\documentclass{article}
\usepackage{url}

\def\quest#1{\textbf{#1}}
\def\answ#1{#1 \vspace{0.5em}}

\newcounter{SQcount}
\newenvironment{subq}{
\begin{list}
    {} %    {\textbf{\alph{SQcount}}}
    {   \usecounter{SQcount}
        \renewcommand{\makelabel}[1]{\stepcounter{SQcount} \textbf{\alph{SQcount}) \parbox[t]{0.8\linewidth}{{##1}}}  }
        \setlength{\rightmargin}{\leftmargin}
    }
}
{\end{list}}



\author{William Blum}
\title{Research student's self-assessment}
\date{April 20, 2006}


\begin{document}
\maketitle

\newcounter{Lcount}
\begin{list}
    {} %    {\textbf{\arabic{Lcount}}}
    {   \usecounter{Lcount}
        \renewcommand{\makelabel}[1]{\stepcounter{Lcount} \textbf{\arabic{Lcount} {{#1\hfil}}:}  }
        \setlength{\leftmargin}{0pt}
        \setlength{\rightmargin}{\leftmargin}
    }

\item[Name] \answ{William Blum}

\item[College] \answ{Linacre College}

\item[Current postal address for communication]

\parbox[t]{3cm}{
\answ{Linacre College\\
St cross road\\
Oxford OX13JA \\
0870 276 0524}}

% question 4
\item[]
\begin{subq}
\item[Degree for which you are now registered:] \

   \answ{\indent PRS}

\item[If relevant, have you transferred or applied to
transfer to D.Phil. status?] \

   \answ{Not yet}

\item[If relevant, have you confirmed or applied to
confirm to D.Phil. status?] \

   \answ{No}

\end{subq}



% question 5
\item[Current dissertation title]

\answ{Game semantics for program analysis}

% question 6
\item[Supervisor] \answ{Luke Ong}

% question 7
\item[How frequently do you meet with your supervisor?]

% question 8
\item[]
\begin{subq}
\item[Source of funds by which you are supported:] \

    \answ{Lukes Ong's Game \& Verfification grant }

\item[When do these funds run out?] \

   \answ{January 2008}

\end{subq}

% question 9
\item[How much teaching are you doing?] \

\answ{I tutored two groups of students for the classes of
\emph{Introduction to Specification }(Hilary 2006). I also did the
marking for one group.}

% question 10
\item[Do you have any form of employment? Give brief details]
\answ{No}

% question 11
\item[Lectures/seminars/conferences attended since last report] \

\answ{Courses taken: Domain theory, Categories, Proofs and programs. Conferences: I helped organizing CSL 2005,
In August 2005 I attended the \emph{Marktoberdorf Summer School}. In July 2005 I attended \emph{Program transformation and Analysis} (PAT) in Copenhagen.
In February 2006 I visited the Isaac Newton Institute in Cambridge. I have presented a project during the Computer Laboratories open days.
}

% question 12
\item[How has your work gone in the past six months?] \

\answ{ I studied a restriction of lambda-calculus called ``safe
lambda-calculus''. \emph{Safety} is a syntactic property originally
defined by Knapik et al. in \cite{KNU02} for higher-order recursion
schemes (grammars). In their paper they proved that the MSO theory
of the term tree generated by a safe recursion scheme of level $n$
is decidable. More recently, Ong proved in \cite{OngLics2006} that
the safety assumption is in fact not necessary.}

I am interested in the transposition of the safety property for
grammar to the safety property for lambda terms. A definition of the
safe $\lambda$-calculus was first given in a technical report by
Aehlig, de Miranda and Ong in \cite{safety-mirlong2004}. One
interesting property of safe lambda terms is that performing
substitution on such terms does not involve renaming of the
variable.

I have investigated different possible definitions of a safe lambda
calculus. I also tried to find a more general notion of safety that
do not assume homogeneity of types. Unfortunately, I have not yet
found a definition that would preserve the \emph{no variable
renaming} property.

Another direction of research that I have been investigating
consisted in relating the safety restriction and the
\emph{size-change termination} property defined in
\cite{jones01,jones04}. Jones conjectured that any simply typed term
is size-change terminating, however Damien Sereni disproved this
conjecture by exhibiting a class of counter-examples
(\cite{serenistypesct05}). The terms of this class are not all of
homogeneous type but there is a subclass of homogenous and safe
simply typed terms that are not size-change terminating. This
suggests that there is no obvious link between safety and
size-change termination.


Recently, inspired by my reading on game semantics
\cite{abramsky:game-semantics} and by the technics developed by Luke
Ong in \cite{OngLics2006}, I have proved a result on the game
semantics of safe terms: the pointers in the game semantics of safe
simply type terms can be recovered uniquely from the sequence of
moves. This result is similar to the standard result in game
semantics which says that pointers of strategies can be recovered
uniquely for arena of order 2 at most. A consequence of this result
is that the semantics of such terms can be described by (extended)
regular expressions \cite{ghicamccusker00}.



In parallel, I worked on a separate project with Matthew Hagues and
Luke Ong. We developed a SAT-based approach to the LTL model
checking problem based on the work of Merz \emph{et al}
\cite{hammer:truly} and McMillan \cite{DBLP:conf/cav/McMillan03}.

We have produced an experimental implementation in OCaml and C. The
program input is an NuSMV file (\cite{CAV02:nusmv}) containing the
model and the LTL properties to verify.

The result obtained at the moment are not too encouraging but it is
apparently due to an unfortunate bug in the implementation that we
are tracking down. There are also some optimizations that we have
not finished to implement.

We are currently finishing to write a paper about it and we hope to
submit it to the BMC workshop.

% question 13
\item[Research plan for the next term and vacation] \

First I need to extend the safety restriction to Idealized Algol
terms. Then I would like to extend the result I obtained for game
semantics of safe simply typed term to safe Idealized Algol terms
and investigate if it leads to some application in algorithmic game
semantics.


In \cite{abramsky:mchecking_ia}, Abramsky has studied a language
called Serially Re-entrant Idealized Algol, or SRIA for short. This
language allows multiple occurrences or uses of arguments, as long
as they do not overlap in time. In the games of this language there
is at most one pending occurrence of a question at any time, so that
each move has a unique justifier and so justification pointers may
be ignored.

I would like to find out whether there is a relationship between
SRIA and safe Idealized Algol.


I also want to investigate some application of game semantics to
program analysis and transformation by trying to extend the work of
Dimovski et al. (\cite{DBLP:conf/sas/DimovskiGL05}) on
data-absraction refinement based on game semantics.

Finally I will continue to work with Matthew Hagues and Luke Ong on
the LTL model checking problem.

% question 14
% \item[Comments]

\end{list}

\bibliography{selfasse20060428}
\bibliographystyle{plain}

\end{document}
