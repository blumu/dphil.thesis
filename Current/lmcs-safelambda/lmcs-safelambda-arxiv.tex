% Commands to be executed by LatexDaemon
%Daemon> ini=latex
%Daemon> afterjob=dvipspdf
%Daemon> filter=err+warn
%Daemon> custom_args=-synctex=-1

% -*- TeX -*- -*- Soft -*-
\documentclass{lmcs}

%%%%%%%%%%%%%%%%%%%%%%%% PACKAGES
\usepackage{amssymb}
\usepackage{bussproofs} % proof tree
\usepackage{pst-tree}
\usepackage[defblank]{paralist}
\usepackage{picins}
\usepackage{fancybox}
\usepackage{manfnt}
\usepackage{pstring}
\usepackage{mathrsfs}

% set the color of the justification pointers (pstring package)
\definecolor{darkGreen}{rgb}{0.03,0.35,0.05}
\pstrSetArrowColor{darkGreen}


%%%%%%%%%%%%%%%%% MACROS

% notes to myself
\newcommand\notetoself[1]{
\bigskip \noindent \fbox{
\begin{tabular}{cl}
\textdbend &
\begin{minipage}{0.80\textwidth}
#1
\end{minipage}
\end{tabular}
} }

% framed table
\newlength{\mylength}
\newenvironment{FramedTable}%
{\begin{table}[htbp]
\begin{Sbox}%
\setlength{\mylength}{\textwidth}%
\addtolength{\mylength}{-4\fboxsep}%
\addtolength{\mylength}{-2\fboxrule}%
\begin{minipage}{\mylength}}%
{\end{minipage}\end{Sbox}\shadowbox{\TheSbox}\end{table}}%


% ABBREVIATIONS
\newcommand\ie{{\it i.e.}} % i.e. should always be followed by a comma and by used in a parenthesis (i.e., as in the present example).
\newcommand\eg{{\it e.g.}} % same usage as i.e.
\newcommand\cf{{\it cf.}\ } % always followed by a space


% highlight for a new definition
% by default the text itself is added to the index.
% this can be overridden using the optional argument \dename[index text]{text}
%\newcommand\defname[1]{{\bf\em #1}\index{#1}}
\makeatletter
\newcommand\defname[2][]{{\bf\em #2}%
  \edef\options{#1}%
  \ifx\options\@empty%
   \index{#2}%
  \else%
   \index{#1}%
  \fi%
}
\makeatother

\newcommand\emphind[1]{\emph{#1}\index{#1}}

% reduction, substitution
\newcommand\betared{\rightarrow_\beta}
\newcommand\betasred{\rightarrow_{\beta_s}}
\newcommand\betaredtr{\twoheadrightarrow_\beta} % transitive closure of the beta reduction
\newcommand\subst[2]{\left[ #1/#2 \right]}
\newcommand\captsubst[2]{\{#1/#2 \}}

% computation tree, eta normal form, traversals
\newcommand{\elnf}[1]{\lceil #1\rceil} % eta long normal form
\newcommand\travset{\mathcal{T}rv}

% lambda calculus, reduction, types
\newcommand\dps{\displaystyle}
\newcommand\rulef[2]{\frac{\dps #1}{\dps #2}}

\newcommand\rulefex[3][5pt]{\frac{\dps
    #2}{\stackrel{\rule{0pt}{#1}}{\dps #3}}}
\newcommand{\ord}{\mathop{\mathrm{ord}}}
\newcommand{\rulename}[1]{\mathbf{({\sf #1})}}
\newcommand{\rulenamet}[1]{${\sf (#1)}$}
\newcommand{\encode}[1]{\overline{#1}}
\newcommand\seq[2]{{{#1} \vdash {#2}}}
\newcommand\typear{\rightarrow}


% set theory
\newcommand{\makeset}[1]{\{\,{#1}\,\}}
\newcommand\inter{\cap}
\newcommand\union{\cup}
\newcommand\Union{\bigcup}
\newcommand\prefset{\textsf{Pref}}
\newcommand{\relimg}[1]{{(\!| #1 |\!)}}
\newcommand\nat{\mathbb{N}}

% game semantics
\newcommand{\theroot}{\circledast} % the root of the computation tree
\newcommand{\lsem}{[\![} % \llbracket
\newcommand{\rsem}{]\!]} % \rrbracket
\newcommand{\sem}[1]{{[\![ #1 ]\!]}}
\newcommand{\revsem}[1]{{\langle\!\langle #1 \rangle\!\rangle}}
\newcommand{\syntrevsem}[1]{{\langle\!\langle #1 \rangle\!\rangle}_{\sf s}}
\newcommand\intercomp{\fatsemi{^\|}}

% justified sequence of moves
\newcommand{\oview}[1]{\llcorner #1 \lrcorner}
\newcommand{\pview}[1]{\ulcorner #1 \urcorner}
\newcommand{\filter}{\upharpoonright}


% logic
\newcommand\imp{\Longrightarrow}
\newcommand\zor{\vee}
\newcommand\zand{\wedge}
\newcommand\entail{\vdash}

% pcf and ia
\newcommand\ialgol{{IA}}
\newcommand\iacom{\texttt{com}}
\newcommand\iaexp{\texttt{exp}}
\newcommand\iavar{\texttt{var}}
\newcommand\pcf{{PCF}}
\newcommand\ycomb{\textsf{Y}}


% trees
\newcommand{\tree}[2][]{\pstree[#1]{\TR{#2}}}
% pstricks parameters for drawing computation trees
\newcommand{\pssetcomptree}{\psset{levelsep=4ex,linewidth=0.5pt}}

\newcommand{\arclabel}[1]{\mput*{\mbox{{\small $#1$}}}}


%%%%%%%%%%%%%%%%%%% THEOREM STYLES

%% define non-standard environments here, for example
\theoremstyle{plain}% default
\newtheorem{theorem}[thm]{Theorem}
\newtheorem{corollary}[thm]{Corollary}
\newtheorem{lemma}[thm]{Lemma}
\newtheorem{proposition}[thm]{Proposition}

\theoremstyle{definition}
\newtheorem{definition}[thm]{Definition}
\newtheorem{conjecture}[thm]{Conjecture}
\newtheorem{example}[thm]{Example}

\theoremstyle{remark}
\newtheorem{remark}[thm]{Remark}


% Entailments for different kinds of judgments
\newcommand\stentail{\vdash_{\sf st}} % entailment for simply-typed term
\newcommand\sentail{\vdash_{\sf s}} % safe entailment
\newcommand\asentail{\vdash_{\sf as}} % almost safe entailment
\newcommand\asappentail{\vdash_{\sf asa}} % almost-safe application entailment
\newcommand\lsentail{\vdash_{\sf l}} % long safe entailment

\newcommand\enable\vdash % enabling relation


\newcommand\Nodes{N}% set of nodes
\newcommand\INodes{IN}% set of inner nodes
\newcommand\LNodes{L}% set of leaf nodes



%% due to the dependence on amsart.cls, \begin{document} has to occur
%% BEFORE the title and author information:
\begin{document}

%\title[The safe lambda calculus]{The safe lambda calculus \\ {\small \it version of \today}}
\title[The safe lambda calculus]{The safe lambda calculus}

\author[W.Blum and C.-H. L.Ong]{William Blum}   %required
\address{Oxford University Computing Laboratory --
School of Informatics, University of Edinburgh, UK}    %required
% Wolfson Building, Parks Road, Oxford OX1 3QD, ENGLAND
\email{william.blum@comlab.ox.ac.uk}  %optional
%\thanks{thanks 1, optional.}   %optional

\author[]{C.-H.~Luke~Ong}   %optional
\address{Oxford University Computing Laboratory, Oxford, UK} %optional ; addresses should be duplicated when authors share an  affiliation
% Wolfson Building, Parks Road, Oxford OX1 3QD, ENGLAND
\email{luke.ong@comlab.ox.ac.uk}  %optional ; ditto for email addresses
%\thanks{thanks 2, optional.}    %optional


%% etc.

%% required for running head on odd and even pages, use suitable
%% abbreviations in case of long titles and many authors:

%% mandatory lists of keywords and classifications:
\keywords{lambda calculus, higher-order recursion scheme, safety
restriction, game semantics} \subjclass{F.3.2, F.4.1}

% OPTIONAL comment concerning the title, \eg, if a variant
%or an extended abstract of the paper has appeared elsewhere
\titlecomment{Some of the results presented here were first published in TLCA proceedings \cite{blumong:safelambdacalculus}}
%%%%%%%%%%%%%%%%%%%%%%%%%%%%%%%%%%%%%%%%%%%%%%%%%%%%%%%%%%%%%%%%%%%%%%%%%%%

%% the abstract has to PRECEED the command \maketitle:
%% be sure not to issue the \maketitle command twice!

\begin{abstract}
  Safety is a syntactic condition of higher-order grammars that
  constrains occurrences of variables in the production rules
  according to their type-theoretic order. In this paper, we introduce
  the \emph{safe lambda calculus}, which is obtained by transposing
  (and generalizing) the safety condition to the setting of the
  simply-typed lambda calculus. In contrast to the original definition
  of safety, our calculus does not constrain types (to be
  homogeneous). We show that in the safe lambda calculus, there is no
  need to rename bound variables when performing substitution, as
  variable capture is guaranteed not to happen.  We also propose an
  adequate notion of $\beta$-reduction that preserves safety.  In the
  same vein as Schwichtenberg's 1976 characterization of the
  simply-typed lambda calculus, we show that the numeric functions
  representable in the safe lambda calculus are exactly the
  multivariate polynomials; thus conditional is not definable. We
  also give a characterization of representable word functions.
  We then study the complexity of deciding beta-eta equality of two safe simply-typed terms and show that this
  problem is PSPACE-hard.
  Finally we give a game-semantic analysis of safety: We show that
  safe terms are denoted by \emph{P-incrementally justified
    strategies}. Consequently pointers in the game semantics of safe
  $\lambda$-terms are only necessary from order 4 onwards.
\end{abstract}

    \maketitle              % typeset the title of the contribution

    \section{Introduction}

\subsection*{Background}

The \emph{safety condition} was introduced by Knapik, Niwi{\'n}ski and
Urzyczyn at FoSSaCS 2002 \cite{KNU02} in a seminal study of the
algorithmics of infinite trees generated by higher-order grammars. The
idea, however, goes back some twenty years to Damm \cite{Dam82} who
introduced an essentially equivalent\footnote{See de Miranda's
 thesis \cite{demirandathesis} for a proof.} syntactic
restriction (for generators of word languages) in the form of
\emph{derived types}.
% Level-$n$ tree grammars as defined by Damm correspond exactly to a
% subset of safe level-$n$ grammars -- namely the safe complete grammars
% -- and every safe grammar corresponds to a safe complete one.
A higher-order grammar (that is assumed to be \emph{homogeneously
  typed}) is said to be \emph{safe} if it obeys certain syntactic
conditions that constrain the occurrences of variables in the
production (or rewrite) rules according to their type-theoretic
order. Though the formal definition of safety is somewhat intricate,
the condition itself is manifestly important. As we survey in the
following, higher-order \emph{safe} grammars capture fundamental
structures in computation, offer clear algorithmic advantages, and
lend themselves to a number of compelling characterizations:

\begin{itemize}
\item \emph{Word languages}. Damm and Goerdt \cite{DG86} have shown
  that the word languages generated by order-$n$ \emph{safe} grammars
  form an infinite hierarchy as $n$ varies over the natural numbers.
  The hierarchy gives an attractive classification of the
  semi-decidable languages: Levels 0, 1 and 2 of the hierarchy are
  respectively the regular, context-free, and indexed languages (in
  the sense of Aho \cite{Aho68}), although little is known about
  higher orders.

  Remarkably, for generating word languages, order-$n$ \emph{safe}
  grammars are equivalent to order-$n$ pushdown automata \cite{DG86},
  which are in turn equivalent to order-$n$ indexed grammars
  \cite{Mas74,Mas76}.

\item \emph{Trees}. Knapik \emph{et al.} have shown that the Monadic
  Second Order (MSO) theories of trees generated by \emph{safe}
  (deterministic) grammars of every finite order are
  decidable\footnote{It has recently been shown
    \cite{OngLics2006} that trees generated by \emph{unsafe}
    deterministic grammars (of every finite order) also have decidable
    MSO theories. More precisely, the MSO theory of trees generated by order-$n$
recursion schemes is $n$-EXPTIME complete.}.

  They have also generalized the equi-expressivity result due to Damm
  and Goerdt \cite{DG86} to an equivalence result with respect to
  generating trees: A ranked tree is generated by an order-$n$ \emph{safe}
  grammar if and only if it is generated by an order-$n$ pushdown
  automaton.

\item \emph{Graphs}. Caucal \cite{Cau02} has shown that the MSO
  theories of graphs generated\footnote{These are precisely the
    configuration graphs of higher-order pushdown systems.} by
  \emph{safe} grammars of every finite order are decidable. In a recent preprint \cite{hague-sto07}, however,
  Hague \emph{et al.} have
  shown that the MSO theories of graphs generated by order-$n$
  \emph{unsafe} grammars are undecidable, but deciding their modal
  mu-calculus theories is $n$-EXPTIME complete.
\end{itemize}

\subsection*{Overview}

In this paper, we aim to understand the safety condition in the
setting of the lambda calculus. Our first task is to transpose it to
the lambda calculus and pin it down as an appropriate sub-system of
the simply-typed theory. A first version of the \emph{safe lambda
  calculus} has appeared in an unpublished technical report
\cite{safety-mirlong2004}. Here we propose a more general and cleaner
version where terms are no longer required to be homogeneously typed
(see Section~\ref{sec:safe} for a definition). The formation rules of
the calculus are designed to maintain a simple invariant: Variables
that occur free in a safe $\lambda$-term have orders no smaller than
that of the term itself.  We can now explain the sense in which the
safe lambda calculus is safe by establishing its salient property: No
variable capture can ever occur when substituting a safe term into
another. In other words, in the safe lambda calculus, it is
\emph{safe} to use capture-\emph{permitting} substitution when
performing $\beta$-reduction.


There is no need for new names when computing $\beta$-reductions of
safe $\lambda$-terms, because one can safely ``reuse'' variable names
in the input term. Safe lambda calculus is thus cheaper to compute in
this na\"ive sense. Intuitively one would expect the safety constraint
to lower the expressivity of the simply-typed lambda calculus. Our
next contribution is to give a precise measure of the expressivity
deficit of the safe lambda calculus. An old result of Schwichtenberg
\cite{citeulike:622637} says that the numeric functions representable
in the simply-typed lambda calculus are exactly the multivariate
polynomials \emph{extended with the conditional function}.  In the
same vein, we show that the numeric functions representable in the
safe lambda calculus are exactly the multivariate polynomials.

Our last contribution is to give a game-semantic account of the safe
lambda calculus.
% Not much is known about the safe $\lambda$-calculus, and many problems
% remain to be studied concerning its computational power, the
% complexity classes that it characterizes, its interpretation under the
% Curry-Howard isomorphism and its game-semantic characterization. This
% paper is a contribution to the last problem.
%
% The difficulty in giving a game-semantic account of safety lies in the
% fact that it is a syntactic restriction whereas game semantics is
% syntax-independent. The solution consists in finding a particular
% syntactic representation of terms on which the plays of the game
% denotation can be represented.  To achieve this, we use ideas recently
% introduced by the second author \cite{OngLics2006}: a term is
% canonically represented by a certain abstract syntax tree of its
% $\eta$-long normal form referred as the \emph{computation tree}. This
% abstract syntax tree is specially designed to establish a
% correspondence with the game arena of the term. A computation is
% described by a justified sequence of nodes of the computation tree
% respecting some formation rules and called a
% \emph{traversal}. Traversals permit us to model $\beta$-reductions
% without altering the structure of the computation tree via
% substitution. A notable property is that \emph{P-views} (in the
% game-semantic sense) of traversals corresponds to paths in the
% computation tree.  We show that traversals are just representations of
% the uncovering of plays of the game-semantic denotation. We then
% define a \emph{reduction} operation which eliminates traversal nodes
% that are ``internal'' to the computation, this implements the
% counterpart of the hiding operation of game semantics. Thus, we obtain
% an isomorphism between the strategy denotation of a term and the set
% of reductions of traversals of its computation tree.
Using a correspondence result relating the game semantics of a
$\lambda$-term $M$ to a set of \emph{traversals} \cite{OngLics2006}
over a certain abstract syntax tree of the $\eta$-long form of $M$
(called \emph{computation tree}), we show that safe terms are denoted
by \emph{P-incrementally justified strategies}. In such a strategy,
pointers emanating from the P-moves of a play are uniquely
reconstructible from the underlying sequence of moves and the pointers
associated to the O-moves therein: Specifically, a P-question always
points to the last pending O-question (in the P-view) of a greater
order. Consequently pointers in the game semantics of safe
$\lambda$-terms are only necessary from order 4 onwards. Finally we
prove that a $\eta$-long $\beta$-normal $\lambda$-term is \emph{safe}
if and only if its strategy denotation is (innocent and)
\emph{P-incrementally justified}.



% \subsection*{Related work}

% \noindent\emph{The safety condition for higher-order grammars}

% \smallskip

% \noindent We have mentioned the result of Knapik \emph{et al.}~\cite{KNU02} that
% infinite trees generated by \emph{safe} higher-order grammars have
% decidable MSO theories.  A natural question to ask is whether the
% \emph{safety condition} is really necessary.  This has then been
% partially answered by Aehlig \emph{et al.}
% \cite{DBLP:conf/tlca/AehligMO05} where it was shown that safety is not
% a requirement at level $2$ to guarantee MSO decidability. Also, for
% the restricted case of word languages, the same authors have shown
% \cite{DBLP:conf/fossacs/AehligMO05} that level $2$ safe higher-order
% grammars are as powerful as (non-deterministic) unsafe ones.  De
% Miranda's thesis \cite{demirandathesis} proposes a unified framework
% for the study of higher-order grammars and gives a detailed analysis
% of the safety constraint at level 2.

% More recently, one of us obtained a more general result and showed
% that the MSO theory of infinite trees generated by higher-order
% grammars of any level, \emph{whether safe or not}, is decidable
% \cite{OngLics2006}.  Using an argument based on innocent
% game-semantics, he establishes a correspondence between the tree
% generated by a higher-order grammar called \emph{value tree} and a
% certain regular tree called \emph{computation tree}. Paths in the
% value tree correspond to traversals in the computation tree.
% Decidability is then obtain by reducing the problem to the acceptance
% of the (annotated) computation tree by a certain alternating parity
% tree automaton.  The approach that we follow in
% Sec. \ref{sec:correspondence} uses many ingredients introduced in this
% paper.


% The equivalence of \emph{safe} higher-order grammars and higher-order
% deterministic push-down automata for the purpose of generating
% infinite trees \cite{KNU02} has its counterpart in the general (not
% necessarily safe) case: the forthcoming paper \cite{hague-sto07}
% establishes the equivalence of order-$n$ higher-order grammars and
% order-$n$ \emph{collapsible pushdown automata}. Those automata form a
% new kind of pushdown systems in which every stack symbol has a link to
% a stack situated somewhere below it and with an additional stack
% operation whose effect is to ``collapse'' a stack $s$ to the state
% indicated by the link from the top stack symbol.

% \medskip

% \noindent\emph{Computation trees and traversals}

% \smallskip

% \noindent In \cite{DBLP:conf/lics/AspertiDLR94}, a notion of graph
% based on Lamping's graphs \cite{lamping} is introduced to represent
% $\lambda$-terms. The authors unify different notions of paths
% (regular, legal, consistent and persistent paths) that have appeared
% in the literature as ways to implement graph-based reduction of
% $\lambda$-expressions. We can regard a traversal as an alternative
% notion of path adapted to the graph representation of
% $\lambda$-expressions given by computation trees.

% The traversals of a computation tree provide a way to perform
% \emph{local computation} of $\beta$-reductions as opposed to a global
% approach where the $\beta$-reduction is implemented by performing
% substitutions. A notion of local computation of $\beta$-reduction has
% been investigated by Danos and Regnier
% \cite{DanosRegnier-Localandasynchronou} through the use of special
% graphs called ``virtual nets'' that embed the lambda-calculus.


\section{The safe lambda calculus}
\label{sec:safe}
\subsection*{Higher-order safe grammars}
We first present the safety restriction as it was originally defined
\cite{KNU02}. We consider simple types generated by the grammar $A \,
::= \, o \; | \; A \typear A$. By convention, $\rightarrow$ associates
to the right. Thus every type can be written as $A_1 \typear \cdots
\typear A_n \typear o$, which we shall abbreviate to $(A_1, \cdots,
A_n, o)$ (in case $n = 0$, we identify $(o)$ with $o$). The
\emph{order} of a type is given by $\ord{o} = 0$ and $\ord{A \typear
  B} = \max(\ord{A}+1, \ord{B})$. We assume an infinite set of typed
variables. The order of a typed term or symbol is defined to be the
order of its type.

A (higher-order) \defname{grammar} is a tuple $\langle
\Sigma, \mathcal{N}, \mathcal{R}, S \rangle$, where $\Sigma$ is a
ranked alphabet (in the sense that each symbol $f \in \Sigma$ has an
arity $\mathit{ar}(f) \geq 0$) of \emph{terminals}\footnote{Each $f \in
  \Sigma$ of arity $r \geq 0$ is assumed to have type $(\underbrace{o,
    \cdots, o}_r, o)$.}; $\mathcal{N}$ is a finite set of typed
\emph{non-terminals}; $S$ is a distinguished ground-type symbol of
$\mathcal{N}$, called the start symbol; $\mathcal{R}$ is a finite set
of production (or rewrite) rules, one for each non-terminal $F : (A_1,
\ldots, A_n, o) \in \mathcal{N}$, of the form $ F z_1 \ldots z_m
\rightarrow e$ where each $z_i$ (called \emph{parameter}) is a
variable of type $A_i$ and $e$ is an applicative term of type $o$
generated from the typed symbols in $\Sigma \union \mathcal{N} \union \{z_1,
\ldots, z_m \}$. We say that the grammar is \emph{order-$n$} just in
case the order of the highest-order non-terminal is $n$.

The \defname{tree generated by a recursion scheme} $G$ is a possibly
infinite applicative term, but viewed as a $\Sigma$-labelled tree;
it is \emph{constructed from the terminals in $\Sigma$}, and is obtained by
unfolding the rewrite rules of $G$ \emph{ad infinitum}, replacing
formal by actual parameters each time, starting from the start symbol
$S$. See e.g.~\cite{KNU02} for a formal definition.

\pssetcomptree
\parpic[r]{
$\tree[levelsep=3ex,nodesep=1pt,treesep=1cm,linewidth=0.5pt]{g}
{  \TR{a}
    \tree{g}{\TR{a} \tree{h}{\tree{h}{\vdots}}}
}$
}
\begin{example}\rm\label{eg:running}
  Let $G$ be the following order-2 recursion scheme:
\[\begin{array}{rll}
  S & \rightarrow & H \, a\\
  H \, z^o & \rightarrow & F \, (g \,
  z)\\
  F \, \phi^{(o, o)} & \rightarrow & \phi \, (\phi \, (F \, h))\\
\end{array}\]
where the arities of the terminals $g, h, a$ are $2, 1, 0$ respectively.
The tree generated by $G$ is defined by the infinite term $g \, a \, (g \, a \, (h \, (h \, (h \,
\cdots))))$.%  The only infinite \emph{path} in the
% tree is the node-sequence $\epsilon \cdot 2 \cdot 22 \cdot 221 \cdot
% 2211 \cdots$.

%(with the corresponding \textbfit{trace} $g \, g \, h \, h \, h \,
%\cdots \; \in \; \Sigma^\omega$).
\end{example}

A type $(A_1, \cdots, A_n, o)$ is said to be \defname{homogeneous} if
$\ord{A_1} \geq \ord{A_2}\geq \cdots \geq \ord{A_n}$, and each $A_1$,
\ldots, $A_n$ is homogeneous \cite{KNU02}.  We reproduce the following
definition from \cite{KNU02}.

\begin{definition}[Safe grammar]\rm
  (All types are assumed to be homogeneous.) A term of order $k > 0$
  is \emph{unsafe} if it contains an occurrence of a parameter of
  order strictly less than $k$, otherwise the term is \emph{safe}. An
  occurrence of an unsafe term $t$ as a subexpression of a term $t'$
  is \emph{safe} if it is in the context $\cdots (ts) \cdots$,
  otherwise the occurrence is \emph{unsafe}. A grammar is
  \defname{safe} if no unsafe term has an unsafe occurrence at a
  right-hand side of any production.
%   A rewrite rule $F z_1 \ldots z_m \rightarrow e$ is said to be
%   \defname{unsafe} if the righthand term $e$ has a subterm $t$ such
%   that
% \begin{enumerate}[(i)]
% \item $t$ occurs in an {\em operand} ({\it i.e.}~second) position of some
%   occurrence of the implicit application operator {\it i.e.}~$e$ has the
%   form $\cdots (s \, t) \cdots $ for some $s$
% \item $t$ contains an occurrence of a parameter $z_i$ (say) whose
%   order is less than that of $t$.
% \end{enumerate}
% A homogeneous grammar is said to be \defname{safe} if none of its
% rewrite rules is unsafe.
\end{definition}

\begin{example}\begin{inparaenum}[(i)] \item Take $\; H : ((o, o), o), \; f : (o, o, o)$; the
    following rewrite rules are unsafe (in each case we underline the
    unsafe subterm that occurs unsafely):
\[\begin{array}{rll}
G^{(o, o)} \, x & \quad \rightarrow \quad & H \, \underline{(f \, {x})} \\
F^{((o, o), o, o, o)} \, z \, x \, y & \quad \rightarrow \quad & f \, (F \, \underline{(F \, z
\, {y})} \, y \, (z \, x) ) \, x
\end{array}\]
\item The order-2 grammar defined in Example~\ref{eg:running} is
  unsafe.
\end{inparaenum}
% The
% reader is referred to the literature
% \cite{KNU02,demirandathesis,safety-mirlong2004}
% for details about the safety restriction for higher-order grammars.
\end{example}

\subsection*{Safety adapted to the lambda calculus}
We assume a set $\Xi$ of higher-order constants.
We use sequents of the form $\Gamma \vdash_\Xi M : A$ to represent
terms-in-context where $\Gamma$ is the context and $A$ is the type of
$M$. For simplicity
we write $(A_1, \cdots, A_n, B)$ to mean $A_1 \typear \cdots \typear
A_n \typear B$, where $B$ is not necessarily ground.

\begin{definition}\rm
\begin{inparaenum}[(i)]
\item The \defname{safe lambda calculus} is a sub-system of the
  simply-typed lambda calculus defined by induction over the
  following rules:
$$ \rulename{var} \ \rulef{}{x : A\vdash_\Xi x : A} \quad
\rulename{const} \ \rulef{}{\vdash_\Xi f : A} \quad f \in \Xi \quad
\rulename{wk} \ \rulef{\Gamma \vdash_\Xi s : A}{\Delta \vdash_\Xi s : A} \quad
\Gamma \subset \Delta$$
$$ \rulename{app} \ \rulef{\Gamma \vdash_\Xi s : (A_1,\ldots,A_n,B) \
  \Gamma \vdash_\Xi t_1 : A_1 \; \ldots \; \Gamma \vdash_\Xi t_n : A_n
} {\Gamma \vdash_\Xi s t_1 \ldots t_n : B} \ \ord{B} \sqsubseteq
\ord{\Gamma}$$
$$ \rulename{abs} \ \rulef{\Gamma, x_1 : A_1, \ldots, x_n : A_n
  \vdash_\Xi s : B} {\Gamma \vdash_\Xi \lambda x_1 \ldots x_n . s :
  (A_1, \ldots ,A_n,B)} \ \ord{A_1, \ldots ,A_n,B} \sqsubseteq
\ord{\Gamma}$$ where $\ord{\Gamma}$ denotes the set $\{ \ord{y} : y
\in \Gamma \}$ and ``$c \sqsubseteq S$'' means that $c$ is a
lower-bound of the set $S$. For convenience, we shall omit the
subscript from $\vdash_\Xi$ whenever the generator-set $\Xi$ is clear from
the context.

\noindent \item The sub-system that is defined by the same rules in
(i), such that all types that occur in them are homogeneous, is called
the \defname{homogeneous safe lambda calculus}.
\end{inparaenum}
\end{definition}

The safe lambda calculus deviates from the standard definition of the simply-typed lambda calculus in a number of ways. First the rules $\rulename{app}$ and $\rulename{abs}$
respectively can perform multiple applications and abstract several
variables at once. (Of course this feature alone does not alter
expressivity.) Crucially, the side-conditions in the application rule
and abstraction rules require that variables in the typing context
have order no smaller than that of the term being formed.  We do not
impose any constraint on types. In particular, type-homogeneity as
used originally to define safe grammars \cite{KNU02} is not required
here. Another difference is that we allow $\Xi$-constants to have
arbitrary higher-order types.  % Thus our formulation
% of the safe lambda calculus is more general than the one proposed in
% the technical report \cite{safety-mirlong2004}. (It is possible to
% reconcile the two definitions by adding the further constraint that
% each type occurring in our rules is homogeneous and by restricting
% constants to at most order 1.)

\begin{example}[Kierstead terms]
\label{ex:kierstead}
Consider the terms $M_1 = \lambda f . f (\lambda x . f (\lambda y . y
))$ and $M_2 = \lambda f . f (\lambda x . f (\lambda y .x ))$ where
$x,y:o$ and $f:((o,o),o)$. The term $M_2$ is not safe because in the
subterm $f (\lambda y . x)$, the free variable $x$ has order $0$ which
is smaller than $\ord{\lambda y . x} = 1$.  On the other hand, $M_1$
is safe.
%On the other hand, $M_1$ is safe as the following proof tree shows:
%$$
% \rulef{
%     \rulef{
%        \rulef{}{f \vdash f} {\sf(var)}
%        \
%        \rulef{
%             \rulef{
%                \rulef{
%                    \rulef{}{f \vdash f} {\sf(var)}
%                }
%                {f , x \vdash f } {\sf(wk)}
%                \
%                \rulef{
%                    \rulef{
%                        \rulef{}{y \vdash y} {\sf(var)}
%                    }
%                    {y \vdash \lambda y . y } \rulenamet{abs}
%                }
%                {f , x \vdash \lambda y .y } {\sf(wk)}
%             }
%             {f , x \vdash f (\lambda y .y )} {\sf(app)}
%        }
%        { f  \vdash \lambda x . f (\lambda y .y )} \rulenamet{abs}
%     }
%     {
%        f  \vdash f (\lambda x . f (\lambda y .y ))} {\sf(app)}
%     }
% { \vdash M_1 = \lambda f . f (\lambda x . f (\lambda y .y )) } \rulenamet{abs}
%$$
\end{example}

It is easy to see that valid typing judgements of the safe lambda
calculus satisfy the following simple invariant:
\begin{lemma}
\label{lem:ordfreevar}
If $\Gamma \vdash M : A$ then every variable in $\Gamma$ occurring
free in $M$ has order at least $ord(M)$.
\end{lemma}


When restricted to the homogeneously-typed
sub-system, the safe lambda calculus captures the original notion
of safety due to Knapik \emph{et al.} in the context of higher-order
grammars:

\begin{proposition} Let $G = \langle \Sigma, \mathcal{N}, \mathcal{R},
  S \rangle$ be a grammar and let $e$ be an applicative term generated
  from the symbols in $\mathcal{N} \cup \Sigma \cup \makeset{z_1^{A_1},
    \cdots, z_m^{A_m}}$.  A rule $F z_1 \ldots z_m \rightarrow e$ in
  $\mathcal{R}$ is safe if and only if $ z_1 : A_1, \cdots, z_m : A_m
  \vdash_{\Sigma \cup \mathcal{N}} e : o$ is a valid typing judgement
  of the \emph{homogeneous} safe lambda calculus.
\end{proposition}

\emph{In what sense is the safe lambda calculus safe?} A basic idea
in the lambda calculus is that when performing $\beta$-reduction, one
must use capture-\emph{avoiding} substitution, which is standardly
implemented by renaming bound variables afresh upon each substitution.
In the safe lambda calculus, however, variable capture can never
happen (as the following lemma shows). Substitution can therefore be
implemented simply by capture-\emph{permitting} replacement, without
any need for variable renaming. In the following, we write
$M\captsubst{N}{x}$ to denote the capture-\emph{permitting}
substitution\footnote{This substitution is done by
textually replacing all free occurrences of $x$ in $M$ by $N$ without performing variable renaming.  In particular for the abstraction
  case we have
$(\lambda y_1\ldots y_n . M)\captsubst{N}{x} = \lambda y_1\ldots y_n . M\captsubst{N}{x}$ when $x\not\in
  \{ y_1\ldots y_n \}$.}
%\footnote{This substitution is implemented by textually
%  replacing all free occurrences of $x$ in $M$ by $N$ without
%  performing variable renaming.  In particular for the abstraction
%  case $(\lambda \overline{y} . P)\captsubst{N}{x}$ is defined as
%  $\lambda \overline{y} . P\captsubst{N}{x}$ if $x\not\in
%  \overline{y}$ and $\lambda \overline{y} . P$ elsewhere.}
of $N$ for $x$ in $M$.

\begin{lemma}[No variable capture]\label{lem:nvc}
\label{lem:homog_nocapture} There is
no variable capture when performing capture-permitting
substitution of $N$ for $x$ in $M$
provided that $\Gamma, x:B \vdash M : A$ and $\Gamma \vdash  N : B$ are valid judgments of the safe lambda calculus.
\end{lemma}

\proof
  We proceed by structural induction. The variable, constant and
  application cases are trivial. For the abstraction case, suppose $M = \lambda \overline{y}. R$ where $\overline{y} = y_1
  \ldots y_p$. If $x \in \overline{y}$ then $M \captsubst{N}{x} = M$ and there is no variable capture.

 If $x \not\in \overline{y}$ then we have $M \captsubst{N}{x} = \lambda \overline{y} . R \captsubst{N}{x}$.  By the induction hypothesis there is no variable capture in $R \captsubst{N}{x}$.  Thus variable capture can only happen if the following two conditions are met: $x$ occurs freely in $R$, and some variable $y_i$ for $1 \leq i \leq p$ occurs freely in $N$. By Lemma \ref{lem:ordfreevar}, the latter condition  implies $\ord{y_i} \geq \ord{N} = \ord{x}$.  Since $x \not \in \overline{y}$, the former condition implies that $x$ occurs freely in the safe term $\lambda \overline{y}. R$
  therefore Lemma \ref{lem:ordfreevar} gives $ \ord{x} \geq
  \ord{\lambda \overline{y} . R} \geq 1+ \ord{y_i} > \ord{y_i}$ which  gives a contradiction.
\qed


\begin{remark}
  A version of the No-variable-capture Lemma also holds in safe
  grammars, as is implicit in (for example Lemma 3.2 of) the original
  paper \cite{KNU02}.
\end{remark}

\begin{example}
  In order to contract the $\beta$-redex in the term
\[f:(o,o,o),x:o
  \vdash (\lambda \varphi^{(o,o)} x^o . \varphi \, x) (\underline{f \,
    x}) : (o,o)\] one should rename the bound variable $x$ with a fresh name to
  prevent the capture of the free occurrence of $x$ in the underlined term during substitution. Consequently, by the previous lemma,
  the term is not safe. Indeed, it cannot be because $\ord{x} = 0 < 1
  = \ord{f x}$.
\end{example}

Note that it is not the case that $\lambda$-terms
that satisfy the No-variable-capture Lemma are necessarily safe. For instance the $\beta$-redex in $\lambda y^o
z^o. (\lambda x^o .y) z$ can be contracted using capture-permitting
substitution, even though the term is not safe.

\subsection*{Reductions and transformations preserving safety}

From now on we will use the standard notation $M\subst{N}{x}$ to
denote the substitution of $N$ for $x$ in $M$.  It is understood that,
provided that $M$ and $N$ are safe, this substitution is
capture-permitting.


\begin{lemma}[Substitution preserves safety]
\label{lem:subst_preserve_safety}
If $\Gamma, x :B \vdash M : A$ and $\Gamma \vdash N : B$ then $\Gamma \vdash M[N/x] : A$.
\end{lemma}
This is proved by an easy induction on the structure of the safe term $M$.


It is desirable to have an appropriate notion of reduction for our
calculus. However the standard $\beta$-reduction rule is not
adequate. Indeed, safety is not preserved by $\beta$-reduction as the
following example shows. Suppose that $w,x,y,z : o$ and $f : (o,o,o)
\in \Sigma$ then the safe term $(\lambda x y . f x y) z w$
$\beta$-reduces to $(\underline{\lambda y . f z y}) w$ which is unsafe
since the underlined order-1 subterm contains a free occurrence of the
ground-type $z$. However if we perform one more reduction we obtain
the safe term $f z w$. This suggests an alternative notion of
reduction that performs simultaneous reduction of ``consecutive''
$\beta$-redexes. In order to define this reduction we first introduce
the appropriate notion of redex.

In the simply-typed lambda calculus a redex is a term of the form
$(\lambda x . M) N$. In the safe lambda calculus, a redex is a
succession of several standard redexes:

\begin{definition}\rm
Let $l\geq 1$ and $n\geq 1$. We use the abbreviations $\overline{x}$ and $\overline{x}:\overline{A}$  for $x_1 \ldots x_n$ and $x_1:A_1, \ldots, x_n : A_n$ respectively.

A \defname{safe redex} is a safe term of the form $(\lambda
\overline{x} . M) N_1 \ldots N_l$ such that the variables
$\overline{x}$ are abstracted altogether by one instance of the
\rulenamet{abs} rule (possibly followed by \rulenamet{wk}) and the
term $(\lambda \overline{x}.M)$ is applied to $N_1$, \ldots, $N_l$
by one instance of the \rulenamet{app} rule. Thus $M$, the and the
$N_i$'s are also safe.
\end{definition}
For instance, in the case $n<l$, a safe redex has a derivation tree of the following  form:
$$   \rulef{
            \rulef{\rulef{\rulef{\ldots}{\Gamma', \overline{x}:\overline{A} \vdash M : (A_{n+1}, \ldots, A_l, B)}}{\Gamma' \vdash \lambda \overline{x} . M : (A_1, \ldots, A_l, B)} \rulename{abs}}{\Gamma \vdash \lambda \overline{x} . M : (A_1, \ldots, A_l, B)}\rulename{wk}
            \quad
            \rulef{\ldots}{\Gamma \vdash N_1 :A_1}  \ \ldots \  \rulef{\ldots}{\Gamma \vdash N_l :A_l}
    }
    {
       \Gamma \vdash (\lambda \overline{x} . M) N_1 \ldots N_l : B
    } \rulename{app}
$$


We are now in a position to define a notion of reduction for safe terms.
\begin{definition}\rm
\label{dfn:safereduction} We use the
abbreviations $\overline{x} = x_1 \ldots x_n$,
$\overline{N} = N_1 \ldots N_l$.
The relation $\beta_s$ is defined on the set of safe redexes as:
\begin{eqnarray*}
  \beta_s &=&
  \{  \ (\lambda \overline{x} . M) N_1 \ldots N_l \mapsto \lambda x_{l+1} \ldots x_n. M\subst{\overline{N}}{x_1 \ldots x_l} \mbox{, for $n> l$}
  \} \\
  &\cup&
  \{ \ (\lambda \overline{x}  . M) N_1 \ldots N_l \mapsto M\subst{N_1 \ldots N_n}{\overline{x}} N_{n+1} \ldots N_l
  \mbox{, for $n\leq l$} \} \ .
\end{eqnarray*}
where $M\subst{R_1 \ldots R_k}{z_1 \ldots z_k}$ denotes the simultaneous substitution in $M$ of $R_1$,\ldots,$R_k$ for $z_1, \ldots, z_k$.  The
\defname{safe $\beta$-reduction}, written $\betasred$, is the
compatible closure of the relation $\beta_s$ with respect to the
formation rules of the safe lambda calculus.
\end{definition}

\noindent \emph{Remark:} The $\beta_s$-reduction is a multi-step
$\beta$-reduction \ie it is a subset of the transitive closure of $\betared$.


\begin{lemma}[$\beta_s$-reduction preserves safety]
\label{lem:safered_preserve_safety}
If $\Gamma \vdash s :A$ and $s \betasred t$ then $\Gamma \vdash t :A$.
\end{lemma}

\proof
  It suffices to show that the relation $\beta_s$ preserves safety.
Suppose that $s\ \beta_s\ t$ where $s$ is the
safe-redex $(\lambda x_1 \ldots x_n . M) N_1
  \ldots N_l $ with $x_1 : B_1, \ldots, x_n: B_n$
and $M$ of type $C$.  W.l.o.g we can assume that the last rule used
to form the term $s$ is \rulenamet{app} (and not the weakening rule
\rulenamet{wk}, thus  we have $\Gamma = fv(s)$.

Suppose $n>l$ then $A = (B_{l+1}, \ldots, B_n, C)$. By Lemma \ref{lem:subst_preserve_safety} we can form the safe term %\begin{equation}
$\Gamma, x_{l+1}:B_{l+1}, \ldots x_n :B_{n}\vdash M\subst{\overline{N}}{x_1 \ldots x_l} : C$. %\label{jud:substsafe}\ .
%\end{equation}
By Lemma \ref{lem:ordfreevar}, since $s$ is safe, all the variables
in $\Gamma$ have order $\geq \ord{A}$. This ensures that the
side-condition of the \rulenamet{abs} rule is verified when
abstracting the variables $x_{l+1} \ldots x_n$, which gives us the
judgement $\Gamma \vdash t :A$.

Suppose $n \leq l$. The substitution lemma gives
$\Gamma \vdash M\subst{N_1 \ldots N_n}{\overline{x}} : C$ and using \rulenamet{app} we form $\Gamma \vdash t :A$.
  \qed


In general, safety is not preserved by $\eta$-expansion; for instance we have
% $f:o,o \vdash f$ but $f:o,o \not \vdash \lambda x^o . f x$.
%This remark remains true for closed terms, for instance
$\vdash \lambda y^o z^o . y : (o,o,o)$ but
$\not \vdash \lambda x^o . (\lambda y^o z^o . y) x : (o,o,o)$.
However safety is preserved by $\eta$-reduction:

\begin{lemma}[$\eta$-reduction preserves safety]
  $\Gamma \vdash \lambda \varphi . s \varphi :A $ with $\varphi$ not
  occurring free in $s$ implies $\Gamma \vdash s :A$.
\end{lemma}
\proof
  Suppose $\Gamma \vdash \lambda \varphi . s \varphi :A$. If $s$ is an  abstraction then by construction of the safe term $\lambda \varphi . s \varphi$, $s$ is necessarily safe.  If $s = N_0 \ldots N_p$ with
  $p\geq 1$ then again, since $\lambda \varphi . N_0 \ldots N_p
  \varphi$ is safe, each of the $N_i$ is safe for $0 \leq i \leq p$
  and for any $z\in fv(\lambda \varphi . s \varphi)$, $\ord{z} \geq
  \ord{\lambda \varphi . s \varphi} = \ord{s}$. Since  $\varphi$ does not occur free in $s$ we have $fv(s) = fv(\lambda \varphi . s \varphi)$, thus we can use the application rule to form $fv(s) \vdash N_0 \ldots N_p : A$. The weakening rules permits us to conclude $\Gamma \vdash s :A$. \qed



The $\eta$-long normal form (or simply $\eta$-long form) of a term
% (also called \emph{long reduced form}, \emph{$\eta$-normal form} and
% \emph{extensional form} in the literature
% \cite{DBLP:journals/tcs/JensenP76,DBLP:journals/tcs/Huet75,huet76})
is obtained by hereditarily $\eta$-expanding every subterm occurring
at an operand position. Formally the \defname{$\eta$-long form}
$\elnf{t}$ of a term $t: (A_1,\ldots,A_n,o)$ with $n \geq 0$ is
defined by cases according to the syntactic shape of $t$:
\begin{eqnarray*}
  \elnf{\lambda x . s } &=& \lambda x . \elnf{s} \\
  \elnf{x s_1 \ldots s_m } &=& \lambda \overline{\varphi} . x \elnf{s_1}\ldots \elnf{s_m} \elnf{\varphi_1} \ldots \elnf{\varphi_n} \\
  \elnf{(\lambda x . s) s_1 \ldots s_p } &=& \lambda \overline{\varphi} . (\lambda x . \elnf{s}) \elnf{s_1} \ldots \elnf{s_p} \elnf{\varphi_1} \ldots \elnf{\varphi_n}
\end{eqnarray*}
where $m \geq 0$, $p\geq 1$, $x$ is a  variable or constant, $\overline{\varphi} = \varphi_1 \ldots \varphi_n$ and each $\varphi_i : A_i$ is a fresh variable.

%\begin{remark}
%  Converting a term to its $\eta$-long normal form does not introduce
%  new redex therefore the $\eta$-long normal form of $\beta$-normal
%  term is a $\beta$-normal term.
%\end{remark}

\begin{lemma}[$\eta$-long normalization preserves safety]
\label{lem:elnf_preserves_safety}
If $\Gamma \vdash s :A$ then $\Gamma \vdash \elnf{s} :A$.
\end{lemma}
\proof

 First we observe that for any variable or constant $x:A$ we have $x:A \vdash \elnf{x} :A$. We show this by induction on $\ord{x}$.
It is verified for any ground type variable $x$
since $x = \elnf{x}$.
Step case: $x:A$ with $A=(A_1, \ldots, A_n,o)$ and $n>0$. Let $\varphi_i:A_i$ be fresh variables for $1\leq i\leq n$.
Since $\ord{A_i} < \ord{x}$ the induction hypothesis gives $\varphi_i :A_i \vdash \elnf{\varphi_i} : A_i$. Using \rulenamet{wk} we obtain $x:A, \overline{\varphi} : \overline{A}
  \vdash \elnf{\varphi_i} :A_i$.  The application rule gives $x :A, \overline{\varphi} : \overline{A} \vdash x \elnf{\varphi_1} \ldots \elnf{\varphi_n}
  : o$ and the abstraction rule gives $ x :A \vdash \lambda
  \overline{\varphi} . x \elnf{\varphi_1} \ldots \elnf{\varphi_n} =
  \elnf{x} :A$.


We now prove the lemma by induction on $s$.
The base case is covered by the previous observation.
\emph{Step case:}
\begin{compactitem}
\item $s = x s_1 \ldots s_m$ with $x: (B_1, \ldots, B_m, A)$, $A = (A_1, \ldots, A_n, o)$ for some $m\geq 0$, $n>0$ and $s_i : B_i$ for $1 \leq i \leq
  m$.  Let $\varphi_i: A_i$ be fresh variables for $1\leq i \leq
  n$. By the previous observation we have $\varphi_i :A_i \vdash \elnf{\varphi_i} :A_i$, the weakening rule then gives us $\Gamma , \overline{\varphi} : \overline{A}
  \vdash \elnf{\varphi_i} : A_i$.  Since the judgement
  $\Gamma \vdash x s_1 \ldots s_m : A$ is formed using the \rulenamet{app} rule, each $s_j$ must be safe for $1\leq j \leq m$, thus by the induction hypothesis we have $\Gamma \vdash \elnf{s_j} : B_j$ and by weakening we get $\Gamma, \overline{\varphi} :\overline{A} \vdash \elnf{s_j} : B_j$.  The \rulenamet{app}
  rule then gives $\Gamma, \overline{\varphi} :\overline{A} \vdash x \elnf{s_1} \ldots \elnf{s_m} \elnf{\varphi_1} \ldots \elnf{\varphi_n} : o$. Finally
  the \rulenamet{abs} rule gives $\Gamma \vdash \lambda \overline{\varphi} . x
  \elnf{s_1} \ldots \elnf{s_m} \elnf{\varphi_1} \ldots
  \elnf{\varphi_n} = \elnf{s} : A$, the side-condition of \rulenamet{abs} being verified since $\ord{\elnf{s}} = \ord{s}$.


\item $s = t s_0 \ldots s_m$ where $t$ is an abstraction.
For some fresh variables $\varphi_1$, \ldots, $\varphi_n$
we have $\elnf{s} = \lambda \overline{\varphi}. \elnf{t} \elnf{s_0} \ldots \elnf{s_m} \elnf{\varphi_1}
  \ldots \elnf{\varphi_n}$. Again, using the induction hypothesis we can easily derive $\Gamma \vdash
 \lambda \overline{\varphi}. \elnf{t} \elnf{s_0} \ldots \elnf{s_m} \elnf{\varphi_1} \ldots \elnf{\varphi_n} : A$.

\item $s = \lambda \overline{\eta} . t $ where
$\overline{\eta} : \overline{B}$ and $t:C$ is not an abstraction. The induction hypothesis gives $\Gamma,
  \overline{\eta} : \overline{B} \vdash \elnf{t} : C$ and using
\rulenamet{abs} we get $\Gamma \vdash \lambda \overline{\eta} . \elnf{t} = \elnf{s} : A$.  \qed
\end{compactitem}


Note that the converse does not hold in general, for instance $\lambda
x^o . f^{(o,o,o)} x^o$ is unsafe although $\elnf{\lambda x . f x} =
\lambda x^o y^o . f x y$ is safe.


%\notetoself{Check and prove the following lemma}
% For terms with homogeneous types however, the converse does hold:
%\begin{lemma}
%If $\Gamma \vdash \elnf{s} : T$ is homogeneously safe (i.e. it is a
%safe judgement of the safe $\lambda$-calculus and each sequent
%occurring at the nodes of the proof tree is homogeneously typed)
%then $\Gamma \vdash s :T$ is homogeneously safe.
%\end{lemma}


    \section{Expressivity}
\subsection{Numeric functions representable in the safe lambda
calculus}

Natural numbers can be encoded into the simply-typed lambda calculus
using the Church Numerals: each $n\in\nat$ is encoded into the term
$\encode{n} = \lambda s z. s^n z$ of type $I = ((o,o),o,o)$ where
$o$ is a ground type. In 1976 Schwichtenberg \cite{citeulike:622637}
showed the following:


\begin{theorem}[Schwichtenberg 1976]
The numeric functions representable by simply-typed $\lambda$-terms
of type $I\rightarrow \ldots \rightarrow I$ using the Church Numeral
encoding are exactly the multivariate polynomials \emph{extended
with the conditional function}.
\end{theorem}

If we restrict ourselves to safe terms, the representable functions
are exactly the multivariate polynomials:
\begin{theorem}
\label{thm:polychar} The functions representable by safe
$\lambda$-expressions of type $I\rightarrow \ldots \rightarrow I$
are exactly the multivariate polynomials.
\end{theorem}

\begin{corollary}
The conditional operator $C:I\rightarrow I\rightarrow I \rightarrow
I$ verifying  $C t y z \rightarrow_\beta y$  if $t \rightarrow_\beta
\encode{0}$ and $C t y z \rightarrow_\beta z$ if $t
\rightarrow_\beta \encode{n+1}$ is not definable in the safe
simply-typed lambda calculus.
\end{corollary}
\proof
  Natural numbers are encoded using Church Numerals: $\encode{n} =
  \lambda s z. s^n z$.  Addition: For $n,m \in \nat$, $\encode{n+m} =
  \lambda \alpha^{(o,o)} x^o . (\encode{n} \alpha) (\encode{m} \alpha
  x)$. Multiplication: $\encode{n . m} = \lambda \alpha^{(o,o)}
  . \encode{n} (\encode{m} \alpha)$.  All these terms are safe and
  clearly any multivariate polynomial $P(n_1, \ldots, n_k)$ can be
  computed by composing the addition and multiplication terms as
  appropriate.

For the converse, let $U$ be a safe $\lambda$-term of type
$I\rightarrow I\rightarrow I$.  The generalization to terms of type
$I^n \rightarrow I$ for $n>2$ is immediate (they correspond to
polynomials with $n$ variables). W.l.o.g we can assume that $U =
\lambda x y \alpha z. u$ where $u$ is a safe term of ground type in
$\beta$-normal form with $fv(u) \subseteq \{ x, y : I, z :o, \alpha
: o\rightarrow o \}$.

\emph{Notation:} Let $T$ be a set of terms of type $\tau \rightarrow
\tau$ and $T'$ be a set of terms of type $\tau$ then $T \cdot T'$
denotes the set of terms $\{ s s' : \tau \ | \ s \in T \wedge s' \in
T' \}$. We also define $T^k \cdot T'$ recursively as follows:  $T^0
\cdot T' = T'$ and for $k\geq 0$, $T^{k+1} \cdot T' = T \cdot (T^k
\cdot T')$ ({\it i.e.}~$T^k \cdot T'$ denotes $\{ s_1( \ldots (s_k
s'))  \ | \ s_1, \ldots, s_k \in T \wedge s' \in T' \}$). We define
$T^+\cdot T' = \Union_{k > 0} T^k \cdot T'$ and $T^*\cdot T' =
(T^+\cdot T') \union T'$. For two sets of terms $T$ and $T'$, we
write $T =_\beta T'$ to express that any term of $T$ is
$\beta$-convertible to some term $t'$ of $T'$ and reciprocally.

Let us write $\mathcal{N}^\tau$ for the set of $\beta$-normal terms
of type $\tau$ where $\tau$ ranges in $\{ o, o\rightarrow o, I \}$
and with free variables in $\{ x,y:I, z:o, \alpha:o\rightarrow o\}$.
We write $\mathcal{A}^\tau$ for the subset of $\mathcal{N}^\tau$
consisting of applications only ({\it i.e.}~not abstractions). Let
$B$ be the set of terms of type $(o,o)$ defined by $B = \{ \alpha \}
\union \{ \lambda a.b \ | \ b \in \{a,z\}, a \neq z \}$. It is easy
to see that the following equations hold:
\begin{eqnarray*}
\mathcal{A}^I &=& \{ x,y \} \\
\mathcal{N}^{(o,o)} &=& B \union \mathcal{A}^I \cdot
\mathcal{N}^{(o,o)} = (\mathcal{A}^I)^* \cdot B \\
\mathcal{A}^{(o,o)} &=& \{ \alpha \} \union (\mathcal{A}^I)^+ \cdot B \\
\mathcal{A}^o = \mathcal{N}^o &=& \{ z \} \union \mathcal{A}^{(o,o)} \cdot \mathcal{N}^o = (\mathcal{A}^{(o,o)})^* \cdot \{ z \}
\end{eqnarray*}
Hence $\mathcal{A}^o = \left( \{\alpha \} \union \{x,y\}^+ \cdot
\left( \{\alpha \} \union \{\lambda a.b \ | \ b \in \{a,z\}, a \neq
z \} \right) \right)^* \cdot \{ z \}$. Since $u$ is safe, it cannot
contain terms of the form $\lambda a . z$ with $a \neq z$ occurring
at an operand position, therefore since $u$ belongs to
$\mathcal{A}^o$ we have:
\begin{equation}
u \in \left( \{\alpha\} \union \{x,y\}^+ \cdot \{\alpha,
\underline{i} \} \right)^* \cdot \{ z \} \label{eqn:u}
\end{equation}
where $\underline{i}$ is the identity term of type $o\rightarrow o$.


We observe that $\encode{k} \underline{i} =_\beta \underline{i}$ for
all $k \in \nat$ and for $l\geq 1$, for all $k_1, \ldots k_l \in
\nat$, $\encode{k_1}\ldots \encode{k_l} \alpha =_\beta
\encode{k_1\times \ldots \times k_l} \alpha$. Hence for all $m,n \in
\nat$ we have:
\begin{equation}
\begin{array}{llr}
\{\encode{m},\encode{n}\}^+ \cdot \{\alpha, \underline{i} \} &=_\beta
\{ \underline{i} \} \union
\{ \encode{m^i n^j} \alpha \ |\ i+j \geq 1 \} \nonumber \\
&= \{ \encode{m^i n^j} \alpha \ |\ i,j \geq 0 \} & ( \mbox{since } \underline{i} = \encode{0} \alpha) \end{array}
\label{eqn:intermediate}
\end{equation}
therefore:
$$\begin{array}{llr}
u[\encode{m}, \encode{n}/x,y] &\in \left( \{ \alpha \} \union \{\encode{m},\encode{n}\}^+ \cdot \{\alpha, \underline{i} \} \right)^* \cdot \{ z \}  & \mbox{(by Eq.\ \ref{eqn:u})} \\
&=_\beta \left( \{\alpha \} \union \{ \encode{m^i n^j}
\alpha \ | \ i,j \geq 0 \} \right)^* \cdot \{ z \} & \mbox{(by Eq.\ \ref{eqn:intermediate})}  \\
&=_\beta \left\{ \encode{m^i n^j}
\alpha \ | \ i,j \geq 0 \right\}^* \cdot \{ z \} & \mbox{($\alpha z =_\beta \encode{1} \alpha z$)}.
\end{array}$$

Furthermore, for all $m,n,r,i,j\in \nat$ we have $\encode{m^i n^j}
\alpha (\alpha^r z) =_\beta \alpha^{r + m^i n^j} z$, hence
$u[\encode{m} \encode{n}/x,y] =_\beta \alpha^{p(m,n)} z$ where
$p(m,n) = \sum_{0\leq k \leq d} m^{i_k} n^{j_k}$ for some $i_k,j_k
\geq 0$, $k \in\{ 0,..,d \}$ and $d\geq 0$. Thus $U \encode{m}
\encode{n} =_\beta \encode{p(m,n)}$. \qed


For instance, the term $ C = \lambda F G H \alpha x . H (
\underline{\lambda y . G \alpha x} ) (F \alpha x)$ used by
Schwichtenberg \cite{citeulike:622637} to define the conditional
operator is unsafe since the underlined subterm is of order $1$,
occurs at an operand position and contains an occurrence of $x$ of
order $0$.

    \newcommand\bigo{\mathcal{O}} % big O notation
\newcommand\booltype{\mathbf{B}}

\section{Complexity of the Safe Lambda Calculus}
Here we study the problem of deciding beta-eta equivalence of two safe lambda terms.

\subsection{Statman's result}

A famous result by Statman  states that deciding the $\beta\eta$-equality of two first-order typable lambda terms is not elementary recursive \cite{Statman:1979:TLE}.
The idea of the proof is to encode the Henkin quantifier elimination of Type Theory into the simply-typed lambda calculus. The encoding relies on the fact that the function $\sf sg$ (conditional) can be encoded in the lambda-calculus. Hence the argument does not carry on   in the Safe Lambda Calculus since the conditional operator is not definable (\cite{blumong:safelambdacalculus}).

Mairson gave a simpler proof of Statman's theorem in \cite{mairson1992spt} which also proceeds by encoding the Henkin quantifier elimination procedure into the lambda-calculus but is much easier to understand as it makes use of list iteration to perform quantifier elimination.

It turns out that both encodings rely on the use of unsafe terms in order to implement the quantifier elimination procedure.

%Here we adapt Mairson's proof to produce a safe encoding of the quantifier elimination procedure, thus showing:
%\begin{theorem}
%The Safe Lambda Calculus is not elementary recursive.
%\end{theorem}

We recall the definition of the theory. Let $\mathcal{D}_0 = \{\mathbf{true},\mathbf{false}\}$ and $\mathcal{D}_{k+1} =powerset(\mathcal{D}_k)$.
For any $k\geq0$, we write $x^k$, $y^k$ and $z^k$ to denote variables ranging over $\mathcal{D}_k$. Prime formulas are $x^0$, $\mathbf{true}\in y^1$, $\mathbf{false}\in y^1$, and  $x^k \in y^{k+1}$. Formulae are built up from prime formulas using the logical connectives $\zand$,$\zor$,$\rightarrow$,$\neg$ and the quantifiers
$\forall$ and $\exists$. Meyer showed that deciding the truth of such formulae requires nonelementary time \cite{Meyer1974}.
\smallskip

In Mairson's encoding, all formula variables of a given order $k$ are encoded by terms of the same type $\Delta_k$. Using this encoding,
unsafety manifests itself in two different ways.
\begin{enumerate}[1.]
  \item
        First in the encoding of set membership. The prime formula $x^k \in y^{k+1}$ is encoded as \begin{equation} x^k : \Delta_k, y^{k+1}:\Delta_{k+1} \vdash y^{k+1} (\lambda y^k : \Delta_k . OR (eq_k~\underline{x^k}~y^k)~F : \Delta_k \typear \Delta_{k+1} \typear \Delta_0 \label{eqn:setmembership}\end{equation}
for some terms $OR$, $F$, $eq_k$.
This term is unsafe because of the underline occurrence of $x^k$ which is not abstracted together with $y^k$.

\item Secondly, quantifier elimination is performed by using a list iterator $\mathbf{D}_{k+1}$ which acts like the $fold\_left$ function from functional programming languages over the list of all elements of $\mathcal{D}_k$.
Thus for instance the formula $\forall x^0 . \exists y^0 . x^0 \zor y^0$
is encoded as $$\vdash \mathbf{D}_1 (\lambda x^0:\Delta_0. AND (\mathbf{D}_1 (\lambda y^0:\Delta_0. OR (\underline{x^0} \zor y^0)) F)) T$$ which is unsafe because of the underlined occurrence.

More generally, supposing that we find a way to encode set membership with a safe term, then the encoding of the formula will be safe if and only if for any variable $x$ in the formula, its binder is precisely the first quantifier $\exists z$ or $\forall z$ in the path to the root of the formula AST verifying $\ord{z} \geq \ord{x}$. For instance the formula $\forall x^k . \exists y^{k+1} . x^k \in y^{k+1}$ would be encoded by an unsafe term whereas the encoding of $\forall y^{k+1} . \exists x^k . x^k \in y^{k+1}$ would be safe.
\end{enumerate}

Surprisingly, the unsafety of the quantifier elimination procedure can be
easily overcome. The idea is as follows. We introduce multiple domains of representation for a given formula. An element of $\mathcal{D}_k$ is thereby represented by countably many terms of type $\Delta_k^n$ where $n\in\nat$ indicates the level of the representation. The type $\Delta_k^n$ is defined in such a way that its order strictly increases as $n$ grows. Moreover there exists a term that can reduce the level of representation of a given term. In the formula representation, each variable can now be encoded with a different level of representation. Since there are infinitely many levels, it is always possible to find an assignment of levels to variables such that the resulting encoding term is safe.

For set-membership, however, there is no obvious way to obtain a safe encoding. The set-member function from Eq.\ \ref{eqn:setmembership} can be turned into a safe term provided that we have access to a function permitting us to increase the representation level of term, but to our knowledge, such transformation cannot be expressed in the simply-typed lambda-calculus.



\subsubsection{Encoding basic boolean operations}

We assume a ground type $o$.
%For any type $\mu$ we define the type $\booltype_\mu \equiv (\mu\typear\mu)\typear \mu$.
%We abbreviate $\booltype_0$ into $\booltype$.
%We introduce the following hierarchy of types: $\sigma_0 \equiv o$, $\sigma_{n+1} \equiv \booltype_{\sigma_n}$ for $n\geq1$.
%Note that the order of $\sigma_n$ strictly increases as $n$ increases.
We introduce a parameterized type for encoding booleans defined by $\booltype_{-1} \equiv o$ and $\booltype_{n+1} \equiv \booltype_n\typear\booltype_n\typear\booltype_n$ for $n\geq0$.
We have $\ord{\booltype_n} = n+1$ for $n\geq-1$.


The representation of the truth values $\mathbf{true}$ and $\mathbf{false}$ will be parameterized by $n \in \nat$ as follows
\begin{align*}
  T^n &\equiv \lambda x^{\booltype_{n-1}} y^{\booltype_{n-1}} .x : \booltype_{n}\\
  F^n &\equiv \lambda x^{\booltype_{n-1}} y^{\booltype_{n-1}} .y : \booltype_{n}
\end{align*}
Clearly these terms as safe. Moreover the following relations hold for all $n$:
\begin{align*}
  T^{n+1}~T^n~F^n &\betared^*  T^n \\
  F^{n+1}~T^n~F^n &\betared^*  F^n
\end{align*}
Hence it is possible to lower the representation level of a term encoding a boolean value by applying the two terms $T^n$ and $F^n$ to it.
For $i\in\nat$, we define the function $\_ \downarrow_i$ that lowers the level-representation of a term, turning a term of type $\booltype_n$ for some $n\in\nat$ to a term of type $\booltype_{\min(i,l)}$:
$$ (M :\booltype_n)\downarrow_i = \left\{
  \begin{array}{ll}
    M~T^{n-1}~F^{n-1}~\ldots~T^{i+1}~F^{i+1}:\booltype_i, & \hbox{if $n>i$;} \\
M:\booltype_n, & \hbox{otherwise.}
  \end{array}
  \right.
$$


Boolean functions are encoded by the following level-parameterized terms:
\begin{align*}
AND^n &\equiv \lambda p : \booltype_n \lambda q : \booltype_n \lambda x:\booltype_{n-1} \lambda y:\booltype_{n-1} . p~(q~x~y)~y : \booltype_n\typear\booltype_n\typear\booltype_n \\
OR^n &\equiv \lambda p : \booltype_n \lambda q : \booltype_n \lambda x:\booltype_{n-1} \lambda y:\booltype_{n-1} . p~x~(q~x~y) : \booltype_n\typear\booltype_{n}\typear\booltype_n \\
NOT^n &\equiv \lambda p : \booltype_n \lambda q : \booltype_n \lambda x:\booltype_{n-1} \lambda y:\booltype_{n-1} . p~y~x : \booltype_n\typear\booltype_n\typear\booltype_n \\
IF^n &\equiv \lambda p : \booltype_n \lambda q : \booltype_n \lambda x:\booltype_{n-1} \lambda y:\booltype_{n-1} . OR^n (NOT^n p)~q : \booltype_n\typear\booltype_n\typear\booltype_n
\end{align*}
which are all safe terms.

\subsubsection{Coding elements of the type hierarchy}
For any $n\in\nat$ we define the hierarchy of type $\Delta_k^n$ as follows:
$\Delta_0^n \equiv \booltype_n$ and $\Delta_{k+1}^n \equiv {\Delta_k^n}^*$ where for any type $\alpha$, $\alpha^* = (\alpha \typear \tau \typear \tau)\typear \tau \typear \tau$.

An occurrence of a formula variable $x^k$ will be encoded as a term variable $x^k:\Delta_{k}^n$ for some level of representation $n\in\nat$.

Following Mairson's  proof, we encode the set $\mathcal{D}_0$ as the list $\mathbf{D}_0$ containing $\mathbf{true}$ and $\mathbf{false}$, and we parameterized this representation by $n\in \nat$:
$$\mathbf{D}_0^n \equiv \lambda c:\booltype_n \typear \tau \typear \tau . \lambda e : \tau . c~T^n~(c~F^n~e) : \Delta_1^n$$
and for $k\geq 0$, the higher-order set $\mathcal{D}_{k+1}$ is represented by the parameterized term:
$$\mathbf{D}_{k+1}^n \equiv powerset~\mathbf{D}_k^n : \Delta_{k+2}^n$$
where the term $powerset$ taken from \cite{mairson1992spt} is reproduced here:
\begin{align*}
  powerset &\equiv \lambda A^* :(\alpha \typear \alpha^{**} \typear \alpha^{**}) \typear \alpha^{**} \typear \alpha^{**}.\\
&\qquad  A^*~double~(\lambda c:\alpha^* \typear \tau\typear \tau.\lambda b:\tau . c ( \lambda c':\alpha\typear \tau\typear \tau. \lambda b':\tau.b') b)\\
powerset &: ((\alpha \typear \alpha^{**} \typear \alpha^{**}) \typear \alpha^{**} \typear \alpha^{**})\typear \alpha^{**}
\end{align*}
with
\begin{align*}
  double &\equiv \lambda x :\alpha.\lambda l : (\alpha^* \typear \tau\typear \tau)\typear \tau\typear \tau. \\
  & \qquad \lambda c:\alpha^*\typear \tau\typear \tau.\lambda b:\tau. \\
  & \qquad \qquad l(\lambda e:\alpha^*.c (\lambda c':\alpha\typear \tau\typear \tau.\lambda b':\tau.c'~x~(e~c'~b')))(l~c~b)\\
double &: \alpha \typear \alpha^{**} \typear \alpha^{**}
\end{align*}

It can be checked that these two terms are safe.

\subsubsection{Quantifier elimination}
Following \cite{mairson1992spt}, quantifier elimination interprets $\forall x^k.\Phi(x^k)$ as the iterated conjunction $\mathbf{D}_k(\lambda x^k:\Delta^k.AND(\hat\Phi~x^k))~T$ where $\hat\Phi$ is the interpretation of $\Phi$; similarly $\exists x^k.\Phi(x^k)$  is interpreted by the iterated disjunction $\mathbf{D}_k(\lambda x^k:\Delta^k.AND(\hat\Phi~x^k))~T$.

Let $x^{k_p}_p \ldots x^{k_1}_1$ for $p\geq1$ be the list of variables appearing in the formula. W.l.o.g.\ we can assume that they are given in the order of appearance of their binder in the formula \ie $x^{k_p}_p$ is bound by the leftmost binder. We assign representation levels to variables as follows. The right-most variable is assigned level $1$ \ie $x^{k_1}_1 : \Delta^1_{k_1}$; suppose that $x^{k_i}_i :\Delta^l_{k_l}$ for $1\leq i< p$ then the representation level of variable $x^{k_{i+1}}_{i+1}$ is defined as
the smallest $l'\in\nat$ such that $\ord{\Delta^{l'}_{k_i}} > \ord{\Delta^{l}_{k_{i-1}}}$.

This way, since variables that are bound first have higher order, the variables
 that are bound in the nested list-iterations (corresponding to the nested quantifiers in the formula) are necessarily safely bound.


\subsubsection{Coding set theory in the $\mathcal{D}_k$}
To complete the interpretation of prime formulas, we would need to show how to encode set membership. Unfortunately, this seems to be impossible in the safe lambda calculus. It would turn to be possible if we had at hand a function $\_ \uparrow^k$, counterpart of $\_ \downarrow_k$, that increases the representation level of a term to level $k$. Here is how we would proceed if such function were representable in the safe lambda calculus.

Firstly, the formulae ``$\mathbf{true} \in y^1$'' and ``$\mathbf{false} \in y^1$'' can be encoded by the safe terms $y^1 (\lambda x^0 . OR^0~x^0) F^0$ and $y^1 (\lambda x^0. OR^0(NOT^0~x^0)) F^0$ respectively.
For the general case ``$x^k\in y^{k+1}$''
we proceed as in \cite{mairson1992spt} by introducing lambda-terms encoding set equality, set membership and subset tests, and we further parameterize these encoding by $n\in\nat$.

Equality of booleans is encoded by:
$$ eq_0^n \equiv \lambda x^0 : \booltype_n .\lambda y^0 : \booltype_n. OR^n (AND^n~x^0~y^0) (AND^n (NOT^n~x^0)(NOT^n~y^0)) \ .$$

We now use variable of type $\Delta_{k+1}^n$ as iterators over list of elements of type $\Delta_k^n$ and we instantiate the type variable $\tau$ as $\booltype_n$ in order to iterate a level-$n$ Boolean function. We define the set membership function as follows. Note that
the level of representation differs from input to output: \begin{align*}
  member_{k+1}^{n+1} &\equiv \lambda x^k : \Delta_k^{n+1}.\lambda y^{k+1}:\Delta_{k+1}^{n+1}. \\
& \qquad (y^{k+1}\downarrow_n) (\lambda y^k : \Delta_k^n . OR^n (eq_k^{n+1}~x^k~(y^k\uparrow^{n+1})))~F^n \\
  & : \Delta_k^{n+1} \typear \Delta_{k+1}^{n+1} \typear \booltype_n
\\
  subset_{k+1}^{n+1} &\equiv \lambda x^{k+1} : \Delta_{k+1}^{n+1}.\lambda y^{k+1}:\Delta_{k+1}^{n+1}. \\
  & \qquad (x^{k+1}\downarrow_n) (\lambda x^k : \Delta_k^n . AND^n (member_{k+1}^{n+1}~x^k~y^{k+1}))~T^n \\
  & : \Delta_{k+1}^{n+1} \typear \Delta_{k+1}^{n+1} \typear\booltype_n
\\
  eq_{k+1}^{n+1} &\equiv \lambda x^{k+1} : \Delta_{k+1}^{n+1}.\lambda y^{k+1}:\Delta_{k+1}^{n+1}. \\
   & \qquad
   (\lambda op:\Delta_{k+1}^n\typear\Delta_{k+1}^n\typear\booltype_n. AND^n (op~x^{k+1}~y^{k+1})(op~y^{k+1}~x^{k+1}))~subset_{k+1}^{n+1} \\
  & : \Delta_{k+1}^{n+1} \typear \Delta_{k+1}^{n+1} \typear \booltype_n
\end{align*}
The terms $eq_{k+1}^n$ and $subset_{k+1}^n$ are safe, and so is $member_{k+1}^n$ thanks to the fact that $y^k$ has a lower representation level than $x^k$.

The formula $x^k\in y^{k+1}$ is then encoded by the term
$$x^k:\Delta_k^n, y^{k+1}:\Delta_{k+1}^{n'}\vdash \left(member_{k+1}^{\min(n,n')} (x^k\downarrow_{\min(n,n')})~(y^{k+1}\downarrow_{\min(n,n')})\right)\downarrow_0$$


\subsection{NP-hardness}
To show NP-hardness it suffices to observe that the encoding of SAT in the simply-typed lambda calculus from the paper\cite{asperti-np} relies only on safe terms.

\subsection{PSPACE-hardness}

We encode QBF into the calculus.
We assume that the quantified propositional formula is given in prenex form:
$$\$_{n-1} x_{n-1} \ldots \$_0 x_0 . \psi(x_0, \ldots, x_{n-1})$$
where $\$_i \in \{\exists,\forall\}$ for $0\leq i\leq n-1$.

The encoding is as follows:
\begin{align*}
\sem{1} &= T^0  : \booltype \\
\sem{0} &= F^0 : \booltype \\
\sem{x_i} &= x_i\downarrow_0 = x_i~T^{i-1}~F^{i-1}\ldots T^1~F^1: \booltype \qquad \hbox{where $x_i:\booltype_i$}\\
\sem{\psi_1\zand \psi_2} &= AND^0~\sem{\psi_1}~\sem{\psi_2}
:\booltype  \\
\sem{\psi_1\zor \psi_2} &= OR^0~\sem{\psi_1}~\sem{\psi_2}
:\booltype  \\
\sem{\neg \psi} &= NOT^0~\sem{\psi}
:\booltype  \\
\sem{\forall x_i.\psi(\ldots, x_i, \ldots)} & = \mathbf{D}_0^i(\lambda x^{\booltype_i} AND^0~\sem{\psi(\ldots, x_i, \ldots)})~T^0 :\booltype\\
\sem{\exists x_i.\psi(\ldots, x_i, \ldots)} & = \mathbf{D}_0^i(\lambda x^{\booltype_i}.OR^0~\sem{\psi(\ldots, x_i, \ldots)})~F^0 :\booltype
\end{align*}
The size of $\sem{\psi}$ is in $\bigo(|\psi|^2)$.

It is easy to check that this encoding is safe.
\begin{example}
  The formula $\forall x \exists y \exists z (x\zor y\zor z)\zand(\neg x\zor \neg y\zor \neg z)$ is represented by the safe term:
\begin{align*}
\vdash &\mathbf{D}_0^2(\lambda x^{\booltype_2}. AND^0\\
&\quad\quad (\mathbf{D}_0^1(\lambda x^{\booltype_1}.OR^0\\
&\quad\quad\quad (\mathbf{D}_0^0(\lambda x^{\booltype_0}.OR^0\\
&\quad\quad\quad\quad (AND^0 (OR^0(OR^0~(x~T^1 F^1 T^0 F^0)~(y~T^0 F^0))z) \\
&\quad\quad\quad\quad\quad (OR^0(OR^0(NEG^0 (x~T^1 F^1 T^0 F^0))(NEG^0 (y~T^0 F^0)))(NEG^0~z))) \\
&\quad\quad\quad )F^0)\\
&\quad\quad)F^0)\\
&\quad) T^0
\end{align*}
\end{example}
This gives us:
\begin{theorem}
  Deciding $\beta\eta$-equality of two terms of the Safe Lambda Calculus is PSPACE-hard.
\end{theorem}

% NP \subseteq PSPACE \subseteq EXP



    
\section{A game-semantic account of safety}
\label{sec:gamesemaccount} Our aim is to characterize safety by game
semantics. We shall assume that the reader is familiar with the
basics of game semantics; For an introduction, we recommend
\cite{abramsky:game-semantics-tutorial}. Recall that a
\emph{justified sequence} over an arena is an alternating sequence
of O-moves and P-moves such that every move $m$, except the opening
move, has a pointer to some earlier occurrence of the move $m_0$
such that $m_0$ enables $m$ in the arena. A \emph{play} is just a
justified sequence that satisfies Visibility and Well-Bracketing. A
basic result in game semantics is that $\lambda$-terms are denoted
by \emph{innocent strategies}, which are strategies that depend only
on the \emph{P-view} of a play. The main result
(Theorem~\ref{thm:safeincrejust}) of this section is that if a
$\lambda$-term is safe, then its game semantics (is an innocent
strategy that) is, what we call, \emph{P-incrementally justified}. In such a
strategy, pointers emanating from the P-moves of a play are uniquely
reconstructible from the underlying sequence of moves and pointers
from the O-moves therein: specifically a P-question always points to
the last pending O-question (in the P-view) of a greater order.

The proof of Theorem~\ref{thm:safeincrejust} depends on a
Correspondence Theorem (see the Appendix) that relates the strategy
denotation of a $\lambda$-term $M$ to the set of \emph{traversals}
over a souped-up abstract syntax tree of the $\eta$-long form of $M$.
In the language of game semantics, traversals are just (concrete
representations of) the \emph{uncovering} (in the sense of Hyland
and Ong \cite{hylandong_pcf}) of plays in the strategy denotation.

The useful transference technique between plays and traversals was
originally introduced by one of us \cite{OngLics2006} for studying
the decidability of monadic second-order theories of infinite structures generated by
higher-order grammars (in which the $\Sigma$-constants or terminal symbols are at most
order 1, and \emph{uninterpreted}).
% In this setting, free variables are interpreted
% as constructors and therefore they do not have the ``full power'' of
% true free variables and are limited to order $1$ at most. Also,
% although the grammar can perform higher-order computations, the
% structure being studied is itself of ground type.
In the Appendix, we present an extension of this framework to the
general case of the simply-typed lambda calculus with free variables
of any order. A new traversal rule is introduced to handle nodes
labelled with free variables. Also new nodes are added to the
computation tree to account for the answer moves of the game
semantics, thus enabling the framework to model languages with
interpreted constants such as \pcf~(by adding traversal rules to
handle constant nodes).

\subsection*{Incrementally-bound computation tree}
 In \cite{OngLics2006} the computation tree of a grammar is
defined as the unravelling of a finite graph representing the \emph{long
transform} of a grammar. Similarly we define the computation tree of
a $\lambda$-term as an abstract syntax tree of its $\eta$-long
normal form.  We write $l\langle t_1, \ldots, t_n \rangle$ with $n
\geq 0$ to denote the ordered tree with a root labelled $l$ with $n$
child-subtrees $t_1$, \ldots, $t_n$. In the following we consider arbitrary
simply-typed terms.

\begin{definition}\rm
\label{dfn:comptree}
  The \defname{computation tree} $\tau(M)$ of a simply-typed term
  $\Gamma \stentail M:T$ with variable names in a countable set
  $\mathcal{V}$ is a tree with labels in $$ \{ @ \} \union \mathcal{V}
  \union \{ \lambda x_1 \ldots x_n \ | \ x_1 ,\ldots, x_n \in
  \mathcal{V}, n\in\nat \}$$ defined from its $\eta$-long form as follows. Suppose $\overline{x} = x_1 \ldots x_n$ for $n\geq 0$ then
\begin{eqnarray*}
  \mbox{for $m\geq 0$, $z \in \mathcal{V}$: } \tau(\lambda \overline{x} . z s_1 \ldots s_m : o) &=& \lambda \overline{x} \langle z \langle\tau(s_1),\ldots,\tau(s_m)\rangle\rangle \\
  \mbox{for $m \geq 1$: } \tau(\lambda \overline{x} . (\lambda y.t) s_1 \ldots s_m :o) &=& \lambda \overline{x} \langle @ \langle \tau(\lambda y.t),\tau(s_1),\ldots,\tau(s_m) \rangle \rangle \ .
\end{eqnarray*}
\end{definition}

\begin{example}
\label{examp:comptree}
  Take $\stentail \lambda f^{o \typear o} .
(\lambda u^{o \typear o} . u) f : (o \typear o) \typear
o \typear o$.
\bigskip

\noindent
\begin{tabular}{cc}
Its $\eta$-long normal form is: & Its computation tree is:\\[8pt]
\begin{minipage}{0.45\textwidth}
\centering
$\begin{array}{ll}
 &\stentail  \lambda f^{o \typear o} z^o . \\
&\qquad(\lambda u^{o \typear o} v^o . u (\lambda.v)) \\
&\qquad(\lambda y^o. f y) \\
&\qquad(\lambda.z) \\
&: (o \typear o) \typear o \typear o
\end{array}$
\end{minipage}
&
\begin{minipage}{0.45\textwidth}
\centering
\psset{levelsep=5ex,linewidth=0.5pt,nodesep=1pt,arcangle=-20,arrowsize=2pt 1}
${\pstree{\TR{\lambda f z}}{\pstree{\TR{@}}{\pstree{\TR{\lambda u v}}{\pstree{\TR{u}}{\pstree{\TR{\lambda }}{\TR{v}}}}\pstree{\TR{\lambda y}}{\pstree{\TR{f}}{\pstree{\TR{\lambda }}{\TR{y}}}} \pstree{\TR{\lambda }}{\TR{z}}}}
}$
\end{minipage}
\end{tabular}
\end{example}

\begin{example}
  Take $\stentail \lambda u^o v^{((o \typear o) \typear o)} . (\lambda x^o . v (\lambda z^o . x)) u : o \typear ((o \typear o) \typear o) \typear o$.
  \bigskip

\noindent
\begin{tabular}{cc}
Its $\eta$-long normal form is: & Its computation tree is:\\[8pt]
\begin{minipage}{0.45\textwidth}
\centering
$\begin{array}{ll}
 &\stentail  \lambda u^o v^{((o \typear o) \typear o)} . \\
&\qquad(\lambda x^o . v (\lambda z^o . x)) u \\
&: o \typear ((o \typear o) \typear o) \typear o
\end{array}$
\end{minipage}
&
\begin{minipage}{0.45\textwidth}
\centering
$\pstree{\TR{\lambda u v}}{\pstree{\TR{@}}{\pstree{\TR{\lambda x}}{\pstree{\TR{v}}{\pstree{\TR{\lambda z}}{\TR{x}}}}\pstree{\TR{\lambda }}{\TR{u}}}}
$
\end{minipage}
\end{tabular}
\end{example}

Even-level nodes are $\lambda$-nodes (the root is on level 0). A
single $\lambda$-node can represent several consecutive variable
abstractions or it can just be a \emph{dummy lambda} if the
corresponding subterm is of ground type.  Odd-level nodes are
variable or application nodes.

The \defname{order} of a node $n$, written $\ord{n}$, is defined as
follows: @-nodes have order $0$. The order of a variable-node is the
type-order of the variable labelling it. The order of the root node
is the type-order of $(A_1,\ldots,A_p, T)$ where $A_1,\ldots, A_p$
are the types of the variables in the context $\Gamma$. Finally, the
order of a lambda node different from the root is the type-order of
the term represented by the sub-tree rooted at that node.

We say that a variable node $n$ labelled $x$ is \defname{bound} by a
node $m$, and $m$ is called the \defname{binder} of $n$, if $m$ is
the closest node in the path from $n$ to the root such that $m$ is
labelled $\lambda \overline{\xi}$ with $x\in \overline{\xi}$.


We introduce a class of computation trees in which the binder node
is uniquely determined by the nodes' orders:
\begin{definition}\rm
  A computation tree is \defname{incrementally-bound} if for all
  variable node $x$, either $x$ is \emph{bound} by the first
  $\lambda$-node in the path to the root with order $> \ord{x}$, or $x$
  is a \emph{free variable} and all the $\lambda$-nodes in the path to
  the root except the root have order $\leq \ord{x}$.
\end{definition}

\begin{proposition}[Safety and incremental-binding] \hfill
\label{prop:safe_imp_incrbound}
\begin{enumerate}[(i)]
\item If $M$ is safe then $\tau(M)$ is incrementally-bound.
\item Conversely, if $M$ is a \emph{closed} simply-typed term and $\tau(M)$
is incrementally-bound then $M$ is safe.
\end{enumerate}
\end{proposition}
\proof
  (i) Suppose that $M$ is safe. By Lemma
  \ref{prop:safe_iff_elnfsafe} the $\eta$-long form of $M$ is safe
  therefore $\tau(M)$ is the tree representation of a safe term.

In the safe lambda calculus, the variables in the context with the
lowest order must be all abstracted at once when using the
abstraction rule. Since the computation tree merges consecutive
abstractions into a single node, any variable $x$ occurring free in
the subtree rooted at a node $\lambda \overline{\xi}$ different from
the root must have order greater or equal to $\ord{\lambda
  \overline{\xi}}$. Conversely, if a lambda node $\lambda
\overline{\xi}$ binds a variable node $x$ then $\ord{\lambda
  \overline{\xi}} = 1+\max_{z\in\overline{\xi}} \ord{z} > \ord{x}$.

Let $x$ be a bound variable node. Its binder occurs in the path from
$x$ to the root, therefore, according to the previous observation,
$x$ must be bound by the first $\lambda$-node occurring in this path
with order $>\ord{x}$. Let $x$ be a free variable node then $x$ is
not bound by any of the $\lambda$-nodes occurring in the path to the
root. Once again, by the previous observation, all these
$\lambda$-nodes except the root have order smaller than $\ord{x}$.
Hence $\tau$ is incrementally-bound.

(ii) Let $M$ be a closed term such that $\tau(M)$ is
incrementally-bound.  W.l.o.g. we can assume that $M$ is in $\eta$-long
form.  We prove that $M$ is safe by induction on its structure. The
base case $M = \lambda \overline{\xi} . x$ for some variable $x$ is
trivial.  \emph{Step case:} If $M = \lambda \overline{\xi} . N_1
\ldots N_p$.  Let $i$ range over $1..p$. We have $N_i \equiv \lambda
\overline{\eta_i} . N'_i$ for some non-abstraction term $N'_i$. By
the induction hypothesis, $\lambda \overline{\xi} . N_i = \lambda
\overline{\xi} \overline{\eta_i} . N'_i$ is a safe closed term, and
consequently $N'_i$ is necessarily safe. Let $z$ be a free variable
of $N'_i$ not bound by $\lambda \overline{\eta_i}$ in $N_i$. Since
$\tau(M)$ is incrementally-bound we have $\ord{z} \geq \ord{\lambda
  \overline{\eta_1}} = \ord{N_i}$, thus we can abstract the variables $\overline{\eta_1}$ using \rulenamet{abs} which shows that $N_i$ is safe.  Finally
we conclude $\sentail M = \lambda \overline{\xi} . N_1 \ldots N_p :
T$ using the rules \rulenamet{app} and \rulenamet{abs}.  \qed



The assumption that $M$ is closed is necessary. For instance for
$x,y:o$, the computation trees $\tau(\lambda x y .x)$ and
$\tau(\lambda y . x)$ are both incrementally-bound but $\lambda x y
.x$ is safe and $\lambda y . x$ is not.

\subsection*{P-incrementally justified strategy}

We now consider the game-semantic model of the simply-typed lambda
calculus. The strategy denotation of a term-in-context $\Gamma
\stentail M : T$ is written $\sem{\Gamma
\stentail M : T}$. We define the \defname{order} of a move $m$,
written $\ord{m}$, to be the length of the path from $m$ to its
furthest leaf in the arena minus 1. (There are several ways to
define the order of a move; the definition chosen here is sound in
the current setting where each question move in the arena enables at
least one answer move.)
%{\it i.e.}~height of the subarena rooted at $q$ minus 2.

\begin{definition}\rm
  A strategy $\sigma$ is said to be \defname{P-incrementally
    justified} if for any play $s \, q \in \sigma$ where $q$ is a
  P-question, $q$ points to the last unanswered O-question in $\pview{s}$ with
  order strictly greater than $\ord{q}$.
\end{definition}
Note that although the pointer is determined by the P-view, the
choice of the move itself can be based on the whole history of the
play. Thus P-incremental justification does not imply innocence.

The definition suggests an algorithm that, given a play of a
P-incrementally justified denotation, uniquely recovers the pointers
from the underlying sequence of moves and from the pointers
associated to the O-moves therein. Hence:
\begin{lemma}
\label{lem:incrjustified_pointers_uniqu_recover} In P-incrementally
justified strategies, pointers emanating from P-moves are
superfluous.
\end{lemma}

\begin{example}
Copycat strategies, such as the identity strategy $id_A$ on game $A$
or the evaluation map $ev_{A,B}$ of type $(A \Rightarrow B) \times A
\typear B$, are all P-incrementally justified.\footnote{In such
strategies, a P-move $m$ is justified as follows: either $m$ points
to the preceding move in the P-view or the preceding move is of
smaller order and $m$ is justified by the second last O-move in the
P-view.}
\end{example}
%%%% the following example is wrong : ev is P-ij.
%
%\begin{example}
%Take the evaluation map $ev : (o^1 \Rightarrow o^2) \times o^3 \rightarrow o^4$ and the play $s = q^4 q^2 q^1 q^3 \in \sem{ev}$. We have $\ord{q^2} = 1 > \ord{q^1} = \ord{q^3} = 0$. Now $q^3$ points to $q^4$ but $q^2$ is the last unanswered O-question in $\pview{s}= s$ with order $>\ord{q^3}$, hence $\sem{ev}$ is not P-incrementally justified.
%\end{example}



The Correspondence Theorem~\ref{thm:correspondence}
% and Lemma \ref{lem:betanf_wellbehavedconst_trav_pview_red}
gives us the following equivalence:
\begin{proposition} % [Incremental-binding vs P-incremental justification]
\label{prop:Nher_incrbound_and_incrjustified} Let $\Gamma \stentail
M : T$ be a $\beta$-normal term. The computation tree $\tau(M)$ is
incrementally-bound if and only if $\sem{\Gamma \stentail M : T}$ is
P-incrementally justified.
\end{proposition}


\parpic[r]{
\pssetcomptree
\raisebox{-12pt}
{$\tree{\lambda^3}{\tree{f^2}{ \tree{\lambda y^1}{\TR{x^0} }}}$}
}
%\noindent \emph{Example:}
\begin{example}
Consider the $\beta$-normal term $\Gamma\stentail f (\lambda y .x) :
o$ where $y:o$ and $\Gamma =f:((o,o),o),~x:o$. The figure on the
right represents its computation tree with the node orders given as
superscripts.  The node $x$ is not incrementally-bound therefore $\tau(f
(\lambda y .x))$ is not incrementally-bound and by Proposition
\ref{prop:Nher_incrbound_and_incrjustified}, $\sem{\Gamma \stentail
f (\lambda y .x) : o}$ is not incrementally-justified (although
$\sem{\Gamma \stentail f : ((o,o),o)}$ and $\sem{\Gamma \stentail
\lambda
  y. x : (o,o)}$ are).
\end{example}
\smallskip

Propositions \ref{prop:safe_imp_incrbound} and
\ref{prop:Nher_incrbound_and_incrjustified} allow us to show the
following:
\begin{theorem}[Safety and P-incremental justification]
\label{thm:safeincrejust} \hfill
\begin{enumerate}[(i)]
\item If $\Gamma \sentail M : T$ then $\sem{\Gamma \sentail M : T}$ is P-incrementally justified.
\item If $\stentail M : T$ is a closed simply-typed term and $\sem{\stentail M : T}$ is P-incrementally justified then the $\beta$-normal form of $M$ is safe.
\end{enumerate}
\end{theorem}
\proof (i) Let $M$ be a safe simply-typed term. By Lemma
\ref{lem:safered_preserves_safety}, its $\beta$-normal form $M'$ is
also safe. By Proposition \ref{prop:safe_imp_incrbound}(i),
$\tau(M')$ is incrementally-bound and by Proposition
\ref{prop:Nher_incrbound_and_incrjustified}, $\sem{M'}$ is an
incrementally-justified. Finally the soundness of the game model
gives $\sem{M} = \sem{M'}$.  (ii) is a consequence of Lemma
\ref{lem:safered_preserves_safety}, Proposition
\ref{prop:Nher_incrbound_and_incrjustified} and
\ref{prop:safe_imp_incrbound}(ii) and soundness of the game model.
\qed



Putting Theorem \ref{thm:safeincrejust}(i) and Lemma
\ref{lem:incrjustified_pointers_uniqu_recover} together gives:
\begin{proposition}
  \label{prop:safe_ptr_recoverable} In the game semantics of safe
  $\lambda$-terms, pointers emanating from P-moves are unnecessary
  {\it i.e.}~they are uniquely recoverable from the underlying sequences of
  moves and from O-moves' pointers.
\end{proposition}

 \begin{example} If justification pointers are omitted then the denotations of the
   two Kierstead terms from Example~\ref{ex:kierstead} are not distinguishable.
   In the safe lambda calculus this ambiguity disappears
   since $M_1$ is safe whereas $M_2$ is not.
 \end{example}

In fact, as the last example highlights, pointers are superfluous at
order $3$ for safe terms whether from P-moves or O-moves. This is
because for question moves in the first two levels of an arena
(initial moves being at level $0$), the associated pointers are
uniquely recoverable thanks to the visibility condition. At the
third level, the question moves are all P-moves therefore their
associated pointers are uniquely recoverable by P-incremental
justification. This is not true anymore at order $4$: Take the safe
term $\psi:(((o^4,o^3),o^2),o^1) \sentail \psi (\lambda \varphi .
\varphi a) : o^0$ for some constant $a:o$, where $\varphi:(o,o)$.
Its strategy denotation contains plays whose underlying sequence of
moves is $q_0 \, q_1 \, q_2 \, q_3 \, q_2 \, q_3 \, q_4$. Since
$q_4$ is an O-move, it is not constrained by P-incremental
justification and thus it can point to any of the two occurrences of
$q_3$.\footnote{More generally, a P-incrementally justified strategy
can contain plays that are not ``O-incrementally justified'' since
it must take into account any possible strategy incarnating its
context, including those that are not P-incrementally justified. For
instance in the given example, there is one version of the play that
is not O-incrementally justified (the one where $q_4$ points to the
first occurrence of $q_3$). This play is involved in the strategy
composition $\sem{ \stentail M_2 : (((o,o),o),o)} ; \sem{
\psi:(((o,o),o),o) \stentail \psi (\lambda \varphi . \varphi a):o}$
where $M_2$ denotes the unsafe Kierstead term.}


\subsection*{Towards a fully abstract game model}\hfill

The standard game models which have been shown to be fully abstract
for PCF \cite{abramsky94full,hylandong_pcf} are of course also fully
abstract for the restricted language safe PCF. One may ask, however,
whether there exists a fully abstract model with respect to safe
context only.

Such model may be obtain by considering P-incrementally justified strategies
- which have been shown to compose in \cite{Blumphd}. Its is reasonable to think that
 O-moves also needs to be constrained by the symmetrical O-incremental justification, which corresponds to the requirement that contexts are safe. This line of work is still in progress.


\subsection*{Safe PCF and safe Idealised Algol}

\pcf\ is the simply-typed lambda calculus augmented with basic
arithmetic operators, if-then-else branching and a family of
recursion combinator $Y_A : ((A,A),A)$ for any type $A$.  We define
\emph{safe} \pcf\ to be \pcf\ where the application and abstraction
rules are constrained in the same way as the safe lambda calculus.
This language inherits the good properties of the safe lambda
calculus: No variable capture occurs when performing substitution
and safety is preserved by the reduction rules of the small-step
semantics of \pcf.

\subsubsection{Correspondence}

The computation tree of a \pcf\ term is defined as the least
upper-bound of the chain of computation trees of its \emph{syntactic
approximants} \cite{abramsky:game-semantics-tutorial}.  It is
obtained by infinitely expanding the $Y$ combinator, for instance
$\tau(Y (\lambda f x. f x))$ is the tree representation of the
$\eta$-long form of the infinite term $(\lambda f x. f x)
 ((\lambda f x. f x) ((\lambda f x. f x) ( \ldots$

It is straightforward to define the traversal rules modeling the
arithmetic constants of \pcf. Just as in the safe lambda calculus we
had to remove @-nodes in order to reveal the game-semantic
correspondence, in safe \pcf\ it is necessary to filter out the
constant nodes from the traversals. The Correspondence Theorem for
\pcf\ says that the interaction game semantics is isomorphic to the
set of traversals disposed of these superfluous nodes. This can
easily be shown for term approximants. It is then lifted to full
\pcf\ using the continuity of the function $\travset(\_)^{\filter
\theroot}$ from the set of computation trees (ordered by the
approximation ordering) to the set of sets of justified sequences of
nodes (ordered by subset inclusion). Finally computation trees of
safe \pcf\ terms are incrementally-bound thus we have
%Computation trees of safe \pcf\ terms are incrementally-bound.
%Moreover since \pcf\ constant are of order $1$ at most, the constant
%traversal rules are all \emph{well-behaved} (Lemma
%\ref{lem:sigma_order1_are_wellbehaved}) hence Lemma
%\ref{lem:betanf_wellbehavedconst_trav_pview_red} (from the Appendix)
%still holds and the game-semantic analysis of safety remains valid
%for \pcf. Hence we have:
\begin{theorem}
\label{thm:safepcfpincr} Safe PCF terms have P-incrementally
justified denotations. \qed
\end{theorem}


Similarly, we can define safe \ialgol\ to be safe \pcf\ augmented
with the imperative features of Idealized Algol (\ialgol\ for short)
\cite{Reynolds81}.  Adapting the game-semantic correspondence and
safety characterization to \ialgol\ seems feasible although the
presence of the base type \iavar, whose game arena $\iacom^{\nat}
\times \iaexp$ has infinitely many initial moves, causes a mismatch
between the simple tree representation of the term and its game
arena. It may be possible to overcome this problem by replacing the
notion of computation tree by a ``computation directed acyclic
graph''.

The possibility of representing plays \emph{without some or all of
  their pointers} under the safety assumption suggests potential
applications in algorithmic game semantics. Ghica and McCusker
\cite{ghicamccusker00} were the first to observe that pointers are
unnecessary for representing plays in the game semantics of the
second-order finitary fragment of Idealized Algol ($\ialgol_2$ for
short). Consequently observational equivalence for this fragment can
be reduced to the problem of equivalence of regular expressions.  At
order $3$, although pointers are necessary, deciding observational
equivalence of $\ialgol_3$ is EXPTIME-complete
\cite{DBLP:journals/apal/Ong04,DBLP:conf/fossacs/MurawskiW05}.
Restricting the problem to the safe fragment of $\ialgol_3$ may lead
to a lower complexity.

% (note that it is unlikely to obtain the complexity PSPACE because the
% set of complete plays of the safe term $\lambda f^{(o,o),o} . f
% (\lambda x^o . x)$ is not regular \cite{DBLP:journals/apal/Ong04}).

% Murawski showed the undecidability of program equivalence in
% $\ialgol_i$ for $i\geq4$ by encoding Turing machine computations
% into a finitary $IA_4$ term \cite{murawski03program}. The term
% constructed being not safe, the proof cannot be transposed to the
% safe fragments. Hence the question remains of whether observational
% equivalence is decidable for the \emph{safe} fragments of these
% language.

%In \cite{Ong02}, one of us showed that observational equivalence for
% finitary second-order \ialgol\ with recursion ($\ialgol_2 + Y_1$) is
% undecidable. The proof consists in reducing the Queue-Halting
% problem to the observational equivalence of two $\ialgol_2 + Y_1$
% terms. The same reduction is still valid in the safe fragment of
% $\ialgol_2 + Y_1$.  Consequently, observational equivalence of safe
% $\ialgol_2 + Y_1$ is also undecidable.


    \section{Further Questions}

\begin{itemize}
\item What are the function over free-algebras definable in the
safe simply-typed lambda calculus?
\cite{DBLP:journals/tcs/Leivant93,DBLP:journals/apal/Zaionc91}
  

\end{itemize}


    \bibliographystyle{abbrv} % {nat} % {splncs}
    \bibliography{../bib/dphil-all}

    \section{Appendix -- Computation tree, traversals and correspondence}
    \label{sec:correspondence}
    In \cite{OngLics2006}, one of us introduced the notion of
computation tree and traversals over a computation tree for the
purpose of studying trees generated by higher-order recursion
scheme. Here we extend these concepts to the simply-typed lambda
calculus. Our setting allows the presence of free variables of any
order and the term studied is not required to be of ground type.
(This contrasts with \cite{OngLics2006}'s setting where the term is
of ground type and contains only \emph{uninterpreted
constant}\footnote{A constant $f$ is \emph{uninterpreted} if the
small-step semantics of the language
  does not contain any rule of the form $f \dots \rightarrow e$. $f$
  can be regarded as a data constructor.} of order 1 at most and no
free variables.) Note that we automatically accounts for the
presence of uninterpreted constants since they can just be regarded
as free variables. We will then state the \emph{Correspondence
Theorem} (Theorem \ref{thm:correspondence}) that was used in Sec.
\ref{sec:gamesemaccount}.

In the following we fix a simply typed term $\Gamma \stentail M :T$
(not necessarily safe) and we consider its computation tree
$\tau(M)$ as defined in Def.\ \ref{dfn:comptree}.

\subsection{Notations}
We first fix some notations. We write $\theroot$ to denote the root
of the computation tree $\tau(M)$. The set of nodes of the
computation tree is denoted by $N$. The sets $N_@$, $N_\lambda$ and
$N_{\sf var}$ are respectively the subset of @-nodes,
$\lambda$-nodes and variable nodes. For $n \in N$ we will write
$\kappa(n)$ to denote the subterm of $\elnf{M}$ corresponding to the
subtree of $\tau(M)$ rooted at $n$. In particular $\kappa(\theroot)
= \elnf{M}$. The \defname{type} of a variable-labelled node is the
type of the variable which labels it; the type of the root is
$(A_1,\ldots,A_p, T)$ where $x_1:A_1,\ldots, x_p:A_p$ are the
variables in the context $\Gamma$; and the type of a node $n\in
(N_\lambda \union N_@) \setminus \{ \theroot \}$ is the type of the
corresponding subterm $\kappa(n)$.


\subsection{Pointers and justified sequences of nodes}

We define the \defname{enabling relation} on the set of nodes of the
computation tree as follows: $m$ enables $n$, written $m \enable n$,
if and only if $n$ is bound by $m$ (and we sometimes write $m
\enable_i n$ to precise that $n$ is the $i^{\sf th}$ variable bound
by $m$); or $m$ is the root $\theroot$ and $n$ is a free variable;
or $n$ is a $\lambda$-node and $m$ is its parent node.


We say that a node $n_0$ of the computation tree is
\defname{hereditarily enabled} by $n_p \in N$ if there are nodes
$n_1,\ldots, n_{p-1} \in N$ such that $n_{i+1}$ enables $n_{i}$ for
all $i\in 0..p-1$.

For any set of nodes $S, H \subseteq N$ we write $S^{H\enable}$ for
$S \inter \enable^*(H) = \{ n \in S \ | \exists n_0 \in H \mbox{
s.t. }n_0  \enable^* n \}$ -- the subset of $S$ constituted of nodes
hereditarily enabled by some node in $H$. We will abbreviate
$S^{\{n_0\}\enable}$ into $S^{n_0\enable}$.

We call \defname{input-variables nodes} the elements of $V_{\sf
var}^{\theroot\enable}$ \ie variables that are hereditarily enabled
by the root of $\tau(M)$. Thus we have $V_{\sf
var}^{\theroot\enable} = V \setminus ( V_{\sf var}^{N_@\enable}
\union V_{\sf var}^{N_\Sigma\enable})$.

A \defname{justified sequence of nodes} is a sequence of nodes with
pointers such that each occurrence of a variable or $\lambda$-node
$n$ different from the root has a pointer to some preceding
occurrence $m$ verifying $m \enable n$. In particular, occurrences
of @-nodes do not have pointer. We represent the pointer in the
sequence as follows \Pstr[0.4cm]{ (m){m} \ldots (n-m,45:i) n }.
 where the label indicates that either $n$ is labelled with the $i$th variable
abstracted by the $\lambda$-node $m$ or that $n$ is the $i^{\sf th}$
child of $m$.  Children nodes are numbered from $1$ onward except for
@-nodes where it starts from $0$. Abstracted variables are numbered
from $1$ onward. The $i^{\sf th}$ child of $n$ is denoted by $n.i$.

We say that a node $n_0$ of a justified sequence is
\defname{hereditarily justified} by $n_p$ if there are occurrences $n_1,
\ldots, n_{p-1}$ in the sequence such that $n_i$ points to $n_{i+1}$
for all $i\in 0..p-1$. For any occurrence $n$ in a justified
sequence $s$, we write $s \filter n$ to denote the subsequence of
$s$ consisting of occurrences that are hereditarily justified by
$n$.


The notion of \defname{P-view} $\pview{t}$ of a justified sequence of
nodes $t$ is defined the same way as the P-view of a justified
sequences of moves in Game Semantics:\footnote{ The equalities in the
  definition determine pointers implicitly. For instance in the second
  clause, if in the left-hand side, $n$ points to some node in $s$
  that is also present in $\pview{s}$ then in the right-hand side, $n$
  points to that occurrence of the node in $\pview{s}$.}
$$\begin{array}{rclrcl}
\pview{\epsilon} &=&  \epsilon
& \pview{\Pstr[0.5cm]{ s \cdot (m) m \cdot \ldots \cdot (lmd-m,40){\lambda\overline{\xi}}
}}
 &=& \Pstr{
\pview{s} \cdot (m2) m \cdot (lm2-m2,50) {\lambda \overline{\xi}} } \\
\mbox{for $n \notin N_\lambda$, } \pview{s \cdot n }  &=&  \pview{s} \cdot n \qquad
& \pview{s \cdot \theroot }  &=&  \theroot
\end{array}$$

The O-view of $s$, written $\oview{s}$, is defined dually. We will
borrow the game semantic terminology: A justified sequences of nodes
satisfies \defname{alternation} if for any two consecutive nodes one
is a $\lambda$-node and the other is not, and \defname{P-visibility}
if every variable node points to a node occurring in the P-view a
that point.

\subsection{Computation tree with value-leaves}


We now add another ingredient to the computation tree that was not
originally used in \cite{OngLics2006}.  We write $\mathcal{D}$ to
denote the set of values of the base type $o$.  We add
\defname{value-leaves} to $\tau(M)$ as follows: For each value $v
\in \mathcal{D}$ and for each node $n \in N$ we attach the child
leaf $v_n$ to $n$.  We write $V$ for the set of vertices of the
resulting tree (\ie inner nodes and leaf nodes). For $\$$ ranging in
$\{@, \lambda, var \}$, we write $V_\$$ to denote the set of inner
nodes from $N_\$$ plus the leaf-nodes whose parent is in $N_\$$ \ie
$V_\$ = N_\$ \union \{ v_n \ | \ n \in N_\$, v \in \mathcal{D} \}$.


Everything that we have defined can be lifted to this new version of
computation tree. A value-leaf has order $0$. The enabling relation
$\enable$ is extended so that every leaf is enabled by its parent
node. A link going from a value-leaf $v_n$ to a node $n$ is labelled
by $v$: \Pstr[0.4cm]{ (n) n \ldots (vn-n,35:v){v_n} }. For the
definition of P-view and visibility, value-leaves are treated as
$\lambda$-nodes if they are at an odd level in the computation tree,
and as variable nodes if they are at an even level.

We say that an occurrence of an inner node $n \in N$ is
\defname{answered} by an occurrence $v_n$ if $v_n$ in
the sequence that points to $n$, otherwise we say that $n$ is
\defname{unanswered}. The last unanswered node is called the
\defname{pending node}.  A justified sequence of nodes is
\defname{well-bracketed} if each value-leaf occurring in it is justified by the pending node at that point.  If $t$ is a traversal then we write
$?(t)$ to denote the subsequence of $t$ consisting only of
unanswered nodes.

\subsection{Traversals of the computation tree}
\label{subsec:traversal}

A \emph{traversal} is a justified sequence of nodes of the computation tree where each node
indicates a step that is taken during the evaluation of the term.

\begin{definition}[Traversals for simply-typed $\lambda$-terms] \rm
\label{def:traversal} The set $\travset(M)$ of \defname{traversals}
over $\tau(M)$ is defined by induction over the rules of Table
\ref{tab:trav_rules}. A traversal that cannot be extended by any
rule is said to be \emph{maximal}.
\end{definition}
\begin{FramedTable}
\noindent {\bf Initialization rules}
\begin{itemize}[]
\item\rulenamet{Empty} $\epsilon \in \travset(M)$.
\item\rulenamet{Root} The sequence constituted of a single occurrence of $\tau(M)$'s root is a traversal.
\end{itemize}

\noindent {\bf Structural rules}
\begin{itemize}[]
    \item \rulenamet{Lam} If $t \cdot \lambda \overline{\xi}$ is a traversal then so is
        $t \cdot \lambda \overline{\xi} \cdot n$ where $n$
        denotes $\lambda \overline{\xi}$'s child and:
        \begin{compactitem}
            \item If $n \in N_@ \union N_\Sigma$ then it has no justifier;
            \item if  $n \in N_{\sf var}\setminus N_{\sf fv}$ then it points to the only occurrence\footnote{Prop.\ \ref{prop:pviewtrav_is_path} shows that P-views
            are paths in the tree thus $n$'s enabler occurs
            exactly once in the P-view.} of its enabler in
            $\pview{t\cdot \lambda \overline{\xi}}$;
            \item if  $n \in N_{\sf fv}$ then it points
            to the only occurrence of the root $\theroot$ in
            $\pview{t \cdot \lambda \overline{\xi}}$.
        \end{compactitem}
    \item \rulenamet{App} If $t \cdot @$ is a traversal then so is \Pstr[0.4cm]{t \cdot (m) @  \cdot (n-m,40:0) n}.
\end{itemize}

\emph{\bf Input-variable rules}
\begin{itemize}[]
\item \rulenamet{InputVar} If $t$ is a traversal where $t^\omega \in N_{\sf var}^{\theroot\enable} \union L_\lambda^{\theroot\enable}$
and $x$ is an occurrence of a variable node in $\oview{t}$ then
so is $t \cdot n$ for any child $\lambda$-node $n$ of $x$, $n$
pointing to $x$.



\item \rulenamet{InputValue} If $t_1
\cdot x \cdot t_2$ is a traversal with pending node $x \in
N_{\sf var}^{\theroot\enable}$ then so is \Pstr[0.5cm]{t_1 \cdot
(x){x} \cdot t_2 \cdot (xv-x,38:v){v_x} } for all $v \in
\mathcal{D}$.
\end{itemize}

\emph{\bf Copy-cat rules}
\begin{itemize}[]
\item\rulenamet{Var}
If \Pstr[0.5cm]{t \cdot (n){n} \cdot (lx){\lambda \overline{x}}
    \ldots (x-lx,50:i){x_i} } is a traversal where $x_i \in
    N_{\sf var}^{@\enable}$ then so is \Pstr[0.5cm]{ t \cdot
(n){n} \cdot (lx){\lambda \overline{x}}  \ldots (x-lx,30:i){x_i}
    \cdot (letai-n,40:i){\lambda \overline{\eta_i}}
     }.

\item\rulenamet{Value}
  If \Pstr{t \cdot (m){m} \cdot (n){n}  \ldots
(vn-n,60:v){v}_{n} } is a traversal where $n\in N$ then so is
\Pstr[0.6cm]{t \cdot (m){m} \cdot (n){n} \ldots
(vn-n,60:v){v}_{n} \cdot (vm-m,45:v){v}_m}.
\end{itemize}
\caption[Traversal rules for the simply-typed
lambda-calculus]{Traversal rules for the simply-typed
$\lambda$-calculus.} \label{tab:trav_rules}
\end{FramedTable}

\parpic[r]{
 $\pssetcomptree\tree[levelsep=3ex,treesep=0.5cm]{\lambda} {
    \tree{@}{
        \pstree[linestyle=dotted]{\TR{\lambda y}\arclabel{0} }{
            \tree{y}{
                \tree{\lambda \overline{\eta_1}}{\vdots}%\arclabel{1}
                \tree{\lambda \overline{\eta_i}}{\vdots}%\arclabel{i}
                \tree{\lambda \overline{\eta_n}}{\vdots}%\arclabel{n}
            }
        }
        \pstree[linestyle=dotted]{\TR{\lambda \overline{x}}
            \arclabel{1}}{ \tree{x_i}{\TR{} \TR{}}}
}}$ } A traversal always starts by visiting the root. Then it mainly
follows the structure of the tree. The (Var) rule permits to jump
across the computation tree. The idea is that after visiting a
variable node $x$, a jump is allowed to the node corresponding to
the subterm that would be substituted for $x$ if all the
$\beta$-redexes occurring in the term were reduced. The sequence
\Pstr[0.8cm]{\lambda \cdot (app) @  \cdot (ly) {\lambda y}  \ldots
(y-ly,35:1) y  \cdot (lx-app,38:1) {\lambda \overline{x}} \ldots
(x-lx,30:i) {x_i} \cdot (leta-y,40:i) {\lambda \overline{\eta_i} }
\ldots}
 is an example of traversal of the computation tree shown on the right.

\begin{proposition}[counterpart of proposition 6 from \cite{OngHoMchecking2006}]
\label{prop:pviewtrav_is_path}
Let $t$ be a traversal. Then:
\begin{enumerate}[(i)]
\item $t$ is a well-defined and well-bracketed justified sequence;
\item $t$ is a well-defined justified sequence verifying alternation, P-visibility and O-visibility;
\item If $t$'s last node is not a value-leaf, then $\pview{t}$ is the path in the computation tree going from the root to $t$'s last node.
\end{enumerate}
\end{proposition}

The \defname{reduction} of a traversal $t$ is the subsequence of $t$
obtained by keeping only occurrences of nodes that are hereditarily
enabled by the root $\theroot$. This has the effect of eliminating
the ``internal nodes'' of the computation. If $t$ is a non-empty
traversal then the root $\theroot$ occurs exactly once in $t$ thus
the reduction of $t$ is equal to $t \filter r$ where $r$ is the
first occurrence in $t$ (the only occurrence of the root). We write
$\travset(M)^{\filter \theroot}$ for the set or reductions of
traversals of $M$.

Application nodes are used to connect the operator and the operand
of an application in the computation tree but since they do not play
any role in the computation of the term, we can remove them from the
traversals.  We write $t-@$ for the sequence of nodes-with-pointers
obtained by removing from $t$ all @-nodes and value-leaves of
@-nodes, any link pointing to an @-node being replaced by a link
pointing to the immediate predecessor of @ in $t$. We write
$\travset(M)^{-@}$ for the set $\{ t - @ \ | \  t \in \travset(M)
\}$.
\begin{remark}
Clearly if $M$ is $\beta$-normal then $\tau(M)$ does not contain any
@-node therefore all nodes are hereditarily enabled by the root and
we have $\travset(M)^{-@} = \travset(M) = \travset(M)^{\filter
\theroot }$.
\end{remark}



\begin{lemma}
% version of the lemma for the pure simply-typed lambda calculus
\label{lem:betanf_trav_pview_red} Suppose that $M$ is a
$\beta$-normal simply-typed term. Let $t$ be a non-empty traversal
of $M$ and $r$ denote the only occurrence of $\tau(M)$'s root in
$t$. If $t$'s last occurrence is not a leaf then
$$ \pview{t} \filter r = \pview{?(t) \filter  r }\ .$$
\end{lemma}
In the lambda calculus without interpreted constants this lemma
follows immediately from the fact that $\travset(M) =
\travset(M)^{\filter \theroot}$. But this Lemma remains valid in the
presence of interpreted constants provided that the traversal rules
implementing the constants are \emph{well-behaved}\footnote{A
traversal rule is \defname{well-behaved} if it can be stated under
the form ``$t = t_1\cdot n \cdot t_2 \in \travset(M)\ \zand\ ?(t) =
?(t_1) \cdot n \zand n \in N_{\Sigma}\union N_{\sf var} \zand P(t)
\zand m\in S(t) \imp\ $\Pstr{ t_1\cdot (n){n} \cdot t_2 \cdot
(m-n,25){m} \in {\travset(M)}}'' for some expression $P$ expressing
a condition on $t$ and function $S$ mapping traversals of the form
of $t$ to a subset of the children of $n$.}.

\subsection{Computation trees and arenas}
We consider the well-bracketed game model of the simply-typed lambda
calculus.  We choose to represent strategies using ``prefix-closed
set of plays''.\footnote{In the literature, a strategy is commonly
defined as a set of plays closed by taking a prefix of \emph{even}
length. However for the purpose of showing the correspondence with
traversals, the ``prefix-closed''-based definition is more
adequate.} We fix a term $\Gamma \stentail M : T$ and write
$\sem{\Gamma \stentail M : T}$ for its strategy denotation. The
answer moves of a question $q$ are written $v_q$ where $v$ ranges in
$\mathcal{D}$.

\begin{proposition}
There exists a function $\varphi_M$, constructible from $\tau(M)$,
that maps nodes from $V\setminus (V_@ \union V_\Sigma)$ to moves of
the interaction arena underlying the revealed strategy
$\syntrevsem{\Gamma \stentail M : T}$ and such that
\begin{itemize}
\item $\varphi$ maps $\lambda$-nodes to O-questions, variable
nodes to P-questions, value-leaves of $\lambda$-nodes to
P-answers and value-leaves of variable nodes to O-answers.

\item $\varphi$ maps nodes of a given order to moves of
the same order.
\end{itemize}
\end{proposition}
If $t = t_0 t_1 \ldots$ is a justified sequence of nodes in
$V_\lambda \union V_{\sf var}$ then $\varphi(t)$ is defined to
be the sequence of moves $\varphi(t_0)\ \varphi(t_1) \ldots$
equipped with the pointers of $t$.


\begin{example}
Take $\lambda x . (\lambda g . g x) (\lambda y . y)$ with $x,y:o$ and $g:(o,o)$.
The diagram below represents the computation tree (middle), the arenas
$\sem{(o,o), o}$ (left), $\sem{o , o}$ (right), $\sem{o\rightarrow o}$ (rightmost)
and $\varphi = \psi \union \psi_{\lambda g.g x}^{\lambda g, q_{\lambda g}} \union
\psi_{\lambda y.y}^{\lambda y, q_{\lambda y}}$
(dashed-lines).
$$\psset{levelsep=3.5ex}
\pstree{\TR[name=root]{\lambda x}}
{
    \pstree{\TR[name=App]{@}}
    {
            \pstree{\TR[name=lg]{\lambda g}}
                { \pstree{\TR[name=lgg]{g}}{
                        \pstree{\TR[name=lgg1]{\lambda}}
                        { \TR[name=lgg1x]{x}  } } }
            \pstree{\TR[name=ly]{\lambda y}}
                    {\TR[name=lyy]{y}}
    }
}
\rput(4.5cm,-1cm){
  \pstree{\TR[name=A1lx]{q_{\lambda x}}}
        { \TR[name=A1x]{q_x} }
}
\rput(-6cm,-1.5cm){
    \pstree{\TR[name=A2lg]{q_{\lambda g}}}
    {
        \pstree{\TR[name=A2g]{q_g}}
        {  \TR[name=A2g1]{q_{g_1}}   }
    }}
\rput(2.5cm,-1.5cm){
    \pstree{\TR[name=A3ly]{q_{\lambda y}}}
        { \TR[name=A3y]{q_y}
        }
}
\psset{nodesep=1pt,arrows=->,arcangle=-20,arrowsize=2pt 1,linestyle=dashed,linewidth=0.3pt}
\ncline{->}{root}{A1lx} \mput*{\psi}
\ncarc{->}{lgg1x}{A1x}
\ncline{->}{lg}{A2lg} \mput*{\psi_{\lambda g.g x}^{\lambda g, q_{\lambda g}}}
\ncline{->}{lgg}{A2g}
\ncline{->}{lgg1}{A2g1}
\ncline{->}{ly}{A3ly} \mput*{\psi_{\lambda y.y}^{\lambda y, q_{\lambda y}}}
\ncline{->}{lyy}{A3y}
$$
\end{example}


\subsection{The Correspondence Theorem}

In game semantics, strategy composition is performed using a
CSP-like ``composition + hiding''. If some of the internal moves are
not hidden then we obtain alternative denotations called
\defname{revealed semantics} in \cite{willgreenlandthesis} and
\emph{interaction} semantics in \cite{DBLP:conf/sas/DimovskiGL05}.
We obtain different notions of revealed semantics depending on the
choice of internal moves that we hide. For instance the
\defname{fully revealed denotation} of $\Gamma \stentail M
:T$, written $\revsem{\Gamma \stentail M : T}$, is obtained by
uncovering all the internal moves from $\sem{\Gamma \stentail M :
T}$ that are generated during composition.\footnote{An algorithm
that uniquely recovers hidden moves from $\sem{\Gamma \stentail M :
T}$ is given in Part II of
  \cite{hylandong_pcf}.} The inverse operation consists in filtering out the internal moves.

The \defname{syntactically-revealed denotation}, written
$\syntrevsem{\Gamma \stentail M : T}$, differs from the
fully-revealed one in that only certain internal moves are preserved
during composition: when computing the denotation of an application
joint by an @-node in the computation tree, all the internal moves
are preserved. When computing the denotation of $\revsem{y_i N_1
\ldots N_p}$ for some variable $y_i$, however, we only preserve the
internal moves of $N_1$, \ldots, $N_p$ while omitting the internal
moves produced by the copy-cat projection strategy denoting $y_i$.


In the simply-typed lambda calculus, the set $\travset(M)$ of
traversals of the computation tree is isomorphic to the
syntactically-revealed denotation, and the set of traversal
reductions is isomorphic to the standard strategy denotation:
\begin{theorem}[The Correspondence Theorem]
\label{thm:correspondence} $\varphi_M$ gives us the following two
isomorphisms:
\begin{eqnarray*}
(i)~\varphi_M  &: \travset(M)^{-@} \stackrel{\cong}{\longrightarrow} \syntrevsem{\Gamma \stentail M :T} \\
(ii)~\varphi_M  &: \travset(M )^{\filter \theroot} \stackrel{\cong}{\longrightarrow} \sem{\Gamma \stentail M : T} \ .
\end{eqnarray*}
\end{theorem}

\begin{example}
Take $M = \lambda f z . (\lambda g x . f x) (\lambda y. y) (f z) :
((o,o),o, o)$.  The figure below represents the computation tree
(left tree), the arena $\sem{((o,o),o, o)}$ (right tree) and
$\psi_M$ (dashed line). (Only question moves are shown for clarity.)
The justified sequence of nodes $t$ defined hereunder is an example
of traversal:

\begin{tabular*}{0.9\textwidth}{@{\extracolsep{\fill}}p{6cm}p{7cm}}
$\pssetcomptree\pstree[levelsep=2.5ex,treesep=0.3cm]{ \TR[name=root]{\lambda f z} }
     {  \tree[levelsep=4ex]{@}
        {   \tree{\lambda g x}{
                  \pstree{\TR[name=f]{f^{[1]}}}{
                            \pstree{\TR[name=lmd]{\lambda^{[2]}}}
                                {\TR{x}}
                  }
                }
            \tree{\lambda y }{\TR{y}}
            \tree{\lambda ^{[3]}}{
                \pstree{\TR[name=f2]{f^{[4]}}} {
                \pstree{\TR[name=lmd2]{\lambda^{[5]}}}{\TR[name=z]{z}}
                }
            }
        }
     }
\hspace{2cm}
  \pstree[levelsep=8ex, treesep=0.3cm]{ \TR[name=q0]{q^0} }
    {   \pstree[levelsep=4ex]{\TR[name=q1]{q^1}} {\TR[name=q2]{q^2}}
        \TR[name=q3]{q^3}
    }
\psset{nodesep=1pt,arrows=->,arrowsize=2pt 1,linestyle=dashed,linewidth=0.3pt}
\ncline{->}{root}{q0} \mput*{\psi_M}
\ncarc[arcangle=-25]{->}{z}{q3}
\ncarc[arcangle=10]{->}{f}{q1}
\ncarc[arcangle=10]{->}{lmd}{q2}
\ncline{->}{f2}{q1}
\ncline{->}{lmd2}{q2}$
&
\begin{asparablank}
  \item  \Pstr[0.8cm]{
t = (n){\lambda f z} \
(n2){@} \
(n3-n2,60){\lambda g x} \
(n4-n,45){f^{[1]}} \
(n5-n4,45){\lambda^{[2]}} \
(n6-n3,45){x} \
(n7-n2,35){\lambda^{[3]}} \
(n8-n,35){f^{[4]}} \
(n9-n8,45){\lambda^{[5]}} \
(n10-n,35){z}
}

\item \Pstr[0.9cm]{
t\filter r = (n){\lambda f z} \ (n4-n,50){f}^{[1]} \
(n5-n4,60){\lambda}^{[2]} \ (n8-n,45){f}^{[4]} \
(n9-n8,60){\lambda}^{[5]} \ (n10-n,40){z}}
\item
\Pstr[0.8cm]{ {\varphi_M(t\filter r) =\ } (n){q^0}\ (n4-n,60){q^1}\
(n5-n4,60){q^2}\ (n8-n,45){q^1}\ (n9-n8,60){q^2}\ (n10-n,38){q^3}
\in \sem{M}\ .}
\end{asparablank}
\end{tabular*}
\end{example}


\end{document}
