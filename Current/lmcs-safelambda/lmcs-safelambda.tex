% Commands to be executed by LatexDaemon
%Daemon> ini=latex
%Daemon> afterjob=dvipspdf
%Daemon> filter=err+warn
%Daemon> custom_args=-synctex=-1

%%%%%%%%%%%%%%%%%%%%%%%%%%%%%%%%%%%%%%%%%%%%%%%%%%%%%%%%%%%%%%%%%%%%%%%%%%
%% Modification log
%%
%% 2004/03/25 v0.1 based on amsart.cls, inspired by jair.sty 
%% 2004/09/01 v0.2 based on amsart.cls
%% 2004/10/12 v0.3 based on amsart.cls
%% 2004/12/16 v0.4 based on amsart.cls
%% 2005/01/24 v0.5 based on amsart.cls
%% 2005/03/10 v0.6 based on amsart.cls
%% 2006/07/24 v0.7 based on amsart.cls
%%                
%%                 Juergen Koslowski, Stefan Milius
%%
%%%%%%%%%%%%%%%%%%%%%%%%%%%%%%%%%%%%%%%%%%%%%%%%%%%%%%%%%%%%%%%%%%%%%%%%%%%

\NeedsTeXFormat{LaTeX2e}
\ProvidesClass{lmcs}
              [2006/07/24 v0.7 LMCS Layout Editor Class]
\DeclareOption*{\PassOptionsToClass{\CurrentOption}{amsart}}
\ProcessOptions\relax

\LoadClass[11pt,reqno]{amsart}
\usepackage{helvet}
%%%%%%%%%%%%%%%%%%%%%%%%%%%%%%%%%%%%%%%%%%%%%%%%%%%%%%%%%%%%%%%%%%%%%%%%%%%
%   		Use of this class, cf. also lmcs-smp.tex
%%%%%%%%%%%%%%%%%%%%%%%%%%%%%%%%%%%%%%%%%%%%%%%%%%%%%%%%%%%%%%%%%%%%%%%%%%%
%
% This class builds upon the amsart class of AMS-LaTeX and requires use
% of LaTeX 2e. 
%
%%%%%%%%%%%%%%%%%%%%%%%%%%%%%%%%%%%%%%%%%%%%%%%%%%%%%%%%%%%%%%%%%%%%%%%%%%%
% 		Start of the paper
%%%%%%%%%%%%%%%%%%%%%%%%%%%%%%%%%%%%%%%%%%%%%%%%%%%%%%%%%%%%%%%%%%%%%%%%%%%
%
% \documentclass{lmcs} 
%
% without any options followed optionally by
%
% \usepackage{package1,package2,...}
%
% loading additional macro packages you may wish to use, (eg, xypic, etc.)
% (This also is the place to define further theorem-environments, in case
% those provided by default do not suffice, cf. below.)
%
% \begin{document}
% \title[short_title]{real title}
%
% and a list of author information of the form
%
% \author[short_author1]{Author 1}
% \address{address 1}
% \email{author1@email1}
% \thanks{thanks 1}
%
% \author[short_author2]{Author 2}
% \address{address 2}
% \email{author2@email2}
% \thanks{thanks 2}
%
% \author[short_author3]{Author 3}
% \address{address 3}
% \email{author3@email3}
% \thanks{thanks 3}
%
% The \email and \thanks fields are optional.  The \thanks fields appear
% in footnotes on the title page, the addresses and email information
% is relegated to the end of the article.  The optional arguments to
% the \title and \author macros determine a running head on the odd
% and even pages, respectively.
%
% Lists of keywords and phrases as well as an ACM Subject
% classification are mandatory; these appear in footnotes on the
% title page, preceeding any \thanks fields.
%
%%%%%%%%%%%%%%%%%%%%%%%%%%%%%%%%%%%%%%%%%%%%%%%%%%%%%%%%%%%%%%%%%%%%%%%%%%%
%		Body of the paper
%%%%%%%%%%%%%%%%%%%%%%%%%%%%%%%%%%%%%%%%%%%%%%%%%%%%%%%%%%%%%%%%%%%%%%%%%%%
%
% We encourage the use of LaTeX's crossreferencing capabilities with the
% \label and \ref commands, for sections, subsections, theorems etc., and
% displayed equations and figures.
%
%%%%%%%%%%%%%%%%%%%%%%%%%%%%%%%%%%%%%%%%%%%%%%%%%%%%%%%%%%%%%%%%%%%%%%%%%%%
% 		Theorems, Definitions etc.
%%%%%%%%%%%%%%%%%%%%%%%%%%%%%%%%%%%%%%%%%%%%%%%%%%%%%%%%%%%%%%%%%%%%%%%%%%%
%
% The following theorem-like environments are available.  The
% numbering is consecutive within sections.
%
% thm   Theorem
% cor   Corollary
% lem   Lemma
% prop  Proposition
% asm   Asumption
%
% defi  Definition
% rem   Remark
% rems  Remarks (intended for use with itemized remarks)
% exa   Example
% exas  Examples (intended for use with itemized examples)
% conj  Conjecture
% prob  Problem
% oprob Open Problem
% algo  Algorithm
% obs   Observation
%
% If you require additional environments, you can add them before
% \begin{document} by means of
%
% \theoremstyle{plain}\newtheorem{env}[thm]{Environment}
%
% or
%
% \theoremstyle{definition}\newtheorem{env}[thm]{Environment}
%
% In the first case the font within the new environment will be italicised.
%
%%%%%%%%%%%%%%%%%%%%%%%%%%%%%%%%%%%%%%%%%%%%%%%%%%%%%%%%%%%%%%%%%%%%%%%%%%%
%		Proofs
%%%%%%%%%%%%%%%%%%%%%%%%%%%%%%%%%%%%%%%%%%%%%%%%%%%%%%%%%%%%%%%%%%%%%%%%%%%
%
% Proofs start with the command \proof and should be ended by \qed,
% which provides an end-of-proof box at the right margin:
%
% \proof ... \qed
%
% In itemized or enumerated proofs the \qed command has to occur BEFORE 
% \end{itemize} or \end{enumerate} to ensure proper placement of the box:
%
% \proof
% \begin{itemize}
% \item[(1)] ...
% \item[(2)] ...
% ...
% \item[(n)] ... \qed
% \end{itemize}
%
% Similarly, the box may be used within theorem environments, when no
% explicit proof is given:
%
% \begin{thm} ... \qed \end{thm}
%
%%%%%%%%%%%%%%%%%%%%%%%%%%%%%%%%%%%%%%%%%%%%%%%%%%%%%%%%%%%%%%%%%%%%
%		Itemized or enumerated environments and proofs
%%%%%%%%%%%%%%%%%%%%%%%%%%%%%%%%%%%%%%%%%%%%%%%%%%%%%%%%%%%%%%%%%%%%
%
% By default, the first item of an itemized (or enumerated) environment
% or proof appears inlined on the same line as the environment title.
% This can be prevented by placing \hfill before the itemization, e.g.:
%
% \begin{thm}\label{T:abc}\hfill
% \begin{itemize} ...
%
% \proof\hfill
% \begin{itemize} ...


%
%		End of the paper
%%%%%%%%%%%%%%%%%%%%%%%%%%%%%%%%%%%%%%%%%%%%%%%%%%%%%%%%%%%%%%%%%%%%
%
% Acknowledgements should be placed in a non-numbered section:
%
% \section*{Acknowledgement}
%
% The bibliography uses alpha.bst where references are built from the 
% authors' initials and the year of publication.  The use of bibtex
% for creating the bibliography is strongly encouraged.  Then the
% end of the paper takes the form
%
% \begin{thebibliography}{key}
% \end{thebibliography}
% \end{document}
%
% where ``key'' is the longest alphanumerical key expected to occur.
%
% Optionally, appendices can be inserted after the bibliography by
% means of
%
% \end{thebibliography}
% \appendix
% \section{} % Appendix A
% \section{} % Appendix B
% % etc.
% \end{document}


%%%%%%%%%%%%%%%%%%%%%%%%%%%%%%%%%%%%%%%%%%%%%%%%%%%%%%%%%%%%%%%%%%%%
%%%                   actual macros
%%%%%%%%%%%%%%%%%%%%%%%%%%%%%%%%%%%%%%%%%%%%%%%%%%%%%%%%%%%%%%%%%%%%

\count255=\the\catcode`\@ \catcode`\@=11 \edef\catc@de{\the\count255}

\def\titlecomment#1{\def\@titlecomment{#1}}
\let\@titlecomment=\@empty
\renewcommand{\sfdefault}{phv}
\renewcommand*\subjclass[2][1991]{%
  \def\@subjclass{#2}%
  \@ifundefined{subjclassname@#1}{%
    \ClassWarning{\@classname}{Unknown edition (#1) of ACM
      Subject Classification; using '1991'.}%
  }{%
    \@xp\let\@xp\subjclassname\csname subjclassname@2000\endcsname
  }%
}
\@namedef{subjclassname@2000}{2000 ACM Subject Classification}
\newcommand{\revisionname}{Revision Note}
\newbox\revisionbox
\newenvironment{revision}{%
  \ifx\maketitle\relax
    \ClassWarning{\@classname}{Revision should precede
      \protect\maketitle\space in LMCS documentclasses; reported}%
  \fi
  \global\setbox\revisionbox=\vtop \bgroup
    \normalfont\Small
    \list{}{\labelwidth\z@
      \leftmargin3pc \rightmargin\leftmargin
      \listparindent\normalparindent \itemindent\z@
      \parsep\z@ \@plus\p@
      \let\fullwidthdisplay\relax
    }%
    \item[\hskip\labelsep\scshape\revisionname.]%
}{%
  \endlist\egroup
  \ifx\@setrevision\relax \@setrevisiona \fi
}
\def\@setrevision{\@setrevisiona \global\let\@setrevision\relax}
\def\@setrevisiona{%
  \ifvoid\revisionbox
  \else
    \skip@20\p@ \advance\skip@-\lastskip
    \advance\skip@-\baselineskip \vskip\skip@
    \box\revisionbox
    \prevdepth\z@ % because \revisionbox is a vtop
    \bigskip\hrule\medskip
  \fi
}
\def\@setsubjclass{%
  {\itshape\subjclassname:}\enspace\@subjclass\@addpunct.}
\def\@setkeywords{%
  {\itshape \keywordsname:}\enspace \@keywords\@addpunct.}
\def\@settitlecomment{\@titlecomment\@addpunct.}
\def\@maketitle{%
  \normalfont\normalsize
  \let\@makefnmark\relax  \let\@thefnmark\relax
  \ifx\@empty\@date\else \@footnotetext{\@setdate}\fi
  \ifx\@empty\@subjclass\else \@footnotetext{\@setsubjclass}\fi
  \ifx\@empty\@keywords\else \@footnotetext{\@setkeywords}\fi
  \ifx\@empty\@titlecomment\else \@footnotetext{\@settitlecomment}\fi
  \ifx\@empty\thankses\else \@footnotetext{%
    \def\par{\let\par\@par}\@setthanks\par}\fi
  \@mkboth{\@nx\shortauthors}{\@nx\shorttitle}%
  \global\topskip12\p@\relax % 5.5pc   "   "   "     "     "
  \topskip42 pt\@settitle
  \ifx\@empty\authors \else \@setauthors \fi
  \@setaddresses
  \ifx\@empty\@dedicatory
  \else
    \baselineskip18\p@
    \vtop{\centering{\footnotesize\itshape\@dedicatory\@@par}%
      \global\dimen@i\prevdepth}\prevdepth\dimen@i
  \fi
  \endfront@text
  \bigskip\hrule\medskip
  \@setrevision
  \@setabstract
  \vskip-\bigskipamount
  \normalsize
  \if@titlepage
    \newpage
  \else
    \dimen@34\p@ \advance\dimen@-\baselineskip
    \vskip\dimen@\relax
  \fi
}
\def\@setaddresses{\par
  \nobreak \begingroup
\footnotesize
  \def\author##1{\nobreak\addvspace\bigskipamount}%
  \def\\{\unskip, \ignorespaces}%
  \interlinepenalty\@M
  \def\address##1##2{\begingroup
    \par\addvspace\bigskipamount\noindent\narrower
    \@ifnotempty{##1}{(\ignorespaces##1\unskip) }%
    {\ignorespaces##2}\par\endgroup}%
  \def\curraddr##1##2{\begingroup
    \@ifnotempty{##2}{\nobreak\indent{\itshape Current address}%
      \@ifnotempty{##1}{, \ignorespaces##1\unskip}\/:\space
      ##2\par}\endgroup}%
  \def\email##1##2{\begingroup
    \@ifnotempty{##2}{\nobreak\indent{\itshape e-mail address}%
      \@ifnotempty{##1}{, \ignorespaces##1\unskip}\/:\space
      {##2}\par}\endgroup}%
  \def\urladdr##1##2{\begingroup
    \@ifnotempty{##2}{\nobreak\indent{\itshape URL}%
      \@ifnotempty{##1}{, \ignorespaces##1\unskip}\/:\space
      \ttfamily##2\par}\endgroup}%
  \addresses
  \endgroup
}
\copyrightinfo{}{}

\newinsert\copyins
\skip\copyins=3pc
\count\copyins=1000 % magnification factor, 1000 = 100%
\dimen\copyins=.5\textheight % maximum allowed per page

\renewcommand{\topfraction}{0.95}   % let figure take up nearly whole page
\renewcommand{\textfraction}{0.05}  % let figure take up nearly whole page

%% Specify the dimensions of each page

\setlength{\oddsidemargin}{.25 in}  %   Note \oddsidemargin = \evensidemargin
\setlength{\evensidemargin}{.25 in}
\setlength{\marginparwidth}{0.07 true in}
\setlength{\topmargin}{-0.3 in}
\addtolength{\headheight}{1.84 pt}
\addtolength{\headsep}{0.25in}
\addtolength{\voffset}{0.7 in}
\setlength{\textheight}{8.5 true in}  % Height of text (including footnotes & figures)
\setlength{\textwidth}{6.0 true in}   % Width of text line.
\setlength{\parindent}{20 pt}   % Width of text line.
\widowpenalty=10000
\clubpenalty=10000
\@twosidetrue \@mparswitchtrue \def\ds@draft{\overfullrule 5pt}
\raggedbottom

%% Pagestyle

%% Defines the pagestyle for the title page.
%% Usage: \lmcsheading{vol}{issue}{year}{pages}{subm}{publ}{rev}{spec_iss}{title}

\def\lmcsheading#1#2#3#4#5#6#7{\def\ps@firstpage{\let\@mkboth\@gobbletwo%
\def\@oddhead{%
\hbox{%
  \vbox to 30 pt{\scriptsize\vfill
    \hbox{\textsf{Logical Methods in Computer Science}\hfil}
    \hbox{\textsf{Vol.~? (?:?) 2???, ? pages}}
    \hbox{\textsf{www.lmcs-online.org}}
    \rlap{\vrule width\hsize depth .4 pt}}}\hfill
\raise 4pt
\hbox{%
  \vbox to 30 pt{\scriptsize\vfill
    \hbox{\textsf{}}
    \hbox{\textsf{}}}}\hfill
\raise 4pt
\hbox{%
  \vbox to 30 pt{\scriptsize\vfill
    \hbox to 94 pt{\textsf{Submitted\hfill date}}
    \hbox to 94 pt{\textsf{Published\hfill date}}
    }}}
\def\@evenhead{}\def\@evenfoot{}}%
\thispagestyle{firstpage}}

\def\endfront@text{%
    \insert\copyins{\hsize\textwidth
      \fontsize{6}{7\p@}\normalfont\upshape
      \noindent
\hbox{
  \vbox{\fontsize{6}{8 pt}\baselineskip=6 pt\vss
    \hbox{\hbox to 20 pt{\hfill}
          \textsf{LOGICAL METHODS}\hfil}
    \hbox{\hbox to 20 pt{\phantom{x}}
          \textsf{IN COMPUTER SCIENCE}}}}
\hfill\textsf{DOI:10.2168/LMCS-???}
%\hfill\textsf{\copyright\shortauthors}
\hfill
\hbox{
  \vbox{\fontsize{6}{8 pt}\baselineskip=6 pt\vss
    \hbox{\textsf{\,\,\copyright\quad \shortauthors}\hfil}
    \hbox{\textsf{Creative Commons}\hfil}}}
\par\kern\z@}%
}
%\def\endfront@text{}

\def\enddoc@text{}

%% Defines the pagestyle for the rest of the pages
%% Usage: \ShortHeadings{Minimizing Conflicts}{Minton et al}
%%	  \ShortHeadings{short title}{short authors}

\def\firstpageno#1{\setcounter{page}{#1}}
\def\ShortHeadings#1#2{\def\ps@lmcsps{\let\@mkboth\@gobbletwo%
\def\@oddhead{\hfill {\small\sc #1} \hfill}%
\def\@oddfoot{\hfill \small\rm \thepage \hfill}%
\def\@evenhead{\hfill {\small\sc #2} \hfill}%
\def\@evenfoot{\hfill \small\rm \thepage \hfill}}%
\pagestyle{lmcsps}}

%% MISCELLANY

\def\@startsection#1#2#3#4#5#6{\bigskip%
 \if@noskipsec \leavevmode \fi
 \par \@tempskipa #4\relax
 \@afterindenttrue
 \ifdim \@tempskipa <\z@ \@tempskipa -\@tempskipa \@afterindentfalse\fi
 \if@nobreak \everypar{}\else
     \addpenalty\@secpenalty\addvspace\@tempskipa\fi
 \@ifstar{\@dblarg{\@sect{#1}{\@m}{#3}{#4}{#5}{#6}}}%
         {\@dblarg{\@sect{#1}{#2}{#3}{#4}{#5}{#6}}}%
}

\def\figurecaption#1#2{\noindent\hangindent 40pt
                       \hbox to 36pt {\small\sl #1 \hfil}
                       \ignorespaces {\small #2}}
% Figurecenter prints the caption title centered.
\def\figurecenter#1#2{\centerline{{\sl #1} #2}}
\def\figurecenter#1#2{\centerline{{\small\sl #1} {\small #2}}}

%
%  Allow ``hanging indents'' in long captions
%
\long\def\@makecaption#1#2{
   \vskip 10pt 
   \setbox\@tempboxa\hbox{#1: #2}
   \ifdim \wd\@tempboxa >\hsize               % IF longer than one line:
       \begin{list}{#1:}{
       \settowidth{\labelwidth}{#1:}
       \setlength{\leftmargin}{\labelwidth}
       \addtolength{\leftmargin}{\labelsep}
        }\item #2 \end{list}\par   % Output in quote mode
     \else                                    %   ELSE  center.
       \hbox to\hsize{\hfil\box\@tempboxa\hfil}  
   \fi}


% Define strut macros for skipping spaces above and below text in a
% tabular environment.
\def\abovestrut#1{\rule[0in]{0in}{#1}\ignorespaces}
\def\belowstrut#1{\rule[-#1]{0in}{#1}\ignorespaces}

%%% Theorem environments

% the following environments switch to a slanted font:
\theoremstyle{plain}

\newtheorem{thm}{Theorem}[section]
\newtheorem{cor}[thm]{Corollary}
\newtheorem{lem}[thm]{Lemma}
\newtheorem{prop}[thm]{Proposition}
\newtheorem{asm}[thm]{Assumption}

% the following environments keep the roman font:
\theoremstyle{definition}

\newtheorem{rem}[thm]{Remark}
\newtheorem{rems}[thm]{Remarks}
\newtheorem{exa}[thm]{Example}
\newtheorem{exas}[thm]{Examples}
\newtheorem{defi}[thm]{Definition}
\newtheorem{conv}[thm]{Convention}
\newtheorem{conj}[thm]{Conjecture}
\newtheorem{prob}[thm]{Problem}
\newtheorem{oprob}[thm]{Open Problem}
\newtheorem{algo}[thm]{Algorithm}
\newtheorem{obs}[thm]{Observation}
\newtheorem{qu}[thm]{Question}

\numberwithin{equation}{section}

% end-of-proof sign, to appear at right margin
% Paul Taylor and Chris Thompson, Cambridge, 1986
%
\def\pushright#1{{%        set up
   \parfillskip=0pt            % so \par doesn't push #1 to left
   \widowpenalty=10000         % so we dont break the page before #1
   \displaywidowpenalty=10000  % ditto
   \finalhyphendemerits=0      % TeXbook exercise 14.32
  %
  %                 horizontal
   \leavevmode                 % \nobreak means lines not pages
   \unskip                     % remove previous space or glue
   \nobreak                    % don't break lines
   \hfil                       % ragged right if we spill over
   \penalty50                  % discouragement to do so
   \hskip.2em                  % ensure some space
   \null                       % anchor following \hfill
   \hfill                      % push #1 to right
   {#1}                        % the end-of-proof mark (or whatever)
  %
  %                   vertical
   \par}}                      % build paragraph

\def\qEd{{\lower1 pt\hbox{\vbox{\hrule\hbox{\vrule\kern4 pt
    \vbox{\kern4 pt\hbox{\hss}\kern4 pt}\kern4 pt\vrule}\hrule}}}}
\def\qed{\pushright{\qEd}
    \penalty-700 \par\addvspace{\medskipamount}}

\newenvironment{Proof}[1][\proofname]{\par
  \pushQED{\qed}%
  \normalfont \topsep6\p@\@plus6\p@\relax
  \trivlist
  \item[\hskip\labelsep
        \itshape
    #1]\ignorespaces
}{%
  \popQED\endtrivlist\@endpefalse
}
% Bibliographystyle

\bibliographystyle{alpha}

\endinput


%% due to the dependence on amsart.cls, \begin{document} has to occur
%% BEFORE the title and author information:
\begin{document}

\title[The safe lambda calculus]{The safe lambda calculus \\ {\small \it version of \today}}

\author[W.Blum and C.-H. L.Ong]{William Blum}   %required
\address{Oxford University Computing Laboratory --
School of Informatics, University of Edinburgh, UK}    %required
% Wolfson Building, Parks Road, Oxford OX1 3QD, ENGLAND
\email{william.blum@comlab.ox.ac.uk}  %optional
%\thanks{thanks 1, optional.}   %optional

\author[]{C.-H.~Luke~Ong}   %optional
\address{Oxford University Computing Laboratory, Oxford, UK} %optional ; addresses should be duplicated when authors share an  affiliation
% Wolfson Building, Parks Road, Oxford OX1 3QD, ENGLAND
\email{luke.ong@comlab.ox.ac.uk}  %optional ; ditto for email addresses
%\thanks{thanks 2, optional.}    %optional


%% etc.

%% required for running head on odd and even pages, use suitable
%% abbreviations in case of long titles and many authors:

%% mandatory lists of keywords and classifications:
\keywords{lambda calculus, higher-order recursion scheme, safety
restriction, game semantics} \subjclass{F.3.2, F.4.1}

% OPTIONAL comment concerning the title, \eg, if a variant
%or an extended abstract of the paper has appeared elsewhere
\titlecomment{Some of the results presented here were first published in TLCA proceedings \cite{blumong:safelambdacalculus}}
%%%%%%%%%%%%%%%%%%%%%%%%%%%%%%%%%%%%%%%%%%%%%%%%%%%%%%%%%%%%%%%%%%%%%%%%%%%

%% the abstract has to PRECEED the command \maketitle:
%% be sure not to issue the \maketitle command twice!

\begin{abstract}
  Safety is a syntactic condition of higher-order grammars that
  constrains occurrences of variables in the production rules
  according to their type-theoretic order. In this paper, we introduce
  the \emph{safe lambda calculus}, which is obtained by transposing
  (and generalizing) the safety condition to the setting of the
  simply-typed lambda calculus. In contrast to the original definition
  of safety, our calculus does not constrain types (to be
  homogeneous). We show that in the safe lambda calculus, there is no
  need to rename bound variables when performing substitution, as
  variable capture is guaranteed not to happen.  We also propose an
  adequate notion of $\beta$-reduction that preserves safety.  In the
  same vein as Schwichtenberg's 1976 characterization of the
  simply-typed lambda calculus, we show that the numeric functions
  representable in the safe lambda calculus are exactly the
  multivariate polynomials; thus conditional is not definable. We
  also give a characterization of representable word functions.
  We then study the complexity of deciding beta-eta equality of two safe simply-typed terms and show that this
  problem is PSPACE-hard.
  Finally we give a game-semantic analysis of safety: We show that
  safe terms are denoted by \emph{P-incrementally justified
    strategies}. Consequently pointers in the game semantics of safe
  $\lambda$-terms are only necessary from order 4 onwards.
\end{abstract}

    \maketitle              % typeset the title of the contribution

    \section*{Introduction}

\subsection*{Background}

The \emph{safety condition} was introduced by Knapik, Niwi{\'n}ski and
Urzyczyn at FoSSaCS 2002 \cite{KNU02} in a seminal study of the
algorithmics of infinite trees generated by higher-order grammars. The
idea, however, goes back some twenty years to Damm \cite{Dam82} who
introduced an essentially equivalent\footnote{See de Miranda's
 thesis \cite{demirandathesis} for a proof.} syntactic
restriction (for generators of word languages) in the form of
\emph{derived types}.
% Level-$n$ tree grammars as defined by Damm correspond exactly to a
% subset of safe level-$n$ grammars -- namely the safe complete grammars
% -- and every safe grammar corresponds to a safe complete one.
A higher-order grammar (that is assumed to be \emph{homogeneously
  typed}) is said to be \emph{safe} if it obeys certain syntactic
conditions that constrain the occurrences of variables in the
production (or rewrite) rules according to their type-theoretic
order. Though the formal definition of safety is somewhat intricate,
the condition itself is manifestly important. As we survey in the
following, higher-order \emph{safe} grammars capture fundamental
structures in computation, offer clear algorithmic advantages, and
lend themselves to a number of compelling characterizations:

\begin{itemize}
\item \emph{Word languages}. Damm and Goerdt \cite{DG86} have shown
  that the word languages generated by order-$n$ \emph{safe} grammars
  form an infinite hierarchy as $n$ varies over the natural numbers.
  The hierarchy gives an attractive classification of the
  semi-decidable languages: Levels 0, 1 and 2 of the hierarchy are
  respectively the regular, context-free, and indexed languages (in
  the sense of Aho \cite{Aho68}), although little is known about
  higher orders.

  Remarkably, for generating word languages, order-$n$ \emph{safe}
  grammars are equivalent to order-$n$ pushdown automata \cite{DG86},
  which are in turn equivalent to order-$n$ indexed grammars
  \cite{Mas74,Mas76}.

\item \emph{Trees}. Knapik \emph{et al.} have shown that the Monadic
  Second Order (MSO) theories of trees generated by \emph{safe}
  (deterministic) grammars of every finite order are
  decidable\footnote{It has recently been shown
    \cite{OngLics2006} that trees generated by \emph{unsafe}
    deterministic grammars (of every finite order) also have decidable
    MSO theories. More precisely, the MSO theory of trees generated by order-$n$
recursion schemes is $n$-EXPTIME complete.}.

  They have also generalized the equi-expressivity result due to Damm
  and Goerdt \cite{DG86} to an equivalence result with respect to
  generating trees: A ranked tree is generated by an order-$n$ \emph{safe}
  grammar if and only if it is generated by an order-$n$ pushdown
  automaton.

\item \emph{Graphs}. Caucal \cite{Cau02} has shown that the MSO
  theories of graphs generated\footnote{These are precisely the
    configuration graphs of higher-order pushdown systems.} by
  \emph{safe} grammars of every finite order are decidable. In a recent preprint, however,
  Hague \emph{et al.} have
  shown that the MSO theories of graphs generated by order-$n$
  \emph{unsafe} grammars are undecidable, but deciding their modal
  mu-calculus theories is $n$-EXPTIME complete \cite{hmos-lics08}.
\end{itemize}

\subsection*{Overview}

In this paper, we aim to understand the safety condition in the
setting of the lambda calculus. Our first task is to transpose it to
the lambda calculus and pin it down as an appropriate sub-system of
the simply-typed theory. A first version of the \emph{safe lambda
  calculus} has appeared in an unpublished technical report
\cite{safety-mirlong2004}. Here we propose a more general and cleaner
version where terms are no longer required to be homogeneously typed
(see Section~\ref{sec:safe} for a definition). The formation rules of
the calculus are designed to maintain a simple invariant: Variables
that occur free in a safe $\lambda$-term have orders no smaller than
that of the term itself.  We can now explain the sense in which the
safe lambda calculus is safe by establishing its salient property: No
variable capture can ever occur when substituting a safe term into
another. In other words, in the safe lambda calculus, it is
\emph{safe} to use capture-\emph{permitting} substitution when
performing $\beta$-reduction.


There is no need for new names when computing $\beta$-reductions of
safe $\lambda$-terms, because one can safely ``reuse'' variable names
in the input term. Safe lambda calculus is thus cheaper to compute in
this na\"ive sense. Intuitively one would expect the safety constraint
to lower the expressivity of the simply-typed lambda calculus. Our
next contribution is to give a precise measure of the expressivity
deficit of the safe lambda calculus. An old result of Schwichtenberg
\cite{citeulike:622637} says that the numeric functions representable
in the simply-typed lambda calculus are exactly the multivariate
polynomials \emph{extended with the conditional function}.  In the
same vein, we show that the numeric functions representable in the
safe lambda calculus are exactly the multivariate polynomials.

Our last contribution is to give a game-semantic account of the safe
lambda calculus.
% Not much is known about the safe $\lambda$-calculus, and many problems
% remain to be studied concerning its computational power, the
% complexity classes that it characterizes, its interpretation under the
% Curry-Howard isomorphism and its game-semantic characterization. This
% paper is a contribution to the last problem.
%
% The difficulty in giving a game-semantic account of safety lies in the
% fact that it is a syntactic restriction whereas game semantics is
% syntax-independent. The solution consists in finding a particular
% syntactic representation of terms on which the plays of the game
% denotation can be represented.  To achieve this, we use ideas recently
% introduced by the second author \cite{OngLics2006}: A term is
% canonically represented by a certain abstract syntax tree of its
% $\eta$-long normal form referred as the \emph{computation tree}. This
% abstract syntax tree is specially designed to establish a
% correspondence with the game arena of the term. A computation is
% described by a justified sequence of nodes of the computation tree
% respecting some formation rules and called a
% \emph{traversal}. Traversals permit us to model $\beta$-reductions
% without altering the structure of the computation tree via
% substitution. A notable property is that \emph{P-views} (in the
% game-semantic sense) of traversals corresponds to paths in the
% computation tree.  We show that traversals are just representations of
% the uncovering of plays of the game-semantic denotation. We then
% define a \emph{reduction} operation which eliminates traversal nodes
% that are ``internal'' to the computation, this implements the
% counterpart of the hiding operation of game semantics. Thus, we obtain
% an isomorphism between the strategy denotation of a term and the set
% of reductions of traversals of its computation tree.
Using a correspondence result relating the game semantics of a
$\lambda$-term $M$ to a set of \emph{traversals} \cite{OngLics2006}
over a certain abstract syntax tree of the $\eta$-long form of $M$
(called \emph{computation tree}), we show that safe terms are denoted
by \emph{P-incrementally justified strategies}. In such a strategy,
pointers emanating from the P-moves of a play are uniquely
reconstructible from the underlying sequence of moves and the pointers
associated to the O-moves therein: Specifically, a P-question always
points to the last pending O-question (in the P-view) of a greater
order. Consequently pointers in the game semantics of safe
$\lambda$-terms are only necessary from order 4 onwards. Finally we
prove that a $\beta$-normal $\lambda$-term is \emph{safe}
if and only if its strategy denotation is (innocent and)
\emph{P-incrementally justified}.



% \subsection*{Related work}

% \noindent\emph{The safety condition for higher-order grammars}

% \smallskip

% \noindent We have mentioned the result of Knapik \emph{et al.}~\cite{KNU02} that
% infinite trees generated by \emph{safe} higher-order grammars have
% decidable MSO theories.  A natural question is whether the
% \emph{safety condition} is really necessary.  This has then been
% partially answered by Aehlig \emph{et al.}
% \cite{DBLP:conf/tlca/AehligMO05} where it was shown that safety is not
% a requirement at level $2$ to guarantee MSO decidability. Also, for
% the restricted case of word languages, the same authors have shown
% \cite{DBLP:conf/fossacs/AehligMO05} that level $2$ safe higher-order
% grammars are as powerful as (non-deterministic) unsafe ones.  De
% Miranda's thesis \cite{demirandathesis} proposes a unified framework
% for the study of higher-order grammars and gives a detailed analysis
% of the safety constraint at level 2.

% More recently, the second author obtained a more general result and showed
% that the MSO theory of infinite trees generated by higher-order
% grammars of any level, \emph{whether safe or not}, is decidable
% \cite{OngLics2006}.  Using an argument based on innocent
% game-semantics, he establishes a correspondence between the tree
% generated by a higher-order grammar called \emph{value tree} and a
% certain regular tree called \emph{computation tree}. Paths in the
% value tree correspond to traversals in the computation tree.
% Decidability is then obtained by reducing the problem to the acceptance
% of the (annotated) computation tree by a certain alternating parity
% tree automaton.  The approach that we follow in
% Sec. \ref{sec:correspondence} uses many ingredients introduced in this
% paper.


% The equivalence of \emph{safe} higher-order grammars and higher-order
% deterministic push-down automata for the purpose of generating
% infinite trees \cite{KNU02} has its counterpart in the general (not
% necessarily safe) case: Hague et al. \cite{hmos-lics08}
% establishes the equivalence of order-$n$ higher-order grammars and
% order-$n$ \emph{collapsible pushdown automata}. Those automata form a
% new kind of pushdown systems in which every stack symbol has a link to
% a stack situated somewhere below it and with an additional stack
% operation whose effect is to ``collapse'' a stack $s$ to the state
% indicated by the link from the top stack symbol.

% \medskip

% \noindent\emph{Computation trees and traversals}

% \smallskip

% \noindent Asperti et.~al \cite{DBLP:conf/lics/AspertiDLR94} introduced a notion of graph
% based on Lamping's graphs \cite{lamping} to represent
% $\lambda$-terms. The authors unify different notions of paths
% (regular, legal, consistent and persistent paths) that have appeared
% in the literature as ways to implement graph-based reduction of
% $\lambda$-expressions. We can regard a traversal as an alternative
% notion of path adapted to the graph representation of
% $\lambda$-expressions given by computation trees.

% The traversals of a computation tree provide a way to perform
% \emph{local computation} of $\beta$-reductions as opposed to a global
% approach where the $\beta$-reduction is implemented by performing
% substitutions. A notion of local computation of $\beta$-reduction has
% been investigated by Danos and Regnier
% \cite{DanosRegnier-Localandasynchronou} through the use of special
% graphs called ``virtual nets'' that embed the lambda calculus.


\section{The safe lambda calculus}
\label{sec:safe}
\subsection*{Higher-order safe grammars}
We first present the safety restriction as it was originally defined
\cite{KNU02}. We consider simple types generated by the grammar $A \,
::= \, o \; | \; A \typear A$. By convention, $\rightarrow$ associates
to the right. Thus every type can be written as $A_1 \typear \cdots
\typear A_n \typear o$, which we shall abbreviate to $(A_1, \cdots,
A_n, o)$ (in case $n = 0$, we identify $(o)$ with $o$).
We will also use the notation $A^n \typear B$ for every types $A, B$ and positive natural number $n>0$ defined by induction as: $A^1 \typear B = A \typear B$  and $A^{n+1} \typear B = A\typear (A^n \typear B)$. The \emph{order} of a type is given by $\ord{o} = 0$ and $\ord(A \typear
B) = \max(\ord{A}+1, \ord{B})$. We assume an infinite set of typed
variables. The order of a typed term or symbol is defined to be the
order of its type. The set of \emph{applicative terms} over a set of typed symbols is defined as its closure under the application operation (\ie, if $M : A\rightarrow B$ and $N :A$ are in the closure then so does $M N :B$).

A (higher-order) \defname{grammar} is a tuple $\langle
\Sigma, \mathcal{N}, \mathcal{R}, S \rangle$, where $\Sigma$ is a
ranked alphabet (in the sense that each symbol $f \in \Sigma$ is assumed to have type $o^r\typear o$ where $r$ is the \emph{arity} of $f$) of \emph{terminals}; $\mathcal{N}$ is a finite set of typed
\emph{non-terminals}; $S$ is a distinguished ground-type symbol of
$\mathcal{N}$, called the start symbol; $\mathcal{R}$ is a finite set
of production (or rewrite) rules, one for each non-terminal $F : (A_1,
\ldots, A_n, o) \in \mathcal{N}$, of the form $ F z_1 \ldots z_m
\rightarrow e$ where each $z_i$ (called \emph{parameter}) is a
variable of type $A_i$ and $e$ is an applicative term of type $o$
generated from the typed symbols in $\Sigma \union \mathcal{N} \union \{z_1,
\ldots, z_m \}$. We say that the grammar is \emph{order-$n$} just in
case the order of the highest-order non-terminal is $n$.

We call \defname{higher-order recursion scheme} a higher-order
grammar that is deterministic (\ie, for each non-terminal $F \in
\mathcal{N}$ there is exactly one production rule with $F$ on the
left hand side). Higher-order recursion schemes are used as
generators of infinite trees. The \defname{tree generated by a
recursion scheme} $G$ is a possibly infinite applicative term, but
viewed as a $\Sigma$-labelled tree; it is \emph{constructed from the
terminals in $\Sigma$}, and is obtained by unfolding the rewrite
rules of $G$ \emph{ad infinitum}, replacing formal by actual
parameters each time, starting from the start symbol $S$. See
e.g.~\cite{KNU02} for a formal definition.

\pssetcomptree
\parpic[r]{
\raisebox{-15pt}
{$\tree[levelsep=3ex,nodesep=1pt,treesep=1cm,linewidth=0.5pt]{g}
{  \TR{a}
    \tree{g}{\TR{a} \tree{h}{\tree{h}{\vdots}}}
}$}
}
\begin{example}\rm\label{eg:running}
  Let $G$ be the following order-2 recursion scheme:
\[\begin{array}{rll}
  S & \rightarrow & H \, a\\
  H \, z^o & \rightarrow & F \, (g \,
  z)\\
  F \, \phi^{(o, o)} & \rightarrow & \phi \, (\phi \, (F \, h))\\
\end{array}\]
where the arities of the terminals $g, h, a$ are $2, 1, 0$ respectively.
The tree generated by $G$ is defined by the infinite term $g \, a \, (g \, a \, (h \, (h \, (h \,
\cdots))))$.%  The only infinite \emph{path} in the
% tree is the node-sequence $\epsilon \cdot 2 \cdot 22 \cdot 221 \cdot
% 2211 \cdots$.

%(with the corresponding \textbfit{trace} $g \, g \, h \, h \, h \,
%\cdots \; \in \; \Sigma^\omega$).
\end{example}

A type $(A_1, \cdots, A_n, o)$ is said to be \defname{homogeneous} if
$\ord{A_1} \geq \ord{A_2}\geq \cdots \geq \ord{A_n}$, and each $A_1$,
\ldots, $A_n$ is homogeneous \cite{KNU02}.  We reproduce the following
Knapik et al.'s definition \cite{KNU02}.

\begin{definition}[Safe grammar]\rm
\label{def:safegrammar}
  (All types are assumed to be homogeneous.) A term of order $k > 0$
  is \emph{unsafe} if it contains an occurrence of a parameter of
  order strictly less than $k$, otherwise the term is \emph{safe}. An
  occurrence of an unsafe term $t$ as a subexpression of a term $t'$
  is \emph{safe} if it is in the context $\cdots (ts) \cdots$,
  otherwise the occurrence is \emph{unsafe}. A grammar is
  \defname{safe} if no unsafe term has an unsafe occurrence at a
  right-hand side of any production.
%   A rewrite rule $F z_1 \ldots z_m \rightarrow e$ is said to be
%   \defname{unsafe} if the righthand term $e$ has a subterm $t$ such
%   that
% \begin{enumerate}[(i)]
% \item $t$ occurs in an {\em operand} ({\it i.e.}~second) position of some
%   occurrence of the implicit application operator {\it i.e.}~$e$ has the
%   form $\cdots (s \, t) \cdots $ for some $s$
% \item $t$ contains an occurrence of a parameter $z_i$ (say) whose
%   order is less than that of $t$.
% \end{enumerate}
% A homogeneous grammar is said to be \defname{safe} if none of its
% rewrite rules is unsafe.
\end{definition}

\begin{example}\begin{inparaenum}[(i)] \item Take $H : ((o, o), o)$ and $f : (o, o, o)$; the
    following rewrite rules are unsafe (in each case we underline the
    unsafe subterm that occurs unsafely):
\[\begin{array}{rll}
G^{(o, o)} \, x & \quad \rightarrow \quad & H \, \underline{(f \, {x})} \\
F^{((o, o), o, o, o)} \, z \, x \, y & \quad \rightarrow \quad & f \, (F \, \underline{(F \, z
\, {y})} \, y \, (z \, x) ) \, x
\end{array}\]
\item The order-2 grammar defined in Example~\ref{eg:running} is
  unsafe.
\end{inparaenum}
% The
% reader is referred to the literature
% \cite{KNU02,demirandathesis,safety-mirlong2004}
% for details about the safety restriction for higher-order grammars.
\end{example}

\subsection*{Safety adapted to the lambda calculus}
We assume a set $\Xi$ of higher-order constants. We use sequents of
the form $\Gamma \vdash_\$^\Xi M : A$ to represent term-in-context
where $\Gamma$ is the context and $A$ is the type of $M$. For
convenience, we shall omit the superscript from $\sentail^\Xi$
whenever the set of constants $\Xi$ is clear from the context. The
subscript in $\vdash_\$^\Xi$ specifies which type system we are
using to form the judgment: We use the subscript `st' to refer to
the traditional system of rules of the Church-style simply-typed
lambda calculus augmented with constants from $\Xi$. In the
following we will use new subscripts for each type system that we
introduce. For simplicity we write $(A_1, \cdots, A_n, B)$ to mean
$A_1 \typear \cdots \typear A_n \typear B$, where $B$ is not
necessarily ground.

\begin{definition}\rm
\label{def:safelambda}
\begin{inparaenum}[(i)]
\item The \defname{safe lambda calculus} is a sub-system of the
  simply-typed lambda calculus. It is defined as the set of judgments of the form $\Gamma \sentail M : A$ that are derivable from the following Church-style system of rules:
$$ \rulename{var} \ \rulef{}{x : A\sentail x : A} \qquad
\rulename{const} \ \rulef{}{\sentail f : A}~f \in \Xi \qquad
\rulename{wk} \ \rulef{\Gamma \sentail M : A}{\Delta \sentail M : A} \quad
\Gamma \subset \Delta$$

$$ \rulename{app_{as}} \ \rulef{\Gamma \asappentail M : A\typear B
\quad \Gamma \sentail N : A} {\Gamma \asappentail M\, N : B}
\qquad
\rulename{\delta} \ \rulef{\Gamma \sentail M : A}{\Gamma \asappentail M : A}
$$

$$ \rulename{app} \ \rulef{\Gamma \asappentail M : A\typear B
\quad \Gamma \sentail N : A} {\Gamma \sentail M\, N : B} \quad \ord{B} \leq
\ord{\Gamma}$$

$$ \rulename{abs} \ \rulef{\Gamma, x_1 : A_1, \ldots, x_n : A_n
  \asappentail M : B} {\Gamma \sentail \lambda x_1^{A_1} \ldots x_n^{A_n} . M :
  (A_1, \ldots ,A_n,B)} \quad \ord(A_1, \ldots ,A_n,B) \leq
\ord{\Gamma}$$
\smallskip

\noindent where $\ord{\Gamma}$ denotes the set $\{ \ord{y} : y \in
\Gamma \}$ and ``$c \leq S$'' means that $c$ is a lower-bound of the
set $S$. The subscripts in $\sentail$ and $\asappentail$ stand
``safe'' and ``almost safe application''.

%\noindent
\item The sub-system that is defined by the same rules in
(i), such that all types that occur in them are homogeneous, is called
the \defname{homogeneous safe lambda calculus}.

\item We say that a \emph{term} $M$ is \defname{safe} if the judgement $\Gamma \sentail M : T$ is derivable in the safe lambda calculus for some context $\Gamma$ and type $T$.
\end{inparaenum}
\end{definition}


The safe lambda calculus deviates from the standard definition of
the simply-typed lambda calculus in a number of ways. %First the
%rules $\rulename{app}$ and $\rulename{abs}$ respectively can perform
%multiple applications and abstract several variables at once.
First the rule $\rulename{abs}$ can abstract several variables at once. (Of course this feature alone does not alter expressivity.) Crucially,
the side conditions in the application rule and abstraction rule
require the variables in the typing context to have orders no
smaller than that of the term being formed.  We do not impose any
constraint on types. In particular, type-homogeneity, which was an
assumption of the original definition of safe grammars \cite{KNU02},
is not required here. Another difference is that we allow
$\Xi$-constants to have
arbitrary higher-order types.
\bigskip


\begin{example}[Kierstead terms]
\label{ex:kierstead}
Consider the terms $M_1 = \lambda f^{((o,o),o)} . f (\lambda x^o . f (\lambda y^o . y
))$ and $M_2 = \lambda f^{((o,o),o)} . f (\lambda x^o . f (\lambda y^o .x ))$. The term $M_2$ is not safe because in the subterm $f (\lambda y^o . x)$, the free variable $x$ has order $0$ which
is smaller than $\ord{\lambda y^o . x} = 1$.  On the other hand, $M_1$
is safe.
%On the other hand, $M_1$ is safe as the following proof tree shows:
%$$
% \rulef{
%     \rulef{
%        \rulef{}{f \sentail f} {\sf(var)}
%        \
%        \rulef{
%             \rulef{
%                \rulef{
%                    \rulef{}{f \sentail f} {\sf(var)}
%                }
%                {f , x \sentail f } {\sf(wk)}
%                \
%                \rulef{
%                    \rulef{
%                        \rulef{}{y \sentail y} {\sf(var)}
%                    }
%                    {y \sentail \lambda y . y } \rulenamet{abs}
%                }
%                {f , x \sentail \lambda y .y } {\sf(wk)}
%             }
%             {f , x \sentail f (\lambda y .y )} {\sf(app)}
%        }
%        { f  \sentail \lambda x . f (\lambda y .y )} \rulenamet{abs}
%     }
%     {
%        f  \sentail f (\lambda x . f (\lambda y .y ))} {\sf(app)}
%     }
% { \sentail M_1 = \lambda f . f (\lambda x . f (\lambda y .y )) } \rulenamet{abs}
%$$
\end{example}

It is easy to see that valid typing judgements of the safe lambda
calculus satisfy the following simple invariant:
\begin{lemma}
\label{lem:ordfreevar}
If $\Gamma \sentail M : A$ then every variable in $\Gamma$ occurring
free in $M$ has order at least $\ord(M)$.
\end{lemma}


\begin{definition}
A term is an \defname{almost safe applications}
if it is safe or if it is of the form $N_1 \ldots N_m$ for some $m\geq 1$ where $N_1$ is not an application and for every $1 \leq i\leq m$, $N_i$ is safe.

A term is \defname{almost safe} if either it is an almost safe application, or if it is of the form
$\lambda x_1^{A_1} \ldots x_n^{A_n}. M$ for $n\geq 1$ and some almost safe application $M$.
\end{definition}
An \emph{almost safe application} is not necessarily safe but it can be used to form a safe term by applying sufficiently many safe terms to it. An \emph{almost safe term} can be turned into a safe term by either applying sufficiently many safe terms (if it is an application), or
by abstracting sufficiently many variables (if it is an abstraction).

We have the following immediate lemma:
\begin{lemma}
\label{lem:almostsafeapp_is_appplicative_safe}
A term $M$ is
\begin{enumerate}[(i)]
\item an \emph{almost safe application} iff there is a derivation of $\Gamma \asappentail M : T$ for some $\Gamma, T$;

\item \emph{almost safe} iff $\Gamma \asappentail M : T$ or if $M\equiv \lambda x_1^{A_1} \ldots x_n^{A_n}. N$ and $\Gamma \asappentail N : T$ for some $\Gamma, T$.
\end{enumerate}
\end{lemma}
In particular, terms constructed with the rule \rulenamet{app_{as}} are almost safe applications.
\bigskip

When restricted to the homogeneously-typed sub-system, the safe
lambda calculus captures the original notion of safety due to Knapik
\emph{et al.}~in the context of higher-order grammars:

\begin{proposition}
\label{prop:safegram_safelmd}
 Let $G = \langle \Sigma, \mathcal{N}, \mathcal{R},
  S \rangle$ be a grammar and let $e$ be an applicative term generated
  from the symbols in $\mathcal{N} \cup \Sigma \cup \makeset{z_1^{A_1},
    \cdots, z_m^{A_m}}$.  A rule $F z_1 \ldots z_m \rightarrow e$ in
  $\mathcal{R}$ is safe (in the original sense of Knapik \emph{et al.}) if and only if $ z_1 : A_1, \cdots, z_m : A_m
  \sentail^{\Sigma \cup \mathcal{N}} e : o$ is a valid typing judgement
  of the \emph{homogeneous} safe lambda calculus.
\end{proposition}
\begin{proof}
We show by induction that \begin{asparaenum}
\item  $z_1,\ldots, z_m \asappentail t:A$ is a valid judgment of the homogeneous safe lambda calculus containing no abstraction if and only if in the Knapik sense, all the occurrences of unsafe subterms of $t$ are safe occurrences.
\item $z_1,\ldots, z_m \sentail t:A$ is a valid judgment of the homogeneous safe lambda calculus containing no abstraction if and only if in the Knapik sense, all the occurrences of unsafe subterms of $t$ are safe occurrences, and all parameters occurring in $t$ have order greater than $\ord{t}$. \end{asparaenum}
The constant and variable rule are trivial. Application case: By definition, a term $t_0 \ldots t_n$ is Knapik-safe iff for all $0\leq i \leq n$, all the occurrences of unsafe subterms of $t_i$ are safe occurrences (in the Knapik sense), and for all $1\leq j \leq n$, the operands occurring in $t_j$ have order greater than $\ord{t_j}$. The \rulenamet{app_{as}} rule and the induction hypothesis permit us to conclude.

Now since $e$ is an applicative term of \emph{ground type}, the previous result gives:
$z_1,\ldots, z_m \sentail e:o$ is a valid judgment of the homogeneous safe lambda calculus iff
all the occurrences of unsafe subterms of $e$ are safe occurrences, which is in turn equivalent to ``$F z_1 \ldots z_m \rightarrow e$ is safe'' by definition of Knapik-safety for grammar rules.
\end{proof}

\emph{In what sense is the safe lambda calculus safe?} It is an
elementary fact that when performing $\beta$-reduction in the lambda
calculus, one must use capture-\emph{avoiding} substitution, which
is standardly implemented by renaming bound variables afresh upon
each substitution. In the safe lambda calculus, however, variable
capture can never happen (as the following lemma shows).
Substitution can therefore be implemented simply by
capture-\emph{permitting} replacement, without any need for variable
renaming. In the following, we write $M\captsubst{N}{x}$ to denote
the capture-\emph{permitting} substitution\footnote{This
substitution is done by textually replacing all free occurrences of
$x$ in $M$ by $N$ without performing variable renaming.  In
particular for the abstraction
  case we have
$(\lambda y_1\ldots y_n . M)\captsubst{N}{x} = \lambda y_1\ldots y_n . M\captsubst{N}{x}$ when $x\not\in
  \{ y_1\ldots y_n \}$.}
%\footnote{This substitution is implemented by textually
%  replacing all free occurrences of $x$ in $M$ by $N$ without
%  performing variable renaming.  In particular for the abstraction
%  case $(\lambda \overline{y} . P)\captsubst{N}{x}$ is defined as
%  $\lambda \overline{y} . P\captsubst{N}{x}$ if $x\not\in
%  \overline{y}$ and $\lambda \overline{y} . P$ elsewhere.}
of $N$ for $x$ in $M$.

\begin{lemma}[No variable capture]\label{lem:nvc}
\label{lem:nocapture} There is no variable capture when performing
capture-permitting substitution of $N$ for $x$ in $M$ provided that
$\Gamma, x:B \sentail M : A$ and $\Gamma \sentail  N : B$ are valid
judgments of the safe lambda calculus.
\end{lemma}

\begin{proof}
  We proceed by structural induction on $M$. The variable, constant and
  application cases are trivial. For the abstraction case, suppose $M \equiv \lambda \overline{y}. R$ where $\overline{y} = y_1 \ldots y_p$. If $x \in \overline{y}$ then $M \captsubst{N}{x} = M$ and there is no variable capture.

 Otherwise, $x \not\in \overline{y}$. By Lemma \ref{lem:almostsafeapp_is_appplicative_safe} $R$ is of the
  form $M_1 \ldots M_m$ for some $m\geq 1$ where $M_1$ is not an application and for every $1 \leq i\leq m$, $M_i$ is safe.
 Thus we have $M \captsubst{N}{x} \equiv \lambda \overline{y} . M_1 \captsubst{N}{x} \ldots M_m \captsubst{N}{x}$.  Let $i\in\{1..m\}$. By the induction hypothesis there is no variable capture in $M_i \captsubst{N}{x}$.  Thus variable capture can only happen if the following two conditions are met: $x$ occurs freely in $M_i$, and some variable $y_i$ for $1 \leq i \leq p$ occurs freely in $N$. By Lemma \ref{lem:ordfreevar}, the latter condition implies $\ord{y_i} \geq \ord{N} = \ord{x}$ and  since $x \not \in \overline{y}$, the former condition implies that $x$ occurs freely in the safe term $\lambda \overline{y}. R$
  thus by Lemma \ref{lem:ordfreevar} we have $ \ord{x} \geq
  \ord{\lambda \overline{y} . R} \geq 1+ \ord{y_i} > \ord{y_i}$ which  gives a contradiction.
\end{proof}

\begin{remark}
  A version of the No-variable-capture Lemma also holds in safe
  grammars, as is implicit in (for example Lemma 3.2 of) the original
  paper \cite{KNU02}.
\end{remark}

\begin{example}
  In order to contract the $\beta$-redex in the term
\[f:(o,o,o),x:o
  \stentail (\lambda \varphi^{(o,o)} x^o . \varphi \, x) (\underline{f \,
    x}) : (o,o)\] one should rename the bound variable $x$ to a fresh name to
  prevent the capture of the free occurrence of $x$ in the underlined term during substitution. Consequently, by the previous lemma,
  the term is not safe (because $\ord{x} = 0 < 1
  = \ord{f x}$).
\end{example}

Note that $\lambda$-terms that `satisfy' the No-variable-capture
Lemma are not necessarily safe. For instance the $\beta$-redex in
$\lambda y^o z^o. (\lambda x^o .y) z$ can be contracted using
capture-permitting substitution, even though the term is not safe.
\bigskip

\emph{Related work:} In her thesis \cite{demirandathesis}, de Miranda proposed a different notion of safe lambda calculus. This notion corresponds to (a less general version of) our notion of \emph{homogeneous} safe lambda calculus. It can be showed that for pure applicative terms (\ie, with no lambda-abstraction) the two systems coincides. In particular a version of Proposition \ref{prop:safegram_safelmd} also holds in de Miranda's setting \cite{demirandathesis}. In the presence of lambda abstraction, however, our system is less restrictive. For instance the term $\lambda f^{(o,o,o)} x^o.  f x : (o,o)$ is typable in the homogeneous safe lambda calculus but not in the safe lambda calculus \emph{\`a la} de Miranda. One can show that de Miranda's system is in fact equivalent to the \emph{homogeneous long-safe lambda calculus} (\ie, the restriction of the system of Def.\ \ref{dfn:longsafe} to homogeneous types).

\subsection*{Safe beta reduction}

From now on we will use the standard notation $M\subst{N}{x}$ to
denote the substitution of $N$ for $x$ in $M$.  It is understood that,
provided that $M$ and $N$ are safe, this substitution is
capture-permitting.


\begin{lemma}[Substitution preserves safety]
\label{lem:subst_preserves_safety}
Let $\Gamma \sentail N : B$. Then
\begin{enumerate}[(i)]
  \item $\Gamma, x :B \sentail M : A$ implies $\Gamma \sentail M[N/x] : A$;
  \item $\Gamma, x :B \asappentail M : A$ implies $\Gamma \asappentail M[N/x] : A$.
\end{enumerate}
\end{lemma}
This is proved by an easy induction on the structure of the safe term $M$.
\smallskip

It is desirable to have an appropriate notion of reduction for our
calculus. However the standard $\beta$-reduction rule is not
adequate. Indeed, safety is not preserved by $\beta$-reduction as
the following example shows. Suppose that $w,x,y,z : o$ and $f :
(o,o,o) \in \Sigma$ then the safe term $(\lambda x y . f x y) z w$
$\beta$-reduces to $(\underline{\lambda y . f z y}) w$, which is
unsafe since the underlined order-1 subterm contains a free
occurrence of the ground-type $z$. However if we perform one more
reduction we obtain the safe term $f z w$. This suggests
simultaneous contraction of ``consecutive'' $\beta$-redexes. In
order to define this notion of reduction we first introduce the
corresponding notion of redex.

In the simply-typed lambda calculus a redex is a term of the form
$(\lambda x . M) N$. In the safe lambda calculus, a redex is a
succession of several standard redexes:

\begin{definition}\rm
A \defname{safe redex} is an almost safe application of the form
$$(\lambda x_1^{A_1} \ldots x_n^{A_n}. M) N_1 \ldots N_l$$ for $l,n\geq 1$ such that $M$ is an almost safe application.
(Consequently each $N_i$ is safe as well as $\lambda x_1^{A_1} \ldots x_n^{A_n} . M$, and $M$ is either safe or is an application of safe terms.)
\end{definition}
For instance, in the case $n<l$, a safe redex has a derivation tree of the following form:
\def\defaultHypSeparation{}
\begin{prooftree}
    \AxiomC{\ldots}
  \UnaryInfC{$\Gamma', \overline{x}:\overline{A} \sentail M : (A_{n+1}, \ldots, A_l, B)$}
  \RightLabel{\rulenamet{abs}}
  \UnaryInfC{$\Gamma' \sentail \lambda \overline{x}^{\overline A} . M : (A_1, \ldots, A_l, B)$}
  \RightLabel{\rulenamet{wk}}
  \UnaryInfC{$\Gamma \sentail \lambda \overline{x}^{\overline A} . M : (A_1, \ldots, A_l, B)$}
  \RightLabel{\rulenamet{\delta}}
  \UnaryInfC{$\Gamma \asappentail \lambda \overline{x}^{\overline A} . M : (A_1, \ldots, A_l, B)$}
  \AxiomC{\ldots}
  \UnaryInfC{$\Gamma \sentail N_1 :A_1$}
  \RightLabel{\rulenamet{app_{as}}}
   \BinaryInfC{$\Gamma \asappentail (\lambda \overline{x}^{\overline A} . M) N_1 : (A_2, \ldots A_l, B)$}
%  \AxiomC{\ldots}
%  \RightLabel{\rulenamet{app_{as}}}
%  \BinaryInfC{\vdots\raisebox{0.5cm}{}}
  \noLine\UnaryInfC{\vdots\raisebox{0.5cm}{}}
  \RightLabel{\rulenamet{app_{as}}}
  \UnaryInfC{$\Gamma \asappentail (\lambda \overline{x}^{\overline A} . M) N_1 \ldots N_{l-1} : (A_l, B)$}
  \AxiomC{\ldots}
  \UnaryInfC{$\Gamma \sentail N_l :A_l$}
  \RightLabel{\rulenamet{app}}
  \BinaryInfC{$\Gamma \sentail (\lambda \overline{x}^{\overline A} . M) N_1 \ldots N_l : B$}
\end{prooftree}
\smallskip

A \emph{safe redex} is by definition an almost term,
but it is not necessarily a \emph{safe term}. For instance the term $(\lambda x^o y^o. x) z$ is a safe redex but it is only an \emph{almost} safe term.
The reason why we call such redexes ``safe'' is because when they occur within a safe term, it is possible to contract them without braking the safety of the whole term. Before showing this result, we first need to define how to contract safe redexes:
\begin{definition}[Redex contraction]\rm
\label{dfn:saferedex_contraction} We use the abbreviations $\overline{x} =
x_1 \ldots x_n$, $\overline{N} = N_1 \ldots N_l$. The relation
$\beta_s$ (when viewed as a function) is defined on the set of \emph{safe redexes} as follows:
\begin{eqnarray*}
  \beta_s &=&
  \{  \ (\lambda \overline{x}^{\overline A} . M) N_1 \ldots N_l \mapsto \lambda x_{l+1}^{A_{l+1}} \ldots x_n^{A_{n}}. M\subst{\overline{N}}{x_1 \ldots x_l} \ | \ n> l  \} \\
  &\cup&
  \{ \ (\lambda \overline{x}^{\overline A}  . M) N_1 \ldots N_l \mapsto M\subst{N_1 \ldots N_n}{\overline{x}} N_{n+1} \ldots N_l
  \ | \ n\leq l \} \ .
\end{eqnarray*}
where $M\subst{R_1 \ldots R_k}{z_1 \ldots z_k}$ denotes the simultaneous substitution in $M$ of $R_1$,\ldots,$R_k$ for $z_1, \ldots, z_k$.
\end{definition}

\begin{lemma}[$\beta_s$-reduction preserves safety]
\label{lem:betas_preserves_safety}
Suppose that $M_1\, \beta_s\, M_2$. Then
\begin{enumerate}[(i)]
  \item $M_2$ is almost safe;
  \item If $M_1$ is safe then so is $M_2$.
\end{enumerate}
\end{lemma}
\begin{proof}
Let $M_1\, \beta_s\, M_2$ for some safe redex $M_1$ and term $M_2$ of type $A$. By definition, $M_1$ is of the form $(\lambda x_1^{B_1} \ldots x_n^{B_n} . M) N_1 \ldots N_l $ for some safe terms $N_1$, \ldots, $N_l$ and almost safe term $M$ of type $C$ such that $(\lambda x_1^{B_1} \ldots x_n^{B_n} . M)$ is safe.
\begin{compactitem}[-]
\item
Suppose $n>l$ then $A = (B_{l+1}, \ldots, B_n, C)$.
(i) By the Substitution Lemma
\ref{lem:subst_preserves_safety}, the term $M\subst{\overline{N}}{x_1
\ldots x_l} : C$ is an almost safe application (\ie, $\Gamma, x_{l+1}:B_{l+1}, \ldots x_n :B_{n}\asappentail M\subst{\overline{N}}{x_1
\ldots x_l} : C$). (If $M$ is safe then we apply the lemma once; otherwise it is of the form $R_1 \ldots R_q$ where $R_i$ is a safe term and we apply the lemma on each $R_i$.)
Thus by definition, $\lambda x_{l+1}^{B_{l+1}} \ldots x_n^{B_n} . M\subst{\overline{N}}{x_1
\ldots x_l} \equiv M_2$ is almost safe.

(ii) Suppose that $M_1$ is safe. W.l.o.g.~we can assume that the last rule used to form  $M_1$ is \rulenamet{app} (and not the weakening rule
\rulenamet{wk}), thus the variables of the typing context $\Gamma$ are precisely the free variables of $M_1$, and Lemma \ref{lem:ordfreevar} gives us $\ord{A} \leq \ord{\Gamma}$. This allows us to use the rule \rulenamet{abs} to form the safe term $\Gamma
\sentail \lambda x_{l+1}^{B_{l+1}} \ldots x_n^{B_n} . M\subst{\overline{N}}{x_1
\ldots x_l} \equiv M_2 : A$.

\item Suppose $n \leq l$. (i) Again by the Substitution Lemma
we have that $M\subst{N_1 \ldots N_n}{\overline{x}}$ is an almost safe application: $\Gamma \asappentail M\subst{N_1 \ldots N_n}{\overline{x}} : C$. If $n=l$ then the proof is finished; otherwise ($n<l$) we further apply the rule \rulenamet{app_{as}} $l-n$ times which gives us the almost safe application $\Gamma \asappentail M_2 :A$.

(ii) Suppose that $M_1$ is safe.  If $n=l$ then $M_2\equiv M\subst{N_1 \ldots N_n}{\overline{x}}$ is safe by the Substitution Lemma;
If $n<l$ then we obtain the judgement $\Gamma \sentail M_2 :A$ by
applying the rule \rulenamet{app_{as}} $l-n-1$ times on $\Gamma \sentail M\subst{N_1 \ldots N_n}{\overline{x}} : C$ followed by one application of \rulenamet{app}.
\qedhere
\end{compactitem}
\end{proof}

We can now define a notion of reduction for safe terms.
\begin{definition}\rm
The \defname{safe $\beta$-reduction}, written $\betasred$, is the
compatible closure of the relation $\beta_s$ with respect to the
formation rules of the safe lambda calculus (\ie, it is the smallest relation such that
if $M_1\, \beta_s\, M_2$ and $C[M]$ is a safe term for some context $C[-]$ formed with the rules of the simply typed lambda calculus then $C[M_1]\betasred C[M_2]$).
\end{definition}

\begin{lemma}[$\beta_s$-reduction preserves safety]
\label{lem:safered_preserves_safety}
If $\Gamma \sentail M_1 :A$ and $M_1\betasred M_2$ then $\Gamma \sentail M_2 :A$.
\end{lemma}
\begin{proof}
Follows from Lemma \ref{lem:betas_preserves_safety} by an easy induction.
\end{proof}


\begin{lemma} The safe reduction relation $\betasred$:
\begin{enumerate}[(i)]
\item is a subset of the transitive closure of $\betared$ ($\betasred \subset \betaredtr$);
\item is strongly normalizing;
\item has the unique normal form property;
\item has the Church-Rosser property.
\end{enumerate}
\end{lemma}
\begin{proof}
(i) Immediate from the definition: Safe $\beta$-reduction is just a multi-step $\beta$-reduction.
(ii) This is because $\betasred \subset \betaredtr$ and, $\betared$ is
strongly normalizing in the simply typed $\lambda$-calculus. (iii) It is easy to see that if a safe term has a beta-redex if and only if it has a safe beta-redex (because a beta-redex can always be ``widen'' into consecutive beta-redex of the shape of those in Def.~\ref{dfn:saferedex_contraction}. Therefore the set
of $\beta_s$-normal forms is equal to the set of $\beta_s$-normal
forms. The unicity of $\beta$-normal form then implies the unicity
of $\beta_s$-normal form. (iv) is a consequence of (i) and (ii).
\end{proof}

\subsection*{Eta-long expansion}

The $\eta$-long normal form (or simply $\eta$-long form) of a term
% (also called \emph{long reduced form}, \emph{$\eta$-normal form} and
% \emph{extensional form} in the literature
% \cite{DBLP:journals/tcs/JensenP76,DBLP:journals/tcs/Huet75,huet76})
is obtained by hereditarily $\eta$-expanding the body of every
lambda abstraction as well as every subterm occurring in an
\emph{operand position} (\ie, occurring as the second argument of
some occurrence of the binary application operator). Formally the
\defname{$\eta$-long form}, written $\elnf{M}$, of a term $M:
(A_1,\ldots,A_n,o)$ with $n \geq 0$ is defined by cases according to
the syntactic shape of $M$:
\begin{eqnarray*}
  \elnf{\lambda x^\tau . N } &\equiv& \lambda x^\tau . \elnf{N} \\
  \elnf{x N_1 \ldots N_m } &\equiv& \lambda \overline{\varphi}^{\overline{A}} . x \elnf{N_1}\ldots \elnf{N_m} \elnf{\varphi_1} \ldots \elnf{\varphi_n} \\
  \elnf{(\lambda x^\tau . N) N_1 \ldots N_p } &\equiv& \lambda \overline{\varphi}^{\overline{A}} . (\lambda x^\tau . \elnf{N}) \elnf{N_1} \ldots \elnf{N_p} \elnf{\varphi_1} \ldots \elnf{\varphi_n}
\end{eqnarray*}
where $m \geq 0$, $p\geq 1$, $x$ is either a variable or constant, $\overline{\varphi} = \varphi_1 \ldots \varphi_n$ and each $\varphi_i : A_i$ is a fresh variable.

\begin{remark}
  This transformation does not introduce
  new redexes therefore the $\eta$-long normal form of a $\beta$-normal
  term is also $\beta$-normal.
\end{remark}

Let us introduce a new typing system:
\begin{definition}
\label{dfn:longsafe}
We define the set of \defname{long-safe terms}
by induction over the following system of rules:
  $$ \rulename{var_l} \ \rulef{}{x : A\lsentail x : A} \qquad
\rulename{const_l} \ \rulef{}{\lsentail f : A}\quad f \in \Xi \qquad
\rulename{wk_l} \ \rulef{\Gamma \lsentail M : A}{\Delta \lsentail M : A}\quad
\Gamma \subset \Delta$$

$$ \rulename{app_l} \ \rulef{\Gamma \lsentail M : (A_1,\ldots,A_n,B)
\quad
  \Gamma \lsentail N_1 : A_1 \quad \ldots \quad \Gamma \lsentail N_n : A_n
} {\Gamma \lsentail M N_1 \ldots N_n : B} \quad \ord B \leq
\ord \Gamma$$

$$ \rulename{abs_l} \ \rulef{\Gamma, x_1 : A_1, \ldots, x_n : A_n
  \lsentail M : B} {\Gamma \lsentail \lambda x_1^{A_1} \ldots x_n^{A_n} . M :
  (A_1, \ldots ,A_n,B)} \quad \ord(A_1, \ldots ,A_n,B) \leq
\ord \Gamma$$
\smallskip

The subscript in $\lsentail$ stands for ``long''.
\end{definition}
The terminology ``long-safe'' is deliberately suggestive of a forthcoming lemma. Note that long-safe terms are not necessarily in $\eta$-long normal form.

Observe that the system of rules from Def.~\ref{dfn:longsafe} is a sub-system of the typing system of Def.~\ref{def:safelambda} where the application rule is  restricted the same way as the abstraction rule (\ie, it can perform multiple applications at once provided that all the variables in the context of the resulting term have order greater than the order of the term itself). Thus we clearly have:
\begin{lemma}
\label{lem:longsafe_imp_safe}
If a term is long-safe then it is safe.
\end{lemma}
\smallskip

In general, long-safety is not preserved by $\eta$-expansion; For
instance we have
% $f:o,o \lsentail f$ but $f:o,o \not \lsentail \lambda x^o . f x$.
%This remark remains true for closed terms, for instance
$\lsentail \lambda y^o z^o . y : (o,o,o)$ but
$\not \lsentail\lambda x^o . (\lambda y^o z^o . y) x : (o,o,o)$.
On the other hand, $\eta$-reduction preserves long-safety:

\begin{lemma}[$\eta$-reduction of one variable preserves long-safety]
\label{lem:etared_preserve_longsafety}
  $\Gamma \lsentail \lambda \varphi^\tau . M\, \varphi :A $ with $\varphi$ not
  occurring free in $s$ implies $\Gamma \lsentail M :A$.
\end{lemma}
\begin{proof}
  Suppose $\Gamma \lsentail \lambda \varphi^\tau . M\, \varphi :A$. If $M$ is an  abstraction then by construction of $M$ is necessarily safe.  If $M \equiv N_0 \ldots N_p$ with
  $p\geq 1$ then again, since $\lambda \varphi^\tau . N_0 \ldots N_p
  \varphi$ is safe, each of the $N_i$ is safe for $0 \leq i \leq p$
  and for any variable $z$ occurring free in $\lambda \varphi . M\, \varphi$, $\ord{z} \geq \ord(\lambda \varphi^\tau . M\, \varphi) = \ord M$. Since  $\varphi$ does not occur free in $M$, the terms $M$ and $\lambda \varphi^\tau . M\, \varphi$ have the same set of free variables, thus we can use the application rule to form $\Gamma' \lsentail N_0 \ldots N_p : A$ where $\Gamma'$ consists of the typing-assignments for the free variables of $M$. The weakening rules permits us to conclude $\Gamma \lsentail M :A$.
\end{proof}

\begin{lemma}[Long-safety is preserved by $\eta$-long expansion]
\label{lem:longsafe_imp_elnf_longsafe}
$\Gamma \lsentail M :A$ then $\Gamma \lsentail \elnf{M} :A$.
\end{lemma}
\begin{proof}
 First we observe that for any variable or constant $x:A$ we have $x:A \lsentail \elnf{x} :A$. We show this by induction on $\ord{x}$.
It is verified for any ground type variable $x$
since $x = \elnf{x}$.
Step case: $x:A$ with $A=(A_1, \ldots, A_n,o)$ and $n>0$. Let $\varphi_i:A_i$ be fresh variables for $1\leq i\leq n$.
Since $\ord{A_i} < \ord{x}$ the induction hypothesis gives $\varphi_i :A_i \lsentail \elnf{\varphi_i} : A_i$. Using \rulenamet{wk_l} we obtain $x:A, \overline{\varphi} : \overline{A}
  \lsentail \elnf{\varphi_i} :A_i$.  The application rule gives $x :A, \overline{\varphi} : \overline{A} \lsentail x \elnf{\varphi_1} \ldots \elnf{\varphi_n}
  : o$ and the abstraction rule gives $ x :A \lsentail \lambda
  \overline{\varphi} . x \elnf{\varphi_1} \ldots \elnf{\varphi_n} =
  \elnf{x} :A$.


We now prove the lemma by induction on $M$.
The base case is covered by the previous observation.
\emph{Step case:}
\begin{compactitem}
\item $M \equiv x N_1 \ldots N_m$ with $x: (B_1, \ldots, B_m, A)$, $A = (A_1, \ldots, A_n, o)$ for some $m\geq 0$, $n>0$ and $N_i : B_i$ for $1 \leq i \leq
  m$.  Let $\varphi_i: A_i$ be fresh variables for $1\leq i \leq
  n$. By the previous observation we have $\varphi_i :A_i \lsentail \elnf{\varphi_i} :A_i$, the weakening rule then gives us $\Gamma , \overline{\varphi} : \overline{A}
  \lsentail \elnf{\varphi_i} : A_i$.  Since the judgement
  $\Gamma \lsentail x N_1 \ldots N_m : A$ is formed using the \rulenamet{app_l} rule, each $N_j$ must be long-safe for $1\leq j \leq m$, thus by the induction hypothesis we have $\Gamma \lsentail \elnf{N_j} : B_j$ and by weakening we get $\Gamma, \overline{\varphi} :\overline{A} \lsentail \elnf{N_j} : B_j$.  The \rulenamet{app_l}
  rule then gives $\Gamma, \overline{\varphi} :\overline{A} \lsentail x \elnf{N_1} \ldots \elnf{N_m} \elnf{\varphi_1} \ldots \elnf{\varphi_n} : o$. Finally
  the \rulenamet{abs_l} rule gives $\Gamma \lsentail \lambda \overline{\varphi} . x
  \elnf{N_1} \ldots \elnf{N_m} \elnf{\varphi_1} \ldots
  \elnf{\varphi_n} \equiv \elnf{M} : A$, the side-condition of \rulenamet{abs_l} being verified since $\ord{\elnf{s}} = \ord{s}$.


\item $M \equiv N_0 \ldots N_m$ where $N_0$ is an abstraction and $m\geq 1$.
The eta-long normal form is $\elnf{M} \equiv \lambda \overline{\varphi}. \elnf{N_0} \ldots \elnf{N_m} \elnf{\varphi_1}  \ldots \elnf{\varphi_n}$ for some fresh variables $\varphi_1$, \ldots, $\varphi_n$. Again, using the induction hypothesis we can easily derive $\Gamma \lsentail
 \elnf{M} : A$.

\item $M \equiv \lambda \overline{\eta}^{\overline{B}} . N $ where
$N$ of type $C$ and is not an abstraction. The induction hypothesis gives $\Gamma,
  \overline{\eta} : \overline{B} \lsentail \elnf{N} : C$ and using
\rulenamet{abs_l} we get $\Gamma \lsentail \lambda \overline{\eta} . \elnf{N} \equiv \elnf{M} : A$.  \qedhere
\end{compactitem}
\end{proof}

\begin{remark}\hfill
\begin{enumerate}
\item
The converse of this lemma does not hold in general: Performing
$\eta$-reduction over a large abstraction does not in general
preserve long-safety. This does not contradict Lemma
  \ref{lem:etared_preserve_longsafety} which states that safety is
  preserved when performing $\eta$-reduction on an abstraction
  of a \emph{single} variable. The simplest counter-example is
  the
 term $f^{(o,o,o)} \stentail \lambda x^o . f \underline{x}$ which is not long-safe and
whose eta-long normal form $f^{(o,o,o)} \lsentail \lambda x^o y^o .
f x y$ is long-safe. Even for closed terms the converse does not
hold: $\lambda f^{(o,o,o)} g^{((o,o,o),o)} . g(\lambda x^o . f
\underline{x})$ is not long-safe but its eta-normal form $\lambda f^{(o,o,o)}
g^{((o,o,o),o)} . g(\lambda x^o y^o. f x y)$ is long-safe. In fact
even the closed $\beta\eta$-normal term $\lambda
f^{(o,(o,o),o,o)} g^{((o,o),o,o,o),o)} . g(\lambda y^{(o,o)} x^o
. f \underline{x} y)$ which is not long-safe has a long-safe $\eta$-long normal form!

  \item After performing $\eta$-long expansion of a term, all the occurrences of the application rule are made long-safe. Thus if a term remains not long-safe after $\eta$-long expansion, this means that
  some variable occurrence is not bound by the
  first following application of the \rulenamet{abs} rule in the
  typing tree.
  \end{enumerate}
\end{remark}

\begin{lemma}
  \label{lem:safe_iff_etalong_lsafe}
  A simply-typed term is safe if and only if its $\eta$-long normal form is long-safe.
\end{lemma}
\begin{proof}
Let $\Gamma \stentail M : T$. We want to show that we have $\Gamma \sentail M : T$ if and only if $\Gamma \lsentail \elnf{M} : T$. The
`Only if' part can be proved by a trivial induction on the structure of
$\Gamma \sentail M : T$. For the `if' part we proceed by
induction on the structure of the simply-typed term $M$: The variable and constant cases are
trivial. Suppose that $M$ is an application of the form $x N_1
\ldots N_m : A$ for $m\geq 1$. Its $\eta$-long normal form is of the
form $x \elnf{N_1} \ldots \elnf{N_m} \elnf{\varphi_1} \ldots
\elnf{\varphi_m}: o$ for some fresh variables $\varphi_1$, \ldots
$\varphi_m$. By assumption this term is long-safe term therefore we
have $\ord{A}\leq\ord{\Gamma}$ and for $1\leq i \leq m$,
$\elnf{N_i}$ is also long-safe. By the induction hypothesis this
implies that the $N_i$s are all safe. We can then form the judgment
$\Gamma \sentail x N_1 \ldots N_m : A$ using the rules
$\rulename{var}$ and $\rulename{\delta}$ followed by $m-1$
applications of the rule $\rulename{app_{as}}$ and one application
of $\rulename{app}$ (this is allowed since we have
$\ord{A}\leq\ord{\Gamma}$). The case $M\equiv (\lambda x. N) N_1
\ldots N_m$ for $m\geq 1$ is treated identically.

Suppose that $M \equiv \lambda \overline{x}^{\overline B} . N : A$. By assumption,
its  $\eta$-long n.f.\ $\lambda \overline{x} \overline{\varphi} .
\elnf{N} \elnf{\varphi_1} \ldots \elnf{\varphi_m}: A$ (for some
fresh variables $\overline\varphi = \varphi_1 \ldots \varphi_m$) is long-safe. Thus
we have $\ord{A}\leq\ord{\Gamma}$. Furthermore the long-safe subterm
$\elnf{N} \elnf{\varphi_1} \ldots \elnf{\varphi_m}$ is precisely the
eta-long normal form of $N\, \varphi_1 \ldots \varphi_m : o$ therefore by
the induction hypothesis we have that $N \varphi_1 \ldots \varphi_m
:o$ is safe. Since the $\varphi_i$'s are all safe (by rule
$\rulename{var}$), we can ``peel-off'' $m$ applications (performed using
the rules $\rulename{app_{as}}$ or $\rulename{app}$) from the sequent $\Gamma,
\overline{x}:\overline B,
\overline{\varphi}:\overline C \sentail N\, \varphi_1 \ldots
\varphi_m :o$ which gives us the sequent $\Gamma, \overline{x}:\overline B,
\overline{\varphi}:\overline C \asappentail N : A$. Since the $\overline{\varphi}$
variables are fresh for $N$, we can further peel-off an
application of the weakening rule to obtain the judgment $\Gamma,
\overline{x}:\overline B \sentail N : A$. Finally we obtain $\Gamma \sentail
\lambda \overline{x}^{\overline{B}} . N : A$ using the rule $\rulename{abs}$ (which
is permitted since we have $\ord{A}\leq\ord{\Gamma}$).
\end{proof}



\begin{proposition}
\label{prop:safe_iff_elnfsafe}
A term is safe if and only if its $\eta$-long normal form is safe.
\end{proposition}
\begin{proof}
\begin{align*}
  \mbox{(If):}\qquad  \Gamma \sentail \elnf{M}:T &\implies   \Gamma \lsentail \elnf{M}:T &  \mbox{By Lemma \ref{lem:safe_iff_etalong_lsafe} (only if),} \\
  &\implies   \Gamma \sentail M:T &  \mbox{By Lemma \ref{lem:safe_iff_etalong_lsafe} (if).}
\end{align*}
\begin{align*}
  \mbox{(Only if):}\qquad \Gamma \sentail M:T &\implies   \Gamma \lsentail \elnf{M}:T &  \mbox{By Lemma \ref{lem:safe_iff_etalong_lsafe} (only if),} \\
  &\implies \Gamma \sentail \elnf{M}:T &  \mbox{By Lemma \ref{lem:longsafe_imp_safe}\qquad  }
\end{align*}
\qedhere
\end{proof}

%%%% The statement
%%%%  ``In the homogeneous safe lambda calculus, the notion of safe terms and long-safe terms coincide''
%%%% is wrong because of the example given earlier (which is homogeneously typed).

\subsection*{The type inhabitation problem}

It is well known that the simply-typed lambda calculus corresponds
to intuitionistic implicative logic via the Curry-Howard
isomorphism. The theorems of the logic correspond to inhabited
types; Further every inhabitant of a type represents a proof of the
corresponding formula. Similarly, we can consider the fragment of
intuitionistic implicative logic that corresponds to the safe lambda
calculus under the Curry-Howard isomorphism; We call it the
\emph{safe fragment of intuitionistic implicative logic}.

We would like to compare the reasoning power of these
two logics, in other words, to determine which types are inhabited
in the lambda calculus but not in the safe lambda
calculus.\footnote{This problem was raised to our attention by Ugo
dal Lago.}

If types are generated from a single atom $o$, then there is a
positive answer: Every type generated from one atom that is
inhabited in the lambda calculus is also inhabited in the safe
lambda calculus. Indeed, one can transform any unsafe inhabitant $M$
into a safe one of the same type as follows: Compute the eta-long
beta normal form of $M$. Let $x$ be an occurrence of a ground-type
variable in a subterm of the form $\lambda \overline{x} . C[x]$
where $\lambda \overline{x}$ is the binder of $x$ and for some
context $C[\_~]$ different from the identity $C[R]=R$. We replace
the subterm $\lambda \overline{x} . C[x]$ by $\lambda \overline{x}.
x$ in $M$. This transformation is sound because both $C[x]$ and $x$
are of the same ground type. We repeat this procedure until the term
stabilizes. This procedure clearly terminates since the size of the
term decreases strictly after each step. The final term obtained is
safe and of the same type as $M$.

This argument cannot be generalized to types generated from multiple
atoms. In fact there are order-$3$ types with only $2$ atoms that
are inhabited in the simply-typed lambda calculus but not in the
safe lambda calculus. Take for instance the order-$3$ type
 $( ((b, a), b),  ((a, b), a),  a)$ for some distinct atoms $a$ and $b$. It is only inhabited by the following family of terms which are all unsafe:
 \begin{align*}
& \lambda f^{((b, a), b)} g^{((a, b), a)} . g (\lambda x_1^a . f (\lambda y_1^b . x_1)) \\
&\lambda f^{((b, a), b)} g^{((a, b), a)} . g (\lambda x_1^a . f (\lambda y_1^b . g (\lambda x_2^a . y_1))) \\
&\lambda f^{((b, a), b)} g^{((a, b), a)} . g (\lambda x_1^a . f (\lambda y_1^b . g (\lambda x_2^a . f (\lambda y_2^b . x_i))) \qquad\mbox{where $i = 1, 2$} \\
&\lambda f^{((b, a), b)} g^{((a, b), a)} . g (\lambda x_1^a . f (\lambda y_1^b . g (\lambda x_2^a . f (\lambda y_2^b . g (\lambda x_3^a . y_i))) \qquad\mbox{where $i = 1, 2$} \\
&\ldots
\end{align*}

Another example is the type of function composition. For any atom
$a$ and natural number $n\in\nat$, we define the types $n_a$ as
follows: $0_a = a$ and $(n+1)_a = n_a \typear a$. Take three
distinct atoms $a$, $b$ and $c$. For any $i,j,k\in\nat$, we write
$\sigma(i,j,k)$ to denote the type
$$\sigma(i,j,k) \equiv (i_a \typear j_b) \typear (j_b \typear k_c) \typear i_a \typear
k_c \ .$$

For all $i$, $j$, $k$, this type is inhabited in the lambda calculus
by the ``function composition term'':
$$\lambda x y z . y (x\,z) \enspace .$$
This term is safe if and only if $i\geq j$ (for the subterm $x\,z$ is safe iff $i = \ord(i_a) = \ord z \geq \ord(x\,z) = \ord(j_b) = j$). In the case $i<j$, the type
$\sigma(i,j,k)$ may still be safely inhabited. For instance
$\sigma(1,3,4)$ is inhabited by the safe term
$$ \lambda x^{1_a \typear 3_b} y^{3_b \typear 4_c} z^{1_c} . y (x (\lambda u^a . u)) \ .$$
The order-$4$ type $\sigma(0,2,0)$, however, is only inhabited by the unsafe term $\lambda x y z . y (x z) $.


Statman showed \cite{Statman1979} that the problem of deciding
whether a type \emph{defined over an infinite number of ground
atoms} is inhabited (or equivalently of deciding validity of an
intuitionistic implicative formula) is PSPACE-complete. The previous
observations suggest that the validity problem for the safe fragment
of implicative logic may not be PSPACE-hard.


    \allowdisplaybreaks

\section{Expressivity}
\subsection{Numeric functions representable in the safe lambda
calculus}

Natural numbers can be encoded in the simply-typed lambda calculus
using the Church Numerals: each $n\in\nat$ is encoded as the term
$\encode{n} = \lambda s z. s^n z$ of type $I = ((o,o),o,o)$ where
$o$ is a ground type. In 1976 Schwichtenberg \cite{citeulike:622637}
showed the following:


\begin{theorem}[Schwichtenberg 1976]
The numeric functions representable by simply-typed $\lambda$-terms
of type $I\rightarrow \ldots \rightarrow I$ using the Church Numeral
encoding are exactly the multivariate polynomials \emph{extended
with the conditional function}.
\end{theorem}

If we restrict ourselves to safe terms, the representable functions
are exactly the multivariate polynomials:
\begin{theorem}
\label{thm:polychar} The functions representable by safe
$\lambda$-expressions of type $I\rightarrow \ldots \rightarrow I$
are exactly the multivariate polynomials.
\end{theorem}
\proof
  Natural numbers are encoded using the Church Numerals: $\encode{n} =
  \lambda s z. s^n z$.  Addition: For $n,m \in \nat$, $\encode{n+m} =
  \lambda \alpha^{(o,o)} x^o . (\encode{n} \alpha) (\encode{m} \alpha
  x)$. Multiplication: $\encode{n . m} = \lambda \alpha^{(o,o)}
  . \encode{n} (\encode{m} \alpha)$.  All these terms are safe and
  clearly any multivariate polynomial $P(n_1, \ldots, n_k)$ can be
  computed by composing the addition and multiplication terms as
  appropriate.

For the converse, let $U$ be a safe $\lambda$-term of type
$I\rightarrow I\rightarrow I$.  The generalization to terms of type
$I^n \rightarrow I$ for $n>2$ is immediate (they correspond to
polynomials with $n$ variables). W.l.o.g we can assume that $U =
\lambda x y \alpha z. u$ where $u$ is a safe term of ground type in
$\beta$-normal form with $fv(u) \subseteq \{ x, y : I, z :o, \alpha
: o\rightarrow o \}$.

\emph{Notation:} Let $T$ be a set of terms of type $\tau \rightarrow
\tau$ and $T'$ be a set of terms of type $\tau$ then $T \cdot T'$
denotes the set of terms $\{ s s' : \tau \ | \ s \in T \wedge s' \in
T' \}$. We also define $T^k \cdot T'$ recursively as follows:  $T^0
\cdot T' = T'$ and for $k\geq 0$, $T^{k+1} \cdot T' = T \cdot (T^k
\cdot T')$ ({\it i.e.}~$T^k \cdot T'$ denotes $\{ s_1( \ldots
(s_k~s'))  \ | \ s_1, \ldots, s_k \in T \wedge s' \in T' \}$). We
define $T^+\cdot T' = \Union_{k > 0} T^k \cdot T'$ and $T^*\cdot T'
= (T^+\cdot T') \union T'$. For two sets of terms $T$ and $T'$, we
write $T =_\beta T'$ to express that any term of $T$ is
$\beta$-convertible to some term $t'$ of $T'$ and reciprocally.

Let us write $\mathcal{N}^\tau$ for the set of $\beta$-normal terms
of type $\tau$ where $\tau$ ranges in $\{ o, o\rightarrow o, I \}$
and with free variables in $\{ x,y:I, z:o, \alpha:o\rightarrow o\}$.
We write $\mathcal{A}^\tau$ for the subset of $\mathcal{N}^\tau$
consisting of applications only ({\it i.e.}~not abstractions). Let
$B$ be the set of terms of type $(o,o)$ defined by $B = \{ \alpha \}
\union \{ \lambda a.b \ | \ b \in \{a,z\}, a \neq z \}$. It is easy
to see that the following equations hold:
\begin{eqnarray*}
\mathcal{A}^I &=& \{ x,y \} \\
\mathcal{N}^{(o,o)} &=& B \union \mathcal{A}^I \cdot
\mathcal{N}^{(o,o)} = (\mathcal{A}^I)^* \cdot B \\
\mathcal{A}^{(o,o)} &=& \{ \alpha \} \union (\mathcal{A}^I)^+ \cdot B \\
\mathcal{A}^o = \mathcal{N}^o &=& \{ z \} \union \mathcal{A}^{(o,o)} \cdot \mathcal{N}^o = (\mathcal{A}^{(o,o)})^* \cdot \{ z \}
\end{eqnarray*}
Hence $\mathcal{A}^o = \left( \{\alpha \} \union \{x,y\}^+ \cdot
\left( \{\alpha \} \union \{\lambda a.b \ | \ b \in \{a,z\}, a \neq
z \} \right) \right)^* \cdot \{ z \}$. Since $u$ is safe, it cannot
contain terms of the form $\lambda a . z$ with $a \neq z$ occurring
at an operand position, therefore since $u$ belongs to
$\mathcal{A}^o$ we have:
\begin{equation}
u \in \left( \{\alpha\} \union \{x,y\}^+ \cdot \{\alpha,
\underline{i} \} \right)^* \cdot \{ z \} \label{eqn:u}
\end{equation}
where $\underline{i}$ is the identity term of type $o\rightarrow o$.


We observe that $\encode{k}~\underline{i} =_\beta \underline{i}$ for
all $k \in \nat$ and for $l\geq 1$, $k_1, \ldots k_l \in \nat$ we
have $\encode{k_1}\ldots \encode{k_l}~\alpha =_\beta
\encode{k_1\times \ldots \times k_l}~\alpha$. Hence for all $m,n \in
\nat$ we have:
\begin{equation}
\begin{array}{llr}
\{\encode{m},\encode{n}\}^+ \cdot \{\alpha, \underline{i} \} &=_\beta
\{ \underline{i} \} \union
\{ \encode{m^i  n^j}~\alpha \ |\ i+j \geq 1 \} \nonumber \\
&=_\beta \{ \encode{m^i  n^j}~\alpha \ |\ i,j \geq 0 \} & \mbox{since } \underline{i} =_\beta \encode{0}~\alpha \end{array}
\label{eqn:intermediate}
\end{equation}
therefore:
$$\begin{array}{llr}
u[\encode{m}, \encode{n}/x,y] &\in \left( \{ \alpha \} \union \{\encode{m},\encode{n}\}^+ \cdot \{\alpha, \underline{i} \} \right)^* \cdot \{ z \}  & \mbox{by Eq.\ \ref{eqn:u}} \\
&=_\beta \left( \{\alpha \} \union \{ \encode{m^i n^j}~
\alpha \ | \ i,j \geq 0 \} \right)^* \cdot \{ z \} & \mbox{by Eq.\ \ref{eqn:intermediate}}  \\
&=_\beta \left\{ \encode{m^i n^j}~
\alpha \ | \ i,j \geq 0 \right\}^* \cdot \{ z \} & \mbox{since $\alpha~z =_\beta \encode{1}~\alpha~z$}.
\end{array}$$

Furthermore, for all $m,n,r,i,j\in \nat$ we have $\encode{m^i n^j}
~\alpha~(\alpha^r z) =_\beta \alpha^{r + m^i n^j}~z$, hence we have
$u[\encode{m},\encode{n}/x,y] =_\beta \alpha^{p(m,n)} z$ where
$p(m,n) = \sum_{0\leq k \leq d} m^{i_k} n^{j_k}$ for some $i_k,j_k
\geq 0$, $k \in\{ 0,..,d \}$ and $d\geq 0$. Thus $U~\encode{m}
~\encode{n} =_\beta \encode{p(m,n)}$. \qed

\begin{corollary}
The conditional operator $C:I\rightarrow I\rightarrow I \rightarrow
I$ verifying:
$$
C~t~y~z \rightarrow_\beta \left\{
                            \begin{array}{ll}
                              y, & \hbox{if $t
\rightarrow_\beta \encode{0}$;} \\
                              z, & \hbox{if
$t \rightarrow_\beta \encode{n+1}$ .}
                            \end{array}
                          \right.
$$
is not definable in the simply-typed safe lambda calculus.
\end{corollary}

\begin{example}
The term $\lambda F G H \alpha x . F ( \underline{\lambda y . G
\alpha x} ) (H \alpha x)$ used by Schwichtenberg
\cite{citeulike:622637} to define the conditional operator is unsafe
since the underlined subterm, which is of order $1$, occurs at an
operand position and contains an occurrence of $x$ of order $0$.
\end{example}

This corollary tells us that the conditional function is not
definable when numbers are represented by the Church Numerals. It
may still be possible, however, to represent the conditional
function using a different encoding for natural numbers. One
possible way to compensate for the loss of expressivity caused by
the safety constraint consists in introducing countably many domains
of representations for natural numbers. This is the technique which
is used to represent the predecessor function in the simply-typed
lambda calculus \cite{DBLP:journals/jacm/FortuneLO83}.

Remark that the boolean conditional can be represented in the safe
lambda calculus as follows: We encode booleans by terms of type
$B=((o,o),o,o)$. The two truth values are then represented by
$\lambda x^o y^o.x$ and $\lambda x^o y^o.y$ and the conditional by
$\lambda F^B G^B H^B . F~G~H$.

It is also possible to define a conditional operator behaving like
$C$ in the second-order lambda calculus
\cite{DBLP:journals/jacm/FortuneLO83}: natural numbers are
represented by terms $\overline{n} \equiv \Lambda t.\lambda
s^{t\typear t} z^t.s^n(z)$ of type $J \equiv \Delta t.(t\typear t)
\typear (t \typear t)$ and the conditional is encoded by $\lambda
F^J G^J H^J.F~J~(\lambda u^J . G)~H$. The question of whether this
term is safe is not of concern as a notion of safety for
second-order typed term has yet to be defined.






\newcommand{\zaioncencode}{\underline} % Zaionc's notation for word encoding in the lambda calculus

\newcommand{\zaiwordtyp}{\mathbf{B}} % Zaionc's type for word encoded in the lambda calculus
\newcommand{\closedof}[1]{{\rm Cl}(#1)} % notation for the set of closed terms of a certain type

\newcommand{\openedof}[2]{{\rm Op}(#1,#2)} % notation for the set of opened terms M such that \fatlambda{M} in \openedof

\newcommand\wordnum[1]{\mathbf{#1}} % Zaionc's encoding of numbers as words
\newcommand\safedefset{$\lambda^{safe}${\rm def}}

\newcommand\fatlambda{\lambda\kern-0.7em\lambda}
\newcommand\wordapp{{\sf app}}
\newcommand\wordsub{{\sf sub}}


\subsection{Word functions definable in the safe lambda calculus.}
Schwichtenberg's result on numeric functions definable in the lambda
calculus was extended to richer structures: Zaionc studied the
problem for words functions, then functions over trees and
eventually the general case of functions over free algebras
\cite{DBLP:journals/tcs/Leivant93,DBLP:journals/apal/Zaionc91,702481,DBLP:journals/tcs/Zaionc87,
zaionc:csl94}. In this section we consider the case of word
functions expressible in the safe lambda calculus.
\smallskip

\emph{Notations.} For any simple type $\tau$, we write
$\closedof{\tau}$ for the set of closed terms of type $\tau$. We
consider a binary alphabet $\Sigma = \{a,b\}$. The result naturally
extend to any alphabet. We consider the set $\Sigma^*$ of all words
over $\Sigma$. The empty words is denoted $\epsilon$. We write $|w|$
to denote the length of the word $w\in\Sigma^*$. For any $k\in \nat$
we write $\wordnum{k}$ to denote the word $a \ldots a$ with $k$
occurrences of $a$, so that $|\wordnum{k}| = k$. For any $n\geq 1$,
$k\geq 0$ we write $c(n,k)$ to denote the $n$-ary function
$(\Sigma^*)^n \rightarrow \Sigma^*$ mapping constantly on the word
$\wordnum{k}$. The function $\wordapp : (\Sigma^*)^2 \rightarrow
\Sigma^*$ is the usual concatenation function: $\wordapp(x,y)$ is
the word obtain by concatenating $x$ and $y$. The substitution
function $\wordsub : (\Sigma^*)^3 \rightarrow \Sigma^*$ is defined
as follows: $\wordsub(x,y,z)$ is the word obtained from $x$ by
substituting the word $y$ for all occurrences of $a$ and $z$ for all
occurrences of $b$.

Take the type $\zaiwordtyp = (o\typear o)\typear ((o\typear
o)\typear (o\typear o))$ called the binary word type in
\cite{DBLP:journals/tcs/Zaionc87}. There is a 1-1 correspondence
between words over $\Sigma$ and closed term of type $\zaiwordtyp$:
the empty word $\epsilon$ is represented by $\lambda u v x.x$, and
if $w\in \Sigma^*$ is represented by a term $W \in
\closedof{\zaiwordtyp}$ then $a \cdot w$ is represented by $\lambda
u v x. u(W uvx)$ and $a \cdot w$ is represented by $\lambda u v x.
v(W uvx)$. The term representing the word $w$ is denoted by
$\zaioncencode{w}$. A closed term of type $\zaiwordtyp^n \typear
\zaiwordtyp$ is called a \defname{$\lambda$-word theoretic
function}. The function on words $h:(\Sigma^*)^n \rightarrow
\Sigma^*$ is \defname{represented} by the term $H \in
\closedof{\zaiwordtyp^n \typear \zaiwordtyp}$ if and only if for all
$x_1, \ldots, x_n \in \zaiwordtyp^*$, $H \zaioncencode{x_1} \ldots
\zaioncencode{x_n} = \zaioncencode{h x_1 \ldots x_n}$. \bigskip

It was shown in \cite{DBLP:journals/tcs/Zaionc87} that there is a
finite base of word functions such that any $\lambda$-definable word
function is some composition of functions from the base:
\begin{theorem}[Zaionc \cite{DBLP:journals/tcs/Zaionc87}]
The set of $\lambda$-definable word functions is the minimal set containing the following word functions and closed by compositions:
\begin{itemize}
  \item concatenation $\wordapp$;
  \item substitution $\wordsub$;
  \item extraction of the maximal prefix containing only a given letter;
  \item non-emptiness check:  returns $\wordnum{0}$ if the word is $\epsilon$ and $\wordnum{1}$ otherwise, as well as emptiness check;
  \item occurrence checking: returns $\wordnum{1}$ if the word contain an occurrence of a given letter and $\wordnum{0}$ otherwise;
  \item first-occurrence check:  test whether the word begins with a given letter;
  \item all the projections;
  \item all the constant functions.
\end{itemize}
\end{theorem}
The lambda terms representing the base functions are:
\begin{align*}
  {\rm APP} &= \lambda c d u v x.c u v(d u v x), & {\rm SUB} &= \lambda x d e u v x.c(\lambda y.d u v y)(\lambda y . e u v y) x, \\
  {\rm CUT}_a &= \lambda c u v x . c u (\lambda y.x) x, & {\rm CUT}_b &= \lambda c u v x . c (\lambda y.x) v x, \\
  {\rm SQ} &= \lambda c u v x . c (\lambda y.u x) (\lambda y.u x) x, & \overline{{\rm SQ}} &= \lambda c u v x . c (\lambda y.x) (\lambda y.x) (u x), \\
  {\rm BEG}_a &= \lambda c u v x . c (\lambda y.u x) (\lambda y.x) x, & {\rm BEG}_b &= \lambda c u v x . c (\lambda y.x) (\lambda y.u x) x, \\
  {\rm OCC}_a &= \lambda c u v x . c (\lambda y.u x) (\lambda y.y) x, & {\rm OCC}_b &= \lambda c u v x . c (\lambda y.y) (\lambda y.u x) x.
\end{align*}
where {\rm APP} represents concatenation, {\rm SUB} substitution,
{\rm SQ} and $\overline{{\rm SQ}}$ non-emptiness and emptiness checking, ${\rm BEG}_a$ and
${\rm BEG}_b$ first-occurrence test, and ${\rm OCC}_a$ and ${\rm OCC}_a$ occurrence test.

We observe that among these terms only {\rm APP} and {\rm SUB} are
safe. All the other terms are unsafe because they contain terms of
the form $ N (\lambda y .x)$ where $x$ and $y$ are of the same
order. It turns out that this constitutes a base of terms generating
all the functions definable in the safe lambda calculus as the
following theorem states:
\begin{theorem}
\label{thm:wordfunctions_safely_definable}
Let \safedefset\ denote the minimal set containing the following word functions and closed by compositions:
\begin{itemize}
  \item concatenation $\wordapp$;
  \item substitution $\wordsub$;
  \item all the projections;
  \item all the constant functions.
\end{itemize}
The set of word-functions definable in the safe lambda calculus is
precisely \safedefset.
\end{theorem}

The proof follows the same steps as Zaionc's proof.
The first direction is immediate: the terms {\rm APP} and {\rm SUB} are safe
and represent concatenation and substitution. Projections are represented by safe terms of the form $\lambda x_1 \ldots x_n . x_i$ for some $i\in\{1..n\}$, and constant
functions by $\lambda x_1 \ldots x_n . \zaioncencode{w}$ for some $w\in\Sigma^*$.
For composition, take a functions $g:(\Sigma^*)^n \rightarrow \Sigma^*$ represented by safe term $G\in \closedof{\zaiwordtyp^n \typear \zaiwordtyp}$ and functions $f_1,\ldots,f_n :
(\Sigma^*)^p \rightarrow \Sigma^*$ represented by
safe terms $F_1,\ldots F_n$ respectively then the function $$(x_1,\ldots,x_p) \mapsto g(f_1(x_1,\ldots,x_p),\ldots,f_n(x_1,\ldots,x_p))$$ is represented by the term
$\lambda c_1\ldots x_p. G (F_1 c_1 \ldots c_p)\ldots (F_n c_1 \ldots c_p)$ which is also safe.
\bigskip

To show the other directions we need to introduce some more definitions.
We will write $\openedof{n}{k}$ to denote the set of open terms
of the form:
$$c_1:\zaiwordtyp, \ldots c_n :\zaiwordtyp, u:(o,o), v:(o,o), x_{k-1}:o, \ldots, x_0 :o \vdash M : o \ .$$
Thus  we have the following equality up to alpha-conversion:
$$\closedof{\tau(n,k)} = \{ \lambda c_1^\zaiwordtyp \ldots c_n^\zaiwordtyp u^{(o,o)} v^{(o,o)} x_{k-1}^o \ldots x_0^o . M \ | \ M \in \openedof{n}{k}  \} \ .$$

We define the type $\tau(n,k)$ where $n, k\geq1$ as
$(\zaiwordtyp^n,(o,o),(o,o),\overbrace{o,\ldots,o}^{k\hbox{
times}},o)$ and we generalized the notion of representability to
terms of type $\tau(n,k)$ as follows:
\begin{definition}[Function pair representation]
A closed term $T\in\closedof{\tau(n,k)}$ \defname{represents the pair of functions}
$(f,p)$ where $f:(\Sigma^*)^n \rightarrow \Sigma^*$ and $p:(\Sigma^*)^n \rightarrow \{\wordnum{0}, \ldots, \wordnum{k-1}\}$ if for all $w_1,\ldots,w_n\in\Sigma^*$
and for every $i\in \{0\ldots,k-1\}$ we have:
$$
T \zaioncencode{w_1} \ldots \zaioncencode{w_n} =_{\beta\eta} \lambda u v x_{k-1}\ldots x_0 . \zaioncencode{f(w_1,\ldots,w_n)} u v x_{|p(w_1,\ldots,w_n)|} \ .
$$
By extension we will say that an \emph{open} term $M$ from $\openedof{n}{k}$
represents the pair $(f,p)$
iif $M[\zaioncencode{w_1}\ldots \zaioncencode{w_n}/c_1\ldots c_n] =_{\beta\eta} \zaioncencode{f(w_1,\ldots,w_n)} u v x_{|p(w_1,\ldots,w_n)|}$.
\end{definition}

We will call \defname{safe pair} any pair of functions of the form
$(w,c(n,i))$ where $0\leq i\leq k-1$ and $w$ is an $n$-ary function
from \safedefset.

\begin{theorem}[Characterization of the representable pairs]
\label{thm:zaionc_pair_characterization_safe} The function pairs
representable in the safe lambda calculus are precisely the safe
pairs.
\end{theorem}

\begin{proof}
  (Soundness). Take a pair $(w,c(n,i))$ where
  $0\leq i\leq k-1$ and $w$ is an $n$-ary function from \safedefset.
  As observed earlier, all the functions from \safedefset\ are representable
  in the safe lambda calculus: let $\zaioncencode{w}$ be the representative of $w$.
  The pair $(w,c(n,i))$ is then represented by the term
  $ \lambda c_1 \ldots c_n u v x_{k-1} \ldots x_0 . \zaioncencode{w} c_1\ldots c_n u v x_i$.
\smallskip

(Completeness) It suffices to consider safe beta-normal terms from
$\openedof{n}{k}$ only. The result then immediately follows for any
closed safe beta-normal term in $\closedof{\tau(n,k)}$. The subset
of $\openedof{n}{k}$ constituted of $\beta$-normal terms is
generated by the following grammar (see
\cite{DBLP:journals/tcs/Zaionc87}):
\begin{eqnarray*}
  (\alpha_i^k) &R^k &\rightarrow\ x_i \\
  (\beta^k) && \quad|\  u R^k \\
  (\gamma^k) && \quad|\  v R^k \\
  (\delta^k_j) && \quad|\  c_j\ (\overbrace{\lambda z^k. R^{k+1}[z^k,x_0,\ldots, x_{k-1}/x_0,x_1, \ldots, x_k]}^{Q^k(R^{k+1})}) \\
  && \quad\  \quad \ (\lambda z^k. R^{k+1}[z^k,x_0,\ldots, x_{k-1}/x_0,x_1, \ldots, x_k]) \\
  && \quad\  \quad \ R^k
\end{eqnarray*}
for $k\geq 1$, $0\leq i< k$, $0\leq j\leq n$. The notation
$M[\ldots/\ldots]$ denotes the usual simultaneous substitution. The
name of each rule is given in parenthesis. The non-terminals are
$R^k$ for $k\geq1$ and the set of terminals is $\{ z^k, \lambda z^k
\ |\ k\geq 1\} \union \{ x_i ~| i\geq 0 \} \union \{ c_1, \ldots,
c_n, u, v \}$.

We identify a rule name with the right-hand side of the
corresponding rule, thus $\alpha_i^k$ belongs to $\openedof{n}{k}$,
$\beta^k$ and $\gamma^k$ are functions from $\openedof{n}{k}$ to
$\openedof{n}{k}$, and $\delta^k_j$ is a function from
$\openedof{n}{k+1} \times \openedof{n}{k+1} \times \openedof{n}{k}$
to $\openedof{n}{k}$.

We now want to characterize the subset consisted of all \emph{safe}
terms generated by this grammar. The term $\alpha_i^k$ is always
safe, $\beta^k(M)$ and $\gamma^k(M)$ are safe if and only if $M$ is,
and  $\delta^k_j(F,G,H)$ is safe if and only if $Q^k(F)$, $Q^k(G)$
and $H$ are safe. We observe that the free variables of $Q^k(F)$ all
belong to $\{ c_1, \ldots c_n, u, v, x_0,\ldots x_{k}\}$. All these
variables have order greater than $\ord{z}$ except the $x_i$s which
have same order as $z$. Hence since the $x_i$s are not abstracted
together with $z$ we have that $Q^k(F)$ is safe if and only if $F$
is safe and the variables $x_0\ldots x_k$ do not appear free in
$F[z^k,x_0,\ldots, x_{k-1}/x_0,x_1, \ldots, x_k]$, which is the same
as saying that the variables $x_1\ldots x_k$ do not appear free in
$F$. Similarly, $Q^k(G)$ is safe if and only if $G$ is safe and
variables $x_1\ldots x_k$ do not appear free in $G$.

We therefore need to identify the subclass of terms generated by the non-terminal $R^k$ which are safe and which do not have free occurrences of variables in $\{x_1 \ldots x_{k-1}\}$. By applying this requirement to the rules of the previous grammar we obtain the following specialized grammar characterizing the desired subclass:
\begin{eqnarray*}
  (\overline\alpha_0^k) &\overline R^k &\rightarrow\ x_0 \\
  (\overline\beta^k) && \quad|\  u \overline R^k \\
  (\overline\gamma^k) && \quad|\  v \overline R^k  \\
  (\overline\delta^k_j) && \quad|\  c_j\ (\lambda z^k. \overline R^{k+1}[z^k/x_0]) \ (\lambda z^k. \overline R^{k+1} [z^k/x_0]) \ \overline R^k \ .
\end{eqnarray*}
For any term $M$, $Q^k(M)$ is safe if and only if $M$ can be
generated from the non-terminal $\overline R^k$. Thus the subset of
$\closedof{\tau(n,k)}$ consisting of safe beta-normal terms is given
by the grammar:
\begin{eqnarray*}
  (\widetilde\pi^k) &\widetilde S &\rightarrow \lambda c_1 \ldots c_n u v x_{k-1} \ldots x_0 . \widetilde R^k \\
  (\widetilde\alpha_i^k) &\widetilde R^k &\rightarrow\ x_i \\
  (\widetilde\beta^k) && \quad|\  u \widetilde R^k \\
  (\widetilde\gamma^k) && \quad|\  v \widetilde R^k \\
  (\widetilde\delta^k_j) && \quad|\  c_j\ (\lambda z^k. \overline{R^{k+1}}[z^k/x_0]) \ (\lambda z^k. \overline{R^{k+1}}[z^k/x_0]) \ \widetilde R^k \ .
\end{eqnarray*}

Thus to conclude the proof, it suffices to show that every term that
can be generated by this grammar starting with the non-terminal
$\widetilde S$ represents a safe pair.

We proceed by induction and show that the non-terminal $\overline
R^k$ generates terms representing pairs of the form $(w,c(n,0))$
while non-terminals $\widetilde S$ and $\widetilde R^k$ generate
terms representing pairs of the form $(w,c(n,i))$ for $0 \leq i<k$
and $w \in$\safedefset.

\emph{Base case:} The term $\overline\alpha_0^k$ represents the safe pair $(c(n,0),c(n,0))$ while
$\widetilde\alpha_i^k$ represents the safe pair
$(c(n,0),c(n,i))$. \emph{Step case:} Suppose $T\in
\openedof{n}{k}$ represents
 a pair $(w,p)$.  Then $\overline\alpha^k(T)$ and
 $\widetilde\alpha^k(T)$ represent the pair
 $(\wordapp(a,w),p)$, $\overline\beta^k(T)$ and
 $\widetilde\beta^k(T)$ represent the pair
 $(\wordapp(b,w),p)$, and $\overline\pi^k(T) \in \closedof{\tau(n,k)}$ represents the pair $(w,p)$. Now suppose that $E$, $F$ and $G$ represent the pairs
 $(w_e,c(n,0))$, $(w_f,c(n,0))$ and $(w_g,c(n,i))$ respectively.
 Then we have:
 \begin{alignat*}{2}
   \widetilde \delta^k_j (E,F,G) &[\zaioncencode{w_1}\ldots \zaioncencode{w_n}/c_1\ldots c_n] \\
   &= \zaioncencode{w_j}\  (\lambda z^k. E[z^k/x_0])[\zaioncencode{w_1}\ldots \zaioncencode{w_n}/c_1\ldots c_n] \\
       & \qquad\quad (\lambda z^k. F[z^k/x_0])[\zaioncencode{w_1}\ldots \zaioncencode{w_n}/c_1\ldots c_n] \\
       & \qquad\quad  G[\zaioncencode{w_1}\ldots \zaioncencode{w_n}/c_1\ldots c_n] \\
   &=_{\beta\eta} \zaioncencode{w_j}\  (\lambda z^k. E[\zaioncencode{w_1}\ldots \zaioncencode{w_n}/c_1\ldots c_n][z^k/x_0]) \\
       & \qquad\qquad (\lambda z^k. F[\zaioncencode{w_1}\ldots \zaioncencode{w_n}/c_1\ldots c_n][z^k/x_0]) \\
       & \qquad\qquad  (\zaioncencode{w_g(w_1\ldots w_n)}~u~v~x_i) \hspace{4cm}\mbox{$G$ represents $(h,c(n,i))$}\\
   &=_{\beta\eta} \zaioncencode{w_j}\  (\lambda z^k. (\zaioncencode{w_e(w_1\ldots w_n)}~u~v~x_0)[z^k/x_0]) \hspace{2cm}\mbox{$E$ represents $(f,c(n,0))$} \\
       & \qquad\qquad (\lambda z^k. (\zaioncencode{w_f(w_1\ldots w_n)}~u~v~x_0)[z^k/x_0]) \hspace{1.8cm}\mbox{$F$ represents $(g,c(n,0))$} \\
       & \qquad\qquad  (\zaioncencode{w_g(w_1\ldots w_n)}~u~v~x_i)\\
%
   &=_{\beta\eta} \zaioncencode{w_j}\  (\lambda z^k. \zaioncencode{w_e(w_1\ldots w_n)}~u~v~z^k) \\
       & \qquad\qquad (\lambda z^k. \zaioncencode{w_f(w_1\ldots w_n)}~u~v~z^k) \\
       & \qquad\qquad (\zaioncencode{w_g(w_1\ldots w_n)}~u~v~x_i)\\
%
   &=_\eta \zaioncencode{w_j}\  (\zaioncencode{w_e(w_1\ldots w_n)}~u~v)  \ (\zaioncencode{w_f(w_1\ldots w_n)}~u~v) \  (\zaioncencode{w_g(w_1\ldots w_n)}~u~v~x_i)\\
%
   &=_{\beta\eta}  \zaioncencode{w}~u~v~ x_i
 \end{alignat*}
where the word-function $w$ is defined as
$$w: w_1,\ldots,w_n \mapsto \wordapp(\wordsub(w_j,w_e(w_1,\ldots,w_n),w_f(w_1,\ldots,w_n)),w_g(x_1,\ldots,w_n)) \ .$$
  Hence $\widetilde \delta^k_j (E,F,G)$ represents the pair $(w,c(n,i))$.

  The same argument shows that if $E$, $F$ and $G$ all represent safe pairs
then so does $\overline \delta^k_j (E,F,G)$.
\end{proof}


By setting $k=1$ in Theorem \ref{thm:zaionc_pair_characterization_safe} we obtain that every safe term in $T \in \closedof{\tau(n,k)}$ represents some function from \safedefset. This concludes the proof of the characterization Theorem \ref{thm:wordfunctions_safely_definable}.


\subsection{Representability of functions over other structures}\hfill

There is an isomorphism between binary trees and closed terms of
type $\tau =(o\typear o\typear o) \typear o \typear o$. Thus any
closed term of type $\tau\typear\tau \typear \ldots \typear \tau $
represents an $n$-ary function over trees. Zaionc gave a
characterization of the set of tree functions representable in the
simply-typed lambda calculus \cite{DBLP:conf/aluacs/Zaionc88}: it is
precisely the minimal set containing constant functions, projections
and closed under composition and limited primitive recursion. Zaionc
showed that the same characterization holds for the general case of
functions expressed over free algebras
\cite{DBLP:journals/apal/Zaionc91} (they are again given by the
minimal set containing constant functions, projections and closed
under composition and limited recursion). This result subsumes
Schwichtenberg's result on definable numeric functions as well as
Zaionc's own results on definable word and tree functions.

Among all these basic operations, only limited recursion is unsafe.
We conjecture that reciprocally the set of tree functions
representable in the safe lambda calculus is given by the set
containing constant functions, projections and closed under
composition (but not by limited recursion).

    \newcommand\bigo{\mathcal{O}} % big O notation
\newcommand\booltype{\mathsf{B}}

\section{Complexity of the Safe Lambda Calculus}
This section aims at studying the complexity of the problem of deciding beta-eta equivalence of two safe lambda terms.

Let $2_n^m$ denotes the repeated exponential defined by $2_0^m = m$ and $2_{n+1}^m =
2^{2_{n}^m}$. A program is \defname{elementary recursive} if its run-time can be bounded by $2_K^n$ for some constant $K$ where $n$ is the length of the input.

\subsection{Statman's result}

A famous result by Statman  states that deciding the $\beta\eta$-equality of two first-order typable lambda terms is not elementary recursive \cite{Statman:1979:TLE}.
The proof proceeds by encoding the Henkin quantifier elimination of Type Theory into the simply-typed lambda calculus.
Simpler proofs have subsequently been given, one by Mairson in \cite{mairson1992spt} and one by Loader in  \cite{Loader1998}. Both proceed by encoding the Henkin quantifier elimination procedure into the lambda-calculus, as in the original proof, but their use of list iteration to perform quantifier elimination make them much easier to understand.

It turns out that all these encodings rely on the use of unsafe terms to implement the quantifier elimination procedure: Statman's encoding makes use of the conditional
function $\sf sg$ which is not definable in the safe lambda-calculus (\cite{blumong:safelambdacalculus}), Mairson's encoding uses unsafe terms for encoding both quantifier elimination and set membership, and Loader's encoding uses unsafe term to build list iterators. This leads us to conjecture that Type Theory is intrinsically unsafe in the sense that any of its encoding in the lambda calculus is necessary unsafe. Of course this conjecture does not rule out the possibility of encoding another non-elementary problem into the Safe Lambda Calculus.

We start this section by presenting an adaptation of Mairson's encoding where quantifier elimination is safely encoded and by showing why it is problematic to encode set-membership safely. We will then use this encoding to interpret the True Quantifier Boolean Formula (TQBF) problem into the Safe Lambda Calculus, thus showing that deciding beta-eta equality is PSPACE-hard.

\subsection{Mairson's encoding}
We recall the definition of Finite Type Theory. Let $\mathcal{D}_0 = \{\mathbf{true},\mathbf{false}\}$ and $\mathcal{D}_{k+1} =powerset(\mathcal{D}_k)$.
For any $k\geq0$, we write $x^k$, $y^k$ and $z^k$ to denote variables ranging over $\mathcal{D}_k$. Prime formulae are $x^0$, $\mathbf{true}\in y^1$, $\mathbf{false}\in y^1$, and  $x^k \in y^{k+1}$. Formulae are built up from prime formulae using the logical connectives $\zand$,$\zor$,$\rightarrow$,$\neg$ and the quantifiers
$\forall$ and $\exists$. Meyer showed that deciding the truth of such formulae requires nonelementary time \cite{Meyer1974}.
\smallskip

In Mairson's encoding, variables of a given order $k$ are all encoded by terms of the same type $\Delta_k$. Using this encoding,
unsafety manifests itself in two different ways.
\begin{enumerate}[1.]
  \item
        First in the encoding of set membership. The prime formula $x^k \in y^{k+1}$ is encoded as \begin{equation} x^k : \Delta_k, y^{k+1}:\Delta_{k+1} \vdash y^{k+1} (\lambda y^k : \Delta_k . OR (eq_k~\underline{x^k}~y^k)~F : \Delta_k \typear \Delta_{k+1} \typear \Delta_0 \label{eqn:setmembership}\end{equation}
for some terms $OR$, $F$, $eq_k$.
This term is unsafe because of the underline occurrence of $x^k$ which is not abstracted together with $y^k$.

\item Secondly, quantifier elimination is performed by using a list iterator $\mathbf{D}_{k+1}$ which acts like the $fold\_right$ function from functional programming languages over the list of all elements of $\mathcal{D}_k$.
Thus for instance the formula $\forall x^0 . \exists y^0 . x^0 \zor y^0$
is encoded as $$\vdash \mathbf{D}_1 (\lambda x^0:\Delta_0. AND (\mathbf{D}_1 (\lambda y^0:\Delta_0. OR (\underline{x^0} \zor y^0)) F)) T \ .$$
This term is unsafe since the underlined occurrence is unsafely bound. This is due to the presence of two nested quantifiers in the formula, which are encoded as two nested list iterations. More generally, nested binding will be encoded safely if and only if every variable $x$ in the formula is bound by the first quantifier $\exists z$ or $\forall z$ in the path to the root of the AST of the formula verifying $\ord{z} \geq \ord{x}$. For instance if set-membership could be encoded safely then the interpretation of $\forall x^k . \exists y^{k+1} . x^k \in y^{k+1}$ would be unsafe whereas the encoding of $\forall y^{k+1} . \exists x^k . x^k \in y^{k+1}$ would be safe.
\end{enumerate}

Surprisingly, the unsafety of the quantifier elimination procedure can be easily overcome. The idea is as follows. We introduce multiple domains of representation for a given formula. An element of $\mathcal{D}_k$ is thereby represented by countably many terms of type $\Delta_k^n$ where $n\in\nat$ indicates the level of the domain of representation. The type $\Delta_k^n$ is defined in such a way that its order strictly increases as $n$ grows. Furthermore, there exists a term that can lower the domain of representation of a given term. Thus each formula variable can have a different domain of representation, and since there are infinitely many such domains, it is always possible to find an assignment of representation domains to variables such that the resulting encoding term is safe.

For set-membership, however, there is no obvious way to obtain a safe encoding. In order to turn Mairson's encoding of set-membership (Eq.\ \ref{eqn:setmembership}) into a safe term, we would need to have access to a function that changes the domain of representation of an encoded higher-order value of the type-hierarchy. Unfortunately, such transformation is intrinsically unsafe!
\smallskip

We now present the encoding in details.

\subsubsection{Encoding basic boolean operations}

Let $o$ be a base type and define the family of types $\sigma_0 \equiv o$, $\sigma_{n+1} \equiv \sigma_n\typear\sigma_n$ verifying $\ord{\sigma_n} = n$.


Booleans are encoded over domains $\booltype_n \equiv \sigma_n\typear o\typear o\typear o$ for $n\geq0$, each type $\booltype_n$ being of order $n+1$. We write $\underline{i}_{n+1}$ to denote the term $\lambda x^{\sigma_n}.x : \sigma_{n+1}$ for $n\geq0$.

The truth values $\mathbf{true}$ and $\mathbf{false}$ are represented by the following terms parameterized by $n \in \nat$:
\begin{align*}
  T^n &\equiv \lambda u^{\sigma_n} x^o y^o .x : \booltype_{n}\\
  F^n &\equiv \lambda u^{\sigma_n} x^o y^o .y : \booltype_{n}
\end{align*}
Clearly these terms as safe. Moreover the following relations hold for all $n,n'\geq 0$:
\begin{align*}
  \lambda u^{\sigma_{n'}} . T^{n+1}~\underline{i}_{n+1}  &\betared  T^{n'} \\
  \lambda u^{\sigma_{n'}} . F^{n+1}~\underline{i}_{n+1}  &\betared  F^{n'}
\end{align*}
Hence it is possible to change the domain of representation of a Boolean value from a higher-level to another arbitrary level using the transformation:
$$ \mathbf{C}^{n+1\mapsto n'}_0 \equiv \lambda m^{\booltype_{n+1}} u^{\sigma_{n'}}. m~\underline{i}_{n+1}$$
so that if a term $M : \booltype_n$ for $n\geq1$ is beta-eta convertible to $T^n$ (resp.\ $F^n$) then $\mathbf{C}^{n\mapsto n'}_0~M :\booltype_{n'}$ is beta-eta convertible to $T^{n'}$ (resp.\ $F^{n'}$).

Observe that although the term $\mathbf{C}^{n+1\mapsto n'}_0$ is safe for all $n,n'\geq 0$, if we apply it to a variable then the resulting term
$$ x:B_{n+1} \vdash \mathbf{C}_{n+1\mapsto n'}~x : B_{n}$$
is safe if and only if the transformation decreases the domain of representation of $x$ \ie $\ord{B_{n+1}}\geq\ord{B_{n'}}$.


Boolean functions are encoded by the following safe terms parameterized by $n$:
\begin{align*}
AND^n &\equiv \lambda p^{\booltype_n} q^{\booltype_n} u^{\sigma_n} x^o y^o . p~u~(q~u~x~y)~y : \booltype_n\typear\booltype_n\typear\booltype_n \\
OR^n &\equiv \lambda p^{\booltype_n} q^{\booltype_n} u^{\sigma_n} x^o y^o . p~u~x~(q~u~x~y) : \booltype_n\typear\booltype_{n}\typear\booltype_n \\
NOT^n &\equiv \lambda p^{\booltype_n} u^{\sigma_n} x^o \lambda y^o . p~u~y~x : \booltype_n\typear\booltype_n\typear\booltype_n
%\\ IF^n &\equiv \lambda p^{\booltype_n} q^{\booltype_n} u^{\sigma_n} x^o y^o. OR^n (NOT^n p)~q : \booltype_n\typear\booltype_n\typear\booltype_n
\end{align*}

\subsubsection{Coding elements of the type hierarchy}
For any $n\in\nat$ we define the hierarchy of type $\Delta_k^n$ as follows:
$\Delta_0^n \equiv \booltype_n$ and $\Delta_{k+1}^n \equiv {\Delta_k^n}^*$ where for any type $\alpha$, $\alpha^* = (\alpha \typear \tau \typear \tau)\typear \tau \typear \tau$.

An occurrence of a formula variable $x^k$ will be encoded as a term variable $x^k$ of type $\Delta_{k}^n$ for some level of domain representation $n\in\nat$.
Following Mairson's  encoding, each set $\mathcal{D}_k$ is represented by a list $\mathbf{D}_k^n$ constituted of all its elements:
\begin{align*}
\mathbf{D}_0^n &\equiv \lambda c^{\booltype_n \typear \tau \typear \tau} e^\tau . c~T^n~(c~F^n~e) : \Delta_1^n \\
\mathbf{D}_{k+1}^n &\equiv powerset~\mathbf{D}_k^n : \Delta_{k+2}^n
\end{align*}
where
\begin{align*}
  powerset &\equiv \lambda {A^*}^{(\alpha \typear \alpha^{**} \typear \alpha^{**}) \typear \alpha^{**} \typear \alpha^{**}}. \\
&\qquad  A^*~double~(\lambda c^{\alpha^* \typear \tau\typear \tau} b^\tau . c~(\lambda c'^{\alpha\typear \tau\typear \tau} b'^\tau.b')~b)\\
 &: ((\alpha \typear \alpha^{**} \typear \alpha^{**}) \typear \alpha^{**} \typear \alpha^{**})\typear \alpha^{**} \\
  double &\equiv \lambda x^\alpha~l^{(\alpha^* \typear \tau\typear \tau)\typear \tau\typear \tau}~ c^{\alpha^*\typear \tau\typear \tau}~b^\tau. \\
  & \qquad \qquad l(\lambda e^{\alpha^*}.c~(\lambda c'^{\alpha\typear \tau\typear \tau}~ b'^\tau.c'~\underline{x}~(e~c'~b')))(l~c~b)\\
 &: \alpha \typear \alpha^{**} \typear \alpha^{**}
\end{align*}
In all these terms, the only variable occurrence that is potentially unsafe is the underlined occurrence $x$ in $double$. This occurrence if safely bound just when $\ord{\alpha} \geq \ord{\tau}$.
Consequently for all $k,n\geq0$, $\mathbf{D}_k^n$ is safe if and only if $\ord{\alpha} \geq \ord{\tau}$.


\subsubsection{Quantifier elimination}
Terms of type $\Delta_{k+1}^n$ are now used as iterators over list of elements of type $\Delta_k^n$ and we set $\tau \equiv \booltype_n$ in the type $\Delta_{k+1}^n$ in order to iterate a level-$n$ Boolean function. Since $\ord{\Delta_k^n} \geq \ord{\booltype_n}$ for all $n$, all the instantiations of the terms $\mathbf{D}_k^n$ will be safe. Following \cite{mairson1992spt}, quantifier elimination interprets the formula $\forall x^k.\Phi(x^k)$ as the iterated conjunction:
$\mathbf{C}_0^{n\mapsto 0} \left( \mathbf{D}_k^n(\lambda x^k:\Delta_k^n.AND^n(\hat\Phi~x^k))~T^n \right)$ where $\hat\Phi$ is the interpretation of $\Phi$
and $n$ is the representation level chosen for the variable $x^k$; similarly $\exists x^k.\Phi(x^k)$  is interpreted by the iterated disjunction $\mathbf{C}_0^{n\mapsto 0} \left( \mathbf{D}_k^n(\lambda x^k:\Delta_k^n.AND^n(\hat\Phi~x^k))~T^n\right)$.

Let $x^{k_p}_p \ldots x^{k_1}_1$ for $p\geq1$ be the list of variables appearing in the formula. W.l.o.g.\ we can assume that they are given in the order of appearance of their binder in the formula \ie $x^{k_p}_p$ is bound by the leftmost binder. We fix the domain of representation of each variable as follows. The right-most variable $x^{k_1}_1$ will be encoded in the domain $\Delta^0_{k_1}$; suppose that for $1\leq i< p$ the domain of representation of $x^{k_i}_i$ is $\Delta^l_{k_l}$ then the domain of representation of $x^{k_{i+1}}_{i+1}$ is defined as
$\Delta^{l'}_{k_{i+1}}$ where $l'$ is the smallest natural number such that $\ord{\Delta^{l'}_{k_{i+1}}}$ is strictly greater than $\ord{\Delta^{l}_{k_i}}$.

This way, since variables that are bound first have higher order, the variables that are bound in the nested list-iterations (corresponding to the nested quantifiers in the formula) are necessarily safely bound.

\begin{example}
The formula  $\forall x^0 . \exists y^0 . x^0 \zor y^0$, encoded by an unsafe term using Mairson's encoding, is represented in our encoding by the safe term:
 $$\vdash \mathbf{C}_0^{1\mapsto 0} \left( \mathbf{D}_0^1~(\lambda x^0:\Delta_0^1. AND^0 ( \mathbf{D}_0^0 ~(\lambda y^0:\Delta_0^0. OR^0 (OR^0~(\mathbf{C}_0^{1\mapsto 0}~x^0)~y^0))~F^0))~T^1 \right)\ .$$
\end{example}


\subsubsection{Set-membership}
To complete the interpretation of prime formulae, we would need to show how to encode set membership. The use of multiple domains of representation does not suffice to turn Mairson's encoding into a safe term. We would further need to have a version of the Booleans conversion term $\mathbf{C}^{n+1\mapsto n'}_0$ generalized to higher-order sets.
This transformation can be interpreted as the simply-typed term:
$$ \mathbf{C}^{n\mapsto n'}_{k+1} \equiv \lambda m^{\Delta_{k+1}^n} u^{\Delta_k^n\typear \tau\typear \tau} v^\tau. m (\lambda z^{\Delta_k^n} w^\tau . \underline{u (\underline{\mathbf{C}^{n\mapsto n'}_k z}) w}) v : \Delta_{k+1}^n \typear \Delta_{k+1}^{n'}$$
Unfortunately this term is safe if and only if $n=n'$ (the largest underlined subterm is safe just when $n\geq n'$ and the other underline subterm is safe just when $n'\geq n$), thus making this transformation useless in the Safe Lambda Calculus.

This leads us to conjecture that the set-membership test function is intrinsically unsafe.
\smallskip


%%%%%%%%%%%%%%%%%%%%%%%
%% The following commented section gives an UNSAFE encoding of set-memberhsip.

If $\mathbf{C}^{n\mapsto n'}_{k+1}$ were safely representable then the encoding would go as follows: We set $\tau \equiv \booltype_0$ in the types  $\Delta_{k+1}^n$ for all $n,k\geq 0$ in order to iterate a level-$0$ Boolean function.
Firstly, the formulae ``$\mathbf{true} \in y^1$'' and ``$\mathbf{false} \in y^1$'' can be encoded by the safe terms $y^1 (\lambda x^0 . OR^0~x^0) F^0$ and $y^1 (\lambda x^0. OR^0(NOT^0~x^0)) F^0$ respectively.
For the general case ``$x^k\in y^{k+1}$''
we proceed as in \cite{mairson1992spt} by introducing lambda-terms encoding set equality, set membership and subset tests, and we further parameterize these encoding by $n\in\nat$.
\begin{align*}
member_{k+1}^{n+1} &\equiv \lambda x^{\Delta_k^{n+1}} y^{\Delta_{k+1}^{n+1}}. \\
& \quad\ (\mathbf{C}_{k+1}^{n+1\mapsto n}~y)~(\lambda z^{\Delta_k^n} . OR^0 (eq_k^n~(\mathbf{C}^{n+1\mapsto n}_k~x)~z))~F^0\\
  & : \Delta_k^{n+1} \typear \Delta_{k+1}^{n+1} \typear \booltype_0
\\
subset_{k+1}^n &\equiv \lambda x^{\Delta_{k+1}^n} y^{\Delta_{k+1}^n}. \\
  & \qquad x~(\lambda x^{\Delta_k^n} . AND^0 (member_{k+1}^n~x~y))~T^0 \\
  & : \Delta_{k+1}^n \typear \Delta_{k+1}^n \typear\booltype_0
\\
eq_0^n &\equiv \lambda x^{\booltype_n} .\lambda y^{\booltype_n}. \mathbf{C}_0^{n\mapsto 0}~ \left(OR^n (AND^n~x~y) (AND^n (NOT^n~x)(NOT^n~y))\right) \\
 &: \booltype_n \typear \booltype_n \typear\booltype_0
\\
eq_{k+1}^n &\equiv \lambda x^{\Delta_{k+1}^n}~ y^{\Delta_{k+1}^n}. \\
   & \qquad
   (\lambda op^{\Delta_{k+1}^n\typear\Delta_{k+1}^n\typear\booltype_0}. AND^0 (op~x~y)(op~y~x))~subset_{k+1}^n \\
  & : \Delta_{k+1}^n \typear \Delta_{k+1}^n \typear \booltype_0
\end{align*}
The variables in the definition of $eq_{k+1}^n$ and $subset_{k+1}^n$ are safely bounds. Moreover, the occurrence of $x$ in $member_{k+1}^{n+1}$ is now safely bound (this was not the case in Mairson's original encoding) thanks to the fact that the representation domain of $z$ is lower than that of $x$. Unfortunately, the term is not completely safe because it uses the unsafe conversion terms $\mathbf{C}_k^{n\mapsto n'}$ for $k\geq1$.

The formula $x^k\in y^{k+1}$ can then encoded by
$$x:\Delta_k^n, y:\Delta_{k+1}^{n'}\vdash member_{k+1}^{u}~ (\mathbf{C}_k^{n\mapsto u}~x)~(\mathbf{C}^{n'\mapsto u}_{k+1} ~y) : \booltype_0$$
for some $n,n'\geq 2$ and $u = \min(n,n')+1$.

%%
%%%%%%%%%%%%%%%%%%%%%%%


\subsection{PSPACE-hardness}
We observe that instances of the True Quantified Boolean Formulae satisfaction problem (TQBF) are special instances of the decision problem for Finite Type Theory. These instances corresponds to formulae in which set membership is not allowed and variables are all taken from the base domain $\mathcal{D}_0$.
As we have shown in the previous section, such restricted formulae can be safely encoded in the Safe Lambda Calculus. Therefore since TQBF is PSPACE-complete, deciding $\beta\eta$-equality of two terms of the Safe Lambda Calculus is PSPACE-hard.

%%We assume that the quantified propositional formula is given in prenex form:
%%$$\$_{n-1} x_{n-1} \ldots \$_0 x_0 . \psi(x_0, \ldots, x_{n-1})$$
%%where $\$_i \in \{\exists,\forall\}$ for $0\leq i\leq n-1$.
%%
%%The encoding is as follows:
%%\begin{align*}
%%\sem{1} &= T^0  : \booltype \\
%%\sem{0} &= F^0 : \booltype \\
%%\sem{x_i} &= x_i\downarrow_0 = x_i~T^{i-1}~F^{i-1}\ldots T^1~F^1: \booltype \qquad \hbox{where $x_i:\booltype_i$}\\
%%\sem{\psi_1\zand \psi_2} &= AND^0~\sem{\psi_1}~\sem{\psi_2}
%%:\booltype  \\
%%\sem{\psi_1\zor \psi_2} &= OR^0~\sem{\psi_1}~\sem{\psi_2}
%%:\booltype  \\
%%\sem{\neg \psi} &= NOT^0~\sem{\psi}
%%:\booltype  \\
%%\sem{\forall x_i.\psi(\ldots, x_i, \ldots)} & = \mathbf{D}_0^i(\lambda x^{\booltype_i}. AND^0~\sem{\psi(\ldots, x_i, \ldots)})~T^0 :\booltype\\
%%\sem{\exists x_i.\psi(\ldots, x_i, \ldots)} & = \mathbf{D}_0^i(\lambda x^{\booltype_i}.OR^0~\sem{\psi(\ldots, x_i, \ldots)})~F^0 :\booltype
%%\end{align*}
%%The size of $\sem{\psi}$ is in $\bigo(|\psi|^2)$.
%
%It is easy to check that this encoding is safe.
\begin{example}
Using the encoding where $\tau$ is set to  $\booltype_0$ in the types $\Delta_k^n$ for all $k,n\geq 0$, the formula $\forall x \exists y \exists z (x\zor y\zor z)\zand(\neg x\zor \neg y\zor \neg z)$ is represented by the safe term:
\begin{align*}
\vdash &\mathbf{D}_0^2(\lambda x^{\booltype_2}. AND^0\\
&\quad\quad (\mathbf{D}_0^1(\lambda y^{\booltype_1}.OR^0\\
&\quad\quad\quad (\mathbf{D}_0^0(\lambda z^{\booltype_0}.OR^0\\
&\quad\quad\quad\quad (AND^0 (OR^0(OR^0~(\mathbf{C}_0^{2\mapsto 0}~x)~(\mathbf{C}_0^{1\mapsto 0}~y))z) \\
&\quad\quad\quad\quad\quad (OR^0(OR^0(NEG^0 (\mathbf{C}_0^{2\mapsto 0}~x))(NEG^0 (\mathbf{C}_0^{1\mapsto 0}~y)))(NEG^0~z))) \\
&\quad\quad\quad )F^0)\\
&\quad\quad)F^0)\\
&\quad) T^0
\end{align*}
\end{example}

\begin{remark}
Since the Boolean satisfaction problem (SAT) can be reduced to TQBF (where formulae are restricted to use only existential quantifiers), the Safe Lambda Calculus is also NP-hard. In \cite{asperti-np}, Asperti gave an interpretation of SAT in the simply-typed lambda calculus but his encoding relies on unsafe terms.
\end{remark}

\subsection{Other complexity results}

\subsubsection{The normalization problem}
It is well-known that normalization is non-elementary for the simply-typed lambda calculus. This remains true in the safe simply typed lambda calculus as the following example shows: If we write $\underline{k}$ to denote the $k$th Church Numeral $\lambda s z . \overbrace{s( \ldots (s (s}^{k\hbox{ times}}z) \ldots)$ then the safe term $\overbrace{\underline{2}~\underline{2}\ldots \underline{2}}^{n\hbox{ times}}$ has length $\bigo(n)$ whereas its normal form
$\underline{2_n^0}$ has length $\bigo(2_n^0)$.

\subsubsection{Upper bound}
At present, no upper bound is known for the problem of deciding equality of two safe lambda terms. Proceeding by reduction to the normalization problem does not help since there is no interesting upper-bound for this more general problem (which is non-elementary as shown in the previous paragraph).

\subsubsection{Better lower bound?}
The PSPACE-hard lower bound for the equality problem is very coarse. In fact we suspect the problem to be non-elementary.


    
\section{A game-semantic account of safety}
\label{sec:gamesemaccount} Our aim is to characterize safety by game
semantics. We shall assume that the reader is familiar with the
basics of game semantics; For an introduction, we recommend
\cite{abramsky:game-semantics-tutorial}. Recall that a
\emph{justified sequence} over an arena is an alternating sequence
of O-moves and P-moves such that every move $m$, except the opening
move, has a pointer to some earlier occurrence of the move $m_0$
such that $m_0$ enables $m$ in the arena. A \emph{play} is just a
justified sequence that satisfies Visibility and Well-Bracketing. A
basic result in game semantics is that $\lambda$-terms are denoted
by \emph{innocent strategies}, which are strategies that depend only
on the \emph{P-view} of a play. The main result
(Theorem~\ref{thm:safeincrejust}) of this section is that if a
$\lambda$-term is safe, then its game semantics (is an innocent
strategy that) is, what we call, \emph{P-incrementally justified}. In such a
strategy, pointers emanating from the P-moves of a play are uniquely
reconstructible from the underlying sequence of moves and pointers
from the O-moves therein: specifically a P-question always points to
the last pending O-question (in the P-view) of a greater order.

The proof of Theorem~\ref{thm:safeincrejust} depends on a
Correspondence Theorem (see the Appendix) that relates the strategy
denotation of a $\lambda$-term $M$ to the set of \emph{traversals}
over a souped-up abstract syntax tree of the $\eta$-long form of $M$.
In the language of game semantics, traversals are just (concrete
representations of) the \emph{uncovering} (in the sense of Hyland
and Ong \cite{hylandong_pcf}) of plays in the strategy denotation.

The useful transference technique between plays and traversals was
originally introduced by one of us \cite{OngLics2006} for studying
the decidability of monadic second-order theories of infinite structures generated by
higher-order grammars (in which the $\Sigma$-constants or terminal symbols are at most
order 1, and \emph{uninterpreted}).
% In this setting, free variables are interpreted
% as constructors and therefore they do not have the ``full power'' of
% true free variables and are limited to order $1$ at most. Also,
% although the grammar can perform higher-order computations, the
% structure being studied is itself of ground type.
In the Appendix, we present an extension of this framework to the
general case of the simply-typed lambda calculus with free variables
of any order. A new traversal rule is introduced to handle nodes
labelled with free variables. Also new nodes are added to the
computation tree to account for the answer moves of the game
semantics, thus enabling the framework to model languages with
interpreted constants such as \pcf~(by adding traversal rules to
handle constant nodes).

\subsection*{Incrementally-bound computation tree}
 In \cite{OngLics2006} the computation tree of a grammar is
defined as the unravelling of a finite graph representing the \emph{long
transform} of a grammar. Similarly we define the computation tree of
a $\lambda$-term as an abstract syntax tree of its $\eta$-long
normal form.  We write $l\langle t_1, \ldots, t_n \rangle$ with $n
\geq 0$ to denote the tree with a root labelled $l$ with $n$
children subtrees $t_1$, \ldots, $t_n$. In the following we consider
simply-typed terms not necessarily safe unless mentioned.

\begin{definition}\rm
\label{dfn:comptree}
  The \defname{computation tree} $\tau(M)$ of a simply-typed term
  $\Gamma \stentail M:T$ with variable names in a countable set
  $\mathcal{V}$ is a tree with labels in $$ \{ @ \} \union \mathcal{V}
  \union \{ \lambda x_1 \ldots x_n \ | \ x_1 ,\ldots, x_n \in
  \mathcal{V}, n\in\nat \}$$ defined from its $\eta$-long form as follows. Suppose $\overline{x} = x_1 \ldots x_n$ for $n\geq 0$ then
\begin{eqnarray*}
  \mbox{for $m\geq 0$, $z \in \mathcal{V}$: } \tau(\lambda \overline{x} . z s_1 \ldots s_m : o) &=& \lambda \overline{x} \langle z \langle\tau(s_1),\ldots,\tau(s_m)\rangle\rangle \\
  \mbox{for $m \geq 1$: } \tau(\lambda \overline{x} . (\lambda y.t) s_1 \ldots s_m :o) &=& \lambda \overline{x} \langle @ \langle \tau(\lambda y.t),\tau(s_1),\ldots,\tau(s_m) \rangle \rangle \ .
\end{eqnarray*}
\end{definition}

\begin{example}
\label{examp:comptree}
  Take $\vdash \lambda f^{o \typear o} .
(\lambda u^{o \typear o} . u) f : (o \typear o) \typear
o \typear o$.
\bigskip

\noindent
\begin{tabular}{cc}
Its $\eta$-long normal form is: & Its computation tree is:\\[8pt]
\begin{minipage}{0.45\textwidth}
\centering 
$\begin{array}{ll}
 &\vdash  \lambda f^{o \typear o} z^o . \\
&\qquad(\lambda u^{o \typear o} v^o . u (\lambda.v)) \\
&\qquad(\lambda y^o. f y) \\
&\qquad(\lambda.z) \\
&: (o \typear o) \typear o \typear o
\end{array}$
\end{minipage}
&
\begin{minipage}{0.45\textwidth}
\centering
\psset{levelsep=5ex,linewidth=0.5pt,nodesep=1pt,arcangle=-20,arrowsize=2pt 1}
${\pstree{\TR{\lambda f z}}{\pstree{\TR{@}}{\pstree{\TR{\lambda u v}}{\pstree{\TR{u}}{\pstree{\TR{\lambda }}{\TR{v}}}}\pstree{\TR{\lambda y}}{\pstree{\TR{f}}{\pstree{\TR{\lambda }}{\TR{y}}}} \pstree{\TR{\lambda }}{\TR{z}}}}
}$
\end{minipage}
\end{tabular}
\end{example}

\begin{example}
  Take $\vdash \lambda u^o v^{((o \typear o) \typear o)} . (\lambda x^o . v (\lambda z^o . x)) u : o \typear ((o \typear o) \typear o) \typear o$.
  \bigskip
  
\noindent
\begin{tabular}{cc}
Its $\eta$-long normal form is: & Its computation tree is:\\[8pt]
\begin{minipage}{0.45\textwidth}
\centering
$\begin{array}{ll}
 &\vdash  \lambda u^o v^{((o \typear o) \typear o)} . \\
&\qquad(\lambda x^o . v (\lambda z^o . x)) u \\
&: o \typear ((o \typear o) \typear o) \typear o
\end{array}$
\end{minipage}
&
\begin{minipage}{0.45\textwidth}
\centering
$\pstree{\TR{\lambda u v}}{\pstree{\TR{@}}{\pstree{\TR{\lambda x}}{\pstree{\TR{v}}{\pstree{\TR{\lambda z}}{\TR{x}}}}\pstree{\TR{\lambda }}{\TR{u}}}}
$
\end{minipage}
\end{tabular}
\end{example}

Even-level nodes are $\lambda$-nodes (the root is on level 0). A
single $\lambda$-node can represent several consecutive variable
abstractions or it can just be a \emph{dummy lambda} if the
corresponding subterm is of ground type.  Odd-level nodes are
variable or application nodes.

The \defname{order} of a node $n$, written $\ord{n}$, is defined as
follows: @-nodes have order $0$. The order of a variable-node is the
type-order of the variable labelling it. The order of the root node
is the type-order of $(A_1,\ldots,A_p, T)$ where $A_1,\ldots, A_p$
are the types of the variables in the context $\Gamma$. Finally, the
order of a lambda node different from the root is the type-order of
the term represented by the sub-tree rooted at that node.

We say that a variable node $n$ labelled $x$ is \defname{bound} by a
node $m$, and $m$ is called the \defname{binder} of $n$, if $m$ is
the closest node in the path from $n$ to the root such that $m$ is
labelled $\lambda \overline{\xi}$ with $x\in \overline{\xi}$.


We introduce a class of computation trees in which the binder node
is uniquely determined by the nodes' orders:
\begin{definition}\rm
  A computation tree is \defname{incrementally-bound} if for all
  variable node $x$, either $x$ is \emph{bound} by the first
  $\lambda$-node in the path to the root with order $> \ord{x}$ or $x$
  is a \emph{free variable} and all the $\lambda$-nodes in the path to
  the root except the root have order $\leq \ord{x}$.
\end{definition}

\begin{proposition}[Safety and incremental-binding] \hfill
\label{prop:safe_imp_incrbound}
\begin{enumerate}[(i)]
\item If $M$ is safe then $\tau(M)$ is incrementally-bound.
\item Conversely, if $M$ is a \emph{closed} simply-typed term and $\tau(M)$
is incrementally-bound then $M$ is safe.
\end{enumerate}
\end{proposition}
\proof
  (i) Suppose that $M$ is safe. By Lemma
  \ref{prop:safe_iff_elnfsafe} the $\eta$-long form of $M$ is safe
  therefore $\tau(M)$ is the tree representation of a safe term.

In the safe lambda calculus, the variables in the context with the
lowest order must be all abstracted at once when using the
abstraction rule. Since the computation tree merges consecutive
abstractions into a single node, any variable $x$ occurring free in
the subtree rooted at a node $\lambda \overline{\xi}$ different from
the root must have order greater or equal to $\ord{\lambda
  \overline{\xi}}$. Conversely, if a lambda node $\lambda
\overline{\xi}$ binds a variable node $x$ then $\ord{\lambda
  \overline{\xi}} = 1+\max_{z\in\overline{\xi}} \ord{z} > \ord{x}$.

Let $x$ be a bound variable node. Its binder occurs in the path from
$x$ to the root, therefore, according to the previous observation,
$x$ must be bound by the first $\lambda$-node occurring in this path
with order $>\ord{x}$. Let $x$ be a free variable node then $x$ is
not bound by any of the $\lambda$-nodes occurring in the path to the
root. Once again, by the previous observation, all these
$\lambda$-nodes except the root have order smaller than $\ord{x}$.
Hence $\tau$ is incrementally-bound.

(ii) Let $M$ be a closed term such that $\tau(M)$ is
incrementally-bound.  W.l.o.g. we can assume that $M$ is in $\eta$-long
form.  We prove that $M$ is safe by induction on its structure. The
base case $M = \lambda \overline{\xi} . x$ for some variable $x$ is
trivial.  \emph{Step case:} If $M = \lambda \overline{\xi} . N_1
\ldots N_p$.  Let $i$ range over $1..p$. We have $N_i \equiv \lambda
\overline{\eta_i} . N'_i$ for some non-abstraction term $N'_i$. By
the induction hypothesis, $\lambda \overline{\xi} . N_i = \lambda
\overline{\xi} \overline{\eta_i} . N'_i$ is a safe closed term, and
consequently $N'_i$ is necessarily safe. Let $z$ be a free variable
of $N'_i$ not bound by $\lambda \overline{\eta_i}$ in $N_i$. Since
$\tau(M)$ is incrementally-bound we have $\ord{z} \geq \ord{\lambda
  \overline{\eta_1}} = \ord{N_i}$, thus we can abstract the variables $\overline{\eta_1}$ using \rulenamet{abs} which shows that $N_i$ is safe.  Finally
we conclude $\sentail M = \lambda \overline{\xi} . N_1 \ldots N_p :
T$ using the rules \rulenamet{app} and \rulenamet{abs}.  \qed



The assumption that $M$ is closed is necessary. For instance for
$x,y:o$, the computation trees $\tau(\lambda x y .x)$ and
$\tau(\lambda y . x)$ are both incrementally-bound but $\lambda x y
.x$ is safe and $\lambda y . x$ is not.

\subsection*{P-incrementally justified strategy}

We now consider the game-semantic model of the simply-typed lambda
calculus. The strategy denotation of a term-in-context $\Gamma
\stentail M : T$ is written $\sem{\Gamma
\stentail M : T}$. We define the \defname{order} of a move $m$,
written $\ord{m}$, to be the length of the path from $m$ to its
furthest leaf in the arena minus 1. (There are several ways to
define the order of a move; the definition chosen here is sound in
the current setting where each question move in the arena enables at
least one answer move.)
%{\it i.e.}~height of the subarena rooted at $q$ minus 2.

\begin{definition}\rm
  A strategy $\sigma$ is said to be \defname{P-incrementally
    justified} if for any play $s \, q \in \sigma$ where $q$ is a
  P-question, $q$ points to the last unanswered O-question in $\pview{s}$ with
  order strictly greater than $\ord{q}$.
\end{definition}
Note that although the pointer is determined by the P-view, the
choice of the move itself can be based on the whole history of the
play. Thus P-incremental justification does not imply innocence.

The definition suggests an algorithm that, given a play of a
P-incrementally justified denotation, uniquely recovers the pointers
from the underlying sequence of moves and from the pointers
associated to the O-moves therein. Hence:
\begin{lemma}
\label{lem:incrjustified_pointers_uniqu_recover} In P-incrementally
justified strategies, pointers emanating from P-moves are
superfluous.
\end{lemma}

\begin{example}
Copycat strategies, such as the identity strategy $id_A$ on game $A$
or the evaluation map $ev_{A,B}$ of type $(A \Rightarrow B) \times A
\typear B$, are all P-incrementally justified.\footnote{In such
strategies, a P-move $m$ is justified as follows: either $m$ points
to the preceding move in the P-view or the preceding move is of
smaller order and $m$ is justified by the second last O-move in the
P-view.}
\end{example}
%%%% the following example is wrong : ev is P-ij.
%
%\begin{example}
%Take the evaluation map $ev : (o^1 \Rightarrow o^2) \times o^3 \rightarrow o^4$ and the play $s = q^4 q^2 q^1 q^3 \in \sem{ev}$. We have $\ord{q^2} = 1 > \ord{q^1} = \ord{q^3} = 0$. Now $q^3$ points to $q^4$ but $q^2$ is the last unanswered O-question in $\pview{s}= s$ with order $>\ord{q^3}$, hence $\sem{ev}$ is not P-incrementally justified.
%\end{example}



The Correspondence Theorem
% and Lemma \ref{lem:betanf_wellbehavedconst_trav_pview_red}
gives us the following equivalence:
\begin{proposition} % [Incremental-binding vs P-incremental justification]
\label{prop:Nher_incrbound_and_incrjustified} Let $\Gamma \stentail
M : T$ be a $\beta$-normal term. The computation tree $\tau(M)$ is
incrementally-bound if and only if $\sem{\Gamma \stentail M : T}$ is
P-incrementally justified.
\end{proposition}


\parpic[r]{
\pssetcomptree
\raisebox{-12pt}
{$\tree{\lambda^3}{\tree{f^2}{ \tree{\lambda y^1}{\TR{x^0} }}}$}
}
%\noindent \emph{Example:}
\begin{example}
Consider the $\beta$-normal term $\Gamma\stentail f (\lambda y .x) :
o$ where $y:o$ and $\Gamma =f:((o,o),o),~x:o$. The figure on the
right represents its computation tree with the node orders given as
superscripts.  The node $x$ is not incrementally-bound therefore $\tau(f
(\lambda y .x))$ is not incrementally-bound and by Proposition
\ref{prop:Nher_incrbound_and_incrjustified}, $\sem{\Gamma \stentail
f (\lambda y .x) : o}$ is not incrementally-justified (although
$\sem{\Gamma \stentail f : ((o,o),o)}$ and $\sem{\Gamma \stentail
\lambda
  y. x : (o,o)}$ are).
\end{example}
\smallskip

Propositions \ref{prop:safe_imp_incrbound} and
\ref{prop:Nher_incrbound_and_incrjustified} allow us to show the
following:
\begin{theorem}[Safety and P-incremental justification]
\label{thm:safeincrejust} \hfill
\begin{enumerate}[(i)]
\item If $\Gamma \sentail M : T$ then $\sem{\Gamma \sentail M : T}$ is P-incrementally justified.
\item If $\stentail M : T$ is a closed simply-typed term and $\sem{\stentail M : T}$ is P-incrementally justified then the $\beta$-normal form of $M$ is safe.
\end{enumerate}
\end{theorem}
\proof (i) Let $M$ be a safe simply-typed term. By Lemma
\ref{lem:safered_preserve_safety}, its $\beta$-normal form $M'$ is
also safe. By Proposition \ref{prop:safe_imp_incrbound}(i),
$\tau(M')$ is incrementally-bound and by Proposition
\ref{prop:Nher_incrbound_and_incrjustified}, $\sem{M'}$ is an
incrementally-justified. Finally the soundness of the game model
gives $\sem{M} = \sem{M'}$.  (ii) is a consequence of Lemma
\ref{lem:safered_preserve_safety}, Proposition
\ref{prop:Nher_incrbound_and_incrjustified} and
\ref{prop:safe_imp_incrbound}(ii) and soundness of the game model.
\qed



Putting Theorem \ref{thm:safeincrejust}(i) and Lemma
\ref{lem:incrjustified_pointers_uniqu_recover} together gives:
\begin{proposition}
  \label{prop:safe_ptr_recoverable} In the game semantics of safe
  $\lambda$-terms, pointers emanating from P-moves are unnecessary
  {\it i.e.}~they are uniquely recoverable from the underlying sequences of
  moves and from O-moves' pointers.
\end{proposition}

% \begin{example} If justification pointers are omitted, the two
%   Kierstead terms from Example~\ref{ex:kierstead} have the same
%   denotation. In the safe lambda calculus the ambiguity disappears
%   since $M_1$ is safe whereas $M_2$ is not.
% \end{example}

In fact, as the last example highlights, pointers are 
superfluous at order $3$ for safe terms whether from P-moves or O-moves. This is because for
question moves in the first two levels of an arena, the associated
pointers are uniquely recoverable thanks to the visibility
condition. At the third level, the question moves are all P-moves
therefore their associated pointers are uniquely recoverable by
P-incremental justification. This is not true anymore at order $4$:
Take the safe term $\psi:(((o^4,o^3),o^2),o^1) \sentail \psi
(\lambda \varphi . \varphi a) : o^0$ for some constant $a:o$, where
$\varphi:(o,o)$. Its strategy denotation contains plays whose
underlying sequence of moves is $q_0 \, q_1 \, q_2 \, q_3 \, q_2 \,
q_3 \, q_4$. Since $q_4$ is an O-move, it is not constrained by
P-incremental justification and thus it can point to any of the two
occurrences of $q_3$.\footnote{More generally, a P-incrementally
justified strategy can contain plays that are not ``O-incrementally
justified'' since it must take into account any possible strategy
incarnating its context, including those that are not
P-incrementally justified. For instance in the given example, there
is one version of the play that is not O-incrementally justified
(the one where $q_4$ points to the first occurrence of $q_3$). This
play is involved in the strategy composition $\sem{ \stentail M_2 :
(((o,o),o),o)} ; \sem{ \psi:(((o,o),o),o) \stentail \psi (\lambda
\varphi . \varphi a):o}$ where $M_2$ denotes the unsafe Kierstead
term.}


\subsection*{Towards a fully abstract game model}\hfill

The standard game models which have been shown to be fully abstract
for PCF \cite{abramsky94full,hylandong_pcf} are of course also fully
abstract for the restricted language safe PCF. One may ask, however,
whether there exists a fully abstract model with respect to safe
context only.

Such model may be obtain by considering P-incrementally justified strategies
- which have been shown to compose in \cite{Blumphd}. Its is reasonable to think that
 O-moves also needs to be constrained by the symmetrical O-incremental justification, which corresponds to the requirement that contexts are safe. This line of work is still in progress.


\subsection*{Safe PCF and safe Idealised Algol}

\pcf\ is the simply-typed lambda calculus augmented with basic
arithmetic operators, if-then-else branching and a family of
recursion combinator $Y_A : ((A,A),A)$ for any type $A$.  We define
\emph{safe} \pcf\ to be \pcf\ where the application and abstraction
rules are constrained in the same way as the safe lambda calculus.
This language inherits the good properties of the safe lambda
calculus: No variable capture occurs when performing substitution
and safety is preserved by the reduction rules of the small-step
semantics of \pcf.

\subsubsection{Correspondence}

The computation tree of a \pcf\ term is defined as the least
upper-bound of the chain of computation trees of its \emph{syntactic
approximants} \cite{abramsky:game-semantics-tutorial}.  It is
obtained by infinitely expanding the $Y$ combinator, for instance
$\tau(Y (\lambda f x. f x))$ is the tree representation of the
$\eta$-long form of the infinite term $(\lambda f x. f x)
 ((\lambda f x. f x) ((\lambda f x. f x) ( \ldots$

It is straightforward to define the traversal rules modeling the
arithmetic constants of \pcf. Just as in the safe lambda calculus we
had to remove @-nodes in order to reveal the game-semantic
correspondence, in safe \pcf\ it is necessary to filter out the
constant nodes from the traversals. The Correspondence Theorem for
\pcf\ says that the interaction game semantics is isomorphic to the
set of traversals disposed of these superfluous nodes. This can
easily be shown for term approximants. It is then lifted to full
\pcf\ using the continuity of the function $\travset(\_)^{\filter
\theroot}$ from the set of computation trees (ordered by the
approximation ordering) to the set of sets of justified sequences of
nodes (ordered by subset inclusion). Finally computation trees of
safe \pcf\ terms are incrementally-bound thus we have
%Computation trees of safe \pcf\ terms are incrementally-bound.
%Moreover since \pcf\ constant are of order $1$ at most, the constant
%traversal rules are all \emph{well-behaved} (Lemma
%\ref{lem:sigma_order1_are_wellbehaved}) hence Lemma
%\ref{lem:betanf_wellbehavedconst_trav_pview_red} (from the Appendix)
%still holds and the game-semantic analysis of safety remains valid
%for \pcf. Hence we have:
\begin{theorem}
\label{thm:safepcfpincr} Safe PCF terms have P-incrementally
justified denotations. \qed
\end{theorem}


Similarly, we can define safe \ialgol\ to be safe \pcf\ augmented
with the imperative features of Idealized Algol (\ialgol\ for short)
\cite{Reynolds81}.  Adapting the game-semantic correspondence and
safety characterization to \ialgol\ seems feasible although the
presence of the base type \iavar, whose game arena $\iacom^{\nat}
\times \iaexp$ has infinitely many initial moves, causes a mismatch
between the simple tree representation of the term and its game
arena. It may be possible to overcome this problem by replacing the
notion of computation tree by a ``computation directed acyclic
graph''.

The possibility of representing plays \emph{without some or all of
  their pointers} under the safety assumption suggests potential
applications in algorithmic game semantics. Ghica and McCusker
\cite{ghicamccusker00} were the first to observe that pointers are
unnecessary for representing plays in the game semantics of the
second-order finitary fragment of Idealized Algol ($\ialgol_2$ for
short). Consequently observational equivalence for this fragment can
be reduced to the problem of equivalence of regular expressions.  At
order $3$, although pointers are necessary, deciding observational
equivalence of $\ialgol_3$ is EXPTIME-complete
\cite{DBLP:journals/apal/Ong04,DBLP:conf/fossacs/MurawskiW05}.
Restricting the problem to the safe fragment of $\ialgol_3$ may lead
to a lower complexity.

% (note that it is unlikely to obtain the complexity PSPACE because the
% set of complete plays of the safe term $\lambda f^{(o,o),o} . f
% (\lambda x^o . x)$ is not regular \cite{DBLP:journals/apal/Ong04}).

% Murawski showed the undecidability of program equivalence in
% $\ialgol_i$ for $i\geq4$ by encoding Turing machine computations
% into a finitary $IA_4$ term \cite{murawski03program}. The term
% constructed being not safe, the proof cannot be transposed to the
% safe fragments. Hence the question remains of whether observational
% equivalence is decidable for the \emph{safe} fragments of these
% language.

%In \cite{Ong02}, one of us showed that observational equivalence for
% finitary second-order \ialgol\ with recursion ($\ialgol_2 + Y_1$) is
% undecidable. The proof consists in reducing the Queue-Halting
% problem to the observational equivalence of two $\ialgol_2 + Y_1$
% terms. The same reduction is still valid in the safe fragment of
% $\ialgol_2 + Y_1$.  Consequently, observational equivalence of safe
% $\ialgol_2 + Y_1$ is also undecidable.


    
\section{Further work and open problems}

The safe lambda calculus is still not well understood. Many basic
questions remain. What is a (categorical) model of the safe lambda
calculus? Does the calculus have interesting models?  What kind of
reasoning principles does the safe lambda calculus support, via the
Curry-Howard Isomorphism? Does the safe lambda calculus characterize
a complexity class, in the same way that the simply-typed lambda
calculus characterizes the polytime-computable numeric functions
\cite{DBLP:conf/tlca/LeivantM93}?  Is the addition of unsafe
contexts to safe ones conservative with respect to observational (or
contextual) equivalence?
%Can we obtain a fully abstract model of safe PCF by suitably
%constraining O-moves ({\it i.e.}~``O-incremental justification'')?

With a view to algorithmic game semantics and its applications, it
would be interesting to identify sublanguages of Idealised Algol
whose game semantics enjoy the property that pointers in a play are
uniquely recoverable from the underlying sequence of moves. We name
this class PUR. $\ialgol_2$ is the paradigmatic example of a
PUR-language. Another example is \emph{Serially Re-entrant Idealized
  Algol} \cite{abramsky:mchecking_ia}, a version of \ialgol\ where
multiple uses of arguments are allowed only if they do not ``overlap
in time''.  We believe that a PUR language can be obtained by
imposing the \emph{safety condition} on $\ialgol_3$. Murawski
\cite{Murawski2003} has shown that observational equivalence for
$\ialgol_4$ is undecidable; is observational equivalence for
\emph{safe} $\ialgol_4$ decidable?


    \bibliographystyle{abbrv} % {nat} % {splncs}
    \bibliography{../bib/dphil-all}

    \section*{Appendix -- Computation tree, traversals and correspondence}
    \label{sec:correspondence}
    In \cite{OngLics2006}, one of us introduced the notion of
computation tree and traversals over a computation tree for the
purpose of studying trees generated by higher-order recursion
scheme. Here we extend these concepts to the pure (\ie without
constants) simply-typed lambda calculus. Our setting allows the
presence of free variable of any order. Moreover the term studied is
not necessarily of ground type. (This contrasts with
\cite{OngLics2006}'s setting where the term is of ground type and
contains only \emph{uninterpreted constant}\footnote{A constant $f$
is  \emph{uninterpreted} if the small-step semantics of the language
  does not contain any rule of the form $f \dots \rightarrow e$. $f$
  can be regarded as a data constructor.} of order 1 at most and no
free variables.) Note that our setting automatically accounts for
the presence of uninterpreted constants since they can be regarded
as free variables. We will then state the \emph{Correspondence
Theorem} (Theorem \ref{thm:correspondence}) that was used in Sec.
\ref{sec:gamesemaccount}.

In the following we fix a simply typed term $\Gamma \vdash M :T$ and
we consider its computation tree $\tau(M)$ as defined in Def.\
\ref{dfn:comptree}.

\subsection{Notations}
We first fix some notations. We write $\theroot$ to denote the root
of the computation tree $\tau(M)$. The set of nodes of the
computation tree is denoted by $N$. The sets $N_@$, $N_\lambda$ and
$N_{\sf var}$ are respectively the subset of @-nodes,
$\lambda$-nodes and variable nodes. For $n \in N$ we will write
$\kappa(n)$ to denote the subterm of $\elnf{M}$ corresponding to the
subtree of $\tau(M)$ rooted at $n$. In particular $\kappa(\theroot)
= \elnf{M}$. The \defname{type} of a variable-labelled node is the
type of the variable which labels it; the type of the root is
$(A_1,\ldots,A_p, T)$ where $x_1:A_1,\ldots, x_p:A_p$ are the
variables in the context $\Gamma$; and the type of a node $n\in
(N_\lambda \union N_@) \setminus \{ \theroot \}$ is the type of the
corresponding subterm $\kappa(n)$.


\subsection{Pointers and justified sequences of nodes}

We define the \defname{enabling relation} on the set of nodes of the
computation tree as follows: $m$ enables $n$, written $m \vdash n$,
if and only if $n$ is bound by $m$ (and we sometimes write $m
\vdash_i n$ to precise that $n$ is the $i^{\sf th}$ variable bound
by $m$); or $m$ is the root $\theroot$ and $n$ is a free variable;
or $n$ is a $\lambda$-node and $m$ is its parent node.


We say that a node $n_0$ of the computation tree is
\defname{hereditarily enabled} by $n_p \in N$ if there are nodes
$n_1,\ldots, n_{p-1} \in N$ such that $n_{i+1}$ enables $n_{i}$ for
all $i\in 0..p-1$.

For any set of nodes $S, H \subseteq N$ we write $S^{H\vdash}$ for
$S \inter \vdash^*(H) = \{ n \in S \ | \exists n_0 \in H \mbox{ s.t.
}n_0  \vdash^* n \}$ -- the subset of $S$ constituted of nodes
hereditarily enabled by some node in $H$. We will abbreviate
$S^{\{n_0\}\vdash}$ into $S^{n_0\vdash}$.

We call \defname{input-variables nodes} the elements of $V_{\sf
var}^{\theroot\vdash}$ \ie variables that are hereditarily enabled
by the root of $\tau(M)$. Thus we have $V_{\sf var}^{\theroot\vdash}
= V \setminus ( V_{\sf var}^{N_@\vdash} \union V_{\sf
var}^{N_\Sigma\vdash})$.

A \defname{justified sequence of nodes} is a sequence of nodes with
pointers such that each occurrence of a variable or $\lambda$-node
$n$ different from the root has a pointer to some preceding
occurrence $m$ verifying $m \vdash n$. In particular, occurrences of
@-nodes do not have pointer. We represent the pointer in the
sequence as follows \Pstr[0.4cm]{ (m){m} \ldots (n-m,45:i) n }.
 where the label indicates that either $n$ is labelled with the $i$th variable
abstracted by the $\lambda$-node $m$ or that $n$ is the $i^{\sf th}$
child of $m$.  Children nodes are numbered from $1$ onward except for
@-nodes where it starts from $0$. Abstracted variables are numbered
from $1$ onward. The $i^{\sf th}$ child of $n$ is denoted by $n.i$.

We say that a node $n_0$ of a justified sequence is
\defname{hereditarily justified} by $n_p$ if there are occurrences $n_1,
\ldots, n_{p-1}$ in the sequence such that $n_i$ points to $n_{i+1}$
for all $i\in 0..p-1$. Let $n$ be an occurrence in a justified
sequence $s$. We write $s \filter r$ to denote the subsequence of
$s$ consisting of the occurrences hereditarily justified by $n$.


The notion of \defname{P-view} $\pview{t}$ of a justified sequence of
nodes $t$ is defined the same way as the P-view of a justified
sequences of moves in Game Semantics:\footnote{ The equalities in the
  definition determine pointers implicitly. For instance in the second
  clause, if in the left-hand side, $n$ points to some node in $s$
  that is also present in $\pview{s}$ then in the right-hand side, $n$
  points to that occurrence of the node in $\pview{s}$.}
$$\begin{array}{rclrcl}
\pview{\epsilon} &=&  \epsilon
& \pview{\Pstr[0.5cm]{ s \cdot (m) m \cdot \ldots \cdot (lmd-m,40){\lambda\overline{\xi}}
}}
 &=& \Pstr{
\pview{s} \cdot (m2) m \cdot (lm2-m2,50) {\lambda \overline{\xi}} } \\
\mbox{for $n \notin N_\lambda$, } \pview{s \cdot n }  &=&  \pview{s} \cdot n \qquad
& \pview{s \cdot \theroot }  &=&  \theroot
\end{array}$$

The O-view of $s$, written $\oview{s}$, is defined dually. We will
borrow the game semantic terminology: A justified sequences of nodes
satisfies \defname{alternation} if for any two consecutive nodes one
is a $\lambda$-node and the other is not, and \defname{P-visibility}
if every variable node points to a node occurring in the P-view a
that point.

\subsection{Computation tree with value-leaves}


We now add another ingredient to the computation tree that was not
originally used in \cite{OngLics2006}.  We write $\mathcal{D}$ to
denote the set of values of the base type $o$.  We add
\defname{value-leaves} to $\tau(M)$ as follows: For each value $v
\in \mathcal{D}$ and for each node $n \in N$ we attach the child
leaf $v_n$ to $n$.  We write $V$ for the set of vertices of the
resulting tree (\ie inner nodes and leaf nodes). For $\$$ ranging in
$\{@, \lambda, var \}$, we write $V_\$$ to denote the set of inner
nodes from $N_\$$ plus the leaf-nodes whose parent is in $N_\$$ \ie
$V_\$ = N_\$ \union \{ v_n \ | \ n \in N_\$, v \in \mathcal{D} \}$.


Everything that we have defined can be lifted to this new version of
computation tree. A value-leaf has order $0$. The enabling relation
$\vdash$ is extended so that every leaf is enabled by its parent
node. A link going from a value-leaf $v_n$ to a node $n$ is labelled
by $v$: \Pstr[0.4cm]{ (n) n \ldots (vn-n,35:v){v_n} }. For the
definition of P-view and visibility, value-leaves are treated as
$\lambda$-nodes if they are at an odd level in the computation tree,
and as variable nodes if they are at an even level.

We say that an occurrence of an inner node $n \in N$ is
\defname{answered} by an occurrence $v_n$ if $v_n$ in
the sequence that points to $n$, otherwise we say that $n$ is
\defname{unanswered}. The last unanswered node is called the
\defname{pending node}.  A justified sequence of nodes is
\defname{well-bracketed} if each value-leaf occurring in it is justified by the pending node at that point.  If $t$ is a traversal then we write
$?(t)$ to denote the subsequence of $t$ consisting only of
unanswered nodes.

\subsection{Traversals of the computation tree}
\label{subsec:traversal}

A \emph{traversal} is a justified sequence of nodes of the computation tree where each node
indicates a step that is taken during the evaluation of the term.

\begin{definition}[Traversals for simply-typed $\lambda$-terms] \rm
\label{def:traversal} The set $\travset(M)$ of \defname{traversals}
over $\tau(M)$ is defined by induction over the rules of Table
\ref{tab:trav_rules}. A traversal that cannot be extended by any
rule is said to be \emph{maximal}.
\end{definition}
\begin{FramedTable}
\noindent {\bf Initialization rules}
\begin{itemize}[]
\item\rulenamet{Empty} $\epsilon \in \travset(M)$.
\item\rulenamet{Root} The sequence constituted of a single occurrence of $\tau(M)$'s root is a traversal.
\end{itemize}

\noindent {\bf Structural rules}
\begin{itemize}[]
    \item \rulenamet{Lam} If $t \cdot \lambda \overline{\xi}$ is a traversal then so is
        $t \cdot \lambda \overline{\xi} \cdot n$ where $n$
        denotes $\lambda \overline{\xi}$'s child and:
        \begin{compactitem}
            \item If $n \in N_@ \union N_\Sigma$ then it has no justifier;
            \item if  $n \in N_{\sf var}\setminus N_{\sf fv}$ then it points to the only occurrence\footnote{Prop.\ \ref{prop:pviewtrav_is_path} will show that P-views
            are paths in the tree thus $n$'s enabler occurs
            exactly once in the P-view.} of its enabler in
            $\pview{t\cdot \lambda \overline{\xi}}$;
            \item if  $n \in N_{\sf fv}$ then it points
            to the only occurrence of the root $\theroot$ in
            $\pview{t \cdot \lambda \overline{\xi}}$.
        \end{compactitem}
    \item \rulenamet{App} If $t \cdot @$ is a traversal then so is \Pstr[0.4cm]{t \cdot (m) @  \cdot (n-m,40:0) n}.
\end{itemize}

\emph{\bf Input-variable rules}
\begin{itemize}[]
\item \rulenamet{InputVar} If $t$ is a traversal where $t^\omega \in N_{\sf var}^{\theroot\vdash} \union L_\lambda^{\theroot\vdash}$
and $x$ is an occurrence of a variable node in $\oview{t}$ then
so is $t \cdot n$ for any child $\lambda$-node $n$ of $x$, $n$
pointing to $x$.



\item \rulenamet{InputValue} If $t_1
\cdot x \cdot t_2$ is a traversal with pending node $x \in
N_{\sf var}^{\theroot\vdash}$ then so is \Pstr[0.5cm]{t_1 \cdot
(x){x} \cdot t_2 \cdot (xv-x,38:v){v_x} } for all $v \in
\mathcal{D}$.
\end{itemize}

\emph{\bf Copy-cat rules}
\begin{itemize}[]
\item\rulenamet{Var}
If \Pstr[0.5cm]{t \cdot (n){n} \cdot (lx){\lambda \overline{x}}
    \ldots (x-lx,50:i){x_i} } is a traversal where $x_i \in
    N_{\sf var}^{@\vdash}$ then so is \Pstr[0.5cm]{ t \cdot
(n){n} \cdot (lx){\lambda \overline{x}}  \ldots (x-lx,30:i){x_i}
    \cdot (letai-n,40:i){\lambda \overline{\eta_i}}
     }.

\item\rulenamet{Value}
  If \Pstr{t \cdot (m){m} \cdot (n){n}  \ldots
(vn-n,60:v){v}_{n} } is a traversal where $n\in N$ then so is
\Pstr[0.6cm]{t \cdot (m){m} \cdot (n){n} \ldots
(vn-n,60:v){v}_{n} \cdot (vm-m,45:v){v}_m}.
\end{itemize}
\caption[Traversal rules for the simply-typed
lambda-calculus]{Traversal rules for the simply-typed
$\lambda$-calculus.} \label{tab:trav_rules}
\end{FramedTable}

\parpic[r]{
 $\pssetcomptree\tree[levelsep=3ex,treesep=0.5cm]{\lambda} {
    \tree{@}{
        \pstree[linestyle=dotted]{\TR{\lambda y}\arclabel{0} }{
            \tree{y}{
                \tree{\lambda \overline{\eta_1}}{\vdots}%\arclabel{1}
                \tree{\lambda \overline{\eta_i}}{\vdots}%\arclabel{i}
                \tree{\lambda \overline{\eta_n}}{\vdots}%\arclabel{n}
            }
        }
        \pstree[linestyle=dotted]{\TR{\lambda \overline{x}}
            \arclabel{1}}{ \tree{x_i}{\TR{} \TR{}}}
}}$ } A traversal always starts by visiting the root. Then it mainly
follows the structure of the tree. The (Var) rule permits to jump
across the computation tree. The idea is that after visiting a
variable node $x$, a jump is allowed to the node corresponding to
the subterm that would be substituted for $x$ if all the
$\beta$-redexes occurring in the term were reduced. The sequence
\Pstr[0.8cm]{\lambda \cdot (app) @  \cdot (ly) {\lambda y}  \ldots
(y-ly,35:1) y  \cdot (lx-app,38:1) {\lambda \overline{x}} \ldots
(x-lx,30:i) {x_i} \cdot (leta-y,40:i) {\lambda \overline{\eta_i} }
\ldots}
 is an example of traversal of the computation tree shown on the right.

\begin{proposition}[counterpart of proposition 6 from \cite{OngHoMchecking2006}]
\label{prop:pviewtrav_is_path}
Let $t$ be a traversal. Then:
\begin{enumerate}[(i)]
\item $t$ is a well-defined and well-bracketed justified sequence;
\item $t$ is a well-defined justified sequence verifying alternation, P-visibility and O-visibility;
\item If $t$'s last node is not a value-leaf, then $\pview{t}$ is the path in the computation tree going from the root to $t$'s last node.
\end{enumerate}
\end{proposition}

The \defname{reduction} of a traversal $t$ is the subsequence of $t$
obtained by keeping only occurrences of nodes that are hereditarily
enabled by the root $\theroot$. This has the effect of eliminating
the ``internal nodes'' of the computation. If $t$ is a non-empty
traversal then the root $\theroot$ occurs exactly once in $t$ and
the reduction of $t$ is equal to $t \filter r$. We write
$\travset(M)^{\filter \theroot}$ for the set or reduction of
traversals of $M$.

Application nodes are used to connect the operator and the operand
of an application in the computation tree but since they do not play
any role in the computation of the term, we can remove them from the
traversals.  We write $t-@$ for the sequence of nodes-with-pointers
obtained by removing from $t$ all @-nodes and value-leaves of
@-nodes, any link pointing to an @-node being replaced by a link
pointing to the immediate predecessor of @ in $t$. We write
$\travset(M)^{-@}$ for the set $\{ t - @ \ | \  t \in \travset(M)
\}$.
\begin{remark}
Clearly if $M$ is $\beta$-normal then $\tau$ does not contain any
@-node therefore all nodes are hereditarily justified by $r$ and we
have $\travset(M)^{-@} = \travset(M) = \travset(M)^{\filter \theroot
}$.
\end{remark}



\begin{lemma}
% version of the lemma for the pure simply-typed lambda calculus
\label{lem:betanf_trav_pview_red} Suppose that $M$ is a
$\beta$-normal simply-typed term . Let $t$ be a non-empty traversal
of $M$ and $r$ denote the only occurrence of $\tau(M)$'s root in
$t$. If $t$'s last occurrence is not a leaf then
$$ \pview{t} \filter r = \pview{?(t) \filter  r }\ .$$
\end{lemma}
In the lambda calculus without interpreted constants this lemma
follows immediately from the fact that $\travset(M) =
\travset(M)^{\filter \theroot}$. But this Lemma remains valid in the
presence of interpreted constants provided that the traversal rules
implementing the constants are \emph{well-behaved}\footnote{A
traversal rule is \defname{well-behaved} if it can be stated under
the form ``$t = t_1\cdot n \cdot t_2 \in \travset \zand ?(t) =
?(t_1) \cdot n \zand n \in N_{\Sigma}\union N_{\sf var} \zand P(t)
\zand m\in S(t) \imp\ $\Pstr{ t_1\cdot (n){n} \cdot t_2 \cdot
(m-n,25){m} \in \travset}'' for some expression $P$ expressing a
condition on $t$ and function $S$ mapping traversals of the form of
$t$ to a subset of the children of $n$.}.

\subsection{Computation trees and arenas}
We consider the well-bracketed game model of the simply-typed lambda
calculus.  We choose to represent strategies using ``prefix-closed
set of plays''. \footnote{In the literature, a strategy is commonly
defined as a set of plays closed by taking a prefix of \emph{even}
length. However for the purpose of showing the correspondence with
traversals, the ``prefix-closed''-based definition is more
adequate.} We fix a term $\Gamma \vdash M : T$ and write
$\sem{\Gamma \vdash M : T}$ for its strategy denotation. The answer
moves of a question $q$ are written $v_q$ where $v$ ranges in
$\mathcal{D}$.

\begin{proposition}
There exists a function $\varphi_M$, constructible from $\tau(M)$,
that maps nodes from $V\setminus (V_@ \union V_\Sigma)$ to moves of
the interaction arena underlying the revealed strategy
$\syntrevsem{\Gamma \vdash M : T}$ and such that
\begin{itemize}
\item $\varphi$ maps $\lambda$-nodes to O-questions, variable
nodes to P-questions, value-leaves of $\lambda$-nodes to
P-answers and value-leaves of variable nodes to O-answers.

\item $\varphi$ maps nodes of a given order to moves of
the same order.
\end{itemize}
\end{proposition}
If $t = t_0 t_1 \ldots$ is a justified sequence of nodes in
$V_\lambda \union V_{\sf var}$ then $\varphi(t)$ is defined to
be the sequence of moves $\varphi(t_0)\ \varphi(t_1) \ldots$
equipped with the pointers of $t$.


\begin{example}
Take $\lambda x . (\lambda g . g x) (\lambda y . y)$ with $x,y:o$ and $g:(o,o)$.
The diagram below represents the computation tree (middle), the arenas
$\sem{(o,o), o}$ (left), $\sem{o , o}$ (right), $\sem{o\rightarrow o}$ (rightmost)
and $\varphi = \psi \union \psi_{\lambda g.g x}^{\lambda g, q_{\lambda g}} \union
\psi_{\lambda y.y}^{\lambda y, q_{\lambda y}}$
(dashed-lines).
$$\psset{levelsep=3.5ex}
\pstree{\TR[name=root]{\lambda x}}
{
    \pstree{\TR[name=App]{@}}
    {
            \pstree{\TR[name=lg]{\lambda g}}
                { \pstree{\TR[name=lgg]{g}}{
                        \pstree{\TR[name=lgg1]{\lambda}}
                        { \TR[name=lgg1x]{x}  } } }
            \pstree{\TR[name=ly]{\lambda y}}
                    {\TR[name=lyy]{y}}
    }
}
\rput(4.5cm,-1cm){
  \pstree{\TR[name=A1lx]{q_{\lambda x}}}
        { \TR[name=A1x]{q_x} }
}
\rput(-6cm,-1.5cm){
    \pstree{\TR[name=A2lg]{q_{\lambda g}}}
    {
        \pstree{\TR[name=A2g]{q_g}}
        {  \TR[name=A2g1]{q_{g_1}}   }
    }}
\rput(2.5cm,-1.5cm){
    \pstree{\TR[name=A3ly]{q_{\lambda y}}}
        { \TR[name=A3y]{q_y}
        }
}
\psset{nodesep=1pt,arrows=->,arcangle=-20,arrowsize=2pt 1,linestyle=dashed,linewidth=0.3pt}
\ncline{->}{root}{A1lx} \mput*{\psi}
\ncarc{->}{lgg1x}{A1x}
\ncline{->}{lg}{A2lg} \mput*{\psi_{\lambda g.g x}^{\lambda g, q_{\lambda g}}}
\ncline{->}{lgg}{A2g}
\ncline{->}{lgg1}{A2g1}
\ncline{->}{ly}{A3ly} \mput*{\psi_{\lambda y.y}^{\lambda y, q_{\lambda y}}}
\ncline{->}{lyy}{A3y}
$$
\end{example}


\subsection{The Correspondence Theorem}

In game semantics, strategy composition is performed using a
CSP-like ``composition + hiding''. If the internal moves are not
hidden then we obtain an alternative semantics called
\defname{revealed semantics} in \cite{willgreenlandthesis} and
\emph{interaction} semantics in \cite{DBLP:conf/sas/DimovskiGL05}.
The fully revealed semantics of a term $\Gamma \vdash M :T$, written
$\revsem{\Gamma \vdash M : T}$, is obtained by
uncovering\footnote{An algorithm that uniquely recovers hidden moves
is given in Part II of
  \cite{hylandong_pcf}.}  the internal moves from $\sem{\Gamma \vdash
  M : T}$ that are generated by the composition with the evaluation map
$ev$ at each @-node of the computation tree.  The inverse operation
consists in filtering out the internal moves.

In the simply-typed lambda calculus, the set $\travset(M)$ of
traversals of the computation tree is isomorphic to the set of
uncovered plays of the strategy denotation. Moreover the set of
traversal reductions is isomorphic to the standard strategy
denotation.

\begin{theorem}[The Correspondence Theorem]
\label{thm:correspondence} $\varphi_M$ gives us the following two
isomorphisms:
\begin{eqnarray*}
(i)~\varphi_M  &: \travset(M)^{-@} \stackrel{\cong}{\longrightarrow} \syntrevsem{\Gamma \vdash M :T} \\
(ii)~\varphi_M  &: \travset(M )^{\filter \theroot} \stackrel{\cong}{\longrightarrow} \sem{\Gamma \vdash M : T} \ .
\end{eqnarray*}
\end{theorem}

\begin{example}
Take $M = \lambda f z . (\lambda g x . f x) (\lambda y. y) (f z) :
((o,o),o, o)$.  The figure below represents the computation tree
(left tree), the arena $\sem{((o,o),o, o)}$ (right tree) and
$\psi_M$ (dashed line). (Only question moves are shown for clarity.)
The justified sequence of nodes $t$ defined hereunder is an example
of traversal:

\begin{tabular*}{0.9\textwidth}{@{\extracolsep{\fill}}p{6cm}p{7cm}}
$\pssetcomptree\pstree[levelsep=2.5ex,treesep=0.3cm]{ \TR[name=root]{\lambda f z} }
     {  \tree[levelsep=4ex]{@}
        {   \tree{\lambda g x}{
                  \pstree{\TR[name=f]{f^{[1]}}}{
                            \pstree{\TR[name=lmd]{\lambda^{[2]}}}
                                {\TR{x}}
                  }
                }
            \tree{\lambda y }{\TR{y}}
            \tree{\lambda ^{[3]}}{
                \pstree{\TR[name=f2]{f^{[4]}}} {
                \pstree{\TR[name=lmd2]{\lambda^{[5]}}}{\TR[name=z]{z}}
                }
            }
        }
     }
\hspace{2cm}
  \pstree[levelsep=8ex, treesep=0.3cm]{ \TR[name=q0]{q^0} }
    {   \pstree[levelsep=4ex]{\TR[name=q1]{q^1}} {\TR[name=q2]{q^2}}
        \TR[name=q3]{q^3}
    }
\psset{nodesep=1pt,arrows=->,arrowsize=2pt 1,linestyle=dashed,linewidth=0.3pt}
\ncline{->}{root}{q0} \mput*{\psi_M}
\ncarc[arcangle=-25]{->}{z}{q3}
\ncarc[arcangle=10]{->}{f}{q1}
\ncarc[arcangle=10]{->}{lmd}{q2}
\ncline{->}{f2}{q1}
\ncline{->}{lmd2}{q2}$
&
\begin{asparablank}
  \item  \Pstr[0.8cm]{
t = (n){\lambda f z} \
(n2){@} \
(n3-n2,60){\lambda g x} \
(n4-n,45){f^{[1]}} \
(n5-n4,45){\lambda^{[2]}} \
(n6-n3,45){x} \
(n7-n2,35){\lambda^{[3]}} \
(n8-n,35){f^{[4]}} \
(n9-n8,45){\lambda^{[5]}} \
(n10-n,35){z}
}

\item \Pstr[0.9cm]{
t\filter r = (n){\lambda f z} \ (n4-n,50){f}^{[1]} \
(n5-n4,60){\lambda}^{[2]} \ (n8-n,45){f}^{[4]} \
(n9-n8,60){\lambda}^{[5]} \ (n10-n,40){z}}
\item
\Pstr[0.8cm]{ {\varphi_M(t\filter r) =\ } (n){q^0}\ (n4-n,60){q^1}\
(n5-n4,60){q^2}\ (n8-n,45){q^1}\ (n9-n8,60){q^2}\ (n10-n,38){q^3}
\in \sem{M}\ .}
\end{asparablank}
\end{tabular*}
\end{example}


\end{document}
