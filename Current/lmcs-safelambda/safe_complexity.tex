\newcommand\bigo{\mathcal{O}} % big O notation
\newcommand\booltype{\mathsf{B}}
\newcommand\towerexp[2]{\exp_{#1}(#2)}
%\newcommand\towerexp[2]{2_{#1}(#2)}

\section{Complexity of the safe lambda calculus}
This section is concerned with the complexity of the
beta-eta equivalence problem for the safe lambda calculus:
Given two safe lambda-terms, are they equivalent up to $\beta\eta$-conversion?

\subsection{Statman's result}

Let $\towerexp{h}{m}$ denote the tower-of-exponential function
defined by induction as $\towerexp{0}{m} = m$ and $\towerexp{h+1}{m} =
2^{\towerexp{h}{m}}$.
A program is \defname{elementary recursive} if its run-time can be bounded by
$\towerexp{K}{n}$ for some constant $K$ where $n$ is the length of
the input.

We recall the definition of finite type
theory. We define $\mathcal{D}_0 =
\{\mathbf{true},\mathbf{false}\}$ and $\mathcal{D}_{k+1}
=\mathscr{P}(\mathcal{D}_k)$ (\ie, the powerset of $\mathcal{D}_k$).
For $k\geq0$, we write $x^k$, $y^k$ and
$z^k$ to denote variables ranging over $\mathcal{D}_k$. Prime
formulae are $x^0$, $\mathbf{true}\in y^1$, $\mathbf{false}\in y^1$,
and  $x^k \in y^{k+1}$. Formulae are built up from prime formulae
using the logical connectives $\zand$,$\zor$,$\rightarrow$,$\neg$
and the quantifiers $\forall$ and $\exists$. Meyer showed that
deciding the validity of such formulae requires nonelementary time
\cite{Meyer1974}.


A famous result by Statman  states that deciding the
$\beta\eta$-equality of two first-order typable lambda-terms is not
elementary recursive \cite{Statman:1979:TLE}. The proof proceeds by
encoding the Henkin quantifier elimination of type theory in the
simply-typed lambda calculus and by appealing to Meyer's result \cite{Meyer1974}. Simpler proofs have subsequently been given: one by Mairson \cite{mairson1992spt} and another by Loader
\cite{Loader1998}. Both proceed by encoding the Henkin quantifier
elimination procedure in the lambda calculus, as in the original
proof, but their use of list iteration to implement quantifier
elimination makes them much easier to understand.

It turns out that all these encodings rely on unsafe terms:
Statman's encoding uses the conditional function $\sf sg$ which is
not definable in the safe lambda calculus
\cite{blumong:safelambdacalculus}; Mairson's encoding uses unsafe
terms to encode both quantifier elimination and set membership, and
Loader's encoding uses unsafe terms to build list iterators. We are
thus led to conjecture that finite type theory (see definition in
Sec.~\ref{sec:mairsonenc}) is intrinsically unsafe in the sense
that every encoding of it in the lambda calculus is necessarily
unsafe. Of course this conjecture does not rule out the possibility
that another non-elementary problem is encodable in the safe lambda
calculus.


\subsection{Mairson's encoding}
\label{sec:mairsonenc}
We refer the reader to Mairson's original paper \cite{mairson1992spt} for a detailed account of his encoding. We show here why Mairson's encoding does not work in the safe lambda calculus. We then introduce a variation that eliminates some of the unsafety. Although the resulting encoding does not suffice to interpret type theory in the safe lambda calculus, it enables another interesting encoding: that of the True Quantifier Boolean Formula (TQBF) problem. This implies that deciding beta-eta equality of safe terms is PSPACE-hard.

\subsubsection{Sources of unsafety}
In Mairson's encoding, boolean values are encoded by terms of type
$\booltype = \sigma \typear \sigma\typear \sigma$ for some type
$\sigma$, and variables of order $k \geq 0$ are encoded by terms of
type $\Delta_k$ defined as $\Delta_0 \equiv \booltype$ and
$\Delta_{k+1} \equiv (\Delta_k \typear \tau \typear \tau)\typear \tau \typear
\tau$ for some type $\tau$. Using this encoding, unsafety manifests
itself in three different places:
\begin{enumerate}[(i)]
  \item \emph{Set membership:} The prime formula ``$x^k \in y^{k+1}$'' is encoded by a term-in-context of the form
      \begin{equation} x: \Delta_k, y:\Delta_{k+1} \stentail y (\lambda z^{\Delta_k} . M(x,z))\,F : \Delta_k \typear \Delta_{k+1} \typear \Delta_0 \label{eqn:setmembership}\end{equation}
for some term $F$ and term $M(x,z)$ containing free occurrences of $x$ and $z$.
This is unsafe because the free occurrence of $x$ in $M(x,z)$ is not abstracted together with $z$.

\item \emph{Quantifier elimination} is implemented using a list iterator $\mathbf{D}_{k+1}$ of type $\Delta_{k+2}$ which acts like the {\tt foldr} function (from functional programming) over the list of all elements of $\mathcal{D}_k$.
Nested quantifiers in the formula are encoded by nested list iterations; this can be source of unsafety, for instance the formula ``$\forall x^0 . \exists y^0 . x^0 \zor y^0$'' is encoded as $$\stentail \mathbf{D}_0 (\lambda
x^{\Delta_0}. AND (\mathbf{D}_0 (\lambda y^{\Delta_0}. OR
(\underline{x} \zor y)) F))\ T : \booltype$$
for some terms $AND$, $OR$, $F$ and $T$ and where the type
$\tau$ is instantiated as $\booltype$. This term is unsafe due to
the underlined occurrence which is unsafely bound.

More generally, nested binding will be encoded safely if and only if every variable $x$ in the formula is bound by the first quantifier $\exists z$ or
$\forall z$ satisfying $\ord{z} \geq \ord{x}$ in the path to the root of the formula AST. So for example if set-membership
were safely encodable then the interpretation of ``$\forall x^k
. \exists y^{k+1} . x^k \in y^{k+1}$'' would be unsafe whereas that of ``$\forall y^{k+1} . \exists x^k . x^k \in y^{k+1}$''
would be safe.

\item \emph{Elements of the type hierarchy.} The base set $\mathcal{D}_0$ of booleans is represented by a safe term $\mathbf{D}_0$ of type $\Delta_0$. Higher-order sets $\mathcal{D}_k$ for $k\geq 1$ are represented by unsafe terms $\mathbf{D}_k$: they are constructed from $\mathbf{D}_0$ using a powerset construction that is unsafe.
\end{enumerate}
\bigskip

The second source of unsafety can be easily overcome, the idea is as follows. We introduce multiple domains of representation for a given formula. An element of $\mathcal{D}_k$ is thereby represented by countably many terms of type $\Delta_k^n$ where $n\in\nat$ indicates the level of the domain of representation. The type $\Delta_k^n$ is defined in such a way that its order strictly increases as $n$ grows. Furthermore, there exists a term that can lower the domain of representation of a given term. Thus each formula variable can have a different domain of representation, and since there are infinitely many such domains, it is always possible to find an assignment of representation domains to variables such that the resulting encoding term is safe.

There is no obvious way to eliminate unsafety in the two other cases however. 
For instance in the case of set-membership, Mairson's encoding (\ref{eqn:setmembership}) could be made safe by appealing to a term that changes the domain of representation of an encoded higher-order value of the type-hierarchy. Unfortunately, such transformation is intrinsically unsafe!
\smallskip

In the following paragraphs we present in details a variation over Mairson's encoding in which quantifier elimination is safely encoded.

\subsubsection{Encoding basic boolean operations}

Let $o$ be a base type and define the family of types $\sigma_0
\equiv o$, $\sigma_{n+1} \equiv \sigma_n\typear\sigma_n$ satisfying
$\ord{\sigma_n} = n$. Booleans are encoded over domains $\booltype_n
\equiv \sigma_n\typear o\typear o\typear o$ for $n\geq0$, each type
$\booltype_n$ being of order $n+1$. We write $\underline{i}_{n+1}$
to denote the term $\lambda x^{\sigma_n}.x : \sigma_{n+1}$ for
$n\geq0$. The truth values $\mathbf{true}$ and $\mathbf{false}$ are
represented by the following terms parameterized by $n \in \nat$:
\begin{align*}
  T^n &\equiv \lambda u^{\sigma_n} x^o y^o .x : \booltype_{n}\\
  F^n &\equiv \lambda u^{\sigma_n} x^o y^o .y : \booltype_{n} \enspace .
\end{align*}
Clearly these terms are safe. Moreover the following relations hold
for all $n,n'\geq 0$:
\begin{align*}
  \lambda u^{\sigma_{n'}} . T^{n+1}~\underline{i}_{n+1}  &\betared  T^{n'} \\
  \lambda u^{\sigma_{n'}} . F^{n+1}~\underline{i}_{n+1}  &\betared  F^{n'} \enspace .
\end{align*}
It is then possible to change the domain of representation of a Boolean value from a higher-level to another arbitrary level using the conversion term:
$$ \mathbf{C}^{n+1\mapsto n'}_0 \equiv \lambda m^{\booltype_{n+1}} u^{\sigma_{n'}}. m~\underline{i}_{n+1} : \booltype_{n+1} \typear \booltype_{n'}$$
so that if a term $M$ of type $\booltype_n$, for $n\geq1$, is beta-eta convertible to $T^n$ (resp.\ $F^n$) then $\mathbf{C}^{n\mapsto n'}_0~M$ of type $\booltype_{n'}$ is beta-eta convertible to $T^{n'}$ (resp.\ $F^{n'}$).

Observe that although $\mathbf{C}^{n+1\mapsto n'}_0$ is safe for all $n,n'\geq 0$, if we apply a variable to it then the resulting term
$$ x:B_{n+1} \stentail \mathbf{C}_0^{n+1\mapsto n'}~x : B_{n}$$
is safe if and only if $\ord{B_{n+1}}\geq\ord{B_{n'}}$, that is to say if and only if the transformation decreases the domain of representation of $x$.


Boolean functions are encoded by the following closed safe terms parameterized by $n$:
\begin{align*}
AND^n &\equiv \lambda p^{\booltype_n} q^{\booltype_n} u^{\sigma_n} x^o y^o . p~u~(q~u~x~y)~y : \booltype_n\typear\booltype_n\typear\booltype_n \\
OR^n &\equiv \lambda p^{\booltype_n} q^{\booltype_n} u^{\sigma_n} x^o y^o . p~u~x~(q~u~x~y) : \booltype_n\typear\booltype_{n}\typear\booltype_n \\
NOT^n &\equiv \lambda p^{\booltype_n} u^{\sigma_n} x^o \lambda y^o . p~u~y~x : \booltype_n\typear\booltype_n\typear\booltype_n
%\\ IF^n &\equiv \lambda p^{\booltype_n} q^{\booltype_n} u^{\sigma_n} x^o y^o. OR^n (NOT^n p)~q : \booltype_n\typear\booltype_n\typear\booltype_n
\enspace .
\end{align*}

\subsubsection{Coding elements of the type hierarchy}
For every $n\in\nat$ we define the hierarchy of type $\Delta_k^n$ as
follows: $\Delta_0^n \equiv \booltype_n$ and $\Delta_{k+1}^n \equiv
{\Delta_k^n}^*$ where for a given type $\alpha$, $\alpha^* = (\alpha
\typear \tau \typear \tau)\typear \tau \typear \tau$ for \emph{any} type $\tau$. We encode an occurrence of a formula variable $x^k$ by a term variable $x^k$
of type $\Delta_{k}^n$ for some level of domain representation
$n\in\nat$. Following Mairson's  encoding, each set $\mathcal{D}_k$
is represented by a list $\mathbf{D}_k^n$ consisting of all its
elements:
\begin{align*}
\mathbf{D}_0^n &\equiv \lambda c^{\booltype_n \typear \tau \typear \tau} e^\tau . c~T^n~(c~F^n~e) : \Delta_1^n \\
\mathbf{D}_{k+1}^n &\equiv powerset_{\Delta_k^n}\,\mathbf{D}_k^n : \Delta_{k+2}^n
\end{align*}
where
\begin{align*}
  powerset_\alpha &\equiv \lambda {A^*}^{(\alpha \typear \alpha^{**} \typear \alpha^{**}) \typear \alpha^{**} \typear \alpha^{**}}. \\
&\qquad  A^*~double_\alpha~(\lambda c^{\alpha^* \typear \tau\typear \tau} b^\tau . c~(\lambda c'^{\alpha\typear \tau\typear \tau} b'^\tau.b')~b)\\
 &: ((\alpha \typear \alpha^{**} \typear \alpha^{**}) \typear \alpha^{**} \typear \alpha^{**})\typear \alpha^{**} \\
  double_\alpha &\equiv \lambda x^\alpha~l^{(\alpha^* \typear \tau\typear \tau)\typear \tau\typear \tau}~ c^{\alpha^*\typear \tau\typear \tau}~b^\tau. \\
  & \qquad \qquad l(\lambda e^{\alpha^*}.c~(\lambda c'^{\alpha\typear \tau\typear \tau}~ b'^\tau.c'~\underline{x}~(e~c'~b')))(l~c~b)\\
 &: \alpha \typear \alpha^{**} \typear \alpha^{**} \enspace .
\end{align*}
(In the definition of $\mathbf{D}_{k+1}^n$, to see why it is possible
to apply $powerset_{\Delta_k^n}$ and $\mathbf{D}_k^n$ one needs to understand that the term $\mathbf{D}_k^n$ is of type $\Delta^n_{k+1}$ \emph{polymorphic in $\tau$}. The application can thus by typed by taking $\tau \equiv \Delta^n_{k+2}$ in the term
$\mathbf{D}_k^n$.)

Observe that the term $double$ is unsafe because the underlined variable occurrence $x$ is not bound together with $c'$. 
Consequently for all $n\geq0$, $\mathbf{D}_k^n$ is safe if and
only if $k=0$.


\subsubsection{Quantifier elimination}
Terms of type $\Delta_{k+1}^n$ are now used as iterators over lists of elements of type $\Delta_k^n$ and we set $\tau \equiv \booltype_n$ in the type $\Delta_{k+1}^n$ in order to iterate a level-$n$ Boolean function. Since $\ord{\Delta_k^n} \geq \ord{\booltype_n}$ for all $n$, all the instantiations of the terms $\mathbf{D}_k^n$ will be safe
 (although the terms $\mathbf{D}_k^n$ themselves are not safe for $k>1$).
 Following \cite{mairson1992spt}, quantifier elimination interprets the formula $\forall x^k.\Phi(x^k)$ as the iterated conjunction
$\mathbf{C}_0^{n\mapsto 0} \left( \mathbf{D}_k^n(\lambda x^k:\Delta_k^n.AND^n(\hat\Phi\,x^k))\,T^n \right)$ where $\hat\Phi$ is the interpretation of $\Phi$
and $n$ is the representation level chosen for the variable $x^k$. Similarly we interpret $\exists x^k.\Phi(x^k)$  by the iterated disjunction $\mathbf{C}_0^{n\mapsto 0} \left( \mathbf{D}_k^n(\lambda x^k:\Delta_k^n.AND^n(\hat\Phi\,x^k))\,T^n\right)$.

\subsubsection{Encoding the formula}

Given a formula of type theory, it is possible to encode it in the lambda calculus by inductively applying the above encodings of boolean operations and quantifiers on the formula; each variable occurrence in the formula being assigned some domain of representation.

We now show that there exists an assignment of representation domains for each variable occurrence such that the resulting term is safe. Let $x^{k_p}_p \ldots x^{k_1}_1$ for $p\geq1$ be the list of variables appearing in the formula, given in order of appearance of their binder in the formula (\ie,~$x^{k_p}_p$ is bound by the leftmost binder). We fix the domain of representation of each variable as follows. The right-most variable $x^{k_1}_1$ will be encoded in the domain $\Delta^0_{k_1}$; and if for $1\leq i< p$ the domain of representation of $x^{k_i}_i$ is $\Delta^l_{k_l}$ then the domain of representation of $x^{k_{i+1}}_{i+1}$ is defined as
$\Delta^{l'}_{k_{i+1}}$ where $l'$ is the smallest natural number such that $\ord{\Delta^{l'}_{k_{i+1}}}$ is strictly greater than $\ord{\Delta^{l}_{k_i}}$.

This way, since variables that are bound first have higher order, variables that are bound in  nested list-iterations (corresponding to nested quantifiers in the formula) are guaranteed to be safely bound.

\begin{example}
The formula  $\forall x^0 . \exists y^0 . x^0 \zor y^0$, which is
encoded by an unsafe term in Mairson's encoding, is represented in
our encoding by the safe term:
 $$\sentail \mathbf{C}_0^{1\mapsto 0} \left( \mathbf{D}_0^1~(\lambda x^0:\Delta_0^1. AND^0 ( \mathbf{D}_0^0 ~(\lambda y^0:\Delta_0^0. OR^0 (OR^0~(\mathbf{C}_0^{1\mapsto 0}~x^0)~y^0))~F^0))~T^1 \right) : \booltype_0 \enspace .$$
\end{example}


\subsubsection{Set-membership}
To complete the interpretation of prime formulae, we would need to show how to encode set membership. The use of multiple domains of representation does not suffice to turn Mairson's encoding into a safe term. We would further need to have a version of the Booleans conversion term $\mathbf{C}^{n+1\mapsto n'}_0$ generalized to higher-order sets.
This transformation can be interpreted as the simply-typed term:
$$ \mathbf{C}^{n\mapsto n'}_{k+1} \equiv \lambda m^{\Delta_{k+1}^n} u^{\Delta_k^n\typear \tau\typear \tau} v^\tau. m (\lambda z^{\Delta_k^n} w^\tau . \underline{u (\underline{\mathbf{C}^{n\mapsto n'}_k z}) w}) v : \Delta_{k+1}^n \typear \Delta_{k+1}^{n'} \enspace .$$
Unfortunately this term is safe if and only if $n=n'$---the largest
underlined subterm is safe just when $n\geq n'$ and the other
underline subterm is safe just when $n'\geq n$---in which case the transformation is of no interest.

This leads us to conjecture that the set-membership function is intrinsically unsafe.
\smallskip


If $\mathbf{C}^{n\mapsto n'}_{k+1}$ were safely representable then the encoding would go as follows: We set $\tau \equiv \booltype_0$ in the types  $\Delta_{k+1}^n$ for all $n,k\geq 0$ in order to iterate a level-$0$ Boolean function.
Firstly, the formulae ``$\mathbf{true} \in y^1$'' and ``$\mathbf{false} \in y^1$'' can be encoded by the safe terms $y^1 (\lambda x^0 . OR^0~x^0) F^0$ and $y^1 (\lambda x^0. OR^0(NOT^0~x^0)) F^0$ respectively.
For the general case ``$x^k\in y^{k+1}$''
 we proceed as in Mairson's proof \cite{mairson1992spt}: we introduce lambda-terms encoding set equality, set membership and subset tests, and we further parameterize these encodings by a natural number $n$.
\begin{align*}
member_{k+1}^{n+1} &\equiv \lambda x^{\Delta_k^{n+1}} y^{\Delta_{k+1}^{n+1}}.
%\\ & \quad\
(\mathbf{C}_{k+1}^{n+1\mapsto n}~y)~(\lambda z^{\Delta_k^n} . OR^0 (eq_k^n~(\mathbf{C}^{n+1\mapsto n}_k~x)~z))~F^0\\
  & : \Delta_k^{n+1} \typear \Delta_{k+1}^{n+1} \typear \booltype_0
\\
subset_{k+1}^n &\equiv \lambda x^{\Delta_{k+1}^n} y^{\Delta_{k+1}^n}.
% \\ & \qquad
x~(\lambda x^{\Delta_k^n} . AND^0 (member_{k+1}^n~x~y))~T^0 \\
  & : \Delta_{k+1}^n \typear \Delta_{k+1}^n \typear\booltype_0
\\
eq_0^n &\equiv \lambda x^{\booltype_n} .\lambda y^{\booltype_n}. \mathbf{C}_0^{n\mapsto 0}~ \left(OR^n (AND^n~x~y) (AND^n (NOT^n~x)(NOT^n~y))\right) \\
 &: \booltype_n \typear \booltype_n \typear\booltype_0
\\
eq_{k+1}^n &\equiv \lambda x^{\Delta_{k+1}^n}~ y^{\Delta_{k+1}^n}.
%\\ & \qquad
   (\lambda op^{\Delta_{k+1}^n\typear\Delta_{k+1}^n\typear\booltype_0}. AND^0 (op~x~y)(op~y~x))~subset_{k+1}^n \\
  & : \Delta_{k+1}^n \typear \Delta_{k+1}^n \typear \booltype_0 \enspace .
\end{align*}
The variables in the definition of $eq_{k+1}^n$ and $subset_{k+1}^n$
are safely bounds. Moreover, the occurrence of $x$ in
$member_{k+1}^{n+1}$ is now safely bound---which was not the case in
Mairson's original encoding---thanks to the fact that the
representation domain of $z$ is lower than that of $x$. The formula
$x^k\in y^{k+1}$ can then be encoded as
$$x:\Delta_k^n, y:\Delta_{k+1}^{n'}\stentail member_{k+1}^{u}~ (\mathbf{C}_k^{n\mapsto u}~x)~(\mathbf{C}^{n'\mapsto u}_{k+1} ~y) : \booltype_0$$
for some $n,n'\geq 2$ and $u = \min(n,n')+1$.

Unfortunately, this encoding is not completely safe because it uses
the unsafe conversion terms $\mathbf{C}_k^{n\mapsto n'}$ for
$k\geq1$.



\subsection{PSPACE-hardness}
We observe that instances of the True Quantified Boolean Formulae
satisfaction problem (TQBF) are special instances of the decision
problem for finite type theory. These instances corresponds to
formulae in which set membership is not allowed and variables are
all taken from the base domain $\mathcal{D}_0$. As we have shown in
the previous section, such restricted formulae can be safely encoded
in the safe lambda calculus. Therefore since TQBF is
PSPACE-complete we have:
\begin{theorem}
\label{thm:pspacehardness}
 Deciding $\beta\eta$-equality of two safe lambda-terms is PSPACE-hard. \qed
\end{theorem}

%%We assume that the quantified propositional formula is given in prenex form:
%%$$\$_{n-1} x_{n-1} \ldots \$_0 x_0 . \psi(x_0, \ldots, x_{n-1})$$
%%where $\$_i \in \{\exists,\forall\}$ for $0\leq i\leq n-1$.
%%
%%The encoding is as follows:
%%\begin{align*}
%%\sem{1} &= T^0  : \booltype \\
%%\sem{0} &= F^0 : \booltype \\
%%\sem{x_i} &= x_i\downarrow_0 = x_i~T^{i-1}~F^{i-1}\ldots T^1~F^1: \booltype \qquad \hbox{where $x_i:\booltype_i$}\\
%%\sem{\psi_1\zand \psi_2} &= AND^0~\sem{\psi_1}~\sem{\psi_2}
%%:\booltype  \\
%%\sem{\psi_1\zor \psi_2} &= OR^0~\sem{\psi_1}~\sem{\psi_2}
%%:\booltype  \\
%%\sem{\neg \psi} &= NOT^0~\sem{\psi}
%%:\booltype  \\
%%\sem{\forall x_i.\psi(\ldots, x_i, \ldots)} & = \mathbf{D}_0^i(\lambda x^{\booltype_i}. AND^0~\sem{\psi(\ldots, x_i, \ldots)})~T^0 :\booltype\\
%%\sem{\exists x_i.\psi(\ldots, x_i, \ldots)} & = \mathbf{D}_0^i(\lambda x^{\booltype_i}.OR^0~\sem{\psi(\ldots, x_i, \ldots)})~F^0 :\booltype
%%\end{align*}
%%The size of $\sem{\psi}$ is in $\bigo(|\psi|^2)$.
%
%It is easy to check that this encoding is safe.
\begin{example}
Using the encoding where $\tau$ is set to  $\booltype_0$ in the types $\Delta_k^n$ for all $k,n\geq 0$, the formula $\forall x \exists y \exists z (x\zor y\zor z)\zand(\neg x\zor \neg y\zor \neg z)$ is represented by the safe term:
\begin{align*}
\sentail~&\mathbf{D}_0^2(\lambda x^{\booltype_2}. AND^0\\
&\quad\quad (\mathbf{D}_0^1(\lambda y^{\booltype_1}.OR^0\\
&\quad\quad\quad (\mathbf{D}_0^0(\lambda z^{\booltype_0}.OR^0\\
&\quad\quad\quad\quad (AND^0 (OR^0(OR^0~(\mathbf{C}_0^{2\mapsto 0}~x)~(\mathbf{C}_0^{1\mapsto 0}~y))z) \\
&\quad\quad\quad\quad\quad (OR^0(OR^0(NEG^0 (\mathbf{C}_0^{2\mapsto 0}~x))(NEG^0 (\mathbf{C}_0^{1\mapsto 0}~y)))(NEG^0~z))) \\
&\quad\quad\quad )F^0)\\
&\quad\quad)F^0)\\
&\quad) T^0 \\
& : \booltype_0 \enspace .
\end{align*}
\end{example}

\begin{remark}
The Boolean satisfaction problem (SAT) is just a particular instance of TQBF
where formulae are restricted to use only existential quantifiers, thus
the safe lambda calculus is also NP-hard. Asperti gave an interpretation of SAT in the simply-typed lambda calculus but his encoding relies on unsafe terms \cite{asperti-np}.
\end{remark}


\begin{remark}
\begin{enumerate}[(i)]
\item Because the safety condition restricts expressivity
in a non-trivial way, one can reasonably expect the
beta-eta equivalence problem to have a
lower complexity in the safe case than in the normal case; this intuition is strengthened by our failed attempt to encode type theory in the safe lambda calculus. No upper bounds is known at present. On the other hand our PSPACE-hardness result is probably a coarse lower bound; it would be interesting to know whether we also have EXPTIME-hardness.

\item % Beta-eta equivalence for terms limited to a finite set of types
Statman showed \cite{Statman:1979:TLE} that
when restricted to some finite set of types, the beta-eta equivalence problem is PSPACE-hard.
Such result is unlikely to hold in the safe lambda calculus. This is suggested by the fact that we had to use the entire type hierarchy to encode TQBF in the safe lambda calculus.
In fact we expect the beta-eta equivalence problem for safe terms to have a complexity lower than PSPACE when restricted to any finite set of types.


% Normalization
\item
The \emph{normalization problem} (``Given a (safe) term $M$, what is its $\beta$-normal form?'') is non-elementary. Indeed, let $\tau_{-2}
\equiv o$ and for $n\geq -1$, $\tau_n \equiv \tau_{n-1}\typear
\tau_{n-1}$. For $k,n\in \nat$, let $\overline{k}^n$ denote
the $k^{th}$ Church Numeral $\lambda s^{\tau_{n-1}}
z^{\tau_{n-2}} . s( \cdots (s(s\,z) \cdots)$ (with $k$ applications of $s$) of type $\tau_n$. Then for $n\geq1$, the safe term
$\overline{2}^{n-1}~\overline{2}^{n-2}\cdots \overline{2}^0$ of type
$\tau_0$ has size $\bigo(n)$ and its normal form
$\overline{\towerexp{n}{1}}^0$ has size $\bigo(\towerexp{n}{1})$.

Thus in the simply-typed lambda calculus, beta-eta equivalence is essentially as hard as normalization.
We do not know if this is the case in the safe lambda calculus.

\item % The beta-reduction problem
A related problem is that of \emph{beta-reduction}: ``Given a $\beta$-normal term $M_1$
 and a term $M_2$, does $M_2$ $\beta$-reduce to $M_1$?''. It is known to be PSPACE-complete when restricted to order-$3$ terms \cite{schubert2001cbr}, but no complexity result is known for higher orders. The safe case can potentially give rise to interesting complexity characterizations at higher-orders.

\end{enumerate}
\end{remark}
