\newcommand\bigo{\mathcal{O}} % big O notation
\newcommand\booltype{\mathsf{B}}
\newcommand\towerexp[2]{\exp_{#1}(#2)}
%\newcommand\towerexp[2]{2_{#1}(#2)}

\section{Complexity of the safe lambda calculus}
This section is concerned with the complexity of the
beta-eta equivalence problem for the safe lambda calculus:
given two safe lambda terms, are they equivalent up to $\beta\eta$-conversion?



Let $\towerexp{h}{m}$ denote the tower-of-exponential function
defined by $\towerexp{0}{m} = m$ and $\towerexp{h+1}{m} =
2^{\towerexp{h}{m}}$. Recall that a program is
\defname{elementary recursive} if its run-time can be bounded by
$\towerexp{K}{n}$ for some constant $K$ where $n$ is the length of
the input.

\subsection{Statman's result}

A famous result by Statman  states that deciding the
$\beta\eta$-equality of two first-order typable lambda terms is not
elementary recursive \cite{Statman:1979:TLE}. The proof proceeds by
encoding the Henkin quantifier elimination of type theory in the
simply-typed lambda calculus. Simpler proofs have subsequently been
given, one by Mairson in \cite{mairson1992spt} and another by Loader
in \cite{Loader1998}. Both proceed by encoding the Henkin quantifier
elimination procedure in the lambda-calculus, as in the original
proof, but their use of list iteration to perform quantifier
elimination makes them much easier to understand.

It turns out that all these encodings rely on unsafe terms:
Statman's encoding uses the conditional function $\sf sg$ which is
not definable in the safe lambda-calculus
\cite{blumong:safelambdacalculus}; Mairson's encoding uses unsafe
terms to encode both quantifier elimination and set membership, and
Loader's encoding uses unsafe term to build list iterators. We are
thus led to conjecture that finite type theory (see definition in
Sec.\ \ref{sec:mairsonenc}) is intrinsically unsafe in the sense
that every encoding of it in the lambda calculus is necessarily
unsafe. Of course this conjecture does not rule out the possibility
that another non-elementary problem is encodable in the safe lambda
calculus.

We start this section by presenting an adaptation of Mairson's
encoding. We show that quantifier elimination can be safely encoded
and explain why it is problematic to encode set-membership safely.
We will then use this encoding to interpret the True Quantifier
Boolean Formula (TQBF) problem in the safe lambda calculus, thus
showing that deciding beta-eta equality is PSPACE-hard.

\subsection{Mairson's encoding}
\label{sec:mairsonenc} We recall the definition of finite type
theory. Let us define $\mathcal{D}_0 =
\{\mathbf{true},\mathbf{false}\}$ and $\mathcal{D}_{k+1}
=powerset(\mathcal{D}_k)$. For $k\geq0$, we write $x^k$, $y^k$ and
$z^k$ to denote variables ranging over $\mathcal{D}_k$. Prime
formulas are $x^0$, $\mathbf{true}\in y^1$, $\mathbf{false}\in y^1$,
and  $x^k \in y^{k+1}$. Formulae are built up from prime formulae
using the logical connectives $\zand$,$\zor$,$\rightarrow$,$\neg$
and the quantifiers $\forall$ and $\exists$. Meyer showed that
deciding the validity of such formulae requires nonelementary time
\cite{Meyer1974}.
\smallskip

In Mairson's encoding, boolean values are encoded by terms of type
$\booltype = \sigma \typear \sigma\typear \sigma$ for some type
$\sigma$, and variables of order $k \geq 0$ are encoded by terms of
type $\Delta_k$ defined as $\Delta_0 \equiv \booltype$ and
$\Delta_{k+1} \equiv {\Delta_k}^*$ where for any type $\alpha$,
$\alpha^* = (\alpha \typear \tau \typear \tau)\typear \tau \typear
\tau$ for some type $\tau$. Using this encoding, unsafety manifests
itself in two different ways.
\begin{enumerate}[1.]
  \item
        First in the encoding of set membership. The prime formula $x^k \in y^{k+1}$ is encoded as \begin{equation} x^k : \Delta_k, y^{k+1}:\Delta_{k+1} \stentail y^{k+1} (\lambda y^k : \Delta_k . OR (eq_k~\underline{x^k}~y^k)~F : \Delta_k \typear \Delta_{k+1} \typear \Delta_0 \label{eqn:setmembership}\end{equation}
for some terms $OR$, $F$, $eq_k$.
This term is unsafe because of the underline occurrence of $x^k$ which is not abstracted together with $y^k$.

\item Secondly, quantifier elimination is performed using a list iterator $\mathbf{D}_{k+1}$ of type $\Delta_{k+2}$ which acts like the $fold\_right$ function (from functional programming) over the list of all elements of $\mathcal{D}_k$.
Thus for instance the formula $\forall x^0 . \exists y^0 . x^0
\zor y^0$ is encoded as $$\stentail \mathbf{D}_0 (\lambda
x^0:\Delta_0. AND (\mathbf{D}_0 (\lambda y^0:\Delta_0. OR
(\underline{x^0} \zor y^0)) F))\ T : \booltype$$ where the type
$\tau$ is instanciated as $\booltype$. This term is unsafe since
the underlined occurrence is unsafely bound. This is due to the
presence of two nested quantifiers in the formula, which are
encoded as two nested list iterations. More generally, nested
binding will be encoded safely if and only if every variable $x$
in the formula is bound by the first quantifier $\exists z$ or
$\forall z$ in the path to the root of the AST of the formula
verifying $\ord{z} \geq \ord{x}$. For instance if set-membership
could be encoded safely then the interpretation of $\forall x^k
. \exists y^{k+1} . x^k \in y^{k+1}$ would be unsafe whereas the
encoding of $\forall y^{k+1} . \exists x^k . x^k \in y^{k+1}$
would be safe.
\end{enumerate}

Surprisingly, the `unsafety' of the quantifier elimination procedure can be easily overcome. The idea is as follows. We introduce multiple domains of representation for a given formula. An element of $\mathcal{D}_k$ is thereby represented by countably many terms of type $\Delta_k^n$ where $n\in\nat$ indicates the level of the domain of representation. The type $\Delta_k^n$ is defined in such a way that its order strictly increases as $n$ grows. Furthermore, there exists a term that can lower the domain of representation of a given term. Thus each formula variable can have a different domain of representation, and since there are infinitely many such domains, it is always possible to find an assignment of representation domains to variables such that the resulting encoding term is safe.

For set-membership, however, there is no obvious way to obtain a safe encoding. In order to turn Mairson's encoding of set-membership (Eq.\ \ref{eqn:setmembership}) into a safe term, we would need to have access to a function that changes the domain of representation of an encoded higher-order value of the type-hierarchy. Unfortunately, such transformation is intrinsically unsafe!
\smallskip

We now present the encoding in details.

\subsubsection{Encoding basic boolean operations}

Let $o$ be a base type and define the family of types $\sigma_0
\equiv o$, $\sigma_{n+1} \equiv \sigma_n\typear\sigma_n$ satisfying
$\ord{\sigma_n} = n$. Booleans are encoded over domains $\booltype_n
\equiv \sigma_n\typear o\typear o\typear o$ for $n\geq0$, each type
$\booltype_n$ being of order $n+1$. We write $\underline{i}_{n+1}$
to denote the term $\lambda x^{\sigma_n}.x : \sigma_{n+1}$ for
$n\geq0$. The truth values $\mathbf{true}$ and $\mathbf{false}$ are
represented by the following terms parameterized by $n \in \nat$:
\begin{align*}
  T^n &\equiv \lambda u^{\sigma_n} x^o y^o .x : \booltype_{n}\\
  F^n &\equiv \lambda u^{\sigma_n} x^o y^o .y : \booltype_{n}
\end{align*}
Clearly these terms are safe. Moreover the following relations hold
for all $n,n'\geq 0$:
\begin{align*}
  \lambda u^{\sigma_{n'}} . T^{n+1}~\underline{i}_{n+1}  &\betared  T^{n'} \\
  \lambda u^{\sigma_{n'}} . F^{n+1}~\underline{i}_{n+1}  &\betared  F^{n'}
\end{align*}
Hence it is possible to change the domain of representation of a Boolean value from a higher-level to another arbitrary level using the transformation:
$$ \mathbf{C}^{n+1\mapsto n'}_0 \equiv \lambda m^{\booltype_{n+1}} u^{\sigma_{n'}}. m~\underline{i}_{n+1} : \booltype_{n+1} \typear \booltype_{n'}$$
so that if a term $M : \booltype_n$ for $n\geq1$ is beta-eta convertible to $T^n$ (resp.\ $F^n$) then $\mathbf{C}^{n\mapsto n'}_0~M :\booltype_{n'}$ is beta-eta convertible to $T^{n'}$ (resp.\ $F^{n'}$).

Observe that although the term $\mathbf{C}^{n+1\mapsto n'}_0$ is safe for all $n,n'\geq 0$, if we apply it to a variable then the resulting term
$$ x:B_{n+1} \stentail \mathbf{C}_0^{n+1\mapsto n'}~x : B_{n}$$
is safe if and only if the transformation decreases the domain of representation of $x$ \ie $\ord{B_{n+1}}\geq\ord{B_{n'}}$.


Boolean functions are encoded by the following safe terms parameterized by $n$:
\begin{align*}
AND^n &\equiv \lambda p^{\booltype_n} q^{\booltype_n} u^{\sigma_n} x^o y^o . p~u~(q~u~x~y)~y : \booltype_n\typear\booltype_n\typear\booltype_n \\
OR^n &\equiv \lambda p^{\booltype_n} q^{\booltype_n} u^{\sigma_n} x^o y^o . p~u~x~(q~u~x~y) : \booltype_n\typear\booltype_{n}\typear\booltype_n \\
NOT^n &\equiv \lambda p^{\booltype_n} u^{\sigma_n} x^o \lambda y^o . p~u~y~x : \booltype_n\typear\booltype_n\typear\booltype_n
%\\ IF^n &\equiv \lambda p^{\booltype_n} q^{\booltype_n} u^{\sigma_n} x^o y^o. OR^n (NOT^n p)~q : \booltype_n\typear\booltype_n\typear\booltype_n
\end{align*}

\subsubsection{Coding elements of the type hierarchy}
For any $n\in\nat$ we define the hierarchy of type $\Delta_k^n$ as
follows: $\Delta_0^n \equiv \booltype_n$ and $\Delta_{k+1}^n \equiv
{\Delta_k^n}^*$ where for any type $\alpha$, $\alpha^* = (\alpha
\typear \tau \typear \tau)\typear \tau \typear \tau$. An occurrence
of a formula variable $x^k$ will be encoded as a term variable $x^k$
of type $\Delta_{k}^n$ for some level of domain representation
$n\in\nat$. Following Mairson's  encoding, each set $\mathcal{D}_k$
is represented by a list $\mathbf{D}_k^n$ constituted of all its
elements:
\begin{align*}
\mathbf{D}_0^n &\equiv \lambda c^{\booltype_n \typear \tau \typear \tau} e^\tau . c~T^n~(c~F^n~e) : \Delta_1^n \\
\mathbf{D}_{k+1}^n &\equiv powerset~\mathbf{D}_k^n : \Delta_{k+2}^n
\end{align*}
where
\begin{align*}
  powerset &\equiv \lambda {A^*}^{(\alpha \typear \alpha^{**} \typear \alpha^{**}) \typear \alpha^{**} \typear \alpha^{**}}. \\
&\qquad  A^*~double~(\lambda c^{\alpha^* \typear \tau\typear \tau} b^\tau . c~(\lambda c'^{\alpha\typear \tau\typear \tau} b'^\tau.b')~b)\\
 &: ((\alpha \typear \alpha^{**} \typear \alpha^{**}) \typear \alpha^{**} \typear \alpha^{**})\typear \alpha^{**} \\
  double &\equiv \lambda x^\alpha~l^{(\alpha^* \typear \tau\typear \tau)\typear \tau\typear \tau}~ c^{\alpha^*\typear \tau\typear \tau}~b^\tau. \\
  & \qquad \qquad l(\lambda e^{\alpha^*}.c~(\lambda c'^{\alpha\typear \tau\typear \tau}~ b'^\tau.c'~\underline{x}~(e~c'~b')))(l~c~b)\\
 &: \alpha \typear \alpha^{**} \typear \alpha^{**}
\end{align*}
In all these terms, the only variable occurrence that is potentially
unsafe is the underlined occurrence $x$ in $double$. This occurrence
is safely bound just when $\ord{\alpha} \geq \ord{\tau}$.
Consequently for all $k,n\geq0$, $\mathbf{D}_k^n$ is safe if and
only if $\ord{\alpha} \geq \ord{\tau}$.


\subsubsection{Quantifier elimination}
Terms of type $\Delta_{k+1}^n$ are now used as iterators over list of elements of type $\Delta_k^n$ and we set $\tau \equiv \booltype_n$ in the type $\Delta_{k+1}^n$ in order to iterate a level-$n$ Boolean function. Since $\ord{\Delta_k^n} \geq \ord{\booltype_n}$ for all $n$, all the instantiations of the terms $\mathbf{D}_k^n$ will be safe. Following \cite{mairson1992spt}, quantifier elimination interprets the formula $\forall x^k.\Phi(x^k)$ as the iterated conjunction:
$\mathbf{C}_0^{n\mapsto 0} \left( \mathbf{D}_k^n(\lambda x^k:\Delta_k^n.AND^n(\hat\Phi~x^k))~T^n \right)$ where $\hat\Phi$ is the interpretation of $\Phi$
and $n$ is the representation level chosen for the variable $x^k$; similarly $\exists x^k.\Phi(x^k)$  is interpreted by the iterated disjunction $\mathbf{C}_0^{n\mapsto 0} \left( \mathbf{D}_k^n(\lambda x^k:\Delta_k^n.AND^n(\hat\Phi~x^k))~T^n\right)$.

Let $x^{k_p}_p \ldots x^{k_1}_1$ for $p\geq1$ be the list of variables appearing in the formula. W.l.o.g.\ we can assume that they are given in the order of appearance of their binder in the formula \ie $x^{k_p}_p$ is bound by the leftmost binder. We fix the domain of representation of each variable as follows. The right-most variable $x^{k_1}_1$ will be encoded in the domain $\Delta^0_{k_1}$; suppose that for $1\leq i< p$ the domain of representation of $x^{k_i}_i$ is $\Delta^l_{k_l}$ then the domain of representation of $x^{k_{i+1}}_{i+1}$ is defined as
$\Delta^{l'}_{k_{i+1}}$ where $l'$ is the smallest natural number such that $\ord{\Delta^{l'}_{k_{i+1}}}$ is strictly greater than $\ord{\Delta^{l}_{k_i}}$.

This way, since variables that are bound first have higher order, the variables that are bound in the nested list-iterations (corresponding to the nested quantifiers in the formula) are necessarily safely bound.

\begin{example}
The formula  $\forall x^0 . \exists y^0 . x^0 \zor y^0$, which is
encoded by an unsafe term in Mairson's encoding, is represented in
our encoding by the safe term:
 $$\sentail \mathbf{C}_0^{1\mapsto 0} \left( \mathbf{D}_0^1~(\lambda x^0:\Delta_0^1. AND^0 ( \mathbf{D}_0^0 ~(\lambda y^0:\Delta_0^0. OR^0 (OR^0~(\mathbf{C}_0^{1\mapsto 0}~x^0)~y^0))~F^0))~T^1 \right) : \booltype_0 \ .$$
\end{example}


\subsubsection{Set-membership}
To complete the interpretation of prime formulae, we would need to show how to encode set membership. The use of multiple domains of representation does not suffice to turn Mairson's encoding into a safe term. We would further need to have a version of the Booleans conversion term $\mathbf{C}^{n+1\mapsto n'}_0$ generalized to higher-order sets.
This transformation can be interpreted as the simply-typed term:
$$ \mathbf{C}^{n\mapsto n'}_{k+1} \equiv \lambda m^{\Delta_{k+1}^n} u^{\Delta_k^n\typear \tau\typear \tau} v^\tau. m (\lambda z^{\Delta_k^n} w^\tau . \underline{u (\underline{\mathbf{C}^{n\mapsto n'}_k z}) w}) v : \Delta_{k+1}^n \typear \Delta_{k+1}^{n'}$$
Unfortunately this term is safe if and only if $n=n'$ (the largest
underlined subterm is safe just when $n\geq n'$ and the other
underline subterm is safe just when $n'\geq n$), thus making this
transformation useless in the safe lambda calculus.

This leads us to conjecture that the set-membership test function is intrinsically unsafe.
\smallskip


%%%%%%%%%%%%%%%%%%%%%%%
%% The following commented section gives an UNSAFE encoding of set-memberhsip.

If $\mathbf{C}^{n\mapsto n'}_{k+1}$ were safely representable then the encoding would go as follows: We set $\tau \equiv \booltype_0$ in the types  $\Delta_{k+1}^n$ for all $n,k\geq 0$ in order to iterate a level-$0$ Boolean function.
Firstly, the formulae ``$\mathbf{true} \in y^1$'' and ``$\mathbf{false} \in y^1$'' can be encoded by the safe terms $y^1 (\lambda x^0 . OR^0~x^0) F^0$ and $y^1 (\lambda x^0. OR^0(NOT^0~x^0)) F^0$ respectively.
For the general case ``$x^k\in y^{k+1}$''
we proceed as in \cite{mairson1992spt} by introducing lambda-terms encoding set equality, set membership and subset tests, and we further parameterize these encoding by $n\in\nat$.
\begin{align*}
member_{k+1}^{n+1} &\equiv \lambda x^{\Delta_k^{n+1}} y^{\Delta_{k+1}^{n+1}}.
%\\ & \quad\
(\mathbf{C}_{k+1}^{n+1\mapsto n}~y)~(\lambda z^{\Delta_k^n} . OR^0 (eq_k^n~(\mathbf{C}^{n+1\mapsto n}_k~x)~z))~F^0\\
  & : \Delta_k^{n+1} \typear \Delta_{k+1}^{n+1} \typear \booltype_0
\\
subset_{k+1}^n &\equiv \lambda x^{\Delta_{k+1}^n} y^{\Delta_{k+1}^n}.
% \\ & \qquad
x~(\lambda x^{\Delta_k^n} . AND^0 (member_{k+1}^n~x~y))~T^0 \\
  & : \Delta_{k+1}^n \typear \Delta_{k+1}^n \typear\booltype_0
\\
eq_0^n &\equiv \lambda x^{\booltype_n} .\lambda y^{\booltype_n}. \mathbf{C}_0^{n\mapsto 0}~ \left(OR^n (AND^n~x~y) (AND^n (NOT^n~x)(NOT^n~y))\right) \\
 &: \booltype_n \typear \booltype_n \typear\booltype_0
\\
eq_{k+1}^n &\equiv \lambda x^{\Delta_{k+1}^n}~ y^{\Delta_{k+1}^n}.
%\\ & \qquad
   (\lambda op^{\Delta_{k+1}^n\typear\Delta_{k+1}^n\typear\booltype_0}. AND^0 (op~x~y)(op~y~x))~subset_{k+1}^n \\
  & : \Delta_{k+1}^n \typear \Delta_{k+1}^n \typear \booltype_0
\end{align*}
The variables in the definition of $eq_{k+1}^n$ and $subset_{k+1}^n$
are safely bounds. Moreover, the occurrence of $x$ in
$member_{k+1}^{n+1}$ is now safely bound (this was not the case in
Mairson's original encoding) thanks to the fact that the
representation domain of $z$ is lower than that of $x$. The formula
$x^k\in y^{k+1}$ can then be encoded as
$$x:\Delta_k^n, y:\Delta_{k+1}^{n'}\stentail member_{k+1}^{u}~ (\mathbf{C}_k^{n\mapsto u}~x)~(\mathbf{C}^{n'\mapsto u}_{k+1} ~y) : \booltype_0$$
for some $n,n'\geq 2$ and $u = \min(n,n')+1$.

Unfortunately, this encoding is not completely safe because it uses
the unsafe conversion terms $\mathbf{C}_k^{n\mapsto n'}$ for
$k\geq1$.

%%
%%%%%%%%%%%%%%%%%%%%%%%


\subsection{PSPACE-hardness}
We observe that instances of the True Quantified Boolean Formulae
satisfaction problem (TQBF) are special instances of the decision
problem for finite type theory. These instances corresponds to
formulae in which set membership is not allowed and variables are
all taken from the base domain $\mathcal{D}_0$. As we have shown in
the previous section, such restricted formulae can be safely encoded
in the safe lambda calculus. Therefore since TQBF is
PSPACE-complete, deciding $\beta\eta$-equality of two terms of the
safe lambda calculus is PSPACE-hard.

%%We assume that the quantified propositional formula is given in prenex form:
%%$$\$_{n-1} x_{n-1} \ldots \$_0 x_0 . \psi(x_0, \ldots, x_{n-1})$$
%%where $\$_i \in \{\exists,\forall\}$ for $0\leq i\leq n-1$.
%%
%%The encoding is as follows:
%%\begin{align*}
%%\sem{1} &= T^0  : \booltype \\
%%\sem{0} &= F^0 : \booltype \\
%%\sem{x_i} &= x_i\downarrow_0 = x_i~T^{i-1}~F^{i-1}\ldots T^1~F^1: \booltype \qquad \hbox{where $x_i:\booltype_i$}\\
%%\sem{\psi_1\zand \psi_2} &= AND^0~\sem{\psi_1}~\sem{\psi_2}
%%:\booltype  \\
%%\sem{\psi_1\zor \psi_2} &= OR^0~\sem{\psi_1}~\sem{\psi_2}
%%:\booltype  \\
%%\sem{\neg \psi} &= NOT^0~\sem{\psi}
%%:\booltype  \\
%%\sem{\forall x_i.\psi(\ldots, x_i, \ldots)} & = \mathbf{D}_0^i(\lambda x^{\booltype_i}. AND^0~\sem{\psi(\ldots, x_i, \ldots)})~T^0 :\booltype\\
%%\sem{\exists x_i.\psi(\ldots, x_i, \ldots)} & = \mathbf{D}_0^i(\lambda x^{\booltype_i}.OR^0~\sem{\psi(\ldots, x_i, \ldots)})~F^0 :\booltype
%%\end{align*}
%%The size of $\sem{\psi}$ is in $\bigo(|\psi|^2)$.
%
%It is easy to check that this encoding is safe.
\begin{example}
Using the encoding where $\tau$ is set to  $\booltype_0$ in the types $\Delta_k^n$ for all $k,n\geq 0$, the formula $\forall x \exists y \exists z (x\zor y\zor z)\zand(\neg x\zor \neg y\zor \neg z)$ is represented by the safe term:
\begin{align*}
\sentail~&\mathbf{D}_0^2(\lambda x^{\booltype_2}. AND^0\\
&\quad\quad (\mathbf{D}_0^1(\lambda y^{\booltype_1}.OR^0\\
&\quad\quad\quad (\mathbf{D}_0^0(\lambda z^{\booltype_0}.OR^0\\
&\quad\quad\quad\quad (AND^0 (OR^0(OR^0~(\mathbf{C}_0^{2\mapsto 0}~x)~(\mathbf{C}_0^{1\mapsto 0}~y))z) \\
&\quad\quad\quad\quad\quad (OR^0(OR^0(NEG^0 (\mathbf{C}_0^{2\mapsto 0}~x))(NEG^0 (\mathbf{C}_0^{1\mapsto 0}~y)))(NEG^0~z))) \\
&\quad\quad\quad )F^0)\\
&\quad\quad)F^0)\\
&\quad) T^0 \\
& : \booltype_0
\end{align*}
\end{example}

\begin{remark}
Since the Boolean satisfaction problem (SAT) can be reduced to TQBF
(where formulae are restricted to use only existential quantifiers),
the safe lambda calculus is also NP-hard. In \cite{asperti-np},
Asperti gave an interpretation of SAT in the simply-typed lambda
calculus but his encoding relies on unsafe terms.
\end{remark}

\subsection{Other complexity results}




\subsubsection{Better lower bound?}
Since the safety condition restricts the expressivity of the lambda
calculus in a non-trivial way, one can reasonably expect the
beta-eta equality problem (where types are not restricted) to have a
lower complexity in the safe case than in the normal case. Our
failed attempt to encode type theory in the safe lambda calculus
suggests that the non-elementary lower bounds that holds in the
simply-typed lambda calculus no longer applies in the safe lambda
calculus. (Nevertheless, one may not rule out the possibility that
another non recursive problem may be encodable in the safe lambda
calculus.)

We have shown that the problem is PSPACE-hard but this is probably a coarse lower bound. It would be interesting to know
whether it is also EXPTIME-hard.

\subsubsection{Upper bound}
At present, no upper bound is known for the equivalence problem for safe terms.
%Proceeding by reduction to the
%normalization problem does not help since there is no interesting
%upper-bound for this more general problem (which is non-elementary
%as shown in the previous paragraph).


\subsubsection{Beta-eta equivalence for terms limited to a finite set of types}
Statman showed \cite{Statman:1979:TLE} that there exists a finite
set of types such that the beta-eta equivalence problem restricted to terms
of these types is PSPACE-hard.

The picture is different in the safe lambda calculus since our
encoding of TQBF requires the full type hierarchy. It
was indeed necessary to introduce variables of higher-order in order
to eliminate `unsafety'. Consequently, we had to use simple types of
unbounded order (the order is linear in the size of the QBF
formula). We suspect the decidability problem for safe terms
restricted to any finite set of types to have a complexity lower
than PSPACE.


\subsubsection{Normalization}
The \emph{normalization problem} is: given a term $M$, what is its $\beta$-normal form? This problem is non-elementary even when restricted to safe terms as the following example shows. Let $\tau_{-2}
\equiv o$ and for $n\geq -1$, $\tau_n \equiv \tau_{n-1}\typear
\tau_{n-1}$. For $k,n\in \nat$ we write $\overline{k}^n$ to denote
the $k$th Church Numeral parameterized by $n$ as follows:
$$\overline{k}^n \equiv \lambda s^{\tau_{n-1}}
z^{\tau_{n-2}} . \overbrace{s( \ldots (s (s}^{k\hbox{ times}}z)
\ldots) : \tau_n \ .$$ Then for $n\geq1$, the safe term
$\overline{2}^{n-1}~\overline{2}^{n-2}\ldots \overline{2}^0$ of type
$\tau_0$ has length $\bigo(n)$ whereas its normal form
$\overline{\towerexp{n}{1}}^0$ has length $\bigo(\towerexp{n}{1})$.

Statman's result shows that in the simply-typed lambda calculus, the
beta-eta equality problem is essentially as hard as the
normalization problem i.e.~they are both non-elementary. It is not known whether this is still the
case in the safe lambda calculus. In particular, it may be the case
that the beta-eta equivalence problem is elementary although we know that the normalization problem is not.


\subsubsection{The beta-reduction problem}
The \emph{beta-reduction problem} is related to the beta-eta equivalence problem. It can be stated as follows: given a term $M_1$ in $\beta$-normal form and a term $M_2$ (possibly containing redexes), does $M_2$ $\beta$-reduces to $M_1$?

In \cite{schubert2001cbr}, Schubert
gave a PSPACE algorithm to decide the $\beta$-reduction problem for order-$3$
lambda terms. Since order-$3$ terms are sufficient to encode TQBF in the lambda calculus, Schubert shows that the problem is PSPACE-complete. No complexity result is known for restrictions of this problem to terms of order greater than $3$. A natural question to ask is whether complexity characterizations can be obtained when restricting the problem to safe terms.
