\section*{Appendix A -- Computation tree, traversals and correspondence}
\label{sec:correspondence}

In this section we adapt the notions of computation tree and
traversals over the computation tree originally introduced in
\cite{OngLics2006} to the context of the simply-typed
lambda calculus without constants. Everything remains valid in the
presence of \emph{uninterpreted}\footnote{A constant $f$ is
  \emph{uninterpreted} if the small-step semantics of the language
  does not contain any rule of the form $f \dots \rightarrow e$. $f$
  can be regarded as a data constructor.}  constants since we can just
consider them as free variables. Moreover there is no restriction on
the order of these constants, contrary to \cite{OngLics2006} where
constants need to be of order at most one. We will also state the
\emph{Correspondence Theorem} (Theorem \ref{thm:correspondence}) used
in Sec. \ref{sec:gamesemaccount}.

In the following we fix a simply typed term
$\Gamma \vdash M :T$ and we consider its computation tree $\tau(M)$ as defined in Def. \ref{dfn:comptree}.


\subsection{Pointers and justified sequences of nodes}

We define the \defname{enabling relation} on the set of nodes of the
computation tree as follows: $m$ enables $n$, written $m \vdash n$, if
and only if $n$ is bound by $m$ (we write $m \vdash_i n$ to precise
that $n$ is the $i^{\sf th}$ variable bound by $m$), or $m=r$ and $n$
is a free variable, or $n$ is a $\lambda$-node and $m$ is its parent
node.

For any set of nodes $S$ we write $S^{\upharpoonright r}$ for $\{ n
\in S \ | \ r \vdash^* n \}$ -- the subset of $S$ constituted of nodes
hereditarily enabled by $r$.  We call \defname{input-variables nodes}
the elements of $N_{var}^{\upharpoonright r}$.

A \defname{justified sequence of nodes} is a sequence of nodes with
pointers such that each variable or $\lambda$-node $n$ different
from the root has a pointer to a node $m$ occurring before it the
sequence such that $m \vdash n$.  We represent the pointer in the
sequence as follows \Pstr[0.4cm]{ (m){m} \ldots (n-m,45:i) n }.
  where the label indicates that either $n$ is labelled with the $i$th variable
abstracted by the $\lambda$-node $m$ or that $n$ is the $i^{\sf th}$
child of $m$.  Children nodes are numbered from $1$ onward except for
@-nodes where it starts from $0$. Abstracted variables are numbered
from $1$ onward. The $i^{\sf th}$ child of $n$ is denoted by $n.i$.

We say that a node $n_0$ of a justified sequence is
\defname{hereditarily justified} by $n_p$ if there are nodes $n_1,
\ldots, n_{p-1}$ in the sequence such that $n_i$ points to $n_{i+1}$
for all $i\in 0..p-1$.

The notion of \defname{P-view} $\pview{t}$ of a justified sequence of
nodes $t$ is defined the same way as the P-view of a justified
sequences of moves in Game Semantics:\footnote{ The equalities in the
  definition determine pointers implicitly. For instance in the second
  clause, if in the left-hand side, $n$ points to some node in $s$
  that is also present in $\pview{s}$ then in the right-hand side, $n$
  points to that occurrence of the node in $\pview{s}$.}
$$\begin{array}{rclrcl}
\pview{\epsilon} &=&  \epsilon
& \pview{\Pstr{ s \cdot (m) m \cdot \ldots \cdot (lmd-m,40){\lambda\overline{\xi}}
}}
 &=& \Pstr{
\pview{s} \cdot (m2) m \cdot (lm2-m2,50) {\lambda \overline{\xi}} } \\
\mbox{for $n \notin N_\lambda$, } \pview{s \cdot n }  &=&  \pview{s} \cdot n \qquad
& \pview{s \cdot r }  &=&  r
\end{array}$$

The O-view of $s$, written $\oview{s}$, is defined
dually. We borrow the game semantic terminology: A justified sequences of nodes satisfies
\emph{alternation} if for any two consecutive nodes one is a
$\lambda$-node and the other is not, and \emph{P-visibility} if every
variable node points to a node occurring in the P-view a that point.

\subsection{Computation tree with value-leaves}
Let us first fix some notations. We write $r$ for the root of $\tau(M)$ and $N$, $N_@$, $N_\lambda$ and
$N_{var}$ for the set of nodes, @-labelled nodes, $\lambda$-nodes and variable nodes respectively. For $n \in N$, $\kappa(n)$ denotes the subterm of $\elnf{M}$ corresponding to the subtree rooted at $n$, in particular $\kappa(r) = \elnf{M}$.  The \defname{type} of a variable node labelled $x$ is the type of $x$, the type of the root is $(A_1,\ldots,A_p, T)$ where $x_1:A_1,\ldots, x_p:A_p$ are the free variables of $M$ and the type of $n\in (N_\lambda \union N_@)
\setminus \{ r \}$ is the type of $\kappa(n)$.


We now add another ingredient to the computation tree that was not originally used in \cite{OngLics2006}.  We write $\mathcal{D}$ to denote the set
of values of the base type $o$.  We add \defname{value-leaves} to
$\tau(M)$ as follows: For each value $v \in \mathcal{D}$ and for each
node $n \in N$ we attach the child leaf $v_n$ to $n$.  We write $V$
for the set of nodes and leaves of the computation tree.  For $\$$
ranging in $\{@, \lambda, var \}$, we write $V_\$$ to denote the set
$N_\$ \union \{ v_n \ | \ n \in N_\$, v \in \mathcal{D} \}$.


Everything that we have defined can be lifted to this new version of
computation tree. A value-leaf has order $0$. The enabling relation
$\vdash$ is extended so that every leaf is enabled by its parent
node. A link going from a value-leaf $v_n$ to a node $n$ is labelled
by $v$: \Pstr[0.4cm]{ (n) n \ldots (vn-n,35:v){v_n} }. For the
definition of P-view and visibility, value-leaves are treated as
$\lambda$-nodes if they are at an odd level in the computation tree,
and as variable nodes if they are at an even level.

We say that a node $n$ is \defname{matched} by $v_n$
if there is an occurrence of $v_n$ in the sequence that points to $n$,
otherwise we say that $n$ is \defname{unmatched}.
The last unmatched node is called the \defname{pending node}.  A justified sequence of nodes is
\defname{well-bracketed} if each value-leaf occurring in it is justified by the pending node at that point.  If $t$ is a traversal then we write
$?(t)$ to denote the subsequence of $t$ consisting only of unmatched
nodes.

\subsection{Traversals of the computation tree}
\label{subsec:traversal}

A \emph{traversal} is a justified sequence of nodes of the computation tree where each node
indicates a step that is taken during the evaluation of the term.
\begin{definition}\rm
\label{def:traversal}
The set $\travset(M)$ of \defname{traversals} over $\tau(M)$ is defined by induction over the following rules.

\begin{description}
\item[{\bf (Empty)}] $\epsilon \in \travset(M)$.
\item[{\bf (Root)}] $ r \in \travset(M)$.
\item[{\bf (Lam)}] If $t \cdot \lambda \overline{\xi}$ is a traversal then so is
$t \cdot \lambda \overline{\xi} \cdot n$
where $n$ is $\lambda \overline{\xi}$'s child and if $n$ is a variable node then it points to the only\footnote{This is justified {\it a posteriori} by the fact that P-views are paths in the computation tree.} occurrence of its enabler that is still present in $\pview{t \cdot \lambda \overline{\xi}}$.

\item[{\bf (App)}] If $t \cdot @$ is a traversal then so is \Pstr[0.4cm]{t \cdot (m) @  \cdot (n-m,40:0) n}.

\item[{\bf (InputVar$^{val}$)}] If $t_1 \cdot x \cdot t_2$ is a traversal
with $x \in N_{var}^{\upharpoonright r}$ and $?(t_1 \cdot x \cdot
t_2)=?(t_1) \cdot x$ then so is \Pstr[0.4cm]{t_1 \cdot (x){x} \cdot
t_2 \cdot (xv-x,38:v){v_x} } for all $v \in \mathcal{D}$.

\item[{\bf (InputVar)}] If $t_1 \cdot x \cdot t_2$ is a traversal with
  $x \in N_{var}^{\upharpoonright r}$ and $?(t_1 \cdot x \cdot
  t_2)=?(t_1) \cdot x$ then so is $t_1 \cdot x \cdot t_2 \cdot n$ for any
  $\lambda$-node $n$ whose parent occurs in $\oview{t_1 \cdot x}$, $n$
  pointing to some occurrence of its parent node in $\oview{t_1 \cdot x}$.


\item[{\bf (Copycat$^@$)}]
  If \Pstr{t \cdot (app){@} \cdot (lz-app,60:0){\lambda \overline{z}}  \ldots  (lzv-lz,60:v){v}_{\lambda \overline{z}} } is a traversal then so is
\Pstr[0.6cm]{t \cdot (app){@} \cdot (lz-app,60){\lambda
\overline{z}} \ldots  (lzv-lz,60:v){v}_{\lambda \overline{z}} \cdot
(appv-app,45:v){v}_@}.

\item[{\bf (Copycat$^\lambda$)}] If \Pstr[0.4cm]{t \cdot \lambda \overline{\xi} \cdot (x){x}  \ldots   (xv-x,50:v){v}_x}
is a traversal then so is \Pstr[0.5cm]{t \cdot (lmd){\lambda
\overline{\xi}} \cdot (x){x}  \ldots  (xv-x,50:v){v}_x  \cdot
(lmdv-lmd,30:v){v}_{\lambda \overline{\xi}} }.

\item[{\bf (Copycat$^{var}$)}] If \Pstr[0.4cm]{t \cdot y \cdot (lmd){\lambda \overline{\xi}}
\ldots (lmdv-lmd,50:v){v}_{\lambda \overline{\xi}} } is a traversal
for some variable $y$ not in $N_{var}^{\upharpoonright r}$ then so
is \Pstr[0.7cm]{t \cdot (y){y} \cdot (lmd){\lambda \overline{\xi}}
\ldots (lmdv-lmd,30:v){v}_{\lambda \overline{\xi}}  \cdot
(vy-y,50:v){v}_y }.

\item[{\bf (Var)}]
If \Pstr[0.5cm]{t' \cdot (n){n} \cdot
    (lx){\lambda \overline{x}}  \ldots
    (x-lx,50:i){x_i} } is a traversal for some variable $x_i$ not in $N_{var}^{\upharpoonright r}$ then
so is
\Pstr[0.6cm]{
t' \cdot (n){n} \cdot
    (lx){\lambda \overline{x}}  \ldots
    (x-lx,30:i){x_i}  \cdot
    (letai-n,40:i){\lambda \overline{\eta_i}}
     }.
\end{description}
\end{definition}

\begin{wrapfigure}[7]{r}{3.8cm}
$
% The next line must be commented out if and only if the paragraph appears at the top of the page (otherwise the vertical position of the figure in the margin is not correct)
\rput[t](1.75cm,0.9cm)
{\tree[levelsep=3ex,treesep=0.5cm]{\lambda} {
    \tree{@}{
        \pstree[linestyle=dotted]{\TR{\lambda y}\arclabel{0} }{
            \tree{y}{
                \tree{\lambda \overline{\eta_1}}{\vdots}%\arclabel{1}
                \tree{\lambda \overline{\eta_i}}{\vdots}%\arclabel{i}
                \tree{\lambda \overline{\eta_n}}{\vdots}%\arclabel{n}
            }
        }
        \pstree[linestyle=dotted]{\TR{\lambda \overline{x}}
            \arclabel{1}}{ \tree{x_i}{\TR{} \TR{}}}
}}}
$
\end{wrapfigure}
A traversal always starts by visiting the root. Then it mainly
follows the structure of the tree. The (Var) rule permits to jump
across the computation tree. The idea is that after visiting a
variable node $x$, a jump is allowed to the node corresponding to
the subterm that would be substituted for $x$ if all the
$\beta$-redexes occurring in the term were reduced. The sequence
\Pstr[0.8cm]{\lambda \cdot (app) @  \cdot (ly) {\lambda y}  \ldots
(y-ly,35:1) y  \cdot (lx-app,38:1) {\lambda \overline{x}} \ldots
(x-lx,30:i) {x_i} \cdot (leta-y,40:i) {\lambda \overline{\eta_i} }
\ldots}
 is an example of traversal of the computation tree shown on the right.

\begin{proposition}[counterpart of proposition 6 from \cite{OngHoMchecking2006}]
\label{prop:pviewtrav_is_path}
Let $t$ be a traversal. Then
\begin{enumerate}[(i)]
\item $t$ is a well-defined and well-bracketed justified sequence.
\item $?(t)$ is a well-defined justified sequence verifying alternation, P-visibility and O-visibility.
\item $\pview{?(t)}$ is the path in $\tau(M)$ from $r$ to the last node in $?(t)$.
\end{enumerate}
\end{proposition}

The \defname{reduction} of a traversal $t$ written $ t \upharpoonright
r$, is the subsequence of $t$ obtained by keeping only the nodes that
are hereditarily justified by $r$. This has the effect of eliminating
the ``internal nodes'' of the computation.

Application nodes are used to connect the operator and the operand of
an application in the computation tree but since they do not play any
role in the computation of the term, we can remove them from the
traversals.  We write $t-@$ for the sequence of nodes-with-pointers
obtained by removing from $t$ all @-nodes and value-leaves of @-nodes,
any link pointing to an @-node being replaced by a link pointing to
the immediate predecessor of @ in $t$.

We introduce the two notations $\travset(M)^{-@} = \{ t - @ \ | \  t \in \travset(M) \}$ and $\travset(M)^{\upharpoonright r} = \{ t  \upharpoonright r \ | \  t  \in \travset(M) \}$.
\begin{remark}
Clearly if $M$ is $\beta$-normal then $\tau$ does not contain any @-node therefore all nodes are
hereditarily justified by $r$ and we have $\travset(M)^{-@} = \travset(M) = \travset(M)^{\upharpoonright r }$.
\end{remark}

\begin{lemma}[View of a traversal reduction]
\label{lem:redtrav_trav}
If $M$ is in $\beta$-normal form then for all $t\in \travset(M)$ we have $ \pview{?(t) \upharpoonright  r } = \pview{?(t)} \upharpoonright r$.
\end{lemma}
In the safe lambda calculus without interpreted constants this
lemma follows immediately from the fact that $\travset(M) =
\travset(M)^{\upharpoonright r }$. This remains valid in the presence of interpreted constants provided that the traversal rules
implementing the constants are \emph{well-behaved}.\footnote{A
traversal rule is \defname{well-behaved} if it can be stated under the
form ``$t = t_1\cdot n \cdot t_2 \in \travset \zand ?(t) = ?(t_1)
\cdot n \zand n \in N_{\Sigma}\union N_{var} \zand P(t) \zand m\in
S(t) \imp\ $\Pstr{ t_1\cdot (n){n} \cdot t_2 \cdot (m-n,25){m} \in \travset}'' for some expression $P$
expressing a condition on $t$ and function $S$ mapping traversals of the form of $t$ to a subset of the children of $n$.}

\subsection{Computation trees and arenas}
We consider the well-bracketed game model of the simply-typed
lambda calculus.  We choose to represent strategies using
``prefix-closed set of plays''.
\footnote{In the literature, a strategy is commonly defined as a set of plays
closed by taking a prefix of \emph{even} length. However for the purpose of showing the correspondence with traversals, the ``prefix-closed'' version is more appropriate.}
We fix a term $\Gamma \vdash M : T$ and write $\sem{\Gamma \vdash M : T}$ for its strategy denotation.
The answer moves of a question $q$ are written $v_q$ where $v$ ranges in $\mathcal{D}$.

\begin{definition}[Mapping from nodes to moves]\rm
\label{def:phi_psi mapping}
Let $q$ be a question move of $\sem{T}$ and $n \in N$ such that $n$
and $q$ are of type $(A_1,\ldots,A_p,o)$.  Let $\{ q^1, \ldots, q^p \}
\union \{ v_q \ | \ v \in \mathcal{D} \}$ be the set of moves enabled
by $q$ where each $q^i$ is of type $A_i$. The function $\psi_M^{n,q}$
from $V^{\upharpoonright n}$ to $\sem{T}$ is defined as:
\begin{eqnarray*}
\psi^{n,q}_M &=& \{ n \mapsto q \} \union  \{ v_n \mapsto v_q \ | \ v \in \mathcal{D} \}\\
 &&\union \left\{
                \begin{array}{ll}
                  \emptyset, & \hbox{if $p=0$\ ;} \\
                  \Union_{m \in N | n \vdash_i m} \psi_M^{m, q^i}, & \hbox{if $p\geq1$ and $n\in N_{\lambda}$\ ;} \\
                  \Union_{i=1..p} \psi_M^{n.i, q^i}, & \hbox{if $p\geq1$ and $n\in N_{var}$\ .}
                \end{array}
              \right.
\end{eqnarray*}
Note that $\psi_M^{n,q}$ is only defined on nodes hereditarily enabled
by $n$.  For any $n \in N$ let $A_n$ denote the type of
$\kappa(n)$. We write $\psi_{\kappa(n)}$ for $\psi_{\kappa(n)}^{n,q}$
where $q$ denotes the initial move of $\sem{A_n}$.\footnote{Arenas
involved in the game semantics of simply-typed lambda calculus are
trees.}


For a closed term $\vdash M : T$, the total function $\varphi_M$ from
$V_\lambda \union V_{var}$ to $\sem{T} \uplus \biguplus_{n \in N'_@}
\sem{A_n}$ is defined as $\varphi_M = \psi_M \union \Union_{n \in
N'_@} \psi_{\kappa(n)}$ where $N'_@$ denotes the set of children nodes
of @-nodes.  For an open term $x_1 : X_1, \ldots, x_n : X_n \vdash M :
T$, $\varphi_M$ is defined as $\varphi_{\lambda x_1 \ldots x_n
. M}$. When there is no ambiguity we omit the subscript in $\varphi_M$
and $\psi_M$.
\end{definition}
\begin{remark}
$\varphi$ maps $\lambda$-nodes to O-questions, variable nodes to
P-questions, value-leaves of $\lambda$-nodes to P-answers and
value-leaves of variable nodes to O-answers.
Moreover $\varphi$ maps nodes of a given order to moves of the same order.
\end{remark}
If $t = t_0 t_1 \ldots$ is a justified sequence
of nodes in $V_\lambda \union V_{var}$ then $\varphi(t)$ is defined
to be the sequence of moves $\varphi(t_0)\ \varphi(t_1) \ldots$
equipped with the pointers of $t$.

\begin{example}
Take $\lambda x . (\lambda g . g x) (\lambda y . y)$ with $x,y:o$ and $g:(o,o)$.
The diagram below represents the computation tree (middle), the arenas
$\sem{(o,o), o}$ (left), $\sem{o , o}$ (right), $\sem{o\rightarrow o}$ (rightmost)
and $\varphi = \psi \union \psi_{\lambda g.g x}^{\lambda g, q_{\lambda g}} \union
\psi_{\lambda y.y}^{\lambda y, q_{\lambda y}}$
(dashed-lines).
$$\psset{levelsep=3.5ex}
\pstree{\TR[name=root]{\lambda x}}
{
    \pstree{\TR[name=App]{@}}
    {
            \pstree{\TR[name=lg]{\lambda g}}
                { \pstree{\TR[name=lgg]{g}}{
                        \pstree{\TR[name=lgg1]{\lambda}}
                        { \TR[name=lgg1x]{x}  } } }
            \pstree{\TR[name=ly]{\lambda y}}
                    {\TR[name=lyy]{y}}
    }
}
\rput(4.5cm,-1cm){
  \pstree{\TR[name=A1lx]{q_{\lambda x}}}
        { \TR[name=A1x]{q_x} }
}
\rput(-6cm,-1.5cm){
    \pstree{\TR[name=A2lg]{q_{\lambda g}}}
    {
        \pstree{\TR[name=A2g]{q_g}}
        {  \TR[name=A2g1]{q_{g_1}}   }
    }}
\rput(2.5cm,-1.5cm){
    \pstree{\TR[name=A3ly]{q_{\lambda y}}}
        { \TR[name=A3y]{q_y}
        }
}
\psset{nodesep=1pt,arrows=->,arcangle=-20,arrowsize=2pt 1,linestyle=dashed,linewidth=0.3pt}
\ncline{->}{root}{A1lx} \mput*{\psi}
\ncarc{->}{lgg1x}{A1x}
\ncline{->}{lg}{A2lg} \mput*{\psi_{\lambda g.g x}^{\lambda g, q_{\lambda g}}}
\ncline{->}{lgg}{A2g}
\ncline{->}{lgg1}{A2g1}
\ncline{->}{ly}{A3ly} \mput*{\psi_{\lambda y.y}^{\lambda y, q_{\lambda y}}}
\ncline{->}{lyy}{A3y}
$$
\end{example}


\subsection{The Correspondence Theorem}

In game semantics, strategy composition is achieved by performing a
CSP-like ``composition + hiding''. If the internal moves are not
hidden then we obtain an alternative semantics called
\defname{revealed semantics} in \cite{willgreenlandthesis} and
\emph{interaction} semantics in \cite{DBLP:conf/sas/DimovskiGL05}.
The revealed semantics of a term $\Gamma \vdash M :T$, written
$\intersem{\Gamma \vdash M : T}$, is obtained by uncovering\footnote{An
  algorithm that uniquely recovers hidden moves is given in Part II of
  \cite{hylandong_pcf}.}  the internal moves from $\sem{\Gamma \vdash
  M : T}$ that are generated by the composition with the evaluation map
$ev$ at each @-node of the computation tree.  The inverse operation consists in filtering out the internal moves.

In the simply-typed lambda calculus, the set $\travset(M)$ of
traversals of the computation tree is
isomorphic to the set of uncovered plays of the strategy denotation
(this is the counterpart of the ``Path-Traversal Correspondence'' of
\cite{OngLics2006}). Moreover the set of traversal reductions is
isomorphic to the strategy denotation.

\begin{theorem}[The Correspondence Theorem]
\label{thm:correspondence}
\begin{eqnarray*}
(i)~\varphi_M  &: \travset(M)^{-@} \stackrel{\cong}{\longrightarrow} \intersem{\Gamma \vdash M :T} \\
(ii)~\psi_M  &: \travset(M )^{\upharpoonright r} \stackrel{\cong}{\longrightarrow} \sem{\Gamma \vdash M : T} \ .
\end{eqnarray*}
\end{theorem}

\begin{example}
Take $M = \lambda f z . (\lambda g x . f x)
(\lambda y. y) (f z) : ((o,o),o, o)$.  The figure below represents the computation tree (left tree), the arena $\sem{((o,o),o, o)}$ (right tree) and $\psi_M$ (dashed line). (Only question moves are shown for clarity.)  The justified sequence of nodes $t$ defined hereunder is an example of traversal:

\begin{tabular}{lp{6.3cm}}
$\tree[levelsep=2.5ex,treesep=0.3cm]{ \Rnode{root}{\lambda f z} }
     {  \tree{@}
        {   \tree{\lambda g x}{
                  \tree{\Rnode{f}{f^{[1]}}}{
                            \tree{\Rnode{lmd}{\lambda^{[2]}}}
                            {\TR{x}}
                  }
                }
            \tree{ \lambda y }{\TR{y}}
            \tree{\lambda ^{[3]}}{
                \tree{\Rnode{f2}{f^{[4]}}} {
                \tree{\Rnode{lmd2}{\lambda^{[5]}}}{\TR{\Rnode{z}{z}}}
                }
            }
        }
     }
\hspace{1cm}
  \tree[levelsep=8ex,treesep=0.3cm]{ \Rnode{q0}q^0 }
    {   \pstree[levelsep=4ex]{\TR{\Rnode{q1}{q^1}}}{\TR{\Rnode{q2}{q^2}}}
        \TR{\Rnode{q3}q^3}
        \TR{\Rnode{q4}q^4}
    }
\psset{nodesep=1pt,arrows=->,arrowsize=2pt 1,linestyle=dashed,linewidth=0.3pt}
\ncline{->}{root}{q0} \mput*{\psi_M}
\ncarc[arcangle=-25]{->}{z}{q3}
\ncarc[arcangle=10]{->}{f}{q1}
\ncarc[arcangle=10]{->}{lmd}{q2}
\ncline{->}{f2}{q1}
\ncline{->}{lmd2}{q2}$
\hspace{2cm}
&
\begin{asparablank}
  \item  \Pstr[0.8cm]{
t = (n){\lambda f z} \
(n2){@} \
(n3-n2,60){\lambda g x} \
(n4-n,45){f^{[1]}} \
(n5-n4,45){\lambda^{[2]}} \
(n6-n3,45){x} \
(n7-n2,35){\lambda^{[3]}} \
(n8-n,35){f^{[4]}} \
(n9-n8,45){\lambda^{[5]}} \
(n10-n,35){z}
}

\item \Pstr[0.9cm]{
t\upharpoonright r = (n){\lambda f z} \
(n4-n,50){f}^{[1]} \
(n5-n4,60){\lambda}^{[2]} \
(n8-n,45){f}^{[4]} \
(n9-n8,60){\lambda}^{[5]} \
(n10-n,40){z}}
\item
\Pstr[0.8cm]{ {\psi_M(t\upharpoonright r) =\ } (n){q^0}\
(n4-n,60){q^1}\ (n5-n4,60){q^2}\ (n8-n,45){q^1}\ (n9-n8,60){q^2}\
(n10-n,38){q^3} \in \sem{M}\ .}
\end{asparablank}
\end{tabular}
\end{example}
