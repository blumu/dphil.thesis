 In \cite{OngLics2006} the computation tree of a grammar is
defined as the unravelling of a finite graph representing the long
transform of a grammar. Similarly we define the computation tree of
a $\lambda$-term as an abstract syntax tree of its $\eta$-long
normal form.  We write $l\langle t_1, \ldots, t_n \rangle$ with $n
\geq 0$ to denote the tree with a root labelled $l$ with $n$
children subtrees $t_1$, \ldots, $t_n$. In the following, judgements
of the form $\Gamma \vdash M:T$ refer to simply-typed terms not
necessarily safe unless  mentioned.

\begin{definition}\rm
\label{dfn:comptree}
  The \defname{computation tree} $\tau(M)$ of a simply-typed term
  $\Gamma \vdash M:T$ with variable names in a countable set
  $\mathcal{V}$ is a tree with labels in $ \{ @ \} \union \mathcal{V}
  \union \{ \lambda x_1 \ldots x_n \ | \ x_1 ,\ldots, x_n \in
  \mathcal{V}, n\in\nat \}$ defined from its $\eta$-long form as follows. Suppose $\overline{x} = x_1 \ldots x_n$ for $n\geq 0$ then
\begin{eqnarray*}
  \mbox{for $m\geq 0$, $z \in \mathcal{V}$: } \tau(\lambda \overline{x} . z s_1 \ldots s_m : o) &=& \lambda \overline{x} \langle z \langle\tau(s_1),\ldots,\tau(s_m)\rangle\rangle \\
  \mbox{for $m \geq 1$: } \tau(\lambda \overline{x} . (\lambda y.t) s_1 \ldots s_m :o) &=& \lambda \overline{x} \langle @ \langle \tau(\lambda y.t),\tau(s_1),\ldots,\tau(s_m) \rangle \rangle \ .
\end{eqnarray*}
\end{definition}

Even-level nodes are $\lambda$-nodes (the root is on level 0). A
single $\lambda$-node can represent several consecutive variable
abstractions or it can just be a \emph{dummy lambda} if the
corresponding subterm is of ground type.  Odd-level nodes are
variable or application nodes.

The \defname{order} of a node $n$, written $\ord{n}$, is defined as
follows: @-nodes have order $0$. The order of a variable-node is the
type-order of the variable labelling it. The order of the root node
is the type-order of $(A_1,\ldots,A_p, T)$ where $A_1,\ldots, A_p$
are the types of the variables in the context $\Gamma$. Finally, the
order of a lambda node different from the root is the type-order of
the term represented by the sub-tree rooted at that node.

We say that a variable node $n$ labelled $x$ is \defname{bound} by a
node $m$, and $m$ is called the \defname{binder} of $n$, if $m$ is
the closest node in the path from $n$ to the root such that $m$ is
labelled $\lambda \overline{\xi}$ with $x\in \overline{\xi}$.
