\documentclass{lmcs}

%% read in additional TeX-packages or personal macros here:
%% e.g. \usepackage{xy}

%% define non-standard environments here, for example
\theoremstyle{plain}\newtheorem{satz}[thm]{Satz}
\def\eg{{\em e.g.}}
\def\cf{{\em cf.}}

%% due to the dependence on amsart.cls, \begin{document} has to occur
%% BEFORE the title and author information:
\begin{document}

\title[short title]{How to use lmcs.cls}

\author[Author1 et al.]{Author 1}	%required
\address{address 1}	%required
\email{author1@email1}  %optional
%\thanks{thanks 1, optional.}	%optional

\author[]{Author 2}	%optional
\address{address2; addresses should be duplicated when authors share an affiliation}	%optional
\email{author2@email2; ditto for email addresses}  %optional
\thanks{thanks 2, optional.}	%optional

\author[]{Author 3}	%optional
\address{address 3}	%optional
\email{author3@email3}  %optional
\thanks{thanks 3, optional.}	%optional

%% etc.

%% required for running head on odd and even pages, use suitable
%% abbreviations in case of long titles and many authors:

%% mandatory lists of keywords and classifications:
\keywords{MANDATORY list of keywords}
\subjclass{MANDATORY list of acm classifications}
\titlecomment{OPTIONAL comment concerning the title, \eg, if a variant
or an extended abstract of the paper has appeared elsewehere}
%%%%%%%%%%%%%%%%%%%%%%%%%%%%%%%%%%%%%%%%%%%%%%%%%%%%%%%%%%%%%%%%%%%%%%%%%%%

%% the abstract has to PRECEED the command \maketitle:
%% be sure not to issue the \maketitle command twice!

\begin{abstract}
  \noindent The abstract has to preceed the maketitle command.  Be
  sure not to issue the maketitle command twice!  Preferably, the
  abstact should consist of plain ASCII text, without mathematical
  expressions or \TeX-commands, including explicit references using
  the cite command.  Presently we are not able to automatically
  extract an abstract containing such data and relyably turn it into
  html code.  If you cannot meet these criteria, it is your
  responsibility to provide us with an html-version of your abstract.
  Please keep the abstract fairy short to prevent it from spilling
  over to the second page!
\end{abstract}

\maketitle

%% start the paper here:
\section*{Introduction}\label{S:one}

  Your article should start with an introduction.  This is the place
  to employ mathematical notation and give references, as opposed to
  the abstract.  It is up to the authors to decide, whether to assign
  a section number to he introduction or not.

\section{Multiple authors}

  In papers with multiple authors several points need to be mentioned.
  Do not worry about footnote signs that will link author $n$ to
  address $n$ and the optional thanks $n$.  This will be taken care of
  by the layout editor.  Even if authors share an affiliation and part
  of an email address, they should follow the strict scheme outlined
  above and list their data individually.  The layout editor will
  notice duplication of data and can then arrange for more
  space-efficient formatting.  Alternatively, Authors can write ``same
  data as Author n'' into some field to alert the layout editor.
  Unfortunately, so far we have not been able to devise a system that
  automatically takes care of these issues.  But once the layout
  editor is made aware of some duplication, he can take some fairly
  simple measures to adjust the format accordingly.  Placing the
  responsibility on the layout editor insures that these formatting
  issues are handled uniformly in different papers and that the
  authors do not have to second-guess the Journal's policy.

\section{Use of  Definitions and Theormes etc.}

  Let's define something.

\begin{defi}\label{D:first}
  Blah
\end{defi}

  Please use the supplied proclamation environments (as well as
  LaTeX's cross-referencing facilities), or extend them in the spirit
  of the given ones, if necessary (\cf, Satz \ref{T:big} below).
  Refrain from replacing our proclamation macros by your own
  constructs, especially do not change the numbering scheme: all
  proclamations are to be numbered consecutively!

\subsection{First Subsection}

  This is a test of subsectioning.  It works like numbering of
  paragraphs but is not linked with the numbering of theorems.

\begin{satz}\label{T:big}
  This is a sample for a proclamation environment that can be added
  along with your personal macros, in case the supplied environments
  are insufficient.
\end{satz}

\proof
  Trivial.  Please use the qed-command for the endo-of-proof box.  It
  will not be placed automatically, since that produces awkward output
  if, \eg, the end of the proclamation coincides with the end of a
  list environment.\qed

  Consequently, we have

\begin{cor}\label{C:big}
  Blah.  If no proof is given, an end-of-proof box should conclude a
  proclamation (Theorem, Proposition, Lemma, Corollary).\qed
\end{cor}

  When proclamations or proofs contain list environments (itemize,
  enumerate) without preceeding text, two possibilities exist:

\begin{thm}\label{T:m}\hfill  %% \hfill pushes the first item to a new line
\begin{itemize}
\item[(0)]
  Issuing an hfill-command before the beginning of the list
  environment will push the first item to a new line, like in this case.
\item[(1)]
  blah.
\end{itemize}
\end{thm}

\proof\hfill  %% \hfill pushes the first item to a new line
\begin{itemize}
\item[(0)]
  The same behaviour occurs in proofs; to start the first item on a
  new line an explicit hfill-command is necessary.
\item[(1)]
  blah, according to \cite{koslowski:mib}.\qed
\end{itemize}

  \noindent We strongly recommend using this variant since it produces
  rather orderly output.  The space-saving variant, in contrast, can
  look quite awful, \cf, Theorem \ref{T:en} below.  Please notice that
  this paragraph is not indented, since it is following a proclamation
  that ended with a list environment.  This can be achieved by
  starting the paragraph directly after the end of that environment,
  without inserting a blank line, or by explicit use of the
  noindent-command at the beginning of the paragraph.  The effect
  indentation may have after a list environment is demonstrated after
  the proof of Theorem \ref{T:en}.

\begin{thm}\label{T:en} %% without \hfill the first item is indented

\begin{enumerate}
\item
  Without the hfill-command the first item starts in the same line as
  the name for the proclamation.
\item
  This may be useful when space needs to be conserved, but not in an
  electronic journal.
\end{enumerate}
\end{thm}

\proof %% without \hfill the first item is indented
\begin{enumerate}
\item
  As you can see, the second option produces a somewhat unpleasant effect.
\item
  Hence we would urge authors to use the first variant.  Perhaps a
  \TeX-guru can help us to make that the default, without the need for
  the hfill-command.\qed
\end{enumerate}

  Here we started a new paragraph without suppressing ist
  indentation.  This adds to the rather disorienting appearance
  produced by not turning off the space-saving measures built into
  amsart.cls, on which this style is based.

\section*{Acknowledgement}
  The authors wish to acknowledge fruitful discussions with A and B.

%% in general the use of bibtex is encouraged

\begin{thebibliography}{Kos97}

\bibitem[Kos97]{koslowski:mib}
J{\"u}rgen Koslowski.
\newblock Monads and interpolads in bicategories.
\newblock {\em Theory Appl. Categ.}, 3(8):182--212, 1997.

\end{thebibliography}

\appendix
\section{}

  Doctors recommend to take out the appendix if pain starts to get
  life threatening.

\end{document}
