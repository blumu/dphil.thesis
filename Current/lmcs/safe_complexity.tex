\newcommand\bigo{\mathcal{O}} % big O notation
\newcommand\booltype{\mathbf{B}}

\section{Complexity of the Safe Lambda Calculus}
Here we study the problem of deciding beta-eta equivalence of two safe lambda terms.

\subsection{Statman's result}

A famous result by Statman  states that deciding the $\beta\eta$-equality of two first-order typable lambda terms is not elementary recursive \cite{Statman:1979:TLE}.
The idea of the proof is to encode the Henkin quantifier elimination of Type Theory into the simply-typed lambda calculus. The encoding relies on the fact that the function $\sf sg$ (conditional) can be encoded in the lambda-calculus. Hence the argument does not carry on   in the Safe Lambda Calculus since the conditional operator is not definable (\cite{blumong:safelambdacalculus}).

Mairson gave a simpler proof of Statman's theorem in \cite{mairson1992spt} which also proceeds by encoding the Henkin quantifier elimination procedure into the lambda-calculus but is much easier to understand as it makes use of list iteration to perform quantifier elimination.

It turns out that both encodings rely on the use of unsafe terms in order to implement the quantifier elimination procedure.

%Here we adapt Mairson's proof to produce a safe encoding of the quantifier elimination procedure, thus showing:
%\begin{theorem}
%The Safe Lambda Calculus is not elementary recursive.
%\end{theorem}

We recall the definition of the theory. Let $\mathcal{D}_0 = \{\mathbf{true},\mathbf{false}\}$ and $\mathcal{D}_{k+1} =powerset(\mathcal{D}_k)$.
For any $k\geq0$, we write $x^k$, $y^k$ and $z^k$ to denote variables ranging over $\mathcal{D}_k$. Prime formulas are $x^0$, $\mathbf{true}\in y^1$, $\mathbf{false}\in y^1$, and  $x^k \in y^{k+1}$. Formulae are built up from prime formulas using the logical connectives $\zand$,$\zor$,$\rightarrow$,$\neg$ and the quantifiers
$\forall$ and $\exists$. Meyer showed that deciding the truth of such formulae requires nonelementary time \cite{Meyer1974}.
\smallskip

In Mairson's encoding, all formula variables of a given order $k$ are encoded by terms of the same type $\Delta_k$. Using this encoding,
unsafety manifests itself in two different ways.
\begin{enumerate}[1.]
  \item
        First in the encoding of set membership. The prime formula $x^k \in y^{k+1}$ is encoded as \begin{equation} x^k : \Delta_k, y^{k+1}:\Delta_{k+1} \vdash y^{k+1} (\lambda y^k : \Delta_k . OR (eq_k~\underline{x^k}~y^k)~F : \Delta_k \typear \Delta_{k+1} \typear \Delta_0 \label{eqn:setmembership}\end{equation}
for some terms $OR$, $F$, $eq_k$.
This term is unsafe because of the underline occurrence of $x^k$ which is not abstracted together with $y^k$.

\item Secondly, quantifier elimination is performed by using a list iterator $\mathbf{D}_{k+1}$ which acts like the $fold\_left$ function from functional programming languages over the list of all elements of $\mathcal{D}_k$.
Thus for instance the formula $\forall x^0 . \exists y^0 . x^0 \zor y^0$
is encoded as $$\vdash \mathbf{D}_1 (\lambda x^0:\Delta_0. AND (\mathbf{D}_1 (\lambda y^0:\Delta_0. OR (\underline{x^0} \zor y^0)) F)) T$$ which is unsafe because of the underlined occurrence.

More generally, supposing that we find a way to encode set membership with a safe term, then the encoding of the formula will be safe if and only if for any variable $x$ in the formula, its binder is precisely the first quantifier $\exists z$ or $\forall z$ in the path to the root of the formula AST verifying $\ord{z} \geq \ord{x}$. For instance the formula $\forall x^k . \exists y^{k+1} . x^k \in y^{k+1}$ would be encoded by an unsafe term whereas the encoding of $\forall y^{k+1} . \exists x^k . x^k \in y^{k+1}$ would be safe.
\end{enumerate}

Surprisingly, the unsafety of the quantifier elimination procedure can be
easily overcome. The idea is as follows. We introduce multiple domains of representation for a given formula. An element of $\mathcal{D}_k$ is thereby represented by countably many terms of type $\Delta_k^n$ where $n\in\nat$ indicates the level of the representation. The type $\Delta_k^n$ is defined in such a way that its order strictly increases as $n$ grows. Moreover there exists a term that can reduce the level of representation of a given term. In the formula representation, each variable can now be encoded with a different level of representation. Since there are infinitely many levels, it is always possible to find an assignment of levels to variables such that the resulting encoding term is safe.

For set-membership, however, there is no obvious way to obtain a safe encoding. The set-member function from Eq.\ \ref{eqn:setmembership} can be turned into a safe term provided that we have access to a function permitting us to increase the representation level of term, but to our knowledge, such transformation cannot be expressed in the simply-typed lambda-calculus.



\subsubsection{Encoding basic boolean operations}

We assume a ground type $o$.
%For any type $\mu$ we define the type $\booltype_\mu \equiv (\mu\typear\mu)\typear \mu$.
%We abbreviate $\booltype_0$ into $\booltype$.
%We introduce the following hierarchy of types: $\sigma_0 \equiv o$, $\sigma_{n+1} \equiv \booltype_{\sigma_n}$ for $n\geq1$.
%Note that the order of $\sigma_n$ strictly increases as $n$ increases.
We introduce a parameterized type for encoding booleans defined by $\booltype_{-1} \equiv o$ and $\booltype_{n+1} \equiv \booltype_n\typear\booltype_n\typear\booltype_n$ for $n\geq0$.
We have $\ord{\booltype_n} = n+1$ for $n\geq-1$.


The representation of the truth values $\mathbf{true}$ and $\mathbf{false}$ will be parameterized by $n \in \nat$ as follows
\begin{align*}
  T^n &\equiv \lambda x^{\booltype_{n-1}} y^{\booltype_{n-1}} .x : \booltype_{n}\\
  F^n &\equiv \lambda x^{\booltype_{n-1}} y^{\booltype_{n-1}} .y : \booltype_{n}
\end{align*}
Clearly these terms as safe. Moreover the following relations hold for all $n$:
\begin{align*}
  T^{n+1}~T^n~F^n &\betared^*  T^n \\
  F^{n+1}~T^n~F^n &\betared^*  F^n
\end{align*}
Hence it is possible to lower the representation level of a term encoding a boolean value by applying the two terms $T^n$ and $F^n$ to it.
For $i\in\nat$, we define the function $\_ \downarrow_i$ that lowers the level-representation of a term, turning a term of type $\booltype_n$ for some $n\in\nat$ to a term of type $\booltype_{\min(i,l)}$:
$$ (M :\booltype_n)\downarrow_i = \left\{
  \begin{array}{ll}
    M~T^{n-1}~F^{n-1}~\ldots~T^{i+1}~F^{i+1}:\booltype_i, & \hbox{if $n>i$;} \\
M:\booltype_n, & \hbox{otherwise.}
  \end{array}
  \right.
$$


Boolean functions are encoded by the following level-parameterized terms:
\begin{align*}
AND^n &\equiv \lambda p : \booltype_n \lambda q : \booltype_n \lambda x:\booltype_{n-1} \lambda y:\booltype_{n-1} . p~(q~x~y)~y : \booltype_n\typear\booltype_n\typear\booltype_n \\
OR^n &\equiv \lambda p : \booltype_n \lambda q : \booltype_n \lambda x:\booltype_{n-1} \lambda y:\booltype_{n-1} . p~x~(q~x~y) : \booltype_n\typear\booltype_{n}\typear\booltype_n \\
NOT^n &\equiv \lambda p : \booltype_n \lambda q : \booltype_n \lambda x:\booltype_{n-1} \lambda y:\booltype_{n-1} . p~y~x : \booltype_n\typear\booltype_n\typear\booltype_n \\
IF^n &\equiv \lambda p : \booltype_n \lambda q : \booltype_n \lambda x:\booltype_{n-1} \lambda y:\booltype_{n-1} . OR^n (NOT^n p)~q : \booltype_n\typear\booltype_n\typear\booltype_n
\end{align*}
which are all safe terms.

\subsubsection{Coding elements of the type hierarchy}
For any $n\in\nat$ we define the hierarchy of type $\Delta_k^n$ as follows:
$\Delta_0^n \equiv \booltype_n$ and $\Delta_{k+1}^n \equiv {\Delta_k^n}^*$ where for any type $\alpha$, $\alpha^* = (\alpha \typear \tau \typear \tau)\typear \tau \typear \tau$.

An occurrence of a formula variable $x^k$ will be encoded as a term variable $x^k:\Delta_{k}^n$ for some level of representation $n\in\nat$.

Following Mairson's  proof, we encode the set $\mathcal{D}_0$ as the list $\mathbf{D}_0$ containing $\mathbf{true}$ and $\mathbf{false}$, and we parameterized this representation by $n\in \nat$:
$$\mathbf{D}_0^n \equiv \lambda c:\booltype_n \typear \tau \typear \tau . \lambda e : \tau . c~T^n~(c~F^n~e) : \Delta_1^n$$
and for $k\geq 0$, the higher-order set $\mathcal{D}_{k+1}$ is represented by the parameterized term:
$$\mathbf{D}_{k+1}^n \equiv powerset~\mathbf{D}_k^n : \Delta_{k+2}^n$$
where the term $powerset$ taken from \cite{mairson1992spt} is reproduced here:
\begin{align*}
  powerset &\equiv \lambda A^* :(\alpha \typear \alpha^{**} \typear \alpha^{**}) \typear \alpha^{**} \typear \alpha^{**}.\\
&\qquad  A^*~double~(\lambda c:\alpha^* \typear \tau\typear \tau.\lambda b:\tau . c ( \lambda c':\alpha\typear \tau\typear \tau. \lambda b':\tau.b') b)\\
powerset &: ((\alpha \typear \alpha^{**} \typear \alpha^{**}) \typear \alpha^{**} \typear \alpha^{**})\typear \alpha^{**}
\end{align*}
with
\begin{align*}
  double &\equiv \lambda x :\alpha.\lambda l : (\alpha^* \typear \tau\typear \tau)\typear \tau\typear \tau. \\
  & \qquad \lambda c:\alpha^*\typear \tau\typear \tau.\lambda b:\tau. \\
  & \qquad \qquad l(\lambda e:\alpha^*.c (\lambda c':\alpha\typear \tau\typear \tau.\lambda b':\tau.c'~x~(e~c'~b')))(l~c~b)\\
double &: \alpha \typear \alpha^{**} \typear \alpha^{**}
\end{align*}

It can be checked that these two terms are safe.

\subsubsection{Quantifier elimination}
Following \cite{mairson1992spt}, quantifier elimination interprets $\forall x^k.\Phi(x^k)$ as the iterated conjunction $\mathbf{D}_k(\lambda x^k:\Delta^k.AND(\hat\Phi~x^k))~T$ where $\hat\Phi$ is the interpretation of $\Phi$; similarly $\exists x^k.\Phi(x^k)$  is interpreted by the iterated disjunction $\mathbf{D}_k(\lambda x^k:\Delta^k.AND(\hat\Phi~x^k))~T$.

Let $x^{k_p}_p \ldots x^{k_1}_1$ for $p\geq1$ be the list of variables appearing in the formula. W.l.o.g.\ we can assume that they are given in the order of appearance of their binder in the formula \ie $x^{k_p}_p$ is bound by the leftmost binder. We assign representation levels to variables as follows. The right-most variable is assigned level $1$ \ie $x^{k_1}_1 : \Delta^1_{k_1}$; suppose that $x^{k_i}_i :\Delta^l_{k_l}$ for $1\leq i< p$ then the representation level of variable $x^{k_{i+1}}_{i+1}$ is defined as
the smallest $l'\in\nat$ such that $\ord{\Delta^{l'}_{k_i}} > \ord{\Delta^{l}_{k_{i-1}}}$.

This way, since variables that are bound first have higher order, the variables
 that are bound in the nested list-iterations (corresponding to the nested quantifiers in the formula) are necessarily safely bound.


\subsubsection{Coding set theory in the $\mathcal{D}_k$}
To complete the interpretation of prime formulas, we would need to show how to encode set membership. Unfortunately, this seems to be impossible in the safe lambda calculus. It would turn to be possible if we had at hand a function $\_ \uparrow^k$, counterpart of $\_ \downarrow_k$, that increases the representation level of a term to level $k$. Here is how we would proceed if such function were representable in the safe lambda calculus.

Firstly, the formulae ``$\mathbf{true} \in y^1$'' and ``$\mathbf{false} \in y^1$'' can be encoded by the safe terms $y^1 (\lambda x^0 . OR^0~x^0) F^0$ and $y^1 (\lambda x^0. OR^0(NOT^0~x^0)) F^0$ respectively.
For the general case ``$x^k\in y^{k+1}$''
we proceed as in \cite{mairson1992spt} by introducing lambda-terms encoding set equality, set membership and subset tests, and we further parameterize these encoding by $n\in\nat$.

Equality of booleans is encoded by:
$$ eq_0^n \equiv \lambda x^0 : \booltype_n .\lambda y^0 : \booltype_n. OR^n (AND^n~x^0~y^0) (AND^n (NOT^n~x^0)(NOT^n~y^0)) \ .$$

We now use variable of type $\Delta_{k+1}^n$ as iterators over list of elements of type $\Delta_k^n$ and we instantiate the type variable $\tau$ as $\booltype_n$ in order to iterate a level-$n$ Boolean function. We define the set membership function as follows. Note that
the level of representation differs from input to output: \begin{align*}
  member_{k+1}^{n+1} &\equiv \lambda x^k : \Delta_k^{n+1}.\lambda y^{k+1}:\Delta_{k+1}^{n+1}. \\
& \qquad (y^{k+1}\downarrow_n) (\lambda y^k : \Delta_k^n . OR^n (eq_k^{n+1}~x^k~(y^k\uparrow^{n+1})))~F^n \\
  & : \Delta_k^{n+1} \typear \Delta_{k+1}^{n+1} \typear \booltype_n
\\
  subset_{k+1}^{n+1} &\equiv \lambda x^{k+1} : \Delta_{k+1}^{n+1}.\lambda y^{k+1}:\Delta_{k+1}^{n+1}. \\
  & \qquad (x^{k+1}\downarrow_n) (\lambda x^k : \Delta_k^n . AND^n (member_{k+1}^{n+1}~x^k~y^{k+1}))~T^n \\
  & : \Delta_{k+1}^{n+1} \typear \Delta_{k+1}^{n+1} \typear\booltype_n
\\
  eq_{k+1}^{n+1} &\equiv \lambda x^{k+1} : \Delta_{k+1}^{n+1}.\lambda y^{k+1}:\Delta_{k+1}^{n+1}. \\
   & \qquad
   (\lambda op:\Delta_{k+1}^n\typear\Delta_{k+1}^n\typear\booltype_n. AND^n (op~x^{k+1}~y^{k+1})(op~y^{k+1}~x^{k+1}))~subset_{k+1}^{n+1} \\
  & : \Delta_{k+1}^{n+1} \typear \Delta_{k+1}^{n+1} \typear \booltype_n
\end{align*}
The terms $eq_{k+1}^n$ and $subset_{k+1}^n$ are safe, and so is $member_{k+1}^n$ thanks to the fact that $y^k$ has a lower representation level than $x^k$.

The formula $x^k\in y^{k+1}$ is then encoded by the term
$$x^k:\Delta_k^n, y^{k+1}:\Delta_{k+1}^{n'}\vdash \left(member_{k+1}^{\min(n,n')} (x^k\downarrow_{\min(n,n')})~(y^{k+1}\downarrow_{\min(n,n')})\right)\downarrow_0$$


\subsection{NP-hardness}
To show NP-hardness it suffices to observe that the encoding of SAT in the simply-typed lambda calculus from the paper\cite{asperti-np} relies only on safe terms.

\subsection{PSPACE-hardness}

We encode QBF into the calculus.
We assume that the quantified propositional formula is given in prenex form:
$$\$_{n-1} x_{n-1} \ldots \$_0 x_0 . \psi(x_0, \ldots, x_{n-1})$$
where $\$_i \in \{\exists,\forall\}$ for $0\leq i\leq n-1$.

The encoding is as follows:
\begin{align*}
\sem{1} &= T^0  : \booltype \\
\sem{0} &= F^0 : \booltype \\
\sem{x_i} &= x_i\downarrow_0 = x_i~T^{i-1}~F^{i-1}\ldots T^1~F^1: \booltype \qquad \hbox{where $x_i:\booltype_i$}\\
\sem{\psi_1\zand \psi_2} &= AND^0~\sem{\psi_1}~\sem{\psi_2}
:\booltype  \\
\sem{\psi_1\zor \psi_2} &= OR^0~\sem{\psi_1}~\sem{\psi_2}
:\booltype  \\
\sem{\neg \psi} &= NOT^0~\sem{\psi}
:\booltype  \\
\sem{\forall x_i.\psi(\ldots, x_i, \ldots)} & = \mathbf{D}_0^i(\lambda x^{\booltype_i} AND^0~\sem{\psi(\ldots, x_i, \ldots)})~T^0 :\booltype\\
\sem{\exists x_i.\psi(\ldots, x_i, \ldots)} & = \mathbf{D}_0^i(\lambda x^{\booltype_i}.OR^0~\sem{\psi(\ldots, x_i, \ldots)})~F^0 :\booltype
\end{align*}
The size of $\sem{\psi}$ is in $\bigo(|\psi|^2)$.

It is easy to check that this encoding is safe.
\begin{example}
  The formula $\forall x \exists y \exists z (x\zor y\zor z)\zand(\neg x\zor \neg y\zor \neg z)$ is represented by the safe term:
\begin{align*}
\vdash &\mathbf{D}_0^2(\lambda x^{\booltype_2}. AND^0\\
&\quad\quad (\mathbf{D}_0^1(\lambda x^{\booltype_1}.OR^0\\
&\quad\quad\quad (\mathbf{D}_0^0(\lambda x^{\booltype_0}.OR^0\\
&\quad\quad\quad\quad (AND^0 (OR^0(OR^0~(x~T^1 F^1 T^0 F^0)~(y~T^0 F^0))z) \\
&\quad\quad\quad\quad\quad (OR^0(OR^0(NEG^0 (x~T^1 F^1 T^0 F^0))(NEG^0 (y~T^0 F^0)))(NEG^0~z))) \\
&\quad\quad\quad )F^0)\\
&\quad\quad)F^0)\\
&\quad) T^0
\end{align*}
\end{example}
This gives us:
\begin{theorem}
  Deciding $\beta\eta$-equality of two terms of the Safe Lambda Calculus is PSPACE-hard.
\end{theorem}

% NP \subseteq PSPACE \subseteq EXP

