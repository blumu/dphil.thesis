\section{Expressivity}
\subsection{Numeric functions representable in the safe lambda
calculus}

Natural numbers can be encoded into the simply-typed lambda calculus
using the Church Numerals: each $n\in\nat$ is encoded into the term
$\encode{n} = \lambda s z. s^n z$ of type $I = ((o,o),o,o)$ where
$o$ is a ground type. In 1976 Schwichtenberg \cite{citeulike:622637}
showed the following:


\begin{theorem}[Schwichtenberg 1976]
The numeric functions representable by simply-typed $\lambda$-terms
of type $I\rightarrow \ldots \rightarrow I$ using the Church Numeral
encoding are exactly the multivariate polynomials \emph{extended
with the conditional function}.
\end{theorem}

If we restrict ourselves to safe terms, the representable functions
are exactly the multivariate polynomials:
\begin{theorem}
\label{thm:polychar} The functions representable by safe
$\lambda$-expressions of type $I\rightarrow \ldots \rightarrow I$
are exactly the multivariate polynomials.
\end{theorem}

\begin{corollary}
The conditional operator $C:I\rightarrow I\rightarrow I \rightarrow
I$ verifying  $C t y z \rightarrow_\beta y$  if $t \rightarrow_\beta
\encode{0}$ and $C t y z \rightarrow_\beta z$ if $t
\rightarrow_\beta \encode{n+1}$ is not definable in the safe
simply-typed lambda calculus.
\end{corollary}
\proof
  Natural numbers are encoded using Church Numerals: $\encode{n} =
  \lambda s z. s^n z$.  Addition: For $n,m \in \nat$, $\encode{n+m} =
  \lambda \alpha^{(o,o)} x^o . (\encode{n} \alpha) (\encode{m} \alpha
  x)$. Multiplication: $\encode{n . m} = \lambda \alpha^{(o,o)}
  . \encode{n} (\encode{m} \alpha)$.  All these terms are safe and
  clearly any multivariate polynomial $P(n_1, \ldots, n_k)$ can be
  computed by composing the addition and multiplication terms as
  appropriate.

For the converse, let $U$ be a safe $\lambda$-term of type
$I\rightarrow I\rightarrow I$.  The generalization to terms of type
$I^n \rightarrow I$ for $n>2$ is immediate (they correspond to
polynomials with $n$ variables). W.l.o.g we can assume that $U =
\lambda x y \alpha z. u$ where $u$ is a safe term of ground type in
$\beta$-normal form with $fv(u) \subseteq \{ x, y : I, z :o, \alpha
: o\rightarrow o \}$.

\emph{Notation:} Let $T$ be a set of terms of type $\tau \rightarrow
\tau$ and $T'$ be a set of terms of type $\tau$ then $T \cdot T'$
denotes the set of terms $\{ s s' : \tau \ | \ s \in T \wedge s' \in
T' \}$. We also define $T^k \cdot T'$ recursively as follows:  $T^0
\cdot T' = T'$ and for $k\geq 0$, $T^{k+1} \cdot T' = T \cdot (T^k
\cdot T')$ ({\it i.e.}~$T^k \cdot T'$ denotes $\{ s_1( \ldots (s_k
s'))  \ | \ s_1, \ldots, s_k \in T \wedge s' \in T' \}$). We define
$T^+\cdot T' = \Union_{k > 0} T^k \cdot T'$ and $T^*\cdot T' =
(T^+\cdot T') \union T'$. For two sets of terms $T$ and $T'$, we
write $T =_\beta T'$ to express that any term of $T$ is
$\beta$-convertible to some term $t'$ of $T'$ and reciprocally.

Let us write $\mathcal{N}^\tau$ for the set of $\beta$-normal terms
of type $\tau$ where $\tau$ ranges in $\{ o, o\rightarrow o, I \}$
and with free variables in $\{ x,y:I, z:o, \alpha:o\rightarrow o\}$.
We write $\mathcal{A}^\tau$ for the subset of $\mathcal{N}^\tau$
consisting of applications only ({\it i.e.}~not abstractions). Let
$B$ be the set of terms of type $(o,o)$ defined by $B = \{ \alpha \}
\union \{ \lambda a.b \ | \ b \in \{a,z\}, a \neq z \}$. It is easy
to see that the following equations hold:
\begin{eqnarray*}
\mathcal{A}^I &=& \{ x,y \} \\
\mathcal{N}^{(o,o)} &=& B \union \mathcal{A}^I \cdot
\mathcal{N}^{(o,o)} = (\mathcal{A}^I)^* \cdot B \\
\mathcal{A}^{(o,o)} &=& \{ \alpha \} \union (\mathcal{A}^I)^+ \cdot B \\
\mathcal{A}^o = \mathcal{N}^o &=& \{ z \} \union \mathcal{A}^{(o,o)} \cdot \mathcal{N}^o = (\mathcal{A}^{(o,o)})^* \cdot \{ z \}
\end{eqnarray*}
Hence $\mathcal{A}^o = \left( \{\alpha \} \union \{x,y\}^+ \cdot
\left( \{\alpha \} \union \{\lambda a.b \ | \ b \in \{a,z\}, a \neq
z \} \right) \right)^* \cdot \{ z \}$. Since $u$ is safe, it cannot
contain terms of the form $\lambda a . z$ with $a \neq z$ occurring
at an operand position, therefore since $u$ belongs to
$\mathcal{A}^o$ we have:
\begin{equation}
u \in \left( \{\alpha\} \union \{x,y\}^+ \cdot \{\alpha,
\underline{i} \} \right)^* \cdot \{ z \} \label{eqn:u}
\end{equation}
where $\underline{i}$ is the identity term of type $o\rightarrow o$.


We observe that $\encode{k} \underline{i} =_\beta \underline{i}$ for
all $k \in \nat$ and for $l\geq 1$, for all $k_1, \ldots k_l \in
\nat$, $\encode{k_1}\ldots \encode{k_l} \alpha =_\beta
\encode{k_1\times \ldots \times k_l} \alpha$. Hence for all $m,n \in
\nat$ we have:
\begin{equation}
\begin{array}{llr}
\{\encode{m},\encode{n}\}^+ \cdot \{\alpha, \underline{i} \} &=_\beta
\{ \underline{i} \} \union
\{ \encode{m^i n^j} \alpha \ |\ i+j \geq 1 \} \nonumber \\
&= \{ \encode{m^i n^j} \alpha \ |\ i,j \geq 0 \} & ( \mbox{since } \underline{i} = \encode{0} \alpha) \end{array}
\label{eqn:intermediate}
\end{equation}
therefore:
$$\begin{array}{llr}
u[\encode{m}, \encode{n}/x,y] &\in \left( \{ \alpha \} \union \{\encode{m},\encode{n}\}^+ \cdot \{\alpha, \underline{i} \} \right)^* \cdot \{ z \}  & \mbox{(by Eq.\ \ref{eqn:u})} \\
&=_\beta \left( \{\alpha \} \union \{ \encode{m^i n^j}
\alpha \ | \ i,j \geq 0 \} \right)^* \cdot \{ z \} & \mbox{(by Eq.\ \ref{eqn:intermediate})}  \\
&=_\beta \left\{ \encode{m^i n^j}
\alpha \ | \ i,j \geq 0 \right\}^* \cdot \{ z \} & \mbox{($\alpha z =_\beta \encode{1} \alpha z$)}.
\end{array}$$

Furthermore, for all $m,n,r,i,j\in \nat$ we have $\encode{m^i n^j}
\alpha (\alpha^r z) =_\beta \alpha^{r + m^i n^j} z$, hence
$u[\encode{m} \encode{n}/x,y] =_\beta \alpha^{p(m,n)} z$ where
$p(m,n) = \sum_{0\leq k \leq d} m^{i_k} n^{j_k}$ for some $i_k,j_k
\geq 0$, $k \in\{ 0,..,d \}$ and $d\geq 0$. Thus $U \encode{m}
\encode{n} =_\beta \encode{p(m,n)}$. \qed


For instance, the term $ C = \lambda F G H \alpha x . H (
\underline{\lambda y . G \alpha x} ) (F \alpha x)$ used by
Schwichtenberg \cite{citeulike:622637} to define the conditional
operator is unsafe since the underlined subterm is of order $1$,
occurs at an operand position and contains an occurrence of $x$ of
order $0$.
