\section{Expressivity}
\subsection{Numeric functions representable in the safe lambda
calculus}

Natural numbers can be encoded in the simply-typed lambda calculus
using the Church Numerals: each $n\in\nat$ is encoded as the term
$\encode{n} = \lambda s z. s^n z$ of type $I = ((o,o),o,o)$ where
$o$ is a ground type. In 1976 Schwichtenberg \cite{citeulike:622637}
showed the following:


\begin{theorem}[Schwichtenberg 1976]
The numeric functions representable by simply-typed $\lambda$-terms
of type $I\rightarrow \ldots \rightarrow I$ using the Church Numeral
encoding are exactly the multivariate polynomials \emph{extended
with the conditional function}.
\end{theorem}

If we restrict ourselves to safe terms, the representable functions
are exactly the multivariate polynomials:
\begin{theorem}
\label{thm:polychar} The functions representable by safe
$\lambda$-expressions of type $I\rightarrow \ldots \rightarrow I$
are exactly the multivariate polynomials.
\end{theorem}

\begin{corollary}
The conditional operator $C:I\rightarrow I\rightarrow I \rightarrow
I$ verifying:
$$
C~t~y~z \rightarrow_\beta \left\{
                            \begin{array}{ll}
                              y, & \hbox{if $t
\rightarrow_\beta \encode{0}$;} \\
                              z, & \hbox{if
$t \rightarrow_\beta \encode{n+1}$ .}
                            \end{array}
                          \right.
$$
is not definable in the simply-typed safe lambda calculus.
\end{corollary}
\proof
  Natural numbers are encoded using the Church Numerals: $\encode{n} =
  \lambda s z. s^n z$.  Addition: For $n,m \in \nat$, $\encode{n+m} =
  \lambda \alpha^{(o,o)} x^o . (\encode{n} \alpha) (\encode{m} \alpha
  x)$. Multiplication: $\encode{n . m} = \lambda \alpha^{(o,o)}
  . \encode{n} (\encode{m} \alpha)$.  All these terms are safe and
  clearly any multivariate polynomial $P(n_1, \ldots, n_k)$ can be
  computed by composing the addition and multiplication terms as
  appropriate.

For the converse, let $U$ be a safe $\lambda$-term of type
$I\rightarrow I\rightarrow I$.  The generalization to terms of type
$I^n \rightarrow I$ for $n>2$ is immediate (they correspond to
polynomials with $n$ variables). W.l.o.g we can assume that $U =
\lambda x y \alpha z. u$ where $u$ is a safe term of ground type in
$\beta$-normal form with $fv(u) \subseteq \{ x, y : I, z :o, \alpha
: o\rightarrow o \}$.

\emph{Notation:} Let $T$ be a set of terms of type $\tau \rightarrow
\tau$ and $T'$ be a set of terms of type $\tau$ then $T \cdot T'$
denotes the set of terms $\{ s s' : \tau \ | \ s \in T \wedge s' \in
T' \}$. We also define $T^k \cdot T'$ recursively as follows:  $T^0
\cdot T' = T'$ and for $k\geq 0$, $T^{k+1} \cdot T' = T \cdot (T^k
\cdot T')$ ({\it i.e.}~$T^k \cdot T'$ denotes $\{ s_1( \ldots (s_k
s'))  \ | \ s_1, \ldots, s_k \in T \wedge s' \in T' \}$). We define
$T^+\cdot T' = \Union_{k > 0} T^k \cdot T'$ and $T^*\cdot T' =
(T^+\cdot T') \union T'$. For two sets of terms $T$ and $T'$, we
write $T =_\beta T'$ to express that any term of $T$ is
$\beta$-convertible to some term $t'$ of $T'$ and reciprocally.

Let us write $\mathcal{N}^\tau$ for the set of $\beta$-normal terms
of type $\tau$ where $\tau$ ranges in $\{ o, o\rightarrow o, I \}$
and with free variables in $\{ x,y:I, z:o, \alpha:o\rightarrow o\}$.
We write $\mathcal{A}^\tau$ for the subset of $\mathcal{N}^\tau$
consisting of applications only ({\it i.e.}~not abstractions). Let
$B$ be the set of terms of type $(o,o)$ defined by $B = \{ \alpha \}
\union \{ \lambda a.b \ | \ b \in \{a,z\}, a \neq z \}$. It is easy
to see that the following equations hold:
\begin{eqnarray*}
\mathcal{A}^I &=& \{ x,y \} \\
\mathcal{N}^{(o,o)} &=& B \union \mathcal{A}^I \cdot
\mathcal{N}^{(o,o)} = (\mathcal{A}^I)^* \cdot B \\
\mathcal{A}^{(o,o)} &=& \{ \alpha \} \union (\mathcal{A}^I)^+ \cdot B \\
\mathcal{A}^o = \mathcal{N}^o &=& \{ z \} \union \mathcal{A}^{(o,o)} \cdot \mathcal{N}^o = (\mathcal{A}^{(o,o)})^* \cdot \{ z \}
\end{eqnarray*}
Hence $\mathcal{A}^o = \left( \{\alpha \} \union \{x,y\}^+ \cdot
\left( \{\alpha \} \union \{\lambda a.b \ | \ b \in \{a,z\}, a \neq
z \} \right) \right)^* \cdot \{ z \}$. Since $u$ is safe, it cannot
contain terms of the form $\lambda a . z$ with $a \neq z$ occurring
at an operand position, therefore since $u$ belongs to
$\mathcal{A}^o$ we have:
\begin{equation}
u \in \left( \{\alpha\} \union \{x,y\}^+ \cdot \{\alpha,
\underline{i} \} \right)^* \cdot \{ z \} \label{eqn:u}
\end{equation}
where $\underline{i}$ is the identity term of type $o\rightarrow o$.


We observe that $\encode{k} \underline{i} =_\beta \underline{i}$ for
all $k \in \nat$ and for $l\geq 1$, for all $k_1, \ldots k_l \in
\nat$, $\encode{k_1}\ldots \encode{k_l} \alpha =_\beta
\encode{k_1\times \ldots \times k_l} \alpha$. Hence for all $m,n \in
\nat$ we have:
\begin{equation}
\begin{array}{llr}
\{\encode{m},\encode{n}\}^+ \cdot \{\alpha, \underline{i} \} &=_\beta
\{ \underline{i} \} \union
\{ \encode{m^i n^j} \alpha \ |\ i+j \geq 1 \} \nonumber \\
&= \{ \encode{m^i n^j} \alpha \ |\ i,j \geq 0 \} & ( \mbox{since } \underline{i} = \encode{0} \alpha) \end{array}
\label{eqn:intermediate}
\end{equation}
therefore:
$$\begin{array}{llr}
u[\encode{m}, \encode{n}/x,y] &\in \left( \{ \alpha \} \union \{\encode{m},\encode{n}\}^+ \cdot \{\alpha, \underline{i} \} \right)^* \cdot \{ z \}  & \mbox{(by Eq.\ \ref{eqn:u})} \\
&=_\beta \left( \{\alpha \} \union \{ \encode{m^i n^j}
\alpha \ | \ i,j \geq 0 \} \right)^* \cdot \{ z \} & \mbox{(by Eq.\ \ref{eqn:intermediate})}  \\
&=_\beta \left\{ \encode{m^i n^j}
\alpha \ | \ i,j \geq 0 \right\}^* \cdot \{ z \} & \mbox{($\alpha z =_\beta \encode{1} \alpha z$)}.
\end{array}$$

Furthermore, for all $m,n,r,i,j\in \nat$ we have $\encode{m^i n^j}
\alpha (\alpha^r z) =_\beta \alpha^{r + m^i n^j} z$, hence
we have $u[\encode{m} \encode{n}/x,y] =_\beta \alpha^{p(m,n)} z$ where
$p(m,n) = \sum_{0\leq k \leq d} m^{i_k} n^{j_k}$ for some $i_k,j_k
\geq 0$, $k \in\{ 0,..,d \}$ and $d\geq 0$. Thus $U \encode{m}
\encode{n} =_\beta \encode{p(m,n)}$. \qed


For instance, the term $ C = \lambda F G H \alpha x . H (
\underline{\lambda y . G \alpha x} ) (F \alpha x)$ used by
Schwichtenberg \cite{citeulike:622637} to define the conditional
operator is unsafe since the underlined subterm, which is of order
$1$, occurs at an operand position and contains an occurrence of $x$
of order $0$.






\newcommand{\zaioncencode}{\underline} % Zaionc's notation for word encoding in the lambda calculus

\newcommand{\zaiwordtyp}{\mathbf{B}} % Zaionc's type for word encoded in the lambda calculus
\newcommand{\closedof}[1]{{\rm Cl}(#1)} % notation for the set of closed terms of a certain type

\newcommand{\openedof}[2]{{\rm Op}(#1,#2)} % notation for the set of opened terms M such that \fatlambda{M} in \openedof

\newcommand\wordnum[1]{\mathbf{#1}} % Zaionc's encoding of numbers as words
\newcommand\safedefset{$\lambda^{safe}${\rm def}}

\newcommand\fatlambda{\lambda\kern-0.7em\lambda}
\newcommand\wordapp{{\sf app}}
\newcommand\wordsub{{\sf sub}}


\subsection{Word functions definable in the safe lambda calculus.}
Schwichtenberg's result on numeric functions definable in the lambda
calculus was extended to richer structures: Zaionc studied the
problem for words functions, then functions over trees and
eventually the general case of functions over free algebras
\cite{DBLP:journals/tcs/Leivant93,DBLP:journals/apal/Zaionc91,702481,DBLP:journals/tcs/Zaionc87}.
In this section we consider the case of word functions expressible
in the safe lambda calculus.
\smallskip

\emph{Notations.} For any simple type $\tau$, we write
$\closedof{\tau}$ for the set of closed terms of type $\tau$. We
consider a binary alphabet $\Sigma = \{a,b\}$. The result naturally
extend to any alphabet. We consider the set $\Sigma^*$ of all words
over $\Sigma$. The empty words is denoted $\epsilon$. We write $|w|$
to denote the length of the word $w\in\Sigma^*$. For any $k\in \nat$
we write $\wordnum{k}$ to denote the word $a \ldots a$ with $k$
occurrences of $a$, so that $|\wordnum{k}| = k$. For any $n\geq 1$,
$k\geq 0$ we write $c(n,k)$ to denote the $n$-ary function
$(\Sigma^*)^n \rightarrow \Sigma^*$ mapping constantly on the word
$\wordnum{k}$. The function $\wordapp : (\Sigma^*)^2 \rightarrow
\Sigma^*$ is the usual concatenation function: $\wordapp(x,y)$ is
the word obtain by concatenating $x$ and $y$. The substitution
function $\wordsub : (\Sigma^*)^3 \rightarrow \Sigma^*$ is defined
as folllows: $\wordsub(x,y,z)$ is the word obtained from $x$ by
substituting the word $y$ for all occurrences of $a$ and $z$ for all
occurrences of $b$.

Take the type $\zaiwordtyp = (o\typear o)\typear ((o\typear
o)\typear (o\typear o))$ called the binary word type in
\cite{DBLP:journals/tcs/Zaionc87}. There is a 1-1 correspondence
between words over $\Sigma$ and closed term of type $\zaiwordtyp$:
the empty word $\epsilon$ is represented by $\lambda u v x.x$, and
if $w\in \Sigma^*$ is represented by a term $W \in
\closedof{\zaiwordtyp}$ then $a \cdot w$ is represented by $\lambda
u v x. u(W uvx)$ and $a \cdot w$ is represented by $\lambda u v x.
v(W uvx)$. The term representing the word $w$ is denoted by
$\zaioncencode{w}$. A closed term of type $\zaiwordtyp^n \typear
\zaiwordtyp$ is called a \defname{$\lambda$-word theoretic
function}. The function on words $h:(\Sigma^*)^n \rightarrow
\Sigma^*$ is \defname{represented} by the term $H \in
\closedof{\zaiwordtyp^n \typear \zaiwordtyp}$ if and only if for all
$x_1, \ldots, x_n \in \zaiwordtyp^*$, $H \zaioncencode{x_1} \ldots
\zaioncencode{x_n} = \zaioncencode{h x_1 \ldots x_n}$. \bigskip

It was shown in \cite{DBLP:journals/tcs/Zaionc87} that there is a
finite base of word functions such that any $\lambda$-definable word
function is some composition of functions from the base:
\begin{theorem}[Zaionc \cite{DBLP:journals/tcs/Zaionc87}]
The set of $\lambda$-definable word functions is the minimal set containing the following word functions and closed by compositions:
\begin{itemize}
  \item concatenation $\wordapp$;
  \item substitution $\wordsub$;
  \item extraction of the maximal prefix containing only a given letter;
  \item non-emptiness check:  returns $\wordnum{0}$ if the word is $\epsilon$ and $\wordnum{1}$ otherwise, as well as emptiness check;
  \item occurrence checking: returns $\wordnum{1}$ if the word contain an occurrence of a given letter and $\wordnum{0}$ otherwise;
  \item first-occurrence check:  test whether the word begins with a given letter;
  \item all the projections;
  \item all the constant functions.
\end{itemize}
\end{theorem}
The lambda terms representing the base functions are:
\begin{align*}
  {\rm APP} &= \lambda c d u v x.c u v(d u v x), & {\rm SUB} &= \lambda x d e u v x.c(\lambda y.d u v y)(\lambda y . e u v y) x, \\
  {\rm CUT}_a &= \lambda c u v x . c u (\lambda y.x) x, & {\rm CUT}_b &= \lambda c u v x . c (\lambda y.x) v x, \\
  {\rm SQ} &= \lambda c u v x . c (\lambda y.u x) (\lambda y.u x) x, & \overline{{\rm SQ}} &= \lambda c u v x . c (\lambda y.x) (\lambda y.x) (u x), \\
  {\rm BEG}_a &= \lambda c u v x . c (\lambda y.u x) (\lambda y.x) x, & {\rm BEG}_b &= \lambda c u v x . c (\lambda y.x) (\lambda y.u x) x, \\
  {\rm OCC}_a &= \lambda c u v x . c (\lambda y.u x) (\lambda y.y) x, & {\rm OCC}_b &= \lambda c u v x . c (\lambda y.y) (\lambda y.u x) x.
\end{align*}
where {\rm APP} represents concatenation, {\rm SUB} substitution,
{\rm SQ} and $\overline{{\rm SQ}}$ non-emptiness and emptiness checking, ${\rm BEG}_a$ and
${\rm BEG}_b$ first-occurrence test, and ${\rm OCC}_a$ and ${\rm OCC}_a$ occurrence test.

We observe that among these terms only {\rm APP} and {\rm SUB} are
safe. All the other terms are unsafe because they contain terms of
the form $ N (\lambda y .x)$ where $x$ and $y$ are of the same
order. It turns out that this constitutes a base of terms generating
all the functions definable in the safe lambda calculus as the
following theorem states:
\begin{theorem}
\label{thm:wordfunctions_safely_definable}
Let \safedefset\ denote the minimal set containing the following word functions and closed by compositions:
\begin{itemize}
  \item concatenation $\wordapp$;
  \item substitution $\wordsub$;
  \item all the projections;
  \item all the constant functions.
\end{itemize}
The set of word-functions definable in the safe lambda calculus is
precisely \safedefset.
\end{theorem}

The proof follows the same steps as Zaionc's proof.
The first direction is immediate: the terms {\rm APP} and {\rm SUB} are safe
and represent concatenation and substitution. Projections are represented by safe terms of the form $\lambda x_1 \ldots x_n . x_i$ for some $i\in\{1..n\}$, and constant
functions by $\lambda x_1 \ldots x_n . \zaioncencode{w}$ for some $w\in\Sigma^*$.
For composition, take a functions $g:(\Sigma^*)^n \rightarrow \Sigma^*$ represented by safe term $G\in \closedof{\zaiwordtyp^n \typear \zaiwordtyp}$ and functions $f_1,\ldots,f_n :
(\Sigma^*)^p \rightarrow \Sigma^*$ represented by
safe terms $F_1,\ldots F_n$ respectively then the function $$(x_1,\ldots,x_p) \mapsto g(f_1(x_1,\ldots,x_p),\ldots,f_n(x_1,\ldots,x_p))$$ is represented by the term
$\lambda c_1\ldots x_p. G (F_1 c_1 \ldots c_p)\ldots (F_n c_1 \ldots c_p)$ which is also safe.
\bigskip

To show the other directions we need to introduce some more definitions.
We will write $\openedof{n}{k}$ to denote the set of open terms
of the form:
$$c_1:\zaiwordtyp, \ldots c_n :\zaiwordtyp, u:(o,o), v:(o,o), x_{k-1}:o, \ldots, x_0 :o \vdash M : o \ .$$
Thus  we have the following equality up to alpha-conversion:
$$\closedof{\tau(n,k)} = \{ \lambda c_1^\zaiwordtyp \ldots c_n^\zaiwordtyp u^{(o,o)} v^{(o,o)} x_{k-1}^o \ldots x_0^o . M \ | \ M \in \openedof{n}{k}  \} \ .$$

We define the type $\tau(n,k)$ where $n, k\geq1$ as $(\zaiwordtyp^n,(o,o),(o,o),\overbrace{o,\ldots,o}^{k\hbox{ times}},o)$.
and we generalized the notion of representability to terms of type $\tau(n,k)$ as follows:
\begin{definition}[Function pair representation]
A closed term $T\in\closedof{\tau(n,k)}$ \defname{represents the pair of functions}
$(f,p)$ where $f:(\Sigma^*)^n \rightarrow \Sigma^*$ and $p:(\Sigma^*)^n \rightarrow \{\wordnum{0}, \ldots, \wordnum{k-1}\}$ if for all $w_1,\ldots,w_n\in\Sigma^*$
and for every $i\in \{0\ldots,k-1\}$ we have:
$$
T \zaioncencode{w_1} \ldots \zaioncencode{w_n} =_{\beta\eta} \lambda u v x_{k-1}\ldots x_0 . \zaioncencode{f(w_1,\ldots,w_n)} u v x_{|p(w_1,\ldots,w_n)|} \ .
$$
By extension we will say that an \emph{open} term $M$ from $\openedof{n}{k}$
represents the pair $(f,p)$
iif $M[\zaioncencode{w_1}\ldots \zaioncencode{w_n}/c_1\ldots c_n] =_{\beta\eta} \zaioncencode{f(w_1,\ldots,w_n)} u v x_{|p(w_1,\ldots,w_n)|}$.
\end{definition}

We will call \defname{safe pair} any pair of functions of the form
$(w,c(n,i))$ where $0\leq i\leq k-1$ and $w$ is an $n$-ary function
from \safedefset.

\begin{theorem}[Characterization of the representable pairs]
\label{thm:zaionc_pair_characterization_safe} The function pairs
representable in the safe lambda calculus are precisely the safe
pairs.
\end{theorem}

\begin{proof}
  (Soundness). Take a pair $(w,c(n,i))$ where
  $0\leq i\leq k-1$ and $w$ is an $n$-ary function from \safedefset.
  As observed earlier, all the functions from \safedefset\ are representable
  in the safe lambda calculus: let $\zaioncencode{w}$ be the representative of $w$.
  The pair $(w,c(n,i))$ is then represented by the term
  $ \lambda c_1 \ldots c_n u v x_{k-1} \ldots x_0 . \zaioncencode{w} c_1\ldots c_n u v x_i$.
\smallskip

(Completeness) It suffices to consider safe beta-normal terms from
$\openedof{n}{k}$ only. The result then immediately follows for any
closed safe beta-normal term in $\closedof{\tau(n,k)}$. The subset
of $\openedof{n}{k}$ constituted of $\beta$-normal terms is
generated by the following grammar (see
\cite{DBLP:journals/tcs/Zaionc87}):
\begin{eqnarray*}
  (\alpha_i^k) &R^k &\rightarrow\ x_i \\
  (\beta^k) && \quad|\  u R^k \\
  (\gamma^k) && \quad|\  v R^k \\
  (\delta^k_j) && \quad|\  c_j\ (\overbrace{\lambda z^k. R^{k+1}[z^k,x_0,\ldots, x_{k-1}/x_0,x_1, \ldots, x_k]}^{Q^k(R^{k+1})}) \\
  && \quad\  \quad \ (\lambda z^k. R^{k+1}[z^k,x_0,\ldots, x_{k-1}/x_0,x_1, \ldots, x_k]) \\
  && \quad\  \quad \ R^k
\end{eqnarray*}
for $k\geq 1$, $0\leq i< k$, $0\leq j\leq n$. The notation
$M[\ldots/\ldots]$ denotes the usual simultaneous substitution. The
name of each rule is given in parenthesis. The non-terminals are
$R^k$ for $k\geq1$ and the set of terminals is $\{ z^k, \lambda z^k
\ |\ k\geq 1\} \union \{ x_i ~| i\geq 0 \} \union \{ c_1, \ldots,
c_n, u, v \}$.

We identify a rule name with the right-hand side of the
corresponding rule, thus $\alpha_i^k$ belongs to $\openedof{n}{k}$,
$\beta^k$ and $\gamma^k$ are functions from $\openedof{n}{k}$ to
$\openedof{n}{k}$, and $\delta^k_j$ is a function from
$\openedof{n}{k+1} \times \openedof{n}{k+1} \times \openedof{n}{k}$
to $\openedof{n}{k}$.

We now want to characterize the subset consisted of all \emph{safe}
terms generated by this grammar. The term $\alpha_i^k$ is always
safe, $\beta^k(M)$ and $\gamma^k(M)$ are safe if and only if $M$ is,
and  $\delta^k_j(F,G,H)$ is safe if and only if $Q^k(F)$, $Q^k(G)$
and $H$ are safe. We observe that the free variables of $Q^k(F)$ all
belong to $\{ c_1, \ldots c_n, u, v, x_0,\ldots x_{k}\}$. All these
variables have order greater than $\ord{z}$ except the $x_i$s which
have same order as $z$. Hence since the $x_i$s are not abstracted
together with $z$ we have that $Q^k(F)$ is safe if and only if $F$
is safe and the variables $x_0\ldots x_k$ do not appear free in
$F[z^k,x_0,\ldots, x_{k-1}/x_0,x_1, \ldots, x_k]$, which is the same
as saying that the variables $x_1\ldots x_k$ do not appear free in
$F$. Similarly, $Q^k(G)$ is safe if and only if $G$ is safe and
variables $x_1\ldots x_k$ do not appear free in $G$.

We therefore need to identify the subclass of terms generated by the non-terminal $R^k$ which are safe and which do not have free occurrences of variables in $\{x_1 \ldots x_{k-1}\}$. By applying this requirement to the rules of the previous grammar we obtain the following specialized grammar characterizing the desired subclass:
\begin{eqnarray*}
  (\overline\alpha_0^k) &\overline R^k &\rightarrow\ x_0 \\
  (\overline\beta^k) && \quad|\  u \overline R^k \\
  (\overline\gamma^k) && \quad|\  v \overline R^k  \\
  (\overline\delta^k_j) && \quad|\  c_j\ (\lambda z^k. \overline R^{k+1}[z^k/x_0]) \ (\lambda z^k. \overline R^{k+1} [z^k/x_0]) \ \overline R^k \ .
\end{eqnarray*}
For any term $M$, $Q^k(M)$ is safe if and only if $M$ can be
generated from the non-terminal $\overline R^k$. Thus the subset of
$\closedof{\tau(n,k)}$ consisting of safe beta-normal terms is given
by the grammar:
\begin{eqnarray*}
  (\widetilde\pi^k) &\widetilde S &\rightarrow \lambda c_1 \ldots c_n u v x_{k-1} \ldots x_0 . \widetilde R^k \\
  (\widetilde\alpha_i^k) &\widetilde R^k &\rightarrow\ x_i \\
  (\widetilde\beta^k) && \quad|\  u \widetilde R^k \\
  (\widetilde\gamma^k) && \quad|\  v \widetilde R^k \\
  (\widetilde\delta^k_j) && \quad|\  c_j\ (\lambda z^k. \overline{R^{k+1}}[z^k/x_0]) \ (\lambda z^k. \overline{R^{k+1}}[z^k/x_0]) \ \widetilde R^k \ .
\end{eqnarray*}

Thus to conclude the proof, it suffices to show that every term that
can be generated by this grammar starting with the non-terminal
$\widetilde S$ represents a safe pair.

We proceed by induction and show that the non-terminal $\overline
R^k$ generates terms representing pairs of the form $(w,c(n,0))$
while non-terminals $\widetilde S$ and $\widetilde R^k$ generate
terms representing pairs of the form $(w,c(n,i))$ for $0 \leq i<k$
and $w \in$\safedefset.

\emph{Base case:} The terms $\overline\alpha_0^k$ and
$\widetilde\alpha_i^k$ represents the pairs $(c(n,0),c(n,0))$ and
$(c(n,0),c(n,i))$ respectively. \emph{Step case:} Suppose $T\in
\openedof{n}{k}$ represents
 a pair $(w,p)$.  Then $\overline\alpha^k(T)$ and
 $\widetilde\alpha^k(T)$ represent the pair
 $(\wordapp(a,w),p)$, $\overline\beta^k(T)$ and
 $\widetilde\beta^k(T)$ represent the pair
 $(\wordapp(b,w),p)$, and $\overline\pi^k(T) \in \closedof{\tau(n,k)}$ represents the pair $(w,p)$. Now suppose that $E$, $F$ and $G$ represent the pairs
 $(w_e,c(n,0))$, $(w_f,c(n,0))$ and $(w_g,c(n,i))$ respectively.
 Then we have:
 \begin{align*}
   \widetilde \delta^k_j (E,F,G) &[\zaioncencode{w_1}\ldots \zaioncencode{w_n}/c_1\ldots c_n] \\
   &= \zaioncencode{w_j}\  (\lambda z^k. E[z^k/x_0])[\zaioncencode{w_1}\ldots \zaioncencode{w_n}/c_1\ldots c_n] \\
       & \qquad\quad (\lambda z^k. F[z^k/x_0])[\zaioncencode{w_1}\ldots \zaioncencode{w_n}/c_1\ldots c_n] \\
       & \qquad\quad  G[\zaioncencode{w_1}\ldots \zaioncencode{w_n}/c_1\ldots c_n] \\
   &=_{\beta\eta} \zaioncencode{w_j}\  (\lambda z^k. E[\zaioncencode{w_1}\ldots \zaioncencode{w_n}/c_1\ldots c_n][z^k/x_0]) \\
       & \qquad\qquad (\lambda z^k. F[\zaioncencode{w_1}\ldots \zaioncencode{w_n}/c_1\ldots c_n][z^k/x_0]) \\
       & \qquad\qquad  (\zaioncencode{w_g(w_1\ldots w_n)}~u~v~x_i) \hspace{4cm}\mbox{$G$ represents $(h,c(n,i))$}\\
   &=_{\beta\eta} \zaioncencode{w_j}\  (\lambda z^k. (\zaioncencode{w_e(w_1\ldots w_n)}~u~v~x_0)[z^k/x_0]) \hspace{2cm}\mbox{$E$ represents $(f,c(n,0))$} \\
       & \qquad\qquad (\lambda z^k. (\zaioncencode{w_f(w_1\ldots w_n)}~u~v~x_0)[z^k/x_0]) \hspace{1.8cm}\mbox{$F$ represents $(g,c(n,0))$} \\
       & \qquad\qquad  (\zaioncencode{w_g(w_1\ldots w_n)}~u~v~x_i)\\
%
   &=_{\beta\eta} \zaioncencode{w_j}\  (\lambda z^k. \zaioncencode{w_e(w_1\ldots w_n)}~u~v~z^k) \\
       & \qquad\qquad (\lambda z^k. \zaioncencode{w_f(w_1\ldots w_n)}~u~v~z^k) \\
       & \qquad\qquad (\zaioncencode{w_g(w_1\ldots w_n)}~u~v~x_i)\\
%
   &=_\eta \zaioncencode{w_j}\  (\zaioncencode{w_e(w_1\ldots w_n)}~u~v)  \ (\zaioncencode{w_f(w_1\ldots w_n)}~u~v) \  (\zaioncencode{w_g(w_1\ldots w_n)}~u~v~x_i)\\
%
   &=_{\beta\eta}  \zaioncencode{w}~u~v~ x_i
 \end{align*}
where the word-function $w$ is defined as
$$w: w_1,\ldots,w_n \mapsto \wordapp(\wordsub(w_j,w_e(w_1,\ldots,w_n),w_f(w_1,\ldots,w_n)),w_g(x_1,\ldots,w_n)) \ .$$
  Hence $\widetilde \delta^k_j (E,F,G)$ represents the pair $(w,c(n,i))$.

  The same argument shows that if $E$, $F$ and $G$ all represent safe pairs
then so does $\overline \delta^k_j (E,F,G)$.
\end{proof}


By setting $k=1$ in Theorem \ref{thm:zaionc_pair_characterization_safe} we obtain that every safe term in $T \in \closedof{\tau(n,k)}$ represents some function from \safedefset. This concludes the proof of the characterization Theorem \ref{thm:wordfunctions_safely_definable}.


\subsection{Representability of functions over other structures}\hfill

There is an isomorphism between binary trees and closed terms of
type $\tau =(o\typear o\typear o) \typear o \typear o$. Thus any
closed term of type $\tau\typear\tau \typear \ldots \typear \tau $
represents an $n$-ary function over trees. Zaionc gave a
characterization of the set of tree functions representable in the
simply-typed lambda calculus \cite{DBLP:conf/aluacs/Zaionc88}: it is
precisely the minimal set containing constant functions, projections
and closed under composition and limited primitive recursion. Zaionc
showed that the same characterization holds for the general case of
functions expressed over free algebras
\cite{DBLP:journals/apal/Zaionc91} (they are again given by the
minimal set containing constant functions, projections and closed
under composition and limited recursion). This result subsumes
Schwichtenberg's result on definable numeric functions as well as
Zaionc's own results on definable word and tree functions.

Among all these basic operations, only limited recursion is unsafe.
We conjecture that reciprocally the set of tree functions
representable in the safe lambda calculus is given by the set
containing constant functions, projections and closed under
composition (but not by limited recursion).
