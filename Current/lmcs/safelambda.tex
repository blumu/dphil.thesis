\section{Introduction}

\subsection*{Background}

The \emph{safety condition} was introduced by Knapik, Niwi{\'n}ski and
Urzyczyn at FoSSaCS 2002 \cite{KNU02} in a seminal study of the
algorithmics of infinite trees generated by higher-order grammars. The
idea, however, goes back some twenty years to Damm \cite{Dam82} who
introduced an essentially equivalent\footnote{See de Miranda's
 thesis \cite{demirandathesis} for a proof.} syntactic
restriction (for generators of word languages) in the form of
\emph{derived types}.
% Level-$n$ tree grammars as defined by Damm correspond exactly to a
% subset of safe level-$n$ grammars -- namely the safe complete grammars
% -- and every safe grammar corresponds to a safe complete one.
A higher-order grammar (that is assumed to be \emph{homogeneously
  typed}) is said to be \emph{safe} if it obeys certain syntactic
conditions that constrain the occurrences of variables in the
production (or rewrite) rules according to their type-theoretic
order. Though the formal definition of safety is somewhat intricate,
the condition itself is manifestly important. As we survey in the
following, higher-order \emph{safe} grammars capture fundamental
structures in computation, offer clear algorithmic advantages, and
lend themselves to a number of compelling characterizations:

\begin{itemize}
\item \emph{Word languages}. Damm and Goerdt \cite{DG86} have shown
  that the word languages generated by order-$n$ \emph{safe} grammars
  form an infinite hierarchy as $n$ varies over the natural numbers.
  The hierarchy gives an attractive classification of the
  semi-decidable languages: Levels 0, 1 and 2 of the hierarchy are
  respectively the regular, context-free, and indexed languages (in
  the sense of Aho \cite{Aho68}), although little is known about
  higher orders.

  Remarkably, for generating word languages, order-$n$ \emph{safe}
  grammars are equivalent to order-$n$ pushdown automata \cite{DG86},
  which are in turn equivalent to order-$n$ indexed grammars
  \cite{Mas74,Mas76}.

\item \emph{Trees}. Knapik \emph{et al.} have shown that the Monadic
  Second Order (MSO) theories of trees generated by \emph{safe}
  (deterministic) grammars of every finite order are
  decidable\footnote{It has recently been shown
    \cite{OngLics2006} that trees generated by \emph{unsafe}
    deterministic grammars (of every finite order) also have decidable
    MSO theories. More precisely, the MSO theory of trees generated by order-$n$
recursion schemes is $n$-EXPTIME complete.}.

  They have also generalized the equi-expressivity result due to Damm
  and Goerdt \cite{DG86} to an equivalence result with respect to
  generating trees: A ranked tree is generated by an order-$n$ \emph{safe}
  grammar if and only if it is generated by an order-$n$ pushdown
  automaton.

\item \emph{Graphs}. Caucal \cite{Cau02} has shown that the MSO
  theories of graphs generated\footnote{These are precisely the
    configuration graphs of higher-order pushdown systems.} by
  \emph{safe} grammars of every finite order are decidable. In a recent preprint \cite{hague-sto07}, however,
  Hague \emph{et al.} have
  shown that the MSO theories of graphs generated by order-$n$
  \emph{unsafe} grammars are undecidable, but deciding their modal
  mu-calculus theories is $n$-EXPTIME complete.
\end{itemize}

\subsection*{Overview}

In this paper, we aim to understand the safety condition in the
setting of the lambda calculus. Our first task is to transpose it to
the lambda calculus and pin it down as an appropriate sub-system of
the simply-typed theory. A first version of the \emph{safe lambda
  calculus} has appeared in an unpublished technical report
\cite{safety-mirlong2004}. Here we propose a more general and cleaner
version where terms are no longer required to be homogeneously typed
(see Section~\ref{sec:safe} for a definition). The formation rules of
the calculus are designed to maintain a simple invariant: Variables
that occur free in a safe $\lambda$-term have orders no smaller than
that of the term itself.  We can now explain the sense in which the
safe lambda calculus is safe by establishing its salient property: No
variable capture can ever occur when substituting a safe term into
another. In other words, in the safe lambda calculus, it is
\emph{safe} to use capture-\emph{permitting} substitution when
performing $\beta$-reduction.


There is no need for new names when computing $\beta$-reductions of
safe $\lambda$-terms, because one can safely ``reuse'' variable names
in the input term. Safe lambda calculus is thus cheaper to compute in
this na\"ive sense. Intuitively one would expect the safety constraint
to lower the expressivity of the simply-typed lambda calculus. Our
next contribution is to give a precise measure of the expressivity
deficit of the safe lambda calculus. An old result of Schwichtenberg
\cite{citeulike:622637} says that the numeric functions representable
in the simply-typed lambda calculus are exactly the multivariate
polynomials \emph{extended with the conditional function}.  In the
same vein, we show that the numeric functions representable in the
safe lambda calculus are exactly the multivariate polynomials.

Our last contribution is to give a game-semantic account of the safe
lambda calculus.
% Not much is known about the safe $\lambda$-calculus, and many problems
% remain to be studied concerning its computational power, the
% complexity classes that it characterizes, its interpretation under the
% Curry-Howard isomorphism and its game-semantic characterization. This
% paper is a contribution to the last problem.
%
% The difficulty in giving a game-semantic account of safety lies in the
% fact that it is a syntactic restriction whereas game semantics is
% syntax-independent. The solution consists in finding a particular
% syntactic representation of terms on which the plays of the game
% denotation can be represented.  To achieve this, we use ideas recently
% introduced by the second author \cite{OngLics2006}: a term is
% canonically represented by a certain abstract syntax tree of its
% $\eta$-long normal form referred as the \emph{computation tree}. This
% abstract syntax tree is specially designed to establish a
% correspondence with the game arena of the term. A computation is
% described by a justified sequence of nodes of the computation tree
% respecting some formation rules and called a
% \emph{traversal}. Traversals permit us to model $\beta$-reductions
% without altering the structure of the computation tree via
% substitution. A notable property is that \emph{P-views} (in the
% game-semantic sense) of traversals corresponds to paths in the
% computation tree.  We show that traversals are just representations of
% the uncovering of plays of the game-semantic denotation. We then
% define a \emph{reduction} operation which eliminates traversal nodes
% that are ``internal'' to the computation, this implements the
% counterpart of the hiding operation of game semantics. Thus, we obtain
% an isomorphism between the strategy denotation of a term and the set
% of reductions of traversals of its computation tree.
Using a correspondence result relating the game semantics of a
$\lambda$-term $M$ to a set of \emph{traversals} \cite{OngLics2006}
over a certain abstract syntax tree of the $\eta$-long form of $M$
(called \emph{computation tree}), we show that safe terms are denoted
by \emph{P-incrementally justified strategies}. In such a strategy,
pointers emanating from the P-moves of a play are uniquely
reconstructible from the underlying sequence of moves and the pointers
associated to the O-moves therein: Specifically, a P-question always
points to the last pending O-question (in the P-view) of a greater
order. Consequently pointers in the game semantics of safe
$\lambda$-terms are only necessary from order 4 onwards. Finally we
prove that a $\eta$-long $\beta$-normal $\lambda$-term is \emph{safe}
if and only if its strategy denotation is (innocent and)
\emph{P-incrementally justified}.



% \subsection*{Related work}

% \noindent\emph{The safety condition for higher-order grammars}

% \smallskip

% \noindent We have mentioned the result of Knapik \emph{et al.}~\cite{KNU02} that
% infinite trees generated by \emph{safe} higher-order grammars have
% decidable MSO theories.  A natural question to ask is whether the
% \emph{safety condition} is really necessary.  This has then been
% partially answered by Aehlig \emph{et al.}
% \cite{DBLP:conf/tlca/AehligMO05} where it was shown that safety is not
% a requirement at level $2$ to guarantee MSO decidability. Also, for
% the restricted case of word languages, the same authors have shown
% \cite{DBLP:conf/fossacs/AehligMO05} that level $2$ safe higher-order
% grammars are as powerful as (non-deterministic) unsafe ones.  De
% Miranda's thesis \cite{demirandathesis} proposes a unified framework
% for the study of higher-order grammars and gives a detailed analysis
% of the safety constraint at level 2.

% More recently, one of us obtained a more general result and showed
% that the MSO theory of infinite trees generated by higher-order
% grammars of any level, \emph{whether safe or not}, is decidable
% \cite{OngLics2006}.  Using an argument based on innocent
% game-semantics, he establishes a correspondence between the tree
% generated by a higher-order grammar called \emph{value tree} and a
% certain regular tree called \emph{computation tree}. Paths in the
% value tree correspond to traversals in the computation tree.
% Decidability is then obtain by reducing the problem to the acceptance
% of the (annotated) computation tree by a certain alternating parity
% tree automaton.  The approach that we follow in
% Sec. \ref{sec:correspondence} uses many ingredients introduced in this
% paper.


% The equivalence of \emph{safe} higher-order grammars and higher-order
% deterministic push-down automata for the purpose of generating
% infinite trees \cite{KNU02} has its counterpart in the general (not
% necessarily safe) case: the forthcoming paper \cite{hague-sto07}
% establishes the equivalence of order-$n$ higher-order grammars and
% order-$n$ \emph{collapsible pushdown automata}. Those automata form a
% new kind of pushdown systems in which every stack symbol has a link to
% a stack situated somewhere below it and with an additional stack
% operation whose effect is to ``collapse'' a stack $s$ to the state
% indicated by the link from the top stack symbol.

% \medskip

% \noindent\emph{Computation trees and traversals}

% \smallskip

% \noindent In \cite{DBLP:conf/lics/AspertiDLR94}, a notion of graph
% based on Lamping's graphs \cite{lamping} is introduced to represent
% $\lambda$-terms. The authors unify different notions of paths
% (regular, legal, consistent and persistent paths) that have appeared
% in the literature as ways to implement graph-based reduction of
% $\lambda$-expressions. We can regard a traversal as an alternative
% notion of path adapted to the graph representation of
% $\lambda$-expressions given by computation trees.

% The traversals of a computation tree provide a way to perform
% \emph{local computation} of $\beta$-reductions as opposed to a global
% approach where the $\beta$-reduction is implemented by performing
% substitutions. A notion of local computation of $\beta$-reduction has
% been investigated by Danos and Regnier
% \cite{DanosRegnier-Localandasynchronou} through the use of special
% graphs called ``virtual nets'' that embed the lambda-calculus.


\section{The safe lambda calculus}
\label{sec:safe}
\subsection*{Higher-order safe grammars}
We first present the safety restriction as it was originally defined
\cite{KNU02}. We consider simple types generated by the grammar $A \,
::= \, o \; | \; A \typear A$. By convention, $\rightarrow$ associates
to the right. Thus every type can be written as $A_1 \typear \cdots
\typear A_n \typear o$, which we shall abbreviate to $(A_1, \cdots,
A_n, o)$ (in case $n = 0$, we identify $(o)$ with $o$). The
\emph{order} of a type is given by $\ord{o} = 0$ and $\ord{A \typear
  B} = \max(\ord{A}+1, \ord{B})$. We assume an infinite set of typed
variables. The order of a typed term or symbol is defined to be the
order of its type.

A (higher-order) \defname{grammar} is a tuple $\langle
\Sigma, \mathcal{N}, \mathcal{R}, S \rangle$, where $\Sigma$ is a
ranked alphabet (in the sense that each symbol $f \in \Sigma$ has an
arity $\mathit{ar}(f) \geq 0$) of \emph{terminals}\footnote{Each $f \in
  \Sigma$ of arity $r \geq 0$ is assumed to have type $(\underbrace{o,
    \cdots, o}_r, o)$.}; $\mathcal{N}$ is a finite set of typed
\emph{non-terminals}; $S$ is a distinguished ground-type symbol of
$\mathcal{N}$, called the start symbol; $\mathcal{R}$ is a finite set
of production (or rewrite) rules, one for each non-terminal $F : (A_1,
\ldots, A_n, o) \in \mathcal{N}$, of the form $ F z_1 \ldots z_m
\rightarrow e$ where each $z_i$ (called \emph{parameter}) is a
variable of type $A_i$ and $e$ is an applicative term of type $o$
generated from the typed symbols in $\Sigma \union \mathcal{N} \union \{z_1,
\ldots, z_m \}$. We say that the grammar is \emph{order-$n$} just in
case the order of the highest-order non-terminal is $n$.

The \defname{tree generated by a recursion scheme} $G$ is a possibly
infinite applicative term, but viewed as a $\Sigma$-labelled tree;
it is \emph{constructed from the terminals in $\Sigma$}, and is obtained by
unfolding the rewrite rules of $G$ \emph{ad infinitum}, replacing
formal by actual parameters each time, starting from the start symbol
$S$. See e.g.~\cite{KNU02} for a formal definition.

\parpic[r]{
$\tree[levelsep=3ex,nodesep=1pt,treesep=1cm,linewidth=0.5pt]{g}
{  \TR{a}
    \tree{g}{\TR{a} \tree{h}{\tree{h}{\vdots}}}
}$
}
\begin{example}\rm\label{eg:running}
  Let $G$ be the following order-2 recursion scheme:
\[\begin{array}{rll}
  S & \rightarrow & H \, a\\
  H \, z^o & \rightarrow & F \, (g \,
  z)\\
  F \, \phi^{(o, o)} & \rightarrow & \phi \, (\phi \, (F \, h))\\
\end{array}\]
where the arities of the terminals $g, h, a$ are $2, 1, 0$ respectively.
The tree generated by $G$ is defined by the infinite term $g \, a \, (g \, a \, (h \, (h \, (h \,
\cdots))))$.%  The only infinite \emph{path} in the
% tree is the node-sequence $\epsilon \cdot 2 \cdot 22 \cdot 221 \cdot
% 2211 \cdots$.

%(with the corresponding \textbfit{trace} $g \, g \, h \, h \, h \,
%\cdots \; \in \; \Sigma^\omega$).
\end{example}

A type $(A_1, \cdots, A_n, o)$ is said to be \defname{homogeneous} if
$\ord{A_1} \geq \ord{A_2}\geq \cdots \geq \ord{A_n}$, and each $A_1$,
\ldots, $A_n$ is homogeneous \cite{KNU02}.  We reproduce the following
definition from \cite{KNU02}.

\begin{definition}[Safe grammar]\rm
  (All types are assumed to be homogeneous.) A term of order $k > 0$
  is \emph{unsafe} if it contains an occurrence of a parameter of
  order strictly less than $k$, otherwise the term is \emph{safe}. An
  occurrence of an unsafe term $t$ as a subexpression of a term $t'$
  is \emph{safe} if it is in the context $\cdots (ts) \cdots$,
  otherwise the occurrence is \emph{unsafe}. A grammar is
  \defname{safe} if no unsafe term has an unsafe occurrence at a
  right-hand side of any production.
%   A rewrite rule $F z_1 \ldots z_m \rightarrow e$ is said to be
%   \defname{unsafe} if the righthand term $e$ has a subterm $t$ such
%   that
% \begin{enumerate}[(i)]
% \item $t$ occurs in an {\em operand} ({\it i.e.}~second) position of some
%   occurrence of the implicit application operator {\it i.e.}~$e$ has the
%   form $\cdots (s \, t) \cdots $ for some $s$
% \item $t$ contains an occurrence of a parameter $z_i$ (say) whose
%   order is less than that of $t$.
% \end{enumerate}
% A homogeneous grammar is said to be \defname{safe} if none of its
% rewrite rules is unsafe.
\end{definition}

\begin{example}\begin{inparaenum}[(i)] \item Take $\; H : ((o, o), o), \; f : (o, o, o)$; the
    following rewrite rules are unsafe (in each case we underline the
    unsafe subterm that occurs unsafely):
\[\begin{array}{rll}
G^{(o, o)} \, x & \quad \rightarrow \quad & H \, \underline{(f \, {x})} \\
F^{((o, o), o, o, o)} \, z \, x \, y & \quad \rightarrow \quad & f \, (F \, \underline{(F \, z
\, {y})} \, y \, (z \, x) ) \, x \\
\end{array}\]
\item The order-2 grammar defined in Example~\ref{eg:running} is
  unsafe.
\end{inparaenum}
% The
% reader is referred to the literature
% \cite{KNU02,demirandathesis,safety-mirlong2004}
% for details about the safety restriction for higher-order grammars.
\end{example}

\subsection*{Safety adapted to the lambda calculus}
We assume a set $\Xi$ of higher-order constants.
We use sequents of the form $\Gamma \vdash_\Xi M : A$ to represent
terms-in-context where $\Gamma$ is the context and $A$ is the type of
$M$. For simplicity
we write $(A_1, \cdots, A_n, B)$ to mean $A_1 \typear \cdots \typear
A_n \typear B$, where $B$ is not necessarily ground.

\begin{definition}\rm
\begin{inparaenum}[(i)]
\item The \defname{safe lambda calculus} is a sub-system of the
  simply-typed lambda calculus defined by induction over the
  following rules:
$$ \rulename{var} \ \rulef{}{x : A\vdash_\Xi x : A} \quad
\rulename{const} \ \rulef{}{\vdash_\Xi f : A} \quad f \in \Xi \quad
\rulename{wk} \ \rulef{\Gamma \vdash_\Xi s : A}{\Delta \vdash_\Xi s : A} \quad
\Gamma \subset \Delta$$
$$ \rulename{app} \ \rulef{\Gamma \vdash_\Xi s : (A_1,\ldots,A_n,B) \
  \Gamma \vdash_\Xi t_1 : A_1 \; \ldots \; \Gamma \vdash_\Xi t_n : A_n
} {\Gamma \vdash_\Xi s t_1 \ldots t_n : B} \ \ord{B} \sqsubseteq
\ord{\Gamma}$$
$$ \rulename{abs} \ \rulef{\Gamma, x_1 : A_1, \ldots, x_n : A_n
  \vdash_\Xi s : B} {\Gamma \vdash_\Xi \lambda x_1 \ldots x_n . s :
  (A_1, \ldots ,A_n,B)} \ \ord{A_1, \ldots ,A_n,B} \sqsubseteq
\ord{\Gamma}$$ where $\ord{\Gamma}$ denotes the set $\{ \ord{y} : y
\in \Gamma \}$ and ``$c \sqsubseteq S$'' means that $c$ is a
lower-bound of the set $S$. For convenience, we shall omit the
subscript from $\vdash_\Xi$ whenever the generator-set $\Xi$ is clear from
the context.

\noindent \item The sub-system that is defined by the same rules in
(i), such that all types that occur in them are homogeneous, is called
the \defname{homogeneous safe lambda calculus}.
\end{inparaenum}
\end{definition}

The safe lambda calculus deviates from the standard definition of the simply-typed lambda calculus in a number of ways. First the rules $\rulename{app}$ and $\rulename{abs}$
respectively can perform multiple applications and abstract several
variables at once. (Of course this feature alone does not alter
expressivity.) Crucially, the side-conditions in the application rule
and abstraction rules require that variables in the typing context
have order no smaller than that of the term being formed.  We do not
impose any constraint on types. In particular, type-homogeneity as
used originally to define safe grammars \cite{KNU02} is not required
here. Another difference is that we allow $\Xi$-constants to have
arbitrary higher-order types.  % Thus our formulation
% of the safe lambda calculus is more general than the one proposed in
% the technical report \cite{safety-mirlong2004}. (It is possible to
% reconcile the two definitions by adding the further constraint that
% each type occurring in our rules is homogeneous and by restricting
% constants to at most order 1.)

\begin{example}[Kierstead terms]
\label{ex:kierstead}
Consider the terms $M_1 = \lambda f . f (\lambda x . f (\lambda y . y
))$ and $M_2 = \lambda f . f (\lambda x . f (\lambda y .x ))$ where
$x,y:o$ and $f:((o,o),o)$. The term $M_2$ is not safe because in the
subterm $f (\lambda y . x)$, the free variable $x$ has order $0$ which
is smaller than $\ord{\lambda y . x} = 1$.  On the other hand, $M_1$
is safe.
%On the other hand, $M_1$ is safe as the following proof tree shows:
%$$
% \rulef{
%     \rulef{
%        \rulef{}{f \vdash f} {\sf(var)}
%        \
%        \rulef{
%             \rulef{
%                \rulef{
%                    \rulef{}{f \vdash f} {\sf(var)}
%                }
%                {f , x \vdash f } {\sf(wk)}
%                \
%                \rulef{
%                    \rulef{
%                        \rulef{}{y \vdash y} {\sf(var)}
%                    }
%                    {y \vdash \lambda y . y } {\sf(abs)}
%                }
%                {f , x \vdash \lambda y .y } {\sf(wk)}
%             }
%             {f , x \vdash f (\lambda y .y )} {\sf(app)}
%        }
%        { f  \vdash \lambda x . f (\lambda y .y )} {\sf(abs)}
%     }
%     {
%        f  \vdash f (\lambda x . f (\lambda y .y ))} {\sf(app)}
%     }
% { \vdash M_1 = \lambda f . f (\lambda x . f (\lambda y .y )) } {\sf(abs)}
%$$
\end{example}

It is easy to see that valid typing judgements of the safe lambda
calculus satisfy the following simple invariant:
\begin{lemma}
\label{lem:ordfreevar}
If $\Gamma \vdash M : A$ then every variable in $\Gamma$ occurring
free in $M$ has order at least $ord(M)$.
\end{lemma}


When restricted to the homogeneously-typed
sub-system, the safe lambda calculus captures the original notion
of safety due to Knapik \emph{et al.} in the context of higher-order
grammars:

\begin{proposition} Let $G = \langle \Sigma, \mathcal{N}, \mathcal{R},
  S \rangle$ be a grammar and let $e$ be an applicative term generated
  from the symbols in $\mathcal{N} \cup \Sigma \cup \makeset{z_1^{A_1},
    \cdots, z_m^{A_m}}$.  A rule $F z_1 \ldots z_m \rightarrow e$ in
  $\mathcal{R}$ is safe if and only if $ z_1 : A_1, \cdots, z_m : A_m
  \vdash_{\Sigma \cup \mathcal{N}} e : o$ is a valid typing judgement
  of the \emph{homogeneous} safe lambda calculus.
\end{proposition}

\emph{In what sense is the safe lambda calculus safe?} A basic idea
in the lambda calculus is that when performing $\beta$-reduction, one
must use capture-\emph{avoiding} substitution, which is standardly
implemented by renaming bound variables afresh upon each substitution.
In the safe lambda calculus, however, variable capture can never
happen (as the following lemma shows). Substitution can therefore be
implemented simply by capture-\emph{permitting} replacement, without
any need for variable renaming. In the following, we write
$M\captsubst{N}{x}$ to denote the capture-\emph{permitting}
substitution\footnote{This substitution is done by
textually replacing all free occurrences of $x$ in $M$ by $N$ without performing variable renaming.  In particular for the abstraction
  case we have
$(\lambda y_1\ldots y_n . M)\captsubst{N}{x} = \lambda y_1\ldots y_n . M\captsubst{N}{x}$ when $x\not\in
  \{ y_1\ldots y_n \}$.}
%\footnote{This substitution is implemented by textually
%  replacing all free occurrences of $x$ in $M$ by $N$ without
%  performing variable renaming.  In particular for the abstraction
%  case $(\lambda \overline{y} . P)\captsubst{N}{x}$ is defined as
%  $\lambda \overline{y} . P\captsubst{N}{x}$ if $x\not\in
%  \overline{y}$ and $\lambda \overline{y} . P$ elsewhere.}
of $N$ for $x$ in $M$.

\begin{lemma}[No variable capture]\label{lem:nvc}
\label{lem:homog_nocapture} There is
no variable capture when performing capture-permitting
substitution of $N$ for $x$ in $M$
provided that $\Gamma, x:B \vdash M : A$ and $\Gamma \vdash  N : B$ are valid judgments of the safe lambda calculus.
\end{lemma}

\proof
  We proceed by structural induction. The variable, constant and
  application cases are trivial. For the abstraction case, suppose $M = \lambda \overline{y}. R$ where $\overline{y} = y_1
  \ldots y_p$. If $x \in \overline{y}$ then $M \captsubst{N}{x} = M$ and there is no variable capture.

 If $x \not\in \overline{y}$ then we have $M \captsubst{N}{x} = \lambda \overline{y} . R \captsubst{N}{x}$.  By the induction hypothesis there is no variable capture in $R \captsubst{N}{x}$.  Thus variable capture can only happen if the following two conditions are met: $x$ occurs freely in $R$, and some variable $y_i$ for $1 \leq i \leq p$ occurs freely in $N$. By Lemma \ref{lem:ordfreevar}, the latter condition  implies $\ord{y_i} \geq \ord{N} = \ord{x}$.  Since $x \not \in \overline{y}$, the former condition implies that $x$ occurs freely in the safe term $\lambda \overline{y}. R$
  therefore Lemma \ref{lem:ordfreevar} gives $ \ord{x} \geq
  \ord{\lambda \overline{y} . R} \geq 1+ \ord{y_i} > \ord{y_i}$ which  gives a contradiction.
\qed


\begin{remark}
  A version of the No-variable-capture Lemma also holds in safe
  grammars, as is implicit in (for example Lemma 3.2 of) the original
  paper \cite{KNU02}.
\end{remark}

\begin{example}
  In order to contract the $\beta$-redex in the term
\[f:(o,o,o),x:o
  \vdash (\lambda \varphi^{(o,o)} x^o . \varphi \, x) (\underline{f \,
    x}) : (o,o)\] one should rename the bound variable $x$ with a fresh name to
  prevent the capture of the free occurrence of $x$ in the underlined term during substitution. Consequently, by the previous lemma,
  the term is not safe. Indeed, it cannot be because $\ord{x} = 0 < 1
  = \ord{f x}$.
\end{example}

Note that it is not the case that $\lambda$-terms
that satisfy the No-variable-capture Lemma are necessarily safe. For instance the $\beta$-redex in $\lambda y^o
z^o. (\lambda x^o .y) z$ can be contracted using capture-permitting
substitution, even though the term is not safe.

\subsection*{Reductions and transformations preserving safety}

From now on we will use the standard notation $M\subst{N}{x}$ to
denote the substitution of $N$ for $x$ in $M$.  It is understood that,
provided that $M$ and $N$ are safe, this substitution is
capture-permitting.


\begin{lemma}[Substitution preserves safety]
\label{lem:subst_preserve_safety}
If $\Gamma, x :B \vdash M : A$ and $\Gamma \vdash N : B$ then $\Gamma \vdash M[N/x] : A$.
\end{lemma}
This is proved by an easy induction on the structure of the safe term $M$.


It is desirable to have an appropriate notion of reduction for our
calculus. However the standard $\beta$-reduction rule is not
adequate. Indeed, safety is not preserved by $\beta$-reduction as the
following example shows. Suppose that $w,x,y,z : o$ and $f : (o,o,o)
\in \Sigma$ then the safe term $(\lambda x y . f x y) z w$
$\beta$-reduces to $(\underline{\lambda y . f z y}) w$ which is unsafe
since the underlined order-1 subterm contains a free occurrence of the
ground-type $z$. However if we perform one more reduction we obtain
the safe term $f z w$. This suggests an alternative notion of
reduction that performs simultaneous reduction of ``consecutive''
$\beta$-redexes. In order to define this reduction we first introduce
the appropriate notion of redex.

In the simply-typed lambda calculus a redex is a term of the form
$(\lambda x . M) N$. In the safe lambda calculus, a redex is a
succession of several standard redexes:

\begin{definition}\rm
Let $l\geq 1$ and $n\geq 1$. We use the abbreviations $\overline{x}$ and $\overline{x}:\overline{A}$  for $x_1 \ldots x_n$ and $x_1:A_1, \ldots, x_n : A_n$ respectively.

A \defname{safe redex} is a safe term of the form $(\lambda
\overline{x} . M) N_1 \ldots N_l$ such that the variables
$\overline{x}$ are abstracted altogether by one instance of the
$(\sf{abs})$ rule and the term $(\lambda \overline{x} . M)$ is
applied to $N_1$, \ldots, $N_l$ by one instance of the $(\sf{app})$ rule.
\end{definition}
Thus $M$, the $N_i$'s and the redex itself are all safe terms.
For instance, in the case $n<l$, a safe redex has a derivation tree of the following  form:
$$   \rulef{
            \rulef{\rulef{\ldots}{\Gamma, \overline{x}:\overline{A} \vdash M : (A_{n+1}, \ldots, A_l, B)}}{\Gamma \vdash \lambda \overline{x} . M : (A_1, \ldots, A_l, B)} ({\sf abs })
            \quad
            \rulef{\ldots}{\Gamma \vdash N_1 :A_1}  \ \ldots \  \rulef{\ldots}{\Gamma \vdash N_l :A_l}
    }
    {
       \Gamma \vdash (\lambda \overline{x} . M) N_1 \ldots N_l : B
    } ({\sf app})
$$


We are now in a position to define a notion of reduction for safe terms.

\begin{definition}\rm
\label{dfn:safereduction} We use the
abbreviations $\overline{x} = x_1 \ldots x_n$,
$\overline{N} = N_1 \ldots N_l$.
The relation $\beta_s$ is defined on the set of safe redexes as:
\begin{eqnarray*}
  \beta_s &=&
  \{  \ (\lambda \overline{x} . M) N_1 \ldots N_l \mapsto \lambda x_{l+1} \ldots x_n. M\subst{\overline{N}}{x_1 \ldots x_l} \mbox{, for $n> l$}
  \} \\
  &\cup&
  \{ \ (\lambda \overline{x}  . M) N_1 \ldots N_l \mapsto M\subst{N_1 \ldots N_n}{\overline{x}} N_{n+1} \ldots N_l
  \mbox{, for $n\leq l$} \} \ .
\end{eqnarray*}
where $M\subst{R_1 \ldots R_k}{z_1 \ldots z_k}$ denotes the simultaneous substitution of $R_1$,\ldots,$R_k$ for $z_1, \ldots, z_k$ in $M$.  The
\defname{safe $\beta$-reduction}, written $\betasred$, is the
compatible closure of the relation $\beta_s$ with respect to the
formation rules of the safe lambda calculus.
\end{definition}

\noindent \emph{Remark:} The $\beta_s$-reduction is a multi-step
$\beta$-reduction {\it i.e.}~$\betared \subset \betasred \subset
\betaredtr$\ .


\begin{lemma}[$\beta_s$-reduction preserves safety]
\label{lem:safered_preserve_safety}
If $\Gamma \vdash s :A$ and $s \betasred t$ then $\Gamma \vdash t :A$.
\end{lemma}

\proof
  It suffices to show that the relation $\beta_s$ preserves safety.
Suppose that $s\ \beta_s\ t$ where $s$ is the
safe-redex $(\lambda x_1 \ldots x_n . M) N_1
  \ldots N_l $ with $x_1 : B_1, \ldots, x_n: B_n$
and $M$ of type $C$.  W.l.o.g we can assume that the last rule   used to form the term $s$ is {\sf(app)} {\it i.e.}~not the weakening rule {\sf(wk)}, thus  we have $\Gamma = fv(s)$.

Suppose $n>l$ then $A = (B_{l+1}, \ldots, B_n, C)$. By Lemma \ref{lem:subst_preserve_safety} we can form the safe term %\begin{equation}
$\Gamma, x_{l+1}:B_{l+1}, \ldots x_n :B_{n}\vdash M\subst{\overline{N}}{x_1 \ldots x_l} : C$. %\label{jud:substsafe}\ .
%\end{equation}
By Lemma \ref{lem:ordfreevar}, since $s$ is safe, all the variables in $\Gamma$ have order $\geq \ord{A}$. This ensures that the side-condition of the {\sf(abs)} rule is verified if we abstract the variables $x_{l+1} \ldots x_n$, which gives us the judgement $\Gamma \vdash t :A$.

Suppose $n \leq l$. The substitution lemma gives
$\Gamma \vdash M\subst{N_1 \ldots N_n}{\overline{x}} : C$ and using {\sf(app)} we form $\Gamma \vdash t :A$.
  \qed


In general, safety is not preserved by $\eta$-expansion; for instance we have
% $f:o,o \vdash f$ but $f:o,o \not \vdash \lambda x^o . f x$.
%This remark remains true for closed terms, for instance
$\vdash \lambda y^o z^o . y : (o,o,o)$ but
$\not \vdash \lambda x^o . (\lambda y^o z^o . y) x : (o,o,o)$.
However safety is preserved by $\eta$-reduction:

\begin{lemma}[$\eta$-reduction preserves safety]
  $\Gamma \vdash \lambda \varphi . s \varphi :A $ with $\varphi$ not
  occurring free in $s$ implies $\Gamma \vdash s :A$.
\end{lemma}
\proof
  Suppose $\Gamma \vdash \lambda \varphi . s \varphi :A$. If $s$ is an  abstraction then by construction of the safe term $\lambda \varphi . s \varphi$, $s$ is necessarily safe.  If $s = N_0 \ldots N_p$ with
  $p\geq 1$ then again, since $\lambda \varphi . N_0 \ldots N_p
  \varphi$ is safe, each of the $N_i$ is safe for $0 \leq i \leq p$
  and for any $z\in fv(\lambda \varphi . s \varphi)$, $\ord{z} \geq
  \ord{\lambda \varphi . s \varphi} = \ord{s}$. Since  $\varphi$ does not occur free in $s$ we have $fv(s) = fv(\lambda \varphi . s \varphi)$, thus we can use the application rule to form $fv(s) \vdash N_0 \ldots N_p : A$. The weakening rules permits us to conclude $\Gamma \vdash s :A$. \qed



The $\eta$-long normal form (or simply $\eta$-long form) of a term
% (also called \emph{long reduced form}, \emph{$\eta$-normal form} and
% \emph{extensional form} in the literature
% \cite{DBLP:journals/tcs/JensenP76,DBLP:journals/tcs/Huet75,huet76})
is obtained by hereditarily $\eta$-expanding every subterm occurring
at an operand position. Formally the \defname{$\eta$-long form}
$\elnf{t}$ of a term $t: (A_1,\ldots,A_n,o)$ with $n \geq 0$ is
defined by cases according to the syntactic shape of $t$:
\begin{eqnarray*}
  \elnf{\lambda x . s } &=& \lambda x . \elnf{s} \\
  \elnf{x s_1 \ldots s_m } &=& \lambda \overline{\varphi} . x \elnf{s_1}\ldots \elnf{s_m} \elnf{\varphi_1} \ldots \elnf{\varphi_n} \\
  \elnf{(\lambda x . s) s_1 \ldots s_p } &=& \lambda \overline{\varphi} . (\lambda x . \elnf{s}) \elnf{s_1} \ldots \elnf{s_p} \elnf{\varphi_1} \ldots \elnf{\varphi_n}
\end{eqnarray*}
where $m \geq 0$, $p\geq 1$, $x$ is a  variable or constant, $\overline{\varphi} = \varphi_1 \ldots \varphi_n$ and each $\varphi_i : A_i$ is a fresh variable.

%\begin{remark}
%  Converting a term to its $\eta$-long normal form does not introduce
%  new redex therefore the $\eta$-long normal form of $\beta$-normal
%  term is a $\beta$-normal term.
%\end{remark}

\begin{lemma}[$\eta$-long normalization preserves safety]
\label{lem:elnf_preserves_safety}
If $\Gamma \vdash s :A$ then $\Gamma \vdash \elnf{s} :A$.
\end{lemma}
\proof

 First we observe that for any variable or constant $x:A$ we have $x:A \vdash \elnf{x} :A$. We show this by induction on $\ord{x}$.
It is verified for any ground type variable $x$
since $x = \elnf{x}$.
Step case: $x:A$ with $A=(A_1, \ldots, A_n,o)$ and $n>0$. Let $\varphi_i:A_i$ be fresh variables for $1\leq i\leq n$.
Since $\ord{A_i} < \ord{x}$ the induction hypothesis gives $\varphi_i :A_i \vdash \elnf{\varphi_i} : A_i$. Using {\sf(wk)} we obtain $x:A, \overline{\varphi} : \overline{A}
  \vdash \elnf{\varphi_i} :A_i$.  The application rule gives $x :A, \overline{\varphi} : \overline{A} \vdash x \elnf{\varphi_1} \ldots \elnf{\varphi_n}
  : o$ and the abstraction rule gives $ x :A \vdash \lambda
  \overline{\varphi} . x \elnf{\varphi_1} \ldots \elnf{\varphi_n} =
  \elnf{x} :A$.


We now prove the lemma by induction on $s$.
The base case is covered by the previous observation.
\emph{Step case:}
\begin{compactitem}
\item $s = x s_1 \ldots s_m$ with $x: (B_1, \ldots, B_m, A)$, $A = (A_1, \ldots, A_n, o)$ for some $m\geq 0$, $n>0$ and $s_i : B_i$ for $1 \leq i \leq
  m$.  Let $\varphi_i: A_i$ be fresh variables for $1\leq i \leq
  n$. By the previous observation we have $\varphi_i :A_i \vdash \elnf{\varphi_i} :A_i$, the weakening rule then gives us $\Gamma , \overline{\varphi} : \overline{A}
  \vdash \elnf{\varphi_i} : A_i$.  Since the judgement
  $\Gamma \vdash x s_1 \ldots s_m : A$ is formed using the {\sf (app)} rule, each $s_j$ must be safe for $1\leq j \leq m$, thus by the induction hypothesis we have $\Gamma \vdash \elnf{s_j} : B_j$ and by weakening we get $\Gamma, \overline{\varphi} :\overline{A} \vdash \elnf{s_j} : B_j$.  The {\sf(app)}
  rule then gives $\Gamma, \overline{\varphi} :\overline{A} \vdash x \elnf{s_1} \ldots \elnf{s_m} \elnf{\varphi_1} \ldots \elnf{\varphi_n} : o$. Finally
  the {\sf (abs)} rule gives $\Gamma \vdash \lambda \overline{\varphi} . x
  \elnf{s_1} \ldots \elnf{s_m} \elnf{\varphi_1} \ldots
  \elnf{\varphi_n} = \elnf{s} : A$, the side-condition of {\sf (abs)} being verified since $\ord{\elnf{s}} = \ord{s}$.


\item $s = t s_0 \ldots s_m$ where $t$ is an abstraction.
For some fresh variables $\varphi_1$, \ldots, $\varphi_n$
we have $\elnf{s} = \lambda \overline{\varphi}. \elnf{t} \elnf{s_0} \ldots \elnf{s_m} \elnf{\varphi_1}
  \ldots \elnf{\varphi_n}$. Again, using the induction hypothesis we can easily derive $\Gamma \vdash
 \lambda \overline{\varphi}. \elnf{t} \elnf{s_0} \ldots \elnf{s_m} \elnf{\varphi_1} \ldots \elnf{\varphi_n} : A$.

\item $s = \lambda \overline{\eta} . t $ where
$\overline{\eta} : \overline{B}$ and $t:C$ is not an abstraction. The induction hypothesis gives $\Gamma,
  \overline{\eta} : \overline{B} \vdash \elnf{t} : C$ and using
{\sf(abs)} we get $\Gamma \vdash \lambda \overline{\eta} . \elnf{t} = \elnf{s} : A$.  \qed
\end{compactitem}


Note that the converse does not hold in general, for instance $\lambda
x^o . f^{(o,o,o)} x^o$ is unsafe although $\elnf{\lambda x . f x} =
\lambda x^o y^o . f x y$ is safe.


\notetoself{Check and prove the following lemma}
 For terms with homogeneous types however, the converse does hold:
\begin{lemma}
If $\Gamma \vdash \elnf{s}$ is homogeneously safe (i.e. it is a safe
judgement of the safe $\lambda$-calculus and each sequent occurring
at the nodes of the proof tree is homogeneously typed) then $\Gamma
\vdash s$ is homogeneously safe.
\end{lemma}


\subsection*{Numeric functions representable in the safe lambda calculus}

Natural numbers can be encoded into the simply-typed
lambda calculus using the Church Numerals: each $n\in\nat$ is
encoded into the term $\encode{n} = \lambda s z. s^n z$ of type $I =
((o,o),o,o)$ where $o$ is a ground type. In 1976 Schwichtenberg
\cite{citeulike:622637} showed the following:


\begin{theorem}[Schwichtenberg 1976]
The numeric functions representable by simply-typed $\lambda$-terms of
type $I\rightarrow \ldots \rightarrow I$ using the Church Numeral
encoding are exactly the multivariate polynomials \emph{extended with
the conditional function}.
\end{theorem}

If we restrict ourselves to safe terms, the representable functions are
exactly the multivariate polynomials:
\begin{theorem}
\label{thm:polychar}
The functions representable by safe $\lambda$-expressions of type
$I\rightarrow \ldots \rightarrow I$ are exactly the multivariate
polynomials.
\end{theorem}

\begin{corollary}
The conditional operator $C:I\rightarrow I\rightarrow I \rightarrow I$ verifying  $C t y z \rightarrow_\beta y$  if $t \rightarrow_\beta \encode{0}$ and $C t y z \rightarrow_\beta z$ if $t \rightarrow_\beta \encode{n+1}$ is not definable in the safe simply-typed lambda calculus.
\end{corollary}
\proof
  Natural numbers are encoded using Church Numerals: $\encode{n} =
  \lambda s z. s^n z$.  Addition: For $n,m \in \nat$, $\encode{n+m} =
  \lambda \alpha^{(o,o)} x^o . (\encode{n} \alpha) (\encode{m} \alpha
  x)$. Multiplication: $\encode{n . m} = \lambda \alpha^{(o,o)}
  . \encode{n} (\encode{m} \alpha)$.  All these terms are safe and
  clearly any multivariate polynomial $P(n_1, \ldots, n_k)$ can be
  computed by composing the addition and multiplication terms as
  appropriate.

For the converse, let $U$ be a safe $\lambda$-term of type
$I\rightarrow I\rightarrow I$.  The generalization to terms of type
$I^n \rightarrow I$ for $n>2$ is immediate (they correspond to
polynomials with $n$ variables). W.l.o.g we can
assume that $U = \lambda x y \alpha z. u$ where $u$ is a safe term of
ground type in $\beta$-normal form with $fv(u) \subseteq \{ x, y : I,
z :o, \alpha : o\rightarrow o \}$.

\emph{Notation:} Let $T$ be a set of terms of type $\tau \rightarrow \tau$ and $T'$ be a set of terms of type $\tau$ then $T \cdot T'$ denotes the set of terms $\{ s s' : \tau \ | \ s \in T \wedge s' \in T' \}$. We also define
$T^k \cdot T'$ recursively as follows:  $T^0 \cdot T' = T'$ and
for $k\geq 0$, $T^{k+1} \cdot T' = T \cdot (T^k \cdot T')$ ({\it i.e.}~$T^k \cdot T'$ denotes $\{ s_1( \ldots (s_k s'))  \ | \ s_1, \ldots, s_k \in T \wedge s' \in T' \}$). We define $T^+\cdot T' = \Union_{k > 0} T^k \cdot T'$ and
$T^*\cdot T' = (T^+\cdot T') \union T'$.
For two sets of terms $T$ and $T'$, we write $T =_\beta T'$ to express that any term of $T$ is $\beta$-convertible to some term $t'$ of $T'$ and reciprocally.

Let us write $\mathcal{N}^\tau$ for the set of $\beta$-normal terms of
type $\tau$ where $\tau$ ranges in $\{ o, o\rightarrow o, I \}$ and
with free variables in $\{ x,y:I, z:o, \alpha:o\rightarrow o\}$. We
write $\mathcal{A}^\tau$ for the subset of $\mathcal{N}^\tau$
consisting of applications only ({\it i.e.}~not abstractions).
Let $B$ be the set of terms of type $(o,o)$ defined by $B = \{ \alpha \} \union \{ \lambda a.b \ | \ b \in \{a,z\}, a \neq z \}$.
It is easy to see that the following equations hold:
\begin{eqnarray*}
\mathcal{A}^I &=& \{ x,y \} \\
\mathcal{N}^{(o,o)} &=& B \union \mathcal{A}^I \cdot
\mathcal{N}^{(o,o)} = (\mathcal{A}^I)^* \cdot B \\
\mathcal{A}^{(o,o)} &=& \{ \alpha \} \union (\mathcal{A}^I)^+ \cdot B \\
\mathcal{A}^o = \mathcal{N}^o &=& \{ z \} \union \mathcal{A}^{(o,o)} \cdot \mathcal{N}^o = (\mathcal{A}^{(o,o)})^* \cdot \{ z \}
\end{eqnarray*}
Hence $\mathcal{A}^o = \left( \{\alpha \} \union \{x,y\}^+ \cdot \left( \{\alpha \} \union \{\lambda a.b \ | \ b \in \{a,z\}, a \neq z \} \right) \right)^* \cdot \{ z \}$.
Since $u$ is safe, it cannot contain terms of the form $\lambda a . z$ with $a \neq z$ occurring at an operand position, therefore since $u$ belongs to $\mathcal{A}^o$ we have:
\begin{equation}
u \in \left( \{\alpha\} \union \{x,y\}^+ \cdot \{\alpha,
\underline{i} \} \right)^* \cdot \{ z \} \label{eqn:u}
\end{equation}
where $\underline{i}$ is the identity term of type $o\rightarrow o$.


We observe that $\encode{k} \underline{i} =_\beta \underline{i}$ for all $k \in \nat$ and for $l\geq 1$, for all $k_1, \ldots k_l \in \nat$,
$\encode{k_1}\ldots \encode{k_l} \alpha =_\beta
\encode{k_1\times \ldots \times k_l} \alpha$. Hence for all $m,n \in \nat$ we have:
\begin{equation}
\begin{array}{llr}
\{\encode{m},\encode{n}\}^+ \cdot \{\alpha, \underline{i} \} &=_\beta
\{ \underline{i} \} \union
\{ \encode{m^i n^j} \alpha \ |\ i+j \geq 1 \} \nonumber \\
&= \{ \encode{m^i n^j} \alpha \ |\ i,j \geq 0 \} & ( \mbox{since } \underline{i} = \encode{0} \alpha) \end{array}
\label{eqn:intermediate}
\end{equation}
therefore:
$$\begin{array}{llr}
u[\encode{m}, \encode{n}/x,y] &\in \left( \{ \alpha \} \union \{\encode{m},\encode{n}\}^+ \cdot \{\alpha, \underline{i} \} \right)^* \cdot \{ z \}  & \mbox{(by Eq.\ \ref{eqn:u})} \\
&=_\beta \left( \{\alpha \} \union \{ \encode{m^i n^j}
\alpha \ | \ i,j \geq 0 \} \right)^* \cdot \{ z \} & \mbox{(by Eq.\ \ref{eqn:intermediate})}  \\
&=_\beta \left\{ \encode{m^i n^j}
\alpha \ | \ i,j \geq 0 \right\}^* \cdot \{ z \} & \mbox{($\alpha z =_\beta \encode{1} \alpha z$)}.
\end{array}$$

Furthermore, for all $m,n,r,i,j\in \nat$
we have $\encode{m^i n^j} \alpha (\alpha^r z) =_\beta
\alpha^{r + m^i n^j} z$,
hence $u[\encode{m} \encode{n}/x,y] =_\beta \alpha^{p(m,n)} z$ where $p(m,n) = \sum_{0\leq k \leq d} m^{i_k} n^{j_k}$ for some $i_k,j_k \geq 0$, $k \in\{ 0,..,d \}$ and $d\geq 0$.
Thus $U \encode{m} \encode{n} =_\beta \encode{p(m,n)}$. \qed


For instance, the term $ C = \lambda F G H \alpha x . H (
\underline{\lambda y . G \alpha x} ) (F \alpha x)$ used by
Schwichtenberg \cite{citeulike:622637} to define the conditional
operator is unsafe since the underlined subterm is of order $1$,
occurs at an operand position and contains an occurrence of $x$ of
order $0$.


\section{A game-semantic account of safety}
\label{sec:gamesemaccount}
Our aim here is to characterize safety, which is a syntactic property,
by game semantics. Because of length restriction, we shall assume that
the reader is familiar with the basics of game semantics.  (For an
introduction, we recommend \cite{abramsky:game-semantics-tutorial}). Recall that a \emph{justified sequence} over
an arena is an alternating sequence of O-moves and P-moves such that
every move $m$, except the opening move, has a pointer to some earlier
occurrence of the move $m_0$ such that $m_0$ enables $m$ in the
arena. A \emph{play} is just a justified sequence that satisfies
Visibility and Well-Bracketing. A basic result in game semantics is
that $\lambda$-terms are denoted by \emph{innocent strategies}, which
are strategies that depends only on the \emph{P-view} of a play. The
main result (Theorem~\ref{thm:safeincrejust}) of this section is that
if a $\lambda$-term is safe, then its game semantics (is an innocent
strategy that) is \emph{P-incrementally justified}. In such a
strategy, pointers emanating from the P-moves of a play are uniquely
reconstructible from the underlying sequence of moves and pointers
from the O-moves therein: Specifically a P-question always
points to the last pending O-question (in the P-view) of a greater
order.

The proof of Theorem~\ref{thm:safeincrejust} depends on a
Correspondence Theorem (see the Appendix) that relates the strategy
denotation of a $\lambda$-term $M$ to the set of \emph{traversals}
over a certain abstract syntax tree of the $\eta$-long form of $M$.
In the language of game semantics, traversals are just (concrete
representations of) the \emph{uncovering} (in the sense of Hyland
and Ong \cite{hylandong_pcf}) of plays in the strategy denotation.

The useful transference technique between plays and traversals was
originally introduced by one of us \cite{OngLics2006} for studying the
decidability of MSO theories of infinite structures generated by
higher-order grammars (in which the $\Sigma$-constants are at most
order 1, and \emph{uninterpreted}).
% In this setting, free variables are interpreted
% as constructors and therefore they do not have the ``full power'' of
% true free variables and are limited to order $1$ at most. Also,
% although the grammar can perform higher-order computations, the
% structure being studied is itself of ground type.
In the Appendix, we present an extension of this framework to the
general case of the simply-typed lambda calculus with free variables
of any order. A new traversal rule is introduced to handle nodes
labelled with free variables. Also new nodes are added to the
computation tree to account for the answer moves of the game
semantics, thus enabling the framework to model languages with
interpreted constants such as \pcf~(by adding traversal rules to
handle constant nodes).

\subsection*{Incrementally-bound computation tree}
In \cite{OngLics2006} the computation tree of a grammar is defined as
the unravelling of a finite graph representing the long transform of a
grammar. Similarly we define the computation tree of a $\lambda$-term
as an abstract syntax tree of its $\eta$-long normal form.  We write
$l(t_1, \ldots, t_n)$ with $n \geq 0$ to denote the tree with a root labelled $l$ with $n$ children subtrees $t_1$, \ldots, $t_n$.
In the following, judgements of the form $\Gamma \vdash M:T$
refer to simply-typed terms not necessarily safe unless  mentioned.

\begin{definition}\rm
\label{dfn:comptree}
  The \defname{computation tree} $\tau(M)$ of a simply-typed term
  $\Gamma \vdash M:T$ with variable names in a countable set
  $\mathcal{V}$ is a tree with labels in $ \{ @ \} \union \mathcal{V}
  \union \{ \lambda x_1 \ldots x_n \ | \ x_1 ,\ldots, x_n \in
  \mathcal{V} \}$ defined from its $\eta$-long form as follows:
\begin{eqnarray*}
  \mbox{for $n\geq 0$ and $s:o$, } \tau(\lambda x_1 \ldots x_n . s) &=& \lambda x_1 \ldots x_n(t) \mbox{\quad where $\tau(s) = \lambda(t)$}\\
  \mbox{for $m\geq 0$ and $x \in  \mathcal{V}$, } \tau( x s_1 \ldots s_m : o) &=&  \lambda (x(\tau(s_1),\ldots,\tau(s_m)))\\
  \mbox{for $m \geq 1$, } \tau((\lambda x.t) s_1 \ldots s_m :o) &=& \lambda (@(\tau(\lambda x.t),\tau(s_1),\ldots,\tau(s_m))) \ .
\end{eqnarray*}
\end{definition}

Even-level nodes are $\lambda$-nodes (the root is on level 0). A
single $\lambda$-node can represent several consecutive variable
abstractions or it can just be a \emph{dummy lambda} if the
corresponding subterm is of ground type.  Odd-level nodes are variable or application nodes.

The \defname{order} of a node $n$, written $\ord{n}$, is defined as follows:
@-nodes have order $0$. The order of a variable-node is the type-order of the variable labelling it.
The order of the root node is the type-order of $(A_1,\ldots,A_p, T)$ where $A_1,\ldots, A_p$ are the types of the variables in the context $\Gamma$.
Finally, the order of a lambda node different from the root is the type-order of the term represented by the sub-tree rooted at that node.

We say that a variable node $n$ labelled $x$ is \defname{bound} by a node $m$, and $m$ is called the \defname{binder} of $n$, if $m$ is the closest node in the path from $n$ to the root such that $m$ is labelled
$\lambda \overline{\xi}$ with $x\in \overline{\xi}$.
We introduce a class of computation trees in which the binder
node is uniquely determined by the nodes' orders:
\begin{definition}\rm
  A computation tree is \defname{incrementally-bound} if for all
  variable node $x$, either $x$ is \emph{bound} by the first
  $\lambda$-node in the path to the root with order $> \ord{x}$ or $x$
  is a \emph{free variable} and all the $\lambda$-nodes in the path to
  the root except the root have order $\leq \ord{x}$.
\end{definition}

\begin{proposition}[Safety and incremental-binding] \hfill
\label{prop:safe_imp_incrbound}
\begin{enumerate}[(i)]
\item If $M$ is safe then $\tau(M)$ is incrementally-bound.
\item Conversely, if $M$ is a \emph{closed} simply-typed term and $\tau(M)$
is incrementally-bound then the $\eta$-long form of $M$ is safe.
\end{enumerate}
\end{proposition}
\proof
  (i) Suppose that $M$ is safe. By Lemma
  \ref{lem:elnf_preserves_safety} the $\eta$-long form of $M$ is safe
  therefore $\tau(M)$ is the tree representation of a safe term.

In the safe lambda calculus, the variables in the context with the
lowest order must be all abstracted at once when using the abstraction
rule. Since the computation tree merges consecutive abstractions into
a single node, any variable $x$ occurring free in the subtree rooted
at a node $\lambda \overline{\xi}$ different from the root must have
order greater or equal to $\ord{\lambda
  \overline{\xi}}$. Reciprocally, if a lambda node $\lambda
\overline{\xi}$ binds a variable node $x$ then $\ord{\lambda
  \overline{\xi}} = 1+\max_{z\in\overline{\xi}} \ord{z} > \ord{x}$.

Let $x$ be a bound variable node. Its binder occurs in the path from
$x$ to the root, therefore, according to the previous observation, $x$
must be bound by the first $\lambda$-node occurring in this path with
order $>\ord{x}$. Let $x$ be a free variable node then $x$ is not
bound by any of the $\lambda$-nodes occurring in the path to the
root. Once again, by the previous observation, all these
$\lambda$-nodes except the root have order smaller than
$\ord{x}$. Hence $\tau$ is incrementally-bound.

(ii) Let $M$ be a closed term such that $\tau(M)$ is
incrementally-bound.  We assume that $M$ is already in $\eta$-long
form.  We prove that $M$ is safe by induction on its structure. The
base case $M = \lambda \overline{\xi} . x$ for some variable $x$ is
trivial.  \emph{Step case:} If $M = \lambda \overline{\xi} . N_1
\ldots N_p$.  Let $i$ range over $1..p$. We have $N_i \equiv \lambda
\overline{\eta_i} . N'_i$ for some non-abstraction term $N'_i$. By the induction hypothesis, $\lambda \overline{\xi} . N_i = \lambda \overline{\xi} \overline{\eta_i} . N'_i$ is a safe closed term, and consequently $N'_i$ is necessarily safe. Let $z$ be a free variable
of $N'_i$ not bound by $\lambda \overline{\eta_i}$ in $N_i$. Since
$\tau(M)$ is incrementally-bound we have $\ord{z} \geq \ord{\lambda
  \overline{\eta_1}} = \ord{N_i}$, thus we can abstract the variables $\overline{\eta_1}$ using the {\sf (abs)} which shows that $N_i$ is safe.  Finally
we conclude $\vdash M = \lambda \overline{\xi} . N_1 \ldots N_p : T$ using
the rules {\sf (app)} and {\sf (abs)}.  \qed



The assumption that $M$ is closed is necessary. For instance for
$x,y:o$, the two identical computation trees $\tau(\lambda x y .x)$
and $\tau(\lambda y . x)$ are
incrementally-bound but $\lambda x y .x$ is safe and $\lambda y . x$
is not.

\subsection*{P-incrementally justified strategy}

We now consider the game-semantic model of the simply-typed lambda
calculus. The strategy denotation of a term is written $\sem{\Gamma \vdash M : T}$. We define the \defname{order} of a move $m$, written $\ord{m}$, to be the length of the path from $m$ to its furthest leaf in the arena minus 1. (There are several ways to define the order of a move; the definition chosen here is sound in the current setting where each question move in the arena enables at least one answer move.)
%{\it i.e.}~height of the subarena rooted at $q$ minus 2.

\begin{definition}\rm
  A strategy $\sigma$ is said to be \defname{P-incrementally
    justified} if for any play $s \, q \in \sigma$ where $q$ is a
  P-question, $q$ points to the last unanswered O-question in $\pview{s}$ with
  order strictly greater than $\ord{q}$.
\end{definition}
Note that although the pointer is determined by the P-view, the choice of the move itself
can be based on the whole history of the play. Thus P-incremental justification does not imply innocence.

The definition suggests an algorithm that, given a play of a P-incrementally justified denotation, uniquely recovers the pointers from the underlying sequence of moves and from the pointers associated to the O-moves therein.
Hence:
\begin{lemma}
\label{lem:incrjustified_pointers_uniqu_recover} In P-incrementally justified strategies, pointers emanating from P-moves are superfluous.
\end{lemma}

\begin{example}
Copycat strategies, such as the identity strategy $id_A$ on game $A$ or the evaluation map $ev_{A,B}$ of type $(A \Rightarrow B) \times A \typear B$, are all P-incrementally justified.\footnote{In such strategies, a P-move $m$ is justified as follows: either $m$ points to the preceding move in the P-view or the preceding move is of smaller order and $m$ is justified by the second last O-move in the P-view.}
\end{example}
%%%% the following example is wrong : ev is P-ij.
%
%\begin{example}
%Take the evaluation map $ev : (o^1 \Rightarrow o^2) \times o^3 \rightarrow o^4$ and the play $s = q^4 q^2 q^1 q^3 \in \sem{ev}$. We have $\ord{q^2} = 1 > \ord{q^1} = \ord{q^3} = 0$. Now $q^3$ points to $q^4$ but $q^2$ is the last unanswered O-question in $\pview{s}= s$ with order $>\ord{q^3}$, hence $\sem{ev}$ is not P-incrementally justified.
%\end{example}




The Correspondence Theorem
% and Lemma \ref{lem:betanf_wellbehavedconst_trav_pview_red}
gives us the following equivalence:
\begin{proposition} % [Incremental-binding vs P-incremental justification]
\label{prop:incrbound_imp_incrjustified}
For a $\beta$-normal term $\Gamma \vdash M : T$,
$\tau(M)$ is incrementally-bound if and only if $\sem{\Gamma \vdash M : T}$
is P-incrementally justified.
\end{proposition}

% \parpic(0.5cm,2cm)(5pt,0.75cm)[x]{ $\tree{\lambda^3}{\tree{f^2}{
%       \tree{\lambda y^1}{ \TR{x^0} }}}$ }
%\parpic(0.5cm,1cm)(0.1pt,0.2cm)[r]{ $\tree{\lambda^3}{\tree{f^2}{ \tree{\lambda y^1}{ \TR{x^0} }}}$ }
%\parpic(0.3cm,1cm)[r]{ \rput[t](0,0cm){$\tree{\lambda^3}{\tree{f^2}{
%        \tree{\lambda y^1}{ \TR{x^0} }}}$} }
%%%  ah Latex... i hate you !!!
%\begin{floatingfigure}[r]{60mm}
%$%\rput[t](0cm,0cm)
%{\tree{\lambda^3}{\tree{f^2}{ \tree{\lambda y^1}{ \TR{x^0} }}}}$
%\end{floatingfigure}

%\begin{wrapfigure}[5]{r}{0.6cm}
% the next line must be commented out if and only if the paragraph appears at the top of the page (otherwise
% the figure in the margin is not positioned correctly):
%\rput[t](0.1cm,0.8cm)
% ${\tree{\lambda^3}{\tree{f^2}{ \tree{\lambda y^1}{ \TR{x^0} }}}}$
%\end{wrapfigure}

\parpic[r]{
${\tree{\lambda^3}{\tree{f^2}{ \tree{\lambda y^1}{ \TR{x^0} }}}}$
}
\noindent \emph{Example:} %\begin{example}
Consider the $\beta$-normal term $\Gamma\vdash f (\lambda y .x) : o$ where $y:o$ and $\Gamma =f:((o,o),o),~x:o$. The figure on the right represents its computation tree
with the node orders given as superscripts.  Node $x$ is not
incrementally-bound therefore $\tau(f (\lambda y .x))$ is not
incrementally-bound and by Proposition
\ref{prop:incrbound_imp_incrjustified}, $\sem{\Gamma \vdash f (\lambda y .x) : o}$ is
not incrementally-justified (although $\sem{\Gamma \vdash f : ((o,o),o)}$ and $\sem{\Gamma \vdash \lambda
  y. x : (o,o)}$ are).
%\end{example}
\smallskip

Propositions \ref{prop:safe_imp_incrbound} and
\ref{prop:incrbound_imp_incrjustified} allow us to show the following:
\begin{theorem}[Safety and P-incremental justification]
\label{thm:safeincrejust} \hfill
\begin{enumerate}[(i)]
\item If $\Gamma \vdash M : T$ is safe then $\sem{\Gamma \vdash M : T}$ is P-incrementally justified.
\item If $\vdash M : T$ is a closed simply-typed term and $\sem{\vdash M : T}$ is P-incrementally justified then the $\eta$-long form of the $\beta$-normal form of $M$ is safe.
\end{enumerate}
\end{theorem}
\proof
(i) Let $M$ be a safe simply-typed term. By Lemma
\ref{lem:safered_preserve_safety}, its $\beta$-normal form $M'$ is
also safe. By Proposition \ref{prop:safe_imp_incrbound}(i), $\tau(M')$
is incrementally-bound and by Proposition
\ref{prop:incrbound_imp_incrjustified}, $\sem{M'}$ is an
incrementally-justified. Finally the soundness of the game model gives
$\sem{M} = \sem{M'}$.  (ii) is a consequence of Lemma
\ref{lem:safered_preserve_safety}, Proposition
\ref{prop:incrbound_imp_incrjustified} and
\ref{prop:safe_imp_incrbound}(ii) and soundness of the game model.
\qed



Putting Theorem \ref{thm:safeincrejust}(i) and Lemma
\ref{lem:incrjustified_pointers_uniqu_recover} together gives:
\begin{proposition}
  \label{prop:safe_ptr_recoverable} In the game semantics of safe
  $\lambda$-terms, pointers emanating from P-moves are unnecessary
  {\it i.e.}~they are uniquely recoverable from the underlying sequences of
  moves and from O-moves' pointers.
\end{proposition}

% \begin{example} If justification pointers are omitted, the two
%   Kierstead terms from Example~\ref{ex:kierstead} have the same
%   denotation. In the safe lambda calculus the ambiguity disappears
%   since $M_1$ is safe whereas $M_2$ is not.
% \end{example}

In fact, as the last example highlights, pointers are entirely superfluous at
order $3$ for safe terms. This is because for
question moves in the first two levels of an arena,
the associated pointers are uniquely recoverable thanks to
the visibility condition. At the third level, the question moves are all P-moves therefore their associated pointers are uniquely recoverable by
P-incremental justification. This is not true anymore at order $4$:
Take the safe term $\psi:(((o^4,o^3),o^2),o^1) \vdash \psi (\lambda \varphi . \varphi a) : o^0$
for some constant $a:o$, where $\varphi:(o,o)$. Its strategy denotation contains plays whose underlying sequence of moves is $q_0 \, q_1 \, q_2 \, q_3 \, q_2 \, q_3 \, q_4$.
Since $q_4$ is an O-move, it is not constrained by
P-incremental justification and thus it can point to any of the two occurrences of $q_3$.\footnote{More generally,
a P-incrementally justified strategy can contain plays that are not ``O-incrementally justified'' since it must take into account any possible strategy incarnating its context, including those that are not P-incrementally justified.
In the given example, the version of the play that is not O-incrementally justified is involved in the strategy composition
$\sem{ \vdash M_2 : (((o,o),o),o)} ; \sem{ \psi:(((o,o),o),o) \vdash \psi (\lambda \varphi . \varphi a):o}$ where $M_2$ denotes the unsafe Kierstead term.}

\subsection*{Safe PCF and Safe Idealised Algol}

\pcf\ is the simply-typed lambda calculus augmented with basic
arithmetic operators, if-then-else branching and a family of recursion
combinator $Y_A : ((A,A),A)$ for any type $A$.  We define \emph{safe}
\pcf\ to be \pcf\ where the application and abstraction rules are
constrained in the same way as the safe lambda calculus.  This
language inherits the good properties of the safe lambda calculus: No
variable capture occurs when performing substitution and safety is
preserved by the reduction rules of the small-step semantics of
\pcf.

\subsubsection{Correspondence}

The computation tree of a \pcf\ term is defined as the least
upper-bound of the chain of computation trees of its \emph{syntactic approximants} \cite{abramsky:game-semantics-tutorial}.  It is
obtained by infinitely expanding the $Y$ combinator,
for instance $\tau(Y (\lambda f x. f x))$ is the tree representation
of the $\eta$-long form of the infinite term $(\lambda f x. f x)
 ((\lambda f x. f x) ((\lambda f x. f x) ( \ldots$

It is straightforward to define the traversal rules modeling the
arithmetic constants of \pcf. Just as in the safe lambda calculus
we had to remove @-nodes in order to reveal the game-semantic
correspondence, in safe \pcf\ it is necessary to filter out the
constant nodes from the traversals. The Correspondence Theorem for
\pcf\ says that the interaction game semantics is isomorphic to the
set of traversals disposed of these superfluous nodes. This can easily be shown
for term approximants. It is then lifted to full \pcf\ using the continuity of the function
$\travset(\_)^{\filter r}$ from the set of computation trees (ordered by the approximation ordering) to the set of sets of justified sequences of nodes (ordered by subset inclusion).


\notetoself{Update this: Computation trees of Safe \pcf\ terms are
incrementally-bound. Moreover since \pcf\ constant are of order $1$
at most, the constant traversal rules are all \emph{well-behaved}
(Lemma \ref{lem:sigma_order1_are_wellbehaved}) hence Lemma
\ref{lem:betanf_wellbehavedconst_trav_pview_red} (from the Appendix)
still holds and the game-semantic analysis of safety remains valid
for \pcf. Hence we have:
\begin{theorem}
\label{thm:safepcfpincr}
Safe PCF terms have P-incrementally justified denotations. \qed
\end{theorem}
}

Similarly, we can define safe \ialgol\ to be safe \pcf\ augmented with
the imperative features of Idealized Algol (\ialgol\ for short)
\cite{Reynolds81}.  Adapting the game-semantic correspondence and
safety characterization to \ialgol\ seems feasible although the
presence of the base type \iavar, whose game arena $\iacom^{\nat}
\times \iaexp$ has infinitely many initial moves, causes a mismatch
between the simple tree representation of the term and its game
arena. It may be possible to overcome this problem by replacing the
notion of computation tree by a ``computation directed acyclic
graph''.

The possibility of representing plays \emph{without some or all of
  their pointers} under the safety assumption suggests potential
applications in algorithmic game semantics. Ghica and McCusker
\cite{ghicamccusker00} were the first to observe that pointers are
unnecessary for representing plays in the game semantics of the
second-order finitary fragment of Idealized Algol ($\ialgol_2$ for
short). Consequently observational equivalence for this fragment can
be reduced to the problem of equivalence of regular expressions.  At
order $3$, although pointers are necessary, deciding observational
equivalence of $\ialgol_3$ is EXPTIME-complete
\cite{DBLP:journals/apal/Ong04,DBLP:conf/fossacs/MurawskiW05}. Restricting
the problem to the safe fragment of $\ialgol_3$ may lead to a lower
complexity.

% (note that it is unlikely to obtain the complexity PSPACE because the
% set of complete plays of the safe term $\lambda f^{(o,o),o} . f
% (\lambda x^o . x)$ is not regular \cite{DBLP:journals/apal/Ong04}).

% Murawski showed the undecidability of program equivalence in
% $\ialgol_i$ for $i\geq4$ by encoding Turing machine computations
% into a finitary $IA_4$ term \cite{murawski03program}. The term
% constructed being not safe, the proof cannot be transposed to the
% safe fragments. Hence the question remains of whether observational
% equivalence is decidable for the \emph{safe} fragments of these
% language.

%In \cite{Ong02}, one of us showed that observational equivalence for
% finitary second-order \ialgol\ with recursion ($\ialgol_2 + Y_1$) is
% undecidable. The proof consists in reducing the Queue-Halting
% problem to the observational equivalence of two $\ialgol_2 + Y_1$
% terms. The same reduction is still valid in the safe fragment of
% $\ialgol_2 + Y_1$.  Consequently, observational equivalence of safe
% $\ialgol_2 + Y_1$ is also undecidable.


\section{Further work and open problems}

The safe lambda calculus is still not well understood. Many basic
questions remain. What is a (categorical) model of the safe lambda
calculus? Does the calculus have interesting models?  What kind of
reasoning principles does the safe lambda calculus support, via the
Curry-Howard Isomorphism? Does the safe lambda calculus characterize a complexity class, in the same way that the simply-typed lambda calculus characterizes the polytime-computable numeric functions \cite{DBLP:conf/tlca/LeivantM93}?  Do incrementally-justified strategies compose? Is the addition of
unsafe contexts to safe ones conservative with respect to observational (or contextual) equivalence?
%Can we obtain a fully abstract model of safe PCF by suitably
%constraining O-moves ({\it i.e.}~``O-incremental justification'')?

With a view to algorithmic game semantics and its applications, it
would be interesting to identify sublanguages of Idealised Algol whose
game semantics enjoy the property that pointers in a play are uniquely
recoverable from the underlying sequence of moves. We name this class
PUR. $\ialgol_2$ is the paradigmatic example of a
PUR-language. Another example is \emph{Serially Re-entrant Idealized
  Algol} \cite{abramsky:mchecking_ia}, a version of \ialgol\ where
multiple uses of arguments are allowed only if they do not ``overlap
in time''.  We believe that a PUR language can be obtained by imposing
the \emph{safety condition} on $\ialgol_3$. Murawski \cite{Murawski2003} has
shown that observational equivalence for $\ialgol_4$ is
undecidable; is observational equivalence for \emph{safe} $\ialgol_4$
decidable?
