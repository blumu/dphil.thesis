\documentclass{article}
\usepackage{amsmath,amsthm,amssymb}
\usepackage{pdfsync}
\usepackage{url}
\usepackage{a4wide}
\usepackage{gamesem}

\newtheorem{theorem}{Theorem}[section]
\newtheorem{corollary}[theorem]{Corollary}
\newtheorem{lemma}[theorem]{Lemma}


\author{William Blum}
\title{Notes on HORS}

\begin{document}
\maketitle

\section{Notes on HORS}

What are the infinite words languages generated by HO recursion
scheme?

At order 0, do we get the omega-regular languages? (\ie languages of
infinite words  recognized by Buchi automata.)

At order 1, what do we get? Context-free languages of infinite words? Is
there such a thing?


\subsection{Urzycyn language U}

U is not context-free. The standard pumping lemma for context-free
language (Bar-Hillel lemma) cannot be used to prove it. Instead
one must Ogden's lemma, a a stronger version of the pumping lemma:

Let n be a number > 0.

Take $w = [  [^{n+1}  \underline{]^n}  ]  [  *^{n+3}.$

where the underlined ] denotes the n distinguished position of the Ogden Lemma.

Then w admits no decomposition that would allow to pump two (or even
more) subwords at the same time. Hence by Ogden Lemma, U is not
context free.

It is not even possible to decompose w so that three or more
subwords can be pumped at the same time!

This suggests the following conjecture:

\paragraph{Conjecture: Ogden pumping lemma for n-PDA}
Let L be a language recognized by an order-n PDA A.
Then there is a number n>0 such that for all word w with $|w|>0$
and all marking of at least n distinguished positions in $w$,
there is a decomposition $w = u p0  u2 p1  ...   un pn u{n+1}$ such that
- the word p0 p1 ... pn contains at least one marked position
- the word p0 u2 p1  ...  un pn contains at most n marked position
- for all i>=0, $u p0^i u2 p1^i  ...   un pn^i u{n+1}$ is in L.



\subsection{Conjecture}
The language L defined as follows is intrinsically unsafe (for any order).

$\Sigma = \{ a, b, c \}$

$L = \{ w *_1^{(|w|_b - |w|_a)} *_2^{(|w|_c - |w|_a)} \ | \
   w \in \Sigma^* \zand |w|_b \geq |w|_a \zand |w|_c \geq |w|_a \}$




Lemma Let A be a n-PDA then there is a n+1-PDA recognizing
the reverse of L(A).

Proof: just replace the production of terminals by a $push_1$
of a corresponding element on the stack followed by a $push_{n+1}$.
When the end of string symbol is emitted, the reversed string can be
retrieved using the following algorithm:

    while stack is not empty do
       emit top1(s)
       $pop_{n+1}$
    done
EOP





\subsection{safe HORS to HO-PDA}

Hague, Murawsky, Ong and Serre proposed an algorithm that transforms a
given order-n recursive scheme G to an order-n collapsible pushdown
 automaton (CPDA) $A_G$ that computes the value tree of G (i.e. the tree
 generated by G); precisely, $A_G$ computes exactly the traversals over the
 computation tree of G. (The paper is at
\url{http://users.comlab.ox.ac.uk/luke.ong/publications/stoc07-long.pdf},
the relevant part is section 5).

We prove that if G is safe, then $A_G$ is a pushdown automaton
 (of order n). This gives a new proof of the fact that order-n pushdown automata compute
 trees generated by order-n safe recursion schemes, independent of the
 original KNU02 approach. (There is a similar conjecture/result in the
 other direction.)


Proof idea:
In fact for incrementally-bound computation trees,
the n-CPDA that computes the traversals is  such that only links created with the $push_1$
operation will ever be used by a collapse operation later on.
Links that are created during the duplication of a stack by the
operations $push_j$ (for j>=2) can all be ignored. Hence the
collapse operation can safely be simulated with the operation
$pop_{n-ord(u)+1}$ where u is the $top_1$ lambda-node.


\subsection{2}

For a grammar G of order 2, the eta-long normal form of G introduces variables
of order 0 only.

Proof: First necessarily, the variables introduced are of order 1 at most
(otherwise the eta-expanded grammar would not be of order 2).

Suppose a subterm $M$ is eta-expanded to $\lambda u . M u$ where $u$ is a variable
of order $1$. Then this implies that $M$ is of order $2$. Consequently $M$ is a non-terminal!
Indeed any variable or an applicative term is of order $1$ at most in an order $2$ grammar.

However non-terminals of order $2$ cannot occur at operand position, otherwise this would imply that
the grammar is of order $3$. Hence $M$ is not eta-expanded when computing the eta-long normal form, which contradicts the hypothesis.


\subsection{2-DPDA equivalence problem:}

\emph{Question:} Is 2-DPDA equivalence decidable?
Equivalence of 1-DPDA was shown to be decidable but it is still not known
whether it is the case at higher orders.
Can we encode the set of traversals of a Safe $PCF_2$ term into a 2-DPDA?

\subsubsection{Does Undecidability of Finitary Safe $PCF_2$ implies
that of 2-DPDA equivalence?}

The simply-typed $\lambda$-calculus augmented with the
$Y$-combinator written $\lambda^\rightarrow + Y$ is as expressive as
recursion schemes in the sense that any \emph{ground type} term of
$\lambda^\rightarrow + Y$ with redexes of order $n$ at most can be
encoded into an order-$n$ recursion scheme (furthermore, the
transformation is such that safe terms correspond to safe recursion
schemes). For example, take the ground type term $s^n z$ where
$s:o\rightarrow o$, $z :o$ for some natural number $n$. This term
can be transformed into the recursion scheme containing a single
equation $S = s^n z$. The value tree of this recursion scheme is a
flat tree over the alphabet $\Sigma = \{s:o\rightarrow o, z:o\}$
where nodes are labelled $s$ and leaves are labelled $z$.


Suppose that we are able to do the same for any \pcf\ term, then we
could convert $SIM$ into an equivalent safe recursion scheme. And
since order-$n$ safe recursion schemes are as expressive as
$n$-DPDA\footnote{Knapik et.~al \cite{KNU02} gave an effective
transformation in both directions.}, we could build a
2-DPDA computing $SIM$. So the Queue program terminates iff
this DPDA is equivalent to the constant representing termination of the Queue program. But since Queue-Halting is undecidable, this would
imply the undecidability of 2-DPDA equivalence!



Instead of trying to encode  \pcf\ terms into recursion schemes, we
can look at the equivalent problem of simulating \pcf\ constructs in
$\lambda^\rightarrow + Y$. It is well-known that the Church Numerals
cannot represent the predecessor function in $\lambda^\rightarrow$ \cite{DBLP:journals/jacm/FortuneLO83}. It
is however possible to represent the predecessor function provided
that we allow a different representation of input and output numbers
\cite{DBLP:journals/jacm/FortuneLO83}. Unfortunately, the
substraction function is not definable in $\lambda^\rightarrow$
(equivalently  the equality test function $\lambda xy. {\sf if\ } x
= y {\sf\ then\ } u {\sf\ else\ } v$ is not definable)
\cite{DBLP:journals/jacm/FortuneLO83}. Hence we know that \pcf\
cannot be simulated into the simply-typed $\lambda$-calculus without
the $Y$ combinator.

Would the $Y$ combinator help to simulate the \pcf\ constructs? It
does not seem so. Indeed, as we have just seen, substraction --
which is definable in $\pcf$ using recursion and the conditional
operator -- is not definable in $\lambda^\rightarrow$. Throwing in
the recursion can be useful only if we have some destruction
function, such as the predecessor function. Indeed, only this can
permit us to make recursive call with decreasing parameters (which
is necessary to guarantee termination). Therefore we are bound to
use a numeral representation that can differ from input to output.
In other words, the type of the input may differ from the type of
the output. But with such difference in the representation,
recursion cannot be used anymore! Note that this problem is due to
the fact that \pcf\ uses simple types. With other kinds of types,
for instance second-order types, it would be possible to encode the
substraction operation.

%Suppose that equality test is definable by recursion by an the expression of  the form  $Y (\lambda f x y . e)$.
%
%A \emph{recursive call} in $e$ is an occurrence of $f$ in $e$.
%A recursive call is \emph{decreasing} if it is of the form $f t_x t_y$ where $t_x$ and $t_y$ denote $\lambda$-expressions in which $x$ and $y$ may occur free and such that for all $n_x, n_y \in \nat$: $t_x [\encode{n_x},\encode{n_x} /x,y] \betared^* \encode{m_x}$ and $t_y[\encode{n_x},\encode{n_x} /x,y] \betared^* \encode{m_y}$ for some $m_x,m_y \in \nat$ with  $m_x < n_x \vee  m_y < n_y$.
%
%We can assume that all recursive calls $f t_x t_y$ are such that
%$t_x$ and $t_y$ depends either on $x$ or $y$ or both (otherwise the recursive call can just be replaced by the $\beta$-normal form of $Y (\lambda f x y . e) t_x t_y$).
%
%In order to guarantee termination, at least one of the recursive call to $f$ in $e$ must be decreasing. This suggests that in order to define $e$, one needs to have access to a deconstructing function such as the predecessor function. Such function can only be encoded if we allow input and output value to have different representations.



\bibliographystyle{plain}
\bibliography{../bib/gamesem,../bib/higherorder,../bib/lambdacalculus}


\end{document}
