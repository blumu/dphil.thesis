\documentclass{article}
\usepackage{amsmath, amsthm, amssymb}
\usepackage[defblank]{paralist}
\usepackage{shadowbox}
\usepackage{a4wide}

\newcommand{\ord}{\mathop{\mathrm{ord}}}
\newcommand{\subseqof}{\sqsubseteq}
\newcommand{\prefixof}{\leqslant}
\newcommand{\suffixof}{\geqslant}
\newcommand\defname[1]{{\bf\em #1}\index{#1}}

\newcommand{\theroot}{\circledast} % the root of the computation tree

% misc
\newcommand\assignar\leftarrow

\newtheorem{theorem}{Theorem}[section]
\newtheorem{corollary}[theorem]{Corollary}
\newtheorem{conjecture}[theorem]{Conjecture}

\newtheorem{lemma}{Lemma}[section]
\newtheorem{proposition}{Proposition}[section]

\theoremstyle{remark}
\newtheorem{remark}{Remark}[section]
\newtheorem{example}{Example}[section]
\newtheorem{property}{Property}[section]

\theoremstyle{definition}
\newtheorem{definition}{Definition}[section]
\newtheorem{algorithm}{Algorithm}[section]

% order-decomposition of a 1-stack
\newcommand\orddec\overline


% ABBREVIATIONS
\def\ie{{\it i.e.}\ }
\def\eg{{\em e.g.}}
\def\cf{{\em cf.}}

% set theory
\newcommand{\makeset}[1]{\{\,{#1}\,\}}
\newcommand\inter{\cap}
\newcommand\union{\cup}
\newcommand\Union{\bigcup}
\newcommand\prefset{\textsf{Pref}}
\newcommand{\relimg}[1]{{(\!| #1 |\!)}}
\newcommand\nat{\mathbb{N}}


% Note to myself
\newcommand\notetoself[1]{
\bigskip \noindent \fbox{
\begin{tabular}{cl}
\textdbend &
\begin{minipage}{0.80\textwidth}
#1
\end{minipage}
\end{tabular}
} }


\author{William Blum}
\title{Encoding of a safe order-$n$ recursion scheme into a $n$-PDA}

\begin{document}
\maketitle
\begin{abstract}
In \cite{hague-sto07}, Hague, Murawski, Ong and Serre introduced
higher-order collapsible pushdown automata (CPDA) and showed their
equivalence with higher-order recursion scheme. They proposed an
algorithm that transforms a given order-$n$ recursive  scheme G to
an equivalent order-$n$ CPDA. Here we show that if the recursion
scheme is safe then the generated automaton does not require the
extra power of a CPDA and can in fact be simulated by an order
$n$-PDA ({\it i.e.} push-down automaton). The proof does not make
use of the type-homogeneity constraint (\cite{KNU02}).
\end{abstract}


\section{Preliminaries}

Take an order-$n$ {\bf safe} recursion scheme $G$. Let $Gr(G)$ be the computation graph of $G$.
We consider the $n$-CPDA $CPDA(G)$ as defined in \cite[Definition 5.2]{hague-sto07}.
Recall that the stack-alphabet $\Gamma$ is defined as the set of nodes of the computation graph.
An element of the stack is written $a^{(j,k)}$ where $a\in \Gamma$ and the exponent $(j,k)$
encodes the pointer associated to the stack symbol. We observe that $CPDA(G)$ is such that the link associated to a lambda node $\lambda \overline{\xi}$ is always of the form $(n-\ord{\lambda \overline{\xi}}+1, k)$, therefore the first component of the link is always recoverable from the node itself. For this reason we omit it when representing stack symbols {\it i.e.} we write $\lambda \overline{\xi}^{k}$ to mean $\lambda \overline{\xi}^{(n-\ord{\lambda \overline{\xi}}+1,k)}$.


{\it Remark:} It is possible to redefine $CPDA(G)$ from
\cite{hague-sto07} in a more compact way by merging the subcases of
$(V_1)$ and $(V_0)$. This is done as follows: In case (A), when
pushing the prime child of an application node $@$ on the stack, we
associate a link to it that points to the preceding stack symbol in
the top $1$-stack {\it i.e.} the $@$-node itself.
 This modification permits us to avoid the consideration on $j$ (the child-index of $u$'s binder) in the
 cases ($V_0$) and ($V_1$). The sequences of instruction $pop_1^{p+1}$ can now be replaced by
 $pop_1^p ; collapse$. This gives us the CPDA $CPDA'(G)$ describe in
Figure \ref{fig:cpdaprime}.
\begin{figure}[htbp]
\begin{center}
\makebox{
\begin{shadowbox}[10cm]
Suppose $u$ is a variable which is the $i$-parameter of
its binder and let $p$ be the span of $u$.
\begin{itemize}
\item $(V_1)$ If the variable has order $l\geq 1$, then
$$\delta(u) = push_{n-l+1} ; pop_1^p ; collapse;push_1 E_i(top_1)^{(n-l+1,1)}$$
\item $(V_0)$ If the variable is of ground type then
$$\delta(u) = pop_1^p ; collapse;push_1 E_i(top_1)$$
\end{itemize}
\caption{CPDA'(G): a compact version of CPDA(G).}
\label{fig:cpdaprime}
\end{shadowbox}
}
\end{center}
\end{figure}
This definition is more compact, however it makes it harder to
understand how the CPDA works. We will therefore stick to the
original definition for the purpose of proving our result and we
will make an explicit case analysis on $j$ in the proof.

In the following, we use the term \defname{configuration} to refer to an order-$n$ stack where $n$ is the order
of the recursion scheme (which is also the order of CPDA(G)).


\subsection{Incremental order-decomposition}

Let $s$ be a 1-stack. It can then be written as
$$ s = u_{r+1} \cdot \lambda \overline{\eta}_r^{k_r} \cdot u_r \cdot
\ldots \cdot \lambda \overline{\eta}_1^{k_1} \cdot  u_1 $$
for some $r\geq 0$ where
\begin{compactitem}
\item  $\lambda \overline{\eta}_1^{k_1}$ is the
last $\lambda$-node in $s$ with order strictly positive;

\item for $1 \leq l \leq r$, $\lambda
\overline{\eta}_l^{k_l}$ is the last $\lambda$-node in $s_{\prefixof
\lambda \overline{\eta}_{l-1}^{k_{l-1}}}$ with order strictly
greater than $\ord{\lambda \overline{\eta}_{l-1}^{k_{l-1}}}$,

\item  $r$ is defined as the smallest number such that
$s_{\prefixof \lambda \overline{\eta}_{r}^{k_{r}}}$ does not contain
any lambda-node of order strictly greater than $\lambda
\overline{\eta}_{r}^{k_{r}}$.
\end{compactitem}

Thus:
\begin{compactitem}
\item for $1 \leq k \leq r$, all the lambda-nodes occurring in $u_l$ have order
strictly smaller than $\ord{\lambda \overline{\eta}_l}$;
\item For $l>l'$ we have $\ord{\lambda \overline{\eta}_l^{k_l}}
> \ord{\lambda \overline{\eta}_{l'}^{k_{l'}}}$;
\item We have $r=0$ iif there is no lambda node in $s$ with
positive order.
\end{compactitem}
\smallskip

We call \defname{incremental order-decomposition} (or just
order-decomposition) of $s$ and we write $\orddec{s}$ to denote the
subsequence $\lambda
\overline{\eta}_r^{k_r} \ldots \overline{\eta}_1^{k_1}$ of $s$ consisting of the lambda-nodes $\lambda
\overline{\eta}_l^{k_l}$ defined above. For $1\leq l \leq r$, $l$ is called the \defname{incremental index} of the lambda-node $\lambda \overline{\eta}_l^{k_l}$ in $\orddec{s}$.
 Note that this sequence is uniquely determined.

Convention: Usually the operation $top_i~s$, that returns the top $(i-1)$-stack of a higher-order stack $s$, does not preserve the dangling pointers resulting from the extraction of the stack. Here, we define $top_i$ in such a way that the pointers are preserved. In the implementation side, this just means that we keep the pair of integers that encodes each pointer. This convention allows us to
extend the definition of the order-decomposition to higher-level stacks as follows:
\begin{definition}
The incremental order-decomposition
of a higher-order stack $s$ is defined as the incremental
order-decomposition of $top_2 s$ \ie $\orddec{s} = \orddec{top_2~s}$.
\end{definition}


\begin{lemma}
\label{lem:push1pop1_orderdecompo} Let $s$ be a higher-order stack
with the following order-decomposition $\orddec{s} = \langle \lambda
\overline{\eta}_r^{k_r}, \ldots, \lambda \overline{\eta}_1^{k_1}
\rangle$. Then
\begin{enumerate}[i.]
\item For all lambda node $a \in \Gamma$ and link $(j,k)$ we have
 $$ \orddec{push_1 a^{(j,k)} s} = \left\{
                                       \begin{array}{ll}
                                        \langle a^k \rangle, &  \hbox{if $\ord{a} > \ord{\lambda \overline{\eta}_r^{k_r}}$}; \\
                                         \langle \lambda \overline{\eta}_r^{k_r}, \ldots, \lambda
\overline{\eta}_i^{k_i}, a^k \rangle, & \hbox{otherwise, where $i = \min \{ i \ | \ \ord{a} <
\ord{\lambda \overline{\eta}_i^{k_i}} \}$.}
                                       \end{array}
                                     \right.$$

\item For all non-lambda node $a \in \Gamma$ and link $(j,k)$ we have
$$ \orddec{push_1 a^{(j,k)} s} = \orddec{s} \ .$$

\item If $top_1 s$ is not a lambda-node then
$$ \orddec{pop_1 s} = \orddec{s} \ .$$
\end{enumerate}
\end{lemma}
\begin{proof}
  Trivial from the definition of $\orddec{t}$ and
  $\orddec{s}$.
\end{proof}

We call reachable configuration of $CPDA(G)$, any configuration that can be attained starting from the initial configuration (\ie $push_1 \theroot \bot_n$) by performing one or more application of the steps (A), (S), (L), $(V_1)$, $(V_0)$ of the algorithm defining $CPDA(G)$. Note that within a step, the intermediate configurations reached by the internal transitions are not considered as reachable.

\begin{lemma}[Incremental binders are in the order-decomposition]
\label{lem:binder_in_ordecompos} Let $c$ be a reachable
configuration of $CPDA(G)$ such that $top_1\
c$ is a variable $x$ of order $l\geq 0$. Then
\begin{enumerate}[i.]
\item $\orddec{c}$ contains at least a node with order strictly
greater than $l$.
\item Let $\lambda \overline{\xi}$ be the last lambda node
in $\orddec{c}$ verifying the first condition. Then $\lambda \overline{\xi}$ is precisely $x$'s binder.
\end{enumerate}
\end{lemma}
\begin{proof}
\begin{enumerate}[i.]
\item By Corollary 8 from \cite{hague-sto07}, the $top_2$
     stack of a configuration contains the P-view of some
    traversal whose last visited node is the $top_1$ symbol, and
    by Proposition 6 from \cite{OngLics2006}, the P-view of a
    traversal is exactly the path from the last visited node to
    the root of the computation tree (obtained by unfolding the
    computation graph). Hence the binder of $x$, whose order
    is strictly greater than $\ord{x}$, occurs in $top_2 c$.
    Consequently $\orddec{c}$ must contain at least one node of order strictly greater than $l$.

\item Since the recursion scheme is safe, $x$ is
 incrementally bound (see \cite{blumong:safelambdacalculus})
 which means that its binder is the first $\lambda$-node in the
 path to the root in the computation tree with order strictly
 greater than $x$.
\end{enumerate}
\end{proof}

\subsection{Safe higher-order stack}
\subsubsection{Definition}

\begin{definition}[Safe higher-order stack]
\label{dfn:safestack} Let $s$ be an order-$j$ stack for $j\geq1$ and $r
= |\orddec{s}|$.
We say that $s$ is \defname{safe} iif
    \begin{enumerate}[1.]
    \item $\orddec{s} = \langle \lambda \overline{\eta}_r^1, \ldots ,
    \lambda \overline{\eta}_1^1 \rangle$ \ie the $k$-components in $\orddec{s}$ are all equal to $1$;
    \item for all $1 \leq q \leq r$ such that $n-\ord{\lambda \overline{\eta}_q}+1 < \ord{s}$ we have that
    $collapse (s_{\prefixof \lambda \overline{\eta}_q^1}) = pop_{(n-\ord{\lambda \overline{\eta}_q}+1)}\  s_{\prefixof \lambda \overline{\eta}_q^1}$ is safe.
    \end{enumerate}
We say that a stack $s$ is \defname{$q$-safe} iff $pop_1^{q-1} s$ is safe.
\end{definition}

Equivalently, this definition can be reformulated indictively as follows:
$s$ is safe if and only if
    $\orddec{top_2(s)} = \langle \lambda \overline{\eta}_r^1, \ldots ,
    \lambda \overline{\eta}_1^1 \rangle$
    and for all $1 \leq q \leq r$ such that $n-\ord{\lambda \overline{\eta}_q}+1 < \ord{s}$,
    $collapse (s_{\prefixof \lambda \overline{\eta}_q^1})$ is safe.


%\newcommand\saferel\rightsquigarrow
%\begin{definition}
%Let $q\geq 0$. We define the relation $\saferel$ on higher-order
%stacks of any order as follows. $s \saferel s'$ just when
%$\orddec{s} = \langle \lambda \overline{\eta}_r^1, \ldots ,
%    \lambda \overline{\eta}_1^1 \rangle$ \ie the $k$-components in $\orddec{s}$ are all equal to $1$ and
%     $s' = collapse (s_{\prefixof \lambda \overline{\eta}_q}) $ for some $1 \leq q \leq r$ such that $n-\ord{\lambda \overline{\eta}_q}+1 < \ord{s}$.
%
%We write $\saferel^*$ to denote the transitive reflexive closure of $\saferel$.
%\end{definition}
%
%The definition of safe stacks can be reformulated using the relation $\saferel$ as follows:
%\begin{lemma}
%\label{lem:qsafety_equivdef}
% Let $s$ be an order-$j$ stack with $j\geq 2$. Then
%  $s$ is safe if and only if $top_2~s$ is safe and for all $s'$ such that $s \saferel s'$, $s'$ is also safe.
%\end{lemma}

\subsubsection{Stack operations preserving stack safety}
\begin{lemma}
\label{lem:push1pop1_preserves_safety} Let $s$ be a safe higher-order
stack. Then:
\begin{enumerate}
  \item If $top_1\ s$ is not a lambda-node then $pop\ s$ is safe;
  \item for any any lambda-node $a$, $1 \leq j \leq n$, $push_1\ a^{(j,1)}\ s$ is also safe;
  \item for any non-lambda node $a$, $1 \leq j \leq n$, $k \geq 1$, $push_1\ a^{(j,k)}\ s$ is safe.
\end{enumerate}
\end{lemma}
\begin{proof}
This is a direct consequence of Lemma
\ref{lem:push1pop1_orderdecompo}.
\end{proof}


\begin{lemma}
\label{lem:top_qsafe} If $s$ is safe then $top_{\ord{s}}~s$ is safe.
\end{lemma}


Let $s$ be a higher-order stack. We define $s^{\langle j \rangle}$ as the operation that replaces
every link occurring in $s$ of the form $(j,k)$ by $(j,k+1)$. Formally,
\begin{align*}
{a^{(j,k)}}^{\langle j \rangle} &= a^{(j,k)}   \\
{a^{(j,k)}}^{\langle j' \rangle} &= a^{(j,k+1)} &   \mbox{when $j\neq j'$,}\\
[s_1 \ldots s_p]^{\langle j \rangle} &= [s_1^{\langle j \rangle} \ldots s_p^{\langle j \rangle}] \ .
\end{align*}

\begin{lemma}
\label{lem:incrk_qsafe}
Let $n>l\geq 0$. Let $s$ be a stack of level $2\leq \ord{s} <n$.
 Let $b$ be the incremental index of the last lambda node in $\orddec{s}$
 with order $>l$.
If $s$ is safe then $s^{\langle n-l+1 \rangle}$ is $b$-safe.
\end{lemma}
\begin{proof}
  By induction on the order of $s$.
The base case is $\ord{s} = 2$.
Since $s$ is safe we have:
\begin{equation}
 \orddec{s} = \langle \lambda \overline{\eta}_r^1
, \ldots, \lambda \overline{\eta}_1^1   \rangle \ . \label{eqn:orddec_s}
\end{equation}
By definition of the operation $s^{\langle n-l+1 \rangle}$, the pointers in $s = top_2\ s$ are updated as follows:
the $k$ component of each link is increased by one if the order of
the stack symbol is $l$ and is kept unchanged otherwise.
Hence we have:
\begin{equation}
\orddec{s^{\langle n-l+1 \rangle}} = \langle
\lambda \overline{\eta}_r^1, \ldots,  \lambda \overline{\eta}_{b}^1, \lambda \overline{\eta}_{b-1}^{k_{b-1}}, \ldots,
 \lambda \overline{\eta}_1^{k_1} \rangle
\ .
\end{equation}
for some $k_{b-1}, \ldots , k_1 \leq 2$ and where $q$ is precisely the incremental index of the last lambda-node in $\orddec{s}$ with order $>l$ \ie $q=b$. Thus $s^{\langle n-l+1 \rangle}$ is $b$-safe.

Step case: $n>\ord{s}>2$.
By definition of the order-decomposition we have
$\orddec{s^{\langle n-l+1 \rangle}} = \orddec{top_2 (s^{\langle n-l+1 \rangle})}$.
Clearly we have $top_2 (s^{\langle n-l+1 \rangle}) = (top_2 (s))^{\langle n-l+1 \rangle}$.
Since $s$ is safe so is $top_2(s)$, therefore by the base case of the induction we have that $top_2 (s^{\langle n-l+1 \rangle})$ is $b$-safe. To conclude that $s^{\langle n-l+1 \rangle}$ is also $b$-safe, we need to show the second condition of Def.~\ref{dfn:safestack},
that is to say for all $b\leq q \leq r$ such that
$n-\ord{\lambda \overline{\eta}_{q}} +1 < \ord{s^{\langle n-l+1 \rangle}}$, $collapse(s_{\prefixof \lambda \overline{\eta}_{q}})$ is safe.

This is clearly true. Indeed, the only symbols that are updated in $s$ by the operation $s \mapsto s^{\langle n-l+1 \rangle}$ are those occurring in the top order-$1$ stack $top_2~s$. Thus since the first $r-b+1$ lambda-nodes in $\orddec{s^{\langle n-l+1 \rangle}}$ and $\orddec{s}$ coincide, this implies that $collapse(pop_1^{q} s^{\langle n-l+1 \rangle}) = collapse(pop_1^q s)$ for all $b \leq q \leq r$.
Hence the safety of $s$ implies the safety of $pop_1^{q}~s^{\langle n-l+1 \rangle}$ \ie the $q$-safety of $s^{\langle n-l+1 \rangle}$.
\end{proof}


\notetoself{The proof of the  following lemma needs to be reviewed.}
\begin{lemma}
\label{lem:cons_qsafety} Let $u = [s_1 \ldots s_{k+1}~ t]$ be a higher-order stack. If
\begin{enumerate}[1.]
\item $pop_{\ord{u}} u = [s_1 \ldots s_{k+1}]$ is safe and of order $\geq 2$;
\item and the $(\ord{s}-1)$-stack $top_{\ord{u}} u = t$ is safe,
\end{enumerate}
then $u$ is safe.
\end{lemma}
\begin{proof}
We use Lemma \ref{lem:qsafety_equivdef}. Firstly, since $t$ is
safe so is $top_2 [s_1 \ldots s_{k+1} \cdot t] = top_2 t$. Now suppose
that $[s_1 \ldots s_{k+1} \cdot t] \saferel s'$, we need to show
that $s'$ is safe. In other words we need to prove that for all
$s''$ such that $s' \saferel s''$, $s''$ is safe. We proceed by
induction on the number of $\ord{s'}-1$-stacks in $s'$.

\hrulefill

 if $t\saferel_q t'$ for some $t'$ then for all lambda node
$\lambda \overline{\xi}$ of order $n-\ord{s}+1$ occurring in
$\orddec{t'}$ we have that $collapse([s_1 \ldots s_{k+1} \cdot
t'_{\prefixof{\lambda \overline{\xi}}}])$ is safe.


\hrulefill

The stack symbols that are duplicated by the $push_j$ operation are
the ones lying in the $top$ order-$(j-1)$ stack. More precisely,
suppose $top_{j+1}\ c = [c_1\ \ldots\ c_{l+1}]$ then $top_{j+1}\
(push_j\ c) = [c_1\ \ldots\ c_{l+1}\ c_{l+1}^{\langle j \rangle}]$
and the elements that are created lies in $top_j\ (push_j\ c) =
c_{l+1}^{\langle j \rangle}$.

Take any such element $a^{(l,k)}$. If $l=j$ then the link points to
the same prefix stack as some other stack symbol in $top_j\ c$, thus
the induction hypothesis permits to conclude.

If $l \neq j$ then the link points somewhere in $top_j\ (push_j\ c)$
and in that case $top_2( collapse(c_{\prefixof{a}})) = top_2(
collapse( pop_{j+1} c_{\prefixof{a}}))$ so we can conclude by the
induction hypothesis.

\end{proof}


\begin{lemma}
\label{lem:pushj_safe_implies_b-safe} Let $n>l\geq 0$ and $s$ be a higher-order stack
of level $2 \leq n-l+1 \leq \ord{s} \leq n$. Let $b$ be the incremental index of the last lambda node in $\orddec{s}$ with order $>l$. Then $push_{n-l+1}\ s$ is $b$-safe.\footnote{We reproduce here the definition of the
CPDA operation $push_j$ from \cite{hague-sto07}:
$$ push_j \underbrace{[ s_1 \ldots s_{l+1} ]}_{s} =
\left\{
  \begin{array}{ll}
\    [s_1\ \ldots\ s_{l+1}\ s_{l+1}^{\langle j \rangle}]  &\hbox{if $j = \ord{s}$;}\\
\    [s_1\ \ldots\ s_{l+1}\ push_j\ s_{l+1}]  &\hbox{if $j<\ord{s}$.}
 \end{array}
\right.
$$}
\end{lemma}
\begin{proof}
Let $s=[s_1 \ldots s_{c+1}]$ be any safe higher-order stack of order. We show that $push_{n-l+1}~s$ is $b$-safe by finite induction on the order of $s$.
    \begin{compactitem}
      \item Base case: $\ord{s} = n-l+1 $. We have
    $push_{n-l+1}~s = [ s_1 \ldots s_{c+1} s_{c+1}^{\langle n-l+1
    \rangle}]$.

    Since $s$ is safe, by Lemma \ref{lem:top_qsafe} so is $s_{c+1}$
    which in turn implies that $s_{c+1}^{\langle n-l+1\rangle}$ is
    $b$-safe by Lemma \ref{lem:incrk_qsafe}. Therefore by Lemma
    \ref{lem:cons_qsafety},  $[ s_1 \ldots s_{c+1} s_{c+1}^{\langle n-l+1
    \rangle}]$ is $b$-safe.

      \item Step case: $\ord{s} > n-l+1$. We have
    $push_{n-l+1}~s = [ s_1 \ldots s_{c+1} push_{n-l+1}~s_{c+1}]$.

    $s_{c+1}$ is safe by Lemma \ref{lem:top_qsafe} thus by the
    induction hypothesis $push_{n-l+1}~s_{c+1}$ is $b$-safe and therefore by Lemma \ref{lem:cons_qsafety} so
    is $[ s_1 \ldots s_{c+1} push_{n-l+1}~s_{c+1}]$.
    \end{compactitem}
\end{proof}





\section{Simulation}

We claim that provided that the input recursion scheme is safe, the
$collapse$ operation in $CPDA(G)$ can be simulated by the operation
$pop_{n-\ord{top_1(s)}+1}$. (If we use the formulation $CPDA'(G)$ instead, then
$collapse$ is simulated by $pop_{n-\ord{top_1(s)}+1}$ if $top_1(s)$
is a $j$-child with $j\geq 2$ and by $pop_1$ if it is a prime
child).

\section{Proof of correctness}

\begin{lemma}
Let $s$ be a reachable configuration (with respect to the
$\rightarrow$-step relation\footnote{Recall that $s\rightarrow s'$
just if $s' = \delta(top_1 s)(s)$ where $\delta$ is the transition
function defined in \cite[Figure 2]{hague-sto07}.}). Then:
\begin{enumerate}[i.]
\item $s$ is safe;
\item Whenever the $collapse$ operation is executed during a
transition $s \rightarrow collapse(s)$, the $k$-component of the link attached to the symbol $top_1(s)$ is equal to $1$.
\end{enumerate}
\end{lemma}
\begin{proof}
We prove (i) and (ii) simultaneously by induction on the number of
$\rightarrow$-steps. Let $u = top_1\ s$ and $s' =\delta(u)(s)$.
Suppose that $s$ verifies (i) and (ii) we prove that so does $s'$:
\begin{itemize}
\item[Case (A)] Pushing @ on the top $1$-stack preserves safety (Lemma
\ref{lem:push1pop1_preserves_safety}), therefore by the
induction hypothesis, condition (i) also holds for $s'$. Since
the $collapse$ operation is not executed, (ii) holds as well.

\item Case (S)
We have $s' = push_1 a^{(n-\ord{a}+1,1)} s$ where $a$ is a
lambda-node. Lemma \ref{lem:push1pop1_preserves_safety} and the
induction hypothesis implies that conditions (i) also holds for
$s'$. Condition (ii) holds trivially.

\item Case (L) Again, Lemma \ref{lem:push1pop1_preserves_safety}
shows that (i) holds and (ii) trivially holds.

\item Case ($V_1$) Suppose $u$ is labelled by a variable $x$ of order $l\geq 1$.

Since $s$ is safe we have $\orddec{s} = \langle \lambda
\overline{\eta}_r^1 , \ldots, \lambda \overline{\eta}_1^1
\rangle$.
Since the recursion-scheme $G$ is safe, by Lemma \ref{lem:binder_in_ordecompos}, $x$'s binder is precisely the
last node in $\orddec{s}$ with order strictly greater than $l$.
Hence by Lemma \ref{lem:pushj_safe_implies_b-safe}
we have that $push_{n-l+1}\ s$ is $b$-safe where $b$ is the
incremental index of $x$'s binder in $\orddec{c}$.

%\begin{itemize}
%\item Suppose that $\lambda \overline{\eta}_{b}$ is a prime child, then $top_1 ((push_{n-l+1};pop_1^{p+1}) s) =
%@$ and therefore by Lemma
%\ref{lem:push1pop1_preserves_safety}
%$(push_{n-l+1};pop_1^p;pop_1) s$ is safe.
%
%By Lemma \ref{lem:push1pop1_preserves_safety} again, $s' =
%\left(push_{n-l+1};pop_1^p;push_1 E_i(top_1)^{(n-l+1,1)}
%\right) s$ is safe thus condition (i) holds. Moreover
%condition (ii) holds trivially.
%
%\item Suppose that $\lambda \overline{\eta}_{b}$ is a $j$-child with $j>0$.
%\end{itemize}

After performing $pop_1^p$, the top stack symbol contains $x$'s
binder which is precisely $\lambda \overline{\eta}_{b}^1$. The
link attached to this node has a $k$-component equal to $1$, this guarantees
that condition (ii) is verified when the $collapse$ instruction
is executed.

Furthermore, since $push_{n-l+1}\ s$ is $b$-safe, we have that
$t = collapse \left( (push_{n-l+1}\ s)_{\prefixof \lambda
\overline{\eta}_{b}} \right) = (push_{n-l+1};pop_1^p;collapse)
s$ is safe. Since $s' = push_1 E_i(top_1)^{(n-l+1,1)}\ t$, by
Lemma \ref{lem:push1pop1_preserves_safety} $s'$ is also safe
thus condition (i) holds.

\item ($V_0$) This case is treated identically to ($V_1$).
\end{itemize}
\end{proof}

\bibliographystyle{abbrv}
\bibliography{../bib/dphil-all}

\end{document}
