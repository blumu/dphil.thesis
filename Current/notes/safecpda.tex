\documentclass{article}
\usepackage{gamesem}
\usepackage{amsmath, amsthm, amssymb}
\usepackage[defblank]{paralist}
\usepackage{shadowbox}
\usepackage{a4wide}
\usepackage{manfnt}
\usepackage{pdfsync}

\newtheorem{theorem}{Theorem}[section]
\newtheorem{corollary}[theorem]{Corollary}
\newtheorem{conjecture}[theorem]{Conjecture}

\newtheorem{lemma}{Lemma}[section]
\newtheorem{proposition}{Proposition}[section]

\theoremstyle{remark}
\newtheorem{remark}{Remark}[section]
\newtheorem{example}{Example}[section]
\newtheorem{property}{Property}[section]

\theoremstyle{definition}
\newtheorem{definition}{Definition}[section]
\newtheorem{algorithm}{Algorithm}[section]

% order-decomposition of a 1-stack
\newcommand\orddec{\sf orddec}

\author{William Blum}
\title{Encoding of a safe order-$n$ recursion scheme into a $n$-PDA}

\begin{document}
\maketitle
\begin{abstract}
In \cite{hague-sto07}, Hague, Murawski, Ong and Serre introduced
higher-order collapsible pushdown automata (CPDA) and showed their
equivalence with higher-order recursion scheme. They proposed an
algorithm that transforms a given order-$n$ recursive  scheme G to
an equivalent order-$n$ CPDA. Here we show that if the recursion
scheme is safe then the generated automaton can in fact be simulated by an order
$n$ (non-collapsible) push-down automaton (PDA). The proof does not make
use of the type-homogeneity constraint. This contrasts with the original proof
of equi-expressivity (in term of generated trees) of PDAs and safe recursion schemes \cite{KNU02}.
\end{abstract}


\section{Preliminaries}

\subsection{Notations}

We fix a {\bf safe} higher-order recursion scheme $G$ of order $n$. Let $Gr(G)$ be the computation graph of $G$.
We consider the $n$-CPDA $CPDA(G)$ as defined in \cite[Definition 5.2]{hague-sto07}.
Recall that the stack-alphabet $\Gamma$ is defined as the set of nodes of the computation graph.
An element of the stack is written $a^{(j,k)}$ where $a\in \Gamma$ and the exponent $(j,k)$
encodes the pointer associated to the stack symbol. We observe that $CPDA(G)$ is such that the link associated to a lambda node $\lambda \overline{\xi}$ is always of the form $(n-\ord{\lambda \overline{\xi}}+1, k)$, therefore the first component of the link is always recoverable from the node itself. For this reason we omit it when representing stack symbols {\it i.e.} we write $\lambda \overline{\xi}^{k}$ to mean $\lambda \overline{\xi}^{(n-\ord{\lambda \overline{\xi}}+1,k)}$.


We call \defname{configuration} any order-$n$ stack where $n$ is the order
of $CPDA(G)$ \ie the order of the recursion scheme $G$.


\subsection{Remarks}

The purpose of the CPDA $CPDA(G)$ is to compute the set of traversals of the computation tree of $G$. One can easily transform $CPDA(G)$ into an automaton that ``prints out'' the traversal that is being computed. This can be done by changing the behaviour of the $push_1$ operation to make it print out the input element before pushing it on the stack. The justification pointers can then be recovered
inductively using the node labels: to recover the justifier of a variable node, we just need to look for the only node-occurrence that binds it in the P-view at that point (which is computable by the induction hypothesis).
Prime lambda nodes always point to their immediate predecessor. For non-prime lambda nodes, the justifier is always the occurrence preceding the justifier of the variable node preceding the lambda node.

Since some of the transitions of the CPDA, such as $collapse$, are destructive, the current content of stack does not contain sufficient information in order to reconstruct the traversal that is being computed.
Nevertheless, two very useful pieces of information can be recovered from the configuration-stack: namely the O-view and the P-view of the traversal.

Let $c$ be a configuration. The \defname{long O-view}
(resp.\ O-view, reps.\ P-view) of the traversal that is currently computed by the configuration $c$, written respectively $\longoview{c}$, $\oview{c}$ and $\pview{c}$, can be recovered as follows:
\begin{align*}
  \longoview{s} &= \epsilon & \mbox{if $top_1 s$ is undefined;} \\
      &=   \longoview{pop_1 s} \cdot top_1 s & \mbox{if $top_1 s \in N_{\sf var}$, $top_1 s$ pointing to its immediate predecesor;} \\
      &=   \longoview{pop_1 s} \cdot top_1 s & \mbox{if $top_1 s \in N_@$, $@$ having no pointer;} \\
      &=   \longoview{collapse\,s} \cdot top_1 s & \mbox{if $top_1 s \in N_\lambda^{\sf prime}$, $top_1 s$ pointing to its immediate predecesor;} \\
      &=   \longoview{collapse\,s} \cdot top_1 s & \mbox{if $top_1 s \in N_\lambda\setminus N_\lambda^{\sf prime}$, $top_1 s$ pointing to $\ip(\jp(collapse\,s))$.} \\
\end{align*}
The O-view is defined similarly to the long-O-view except that the calculation stops when an @-node is reached:
\begin{align*}
  \oview{s}  &=   top_1 s & \mbox{if $top_1 s \in N_@$, $@$ having no pointer;} \\
\end{align*}
As shown in the original paper, the P-view is given by $top_2\,c$.


\subsection{Incremental order-decomposition}

We first define the notion incremental order-decomposition for order-$1$ stack and we then generalize it to higher-order stack.

Let $s$ be a 1-stack. For any $l\in \nat$, $s$ can then be written
$$ s = u_{r+1} \cdot \lambda \overline{\eta}_r^{k_r} \cdot u_r \cdot
\ldots \cdot \lambda \overline{\eta}_1^{k_1} \cdot  u_1 $$
where
\begin{compactitem}
\item  $\lambda \overline{\eta}_1^{k_1}$ is the
last $\lambda$-node in $s$ with order strictly greater than $l$;

\item for $1 < l \leq r$, $\lambda
\overline{\eta}_l^{k_l}$ is the last $\lambda$-node in $s_{\prefixof
\lambda \overline{\eta}_{l-1}^{k_{l-1}}}$ with order strictly
greater than $\ord{\lambda \overline{\eta}_{l-1}^{k_{l-1}}}$,

\item  $r$ is defined as the smallest number such that
$s_{\prefixof \lambda \overline{\eta}_{r}^{k_{r}}}$ does not contain
any lambda-node of order strictly greater than $\lambda
\overline{\eta}_{r}^{k_{r}}$.
\end{compactitem}

\noindent In particular:
\begin{compactitem}
\item for $1 \leq k \leq r$, all the lambda-nodes occurring in $u_l$ have order
strictly smaller than $\ord{\lambda \overline{\eta}_l}$;
\item for $1\leq l<l'\leq r$ we have $\ord{\lambda \overline{\eta}_l^{k_l}}
< \ord{\lambda \overline{\eta}_{l'}^{k_{l'}}}$;
\item $r=0$ if and only if all the lambda node in $s$ have order $\geq l$.
\end{compactitem}
\smallskip

We call \defname{incremental order-decomposition}  of an order-$1$ stack $s$ \emph{with respect to $l\in \nat$} and we write $\orddec_l(s)$ to denote the
subsequence $\lambda \overline{\eta}_r^{k_r} \ldots \lambda\overline{\eta}_1^{k_1}$ of $s$ consisting of the lambda-nodes $\lambda
\overline{\eta}_l^{k_l}$ defined above. This sequence is uniquely determined for any given $l\in \nat$. It is formally defined as follows:
\begin{align*}
  \orddec_l(\epsilon) &= \epsilon \\
    \orddec_l(s) &=     \orddec_{\ord{\lambda\overline{\eta}}}(s_{<\lambda\overline{\eta}}) \cdot \lambda\overline{\eta}^k
&    \mbox{for } s\neq\epsilon \\
& \mbox{\quad where } \lambda\overline{\eta}^k \mbox{ is the last node in $top_1(s)$ with order $>l$\ .}
\end{align*}

%For $1\leq j \leq r$, $j$ is called the \defname{incremental index} of the lambda-node $\lambda \overline{\eta}_l^{k_l}$ in $\orddec{s}$.

\emph{Convention:} In the original definition, the operation $top_i$ that returns the top $(i-1)$-stack of a higher-order stack, removes any dangling pointer resulting from the operation. Here, we suppose that $top_i$ is defined in such a way that all pointers are preserved. From an implementation viewpoint, since links are encoded as pairs of integers, this means that $top_i$ just returns an unmodified copy of the top $(i-1)$-stack. This convention allows us to
extend the definition of order-decomposition to higher-level stacks as follows:
\begin{definition}
The \defname{incremental order-decomposition of a higher-order stack} $s$ (or 
order-decomposition for short), written
$\orddec(s)$, is defined as:
$$\orddec(s) = \orddec_{\ord(top_1\ s)}(top_2\ s) \ .$
\end{definition}


\begin{lemma}
\label{lem:push1pop1_orderdecompo} Let $s$ be a higher-order stack
with order-decomposition $\orddec(s) = \langle \lambda
\overline{\eta}_r^{k_r}, \ldots, \lambda \overline{\eta}_1^{k_1}
\rangle$. Then
\begin{enumerate}[i.]
\item For all lambda node $a \in \Gamma$ and link $(j,k)$ we have
 $$ \orddec{push_1 a^{(j,k)}~s} = \left\{
                                       \begin{array}{ll}
                                        \langle a^k \rangle, &  \hbox{if $\ord{a} \geq \ord{\lambda \overline{\eta}_r^{k_r}}$}; \\
                                         \langle \lambda \overline{\eta}_r^{k_r}, \ldots, \lambda
\overline{\eta}_i^{k_i}, a^k \rangle, & \hbox{otherwise, where $i = \min \{ i \ | \ \ord{a} <
\ord{\lambda \overline{\eta}_i^{k_i}} \}$.}
                                       \end{array}
                                     \right.$$

\item For all non-lambda node $a \in \Gamma$ and link $(j,k)$ we have
$$ \orddec{push_1 a^{(j,k)}~s} = \orddec{s} \ .$$

\item If $top_1 s$ is not a lambda-node then
$$ \orddec{pop_1~s} = \orddec{s} \ .$$
\end{enumerate}
\end{lemma}
\begin{proof}
  Trivial from the definition of  $\orddec{s}$.
\end{proof}


We use the notion of reachable configuration
with respect to the $\rightarrow$-step relation introduced in the original paper.\footnote{Recall that $c\rightarrow c'$
just if $c' = \delta(top_1 c)(c)$ where $\delta$ is the transition
function defined in \cite[Figure 2]{hague-sto07}.} Thus a configuration of is reachable if it can be attained starting from the initial configuration ($push_1 \theroot \bot_n$) by performing one or more application of the steps (A), (S), (L), $(V_1)$, $(V_0)$ from the algorithm defining $CPDA(G)$.
In particular, the intermediate configurations visited by the internal transitions of a step are not considered reachable.

\begin{lemma}[Incremental binders are in the order-decomposition]
\label{lem:binder_in_ordecompos} Let $c$ be a reachable
configuration of $CPDA(G)$ such that $top_1\
c$ is a variable $x$ of order $l\geq 0$. Then
\begin{enumerate}[i.]
\item $\orddec{c}$ contains at least a node with order strictly
greater than $l$;
\item The last lambda node in $\orddec{c}$ verifying the first condition is precisely $x$'s binder.
\end{enumerate}
\end{lemma}
\begin{proof}
\begin{enumerate}[i.]
\item By Corollary 8 from \cite{hague-sto07}, the top $1$-stack of a configuration contains the P-view of some
    traversal whose last visited node is the $top_1$ symbol, and
    by Proposition 6 from \cite{OngLics2006}, the P-view of a
    traversal is exactly the path (in the unfolding of) the
    computation graph from the last visited node to
    the root. Hence the binder of $x$, whose order
    is strictly greater than $\ord{x}$, occurs in the top $1$-stack.
    Consequently $\orddec{c}$ must contain at least one node of order strictly greater than $l$.

\item Since the recursion scheme is safe, $x$ is
 incrementally bound (see \cite{blumong:safelambdacalculus})
 which means that its binder is precisely the first $\lambda$-node in the
 path to the root in the computation tree with order strictly
 greater than $x$.
\end{enumerate}
\end{proof}

\subsection{Safe higher-order stack}
\subsubsection{Definition}

\begin{definition}[Safe higher-order stack]
\label{dfn:safestack} Let $s$ be an order-$j$ stack for $j\geq1$ and $r
= |\orddec{s}|$.

We say that $s$ is \defname{safe} iff
    \begin{enumerate}[1.]
    \item $\orddec{s} = \langle \lambda \overline\eta_r^1, \ldots ,
    \lambda \overline{\eta}_1^1 \rangle$ \ie the $k$-components in $\orddec{s}$ are all equal to $1$;
    \item for all $1 \leq q \leq r$ such that $n-\ord{\lambda \overline\eta_q}+1 \leq \ord{s}$,
    $collapse~s_{\prefixof \lambda\overline\eta_q}$ is safe.
    \end{enumerate}

We say that $s$ is \defname{$q$-safe} if and only if the lambda node
$\lambda\overline\eta$ of $\orddec{s}$ with incremental index $q$ verifies
    $collapse~s_{\prefixof \lambda\overline\eta}$ is safe.
\end{definition}

Equivalently, this definition can be reformulated inductively as follows:
an order-$1$ stack $s$ is safe if and only if $\orddec{s} = \langle \lambda \overline{\eta}_r^1, \ldots , \lambda \overline{\eta}_1^1 \rangle$;
a higher-order stack $s$ is safe iff $top_2~s$ is safe and
for any lambda-node $\lambda\overline\eta \in \orddec{s}$ such that $n-\ord{\lambda \overline{\eta}}+1 \leq \ord{s}$,
    $collapse~s_{\prefixof\lambda\overline\eta}$ is safe.


\begin{lemma}
\label{lem:safecollapsesimulation}
If $s$ is safe, then for any lambda node $\lambda\overline\eta$ in $\orddec{s}$ we have:
$$collapse~s_{\prefixof\lambda\overline\eta} = pop_{n-\ord{\lambda\overline\eta}+1} ~s_{\prefixof\lambda\overline\eta} \ . $$
\end{lemma}

\begin{proof}
By definition of the $collapse$ operation, for any stack $s$ if the link attached to $top_1~s$ is $(j,k)$ then $collapse~s = pop_j^k~s$. Here since $s$ is safe, the link attached to the node $\lambda\overline\eta$ is of the form
$(n-\ord{\lambda\overline\eta}+1,1)$.
\end{proof}


\subsubsection{Stack operations preserving stack safety}
\begin{lemma}
\label{lem:push1pop1_preserves_safety} Let $s$ be a safe higher-order
stack. Then:
\begin{enumerate}
  \item If $top_1\ s$ is not a lambda-node then $pop\ s$ is safe;
  \item for any non-lambda node symbol $a$, $1 \leq j \leq n$, $k \geq 1$, $push_1\ a^{(j,k)}\ s$ is safe;
  \item for any lambda-node symbol $a$, $push_1\ a^{(n-\ord{a}+1,1)}\ s$ is safe.
\end{enumerate}
\end{lemma}
\begin{proof}
This is a direct consequence of Lemma
\ref{lem:push1pop1_orderdecompo}.
\end{proof}


\begin{lemma}
\label{lem:top_qsafe} If $s$ is safe then $top_{\ord{s}}~s$ is safe.
\end{lemma}


Let $s$ be a higher-order stack. We define $s^{\langle j \rangle}$ as the operation that replaces
every link occurring in $s$ of the form $(j,k)$ by $(j,k+1)$. Formally,
\begin{align*}
{a^{(j,k)}}^{\langle j \rangle} &= a^{(j,k)}   \\
{a^{(j,k)}}^{\langle j' \rangle} &= a^{(j,k+1)} &   \mbox{when $j\neq j'$,}\\
[s_1 \ldots s_p]^{\langle j \rangle} &= [s_1^{\langle j \rangle} \ldots s_p^{\langle j \rangle}] \ .
\end{align*}

\begin{lemma}
\label{lem:incrk_qsafe}
Let $n>l\geq 0$. Let $s$ be a stack of level $1\leq \ord{s} <n$.
 Let $b$ be the incremental index of the last lambda node in $\orddec{s}$
 with order $>l$.
If $s$ is safe then $s^{\langle n-l+1 \rangle}$ is $b$-safe.
\end{lemma}
\proof Take a safe stack of order $<n$.
 We first show the result for the case $\ord{s} = 1$:
Since $s$ is safe we have
$\orddec{s} = \langle \lambda \overline{\eta}_r^1
, \ldots, \lambda \overline{\eta}_1^1   \rangle$.
By definition, the operation $s^{\langle n-l+1 \rangle}$ updates the pointers in $s$  as follows:
the $k$ component of each link is increased by one if the order of
the stack symbol is $l$ and is kept unchanged otherwise.
Hence we have:
\begin{equation*}
\orddec{s^{\langle n-l+1 \rangle}} = \langle
\lambda \overline{\eta}_r^1, \ldots,  \lambda \overline{\eta}_{q}^1, \lambda \overline{\eta}_{q-1}^{k_{q-1}}, \ldots,
 \lambda \overline{\eta}_1^{k_1} \rangle
\end{equation*}
for some $k_{q-1}, \ldots , k_1 \leq 2$, where $q$ is the incremental index of the last lambda-node in $\orddec{s}$ with order $>l$. By assumption $b$ is precisely equal to $b$ therefore $s^{\langle n-l+1 \rangle}$ is $b$-safe.

Suppose that $1<\ord{s}<n$.
By definition of stack-safety (Def.~\ref{dfn:safestack}), to show that $s^{\langle n-l+1 \rangle}$ is $b$-safe it suffices to prove that
\begin{enumerate}
\item $top_2~s^{\langle n-l+1 \rangle}$ is $b$-safe,
\item for all $b\leq q \leq r$ such that
$n-\ord{\lambda \overline{\eta}_{q}} +1 \leq \ord{s^{\langle n-l+1 \rangle}}$, $collapse~s_{\prefixof \lambda \overline{\eta}_{q}}$ is safe.
\end{enumerate}

First condition: Clearly we have $top_2~s^{\langle n-l+1 \rangle} = (top_2~s)^{\langle n-l+1 \rangle}$. Since $s$ is safe, so is $top_2~s$. We have proven the result for $1$-stacks thus we can apply it to $top_2~s$ which shows that $(top_2~s)^{\langle n-l+1 \rangle}$ is $b$-safe.

Second condition: The only symbols that are updated in $s$ by the operation $s \mapsto s^{\langle n-l+1 \rangle}$ are those occurring in the top order-$1$ stack $top_2~s$. Thus since the first $r-b+1$ lambda-nodes in $\orddec{s^{\langle n-l+1 \rangle}}$ and $\orddec{s}$ coincide, this implies that $collapse(s^{\langle n-l+1 \rangle}_{\prefixof\lambda\overline\eta_q}) = collapse(s_{\prefixof\lambda\overline\eta_q})$ for all $b \leq q \leq r$.
Hence the safety of $s$ implies the safety of $s^{\langle n-l+1 \rangle}_{\prefixof\lambda\overline\eta_b}$ \ie the $b$-safety of $s^{\langle n-l+1 \rangle}$.
\qed
\smallskip

\begin{lemma}
\label{lem:cons_qsafety} Let $s$ be a higher-order stack of level $\geq 2$. If
\begin{enumerate}[1.]
\item $pop_{\ord{s}}~s$ is safe;
\item and $top_{\ord{s}}~s$ (of order $(\ord{s}-1)$) is safe,
\end{enumerate}
then $s$ is safe.
\end{lemma}
\proof
Let $s = [s_1 \ldots s_l~s_{l+1}]$ for some $l\geq0$.
We proceed by induction on  $top_{\ord{s}}~s=s_{l+1}$.
The base case $s_{l+1} = \bot_{\ord{s}-1}$ is trivial.
Suppose that $s_{l+1}$ is not the empty stack. We show that $s$ is safe using the inductive definition of stack-safety \ie we show that $top_2~s$ is safe and that for all lambda-node $\lambda \overline{\eta} \in \orddec{s}$ such that $n-\ord{\lambda \overline{\eta}}+1 \leq \ord{s}$, $collapse~s_{\prefixof \lambda \overline{\eta}}$ is safe:
\begin{itemize}
\item First condition: If $\ord{s}=2$ then $top_2~s = top_{\ord s}~s$ which is safe by the second assumption. Otherwise $\ord{s}>2$ and since $top_{\ord{s}} s$ is safe, so is $top_2(top_{\ord{s}} s)$. By definition of $top_2$ we have $top_2~s = top_2(top_{\ord{s}} s)$ hence $top_2~s$ is safe.

\item Second condition: Let $\lambda \overline{\eta}$ be a lambda-node in $\orddec{s} = \orddec{s_{l+1}}$ such that $n-\ord{\lambda \overline{\eta}}+1 \leq \ord{s}$.
Since $top_2(s)$ is safe, by Lemma \ref{lem:safecollapsesimulation} we have
$collapse~s_{\prefixof \lambda \overline{\eta}} = pop_{n-\ord{\lambda \overline{\eta}}+1}~s_{\prefixof \lambda \overline{\eta}}$.

If $n-\ord{\lambda \overline{\eta}}+1 = \ord{s}$ then
$pop_{n-\ord{\lambda \overline{\eta}}+1}~s_{\prefixof \lambda \overline{\eta}} = pop_{\ord{s}}~s_{\prefixof\lambda\overline\eta} = pop_{\ord{s}}~s$ which is safe by the first assumption.

Otherwise $n-\ord{\lambda \overline{\eta}}+1 < \ord{s}$ and we have:
\begin{align*}
  collapse~s_{\prefixof\lambda\overline\eta}
      &= pop_{n-\ord{\lambda \overline{\eta}}+1}~s_{\prefixof\lambda\overline\eta}
      & \mbox{by Lemma \ref{lem:safecollapsesimulation}, since $s_{\prefixof\lambda\overline\eta}$ is safe} \\
  &= [ s_1 \ldots s_l\ (pop_{n-\ord{\lambda\overline\eta}+1}~ {s_{l+1}}_{\prefixof\lambda\overline\eta}) ]
   & \mbox{$n-\ord{\lambda \overline{\eta}}+1<\ord{s}$}  \\
  &= [s_1 \ldots s_p~collapse({s_{l+1}}_{\prefixof\lambda\overline\eta}) ]
  & \mbox{by Lemma \ref{lem:safecollapsesimulation}, since $s_{l+1}$ is safe}
\end{align*}

By the second assumption, $s_{l+1} = top_{\ord{s}}~s$ is safe therefore by definition
of safety so is $collapse~{s_{l+1}}_{\prefixof\lambda\overline\eta}$, thus
since $|collapse~{s_{l+1}}_{\prefixof\lambda\overline\eta}| < |s_{l+1}|$ we can use the induction hypothesis which shows that $[s_1 \ldots s_p~collapse({s_{l+1}}_{\prefixof\lambda\overline\eta}) ]$ is safe.
\qed
\end{itemize}



\begin{lemma}
\label{lem:pushj_safe_implies_b-safe} Let $n>l\geq 0$ and $s$ be a higher-order stack
of level $2 \leq n-l+1 \leq \ord{s} \leq n$. Let $b$ be the incremental index of the last lambda node in $\orddec{s}$ with order $>l$. Then $push_{n-l+1}\ s$ is $b$-safe.\footnote{We reproduce here the definition of the
CPDA operation $push_j$ from \cite{hague-sto07}:
$$ push_j \underbrace{[ s_1 \ldots s_{l+1} ]}_{s} =
\left\{
  \begin{array}{ll}
\    [s_1\ \ldots\ s_{l+1}\ s_{l+1}^{\langle j \rangle}]  &\hbox{if $j = \ord{s}$;}\\
\    [s_1\ \ldots\ s_{l+1}\ push_j\ s_{l+1}]  &\hbox{if $j<\ord{s}$.}
 \end{array}
\right.
$$}
\end{lemma}
\begin{proof}
Let $s=[s_1 \ldots s_{c+1}]$ be any safe higher-order stack of order. We show that $push_{n-l+1}~s$ is $b$-safe by finite induction on the order of $s$.
    \begin{compactitem}
      \item Base case: $\ord{s} = n-l+1 $. We have
    $push_{n-l+1}~s = [ s_1 \ldots s_{c+1} s_{c+1}^{\langle n-l+1
    \rangle}]$.

    Since $s$ is safe, by Lemma \ref{lem:top_qsafe} so is $s_{c+1}$
    which in turn implies that $s_{c+1}^{\langle n-l+1\rangle}$ is
    $b$-safe by Lemma \ref{lem:incrk_qsafe}. Therefore by Lemma
    \ref{lem:cons_qsafety},  $[ s_1 \ldots s_{c+1} s_{c+1}^{\langle n-l+1
    \rangle}]$ is $b$-safe.

      \item Step case: $\ord{s} > n-l+1$. We have
    $push_{n-l+1}~s = [ s_1 \ldots s_{c+1} push_{n-l+1}~s_{c+1}]$.

    $s_{c+1}$ is safe by Lemma \ref{lem:top_qsafe} thus by the
    induction hypothesis $push_{n-l+1}~s_{c+1}$ is $b$-safe and therefore by Lemma \ref{lem:cons_qsafety} so
    is $[ s_1 \ldots s_{c+1} push_{n-l+1}~s_{c+1}]$.
    \end{compactitem}
\end{proof}





\section{Simulation}

We claim that provided that the input recursion scheme is safe, the
uses of the $collapse$ operation in $CPDA(G)$ can all be replaced by the operation $pop_{n-\ord{top_1(s)}+1}$.

\section{Proof of correctness}

\begin{proposition}
All the reachable configurations of CPDA(G) are safe.
\end{proposition}
\begin{proof}
If $n =\ord{c} =1$ then the result holds trivially since CPDA(G) does not contain
any transition of the form $push_j$ for $j>1$ and therefore all the links associated to symbols in a configuration have a $k$-component equals to $1$.

Suppose that $n\geq 2$. We prove the result by induction on the number of
$\rightarrow$-steps. The initial configuration is trivially safe.
Suppse that $c$ is a safe reachable configuration and that
$c'=\delta(u)(c)$ where $u = top_1\ c$.
We show that $c'$ is safe by case analysis on $u$:
\begin{itemize}
\item[Case (A)] Pushing @ on the top $1$-stack preserves safety (Lemma
\ref{lem:push1pop1_preserves_safety}), therefore $c'$ is safe.

\item Case (S)
We have $c' = push_1 a^{(n-\ord{a}+1,1)} c$ where $a$ is a
lambda-node, therefore by Lemma \ref{lem:push1pop1_preserves_safety},
$c'$ is safe.

\item Case (L) Again, Lemma \ref{lem:push1pop1_preserves_safety}
shows that (i) holds and (ii) trivially holds.

\item Case ($V_1$) Suppose $u$ is labelled by a variable $x$ of order $l\geq 1$.

Since $c$ is safe we have $\orddec{c} = \langle \lambda
\overline{\eta}_r^1 , \ldots, \lambda \overline{\eta}_1^1
\rangle$.
Since the recursion-scheme $G$ is safe, by Lemma \ref{lem:binder_in_ordecompos}, $x$'s binder is precisely the
last node in $\orddec{c}$ with order strictly greater than $l$.
Hence by Lemma \ref{lem:pushj_safe_implies_b-safe}
we have that $push_{n-l+1}\ c$ is $b$-safe where $b$ is the
incremental index of $x$'s binder in $\orddec{c}$.

Consequently by definition of $b$-safety we have that
$t = collapse \left( (push_{n-l+1}~ c)_{\prefixof \lambda
\overline{\eta}_{b}} \right) = (push_{n-l+1};pop_1^p;collapse)
~c$ is safe. Hence by
Lemma \ref{lem:push1pop1_preserves_safety}, $c' = push_1 E_i(top_1)^{(n-l+1,1)}\ t$ is also safe.

\item ($V_0$) This case is treated identically to ($V_1$).
\end{itemize}
\end{proof}

\begin{corollary}
In CPDA(G), the $collapse$ operation is executed only on safe configurations.
\end{corollary}
\begin{proof}
In CPDA(G), the only occurrences of the collapse operation are in the steps ($V_1$) and ($V_0$) and are of the form:
$$ collapse(pop_1^p(push_{n-l+1}~c))$$
for some reachable configuration $c$, where $top_1(u)$ is a variable of order $l$ and of span $p$. In the proof of the previous proposition, we have shown that
$push_{n-l+1}~c$ is $b$-safe where $b$ is the incremental index of $x$'s binder in $\orddec{c}$. After performing $pop_1^p$, the top stack symbol contains $x$'s
binder which is precisely the lambda-node with incremental index $b$. Therefore $pop^p_1(push_{n-l+1}~c)$ is safe.
\end{proof}

Together with Lemma \ref{lem:safecollapsesimulation}, this shows that the simulation is sound.



\section{Remark}
It is possible to redefine $CPDA(G)$ from
\cite{hague-sto07} in a more compact way by merging the two subcases in
$(V_1)$ and $(V_0)$. This is done as follows: In case (A), when
pushing the prime child of an application node $@$ on the stack, we
associate a link to it that points to the preceding stack symbol in
the top $1$-stack \ie the $@$-node itself.
 This modification permits us to avoid the consideration on $j$ (the child-index of $u$'s binder) in the
 cases ($V_0$) and ($V_1$). The sequences of instruction $pop_1^{p+1}$ can now be replaced by
 $pop_1^p ; collapse$. The resulting CPDA is denoted by $CPDA'(G)$ and is described in
Figure \ref{fig:cpdaprime}.
\begin{figure}[htbp]
\begin{center}
\makebox{
\begin{shadowbox}[10cm]
If $u$'s label is not a variables, the action is just a $push_1^v$, where $v$ is an appropriate child of the node $u$. Precisely:
\begin{itemize}
\item $(A)$ If the label is an @ then $\delta(u) = push_1^{E_0(u),1}$.
\item $(S)$ \ldots
\item $(L)$ \ldots
\end{itemize}
Suppose $u$ is a variable which is the $i$-parameter of
its binder and let $p$ be the span of $u$.
\begin{itemize}
\item $(V_1)$ If the variable has order $l\geq 1$, then
$$\delta(u) = push_{n-l+1} ; pop_1^p ; collapse;push_1 E_i(top_1)^{(n-l+1,1)}$$
\item $(V_0)$ If the variable is of ground type then
$$\delta(u) = pop_1^p ; collapse;push_1 E_i(top_1)$$
\end{itemize}
\caption{CPDA'(G): a compact version of CPDA(G).}
\label{fig:cpdaprime}
\end{shadowbox}
}
\end{center}
\end{figure}
Contrary to $CPDA(G)$, in a configuration of this CPDA the link associated to a lambda node $\lambda\overline\eta$ is not necessarily of the form
$(n-\ord{\lambda\overline\eta}+1,k)$. Instead the first component of the link pair $(j,k)$ associated to a lambda node is defined as follows:
$j = 1$ if $\lambda\overline\eta$ is a prime node and $j=n-\ord{\lambda\overline\eta}+1$ otherwise. Therefore we can simulate the collapse operation
using transitions performing $pop_1$ if $top_1~s$ is prime and $pop_{n-\ord{top_1(s)}+1}$ otherwise.
\clearpage

\bibliographystyle{abbrv}
\bibliography{../bib/dphil-all}

\end{document}
