\documentclass{article}
\usepackage{pstricks}
\usepackage{pst-node}
\usepackage{pstring}
\usepackage[all]{xy}
\usepackage{amssymb}
\usepackage{amsmath}
\usepackage{tikz}

\newcommand{\oview}[1]{\llcorner #1 \lrcorner}
\newcommand{\pview}[1]{\ulcorner #1 \urcorner}


%\psset{arrowlength=1,arrowinset=.4}

\begin{document}


%%%%%%%%%% POINTERS USING PSTRICKS
\section{POINTERS USING PSTRICKS}


\subsection{using the commands defined in pstring.sty}
In the following sequences the links are created with the $\backslash$lnk command.
$$\begin{array}{rclrcl}
\mbox{First line} &=& \parbox[t]{8cm}{there should be enough space between the bottom of this line and the arc of the next line.} \\
s &=& \pview{\pstr[28pt][10pt]{
 s \cdot \nd(m) m \cdot \ldots \cdot
\nd(lmd-m,40) {\lambda\overline{\xi}}
 }} \quad \mbox{ the arc is elevated using the raise option of $\backslash$pstr.}
\end{array}
$$

The next example can only be done using the pstricks engine. It can't be done in PGF due to the impossibility to create nodes outside PGF graphics environment
It is possible to achieved this using overlays but it seems to work with the Tikz interface language to PGF
and I do not know how to implement it at the base PGF level. Also the use of overlays requires two compilations in order to resolve the node name references
and it will only work with pdflatex (not with latex).
$$\pstr[0.5cm][5pt]{ s = \underbrace{\cdots\ \nd(r){r}^O \cdots }_{s'} \
\nd(n){n}^P \ \underbrace{\cdots\ \nd(a-r,35){a}^P \cdots }_{u} \
\nd(m-n,30){m}^O } \ .$$



\subsection{other examples using the succinct syntax (defined in pstring.sty)}

\Pstr[0.1cm]{
s \cdot (m) m \cdot \ldots \cdot
(lmd-m,40) {\lambda\overline{\xi}}
}

\Pstr[0.8cm]{ \psi =
(n){q^0}  \ (n4-n,60){q^1}  \ (n5-n4,60){q^2} \  (n8-n,50){q^1}  \ (n9-n8,60){q^2}   \ (n10-n,40){q^3}  \ .}




%%%%%desired syntax:
%\pstr{
%\psi_M(t\upharpoonright r) =
%& [n] q^0 [n4->n]{30} q^1 [n5->n4]{30} q^2 [n8->n]{30} q^1  [n9->n8]{30} q^2 [n10->n]{30} q^3.}

%\hbox{\pstr[0.4cm]{ \psi =
%[n] q^0 | \ [n4-n,60] q^1 | \ [n5-n4,60] q^2 | \  [n8-n,50] q^1 | \ [n9-n8,60] q^2  | \ [n10-n,40] q^3  | \ .
%}


\subsection{examples using directly the command $\backslash$rnode from pstricks and $\backslash$link (from pstring.sty)}
$\backslash$link (from pstring.sty) uses a modification of the $\backslash$psarc command from pstricks and is implemented in pstring.sty.
$$
 \pview{ s \rnode{m}{m} \cdot \ldots \cdot \rnode{lmd}{\lambda \overline{\xi}}} = \pview{s} \cdot \rnode{m2}{m} \cdot \rnode{lmd2}{\lambda \overline{\xi}}   \link*[nodesep=1pt]{30}{lmd}{m}    \link[nodesep=0pt]{35}{lmd2}{m2}
%  \psbezier[linewidth=0.8mm,linecolor=red,showpoints=true]{|->}%
%           {m}{m}(2,2)(0,0)
$$

\begin{itemize}
  \item  \raisebox{0cm}[0.7cm]{$
t = \rnode{n}{\lambda f z} \
\rnode{n2}{@} \
\rnode{n3}{\lambda g x} \
\rnode{n4}{f^{[1]}} \
\rnode{n5}{\lambda^{[2]}} \
\rnode{n6}{x} \
\rnode{n7}{\lambda^{[3]}} \
\rnode{n8}{f^{[4]}} \
\rnode{n9}{\lambda^{[5]}} \
\rnode{n10}{z}
%\psset{ncurv=0.4}
\link*{60}{n3}{n2}
\link*{50}{n4}{n}
\link*{45}{n5}{n4}
\link*{55}{n6}{n3}
\link*{35}{n7}{n2}
\link*{35}{n8}{n}
\link*{45}{n9}{n8}
\link*{35}{n10}{n}$}

\item \raisebox{0cm}[0.8cm]{$
t\upharpoonright r = \rnode{n}{\lambda f z} \
\rnode{n4}{f}^{[1]} \
\rnode{n5}{\lambda}^{[2]} \
\rnode{n8}{f}^{[4]} \
\rnode{n9}{\lambda}^{[5]} \
\rnode{n10}{z}
\link*{50}{n4}{n}
\link*{60}{n5}{n4}
\link*{35}{n8}{n}
\link*{60}{n9}{n8}
\link*{33}{n10}{n}$}
\item \raisebox{0cm}[0.8cm]{$
\psi_M(t\upharpoonright r) = \rnode{n}{q^0}\ \rnode{n4}{q^1}\ \rnode{n5}{q^2}\ \rnode{n8}{q^1}\ \rnode{n9}{q^2}\ \rnode{n10}{q^3}
\link*{60}{n4}{n}
\link*{60}{n5}{n4}
\link*{50}{n8}{n}
\link*{60}{n9}{n8}
\link*{40}{n10}{n}\ .$}
\end{itemize}


%%%%%%%%%% POINTERS USING XYPIC
\section{POINTERS USING XYPIC}

%\SelectTips{eu}{10}
%\UseTips
$\psi_M(t\upharpoonright r) =
\xymatrix@1@=0pt @C=1pt @R=0pt {
q^0 & q^1 \ar@/_2pt/+U-(1,0);[l]+U+(1,0) &
q^2 \ar@/_2pt/+U-(1,0);[l]+U+(1,0) &
 q^1 \ar@/_7pt/+U-(1,0);[lll]+U+(1,0) &
 q^2 \ar@/_2pt/+U-(1,0);[l]+U+(1,0)&
q^3 \ar@/_7pt/+U-(1,0);[lllll]+U+(1,0) \\
}$

\xymatrix{ \bullet
\ar[r]^x
\ar@/^3pc/[rr]^{\quad}_{}="0"
& \bullet
\ar[r]^{\bar{x}}
\ar@/_3pc/[rr]^{\quad}_{}="2"
\ar@{=>}"0"; {} ^{i_x}
& \bullet
\ar[r]^x
\ar@{=>}{}; "2" ^{e_x}
& \bullet }

\xy
(0,0)*+{A d {\frac{\frac{\frac{\frac{A}{V}}{V}}{V}}{V}} }="a";
 (10,0)*+{B}="b";
%(-16,-5)*+{\bullet}="c";
{\ar@/^.25pc/ "a"+U;"b"+U};
%{\ar@/_.25pc/"a";"c"};
%{\ar@/_.15pc/ "b";"c"};
\endxy






%%%%%%%%%% POINTERS USING PGF AND TIKZ
\section{POINTERS USING TIKZ (BASED ON PGF)}

The following sequences are generated using the Tikz interface for PGF.

\tikzstyle{nn}=[%rectangle,draw=blue!50,fill=blue!20,thick,
inner sep=1pt, minimum height = 0cm, minimum width = 0cm,anchor = base west]

\tikzstyle{linklabel}=[rectangle,fill=white,
inner sep=1pt, minimum height = 0cm, minimum width = 0cm]



\def\pstr#1{\tikz[baseline=(ref.base)]{\path node[nn] (ref) {}#1 }}

\def\nd(#1)#2#3{node[nn] (#1) {$#2$} (#1.base east) node[nn] (#1t) {$#3 $} (#1t.base east) }

\def\lnk#1#2#3#4{
\draw[line width=0.3pt,arrows=-stealth](#1.120 )  .. controls +(120:#3) and +(60:#3) .. node[linklabel] {#4} (#2.north) }

\def\lnkb#1#2#3#4{
\count255 = 180 \advance \count255 by -#3
\path[line width=0.3pt,arrows=-stealth] ([xshift=-0.5pt] #1.north)  edge[out=\number\count255,in=#3] node[linklabel] {#4} (#2.north) }

A sequence with links drawn with Bezier curves
t= \pstr{
\nd(n){\lambda f z}{}
\nd(n2){@}{}
\nd(n3){\lambda g x}{}
\nd(n4){f^{[1]}}{}
\nd(n5){\lambda^{[2]}}{}
\nd(n6){x}{}
\nd(n7){\lambda^{[3]}}{}
\nd(n8){f^{[4]}}{}
\nd(n9){\lambda^{[5]}}{}
\nd(n10){z}{};
\lnk{n3}{n2}{5pt} {} ;
\lnk{n4}{n} {10pt}{};
\lnk{n5}{n4}{5pt} {};
\lnk{n6}{n3}{5mm} {};
\lnk{n7}{n2}{5mm} {};
\lnk{n8}{n} {18pt}{};
\lnk{n9}{n8}{5pt} {};
\lnk{n10}{n}{25pt}{};
}.

Same sequence drawn with arcs t= \pstr{
\nd(n){\lambda f z}{}
\nd(n2){@}{}
\nd(n3){\lambda g x}{}
\nd(n4){f^{[1]}}{}
\nd(n5){\lambda^{[2]}}{}
\nd(n6){x}{}
\nd(n7){\lambda^{[3]}}{}
\nd(n8){f^{[4]}}{}
\nd(n9){\lambda^{[5]}}{}
\nd(n10){z}{};
\lnkb{n3}{n2}{60}{};
\lnkb{n4}{n} {45}{};
\lnkb{n5}{n4}{45}{};
\lnkb{n6}{n3}{45}{};
\lnkb{n7}{n2}{35}{};
\lnkb{n8}{n} {35}{};
\lnkb{n9}{n8}{45}{};
\lnkb{n10}{n}{35}{};
}

Another example using Bezier curves
$t = \lambda  \cdot {}$ \pstr{
\nd(app){@}{\cdot}
\nd(ly){\lambda y}{\ldots}
\nd(y){y}{\cdot}
\nd(lx){\lambda \overline{x}}{\ldots}
\nd(x){x_i}{\cdot}
\nd(leta){\lambda \overline{\eta_i} }{\ldots};
\lnk{y}{ly}  {7pt} {\tiny 1};
\lnk{lx}{app}{15pt}{\tiny 1};
\lnk{x}{lx}  {7pt} {\tiny i};
\lnk{leta}{y}{15pt}{\tiny i};
}
.


And the corresponding sequence using arcs
$t = \lambda  \cdot {}$ \pstr{
\nd(app){@}{\cdot}
\nd(ly){\lambda y}{\ldots}
\nd(y){y}{\cdot}
\nd(lx){\lambda \overline{x}}{\ldots}
\nd(x){x_i}{\cdot}
\nd(leta){\lambda \overline{\eta_i} }{\ldots};
\lnkb{y}{ly}  {35}{\tiny 1};
\lnkb{lx}{app}{38}{\tiny 1};
\lnkb{x}{lx}  {30}{\tiny i};
\lnkb{leta}{y}{40}{\tiny i};
}
voila.



\end{document}
