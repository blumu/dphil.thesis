\documentclass[11pt]{article}

\begin{document}
\title{DPhil. Erratum}
\author{William Blum}

\maketitle

\section {Erratum - August 2021}

\begin{itemize}
\item Bug in homogeneity remark: TODO
\item Bug in homogeneity example: TODO

\item Definition of Safety in Section 3.1.2

The notions of `Safe redex' and `Safe beta reduction' defined in the thesis are incorrectly defined on
the more relaxed version of the safe lambda calculus. They should be replaced by the definition found in the TLCA paper 
that apply instead to the `long safe' sub-system from Definition 3.31.
(The system of rules from the TLCA paper are precisely the `long safe' rules from Def3.31 in the thesis.)

Recall that in the long safe system, the abstraction rule requires the body of the lambda to be a safe term, 
unlike the more relaxed version of Table 3.1 where the body can just be an `almost safe application'.

The rules from Table 3.1 capture the `incremental binding of variables' aspect of the original safety restriction (i.e., if you consider AST tree
where consecutive lambdas are merged into a single bulk AST node, then for safe terms you can retrieve the binder node of any variable by traversing the path from 
the variable to the root of the AST and stopping at the first lambda node with order greater than the order of the variable.)
This is the version that gets studied in the last chapter of the thesis, and 
the one that gets characterized semantically by `P-incremental' strategies.

Observe that a long safe term that is not safe can always be turned into an eta-equivalent safe term: by just eta-expanding 
the body of abstractions that are \emph{almost} safe applications. Such eta-expansion has the side-effect
 of instantiating fresh variable names, which partially defeats the benefit of the safety restriction. 
 So in a way, the safety terms from Table 3.1 are not `as safe' as the one from Def 3.31.
 
Because the game semantic model studied in the Chapter 6 is extensional (in the sense that eta-expanding a term does not alter its semantics), 
it cannot possibly capture the syntactic difference between long-safe term and safe terms. 
For this reason, the game semantic characterization result from Chapter 5 (Theorem 5.19) applies to the more relaxed version of 
the calculus from Table 3.1 rather than just long-safety.
(Though the characterization result still implies that long safe terms are denoted by P-incremental strategies.) 

So if one looks for a calculus with a sound notion of beta-reduction (as in "does not get stuck", strongly normalizing) with variable-capture avoidance 
(not only when performing substitution but also when reducing terms) then long-safety is the one to pick.
The `No-variable capture lemma 3.15', defined in the context of a substitution, still applies to both versions of the calculi.
However the definition of `safe redex contraction' should only exist for long-safe term (as in the TLCA paper).

On the other hand, to study the denotational game semantics of the language, the definition from Table 3.1 is more appropriate.

\end{itemize}

\section{Acknowledgment}
Credits to Samuel Frontull for spotting and reporting a few errors in my thesis, including conter-examples, that led to this erratum.

\end{document}
