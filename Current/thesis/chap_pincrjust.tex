\label{chap:pincrjust}

\section{Preliminaries}

We consider an arena $A$ and make the following two assumptions on it:
\begin{itemize}
\item (A1) For $A \neq \bot$ (the arena with a single initial question), each question move in the arena enables at least one answer move.
\item (A2) Answer moves do not enable any other move.
\end{itemize}

An arena is said to be \defname{prime} if it has a single initial move; a type is prime if its arena denotation is prime.

\subsection{Node-order}

\subsubsection{Definition}

We define the \defname{order of a move} $m$ in the arena $A$, written $\ord_A{m}$ (or just $\ord{m}$ where there is no ambiguity), as the length of the path from $m$ to its furthest leaf in $A$ minus 1
({\it i.e.}~the height of the subarena rooted at $m$ minus 2.). Because of assumptions (A1) and (A2),
for any move $m$ of $A \neq \bot$, $m$ is a question move if and only if $\ord{m} \geq 0$, and $m$ is an answer move if and only if $\ord{m} = -1$.

The \defname{order of an arena} $A$ is defined to be the maximal order of its initial moves. The order of a (simple, PCF or IA) type is defined as the order of the arena denoting it - or equivalently as 0 for ground type, $\ord{A\rightarrow B} = \max(1+\ord{A},\ord{B})$ and $\ord(A\times B) = \max(\ord A, \ord B)$. The order of a term is the order of its type.


\subsubsection{Node-order after composition}

Consider the arena $X\lingamear Y$ and let $m$ be a move of
$X\lingamear Y$. We write $\ord_{X\lingamear Y}{m}$ to denote the
order of $m$ in the arena ${X\lingamear Y}$. If $m$ belongs to $X$
(resp.~$Y$) then we write $\ord_X{m}$ (resp.~$\ord_Y{m}$) to denote
the order of the move $m$ in the arena $X$ (resp.~$Y$).

\begin{lemma}
\label{lem:compositionorder} Let $A$, $B$ and $C$ be three arenas.
We have:
$$\begin{array}{lll}
\forall m \in A:
    &  \ord_{A\lingamear B}{m} = \ord_{A\lingamear C}{m} \ ,\\
\forall m \in B:
    & \ord_{A\lingamear B}{m} \geq \ord_{B\lingamear C}{m}  & \mbox{for $m$ initial,}\\
    & \ord_{A\lingamear B}{m} = \ord_{B\lingamear C}{m} & \mbox{for $m$ non initial,} \\
\forall m \in C:
    & \ord_{A\lingamear C}{m} \geq \ord_{B\lingamear C}{m} \iff
\ord{A} \geq \ord{B}\ & \mbox{for $m$ initial,}\\
    & \ord_{A\lingamear C}{m} = \ord_{B\lingamear C}{m}   & \mbox{for $m$ non initial.}
\end{array}
$$
\end{lemma}

\subsection{Well-bracketing}

We call \defname{pending question} of a sequence of moves $s \in L_A$ the last unanswered question in $s$.

\begin{definition}\rm
A strategy $\sigma$ is said to be \defname{P-well-bracketed} if for any play $s \, a \in \sigma$ where $a$ is a  P-answer, $a$ points to the pending question in $s$.
\end{definition}



P-well-bracketing can be restated differently as the following proposition shows:
\begin{proposition}
\label{prop:char_wellbrack}
\rm We make assumption (A1) and (A2).
Let $\sigma$ be a strategy on an arena $A\neq \bot$.
The following statements are equivalent:
\begin{enumerate}
\item[(i)] $\sigma$ is P-well-bracketed,
\item[(ii)] for $s \, a \in \sigma$ with $a$ a P-answer, $a$ points to the pending question in $\pview{s}$,
\item[(iii)] for $s \, a \in \sigma$ with $a$ a P-answer, $a$ points to the last O-question in $\pview{s}$,
\item[(iv)] for $s \, a \in \sigma$ with $a$ a P-answer, $a$ points to the last O-move in $\pview{s}$ with order $>\ord{a}$.
\end{enumerate}
\end{proposition}
\begin{proof}
$(i)\iff(ii)$: \cite[Lemma 2.1]{McC96b} states that if P is to move then the pending question in $s$ is the same as that of $\pview{s}$.

$(ii)\iff(iii)$: Assumption (A2) implies that the pending question in $\pview{s}$ is also the last O-question occurring in $\pview{s}$.

$(iii)\iff(iv)$: Because of assumption (A1) and (A2),
for any move $m$, we have $m$ is a question move
if and only if $\ord{m} \geq 0$ if and only if $\ord{m} > \ord{a} = -1$.
\end{proof}




\begin{lemma}
\label{lem:justfied_by_unanswered}
Under assumption (A2), if $s$ be a justified sequence of moves satisfying alternation and visibility then any O-move (resp. P-move) in $s$ points to an \emph{unanswered} P question (resp. O-question).
\end{lemma}
\begin{proof}
Suppose that an O-move $c$ points to a P-move $d$ that has already been answered by the O-move $a$. The sequence $s$ as the following form:
$$ s= \ldots \Pstr{(d){d}  \ldots  (a-d,20){a}  \ldots  (c-d,20){c}}$$

By O-visibility, $d$ must belong to $\oview{s_{<c}}$. But since $a$ is an answer, by assumption (A2), it cannot justify any P-move, therefore
$\oview{s_{<q}}$ must contain an OP-arc ``hoping'' over $a$. We name the nodes of this arc $d^1$ and $c^1$:
$$ s = \ldots \Pstr[0.7cm]{(d){d}  \ldots  (d1){d^1} \ldots (a-d,20){a} \ldots
 (c1-d1,20){c^1} \ldots (c-d,25){c}}$$

By P-visibility, $d^1$ must belong to $\pview{s_{<c^1}}$. Consequently, $a$ does not belong to $\pview{s_{<c^1}}$ (otherwise the PO-arc $\Pstr[0.5cm]{(d){d} \quad (a-d,45){a}}$ would cause the P-view to jump over $d^1$).
Therefore there must be a PO-arc $\Pstr[0.5cm]{(d2){d^2} \quad (c2-d2,45){c^2}}$ in $\pview{s_{<c^1}}$ hoping over $a$:
$$s = \ldots \Pstr[0.7cm]{(d){d}  \ldots
(d1){d^1} \ldots (d2){c^2} \ldots
(a-d,20){a} \ldots
 (c2-d2,20){d^2} \ldots (c1-d1,20){c^1} \ldots (c-d,25){c}}$$

This process can be repeated infinitely often by using alternatively O-visibility and P-visibility. This gives a contradiction since the sequence of moves $s_{<c}$ has finite length.
Hence $d$ cannot point to a question that has already been answered. Since, by assumption (A2), a question is enabled by another question, $d$ is necessarily justified by an unanswered question.
\end{proof}


\begin{lemma}
\label{lem:oq_in_pview_unanswered}
Under assumption (A2), if $s$ is a P-well-bracketed justified sequence of moves of odd length satisfying alternation and visibility then  all O-questions occurring in $\pview{s}$ are unanswered in $s$.
\end{lemma}
\begin{proof}
We proof the first part by induction on $s$. The base case ($s = q$
with $q$ initial O-move) is trivial. Suppose $\Pstr[0.4cm]{ s = s'
\cdot (q)q \cdot u \cdot (m-q,45){m} }$. We have $\pview{s} =
\pview{s'} \cdot q \cdot m$. Clearly $m$ is unanswered in $s$. Let
$r$ be an O-question in $\pview{s'}$ and suppose that $r$ is
answered in $s$ by some move $a$. By the induction hypothesis, $r$
is unanswered in $s'$ therefore $a$ necessarily appears in the
segment $u$:
\begin{center}
\ifLoadPGFengine
\pstr[0.7cm]{
\nd{s = }(l1){l1}
\nd{\cdots \ }(r){r^O}
\nd(r1){r1}
\nd{\cdots  \ }(q){q^P}
\nd{\ \cdots\ }(l2){}
\nd(a-r,35){a^P}
\nd(r2) {\cdots \ }
\nd(m-q,30){m^O}
\begin{pgfdecoration}{{brace}{\pgfdecoratedpathlength}}
\pgfsetdash{}{0pt}
\pgfpathmoveto{\pgfpointanchor{l1}{north}}
\pgfpathlineto{\pgfpointanchor{r1}{north}}
\end{pgfdecoration}
\pgfusepath{stroke}
} \ .
\begin{tikzpicture}[inner xsep=0pt, inner ysep=0pt]
\node(a0){$t = \ldots \cdot$}; \node[right=0pt of a0](a1){$\ \alpha\
\cdot$}; \node[right=0pt of a1](a2){$\ \lambda'$}; \node[right=0pt
of a2](a3){$\cdot \ldots $}; \node[right=0pt of a3](a4){$\
v_{\lambda'} \ldots$}; \node[right=0pt of a4](a5){$\ \cdot\
\lambda$};
\draw[decorate,decoration={brace,raise=6pt,mirror},thick](a1) --
(a5) ; \path(a1) --node[yshift=-12pt, below]{$u$} (a5) ;
\draw[->,\arrowcolor] (a5.north) .. controls +(150:1cm) and
+(60:20pt) .. (a1.north); \draw[->,\arrowcolor] (a2.north) ..
controls +(160:8pt) .. (a1); \draw[->,\arrowcolor] (a4) .. controls
+(150:20pt) and +(50:20pt) .. (a2);
\end{tikzpicture}
\else
\pstr[0.7cm]{ s = \underbrace{\cdots\ \nd(r){r^O}
\cdots }_{s'} \ \nd(q){q^P} \ \underbrace{\cdots\ \nd(a-r,35){a^P}
\cdots }_{u} \ \nd(m-q,30){m^O} } \ .
\fi
\end{center}


But since $m$ is justified by $q$, by lemma \ref{lem:justfied_by_unanswered} $q$ must be unanswered in $s_{< m}$. In particular, the pending
question at $s_{\prefixof a}$ cannot be $r$ since the unanswered question $q$  is played after $r$. This gives a contradiction since by well-bracketing $a$ should answer the pending question. Hence $r$ is unanswered in $s$.
\end{proof}








\subsection{Interaction sequences} Let us first recall the
definition of an interaction sequence. Let $A$,$B$ and $C$ be three
games. We say that $u$  is an
\defname{interaction sequence} of $A$,$B$ and $C$ whenever $u\filter
A,B$ is a valid position of the game $A\lingamear B$ (i.e.~$u\filter
A,B \in P_{A\lingamear B}$) and  $u\filter B,C$ is a valid position
of the game $B\lingamear C$. We write $Int(A,B,C)$ to denote the set
of all such interaction sequences.

Let $\sigma:A\lingamear B$ and $\mu:B\lingamear C$ be two
strategies. We write $\sigma \parallel \mu$ to denote the set of
interaction sequences that unfold according to the strategy $\sigma$
in the $A,B$-projection of the game and to $\mu$ in the
$B,C$-projection:
$$ \sigma \parallel \mu = \{ u \in Int(A,B,C) \ | \ u\filter A,B \in \sigma \wedge u \filter B,C \in \mu \} \ .$$
The composite of $\sigma$ and $\mu$ is then defined as $\sigma ; \mu
= \{ u \filter A,C \ | \ u \in \sigma \parallel \tau \}$.

The diagram below shows the structure of an interaction sequence
from $\sigma \parallel \mu$. There are four states represented by
the rectangular boxes. The content of the state shows who is to play
in each of the game $A\lingamear B$, $B\lingamear C$ and
$A\lingamear C$. For instance in state $OPP$, it is O's turn to play
in $A\lingamear B$ and P's turn to play in $B\lingamear C$ and
$A\lingamear C$. Arrows represent the moves. When specifying
interaction sequence, the following bullet symbols are used to
represent moves: $\pmove$ for P-moves, $\omove$ for O-moves,
$\pomove$ for a move playing the role of P in $A\lingamear B$ and O
in $B\lingamear C$ and $\opmove$ for the symmetric of $\pomove$. We
sometimes add a subscript to the symbols $\pmove$ and $\omove$ to
denote the component in which the moves is played ($A$ or $C$).


\tikzstyle{state}=[rectangle,draw=blue!50,fill=blue!20,thick,minimum
height = 4ex, text width=4cm] \tikzstyle{move}=[->,shorten
<=1pt,>=latex',line width=1pt] \tikzstyle{intmove}=[dashed]
\tikzstyle{extomove}=[color=\extomovecolor]
\tikzstyle{genomove}=[]%[dashed]
\tikzstyle{genpmove}=[color=\genpmovecolor]
\def\sep{1.5cm}
\begin{figure}[htbp]
\begin{center}
\begin{tikzpicture}[node distance=1.7cm]

% the four states
\path
 node(oooT)  [state] {}
 node(opp)   [state, below of=oooT] {}
 node(pop)   [state, below of=opp]  {}
 node(oooB)  [state, below of=pop] {}
 node(title) [anchor=south, at=(oooT.north), minimum height = 4ex, text width=4cm] { };

\path
% text in the title centered in 3 columns
  ([xshift=-\sep]title) node {$A\lingamear B$}
        (title) node {$B\lingamear C$}
        ([xshift=\sep]title) node {$A\lingamear C$}

% text in the states centered in 3 columns
  ([xshift=-\sep]oooT) node {O}
        (oooT) node {O}
        ([xshift=\sep]oooT) node {O}
  ([xshift=-\sep]opp) node {O}
        (opp) node {P}
        ([xshift=\sep]opp) node {P}
  ([xshift=-\sep]pop) node {P}
        (pop) node {O}
        ([xshift=\sep]pop) node {P}
  ([xshift=-\sep]oooB) node {O}
        (oooB) node {O}
        ([xshift=\sep]oooB) node {O}

% text in between two arrows giving the arena of the move
  (oooT) to node {\bf C} (opp)
  (opp) to node {\bf B} (pop)
  (pop) to node {\bf A} (oooB)

% arrows representing the moves
  (opp.20)    edge[move, genpmove]
        node[right] {$\mu$}
        node[left]{$\pmove$} (oooT.-20)
  (oooT.-160) edge[move, extomove, genomove]
        node[left] {$env_\mu$}
        node[right]{$\omove$} (opp.160)
  (pop.20)    edge[move, genomove,genpmove,intmove]
        node[right] {$\sigma$}
        node[left]{$\pomove$} (opp.-20)
  (opp.-160)  edge[move, genomove, genpmove,intmove]
        node[left] {$\mu$}
        node[right]{$\opmove$}  (pop.160)
  (oooB.20)   edge[move, extomove,genomove]
        node[right] {$env_\sigma$}
        node[left]{$\omove$} (pop.-20)
  (pop.-160)  edge[move, genpmove]
        node[left] {$\sigma$}
        node[right]{$\pmove$} (oooB.160);

%\draw[move, genpmove] (3.5cm,-1cm) -- +(1,0) node[right] {Generalised P-move \& External P-move };
%\draw[move, genomove,genpmove] (3.5cm,-2cm) -- +(1,0) node[right] {Generalised O-move \& Generalised P-move};
%\draw[move, genomove,extomove] (3.5cm,-3cm) -- +(1,0) node[right] {Generalised O-move \& External O-move};
\draw[move] (3.5cm,-1cm) -- +(1cm,0cm) node[right] {External move};
\draw[move,intmove] (3.5cm,-2cm) -- +(1cm,0cm) node[right] {Internal
move}; \draw (3.5cm,-3cm) node[anchor=west]
{\textcolor{\extomovecolor}{External O-moves: $\omove$}}; \draw
(3.5cm,-4cm) node[anchor=west]
{\textcolor{\genpmovecolor}Generalised P-move: $\opmove, \pomove,
\pmove$};
\end{tikzpicture}
\end{center}
\caption{Structure of an interaction sequence.} \label{fig:interseq}
\end{figure}

Note that in state OPP, the alternation condition (for each of the
three games involved) prevents the players from playing in A.
Indeed, the O-moves in component $A$ of $A\lingamear B$ are also
$O$-moves in component $A$ of $A\lingamear C$ however the state name
indicates that the next move in $A\lingamear B$ must be an O-move
and the next move in $A\lingamear C$ must be a P-move.

Similarly, in the top state OOO, the players cannot make move in B
since the O-moves in component B of the game $B\lingamear C$
correspond to P-moves in the component B of $A\lingamear B$. However
the state name indicates that the next move in $A\lingamear B$ and
the next move in $B\lingamear C$ must be played by O.


Let $u \in Int(A,B,C)$ and $m$ be a move of $u$. The
\defname{component} of $m$ is $A,B$ if after playing $m$ the game is
under the control of the strategy $\sigma$ and $B,C$ otherwise (if
$\mu$ has control). In other words, the moves $\omove, \pmove \in A$
and $\opmove \in B$ shown on the diagram of Figure
\ref{fig:interseq} have component $A,B$ and $\omove, \pmove \in C$
and $\pomove \in B$ have component $B,C$.


Also we call \defname{generalized O-move in component $A,B$} moves
that play the role of O in the game $A\lingamear B$, that is to say
moves represented by $\opmove$ and $\omove_A$. Similarly $\pomove$
and $\pmove_A$ moves are the \defname{generalized P-moves in
component $A,B$}, $\omove_C$ and $\pomove$ moves are the
\defname{generalized O-moves in component $B,C$} and  $\pmove_C$ and
$\opmove$ moves are the \defname{generalized P-moves in component
$B,C$}.

The P-view (also called \emph{core} in
\cite{McCusker-GamesandFullAbstrac}) of an interaction sequence $u
\in Int(A,B,C)$, written $\overline{u}$ or $\pview{u}$ is defined
as:
\begin{align*}
\pview{u\cdot \extomove{n}} &= \extomove{n} &
\mbox{ if \extomove{$m$} is an \extomove{external O-move} initial in C,}\\
\pview{\Pstr{u\cdot (m)m\cdot v \cdot (n-m,45){\extomove{n}} }} &= \extomove{n} &\mbox{ if \extomove{$m$} is an \extomove{external O-move} non initial in C,}\\
\pview{u \cdot \genpmove{m}} &= \pview{u}\cdot \genpmove{m}  & \mbox{ if \genpmove{$m$} is a \genpmove{generalised P-move}.}\\
\end{align*}

We can show the following property by an easy induction :
\begin{lemma}
\label{lem:pviewAC_eq_ACpview}
 Let $u$ be an interaction sequence in $Int(A,B,C)$ then
$$\pview{u} \filter A,C = \pview{u \filter A,C} \ .$$
\end{lemma}
\begin{proof}
  By induction on $u$. It is trivial for the empty sequence.
Let $b$ be a move in $B$. We have $\pview{u b} \filter A,C =
\pview{u} \filter A,C$. By the I.H.\ this is equal to $\pview{u
\filter A,C} = \pview{u b\filter A,C}$. Let $m$ be a P-move in $A$
or $C$ then $\pview{u m} \filter A,C = (\pview{u} \filter A,C) m$
and by the I.H.\ this is equal to $\pview{u \filter A,C} m =
\pview{(u \filter A,C) m} = \pview{u m \filter A,C}$. Let $c$ be an
initial move in $C$. We have $\pview{u c \filter A,C}  = \pview{(u
\filter A,C) c} = c =  c \filter A,C = \pview{u c} \filter A,C$. Let
$u = \Pstr{u_1 (m){m} u_2 (n-m){n}}$ with $n$ an O-move in
$A\rightarrow C$. Then necessarily $m\in A,C$ and $ \pview{u\filter
A,C} = \pview{\Pstr[0.5cm]{u_1\filter A,C \cdot (m){m} \cdot
u_2\filter A,C \cdot (n-m,30){n}}} =
 \pview{u_1 \filter A,C} \Pstr{(m){m} (n-m){n}}$. By the I.H.\ this is equal to
$(\pview{u_1}\filter A,C) \Pstr{(m){m} (n-m){n}} = (\pview{u_1}
\Pstr{(m){m} (n-m){n}} ) \filter A,C  = \pview{u_1 \Pstr{(m){m} u_2
(n-m){n}}} \filter A,C$
\end{proof}


\subsection{P-incremental justification}


\begin{definition}\rm
A play $s m$ of even length is said to be \defname{P-incrementally
justified}, or \emph{P-i.j.} for short, if $m$ points to the last
unanswered O-question in $\pview{s}$ with order strictly greater
than $\ord{m}$.

 A strategy $\sigma$ is said to be \defname{P-incrementally justified}, if all plays in $\sigma$ ending with a P-question are
P-incrementally justified.
\end{definition}
Let $\sigma$ be a strategy. We write $Pij(\sigma)$ to denote the set of plays of $\sigma$ that are P-i.j.
We can define equivalently P-i.j.\ strategies as those verifying the relation $\sigma = Pij(\sigma)$.
\begin{proposition}
\label{prop:char_pincr}
\rm We make assumption (A1) and (A2).
Let $\sigma$ be a \emph{P-well-bracketed} strategy on an arena $A\neq \bot$.
The following statements are equivalent:
\begin{enumerate}
\item[(i)] $\sigma$ is P-incrementally justified,
\item[(ii)] for $s \, q \in \sigma$ with $q$ a P-question, $q$ points to the last O-question in $\pview{s}$ with order $>\ord{q}$,
\item[(iii)] for $s \, q \in \sigma$ with $q$ a P-question, $q$ points to the last O-move in $\pview{s}$ with order $>\ord{q}$.
\end{enumerate}
\end{proposition}
\begin{proof}
$(i)\iff(ii)$: By lemma \ref{lem:oq_in_pview_unanswered}, O-question occurring in $\pview{s}$ are all unanswered.

$(ii)\iff(iii)$: Because of (A1) and (A2), $\ord{q} \geq 0$ thus an O-move with order $>\ord{q}$ is necessarily an O-question.
\end{proof}

Putting proposition \ref{prop:char_pincr} and
\ref{prop:char_wellbrack} together we obtain:
\begin{proposition}
Under assumption (A1) and (A2).
A strategy $\sigma$ on $A\neq \bot$
is \emph{P-well-bracketed} and
 \emph{P-incrementally justified} if and only if
for $s \, m \in \sigma$, $m$ points to the last O-move in $\pview{s}$ with order $>\ord{m}$.
\end{proposition}




\section{Closed P-i.j.\ strategies}
\label{sec:closedpij}

\subsection{Definition}

\begin{definition}
\label{def:closedpij} Let $s m$ be an even-length play on some game
$A \rightarrow B$. $s m$ is said to be
\defname{closed P-incrementally justified} (closed P-i.j.\ for short)
just if
\begin{itemize}
\item $s m$ is P-incrementally justified;
\item and if $m$ is an initial move in $A$ then its justifier $n$ (initial in
$B$) verifies $\ord_A m \geq \ord_B n$.
\end{itemize}

\noindent A strategy $\sigma$ is \defname{closed P-i.j.} just if all
plays in $\sigma$ ending with a P-questions are closed P-i.j.
\end{definition}
An example of closed P-i.j.\ strategy is the identity strategy $id_A$
for any game $A$.

\begin{lemma}
\label{lem:closedpij_singleBinitmove} Let $\sigma : A \lingamear B$
be a P-i.j.\ strategy.
\begin{enumerate}[i.]
\item If for each initial move $m$ of $A$ occurring in some play of $\sigma$ we have $\ord_A m \geq \ord{B}$, then $\sigma$ is closed P-i.j.
\item Suppose that $A=A_1\times \ldots \times A_n$ where each of the $A_i$ are prime arenas. If for each initial move $m_i$ of $A_i$, for $i \in \{1..n\}$, occurring in some play of $\sigma$ we have $\ord A_i \geq \ord{B}$, then $\sigma$ is closed P-i.j.
\end{enumerate}
\end{lemma}
\begin{proof}
(i) This is a direct consequence of the definition since $\ord B \geq \ord_B b$ for every move $b$ initial in $B$.

(ii) Take an initial move $m$ of $A$. It is necessary an initial move of $A_i$ for some $i$ hence $\ord_A m = \ord_{A_i} m$ which is equal to $\ord A_i$ since $A_i$ is prime. By hypothesis this is in turn greater than $\ord{B}$ hence we can conclude using (i).
\end{proof}



We observe that every P-i.j.\ strategy $\sigma$ on the game $I
\lingamear A$ is closed P-i.j.\ while $\sigma : A$ is not
necessarily closed P-i.j.\footnote{In particular, every P-i.j.\
strategy $\sigma$ on the game $!A_1 \otimes \ldots \otimes !A_n
\lingamear B$, is isomorphic, up to arena-tagging of the moves, to
the closed P-i.j.\ strategy $\Lambda^n(\sigma)$ on the game $I
\lingamear (A_1,\ldots,A_n,B)$, where $\Lambda$ denotes the usual
{\it currying} isomorphism.}; hence the distinction between $I
\lingamear A$ and $A$ matters. This is because the definition of
closed P-i.j.\ strategy specifically refers to the moves of  the
arena in the left-hand side of the function space arrow
$\lingamear$, therefore the property is not valid up to an
isomorphism that retags the moves such as {\it currying}.

Consequently, it is possible to have two isomorphic strategies $\sigma$ and
$\mu$ such that one is closed P-i.j.\ but not the other. In contrast, the ``ordinary'' P-incremental
justification condition is preserved across the  {\it curry} isomorphism. A consequence of this remark is that the category of closed P-i.j.\ strategies
that we will introduce later on, is not closed (neither monoidal closed nor cartesian closed) and
that it only admits a weak form of {\it curry} isomorphism.

\subsection{Compositionality - A semantic proof}

{\bf Notation} In plays representations, the symbol $\omove$ stands
for an O-move and $\pmove$ for a P-move. Suppose the game considered
is $L\lingamear R$ for some game $L$ and $R$ then whenever the
sub-arena in which the move is played is known, it is specified in
subscripts ($\omove_L$, $\pmove_L$, $\omove_R$ or $\pmove_R$). For
interaction sequences in $Int(A,B,C)$ we use the symbols $\omove_A$,
$\pmove_A$, $\omove_C$, $\pmove_C$, $\opmove$ and $\pomove$ as
defined in Figure \ref{fig:interseq}. We use the variable $X$ to
denote one of the component $A,B$ or $B,C$, the variable  $Y$ then
denotes the other component. We write $s \subseqof t$ to say that
$s$ is a subsequence (with pointers) of $t$, $s \prefixof t$ to say
that $s$ is a prefix (with pointers) of $t$ and  $s \suffixof t$ to
say that $s$ is a suffix of $t$.

We now prove several useful lemmas which will become useful when studying compositionality of P-i.j.\ strategies.

\begin{lemma}
\label{lem:interjump}
Let $X$ be a component (either  $A,B$ or  $B,C$).
Let $u$ be an interaction sequence of the form
$ u =
\Pstr[0.5cm][2pt]{ \ldots (b){\stk \beta \pmove}  \ldots
 {n}  \ldots  (a-b,30){\stk \alpha\omove}
\ldots m}$ where:
\begin{itemize}[-]
\item $\alpha,\beta$ are external moves in component $X$ (necessarily both played in $A$ or in $C$),
\item  $m$ is either played in $B$ or an external P-move in $X$,
\item  $\alpha$ is visible at $m$ in $X$ \emph{i.e.}~$\alpha\in \pview{u \filter X}$ (consequently $\beta$ is also visible).
\end{itemize}
Then $n \not\in \pview{u \filter A, C}$.
\end{lemma}
\begin{proof}
Since $\alpha$ is an O-move, $\alpha$ and $\beta$ are necessarily
played in the same arena ($A$ or $C$). Take $v=u$ if $m$ is a
generalized O-move in $X$ and $v=u_{<z}$ otherwise (if $m$ is a
generalized P-move in $X$). The third assumption implies
$\alpha,\beta\in \pview{v}$. The last move in $v$ is necessarily a
generalized O-move in component $X$ (see diagram of Figure
\ref{fig:interseq}) therefore by \cite[Lemma 3.3.1]{Harmer2005} we
have $\pview{v \filter X} = \pview{\overline{v} \filter X} \subseqof
\overline{v} \subseqof \overline{u}$. Thus $\alpha,\beta \in
\overline{u}$ and since $\alpha,\beta$ are played in $A,C$ we have
$\alpha,\beta  \in \overline{u} \filter A,C = \pview{u
\filter A,C}$ (Lemma \ref{lem:pviewAC_eq_ACpview}). Finally
since $n$ lies underneath the $\beta$-$\alpha$ PO-arc it cannot
appear in the P-view  $\pview{u \filter A,C}$.
\end{proof}

\begin{lemma}
\label{lem:in_pviewAC_imp_in_pviewX}
Let $u$ be an interaction sequence in $Int(A,B,C)$ and
$n$ be a move of $u$ such that $n\in\pview{u \filter A,C}$:
\begin{enumerate}[i.]
\item
if all the moves in $u_{\suffixof n}$
are played in $C$  then $n \in \pview{u \filter B,C}$;
\item
if all the moves in $u_{\suffixof n}$ are played in $A$ then $n \in \pview{u \filter A,B}$.
\end{enumerate}
\end{lemma}
\begin{proof}
\begin{enumerate}[(i)]
\item
We show the contrapositive. Suppose that $n \not\in\pview{u \filter B,C}$. This must be due to one of the following  two
reasons:
\begin{itemize}[-]
\item $\pview{u \filter B,C}$ contains an initial move $c_0 \in C$
occurring after $n$ in $u$.


By \cite[Lemma 3.3.1]{Harmer2005}
we have $\pview{u \filter B,C} = \pview{\overline{u} \filter B,C} \subseqof \pview{u}$, thus $c_0$ also occurs in $\pview{u}$.
Since $c_0$ belongs to $C$ we have
$c_0 \in \pview{u} \filter A,C=
\pview{u \filter A,C}$ (Lemma \ref{lem:pviewAC_eq_ACpview}).
Thus the P-view $\pview{u \filter A,C}$
starts with the initial move $c_0$ and
since $n$ occurs before $c_0$, $n$ does not occur in the P-view.

\item $n$ lies underneath a PO-arc $\beta$-$\alpha$ visible
at $ u \filter B,C$.
By assumption, since $\alpha$ occurs after $n$ in $u$, it must belong to $C$. We can therefore apply Lemma \ref{lem:interjump}
with $X\assignar B,C$ which gives
$n \not\in\pview{u \filter A,C}$.
\end{itemize}

\item Suppose that $n \not\in\pview{u \filter A,B}$ then either:
\begin{itemize}[-]
\item $\pview{u \filter A,B}$ contains an initial move $b_0 \in B$
occurring after $n$ in $u$. But this is impossible since by assumption all the moves occurring after $n$ in $u$ belong to $A$.

\item or $n$ lies underneath a PO-arc $\beta$-$\alpha$ in $A,B$.
By assumption, since $\alpha$ occurs after $n$ it must belong to $A$. We can then conclude using
Lemma \ref{lem:interjump} with $X\assignar A,B$.
\end{itemize}
\end{enumerate}
\end{proof}

Note that we cannot completely relax the assumption
which says that moves in $u_{\suffixof n}$ are all in the same component.
For instance take $u = \Pstr[0.5cm]{(co){\omove_C}\thinspace
(b0-co){\opmove} \thinspace
(n){\stk{\pmove_A}{n}} \thinspace
(b1-co){\opmove}}$ then we have $n\in\pview{u\filter A,C}$ but $n\notin\pview{u\filter A,B}$.


%%%%%%%%%%%
% This commented Lemma could be useful be we did not make use of it eventually.
%
% \begin{lemma}
%\label{lem:oviewsegmentinB}
%For any legal sequence $s = \ldots x \cdot r \cdot y$ of a game $A\lingamear B$ if $x, y \in A$ and $x$ is O-visible from $y$ then any move in $r$ occurring in $\oview{s}$ belongs to $A$.
%\end{lemma}
%\begin{proof}
%We proceed by induction on the length of the segment $r$.
%Base case $r=\epsilon$ is trivial. Suppose $r = r' \cdot m$.
%If $y$ is an O-move then by the Switching Condition
%$m$ is necessarily in $A$. Clearly $x$ is O-visible from $m$ thus  by the I.H.\ any move from $r$ occurring in the O-view is in $A$.
%
%If $y$ is a P-move then it cannot point to an initial move in $B$. Indeed, suppose that it points to an initial O-move $b_0 \in B$ then
%we have $\oview{s} = b_0 \cdot y$ which contradicts the fact that $x\in \oview{s}$.
%Thus $y$ points to a move in $A$ and again we can conclude using the induction hypothesis.
%\end{proof}


\begin{lemma}[P-visibility decomposition (from $C$)]
\label{lem:middlepomove}
Let $u = \ldots n' \cdot r \cdot m \in Int(A,B,C)$ where
$n'$ is a $\omove_A$-move verifying $n' \in \pview{u\filter A,C}$ and $m$ is in $\{ \pmove_C, \opmove, \pomove \}$. Then there is a $\pomove$-move $\gamma$ in $r \cdot m$ such that $\gamma \in \pview{u\filter B,C}$ , $n' \in \pview{u_{\leq \gamma} \filter A,B}$ and $\gamma$ is justified by a move occurring before $n'$.
\end{lemma}
\begin{proof}
By induction on $|r|$.
If $r=\epsilon$ then necessarily $u = \ldots \stk{\omove_A}{n'} \thinspace\stk \pomove m$ where $m$ points before $n'$ ($n'$ being played in $A$ cannot justify $m$ played in $B$) so we just need to take $\gamma = m$.
If $|r|=1$ then either
$u = \ldots \stk{\omove_A}{n'} \pomove\thinspace\stk {\pmove_C} m$
or $u = \ldots \stk{\omove_A}{n'} \pomove\thinspace\stk \opmove m$.
In both cases we can take $\gamma$ to be the $\pomove$-move between $n'$ and $m$.
Suppose $|r|>1$. Let $m^-$ denote the move preceding $m$ in $u$.
We proceed by case analysis:
\begin{enumerate}[i.]
\item Suppose $m = \pmove_C$ and $m^- = \omove_C$.
Let $q$ be the external P-move that justifies $m^-$.
Since $n' \in \pview{u\filter A,C}$, $q$ must occur after $n'$ in $u$:
\begin{center}
\begin{tikzpicture}
\matrix (m) [matrix of math nodes]
{
A & \stackrel\sigma{\longrightarrow} & B & \stackrel\mu{\longrightarrow} & C \\
&\vdots&&\vdots\\
n' \omove\\
&\vdots&&\vdots  \\
&& & &  \node(q){\pmove q};  \\
&\vdots&&\vdots  \\
&& & &  \node(mp){\omove m^-};  \\
&& & &  \pmove m\  \\
};
\draw[tableptr] (mp.west) .. controls +(120:0.5cm) and +(210:0.5cm) .. (q.west);
\end{tikzpicture}
\end{center}

Thus we can use the induction hypothesis (with $u\assignar u_{\prefixof q}$): there is a $\pomove$-move $\gamma$
in $u_{]n',q]}$ pointing before $n'$ such that $\gamma \in \pview{u_{\prefixof q} \filter B,C}$, $n' \in \pview{u_{\prefixof \gamma} \filter A,B}$.
Moreover $\pview{u_{\prefixof q} \filter B,C} \prefixof \pview{u_{\prefixof m} \filter B,C}$ (since $q$ is visible from $m$ in $B,C$) thus we have $\gamma \in \pview{u_{\prefixof m} \filter B,C}$ as required.

\item Suppose $m = \pmove_C$ and $m^- = \pomove \in B$.
Again we can conclude using
the induction hypothesis with $u \assignar u_{\prefixof m^-}$.

\item Suppose $m = \pomove \in B$.

Suppose that all the moves in $r$ are in $A$.
Then $r$ is of the form $(\pmove_A \omove_A)^*$ (where $(\cdot)^*$ denotes the Kleenee star operator).
We just need to take $\gamma = m$.
Indeed, moves in $u_{\suffixof m}$ are all in $A$
and by assumption $n'\in\pview{u\filter A,C}$  therefore
Lemma \ref{lem:in_pviewAC_imp_in_pviewX}(ii) gives
$n'\in\pview{u\filter A,B}$.
Also, since $m$ is a $\pomove$-move,
its justifier is a $\opmove$-move but $r$ contains only $\omove$ and $\pmove$ moves hence $m$'s justifier must occur before $n'$.

Suppose that $r$ contains at least one move in $B$. Let $b$ be the last such move, then $u$ is of the form $\ldots n' \cdot \ldots \cdot \stk\opmove  b \cdot (\pmove_A \omove_A)^* \cdot\thinspace\stk\pomove m $. We then have
$u\filter B,C = \ldots n' \cdot \ldots \cdot
\thinspace\stk\opmove b \thinspace\cdot \stk\pomove m $ thus $b \in \pview{u\filter B,C}$. We can then conclude by applying the induction hypothesis with $u \assignar u_{\prefixof b}$.

\item Suppose $m = \pomove \in B$.
If $m^- = \opmove \in B$ then the I.H.\ with $u \assignar u_{\prefixof m^-}$ permits us to conclude.
If $m^- = \omove \in C$ then we conlude by applying  the I.H.\ on $u \assignar u_{\prefixof q}$ where $q$ is the external P-move in $C$ justifying
$m^-$.
\end{enumerate}
\end{proof}

We now show the lemma symmetric to the previous one:
\begin{lemma}[P-visibility decomposition (from $A$)]
\label{lem:middleopmove}
Let $u = \ldots n' \cdot r \cdot m \in Int(A,B,C)$ where
$n'$ is an O-move \emph{non initial} in $C$ verifying $n' \in \pview{u\filter A,C}$ and $m$ is in $\{\pmove_A, \opmove, \pomove\}$. Then there is a $\opmove$-move $\gamma$ in $r \cdot m$ such that $\gamma \in \pview{u\filter A,B}$ , $n' \in \pview{u_{\leq \gamma} \filter B,C}$ and $\gamma$ is justified by a move occurring before $n'$.
\end{lemma}
\begin{proof}
The proof is almost symmetrical to the previous one (Lemma \ref{lem:middlepomove}). We proceed by induction on $|r|$.
If $r=\epsilon$ then necessarily $u = \ldots \stk {\omove_C} {n'} \thinspace\stk \opmove m$ where $m$ points before $n'$ (it cannot point to $n'$
since $n'$ is not initial in $C$). Thus we just need to take $\gamma = m$.

If $|r|=1$ then either
$u = \ldots \stk {\omove_C} {n'} \thinspace\opmove\thinspace\thinspace\stk{\pmove_A} m$
or $u = \ldots \stk {\omove_C} {n'} \thinspace\opmove\thinspace\thinspace\stk \pomove m$.
In both cases we can take $\gamma$ to be the $\opmove$-move between $n'$ and $m$.
Suppose $|r|>1$. Let $m^-$ denote the move preceding $m$ in $u$.
We do a case analysis:
\begin{enumerate}[i.]
\item Suppose $m = \pmove_A$ and $m^- = \omove_A$.
Let $q$ be the external P-move that justifies $m^-$.
Since $n' \in \pview{u\filter A,C}$, $q$ must occur after $n'$ in $u$:
\begin{center}
\begin{tikzpicture}
\matrix (m) [matrix of math nodes]
{
A & \stackrel\sigma{\longrightarrow} & B & \stackrel\mu{\longrightarrow} & C \\
&\vdots&&\vdots\\
&&&& \omove\ n'\\
&\vdots&&\vdots  \\
\node(q){q \pmove};  \\
&\vdots&&\vdots  \\
\node(mp){m^- \omove};  \\
m \pmove  \\
};
\path (mp) edge[tableptr] (q);
\end{tikzpicture}
\end{center}

Thus we can use the induction hypothesis (with $u\assignar u_{\prefixof q}$): there is a $\opmove$-move $\gamma$
in $u_{]n',q]}$ pointing before $n'$ such that $\gamma \in \pview{u_{\prefixof q} \filter A,B}$, $n' \in \pview{u_{\prefixof \gamma} \filter B,C}$.
Moreover $\pview{u_{\prefixof q} \filter A,B} \prefixof \pview{u_{\prefixof m} \filter A,B}$ (since $q$ is visible from $m$ in $A,B$) thus we have $\gamma \in \pview{u_{\prefixof m} \filter A,B}$ as required.

\item Suppose $m = \pmove_A$ and $m^- = \pomove$ then again we can conclude using the I.H.\ with $u \assignar u_{\prefixof m^-}$.

\item Suppose $m = \opmove$.
\begin{itemize}[-]
\item Suppose that $r$ does not contain any move in $B$  then $r$ is of the form $(\pmove_C \omove_C)^*$.

We just need to take $\gamma = m$.
Indeed:
\begin{enumerate}
\item By lemmma \ref{lem:in_pviewAC_imp_in_pviewX}(i)
we have $n'\in \pview{u\filter B,C}$.

\item  $m$ is justified by a move occurring before $n'$.
Indeed, if $m$ is justified by a $\pomove$-move then since $n' \cdot r$ contains only $\omove$ and $\pmove$ moves, $m$'s justifier must occur before $n'$.
If $m$'s justifier is an initial $\omove_C$-move $c_i$, then
by P-visibility we have $c_i \in \pview{u\filter B,C}$
but since the P-view computation ``stops'' when reaching an initial moves, in order to guarantee that $n'$ also belongs to the P-view (as shown in (a)) it must
occurs after $c_i$.
\end{enumerate}


\item Suppose that $r$ contains some move in $B$. Let $b$ be the last such move. Then $u$ is of the form $u = \ldots n' \cdot \ldots \cdot \stk\opmove  b \cdot (\pmove_A \omove_A)^* \cdot\ \stk\pomove m $.
So we have
$u\filter B,C = \ldots n' \cdot \ldots \cdot \stk\opmove  b \cdot \stk\pomove m $ hence $b \in \pview{u\filter B,C}$. We can now
conclude by applying the I.H.\ with $u \assignar u_{\prefixof b}$.
\end{itemize}

\item Suppose $m = \pomove \in B$.
If $m^- = \pomove \in B$ then the I.H.\ with $u \assignar u_{\prefixof m^-}$ permits us to conclude.
If $m^- = \omove \in A$ then we conclude by applying the I.H.\ on $u \assignar u_{\prefixof q}$ where $q$ is the external P-move in $A$ justifying $m^-$.
\end{enumerate}
\end{proof}

We now use the two preceding Lemmas to show
the following useful result:
\begin{lemma}[Increasing order lemma]
\label{lem:increasing_order}
Let $u = \ldots n' \cdot r \cdot m \in Int(A,B,C)$ where
\begin{enumerate}
\item
$n'$ is an external O-move in compoment $X$
($n'=\omove_A$ and $X=A,B$, or $n'=\omove_C$ and $X=B,C$)  non initial in $C$,
\item $n' \in \pview{u\filter A,C}$,
\item $m$ is either played in $B$
($\opmove$ or $\pomove$) or is an external
 P-move in $Y$
($\pmove_C$ if $n'=\omove_A$ and
$\pmove_A$ if $n'=\omove_C$),
\item $m$'s justifier occurs before $n'$,
\item $u\filter X$ is P-i.j.,
\item $u_{\prefixof b}\filter Y$ is P-i.j.\ for all non-initial B-move $b$ occurring in $u$.
\end{enumerate}
Then:
$$ \ord_{Y} m \geq \ord_{A\lingamear C} n' \ .$$
\end{lemma}
\begin{proof}
If $n' =\omove_C$ (resp.~if $n'=\omove_A$)
then by Lemma \ref{lem:middleopmove}
(resp.~Lemma \ref{lem:middlepomove})
there is an occurrence in $r \cdot m$ of a non-initial B-move $\gamma$ of type $\opmove$
(resp.~$\pomove$) such that $\gamma \in \pview{u\filter Y}$ , $n' \in \pview{u_{\leq \gamma} \filter X}$ and $\gamma$ is justified by a move occurring before $n'$. By the $6^{th}$ hypothesis, $u_{\prefixof \gamma}\filter Y$ is P-i.j.

There are six possible cases depending on
the type of the moves $n'$ and $m$:
$(n',m) \in \{ \omove_A \} \times \{\pmove_C,\opmove,\pomove \}
\union \{ \omove_C \} \times \{\pmove_A,\opmove,\pomove \} $).
The following diagram illustrates the cases $(n',m)
 = (\omove_A,\pmove_C)$ (left)
and  $(n',m)
 = (\omove_C,\pmove_A)$  (right):
\begin{center}
\begin{tikzpicture}
\matrix [matrix of math nodes]
{
A & \longrightarrow & B &
 \longrightarrow & C \\
&\vdots&&\vdots\\
&&&& \node(n){\omove}; \\
&\vdots&\node(gj){\opmove};&\vdots\\
n' \omove \\
&\vdots&&\vdots  \\
&&\node(g){\gamma \pomove}; \\
&\vdots&&\vdots  \\
&&&&\node(m){m \pmove}; \\
};
\path (m) edge[tableptr] (n)
      (g) edge[tableptr] (gj);
\end{tikzpicture}
\hspace{2cm}
\begin{tikzpicture}
\matrix [matrix of math nodes]
{
A & \longrightarrow & B & \longrightarrow & C \\
&\vdots&&\vdots\\
& \node(n){\omove}; \\
&\vdots& &\node(gj){\vdots};\\
&&&&n' \omove \\
&\vdots&&\vdots  \\
&&\node(g){\gamma \opmove}; \\
&\vdots&&\vdots  \\
\node(m){m \pmove}; \\
};
\path (m) edge[tableptr] (n)
      (g) edge[tableptr] (gj);
\end{tikzpicture}
\end{center}



We have:
\begin{equation}
\ord_Y \gamma \geq \ord_X \gamma \label{eqn:gammaorderXY}
\end{equation}
Indeed, if $n' =\omove_C$ then $X=B,C$ and $Y=A,B$ and by Lemma
\ref{lem:compositionorder} we have $\ord_{A\lingamear B} \gamma \geq
\ord_{B\lingamear C} \gamma$. If $n=\omove_A$ then $\gamma$ is a
$\pomove$-move therefore it is not initial in $B$ and Lemma
\ref{lem:compositionorder} gives $\ord_{A\lingamear B} \gamma =
\ord_{B\lingamear C} \gamma$.

Hence:
\begin{align*}
\ord_{A\lingamear C} n'
& = \ord_{X} n' & \mbox{(n' non initial in $C$ \& Lemma \ref{lem:compositionorder})} \\
& \leq \ord_{X} \gamma & \mbox{($u_{\prefixof \gamma}\filter Y$ is P-i.j. \& $\gamma$'s justifier occurs before $n'$)} \\
& \leq \ord_{Y} \gamma & \mbox{(By Eq.\ \ref{eqn:gammaorderXY})} \\
& \leq \ord_{Y} m & \mbox{($u\filter X$ is P-i.j. \&
4$^{th}$ assumption: $m$'s justifier occurs before $\gamma$)}.
\end{align*}
\end{proof}


\begin{lemma}
\label{lem:visibleatprefixofu}
Let $u\in Int(A,B,C)$ such that
$u = \ldots \gamma \ldots \delta \ldots m$
where $m$ is a generalized P-move in $X$,
$\gamma \in \pview{u\filter A,C}$  and $\delta \in \pview{u\filter X}$. Then $\gamma \in \pview{u_{\prefixof \delta} \filter A,C}$.
\end{lemma}
\begin{proof}
First we remark than $\delta$ must occur in $\pview{u}$.
Indeed, $\delta \in \pview{u\filter X} = \pview{u_{< m} \filter X} \cdot m$ therefore $\delta \in \pview{u_{< m} \filter X}$ and since the move preceding $m$ in $u$ is necessarily a generalized O-move in $X$, we can use Lemma 3.3.1 from \cite{Harmer2005}:
\begin{align*}
\delta \in \pview{u_{< m} \filter X}
&= \pview{\pview{u_{<m}}\filter X} & \mbox{(Lemma 3.3.1 from \cite{Harmer2005})}\\
&\subseqof \pview{u_{<m}} \\
&\subseqof \pview{u} \ .
\end{align*}

Clearly, $\pview{u_{\prefixof \delta} \filter A,C}$ is a prefix of $\pview{u \filter A,C}$, indeed:
\begin{align*}
\pview{u_{\prefixof \delta} \filter A,C}
& = \pview{u_{\prefixof \delta}}\filter A,C
  & \mbox{(Lemma \ref{lem:pviewAC_eq_ACpview})}  \\
& \prefixof \pview{u}\filter A,C
  & \mbox{($\delta \in \pview{u}$)} \\
& = \pview{u\filter A,C}
  & \mbox{(Lemma \ref{lem:pviewAC_eq_ACpview})} \ .
\end{align*}

Finally since $\gamma \in \pview{u\filter A,C}$ and $\gamma$ occurs before $\delta$ in $u$, we necessarily have $\gamma \in \pview{u_{\prefixof \delta}\filter A,C}$.
\end{proof}

\begin{lemma}
\label{lem:compos_auxiliary_lemma}
Let $X$ be a component and $u \in Int(A,B,C)$ such that
the projection of $u$ on the component $X$ has the form:
$$ u \filter X =
\Pstr[0.5cm][2pt]{ \ldots (n){n}  \ldots
 {\stk {n'}{\omove}}  \ldots  (m-n,30){\stk m {\pmove}}
}$$
and
\begin{enumerate}
  \item $m$ and $n'$ are external move in $X$ ({\it i.e.}~in $A$ if $X =A,B$ and in $C$ if $X=B,C$);
  \item $u\filter X$ is P-i.j.;
  \item $u_{\prefixof b}\filter Y$ is P-i.j.\ for all non-initial B-move $b$ occurring in $u$.
\end{enumerate}
Then either $\ord_{A\lingamear C} n' \leq \ord_{A\lingamear C} m$ or
$n' \not \in \pview{u\filter A,C}$.
\end{lemma}
\begin{proof}

Suppose that $n'$ occurs in the P-view $\pview{u\filter X}$. Then we have
\begin{equation}
\ord_{A\lingamear C} n'  = \ord_{B\lingamear C} n' \ . \label{eqn:ordnp}
\end{equation}
Indeed, if $X$ is the component $B,C$ then necessarily $n'$ is not initial in $C$ (otherwise it would be the first move in $\pview{u \filter B,C}$, which is not the case since by visibility $n$ must occur before $n'$ in the P-view) and
if $X=A,B$ then $n'$ is in $A$. Thus in both cases, Lemma \ref{lem:compositionorder} gives us the claimed equality.

Hence we have
\begin{align*}
\ord_{A\lingamear C} n'
& = \ord_{X} n' & \mbox{(Eq.\
\ref{eqn:ordnp})} \\
& \leq \ord_{X} m & \mbox{($u\filter X$ is P-i.j.)} \\
& = \ord_{A\lingamear C} m & \mbox{(Lemma \ref{lem:compositionorder} \& $m$ is not initial in $C$)} \ .
\end{align*}

Suppose that $n'$ does not occur in the P-view $\pview{u \filter X}$, then $n'$ lies underneath a PO arc occurring in $\pview{u \filter X}$. Let us denote this arc by $\beta$-$\alpha$ where $\beta$ and $\alpha$ denote the arc's nodes. We have:
$$ u \filter X = \ldots
\Pstr[0.5cm]{
 (n){n} \ldots (b){\stk\beta \pmove} \ldots \stk{n'} {\omove}
\ldots (a-b){\stk\alpha \omove}  \ldots (m-n){\stk m {\pmove} }
} $$
with $\ord_X \alpha \leq \ord_X m$ (by P-i.j.\ of $u \filter X$).

\begin{enumerate}[i.]
\item Suppose $\alpha$ is an external move then so is $\beta$. Indeed, if $X=B,C$ and $\alpha = \omove_C$ then $\alpha$ can only point to another move in $C$ and
if $X=A,B$ and $\alpha = \omove_A$ then since $\alpha$ is an O-move in $A,B$, it is not initial in $A$ and therefore its justifier must also be in $A$.

Then instancing Lemma \ref{lem:interjump} with
$n \assignar n'$ gives us $n' \not\in\pview{u \filter A,C}$.

\item Suppose $\alpha$ is a $B$-move then necessarily so is $\beta$. Indeed, if $X=A,B$ then $\alpha \in B$
can only point to a move in $B$, and if $X=B,C$ then
since $\alpha$ is an O-move in the game $B,C$ it is not initial in $B$ and therefore its justfier must also be in $B$.

Now suppose that $n' \in \pview{u\filter A,C}$,
then by Lemma \ref{lem:visibleatprefixofu}
(with $\delta,\gamma \assignar \alpha,n'$)
we have $n' \in \pview{u_{\prefixof \alpha}\filter A,C}$,
and $u_{\prefixof \alpha}\filter Y$ is P-i.j.\ by hypothesis 3. This permits us to apply Lemma \ref{lem:increasing_order} on $u_{\prefixof \alpha}$:
\begin{align*}
\ord_{A\lingamear C} n'
& \leq \ord_{Y} \alpha & \mbox{(Lemma \ref{lem:increasing_order} with $u\assignar u_{\prefixof \alpha}$)} \\
& = \ord_{X} \alpha & \mbox{(Lemma \ref{lem:compositionorder} \& $\alpha$ non initial in $B$)} \\
& \leq \ord_{X} m & \mbox{($u \filter X$ is P-i.j.)} \\
& = \ord_{A\lingamear C} m & \mbox{(Lemma \ref{lem:compositionorder} \& $m$ is not initial in $C$)} \ .
\end{align*}
\end{enumerate}
\end{proof}


\begin{proposition}
\label{prop:closedpijcompose} Let $\sigma : A \lingamear B$ and $\mu
: B \lingamear C$ be two well-bracketed (P-visible) strategies then
\begin{enumerate}[(I)]
\item $\sigma$ closed P-i.j.\ $\wedge$ $\mu$ P-i.j.
$\implies$ $\sigma ; \mu$  P-i.j.;
\item $\sigma, \mu$ closed P-i.j.
$\implies$ $\sigma ; \mu$ closed P-i.j.
\end{enumerate}
\end{proposition}

\begin{proof}
Well-bracketing is preserved by strategy composition (see \cite[Proposition 2.5]{abramsky94full}) thus
$\sigma ; \mu$ is well-bracketed so we can use the definition of P-i.j.\ from Proposition \ref{prop:char_wellbrack}.

\noindent (I) Let us prove that $\sigma ; \mu$ is P-i.j..
Let $u$ be a play of the interaction $\sigma\ \|\ \mu$ between $\sigma$ and $\mu$
ending with an external P-move $m$
justified by $n$ in $\pview{u \filter A , C}$.
Let $n'$ be an external O-move occurring betweeen $n$ and $m$:
$$ u \filter A,C =
\Pstr[0.5cm][2pt]{ \ldots (n){\stk {n} \omove}  \ldots
 {\stk {n'} \omove}  \ldots  (m-n,30){\stk m \pmove}
}
$$
To show that $u \filter A,C$ is P-incrementally justified, we just
need to prove that either $n'\not\in \pview{u \filter A,C}$ or
$\ord_{A\lingamear C} n' \leq \ord_{A\lingamear C} m$. Note that if
$n'\in \pview{u \filter A,C}$ then necessarily $n'$ is not initial
in $C$ because $n$ occurs before $n'$ in $\pview{u \filter A,C}$.

Let $E$ denote one of the two external arenas ($A$ or $C$), $X$ be
the corresponding component ({\it i.e.}~$X=A,B$ if $E=A$ and $X=B,C$
if $E=C$) and $Y$ denote the other component.
    \begin{enumerate}[1)]
    \item Suppose $m$ and $n$ are two external moves in $E$.

        \begin{enumerate}[{1}.a)]
        \item Suppose $n' \in E$.

        This case corresponds to the situation handled by
        Lemma \ref{lem:compos_auxiliary_lemma}: we have
        either $\ord_{A\lingamear C} n' \leq
        \ord_{A\lingamear C} m$ or $n' \not\in \pview{u
        \filter A,C}$.

        \item Suppose $n' \not\in E$.

        Suppose that $n' \in \pview{u\filter A,C}$, then by
        Lemma \ref{lem:increasing_order} with $X\assignar Y$
        we have $ \ord_{A\lingamear C} n'  \leq \ord_X m$
        and since $m$ is not initial in $C$, Lemma
        \ref{lem:compositionorder} gives $\ord_X m =
        \ord_{A\lingamear C} m$, thus $\ord_{A\lingamear C}
        n' \leq \ord_{A\lingamear C} m$.
        \end{enumerate}

        \item \label{case:mA} Suppose $m \in A$ and $n \in C$.

        Then $m$ is an initial move in $A$
        pointing to a $\opmove$-move
        $b_0$ initial in $B$ which in turn points to the $\omove_C$-move $n$ initial in $C$.

        This situation cannot be handled similarly as the
        previous case. Indeed the pointer associated to the move
        $m$ in the game $A,C$ is not the same as the one
        attached to the corresponding move in the game $A,B$
        (see in \cite{Abr02} for the definition of the
        projection operation over the overall component A,C),
        hence we cannot use Lemma \ref{lem:increasing_order}
        since the condition requiring that $m$ points before
        $n'$ is not necessarily met. A more detailed analysis is
        therefore required.

        Let us assume that $n'\in \pview{u\filter A,C}$ and
        prove that we necessarily have $\ord_{A\lingamear C} n'
        \leq \ord_{A\lingamear C} m$. We do a case analysis:
        \begin{itemize}[-]
        \item Suppose $n'$ occurs before $b_0$.
        Note that we cannot apply Lemma \ref{lem:increasing_order} on $u$
        since $m$ does not point before $b_0$.
        Up to now we have only used the fact that $\sigma$ and $\mu$ are P-i.j. The assumption that $\sigma$ is  \emph{closed} P-i.j.\ now becomes crucial.

        Since $n' \in \pview{u\filter A,C}$ and
        $b_0 \in \pview{u\filter B,C}$, applying Lemma \ref{lem:visibleatprefixofu}
        with $X\assignar B,C$ and $\delta,\gamma \assignar b_0,n'$ gives
        $n' \in \pview{u_{\prefixof b_0}\filter A,C}$. This allows us to apply Lemma \ref{lem:increasing_order} on $u_{\prefixof b_0}$:
            \begin{align*}
            \ord_{A\lingamear C} m
            = \ord_A m
            & \geq \ord_B b_0 & \mbox{($u \filter A,B$ is closed P-i.j., $m$ is initial in $A$)} \\
            & = \ord_{B\lingamear C} b_0  \\
            & \geq \ord_{A\lingamear C} n' & \mbox{(Lemma \ref{lem:increasing_order} on $u_{\prefixof b_0}$ with $X\assignar A,B$)} \ .
            \end{align*}

        \item Suppose $n'$ occurs after $b_0$ (and necessarily before $m$).

            \begin{enumerate}[a.]
            \item Suppose $n'\in C$. Since $m$'s justifier occurs before $n'$ in $u$, we can use Lemma \ref{lem:increasing_order} which gives $\ord_{A\lingamear C} n' \leq \ord_{A\lingamear B} m
                = \ord_{A\lingamear C} m$.

            \item Suppose $n'\in A$.
By Lemma \ref{lem:compos_auxiliary_lemma} with $X
\assignar A,B$, since $n' \in \pview{u \filter A,C}$, we
have $\ord_{A\lingamear C} n' \leq \ord_{A\lingamear C}
m$.
\smallskip

        Note that we could not use Lemma
        \ref{lem:increasing_order} on $u$ directly since
        both $m$ and $n'$ are played in $A$. Also, in
        the ideal case where $n'$ is hereditarily
        enabled by the initial move $m$, we can
        immediately conclude $\ord_{A\lingamear C} n'
        \leq \ord_{A\lingamear C} m$; however this
        argument does not work in general: there may be
        more than one initial move in $A$ in which case
        $n'$ can be hereditarily enabled by an initial
        $A$-move distinct from $m$.
            \end{enumerate}
        \end{itemize}

    \end{enumerate}

\noindent (II) We now show that $\sigma;\mu$ is closed P-i.j.\
provided that both $\sigma$ and $\mu$ are. Take a play $s m \in
\sigma ; \mu$ such that $m$ is initial in $A$ and let $n$ be the
initial move of $C$ justifying $m$. Let $u \in \sigma \ \|\ \mu$ be
the uncovering of $s m$ ($s m = u \filter A,C$) and $b_0$ be the
initial $B$-move justifying $m$ in $u$.
 We have:
\begin{align*}
\ord_A m & \geq \ord_B b_0 & \mbox{($u \filter A,B \in \sigma$ is closed P-i.j.)} \\
 & \geq \ord_C n & \mbox{($u_{\prefixof b_0} \filter B,C \in \mu $ is closed P-i.j.)}.
\end{align*}
\end{proof}

{\it Remark:} The second part of the proposition only gives a
\emph{sufficient} condition for $\sigma ; \mu$ to be closed P-i.j.
In fact it is possible to have that $\sigma ; \mu$ is closed P-i.j.\
although $\mu$ is not.


\subsection{Tensor product}

 Given two strategies $\sigma :\ A
\lingamear B$  and $\tau :\ C\lingamear D$, their tensor product
$\sigma \otimes \tau :\ A\otimes B \lingamear C\otimes D$ is defined
as
$$\sigma \otimes \tau = \{ s \in L_{A\otimes C \lingamear B\otimes
D} \ | \ s \filter A,B \in \sigma \wedge s \filter C,D \in \tau \} $$
 where $A\otimes B$ denotes the tensor product of the games $A$ and $B$ (see \cite{abramsky:game-semantics-tutorial}).
\begin{proposition}
Let $\sigma :\ A \lingamear B$  and $\tau :\ C\lingamear D$.
\begin{enumerate}
\item If $\sigma$ and $\tau$ are P-i.j.\
then so is $\sigma \otimes \tau$;
\item If $\sigma$ and $\tau$ are closed P-i.j.\ then so is $\sigma \otimes \tau$.
\end{enumerate}
\end{proposition}

\begin{proof}
By establishing the state diagram of the game $A\otimes C \lingamear
B\otimes D$ one can show easily that only player O can switch
between the subgames $A\lingamear B$ and $C\lingamear D$.
Consequently, in the P-view of a play of the game $A\otimes C
\lingamear B\otimes D$, all the moves are played in the same subgame
({\it i.e.~} all in $A\lingamear B$ or all in $C\lingamear D$).
Hence if the last move of a play $m$ is played in $A\lingamear B$
then $\pview{s\filter A,B} = \pview{s} \filter A,B = \pview{s}$ (and
reciprocally if $m$ is played in $C\lingamear D$). The first part of
the proposition then follows immediately. The second part is also
straightforward.
\end{proof}


\subsection{Pairing} Given two strategies $\sigma :\ C \lingamear A$
and $\tau :\ C\lingamear B$, the pairing $\langle \sigma , \tau
\rangle :\ C \lingamear A\& B$ is defined as
\begin{align*}
\langle \sigma , \tau \rangle
    &= \{ s \in L_{C \lingamear A\& B} \ | \ s \filter C,A \in \sigma \wedge s \filter B = \epsilon \} \\
    & \union \{ s \in L_{C \lingamear A\& B} \ | \ s \filter C,B \in \tau \wedge s \filter A = \epsilon \}
\ .
\end{align*}
 where $A\& B$ denotes the product of the games $A$ and $B$ (see \cite{abramsky:game-semantics-tutorial}).

\begin{proposition}
\label{prop:pij_paring} Let $\sigma :\ C \lingamear A$  and $\tau :\
C\lingamear B$.
\begin{enumerate}
\item If $\sigma$ and $\tau$ are P-i.j.\ then so is $\langle \sigma , \tau \rangle$;
\item If $\sigma$ and $\tau$ are closed P-i.j.\ then so is $\langle \sigma , \tau \rangle$.
\end{enumerate}
\end{proposition}
The proof is immediate.


\subsection{Promotion} \emph{Notation:} Let $s$ be a play. We call
\defname{thread} a maximal subsequence of $s$ constituted of moves
that are hereditarily justified by the same occurrence of an initial
move. Let $m$ be a move occurring in $s$. We call thread of $m$ the
only thread in $s$ containing $m$.


We recall some definitions. Let $A$ and $B$ be two well-opened
games. Given a strategy  $\sigma :\ !A \lingamear B$, its promotion
$\sigma^\dag :\ !A\lingamear !B$ is defined as
$$ \sigma^\dag = \{ s \in L_{!A\lingamear !B}\ |\ \mbox{for all inital $m$ in $B$, } s\filter m \in \sigma \}$$
and for $\mu :\ !B\lingamear C$ the composite strategy $\sigma
\fatcompos \mu$ is defined as:
$$ \sigma \fatcompos \mu = \sigma^\dag ; \mu \ .$$

Since $B$ is well-opened, plays of $\sigma$ are constituted of a
single thread initiated by some initial $B$-move. Plays of
$\sigma^\dag$ however, are interleaves of potentially infinitely many single-threaded
plays of $\sigma$. One can show easily, using the visibility condition, that the thread of a $P$-move
is always the same as the thread of the preceding $O$-move. Consequently, the P-view of a play is equal to the P-view of the current thread:
if the current thread of a play $s$ is opened by an initial move $b \in B$ then
$\pview{s} = \pview{s \filter b} = \pview{s} \filter b$.


The state of the game is given by an infinite sequence of symbols in $\{O, P\}$, each element of the
sequence indicating who is to play in the corresponding thread.
The diagram on Figure \ref{fig:promotion_state_diagram} illustrates
how the state changes as a play of $\sigma^\dag$ unfolds.
The initial state of the game is $O^\omega$ - an infinite
sequence of O's -- which indicates that O is to play in all the
threads. When O plays an initial move in $B$, it ``opens'' a new
thread so the state of the game becomes $O^k P O^\omega$ where $k$
is the index of the thread being opened. By alternation, $P$ now has to play. His move must be played in a thread
already opened by $O$ and in which $P$ is to play; only one thread is in such state: the $k$th one. Hence after P's move
we are back to state $O^\omega$.

\tikzstyle{state}=[rectangle,draw=blue!50,fill=blue!20,thick,minimum
height = 4ex, text width=1.2cm, text centered]
\tikzstyle{state_nobg}=[thick,minimum
height = 4ex, text width=1.2cm, text centered]
\tikzstyle{omove}=[->,shorten <=1pt,>=latex',line width=0.5pt,bend left=10]
\tikzstyle{pmove}=[->,shorten <=1pt,>=latex',line width=0.5pt,bend left=10, draw=blue!50]
\begin{figure}[htbp]
\begin{center}
\begin{tikzpicture}[node distance=2cm]
% the states
\path
 node(init)  [state, text width=4cm] {$O^\omega$}
 (init)+(-2.8cm,-3cm)
 node(p)     [state, anchor=east,] {$PO^\omega$}
 node(p1)    [state, right of=p]  {$OPO^\omega$}
 node(p2)    [state_nobg, right of=p1] {\ldots}
 node(p3)    [state, right of=p2] {$O^kPO^\omega$}
 node(p4)    [state_nobg, right of=p3] {\ldots} ;
\path
% arrows representing the moves
  ([xshift=-1.4cm]init.south)  edge[omove] node[right]{O} ([xshift=0.2cm]p.north)
  (p.north)    edge[pmove] node[left]{P} ([xshift=-1.5cm]init.south)
  ([xshift=-0cm]init.south)   edge[omove] node[right]{O} ([xshift=0.2cm]p1.north)
  (p1.north)   edge[pmove] node[left]{P} ([xshift=-0.2cm]init.south)
  ([xshift=1cm]init.south)   edge[omove] node[right]{O} ([xshift=0cm]p3.north)
  ([xshift=-0.2cm]p3.north)   edge[pmove] node[left]{P} ([xshift=0.8cm]init.south);
\end{tikzpicture}
\end{center}
\caption{State diagram for plays of $\sigma^\dag$.}
\label{fig:promotion_state_diagram}
\end{figure}



\begin{proposition}
\label{prop:fatcompos_pij} If $A$ and $B$ are two well-opened games
and $\sigma :\ !A \lingamear B$ is a well-bracketed P-i.j.\ strategy
then $\sigma^\dag$ is also well-bracketed and P-i.j. Furthermore if
$\sigma$ is closed P-i.j.\ then so is $\sigma^\dagger$.
\end{proposition}
\begin{proof}
$\sigma^\dag$ is well-bracketed by \cite[Proposition
2.10.]{abramsky94full}. For P-incremental justification, the result is a direct consequence of the
fact that the P-view of a play in $\sigma^\dag$ is equal to the P-view of the current thread.
For closed P-incremental justification, the result is immediate.
\end{proof}

From propositions \ref{prop:closedpijcompose} and
\ref{prop:fatcompos_pij} we obtain:
\begin{corollary}
Let $A$ and $B$ be two well-opened games. Let $\sigma :\ !A
\lingamear B$ and $\mu :\ !B\lingamear C$ be two well-bracketed
strategies then:
\begin{enumerate}
\item If $\sigma$ is closed P-i.j.\ and $\mu$ is P-i.j.\ then $\sigma \fatcompos \mu :\ !A \lingamear
C$ is also P-i.j.;
\item If $\sigma$ and $\mu$ are closed P-i.j.\ then so is $\sigma \fatcompos \mu :\ !A \lingamear C$.
\end{enumerate}
\end{corollary}

\subsection{A category of closed P-i.j.\ strategies}

We define the category $\mathcal{G}_{Pij}$ as follows:
\begin{itemize}
\item the objects are games (as defined in \cite{abramsky:game-semantics-tutorial}),
\item the morphisms from $A$ to $B$ are the closed P-incrementally-justified strategies
on the game $A\rightarrow B$,
\item morphisms are composed using the standard game-semantic strategy composition.
\end{itemize}
This indeed defines a category. Indeed we have shown in the previous
section that closed P-i.j.\ compose, strategy composition is
associative (\cite{abramsky94full,hylandong_pcf}) and finally the
identity strategy $id_A$ for any game $A$ is closed P-i.j.

We mentioned before that such category cannot be cartesian closed. Indeed, remember that
a P-i.j.\ strategy from $A$ to $B$ is said to be \emph{closed} P-i.j.\ provided that some condition
on the arena $A\rightarrow B$ holds. This condition refers precisely to the structure of the arenas $A$ and $B$ and consequently relies on the fact the arena considered is exactly $A\rightarrow B$ (and not any other isomorphic arena).



\section{\texorpdfstring{Modeling the safe lambda-calculus in $\mathcal{G}_{Pij}^{inn}$}
{Categorical model of the safe lambda calculus}}

Consider the category $\mathcal{G}_{Pij}$ defined in section
\ref{sec:closedpij}. In this section we show how the safe lambda
calculus can be modeled in the sub-category
$\mathcal{G}_{Pij}^{inn}$ of innocent P-ij strategies.

\subsection{The language}

We write ``Safe $\Lambda^{\rightarrow}$'' to denote the safe lambda
calculus as defined in Sec.\ \ref{sec:safe}, Def.\
\ref{def:safelambda}.


\subsection{Game-semantic denotation}
In \cite{blumong:safelambdacalculus} we showed\todo{replace by a reference in the thesis. Cite the paper in the introduction of the thesis.} that in the game
semantic model safe lambda terms are denoted by P-i.j.\ strategies.
The argument was syntactic: it is based on the analysis of a special
kind of abstract syntax tree of a term called computation tree
\cite{OngLics2006}. Here we give another proof based on a semantic
argument that uses the results of section \ref{sec:closedpij}.


\begin{proposition}
\label{prop:safe_closepij_sem}
  Safe simply-typed terms are denoted by closed P-i.j.\ strategies.
\end{proposition}
\begin{proof}
  By induction on the formation rules.
  \begin{enumerate}
    \item \rulenamet{var} $\sem{x:A \vdash_s x:A } = id_A$. Clearly the
    identity strategy is closed P-i.j.

    \item \rulenamet{wk} Take $\Gamma \subset \Delta $ and suppose $\sem{\Gamma \vdash_s
    s : A}$ is closed P-i.j. Up to an appropriate retagging of
    the moves the two strategies $\sem{\Delta \vdash_s s : A}$
    and $\sem{\Gamma \vdash_s s : A}$ are isomorphic. Hence
    $\sem{\Delta \vdash_s s : A}$ is P-i.j. It is also closed
    P-i.j.\ since none of the new initial moves introduced by
    $\Delta$ occurs in any play of the strategy.

    \item \rulenamet{app} By Lemma \ref{lem:almostsafe_is_appplicative_safe}, an almost safe term is  an applicative term built out of safe terms therefore any term constructed with the rule \rulenamet{app} is of the form
$\Gamma \sentail t_0 t_1 \ldots t_n : B$
with $\Gamma \sentail t_0 : (A_1,\ldots,A_n,B)$ and $\Gamma \sentail t_i : A_i$ for $i \in\{ 1..n\}$. By the I.H., for $i \in \{0..n\}$ the strategy
 $\sem{t_i}$ is closed P-i.j. We then have $\sem{t_0 t_1 \ldots t_n} = \langle \sem{ t_0}, \sem{t_1}, \ldots, \sem{t_n} \rangle \fatsemi ev^n$
    where $ev^n$ is the $n$-parameter evaluation strategy. By
    Proposition \ref{prop:pij_paring} the strategy $\langle \sem{t_0}, \sem{  t_1} , \ldots, \sem{t_n} \rangle$ is closed P-i.j. Since the evaluation map
    $ev^n$ is P-i.j.\ (but not necessarily closed P-i.j.),
    by Proposition \ref{prop:closedpijcompose}.I.\
    $\sem{ \Gamma \sentail t_0 t_1 \ldots t_n : B}$ is P-i.j. The
    game $\sem{\Gamma}$ is equal to $\sem{Y_1}
    \times \ldots \times \sem{Y_n}$ where the $Y_i$s are the
    types of the variables in $\Gamma$. Every
    $\sem{Y_i}$ is prime since our calculus involves only pure simple
    types without product. Moreover the side-condition of the
    rule gives $\ord Y_i \geq \ord B$ for all $i \in \{1..n\}$.
    Hence by Lemma \ref{lem:closedpij_singleBinitmove}(ii),
    $\sem{ \Gamma \sentail t_0 t_1 \ldots t_n : B}$ is closed
    P-i.j.

    \item \rulenamet{abs} Suppose that $\sem{\Gamma, x_1 : A_1, \ldots, x_n : A_n \vdash_s
    s : B}$ is closed P-i.j. Then the isomorphic strategy $\sigma = \sem{\Gamma \vdash_s \lambda
    x_1 \ldots x_n . s : (A_1,\ldots,A_n,B)}$ is also P-i.j.
    Again, using the side-condition, Lemma \ref{lem:closedpij_singleBinitmove}(ii)
    implies that $\sigma$ is closed P-i.j.
  \end{enumerate}
\end{proof}

\subsection{The safe lambda calculus with product}
It is possible to add product types to the safe lambda calculus as
follows. The grammar of types is extended with a product type:
\begin{align*}
T ::=& \ B,  \quad \mbox{for some base type $B$} \\
   &| \ T \rightarrow T \\
   &| \ T \times T
\end{align*}
and the typing system is extended with three rules corresponding to
pairing, first projection and second projection (respectively
$\rulename{\times}$, $\rulename{\pi_1}$ and $\rulename{\pi_2}$ in
Def.\ \ref{def:safecalc_withprod}).

It is easy to check that most of the good properties of the safe
lambda calculus remains in this extended calculus: the free
variables of a term have order greater than the order of the term
itself; the no-variable-renaming Lemma holds and terms are denoted
by P-i.j.\ strategies. In general, however, terms are not denoted by
\emph{closed} P-i.j.\ strategies. Indeed, in the safe lambda
calculus without product - as defined in the previous section - all
the arenas involved are prime {\it i.e.}~they have a single initial
move. In the present case however, since types can be constructed
using the cartesian product, the corresponding arenas can have many
initial moves. Therefore the hypothesis of Lemma
\ref{lem:closedpij_singleBinitmove}(ii) are not met so we cannot use it to show that a strategy is closed P-i.j. And indeed, when pairing and projection are available, one can build very simple terms with a denotation that is P-i.j.\ but not closed P-.i.j. Take for instance the term $x : (o^1\rightarrow o^2)\times o^3 \vdash \lambda y^o . \pi_2 x : o^4 \rightarrow o^5$; its denotation is P-i.j.\ and contains the play $q^5 q^3$. But we have $\ord_{(o^1\rightarrow o^2)\times o^3} q^3 = 0 < 1 = \ord_{o^4
\rightarrow o^5} q^5$ therefore it is not closed P-i.j.


There are two different approaches to overcome this problem. The
first one consists in restricting the types of the variables
appearing in the context of a term. More precisely we require that
whenever a variable has product type $A \times B$ then we have $\ord
A = \ord B$. This condition rules out the previous counter-example.
One can easily check that it suffices to guaranty that such terms
are indeed denoted by closed P-i.j.\ strategies.

The other approach is less restrictive but requires us to modify
slightly the side-conditions of the application and abstraction
rules: instead of requiring that all variables in the context have
order greater than the order of the term, we require that the order
of \emph{any prime sub-type of any variable} in the context has
order greater that the order of the term, where the set $Pr(A)$ of
prime sub-types of a type $A$ is defined by:
\begin{align*}
Pr(B) &= \{ B \} \qquad \mbox{ for some base type } B \\
Pr(A\rightarrow B) &= \{ A\rightarrow B \} \\
Pr(A\times B) &= Pr(A) \union Pr(B)
\end{align*}

This gives rise the following calculus
\begin{definition}\rm
\label{def:safecalc_withprod}
 The \defname{safe lambda calculus with
product}, or Safe $\Lambda^{\rightarrow}_\times$ for short, is given
by induction over the following rules (the rules inside the frame are those that
are added on top of the usual rules of the safe lambda calculus):
$$ \rulename{var} \ \rulef{}{x : A\sentail x : A} \qquad
\rulename{const} \ \rulef{}{\sentail f : A}~f \in \Xi \qquad
\rulename{wk} \ \rulef{\Gamma \sentail s : A}{\Delta \sentail s : A}
\quad \Gamma \subset \Delta$$
\begin{center}
\framebox{$\dps \rulename{\times} \ \rulef{\Gamma \sentail s : A \qquad \Gamma \sentail t : B}
{\Gamma \sentail \langle s, t \rangle : A \times B}
\qquad \rulename{\pi_1} \ \rulef{\Gamma \sentail s : A \times B}
{\Gamma \sentail \pi_1 s : A} \qquad
 \rulename{\pi_2} \ \rulef{\Gamma \sentail s : A \times B}
{\Gamma \sentail \pi_2 s : B}$}
\end{center}

$$ \rulename{app_{as}} \ \rulef{\Gamma \asentail s : A\typear B
\quad \Gamma \sentail t : A} {\Gamma \asentail s~t : B}
\qquad
\rulename{\delta} \ \rulef{\Gamma \sentail s : A}{\Gamma \asentail s : A}
$$

$$ \rulename{app} \ \rulef{\Gamma \asentail s : A\typear B
\quad \Gamma \sentail t : A} {\Gamma \sentail s~t : B} \ C(\Gamma ; B)$$

$$ \rulename{abs} \ \rulef{\Gamma, x_1 : A_1, \ldots, x_n : A_n
  \asentail s : B} {\Gamma \sentail \lambda x_1 \ldots x_n . s :
  (A_1, \ldots ,A_n,B)} \ C(\Gamma ; (A_1, \ldots ,A_n,B) )$$

\noindent where the side-condition $C(\Gamma ; B)$ expresses that
$\forall y:Y \in \Gamma. \forall Y' \in Pr(Y) . \ord Y' \geq \ord
B$. The subscripts in $\sentail$ and $\asentail$ stand for ``safe''
and ``almost safe'' respectively.
\end{definition}

One can show by induction on the rules that the game denotations of
terms of this calculus are closed P-i.j.\ (the argument is similar to
the one used in the proof of Proposition
\ref{prop:safe_closepij_sem}). However this syntax does not
completely capture all the closed P-i.j.\ strategies. Take for
instance the simply-typed term $x:(o\rightarrow o)\times o \vdash \lambda z^o. \
(\pi_1 x) : o \rightarrow (o \rightarrow o)$; Its denotation is
closed P-i.j.\ but it is not typable in
$\Lambda^{\rightarrow}_\times$.


\section{\texorpdfstring{Modeling safe PCF in $\mathcal{G}_{Pij}^{inn}$}
{Categorical model of safe PCF}}

\subsection{Safe PCF}

\notetoself{Insert here the definition of Safe PCF from transfer
thesis or refer to a definition given in a previous chapter.}

\subsection{Game-semantic denotation (semantic argument)}

\begin{proposition}
\label{prop:safepcf_closedpij} Safe PCF terms are denoted by closed
P-incrementally justified strategies.
\end{proposition}
\begin{proof}
We first prove the result for $\pcf_1$ - the fragment of \pcf\
containing terms of the form $\Omega_A = Y (\lambda x : A.x)$ but
where no other use of Y is allowed (see
\cite{abramsky:game-semantics-tutorial}). The proof is by structural
induction over the structure of the term.
\begin{itemize}
\item The strategy $\sem{\Omega_A} = \bot$ is
clearly closed P-i.j.

\item The functional rules are treated the same way as in the
corresponding proof for the safe lambda calculus.

\item For the arithmetic rules, we observe that the strategies
$succ$, $pred$ and $cond$ are all closed P-i.j. The fact that
pairing and strategy composition preserve closed P-incremental
justification permits us to conclude.
\end{itemize}

We now lift the result to full PCF using the technique of
\emph{syntactic approximant} (see
\cite{abramsky:game-semantics-tutorial}). By \cite[lemma
16]{abramsky:game-semantics-tutorial} we have
$$ \sem{M} = \Union_{n\in\omega} \sem{M_n}$$
where $M_n$ is the $\pcf_1$ term obtained from $M$ by replacing each
subterm of the form $Y N$ with $Y^n N_n$, and $Y^n F$ denotes the
$n$th approximant of $Y F$. Since the $M_n$s are $\pcf_1$ terms, by
the previous result each $\sem{M_n}$ is closed P-i.j.\ and since
closed P-incremental justification is clearly a continuous property,
$\sem{M}$ is also closed P-i.j.
\end{proof}


\subsection{Full abstraction}

\subsubsection{O-incremental justification}

We define \defname{O-incremental justification} as the dual of
P-incremental justification ({\it i.e.}~the role of O and P is
exchanged in the definition). This is the restriction that needs to
be imposed on the Opponent to reflect the fact that the environment
is itself a safe term. This duality is very much like the one
between O-visibility and P-visibility (see
\cite[Sec.~3.6]{Harmer2005}).

This duality can be formally explained as follows: Let $\sigma : A$
and $\mu : A \rightarrow o$ be two strategies, the first
representing the program and the second representing the
environment. Let $q$ be the initial move of the game $A \rightarrow
o$. Then P-views of plays in $A$ correspond to O-views in the game
$A \rightarrow o$. Indeed, for $s\in L_A$ we have $q s \in L_{A
\rightarrow o}$ and due to alternation, $q \pview{s}^A = \oview{q s
}_{A \rightarrow o}$. Consequently, if $\sigma$ is P-i.j.\ then the
play involved in the interaction between $\sigma$ and $\mu$ are all
O-i.j.\ from $\mu$'s perspective. Indeed, let $u \in \sigma \| \mu$
with $|u|\geq1$. Then $u=q v$ and $u\filter A = v \filter A$ is
P-i.j. By the previous remark, this implies that $q (v\filter A) =
(q v)\filter (A \rightarrow o) = u \filter (A \rightarrow o)$ is
O-i.j.
\smallskip

Now if we regard $\sigma$ as the denotation of some closed term
$\vdash M:A$ and $\mu$ as the denotation of some context $x:A \vdash
C[x]:o$ then what the previous remark says is that non O-i.j.\ plays
are useless for the purpose of studying observational equivalence!
This suggests that it is not necessary to include non O-i.j.\ plays
in the game denotation of safe terms. However before removing
completely those plays from the game model, we have to ensure that
this does not prevent us from being able to construct a category:
\begin{lemma}
\label{lem:oij_decomp}
Let $\sigma : A\rightarrow B$ and $\tau : A\rightarrow B$ be closed P-i.j.\ strategies and suppose
that $u\in \sigma \| \tau$ such that for all external O-moves $o$ of $u$, we have that $u_{\prefixof o} \filter A,C$ satisfies
O-incremental justification. Then, for any generalized O-move $m$ of $u$ in component $X$, we have that
$u_{\prefixof m} \filter X$ satisfies O-incremental justification
\end{lemma}
\notetoself{Proof forthcoming! }

 This lemma states that O-i.j plays cannot be obtained from
the interaction of plays that are not O-i.j. In other words, if we
write $\mathcal{O}(\sigma)$ for the set of O-i.j.\ plays of
$\sigma$, then the Lemma can be restated equivalently as:
\begin{eqnarray}
     \forall \sigma, \tau\ \mbox{closed P-i.j.}: \mathcal{O}(\sigma) ; \mathcal{O}(\tau) \supseteq \mathcal{O}(\sigma ; \tau)
     \label{eqn:oijdecomp_1}
\end{eqnarray}
which in turn is equivalent to
\begin{eqnarray}
    \forall \sigma, \tau\ \mbox{closed P-i.j.}: \mathcal{O}( \mathcal{O}(\sigma) ; \mathcal{O}(\tau) ) = \mathcal{O}(\sigma ; \tau)
    \label{eqn:oijdecomp_2}
\end{eqnarray}
Indeed, Eq.~\ref{eqn:oijdecomp_1} implies the right-to-left inclusion and the other inclusion
is given by the fact that $\mathcal{O}(\sigma) ; \mathcal{O}(\tau) \subseteq \sigma;\tau$.


In some sense, Lemma \ref{lem:oij_decomp} is the dual of the proposition stating that closed P-i.j.\
strategies compose, since  the latter can be reexpressed more succinctly with the relation:
\begin{eqnarray}
     \forall \sigma, \tau .\, \mathcal{P}(\sigma) ; \mathcal{P}(\tau) \subseteq \mathcal{P}(\sigma ; \tau)
     \label{eqn:pijcomp_1}
\end{eqnarray}
where $\mathcal{P}(\sigma)$ is define as be the largest even-length-prefix-closed subset of $\sigma$ consisting of closed P-i.j.\ plays.

\subsubsection{A category of incremental strategies}

\notetoself{
-Incremental strategies means O-i.j.\ and closed P-i.j.

}

\subsubsection{Full abstraction}

The fully-abstract game-model of PCF is also fully-abstract for the
safe fragment of PCF when observational equivalence is defined with
respect to unrestricted ({\it i.e.}~possibly unsafe) PCF contexts.
However one may ask what is a fully abstract model of Safe PCF with
respect to \emph{safe} contexts.




\notetoself{
By the definability results for Safe PCF, it should be possible to
prove that the category $\mathcal{C}^{inn}_{OP-incr}$ of
OP-incrementally justified and innocent strategies is fully abstract
for Safe PCF. We can use the same proof as in the PCF case: we have
a compact test strategy $\alpha:A\rightarrow N$ and by definability,
there must be some context $x:A \vdash C[x] : N$ such that $\sem{x:A
\vdash C[x] : N} = \alpha$. The definability result for Safe PCF
gives us that $\lambda x . C[x] : A \rightarrow N$ is safe which in
turns implies that $x:A \vdash C[x] : N$ is safe since $\ord{N} =
0$.
}

\subsubsection{Algorithmic game semantics}
We recall that Strongly Safe IA $\subseteq$ Safe IA $\subseteq$ IA.
Up to order $3$, it is conservative, with respect to observational equivalence, to add unsafe context to safe ones.
At order $4$, it is not conservative anymore.

\paragraph{Observational equivalence}
\begin{table}
\begin{tabular}{|c|c||c|c|c|c|c|}
    \cline{3-7}
  \multicolumn{2}{c|}{}  & \multicolumn{5}{c|}{Finitary fragments} \\ \hline
  \multirow{2}{*}{$L$} & \multirow{2}{*}{$C[\_]$} &   order 2          &  order 2       & order 3     & order 3 & \multirow{2}{*}{order 4}  \\
                       &                          &    + while         &   + $Y_1$      & + while     & +$Y_0$  &          \\ \hline \hline

  \multirow{4}{*}{IA}  & \multirow{2}{*}{IA}      & \multirow{4}{2cm}{PSPACE$^{(1)}$ \\ {\small $\preccurlyeq$ DFA}} & \multirow{4}{*}{U$^{(2)}$} & \multirow{4}{2.8cm}{EXP-complete$^{(3)}$ \\ {\small $\preccurlyeq$ VPA} }  & \multirow{4}{2cm}{D$^{(4)}$ \\ {\small $\preccurlyeq_{exp}$ DPDA\\ $\succcurlyeq$ DPDA} } & \multirow{2}{*}{U$^{(5)}$}\\
                       &                          &                    &                    &  & & \\
\cline{2-2}\cline{7-7} & \multirow{2}{*}{Safe IA} &                    &                    &  & & \multirow{2}{*}{?} \\
                       &                          &                    &                    &  & & \\ \hline

  \multirow{4}{*}{Safe IA} & \multirow{2}{*}{IA}      & \multirow{4}{2cm}{PSPACE \\ {\small $\preccurlyeq$ DFA}} & \multirow{4}{*}{U} & \multirow{4}{2.3cm}{EXP-complete \\ {\small $\preccurlyeq$ VPA}} & \multirow{4}{2cm}{D \\ {\small $\preccurlyeq_{exp}$ DPDA\\ $\succcurlyeq$ DPDA} } & \multirow{2}{*}{U} \\
                           &                          &                    &                & & & \\
\cline{2-2}\cline{7-7}     & \multirow{2}{*}{Safe IA} &                    &                & & & \multirow{2}{*}{?} \\
                           &                          &                    &                & & & \\ \hline

  \multirow{4}{*}{St. Safe IA} & \multirow{2}{*}{IA}           & \multirow{4}{*}{D} & \multirow{4}{*}{?} & \multirow{4}{*}{D} & \multirow{4}{*}{D} & \multirow{2}{*}{?} \\
                               &                               &                    &                    &                    &                    & \\
\cline{2-2} \cline{7-7}        &  \multirow{2}{*}{St. Safe IA} &                    &                    &                    &                    & \multirow{2}{*}{?} \\
                               &                               &                    &                    &                    &                    & \\ \hline
\end{tabular}
\caption{Decidability (and complexity) of observational equivalence for some finitary fragments of IA}

U stands for Undecidable and D stands for decidable with unknown complexity, $\preccurlyeq P$ means ``reducible to problem $P$''
and $\succcurlyeq P$ means ``at least as hard as problem $P$''.
\begin{asparaenum}
\item[1.] See \cite{ghicamccusker00}.
\item[2.] Showed by Ong in \cite{OngLics2006}.
\item[3.] See \cite{DBLP:conf/fossacs/MurawskiW05}.
\item[4.] See \cite{DBLP:conf/icalp/MurawskiOW05}.
\item[5.] By encoding of $\Sigma$-machine (turing complete) into IA$_4$, see \cite{murawski03program}.

\end{asparaenum}
\end{table}

\paragraph{Observational approximation}

Observational approximation has been shown to be undecidable at order $1$ already, for the fragment $IA_1 + Y_0$ (\cite{DBLP:conf/fossacs/MurawskiW05}).


\notetoself{
- Characterization of the set of complete plays for Safe IA.
(Easy adaptation of the corresponding result for IA. In the present case however, the proof relies
on the fact that plays of the strategy are O-i.j. (in order for $\alpha$ to be P-i.j.)
}


\subsection{What is a model of Safe PCF/Safe IA?}

\notetoself{
- Define the notion of incremental category.

- Show that any incremental category is a model of Safe PCF and that any model of Safe PCF
is an incremental category.

- Show that the category of games and OP-i.j. strategies is an incremental category.

}



\section{\texorpdfstring{Modeling Safe IA in $\mathcal{G}_{Pij}$}
{Categorical model of safe IA}}

 \notetoself{ - I need to merge this
section with the other note on Safe IA.

- $\iavar =  \iacom^{\omega}\times \iaexp$

- Any strategy on the game $I \lingamear\ !\iavar$ is P-i.j.\ (and
thus closed P-i.j.) since there is no P-question in the arena
\iavar. Hence the strategy $cell$ is P-i.j. }

\subsubsection{Game-semantic denotation}

In this section, our aim is to extend the game-semantic characterization of safety to Safe IA.

We first observe that the result extends extends trivially from Safe PCF to Strongly Safe IA:
\begin{proposition}
  Strongly Safe IA terms are denoted by closed P-i.j. strategies.
\end{proposition}
\begin{proof}
The proof is an adaptation of the proof for Safe PCF. We first show that the result holds for the
fragment of Strongly Safe IA in which the only allowed uses of $Y$ are in terms of the form $\Omega$.
This is done by induction over the structure of the term:
The functional rules and the arithmetic rules are treated
the same way as in the proof for Safe PCF. For the imperative rules, we
observe that the strategies $assign$, $deref$, $mkvar$, $seq$ and
$cell$ are all closed P-i.j. The fact that pairing, tensor product
and strategy composition all preserve closed P-incremental
justification permits us to conclude.

The result is then lifted to the whole of Strongly Safe IA using syntactic approximants as in the PCF case.
\end{proof}

Now we would like to extend this result to full Safe IA:
\begin{proposition}
\label{prop:safeia_closedpij} Safe IA terms are denoted by closed
P-incrementally justified strategies.
\end{proposition}

This happens to be less trivial than for the previous restricted fragment of Safe IA. We first introduce some definitions:

\begin{definition}[P-i.j. modulo $\mathfrak{M}$]
\label{def:pij_modulo} Let $\sigma$ be a strategy on some game $A$
and $\mathfrak{M}$ be a set of moves. We say that $\sigma$ is P-i.j.
modulo $\mathfrak{M}$ iff for all $s m \in \sigma$ with $m \not\in
\mathfrak{M}$, the play $s m$ is P-i.j.

Similarly we say that $\sigma$ is \emph{closed} P-i.j. modulo
$\mathfrak{M}$ iff for all $s m \in \sigma$ with $m \not\in
\mathfrak{M}$ the play $s m$ is \emph{closed} P-i.j.

Hence a strategy is P-i.j. if and only if it is P-i.j. modulo
$\emptyset$.
\end{definition}

Given a term $\Gamma | \Gamma^{\ianew} \safeentail M : A$, we write
$\sem{\Gamma | \Gamma^{\ianew} \safeentail M : A}$ to denote the
game denotation of the corresponding IA term {\it i.e.}
$\sem{\Gamma, \Gamma^{\ianew} \vdash M : A}$. Instead of showing
Proposition \ref{prop:safeia_closedpij} we will prove the following
more general result:
\begin{proposition}
\label{prop:safeia_closedpijmodulo} Let $\Gamma | \Gamma^{\ianew}
\safeentail M : A $ be a Safe IA term. Its denotation $\sem{\Gamma |
\Gamma^{\ianew} \safeentail M : A}$ is closed P-i.j. modulo
$\mathfrak{M}_{\Gamma^{\ianew}}$ where
$\mathfrak{M}_{\Gamma^{\ianew}}$ is the set of initial moves in
$\Gamma^{\ianew}$.
\end{proposition}

\begin{remark}
Since the context $\Gamma^{\ianew}$ contains variable of type
\iavar\ only, $\mathfrak{M}_{\Gamma^{\ianew}}$ contains only moves
of the form `$read$' or `$write_i$' for some $i\in \nat$.
\end{remark}

\begin{lemma}
\label{lem:leftcompos_preserv_pijmodulo}
 Let $\sigma : A \rightarrow
B$ and $\mu : B \rightarrow C$.
  Let $\mathfrak{M}$ be any set of moves initial in $A$.
  If $\sigma$ is closed  P-i.j. modulo $\mathfrak{M}$ and $\mu$ is
  P-i.j. (resp. closed P-i.j.) then $\sigma \fatsemi \mu$ is P-i.j. (resp. closed P-i.j.) modulo $\mathfrak{M}$.
\end{lemma}
\notetoself{
\begin{proof}
Let us analyze the proof of compositionality for closed P-i.j.
strategies.

\end{proof}
}


\begin{lemma}
\label{lem:cellcomposition_preserve_pijmodulo} Let $\tau : I
\rightarrow C_2$, $\sigma : C_1 \otimes C_2 \rightarrow B$  and
$\mathfrak{M}$ be any set of moves initial in $C_1 \otimes
C_2$.
  If $\tau$ is P-i.j. and
  $\sigma$ is P-i.j. (resp. closed P-i.j.) modulo $\mathfrak{M}$
  then $(id_{C_1} \otimes \tau) \fatsemi \sigma$ is P-i.j. (resp. closed P-i.j.) modulo $\mathfrak{M} \inter C_1$.
\end{lemma}
\notetoself{
\begin{proof}
\end{proof}
}


\notetoself{
\begin{proof}[Proof of Prop.\ \ref{prop:safeia_closedpijmodulo}]
We first prove the result for the fragment of Safe IA where the only allowed uses of the $Y$ combinator is in terms of the form $\Omega$. By induction and and case analysis on the structure of Safe IA terms:
\begin{itemize}
  \item[$\rulename{var}$, $\rulename{var^\ianew}$
  , $\rulename{wk}$, $\rulename{wk^\ianew}$]

  These cases are treated the same way as in the corresponding
  proof for the Safe Lambda Calculus.

  \item[$\rulename{app}$]

  \item[$\rulename{abs}$]


  \item[$\rulename{const}$]
  \item[$\rulename{succ}$, $\rulename{pred}$,$\rulename{cond}$]

  \item[$\rulename{seq}$]
  \item[$\rulename{assign}$]
  \item[$\rulename{deref}$]
  \item[$\rulename{new}$] $\Gamma | \Gamma^\ianew \safeentail \ianewin{x}\ M : B$.

Let $cell : I \rightarrow !\iavar$ denotes the ``storage cell''
strategy (see \cite{abramsky:game-semantics-tutorial}).  Let
$\sigma = \sem{\Gamma | \Gamma^\ianew, x : \iavar \safeentail M
: B}$.  We have $\sem{\Gamma | \Gamma^\ianew \safeentail
\ianewin{x}\  M : B} = (id_{\Gamma,\Gamma^\iavar} \otimes cell)
\fatcompos \sigma$. By induction hypothesis $\sigma$ is closed
P-i.j. modulo $\mathfrak{M}_{\Gamma^{\ianew} \otimes !\iavar}$
and one can easily check that $cell$ is P-incrementally
justified. Instancing Lemma
\ref{lem:cellcomposition_preserve_pijmodulo} with $\tau
\leftarrow cell$, $C_1 \leftarrow \Gamma \otimes  \Gamma^\ianew$
and $C_2\leftarrow !\iavar$ gives us the desired result.
  \item[$\rulename{mkvar}$]

\end{itemize}
The result is then lifted to the whole of Safe IA using the technique of syntactic approximants and using the fact that the ``closed P-i.j.'' property is continuous.
\end{proof}
}


\subsection{Algorithmic game semantics}

There is an important theorem (\cite{AM97a}) in game semantics
which states that two IA terms are equivalent if and only if the set
of complete plays of their game denotations are equal. This result was used in \cite{ghicamccusker00} to show that observational
equivalence for the $IA_2$ fragment of IA is decidable -- the set of
complete plays being representable by regular expressions. In
\cite{Ong02} it was shown that it is still decidable
 for $IA_3+Y_0$. Indeed, for this fragment, the set of complete plays becomes context-free
therefore the problem reduces to the DPDA equivalence problem which
is itself decidable (with an unknown complexity).

Imposing the safety condition should lead to some improvement in
complexity. The complexity of  Safe $IA_3$ (resp. Safe $IA'_3$) for
instance, must be lower than the complexity of the DPDA equivalence
problem. Moreover the fact that Safe $IA_3$ (resp. Safe $IA'_3$)
contains terms whose denotation is context free -- e.g. $\lambda f .
f (\lambda x .x )$ -- strongly suggests that its complexity is
strictly higher than the complexity of regular language equivalence.

Murawski \cite{Murawski2003} has shown that observational
equivalence for $\ialgol_4$ is undecidable. The proofs proceeds by
showing that the computations of $\Gamma$-machine -- some variation
of queue machines that are Turing complete -- are representable
using IA terms.

Does this result extend to the safe fragments? For Safe IA, it does,
simply because the $IA_4$ term exhibited in \cite{Murawski2003} to
represent computations of the $\Gamma$-machine is also a Safe IA
term. The same argument does not carry over to Very Safe IA since
the term is not typable in this language. We do not know whether
observational equivalence is decidable for Strongly Safe $IA_4$.



\subsection{Expressivity of Safe IA/Strongly Safe IA}

Murawski representability : Safe IA representable languages are
exactly the context free languages. For Strongly Safe IA however, we
believe that the representable languages are a proper subclass of
the context free languages.













\section{Remarks}
\subsection{Homogeneity constraint}

Type homogeneity is not preserved after composition. Indeed the
types  $o \typear (o \typear o)$ and $(o \typear o) \typear \left((o
\typear o) \typear o \right)$ are homogeneous but $o \typear
\left((o \typear o) \typear o\right)$ is not.

If $A\typear B$ and $B \typear C$ are homogeneous types then  a
sufficient condition for $A\typear C$ to be homogeneous is
``$\ord{A} \geq \ord{B}$''.
