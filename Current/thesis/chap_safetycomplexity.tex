\def\ie{{\it i.e.}\ }





A famous result by Statman  states that deciding the $\beta\eta$-equality of two first-order typable lambda terms is not elementary recursive \cite{Statman:1979:TLE}.
The idea of the proof is to encode the Henkin quantifier elimination of Type Theory into the simply-typed lambda calculus. The encoding relies on the fact that the function $\sf sg$ (conditional) can be encoded in the lambda-calculus. Hence the argument does not carry on   in the Safe Lambda Calculus since the conditional operator is not definable (\cite{blumong:safelambdacalculus}).

Mairson gave a simpler proof of Statman's theorem in \cite{mairson1992spt} which also proceeds by encoding the Henkin quantifier elimination procedure into the lambda-calculus. However this proof is much easier to understand as it makes use of list iteration to perform quantifier elimination.



\section{Definability}

\subsection{Extending Church numerals to other structures}
Church numerals provide a way to encode numbers and numeric
functions in the simply-typed lambda calculus. The same mechanism
can be used to encode other kinds of structures such as words, lists
and trees as we are about to see.

Take the Church numeral $\lambda s z. s^n z$. One can view the
variable $s$ and $z$ as recursive type constructors: $z$ plays the
role of the base type constructor for the number 0 and $s$ is the
constructor that builds the successor of the number represented by
its parameter. This process is reminiscent of the recursive type
constructor of functional programming languages such as ML. For
instance the type of integer would be defined as
\begin{verbatim}
  type Int = Succ of Int | Zero
\end{verbatim}
and the number $n$ would be encoded as $\overbrace{Succ(Succ( \ldots
Succ(}^{n \hbox{ times}}Zero) \ldots))$.

ML also offers the possibility to define parameterized constructors
which allows us to define the type of lists as:
\begin{verbatim}
  type list = Cons of 'a * list | Nil
\end{verbatim}
For example the list $[a_1;a_2;a_3]$ is encoded by the expression
$Cons(a_1,Cons(a_2,Cons(a_3,Nil))))$. Church Numerals can be
naturally extended to encode this kind of data-structures. For
instance the above-mentioned list can be encoded by the term
$\lambda c^{\tau\typear\sigma\typear\sigma} n^\sigma . c a_1 ( c a_2
(c a_3 n))$ where $\tau$ is the type of the list elements, $\sigma$
is the type of lists and $c$ and $n$ play the role of the
constructors $Cons$ and $Nil$ respectively.

Now suppose that we fix an alphabet $\Sigma = \{ a_1, \ldots, a_p
\}$, then the type of words (or string using the ML terminology)
would be defined in ML as:
\begin{verbatim}
  type word = A1 of word | A2 of word | ... | Ap of word | empty
\end{verbatim}
where each constructor $Ai$ for $i \in\{1..p\}$ can be used to
append the letter $a_i$ at the beginning of a word. For instance the
word $a_1 a_3 a_5$ would be encoded as $A1(A3(A5(empty)))$. Again,
this kind of structures can be defined in the lambda calculus by
extending the Church Numeral encoding: the previous word would be
encoded as $\lambda c_1^{\tau\typear\sigma\typear\sigma}
c_1^{\tau\typear\sigma\typear\sigma} \ldots
c_1^{\tau\typear\sigma\typear\sigma} n^\sigma . c_1 ( c_3 (c_5 n))$.

Similarly we can encode tree structures in the lambda calculus. In
fact it should now become clear that all the structures that are
definable using the inductive types of ML can also be encoded in the
simply-typed lambda calculus.


In theoretical terms, the concepts of recursive ML type is conveyed
by the notion of free algebras: an algebra is determined by a
signature specifying the arity (\ie number of parameters) of each
constructors. For instance the algebra of natural numbers is given
by the signature $[1,0]$, the algebra of words over an alphabet of
size $n$ has signature $[\overbrace{1,\ldots,1}^{n},0]$ and the
algebra of unlabeled binary trees has signature $[2,0]$.


\subsection{Definability}

Schwichtenberg's result on definability of numeric functions in the
lambda calculus was followed by similar results concerning other
structures definable in the simply-type lambda calculus such as
words, trees and free algebras in general
\cite{DBLP:journals/tcs/Leivant93,DBLP:journals/apal/Zaionc91,702481,DBLP:journals/tcs/Zaionc87}.






take a number and . the first Instead of defining numbers, The
notion of By generalizing the notion of Church Numeral, it is
possible to Schwichtenberg's result on definability of numeric
functions in the lambda caclulus has been extended to other


\subsection{Word operation definable in the Safe lambda calculus.}
