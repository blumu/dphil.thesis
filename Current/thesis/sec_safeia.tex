\psset{linecolor=darkGreen,linewidth=0.5pt}


\section{Safe Idealized Algol}

In this section we present two possible approaches to accommodate
the safety restriction to a language featuring block-variable constructs such as Idealized Algol. This gives rise to two
different versions of ``Safe Idealized Algol''. In the first
version, all free variables are required to satisfy the safety
constraint whereas in the second version, only variables that are
not abstracted by the \ianew\ construct (including variables of type
\iavar) are required to satisfy the safety constraint.

We recall the definition of the \defname{safety constraint} $SC(\Gamma ; B)$
for some finite type alphabet $\Gamma$ and type $B$:
$$SC(\Gamma ; B) \quad \mbox{iff} \quad  \forall x:A\; \in \Gamma . \forall A' \in Pr(A) . \ord A' \geq \ord B$$
where $Pr(A)$ denotes the set of prime sub-types of $A$.



\subsection{Strongly Safe IA}

 The obvious way to define safety for IA terms consists in extending the safe lambda calculus by adding rules
 for constants and constructs of IA.
Said equivalently, it is the system of rules of IA where the
application and abstraction rules  have been restricted by the
safety constraint as in the safe lambda calculus. Let us call this
language \defname{Strongly Safe IA}. The rules are formally given in
Table \ref{tab:safeia_formrules}. The circled rules are those that
differs from their IA counterpart. Note that the IA rule $\rulename{new}$ is kept unchanged: it has no side-condition, contrary to the $\rulename{abs}$ rule. In other words, the calculus allows us to abstract variables of type \iavar\ without having to satisfy the safety constraint provided that this abstraction is performed by a block-variable allocation.


\begin{FramedTable}
 {\bf Functional part}

$$ \rulename{var} \ \rulef{}{x : A  \vsafeentail x : A}
\qquad \rulename{wk} \ \rulef{\Gamma \vsafeentail s :
A}{\Delta \vsafeentail s : A} \quad \Gamma \subset
\Delta
$$

\begin{center}
\ovalbox{%
\begin{Bcenter}
$ \dps \rulename{app} \ \rulef{\Gamma  \vsafeentail s :
(A_1,\ldots,A_n,B) \quad \Gamma  \vsafeentail t_1 : A_1 \quad \ldots
\quad \Gamma  \vsafeentail t_n : A_n
} {\dps \Gamma  \vsafeentail s t_1 \ldots t_n : B} \quad SC(\Gamma ; B)$ \\[12pt]
$ \dps  \rulename{abs} \ \rulef{\Gamma, x_1 : A_1, \ldots, x_n : A_n
  \vsafeentail s : B} {\Gamma \vsafeentail \lambda x_1 \ldots x_n . s :
  (A_1, \ldots ,A_n,B)} \quad SC(\Gamma ; (A_1, \ldots ,A_n,B) )$
\end{Bcenter}
}
\end{center}
\smallskip


{\bf Arithmetic and recursion}
$$ \rulename{const} \ \rulef{}{\emptyset  \vsafeentail n :\texttt{exp}}
\qquad \rulename{succ} \ \rulef{\Gamma \vsafeentail M:\texttt{exp} }{\Gamma \vsafeentail \texttt{succ}\ M:\texttt{exp}}
\qquad \rulename{pred} \ \rulef{\Gamma \vsafeentail M:\texttt{exp} }{\Gamma \vsafeentail \texttt{pred}\ M:\texttt{exp}}$$

$$
\rulename{cond} \ \rulef{\Gamma\vsafeentail M : \texttt{exp} \quad \Gamma\vsafeentail N_1 : \texttt{exp} \quad \Gamma\vsafeentail N_2 : \texttt{exp} }{\Gamma \vsafeentail \texttt{cond}\ M\ N_1\ N_2}
\quad  \rulename{rec} \ \rulef{\Gamma \vsafeentail M : A\rightarrow A }{ \Gamma \vsafeentail Y_A M : A}$$

{\bf Imperative constructs}
$$ \rulename{seq} \ \rulef{\Gamma| \Gamma^{\ianew} \vsafeentail M : \texttt{com} \quad \Gamma \vsafeentail N :A}
    {\Gamma \vsafeentail \texttt{seq}_A \ M\ N\ : A} \quad A \in \{ \texttt{com}, \texttt{exp}\}$$

$$ \rulename{assign} \ \rulef{\Gamma \vsafeentail M : \texttt{var} \quad \Gamma \vsafeentail N : \texttt{exp}}
    {\Gamma \vsafeentail \texttt{assign}\ M\ N\ : \texttt{com}}
\qquad \rulename{deref} \
 \rulef{\Gamma \vsafeentail M : \texttt{var}}
    {\Gamma \vsafeentail \texttt{deref}\ M\ : \texttt{exp}}$$

$$ \rulename{new} \ \rulef{\Gamma , x : \texttt{var} \vsafeentail M : A}
    {\Gamma  \vsafeentail \texttt{new } x \texttt{ in } M : A} \quad A \in \{ \texttt{com}, \texttt{exp}\}$$

$$ \rulename{mkvar} \ \rulef{\Gamma \vsafeentail M_1 : \texttt{exp} \rightarrow \texttt{com} \quad \Gamma \vsafeentail M_2 : \texttt{exp}}
    {\Gamma \vsafeentail \texttt{mkvar } M_1\ M_2\ : \texttt{var}}$$

\caption{Formation rules for Strongly Safe IA}
\label{tab:verysafeia_formrules}
\end{FramedTable}
\bigskip

\subsection{Safe IA}

It turns out that the definition of Strongly Safe IA is needlessly
restrictive. Indeed, the major benefit of the safety constraint is
that it avoids variable renaming when performing substitution. To
achieve this, the safety constraint restricts the occurrences of the
variables in a term. In the present setting of a language with
imperative features, one can distinct two types of variables: The
``standard ones'' -- those that we abstract using
$\lambda$-abstraction, for instance $x$ in the term $\lambda x . x$;
and the ``imperative'' ones -- those that are used as identifier for
memory cell, also called block-allocated variables. We observe that
for this second type of variables there is no gain in imposing any
constraint since no substitution can ever occur for such variables
when performing reduction! Inspired by this observation we can now identify a new fragment of IA subsuming Strongly Safe IA and which incarnates the notion ``safety''
 more faithfully.

We consider a new judgment of the form $\Gamma | \Gamma^{\ianew}
\safeentail M : A$ - called a ``split term'' in \cite{abramsky:game-semantics-tutorial} - where the context is partitioned into two
components: The first component contains standard lambda calculus
variables; the second component contains block-allocated variable --
abstracted by an occurrence of the \ianew\ construct. The component
$\Gamma^{\ianew}$ contains variables of type \iavar\ only while the
other component can contain variables of any type including the type
\iavar. It is straightforward to redefine the typing rule of IA in
such a way that these two distinct contexts are maintained
appropriately. In particular the abstraction rule must only be able
to abstract variables taken from the first component of the context;
similarly the \ianew\ block construct must only be used to
``abstract'' variables from the block-variable context. The full
system of rules is given in Table \ref{tab:safeia_formrules}. The
rules that are circled correspond to those that differs from their
counterpart in Table \ref{tab:verysafeia_formrules}; the differences
being that:
\begin{itemize}
\item the rule $\rulename{abs}$ and $\rulename{new}$ can only
abstract variables from their respective context component;
\item the side-condition in $\rulename{abs}$ refers only to
the first context component.
\end{itemize}

\begin{FramedTable}
{\bf Functional part}

$$ \rulename{var} \ \rulef{}{x : A | \emptyset \safeentail x : A} \quad
\rulename{var^{\ianew}} \ \rulef{}{\emptyset | x : \iavar \safeentail x : \iavar} $$
$$
\rulename{wk} \ \rulef{\Gamma | \Gamma^{\ianew} \safeentail s : A}{\Delta | \Gamma^{\ianew} \safeentail s : A} \quad
\Gamma \subset \Delta \qquad
\rulename{wk^{\ianew}} \ \rulef{\Gamma| \Gamma^{\ianew} \safeentail s : A}{\Gamma| \Delta^{\ianew} \safeentail s : A} \quad
\Gamma^{\ianew} \subset \Delta^{\ianew}
$$

where $\Gamma^{\ianew}$ and $\Delta^{\ianew}$ are finite sets of variables
of type \iavar.

$$ \rulename{app} \ \rulef{\Gamma | \Gamma^{\ianew} \safeentail s : (A_1,\ldots,A_n,B) \quad
\Gamma | \Gamma^{\ianew} \safeentail t_1 : A_1 \quad \ldots \quad \Gamma | \Gamma^{\ianew} \safeentail t_n : A_n
} {\Gamma | \Gamma^{\ianew} \safeentail s t_1 \ldots t_n : B} \ SC(\Gamma ; B)$$

\begin{center}\ovalbox{$
 \rulename{abs} \ \dps  \rulef{\dps \Gamma, x_1 : A_1, \ldots, x_n : A_n |
\Gamma^{\ianew}
  \safeentail s : B} {\Gamma | \Gamma^{\ianew} \safeentail \lambda x_1 \ldots x_n . s :
  (A_1, \ldots ,A_n,B)} \ SC(\Gamma ; (A_1, \ldots ,A_n,B) )$}\end{center}


{\bf Arithmetic and recursion}
$$ \rulename{const} \ \rulef{}{\emptyset | \emptyset \safeentail n :\texttt{exp}}
\quad \rulename{succ} \ \rulef{\Gamma | \Gamma^{\ianew}\safeentail M:\texttt{exp} }{\Gamma| \Gamma^{\ianew} \safeentail \texttt{succ}\ M:\texttt{exp}}
\quad \rulename{pred} \ \rulef{\Gamma| \Gamma^{\ianew} \safeentail M:\texttt{exp} }{\Gamma| \Gamma^{\ianew} \safeentail \texttt{pred}\ M:\texttt{exp}}$$

$$
\rulename{cond} \ \rulef{\Gamma| \Gamma^{\ianew} \safeentail M : \texttt{exp} \quad \Gamma| \Gamma^{\ianew} \safeentail N_1 : \texttt{exp} \quad \Gamma| \Gamma^{\ianew} \safeentail N_2 : \texttt{exp} }{\Gamma| \Gamma^{\ianew} \safeentail \texttt{cond}\ M\ N_1\ N_2}
\quad  \rulename{rec} \ \rulef{\Gamma| \Gamma^{\ianew} \safeentail M : A\rightarrow A }{ \Gamma| \Gamma^{\ianew} \safeentail Y_A M : A}$$

{\bf Imperative constructs}
$$ \rulename{seq} \ \rulef{\Gamma| \Gamma^{\ianew} \safeentail M : \texttt{com} \quad \Gamma| \Gamma^{\ianew} \safeentail N :A}
    {\Gamma| \Gamma^{\ianew} \safeentail \texttt{seq}_A \ M\ N\ : A} \quad A \in \{ \texttt{com}, \texttt{exp}\}$$

$$ \rulename{assign} \ \rulef{\Gamma| \Gamma^{\ianew} \safeentail M : \texttt{var} \quad \Gamma| \Gamma^{\ianew} \safeentail N : \texttt{exp}}
    {\Gamma| \Gamma^{\ianew} \safeentail \texttt{assign}\ M\ N\ : \texttt{com}}
\qquad
 \rulename{deref} \ \rulef{\Gamma| \Gamma^{\ianew} \safeentail M : \texttt{var}}
    {\Gamma| \Gamma^{\ianew} \safeentail \texttt{deref}\ M\ : \texttt{exp}}$$


\begin{center}\ovalbox{$
\rulename{new} \ \dps \rulef{\Gamma | \Gamma^{\ianew}, x :
\texttt{var} \safeentail M : A}
    {\Gamma | \Gamma^{\ianew} \safeentail \texttt{new } x \texttt{ in } M : A} \quad A \in \{ \texttt{com}, \texttt{exp}\} $}\end{center}

$$ \rulename{mkvar} \ \rulef{\Gamma| \Gamma^{\ianew} \safeentail M_1 : \texttt{exp} \rightarrow \texttt{com} \quad \Gamma| \Gamma^{\ianew} \safeentail M_2 : \texttt{exp}}
    {\Gamma| \Gamma^{\ianew} \safeentail \texttt{mkvar } M_1\ M_2\ : \texttt{var}}$$

\caption{Formation rules for Safe IA} \label{tab:safeia_formrules}
\end{FramedTable}


Split-terms with an empty block-variable context $\Gamma^\ianew$ are called \emph{semi-closed}. We define \defname{Safe IA} to be the set of semi-closed split-terms typable with the system of rules of Table \ref{tab:safeia_formrules}. For convenience we introduce the additional rule
$$ \rulef{\Gamma | \emptyset \safeentail M : A} {\Gamma \safeentail M : A}$$
so that Safe IA is exactly the set of terms of the form $\Gamma \safeentail M : A$.

\subsubsection{Examples}

Clearly Strongly Safe IA is a subset of Safe IA, the following example
showing that the inclusion is strict:

$$ \safeentail \lambda f^{(\iaexp \rightarrow \iacom) \rightarrow \iaexp} . \ianewin{i}\ f ( \lambda x . \iaassign\ i\ x ) : \iaexp$$
$$ \mbox{ but } \not \vsafeentail \lambda f^{(\iaexp \rightarrow \iacom) \rightarrow \iaexp} . \ianewin{i}\ f ( \lambda x . \iaassign\ i\ x ) : \iaexp \ .$$

\notetoself{Need more examples}


\subsection{Game-semantic analysis via a syntactic argument}

