% -*- TeX -*- -*- Soft -*-
% -*- TeX -*- -*- Soft -*-
% -*- TeX -*- -*- Soft -*-
% -*- TeX -*- -*- Soft -*-
\input{thesis.pre}


\makeindex

%\includeonly{chap_gamesem,chap_pincrjust,sec_safeia,transfer_chap_gsemsafety,../lmcs/safelambda,../corresp/corresp}

%\includeonly{corresp}
%\includeonly{fromlmcs/safelambda}

\author{William Blum}
\title{The safe lambda calculus  \\{\small DPhil thesis}}
\college{Linacre College}
\degree{Doctor of Philosophy}
\degreedate{?}
\renewcommand{\crest}{\beltcrest}

%\institution{Oxford University Computing Laboratory}
\date{Draft of \today}

%set the number of sectioning levels that get number and appear in the contents
\setcounter{secnumdepth}{3}
\setcounter{tocdepth}{3}

\begin{document}
\maketitle

%\setcounter{chapter}{0}
%\chapapp{Chapter}
\begin{abstract}
We consider a syntactic restriction for higher-order grammars called \emph{safety}  that  constrains occurrences of variables in the production rules according to their type-theoretic order. We transpose and generalize this restriction to the setting of the simply-typed lambda-calculus, giving us what we call the \emph{safe lambda calculus}. We study this language under different angles. First we give an account of its game semantic model. For that purpose, we introduce a new concrete presentation of game semantics based on the theory of \emph{traversals}: We show that the \emph{revealed game denotation} of a term can be computed by traversing some souped-up version of the abstract syntax tree of the term using adequately defined traversal rules. This result was presented at the Galop workshop at ETAPS 2008. This allows us to give a game-semantic analysis of safety via syntactic reasoning: We show that  safe lambda-terms are denoted by what we call \emph{P-incrementally justified strategies}. This result was presented at TLCA 2007.

We study the expressivity of the calculus and show a result in the
same vein as Schwichtenberg's 1976 characterization of the
simply-typed lambda calculus, we show that the numeric functions
representable in the safe lambda calculus are exactly the
multivariate polynomials; thus conditional is not definable. We
also give a characterization of representable word functions.
We then study the complexity of deciding beta-eta equality of two safe simply-typed terms and show that this problem is PSPACE-hard.

Finally we consider extension of the safety restriction to functional languages with recursion and references such as Idealized Algol.

\end{abstract}

\begin{romanpages}
\tableofcontents
\listoffigures
\end{romanpages}

    \chapter*{Acknowledgment}
    %\input{acknowledgment.texi}

    \chapter{Introduction}
    \input{chap_introduction.texi}


\part{Background}
    \chapter{Higher-Order Grammars and the Safety Restriction}
    \input{chap_hog_hors.texi}

    \chapter{Lambda Calculus, PCF, Idealized Algol}
    \input{chap_languages.texi}

    % chapter presenting game semantics
    \chapter{Game Semantics}
    \input{chap_gamesem.texi}



\part{Contribution}


In the second chapter we present the \emph{safe $\lambda$-calculus}.
Originally, \emph{safety} has been introduced as a syntactical
restriction on higher-order grammars in order to show a decidability
result about MSO theory of infinite trees \citep{KNU02}. In
\cite{safety-mirlong2004}, Aehlig, de Miranda and Ong  proposed an
adaptation of the safety restriction to the $\lambda$-calculus. This
restriction gives rise to the safe $\lambda$-calculus. We first
present this calculus and then give a more general definition which
does not make any assumption on the types of the terms.

In the third chapter, following ideas described in
\cite{OngLics2006}, we introduce the notions of computation tree of
a simply-typed term and traversal over a computation tree. We prove
a theorem showing a correspondence between traversals of the
computation tree and the game semantics of a term. Based on that
correspondence, we give a characterisation of the game semantics of
safe terms by a property called ``P-incremental-justification''. In
P-incrementally-justified strategies, P-pointers are superfluous (i.e.
they can be recovered uniquely from the underlying sequence of
moves and from O-moves' pointers). This simplification of the game semantics suggests some potential applications in algorithmic game semantics. We finish the
chapter by extending the result to safe \pcf\ and by giving the key
elements for an extension to full Safe Idealized Algol.




\chapter{Safe Higher Order Functional Languages}
\label{chap:safelambda}
%    \section{Safe Lambda Calculus}
%        \subsection{Definition and properties}
%        \subsection{Expressivity of the calculus}
%        Result a la Schwichtenberg \cite{citeulike:622637}
%        Statman's result for Safe Lambda calculus?
    \input{safelambda.texi}
    \input{safe_complexity.texi}
    \input{safe_expressivity.texi}


    %% chapter from the transfer thesis
    \input{transfer_chap_safe_homog.texi}
    \input{transfer_chap_safe_nonhomog.texi}


    \section{Safe PCF}
        \subsection{Definition and properties}
        \subsection{Game-semantic analysis via a syntactic argument}

    %\section{Safe IA}
    \input{sec_safeia.texi}


\chapter{Local Computation of \texorpdfstring{$\beta$}{Beta}-Reduction}
    \label{chap:localbeta}
    \input{corresp.texi}

    \section{Extension to PCF and IA terms}
         \input{corresp_pcf_ia.texi}


    \section{Applications}


\chapter{Game-Semantic Analysis via a Syntactic Argument}
    \label{chap:syntactic_gamesem}
    \input{safe_gamesem.texi}
    \input{chap_safety_gamesem_syntactic.texi}


\chapter{Game-Semantic Models of Safe Languages}
    \label{chap:model}
    \input{chap_pincrjust.texi}



\chapter{Conclusion}
    \label{chap:conclusion}
    \input{chap_further.texi}


\bibliographystyle{plain}
\bibliography{../bib/dphil-all}

\printindex

    %adds the bibliography to the table of contents
    \addcontentsline{toc}{chapter}
         {\protect\numberline{Bibliography\hspace{-96pt}}}


\end{document}



\makeindex

%\includeonly{chap_gamesem,chap_pincrjust,sec_safeia,transfer_chap_gsemsafety,../lmcs/safelambda,../corresp/corresp}

%\includeonly{corresp}
%\includeonly{fromlmcs/safelambda}

\author{William Blum}
\title{The safe lambda calculus  \\{\small DPhil thesis}}
\college{Linacre College}
\degree{Doctor of Philosophy}
\degreedate{?}
\renewcommand{\crest}{\beltcrest}

%\institution{Oxford University Computing Laboratory}
\date{Draft of \today}

%set the number of sectioning levels that get number and appear in the contents
\setcounter{secnumdepth}{3}
\setcounter{tocdepth}{3}

\begin{document}
\maketitle

%\setcounter{chapter}{0}
%\chapapp{Chapter}
\begin{abstract}
We consider a syntactic restriction for higher-order grammars called \emph{safety}  that  constrains occurrences of variables in the production rules according to their type-theoretic order. We transpose and generalize this restriction to the setting of the simply-typed lambda-calculus, giving us what we call the \emph{safe lambda calculus}. We study this language under different angles. First we give an account of its game semantic model. For that purpose, we introduce a new concrete presentation of game semantics based on the theory of \emph{traversals}: We show that the \emph{revealed game denotation} of a term can be computed by traversing some souped-up version of the abstract syntax tree of the term using adequately defined traversal rules. This result was presented at the Galop workshop at ETAPS 2008. This allows us to give a game-semantic analysis of safety via syntactic reasoning: We show that  safe lambda-terms are denoted by what we call \emph{P-incrementally justified strategies}. This result was presented at TLCA 2007.

We study the expressivity of the calculus and show a result in the
same vein as Schwichtenberg's 1976 characterization of the
simply-typed lambda calculus, we show that the numeric functions
representable in the safe lambda calculus are exactly the
multivariate polynomials; thus conditional is not definable. We
also give a characterization of representable word functions.
We then study the complexity of deciding beta-eta equality of two safe simply-typed terms and show that this problem is PSPACE-hard.

Finally we consider extension of the safety restriction to functional languages with recursion and references such as Idealized Algol.

\end{abstract}

\begin{romanpages}
\tableofcontents
\listoffigures
\end{romanpages}

    \chapter*{Acknowledgment}
    %\input{acknowledgment.texi}

    \chapter{Introduction}
    \input{chap_introduction.texi}


\part{Background}
    \chapter{Higher-Order Grammars and the Safety Restriction}
    \newcommand\lcalculrec{\Lambda^{\rightarrow}_\Sigma+Y}

\begin{proposition}
Higher-order recursion schemes are equivalent to the simply-typed lambda calculus extended with recursion and $\Sigma$-constants.
\end{proposition}
This is shown straightforwardly by showing that every higher-order recursion scheme can be converted into an equivalent lambda-term and conversely.

Let $\lcalculrec$ denotes the simply-typed lambda calculus extended with the typed-constants $\Sigma$ and the recursion combinator $Y$.


\begin{itemize}
\item First direction: Take a recursion scheme $\mathcal{R} = \langle \Sigma, \mathcal{N}, \mathcal{R}, S \rangle$.
We can construct an equivalent lambda term over the constant $\Sigma$ by induction on the rewriting rules as follows: we define a function $\Pi : \mathcal{A}(\Sigma,\mathcal{N}) \funto $ 

\end{itemize}

    \chapter{Lambda Calculus, PCF, Idealized Algol}
    \input{chap_languages.texi}

    % chapter presenting game semantics
    \chapter{Game Semantics}
    \chapter{Game semantics}

The aim of this chapter is to introduce game semantics. It starts
with a history of game semantics and a presentation of the full
abstraction problem for PCF which has been solved using game
semantics. It then goes on by introducing the basic notions of game
semantics and by giving a categorical interpretation of games.
Finally we show how games are used to define a syntax-independent
model of programming languages like PCF and Idealized Algol (IA).

This chapter is largely based on the tutorial by Samson Abramsky tutorial on Game Semantics \cite{AM98a}.
Most of the proof will be omitted and we refer the reader to
\cite{hylandong_pcf, abramsky94full} for a deeper description
of game semantics with complete proofs.

\section{History}

\subsection{Game semantics}

In the 1950s, Paul Lorenzen invented Game semantics as a tool to
study semantics of intuitionistic logic \citep{lor61}.

Four decade later, Abramsky proved the full completeness of
Multiplicative Linear Logic (MLL) using game semantics
\citep{abramsky92games}. Shortly after, game semantics has been used
as tool to study models of programming languages. In game semantics,
the meaning of a program is given by a strategy in a two-player
game. One player, the Opponent, represents the environment while the
other, the Proponent, represents the system.


\subsection{Model of programming languages}

Before the 1980s, there were many approaches to define models for
programming languages. Among the successful ones, there were the
axiomatic, operational and denotational semantics:
\begin{itemize}
\item Operational semantics gives a meaning to a program by describing the
behaviour of a machine executing the program. It is defined formally
by giving a state transition system.
\item Axiomatic semantics defined the behaviour of the program
with axioms and is used to prove program correctness by static
analysis of the code of the program.
\item The denotational semantics approach consists in mapping a program to a mathematical structure
having good properties such as compositionality. This mapping is
achieved by structural induction on the syntax of the program.
\end{itemize}

In the 1990s, three different independent research groups: Samson
Abramsky, Radhakrishnan Jagadeesan and Pasquale Malacaria
\citep{abramsky94full}, Martin Hyland and Luke Ong
\citep{hylandong_pcf} and Nickau \citep{Nickau:lfcs94} have
introduced game semantics, a new kind of semantics, in order to
solve a long standing problem in the semanticists community :
finding a fully abstract model for PCF.

\subsection{The problem of full abstraction for PCF}

PCF is a simple programming language introduced in a classical paper
by Plotkin ``LCF considered as a programming language''
(\cite{DBLP:journals/tcs/Plotkin77}). PCF is based on LCF, the Logic
of Computable Functions devised by Dana Scott in \cite{scott_lcf}.
It is a simply typed lambda calculus extended with arithmetic
operators, conditional and recursion.

The problem of the Full Abstraction for PCF goes back to the 1970s.
In \citep{scott93}, Scott gave a model for PCF based on domain
theory. This model gives a sound interpretation of observational
equivalence: if two terms have the same domain theoretic
interpretation then they are observationally equivalent. However the
converse is not true: there exist two PCF terms which are
observationally equivalent but have different domain theoretic
denotation. We say that the model is not fully abstract.

The key reason why the domain theoretic model of PCF is not fully
abstract is that the parallel-or operator defined by the following
truth table
\begin{center}
\begin{tabular}{l|lll}
p-or  & $\bot$ & tt & ff \\ \hline
$\bot$ & $\bot$ & tt & $\bot$\\
tt & tt & tt & tt\\
ff & $\bot$ & tt & ff\\
\end{tabular}
\end{center}
is not definable as a PCF term! It is possible to create two
different PCF terms that always behave the same except when they are
apply to a term computing p-or. Since p-or is not definable in PCF,
these two terms will have the same denotation. This implies that the
model is not fully abstract.

One can patch PCF by adding the operator $p-or$, the resulting
language ``PCF+p-or'' now becomes fully-abstracted by Scott domain
theoretic model \citep{DBLP:journals/tcs/Plotkin77}. However the
language we are now dealing with is strictly more powerful than PCF,
it allows parallel execution of commands whereas PCF only permits
sequential execution.

Another approach consists in getting rid of the undefinable elements
(like p-or) by strengthening the conditions on the function used in
the model (a condition stronger than strictness and continuity) but
unfortunately this approach did not succeed.

The only successful approaches to obtain a fully abstract model for
PCF were the ones taken by Ambramsky, Jagadeesan and Malacaria
\citep{abramsky94full}, Hyland and Ong \citep{hylandong_pcf} and
Nickau \citep{Nickau:lfcs94}, all based on game semantics.

This result has then been adapted to other varieties of programming
paradigm including languages with stores (Idealized Algol),
call-by-value \citep{honda99gametheoretic, abramsky98callbyvalue}
and call-by-name, general referencees
\citep{DBLP:conf/lics/AbramskyHM98}, polymorphism
\citep{DBLP:journals/apal/AbramskyJ05}, control features
(continuation and exception), non determinism, concurrency. In all
these cases, the game semantics model led to a syntax-independent
fully abstract model of the corresponding language.

\section{Games}
\label{sec:catgames}

We now introduce formally the notion of game that will be used in
the following section to give a model of the programming languages
PCF and Idealized Algol. The definitions are taken from
\cite{abramsky:game-semantics, hylandong_pcf, abramsky94full}.


\subsection{Arenas and Games}

The games we are interested in are two-players games. The players are named O for Opponent and P for Proponent.

The game played by O and P is constraint by something called
\emph{arena}. The arena defines the possible moves of the game. By
analogy with real board games, the arena represents the board
together with the rules that tell how players can make their moves
on the board. In fact the analogy with board game stops here. Our
games can be thought as dialog games: one person O interviews
another person P, P tries to answer the initial O-question by
possibly asking O some precisions about its initial question.
Moreover, the notion of winner and winning strategy will not be
relevant in our setting.


More formally, the arena can be seen as a forest of trees whose nodes are possible questions and leaves are possible answers.
The arena is partitioned into two kinds of moves: the moves that can be played by P and the ones that can be played by O.
A move is either a question to the other player or an answer to a question previously asked by the other player.

Each move of the game must be justified by another move that has already been played by the other player. This justification relation
is induced by the edges of the forest arena. Moreover, an answer must always be justified by the question that it answers and a question
is always justified by another question.

\begin{dfn}[Arena]
An arena is a structure $\langle M, \lambda, \vdash \rangle$ where:
\begin{itemize}
\item $M$ is the set of possible moves;
\item $(M,\vdash)$ is a forest of trees;

\item $\lambda : M \rightarrow \{ O, P\} \times \{Q, A\}$ is a labeling functions indicating whether a given move
    is a question or an answer and whether it can be played by O or by P.

    $\lambda = [\lambda^{OP},\lambda^{QA}]$ where $\lambda^{OP} : M \rightarrow  \{ O, P\}$
    and $\lambda^{QA} : M \rightarrow  \{ Q, A\}$.

    \begin{itemize}
    \item If $\lambda^{OP} (m) = O$, we call $m$ and O-move otherwise $m$ is a P-move.
    $\lambda^{QA} (m) = Q$ indicates that $m$ is a question otherwise $m$ is an answer.

    \item For any leaf $l$ of the tree $(M,\vdash)$, $\lambda^{QA} (l) = A$ and for any node
    $n \in (M,\vdash)$, $\lambda^{QA} (n) = Q$.
    \end{itemize}

\item The forest of tree $(M,\vdash)$ respect the following condition:
    \begin{itemize}
    \item[(e1)] The roots are O-moves: for any root $r$ of $(M,\vdash)$, $\lambda^{OP} (r) = O$.
    \item[(e2)] Answers are enabled by questions: $m \vdash n  \zand \lambda^{QA}(n) = A \imp \lambda^{QA}(m) = Q$.
    % Or more succinctly, if we write $\dashv$ the relation $\vdash^-1$: $\lambda^{QA} \left( \dashv( (\lambda^{QA})^{-1}(\{A\}) ) \right) = \{ O \}$
    \item[(e3)] A player move must be justified by a move played by the other player:
         $m\vdash n \imp \lambda^{OP}(m) \neq \lambda^{OP}(n)$.
    \end{itemize}
\end{itemize}
\end{dfn}

For commodity we write the set $\{O,P\} \times \{Q,A\}$ as $\{OQ,OA,PQ,PA\}$.
$\overline{\lambda}$ denotes the labeling function $\lambda$ with the question and answer swapped. For instance:
$$\overline{\lambda(m)} = OQ \iff \lambda(m) = PQ$$

The roots of the forest of tree $(M,\vdash)$ are the \emph{initial moves}.

For example, the simplest possible arena is written $\mathbf{1}$ and
denotes the arena which set of moves $M$ is empty.

\begin{exmp}[The flat arena]
\label{exmp:flatarena}

 Let $A$ be any countable set then the flat arena over $A$
is defined to be the arena $\langle M, \lambda, \vdash \rangle$ such
that $M$ has one move $q$ with $\lambda(q) = OQ$ and for each
element in $A$, there is a corresponding move $a_i$ in $M$ with
$\lambda(a_i) = PA$ for some $i \in \nat$. The enabling relation
$\vdash$ is defined to be $\{ q \vdash a_i \ | i \in \nat \}$.

This arena is represented by the following tree:
\begin{center}
  \pstree[levelsep=6ex]
    { \TR{$q$} }
    {    \TR{$a_1$} \TR{$a_2$} \TR{\ldots} }
\end{center}
The vertices represent the moves and the edges represent the
enabling relation.

The flat arena over $\nat$ and $\mathbb{B}$ is written
$\mathbf{int}$ and  $\mathbf{bool}$ respectively.

\end{exmp}

Once the arena has been defined, the bases of the game are set and the players have something to play with.
We now need to describe the state of the game, for that purpose
we introduced \emph{justified sequences of moves}. Sequence of moves are used to record the history of all the moves that have been
played.

\begin{dfn}[Justified sequence of moves]
A justified sequence is a sequence of moves $s$ together with an associated sequence of pointers. Any
move $m$ in the sequence that is not initial has as pointer that points to a previous move $n$ that justifies it (i.e. $n \vdash m$).
\end{dfn}

The pointers of a justified sequences are represented with arrows.
This is an example of justified sequence of moves:
$$\rnode{q4}{q}^4
\rnode{q3}{q}^3 \rnode{q2}{q}^2 \rnode{q3b}{q}^3 \rnode{q2b}{q}^2
\rnode{q1}{q}^1 \bkptrc{q3}{q4} \bkptrc{q2}{q3}
\bkptrc[ncurv=0.6]{q3b}{q4} \bkptrc{q2b}{q3b}$$

The first move of a justified sequence must be an O-move since
initial moves are all O-moves.

Notation: we write $s t$ or sometimes $s \cdot t$ do denote the
sequences obtain by concatenating $s$ and $t$. The empty sequence is
written $\epsilon$.

 A justified sequence has two particular subsequences which
will be of particular interest later on when we introduce
strategies. These subsequences are called the P-view and the O-view
of the sequence. The idea is that a view describes the local context
of the game. Here is the formal definition:

\begin{dfn}[View]
Given a justified sequence of moves $s$. We define the proponent view (P-view) noted $\pview{s}$ by induction:
\begin{align*}
\pview{\epsilon} &= \epsilon \\
\pview{s \cdot m} &= \pview{s} \cdot \ m && \mbox{ if $m$ is a P-move} \\
\pview{s \cdot m} &= m && \mbox{ if $m$ is initial (O-move) } \\
\pview{ s \cdot \rnode{m}{m} \cdot t \cdot \rnode{n}{n} \bkptra{50}{n}{m} } &=
 \pview{s} \cdot \rnode{mm}{m} \cdot \rnode{nn}{n} \bkptra{70}{nn}{mm} && \mbox{ if $n$ is a non initial O-move }
\end{align*}
The O-view $\oview{s}$ is defined similarly:
\begin{align*}
\oview{\epsilon} &= \epsilon \\
\oview{s \cdot m} &= \oview{s} \cdot \ m && \mbox{ if $m$ is a O-move} \\
\oview{ s \cdot \rnode{m}{m} \cdot t \cdot \rnode{n}{n} \bkptra{50}{n}{m} } &=
 \pview{s} \cdot \rnode{mm}{m} \cdot \rnode{nn}{n} \bkptra{70}{nn}{mm} && \mbox{ if $n$ is a P-move }
\end{align*}
\end{dfn}

In fact not all justified sequences will be of interest for the
games that we will use. We call \emph{legal position} any justified
sequence verifying two additional conditions: alternation and
visibility. Alternation says that players O and P plays
alternatively. Visibility expresses that each non-initial move is
justified by a move situated in the local context at that point.
Intuitively, the visibility condition gives some coherence to the
justification pointers of the sequence.

\begin{dfn}[Legal position]
A legal position is a justified sequence of move $s$ respecting the following constraint:
\begin{itemize}
\item Alternation: For any subsequence $m \cdot n$ of $s$, $\lambda^{OP}(m) \neq \lambda^{OP}(n)$.
\item Visibility: For any subsequence $t m$ of $s$ where $m$ is not initial, if $m$ is a P-move then $m$ points to a move in $\pview{s}$
and if $m$ is a O-move then $m$ points to a move in $\oview{s}$.
\end{itemize}

The set of legal position of an arena $A$ is noted $L_A$.
\end{dfn}

We say that a move $n$ is hereditarily justified by a move $m$ if there is a sequence of move
$m_1, \ldots, m_q$ such that:
$$ m \vdash m_1 \vdash m_2 \vdash \ldots m_q \vdash n$$
If a move has no justification pointer, we says that it is an
\emph{initial move} (in that case it must be a root of the forest
arena).

Suppose that $n$ is an occurrence of a move in the sequence $s$ then
$s \upharpoonright n$ denotes the subsequence of $s$ containing all the moves hereditarily justified by $n$.
Similarly, $s \upharpoonright I$ denotes the
subsequence of $s$ containing all the moves hereditarily justified by the moves in $I$.

\begin{dfn}[Game]
A game is a structure $\langle M, \lambda, \vdash, P \rangle$ such that
\begin{itemize}
\item $ \langle M, \lambda, \vdash \rangle$ is an arena.
\item $P$ is called the set of valid positions, it is:
    \begin{itemize}
    \item a non-empty prefix closed subset of the set of legal position
    \item closed by initial hereditary filtering: if $s$ is a valid position then for any set $I$ of occurrences of initial moves
    in $s$, $s\upharpoonright I$ is also a valid position.
    \end{itemize}
\end{itemize}
\end{dfn}

\begin{exmp}  Consider the flat arena  $\mathbf{int}$.
The set of valid position $P = \{ \epsilon, q \} \union \{ q \cdot
a_i \ | i \in \nat \}$ defines a game on the arena $\mathbf{int}$.
\end{exmp}

\subsection{Constructions on games}
\label{sec:gameconstruction}

We now define game constructors that will be useful later on.

Consider the two functions $f : A \rightarrow C$ and $g : B
\rightarrow C$, we write $[f,g]$ to denote the pairing of $f$ and
$g$ defined on the direct sum $A + B$. Given a game $A$ with a set
of moves $M_A$, we use the filtering operator $s \upharpoonright A$
do denote the subsequence of $s$ consisting of all moves in $M_A$.
Although this notation conflicts with the hereditarily filtering
operator, it should not cause any confusion.

\subsubsection{Tensor product}
Given two games $A$ and $B$ we define the tensor product constructor
$A \otimes B$ as follows:
\begin{eqnarray*}
  M_{A \otimes B} &=& M_A + M_B \\
  \lambda_{A\otimes B} &=& [\lambda_A,\lambda_B] \\
  \vdash_{A\otimes B} & = & \vdash_{A}\ \union\ \vdash_{B} \\
  P_{A\otimes B} & = & \{ s \in L_{A\otimes B} | s \upharpoonright A \in P_A \wedge s \ \upharpoonright B \in P_B  \}.
\end{eqnarray*}

In particular,  $n$ is initial in $A\otimes B$ if and only if $n$ is
initial in A or B. And $m \vdash_{A\otimes B} n$  holds if and only if $m
\vdash_{A} n$ or $m \vdash_{B} n$ holds.

\subsubsection{Function space}
The game $A \otimes B$ is defined as follows:
\begin{eqnarray*}
  M_{A \multimap B} &=& M_A + M_B \\
  \lambda_{A\multimap B} &=& [\overline{\lambda_A},\lambda_B] \\
  \vdash_{A\multimap B} & = & \vdash_{A}\ \union\ \vdash_{B}\ \union\  \{ (m,n) \ |\ m \mbox{ initial in } B \wedge n \mbox{ initial in } A \} \\
  P_{A\otimes B} & = & \{ s \in L_{A\otimes B} | s \upharpoonright A \in P_A \wedge s \ \upharpoonright B \in P_B  \}.
\end{eqnarray*}

\subsubsection{Cartesian product}
The game $A \& B$ is defined as follows:
\begin{eqnarray*}
  M_{A \& B} &=& M_A + M_B \\
  \lambda_{A\& B} &=& [\lambda_A,\lambda_B] \\
  \vdash_{A\& B} & = & \vdash_{A}\ \union\ \vdash_{B} \\
  P_{A\& B} & = & \{ s \in L_{A\otimes B} | s \upharpoonright A \in P_A \wedge s \ \upharpoonright B = \epsilon  \} \\
        &&   \union \{ s \in L_{A\otimes B} | s \upharpoonright A \in P_B \wedge s \ \upharpoonright A = \epsilon  \}.
\end{eqnarray*}

A play of the game $A \& B$ is either a play of $A$ or a play of $B$ whether a play
of the game $A \otimes B$ may be an interleaving of plays on $A$ and plays on $B$.

\subsection{Representation of plays}

Plays of the game are usually represented in a table diagram. The
columns of the table correspond to the different components of the
arena and each row corresponds to one move in the play. The first
row always represents an O-move, this is because O is the only
player who can open a game (since roots of the arena are O-moves).

As an example the play
$$\rnode{q1}{q}\
 \rnode{q2}{q}
 \ \rnode{a2}{8}
\  \rnode{a1}{12}
  \bkptrc{a1}{q1}
\bkptrc{a2}{q2} $$
on the
game $\textbf{int} \multimap \textbf{int} $ can be represented by
the following diagram:

\begin{center}
\begin{tabular}{cccc}
\textbf{int} & $\imp$ & \textbf{int} & \\
&& q & O\\
q  &&& P\\
8  &&& O\\
&& 12 & P
\end{tabular}
\end{center}

When it is necessary, the justification pointers of the play can also
be shown on the diagram.


\subsection{Strategy}

\subsubsection{Definition}

During a game, the player who has to play may have several choices
for his next move. The move that he makes is chosen according to a
given strategy.

A strategy is a rule telling the player which move to make when the
game is in a given position. More abstractly, a strategy is a
partial function mapping legal position where Proponent has to move
to P-moves.

\begin{dfn}[Strategy]
A strategy for player P on a given game $\langle M, \lambda, \vdash, P \rangle$ is a
non-empty set of even-length positions from $P$ such that:
\begin{enumerate}
\item (\emph{no unreachable position}) $sab \in \sigma \imp s \in \sigma$
\item (\emph{determinacy}) $sab, sac \in \sigma \quad \imp \quad  b = c$  and $b$ has the same justifier as
$c$.
\end{enumerate}
\end{dfn}

The idea is that the presence of the even-length sequence $s a b$ in
$\sigma$ tells the player P that whenever the game is in position
$s$ and player O plays the move $a$ then it must respond by playing
the move $b$.

The first condition ensures that the strategy $\sigma$ only
considers positions that the strategy itself could have led to in a
previous move. The second condition in the definition requires that
this choice of move is deterministic (i.e. there is a function $f$
from the set of odd length position to the set of moves $M$ such
that $f(s a) = b$).


For any game $A$, the smallest possible strategy is the strategy
that never respond given by $\{ \epsilon \}$. It is called the
\emph{empty strategy} and denoted $\bot$.

\subsubsection{Copy-cat strategy}

For any arena $A$ there is a strategy on the game $A \multimap A$
called the \emph{copy-cat strategy}. We write $A_1$ and $A_2$ to
denote the first and second copy of the arena $A$ in the game $A
\multimap A$. If $A$ is the arena $A_1$ then $A^\perp$ denotes the
arena $A_2$ and reciprocally.

Let $A$ be one of the arena $A_1$ or $A_2$. The copy-cat strategy
operates as follows: whenever P has to respond to an O-move played
in $A$, it replicates the move played by O in the arena $A^{\perp}$
after that $O$ has to respond in $A^{\perp}$ and $P$ replicates this
response in $(A^\perp)^\perp = A$ and so on and so forth.


More formally, the copy-cat strategy is defined by:
$$ \textsf{id}_A = \{ s \in P^{\textsf{even}}_{A \multimap A} \ | \ \forall t \sqsubseteq^{\textsf{even}} s\ .\ t \upharpoonright A_1 = t \upharpoonright A_2 \}$$
where $P^{\textsf{even}}_A$ denotes the valid position of even
length in the game $A$ and $t \sqsubseteq^{\textsf{even}} s$ denotes
that $t$ is an even length prefix of $s$.

The copy-cat strategy is also called \emph{identity strategy} since
it is the identity for strategy composition as we will see in the
next paragraph.

\begin{exmp} The copy-cat strategy on $\textbf{int}$ is:
$$\begin{array}{ccc}
\textbf{int} & \imp & \textbf{int} \\
&& q\\
q \\
n \\
&& n
\end{array}
$$
Note that we introduced this type of diagram to represent plays of
games but, as we can see here, the same diagrams can be used to
represent strategies when the play represented is general enough.

The copy-cat strategy on $\textbf{int} \typar \textbf{int}$ is given
by the following diagram:
$$\begin{array}{ccccccc}
(\textbf{int} & \imp & \textbf{int}) & \imp & (\textbf{int} & \imp & \textbf{int}) \\
&&&& && q\\
&& q\\
q \\
&&&& q \\
&&&& m \\
m\\
&& n \\
&&&& && n
\end{array}$$
\end{exmp}

\subsubsection{Composition}

It is well-known that any model of the simply typed lambda-calculus
is a cartesian closed category \citep{CroleRL:catt}. Games are used
to give a fully-abstract model of PCF, an extended simply typed
lambda calculus, therefore the game model should fit into a
cartesian closed category. This category will have games as objects
and strategies as morphisms. In a category, morphisms should be able
to compose together, therefore there should be an appropriate notion
of strategy composition.

Composition of strategies is an essential feature of game semantics.
As we will see in the following section, in the game model of PCF,
strategies represent programs. Therefore, strategy composition will
prove to be very useful : obtaining the model of a composed program
boils down to composing the strategies of the composing programs.

The way composition is defined for strategies is similar to
``parallel composition plus hiding'' in the trace semantics of CSP
\citep{hoare_csp}. Consider two strategies $\sigma : A \multimap B$
and $\tau : B \multimap C$ that we wish to compose.

For any sequence of moves $u$ on three arenas $A$, $B$, $C$, we call
projection of $s$ on the game $A \multimap B$ and we note $u
\upharpoonright A,B$ the subsequence of $s$ obtained by removing
from $u$ the moves in $C$ and pointers to moves in $C$. The
projection on $B \multimap C$ is defined similarly.

The definition of the projection on $A \multimap B$ differs
slightly: $u \upharpoonright A,C$ is the subsequence of $u$
consisting of the moves from $A$ and $C$ with some additional
pointers: we add a pointer from $a \in A$ to $c\in C$ whenever $a$
points to some move $b \in B$ itself pointing to $c$. All the
pointers to moves in $B$ are removed.


First we remark that for a given legal position $s$ in the game $A
\multimap C$, there is what is called an \emph{uncovering} of $s$.
The uncovering of $s$ is the maximal justified sequence of moves $u$
from the games $A$, $B$ and $C$ such that:
\begin{itemize}
\item The sequence $s$, considered as a pointer-less sequence, is a subsequence of
$u$;
\item the projection of $u$ on the game $A \multimap B$ lies in the
strategy $\sigma$;
\item the projection of $u$ on the game $B \multimap C$
lies in the strategy $\tau$;
\item and the projection of $u$ on the game $A \multimap C$ is a subsequence of $s$ (here the term ``subsequence'' refers to the sequence of nodes together with the auxiliary sequence of pointers).
\end{itemize}
This uncovering, noted $uncover(s, \sigma, \tau)$, is
defined uniquely for given strategies $\sigma$, $\tau$ and legal
position $s$ (this is proved in part II of \cite{hylandong_pcf}).

We define $\sigma \| \tau $ to be the set of uncovering of legal
positions in $A \multimap C$:
$$ \sigma \| \tau = \{ uncover(s, \sigma, \tau) \ | \ s \mbox{ is a legal position in } A \multimap C \}$$

The composition of $\sigma$, $\tau$ is defined to be the set of
projections of uncovering of legal positions in $A \multimap C$:

\begin{dfn}[Strategy composition]
Consider $\sigma : A \multimap B$ and  $\tau : B \multimap C$ two
strategies. We define $\sigma ; \tau$ to be:
$$ \sigma ; \tau = \{ u \upharpoonright A,C \ | \ u \in \sigma \|
\tau \}$$
\end{dfn}

It can be verified that composition is well-defined and associative
\citep{hylandong_pcf} and that the copy-cat strategy $\textsf{id}_A$ is the identity for composition.

\subsubsection{Constraint on strategies}

Different classes of strategies will be considered depending on the
features of the language that we want to model. Here is a list of
common restrictions that we will consider:
\begin{itemize}
\item \emph{Well-bracketing:} In a well-bracketed strategies the players always answer the last unanswered question (called the pending question) first.
If we represent Opponent's question as ``['', Proponent's answer as
``]'', Proponent's question as ``('' and Opponent's answers as ``)''
then requiring that the last pending question is answered first is
the same as requiring that the string representing the play is a
prefix of a well-bracketed sequence.

\item \emph{History-free strategies:} A strategy is history-free if the Proponent's move at any position of the game where he has to play
is determined by the last move of the Opponent. In other words, the
history prior to the last move is ignored by the Proponent when
deciding how to respond.

\item \emph{History-sensitive strategies:} The Proponent follows a history-sensitive strategy if he needs to have access to the full
history of the moves in order to decide which move to make.

\item \emph{Innocence:} a strategy is innocent if it determines Proponent's moves based on a restricted view of the history of the play, mainly the P-view
at that point. Such strategies can be specified by a partial
function mapping P-views to P-moves. However not every partial
function from P-views to P-moves gives rise to an innocent strategy
(a sufficient condition is given in \cite{hylandong_pcf}).
\end{itemize}

The formal definition of innocence follows:
\begin{dfn}[Innocence]
Given positions $sab, ta \in L_A$ where $sab$ has even length and
$\pview{sa} = \pview{ta}$, there is a unique extension of $ta$ by
the move $b$ together with a justification pointer such that
$\pview{sab} = \pview{sa}$. We write this extension
$\textsf{match}(sab,ta)$.

The strategy $\sigma:A$ is \emph{innocent} if and only if:
$$ \left(
     \begin{array}{c}
       \pview{sa} = \pview{ta} \\
       sab \in \sigma \\
       t\in \sigma \wedge ta \in P_A \\
     \end{array}
   \right)
\quad \imp\quad  \textsf{match}(sab,ta) \in \sigma$$

\end{dfn}


\subsection{Categorical interpretation of games}

In this section we recall some results about the categorical representation of Games.
These results with complete details and proofs can be found in \cite{McC96b,hylandong_pcf,abramsky94full}.
We refer the reader to \cite{CroleRL:catt} for more information about category theory.

We consider the category $\mathcal{G}$ whose objects are games and morphisms are
strategies. A morphism from $A$ to $B$ is a strategy on the game $A \multimap B$.

Three other sub-categories of $\mathcal{G}$ are considered: each of them correspond to some restriction on strategies:
$\mathcal{G}_i$ is the sub-category
of $\mathcal{G}$ whose morphisms are the innocent strategies,
$\mathcal{G}_b$ has only the well-bracketed strategies and $\mathcal{G}_{ib}$ has the innocent and well-bracketed strategies.

\begin{prop}
$\mathcal{G}$, $\mathcal{G}_i$, $\mathcal{G}_b$ and $\mathcal{G}_{ib}$ are categories.
\end{prop}

Proving this requires to prove that composition of strategies is well-defined, associative, has a unit (the copy-cat strategy), preserves innocence and
well-bracketedness. See \cite{hylandong_pcf,abramsky94full} for a proof.


\subsubsection{Monoidal structure}

We have already defined the tensor product on games in section \ref{sec:gameconstruction}.
We now define the corresponding transformation on morphisms:
given two strategies $\sigma : A \multimap B$ and $\tau : C \multimap D$ the strategy
$\sigma \otimes \tau : (A \otimes C) \multimap (B\otimes D)$ is defined by:
$$ \sigma \otimes \tau = \{ s \in L_{A \otimes C \multimap B\otimes D} \ s \upharpoonright A,B \in \sigma
\wedge s \upharpoonright C,D \in \tau \}$$

It can be shown that the tensor product is associative, commutative and has
$I = \langle \emptyset, \emptyset,\emptyset, \{ \epsilon \} \rangle $ as identity.
Hence the game categories $\mathcal{G}$ is a symmetric monoidal categories. Moreover
$\mathcal{G}_i$ and  $\mathcal{G}_b$ are sub-symmetric monoidal categories of $\mathcal{G}$,
and $\mathcal{G}_{ib}$ is a sub-symmetric monoidal category of $\mathcal{G}_i$, $\mathcal{G}_b$ and
$\mathcal{G}$.

\subsubsection{Closed structure}

For any game $A$, $B$ and $C$,
to any strategy $\sigma : A\otimes B \multimap C$, there is a corresponding strategy
$\tau : A\otimes B \multimap C$ obtained by relabeling the moves in $\sigma$. This transformation
is in fact an isomorphism: the hom-set $\mathcal{G}(A\otimes B, C)$ is isomorphic to the hom-set
$\mathcal{G}(A,B\multimap C)$. Hence $\mathcal{G}$ is an autonomous (i.e. symmetric monoidal closed) category.

$\mathcal{G}_i$ and  $\mathcal{G}_b$ are sub-autonomous categories of $\mathcal{G}$,
and $\mathcal{G}_{ib}$ is a sub-autonomous category of $\mathcal{G}_i$, $\mathcal{G}_b$ and
$\mathcal{G}$.

\subsubsection{Cartesian product}
The cartesian product defined in section \ref{sec:gameconstruction} is indeed a cartesian product in the category
$\mathcal{G}$, $\mathcal{G}_i$, $\mathcal{G}_b$ and $\mathcal{G}_{ib}$.

The projections $\pi_1:A \& B \rightarrow A$ and $\pi_1:A \& B \rightarrow B$ are given by the obvious copy-cat strategies.
Given two category morphisms $\sigma :C \rightarrow A$ and $\tau : C \rightarrow B$ the pairing function
$\langle \sigma, \tau \rangle : C \rightarrow A \& B$ is given by:
\begin{eqnarray*}
\langle \sigma, \tau \rangle &=& \{ s \in L_{C\multimap A\&B} \ | \ s \upharpoonright C,A \in \sigma \wedge s \upharpoonright B = \epsilon  \} \\
&\union& \{ s \in L_{C\multimap A\&B} \ | \ s \upharpoonright C,A \in \sigma \wedge s \upharpoonright B = \epsilon  \}
\end{eqnarray*}

\subsubsection{Cartesian closed structure}
Having defined the cartesian product is not enough to turn $\mathcal{G}$ into a cartesian closed category :
we also need to define a terminal object $I$ and the exponential construct $A \imp B$ for any two games $A$ and $B$.
In fact, this cannot be done in the current categories $\mathcal{G}$ and we have to move on to another category
of games noted $\mathcal{C}$ whose objects and morphisms are certain sub-classes of games and strategies.

Before introducing the category $\mathcal{C}$ we need some new definitions:


For any game $A$ we define the exponential game noted $!A$.
The game $!A$ corresponds to a repeated version of the game $A$. Plays of $!A$ are interleaving of plays of
$A$. It is defined as follows:
\begin{eqnarray*}
  M_{!A} &=& M_A \\
  \lambda_{!A} &=& \lambda_A \\
  \vdash_{!A} & = & \vdash_{A} \\
  P_{!A} & = & \{ s \in L_{!A} | \mbox{ for each initial move $m$, } s \upharpoonright m \in P_A \}
\end{eqnarray*}
The following equalities hold:
\begin{eqnarray*}
  !(A \& B) &=& !A \otimes !B\\
  I &=& !I
\end{eqnarray*}

\begin{dfn}[Well-opened games]
A game $A$ is well-opened if for any position $s \in P_A$ the only initial move is the first
one.
\end{dfn}

Well-opened games have single thread of dialog. Then can be turned into games with multiple-thread of dialog
using the promotion operator:

\begin{dfn}[Promotion]
Consider a well-opened game $B$.
Given a strategy on ${!A} \multimap B$, we define it promotion $\sigma^\dagger : {!A} \multimap {!B}$ to be the
strategy which plays several copies of $\sigma$. It is formally defined by:
$$ \sigma^\dagger = \{ s \in L_{{!A} \multimap !B} \ | \ \mbox{ for all initial $m$, } s \upharpoonright m \in \sigma  \}.$$
\end{dfn}

It can be shown that promotion is well-defined (it is indeed a strategy) and that it preserves innocence and
well-bracketedness.


We now introduce the category of well-opened games:
\begin{dfn}[Category of well-opened games]
The category $\mathcal{C}$ of well-opened games is defined as follow:
\begin{enumerate}
\item The objects are the well-opened games,
\item a morphism $\sigma : A \rightarrow B$ is a strategy for the game $!A \multimap B$,
\item the identity map for $A$ is the copy-cat strategy on $!A \multimap A$ (which is well-defined for well-opened games).
It is called dereliction, noted
$\textsf{der}_A$ and defined formally by:
$$ \textsf{der}_A = \{ s \in P^{\textsf{even}}_{{!A} \multimap A} \ | \ \forall t \sqsubseteq^{\textsf{even}} s \ . \ t \upharpoonright {!A} = t \upharpoonright A \},$$
\item composition of morphisms $\sigma : {!A} \multimap B$ and $\tau : {!B} \multimap C$ is defined to be
the strategy $\sigma^\dagger;\tau$ on the game ${!A} \multimap C$.
\end{enumerate}
\end{dfn}
$\mathcal{C}$ is a well-defined category and the three sub-categories
$\mathcal{C}_i$, $\mathcal{C}_b$, $\mathcal{C}_{ib}$ corresponding to sub-category
with innocent strategies, well-bracketed strategies and innocent and well-bracketed strategies respectively.


The category $\mathcal{C}$ has a terminal object $I$, for any two games $A$ and $B$ a product $A \& B$ and
an exponential $A \imp B$. Moreover the hom-sets $\mathcal{C}(A \& B,C)$ and
$\mathcal{C}(A,!B \multimap C)$ are isomorphic. Indeed:
\begin{eqnarray*}
\mathcal{C}(A\& B,C) &=& \mathcal{G}(!(A\& B),C) \\
&=& \mathcal{G}({!A}\otimes {!B}),C) \\
&\cong& \mathcal{G}({!A}, {!B} \multimap C) \qquad  \mbox{($\mathcal{G}$ is a closed monoidal category)}\\
&=& \mathcal{C}(A, {!B} \multimap C)
\end{eqnarray*}
Hence $\mathcal{C}$ is a cartesian closed category. Moreover $\mathcal{C}_i$ and $\mathcal{C}_b$
are sub-cartesian closed caterogies of $\mathcal{C}$ and $\mathcal{C}_{ib}$ is as sub-cartesian closed category
of each of $\mathcal{C}$, $\mathcal{C}_i$ and $\mathcal{C}_b$.



\subsubsection{Order enrichment}

Strategies can be ordered using the inclusion ordering.
The set of strategies on a given game $A$ is a pointed directed complete partial order under this ordering: the
least upper bounds is the union of two strategies and the least element is the empty strategy $\{ \epsilon \}$.

The category  $\mathcal{C}$ and  $\mathcal{G}$ are cpo-enriched.





directe It is possible to define an order on strategies


\subsection{Arena of order at most 2}
In this section, we consider a restricted class of arena and prove a
property on the games played on these arenas.

The height of the arena is the length of the longest sequence of moves
$m_1 \ldots m_h$ in $M$ such that $m_1 \vdash m_2 \vdash \ldots \vdash m_h$.

The order of an arena $\langle M, \lambda, \vdash \rangle$ is defined to be
$h-2$ where $h$ is the height of the forest of trees $(M, \vdash)$.


\begin{lem}[Pointers are superfluous up to order 2]
Let $A$ be the arena of order at most 2. Let $s$ be a justified sequence of moves in the arena $A$ satisfying
 alternation, visibility and well-bracketing then
the pointers of the sequence $s$ can be reconstructed uniquely.
\end{lem}



\begin{proof}
In the graphic representation of the arena, we display the sub-arena by decreasing order of sub-arena order.
It is safe to do so since in the definition of the forest of tree of an arena, the children nodes
are not ordered.

Let $A$ be an arena of order 2. We assume that $A$ has only one root. The arena $A$ has therefore the following shape:
\begin{center}
\
  \pstree[levelsep=6ex]
    { \TR{$q$} }
    {
\SubTree{$T_1$} \SubTree[linestyle=none]{$\ldots$} \SubTree{$T_n$}
    \TR{$a_1$} \TR{$a_2$} \TR{\ldots} }
\end{center}

where each triangle $T_i$ represents an arena of order 0 or 1.

We will see that the following proof can easily be adapted to take into account the general case of forest arenas (multiple roots).

We write $I_k$, for $k=0$ or $1$, the set of indices $i$ such that the arena $T_i$ has order $k$:
$$I_k = \{ i \in 1.. n\ |\ \order{T_i} = k \}$$

Here is a graphic representation of the arenas $T_i$ for $i \in I_0$ and $T_j$ for $j \in I_1$:
\begin{center}
\
  \pstree[levelsep=6ex]
    {\TR{$q^i$}}
    { \TR{$a_1^i$} \TR{$a_2^i$} \TR{\ldots} }
\hspace{2cm}
  \pstree[levelsep=6ex]
    { \TR{$p^j$} }
    {
      \pstree[levelsep=6ex]
        { \TR{$q^j$} }
        { \TR{$a_1^j$} \TR{$a_2^j$} \TR{\ldots} }
      \TR{$b_1^j$} \TR{$b_2^j$} \TR{\ldots}
    }
\end{center}



For any justified sequence of moves $u$, we write $?(u)$ for the
subsequence of $u$ consisting of the questions in the sequence $u$
that are still pending at the end of the sequence.

Let $L$ be the following language $L = \{\ p^i q^i\ | \ i \in I_1
\}$. We consider the following cases:

\begin{center}
\begin{tabular}{c|c|l|l}
Case & $\lambda_{OP}(m)$ & $?(u) \in$ & condition \\ \hline
0 & O & $\{ \epsilon \}$ \\
A & P & $q$ \\
B & O & $q \cdot L^* \cdot p^i$     & $i \in I_1$ \\
C & P & $q \cdot L^* \cdot p^i q^i$ & $i \in I_1$ \\
D & O & $q \cdot L^* \cdot q^i$      & $i \in I_0$ \\
\end{tabular}
\end{center}

We use the notation $\hat{s}$ to denote a legal and well-bracketed
\emph{justified} sequence of moves and $s$ to denote the same
sequence of moves with pointers removed.

Note that the well-bracketing condition already tells us how to
uniquely recover the pointers for P answer moves: a P-answers points
to the last pending question having the same tag. However for O
answers, we will see that the visibility condition already ensures
the unique recoverability of the pointer and that the
well-bracketing condition is not needed.


We prove by induction on the sequence of moves $u$ that $?(u)$
corresponds to either case 0, A, B, C or D and that the pointers in
$u$ can be recovered uniquely.

\textbf{Base cases:}

If $u$ is the empty sequence $\epsilon$ then there is no pointer to
recover and it corresponds to case 0.

If $u$ is a singleton then it must be the initial question $q$ and
there is not pointer to recover. This corresponds to case A.

\textbf{Step case:}

Consider a legal well-bracketed justified sequence $\hat{s}$ where
$s = u \cdot m$ and $m \in M_A$. The induction hypothesis tells us
that the pointers of $u$ can be recovered (and therefore the P-view
or O-view at that point can be computed) and that $u$ corresponds to
one of the cases 0,A,B,C or D.

We proceed by case analysis on $u$:

\begin{description}

\item[case 0] This case cannot happen because $?(u) = \epsilon$ ($u$ is a complete play) implies that there cannot be any further move $m$.

Indeed the visibility condition implies that $m$ must point to a
P-question in the O-view at that point. But since $u$ is a complete
play, the O-view is $\oview{\hat{u}} = q a$ which does not contain
any P-question. Hence the move $m$ cannot be justified and is not
valid.


\item[case A] $?(u) = q$ and the last move $m$ is played by P.
    There are several cases:
    \begin{itemize}
    \item $m$ is an answer $a_k$ (to the initial question
    $q$) for some $k$, then $m$ points to $q$:

    $\hat{s} = \justseq{ q & \ldots & m \pointto{ll}}$

    and $?(s) = \epsilon$ therefore $s$ correspond to the case 0 (complete play).

    \item $m = q^i$ where $q^i$ is an order 0 question ($i \in I_0$).
    Then $q^i$ points to the initial question $q$ and $s$ falls into category D.

    \item $m = p^i$, a first order question, then $p^i$ points to $q$,

    $?(s)= q p^i$ and it is O's turn after $s$ therefore $s$ falls into category B.

    \end{itemize}


\item[case B] $?(u) \in q \cdot L^* \cdot p^i$ where $i \in I_1$ and O plays the move $m$.

We now analyse the different possible O-moves:
\begin{itemize}
\item Suppose that O gives the (tagged) answer $b^j$ for some $j \in I_1$ then
the visibility condition constraints it to point to a question in
the O-view at that point.

We remark that the last move in $\hat{u}$ must be $p^i$. Indeed,
suppose that there is a move $x \in M_A$ such that $\hat{u} =
\justseq{q & \ldots & p^i\ x \pointto{ll}}$ then by visibility, the
O-move $x$ should points to a move in the O-view a that point. The
O-view is $q p^i$, therefore $x$ can only points to $p^i$. But then,
$p^i$ is not a pending question in $s$ which is a contradiction.


Therefore $\oview{\hat{u}} = \oview{ \justseq{ q & \ldots & p^i
\pointto{ll}} } = q p^i$.

Hence $b^j$ can only point to $p^i$ (and therefore $i=j$).

We then have $?(s) = ?(u \cdot b^i) \in  q \cdot L^*$ which is
covered by case A and C.

\item The only other possible O-move is $q^i$ which, again by the visibility condition, points necessarily
to the previous move $p^i$. We then have $?(s) = ?(u \cdot q^i) \in
q \cdot L^* \cdot p^i q^i$. This falls into category C.

\end{itemize}

\item[case C] $?(u) \in q \cdot L^* \cdot p^i q^i$ where $i \in I_1$ and the move $m$ is played by $P$.

Suppose $m$ is an answer, then the well-bracketing condition imposes
to answer to $q^i$ first. The move $m$ is therefore an integer $a^i$
pointing to $q^i$. We then have $?(s) = ?(u \cdot a^i) \in  q \cdot
L^* \cdot p^i$. This correspond to case B.


Suppose $m$ is a question then there are two cases:
\begin{itemize}
\item $m = q^j$ with $j \in I_0$, the pointer goes to the initial question $q$ and $s$ falls into category D.
\item $m = p^j$ with $j \in I_1$, the pointer goes to the initial question $q$ and $s$ falls into category B.
\end{itemize}

\item[case D] $?(u) \in q \cdot L^* \cdot q^i$ where $i \in I_0$ and the move $m$ is played by $O$.

    The same argument as in case B holds. However there is now another possible move:
    the answer $m = a^i_k$ for some $k$.  This moves can only points to
    $q^i$ (this is the only pending question tagged by $i \in I_0$).

    Then $?(\hat{s}) = ?(\hat{u}\cdot a^i_k) = ?(\justseq{ q & \ldots & q^i \pointto{ll} & \ldots & a^i_k \pointto{ll}}) \in q \cdot L^* $ therefore $s$ falls either into category A or C.

\end{description}

This completes the induction.

How to generalize the proof to arenas that have multiple roots
(forest arenas)? In fact there is no ambiguity since all the moves
are implicitly tagged according to the arena that they belong to.
Therefore in the induction, it suffices to ignore the moves that
belong to another tree (as if they were part of a different game
played in parallel).


\end{proof}


\subsection{Pointer-less strategies}
\label{subsec:ptrless_strat}

Up to order 2, the semantics of PCF terms is entirely defined by
pointer-less strategies. In other words, the pointers can be
uniquely reconstructed from any non justified sequence of moves
satisfying the visibility and well-bracketing condition.

At level 3 however, pointers cannot be omitted in general. Here is an example
taken from \cite{abramsky:game-semantics} illustrating this. Consider the
following two terms of type $((\nat \typar \nat) \typar \nat) \typar
\nat$:

$$M_1 = \lambda f . f (\lambda x . f (\lambda y .y ))$$
$$M_2 = \lambda f . f (\lambda x . f (\lambda y .x ))$$

We assign tags to the types in order to identify in which arena the
questions are asked: $((\nat^1 \typar \nat^2) \typar \nat^3) \typar
\nat^4$. Consider now the following pointer-less sequence of moves
$s = q^4 q^3 q^2 q^3 q^2 q^1$. It is possible to retrieve the
pointers of the first five moves but there is an ambiguity for the
last move: does it point to the first or second occurrence of $q^2$
in the sequence $s$?

Note that the visibility condition does not eliminate the ambiguity,
since the two occurrences of $q^2$ both appear in the P-view at that
point (after recovering the pointers of $s$ up to the second last
move we get:
$$s = \rnode{q4}{q}^4
\rnode{q3}{q}^3
\rnode{q2}{q}^2
\rnode{q3b}{q}^3
\rnode{q2b}{q}^2
\rnode{q1}{q}^1
\bkptrc{q3}{q4}
\bkptrc{q2}{q3}
\bkptrc[ncurv=0.6]{q3b}{q4}
\bkptrc{q2b}{q3b}$$

 therefore the P-view of $s$ is $s$ itself.)

In fact these two different possibilities correspond to two
different strategies. Suppose that the link goes to the first
occurrence of $q^2$ then it means that the proponent is requesting
the value of the variable $x$ bound in the subterm $\lambda x . f (
\lambda y. ... )$. If P needs to know the value of $x$, this is
because P is in fact following the strategy of the subterm $\lambda
y . x$. And the entire play is part of the strategy $\sem{M_2}$.

Similarly, if the link points to the second occurrence of $q^2$ then
the play belongs to the strategy $\sem{M_1}$.

\section{Game model for PCF}
\subsection{Syntax of the PCF language}
PCF is a simply-type $\lambda$-calculus with the following
additions: integer constants  (of ground type), first-order
arithmetic operators, if-then-else branching, and the recursion
combinator $Y_A : (A\rightarrow A)\rightarrow A$ for any type $A$.

The types of PCF are given by the following grammar:
$$ T ::= \texttt{exp}\ |\ T \rightarrow T$$

The following grammar gives the structure of terms:
\begin{eqnarray*}
 M ::= x\ |\ \lambda x :A . M \ |\ M M \ |\ \\
\ |\ n \ |\ \texttt{succ } M \ |\  \texttt{pred } M \\
\ |\ \texttt{cond } M M M \ |\ \texttt{Y}_A\ M
\end{eqnarray*}

where $x$ ranges over a set of countably many variables and $n$
ranges over the set of natural numbers.

Terms are generated according to the formation rules given in table
\ref{tab:pcf_formrules} where the judgement is of the form $ \Gamma  \vdash M : A$.

\begin{table}[htbp]
$$ (var) \rulef{}{x_1:A_1, x_2:A_2, \ldots x_n : A_n  \vdash x_i : A_i}\ i \in 1..n$$
$$ (app) \rulef{\Gamma \vdash M : A\rightarrow B \qquad \Gamma \vdash N:A}{\Gamma \vdash M\ N : B}
\qquad (abs) \rulef{\Gamma, x:A \vdash M : B}{\Gamma \vdash \lambda x :A . M : A\rightarrow B}$$

$$ (const) \rulef{}{\Gamma \vdash n :\texttt{exp}}
\qquad (succ) \rulef{\Gamma \vdash M:\texttt{exp} }{\Gamma \vdash \texttt{succ}\ M:\texttt{exp}}
\qquad (pred) \rulef{\Gamma \vdash M:\texttt{exp} }{\Gamma \vdash \texttt{pred}\ M:\texttt{exp}}$$

$$
(cond) \rulef{\Gamma \vdash M : exp \qquad \Gamma \vdash N_1 : exp \qquad \Gamma \vdash N_2 : exp }{\Gamma \vdash \texttt{cond}\ M\ N_1\ N_2}
\qquad  (rec) \rulef{\Gamma \vdash M : A\rightarrow A }{ \Gamma \vdash Y_A M : A}$$

\caption{Formation rules for PCF terms}
\label{tab:pcf_formrules}
\end{table}

\subsection{Operational semantics of PCF}

We give the big-step operational semantics of PCF. The notation $M \eval V$ means
that the closed term $M$ evaluates to the canonical form $V$. The canonical forms are given by the following
grammar:
$$V ::= n\ |\ \lambda x. M$$
In other word, a canonical form is either a number or a function.

The operational semantics is given for closed terms therefore the context $\Gamma$ is not present in
the evaluation rules.

The full operational semantics is given in table \ref{tab:bigstep_pcf}.

\begin{table}[htbp]
$$\rulef{}{V \eval V} \quad \mbox{ provided that $V$ is in canonical form.} $$

$$ \rulef{M \eval \lambda x. M' \quad M'\subst{x}{N}}{M N \eval V}$$

$$\rulef{M \eval n}{\texttt{succ}\ M \eval n+1}
\qquad \rulef{M \eval n+1}{\texttt{pred}\ M \eval n}
\qquad \rulef{M \eval 0}{\texttt{pred}\ M \eval 0}$$

$$\rulef{M \eval 0 \quad N_1 \eval V}{\texttt{cond}\ M N_1 N_2  \eval V}
\qquad
 \rulef{M \eval n+1 \quad N_2 \eval V}{\texttt{cond}\ M N_1 N_2  \eval V}$$

$$\rulef{M (\mathrm{Y} M) \eval V }{\texttt{Y} M \eval V}$$
\label{tab:bigstep_pcf}
\caption{Big-step operational semantics of PCF}
\end{table}



\section{Idealized Algol (IA)}
\label{sec:ia}

\subsection{The syntax of IA}
IA is an extension of PCF introduced by J.C. Reynold in
\cite{Reynolds81}. It adds imperative features such as local variables and sequential composition.

The description of the language that we give here follows the one of \cite{abramsky:game-semantics}.

On top of \texttt{exp}, PCF has the following two new types:
 \texttt{com} for commands and \texttt{var} for variables.

There is a constant \texttt{skip} of type \texttt{com} which corresponds to the command that do
nothing. Commands can be composed using the sequential composition operator \texttt{seq}.
Local variable are declared using the \texttt{new} operator, variable content is written
using \texttt{assign} and retrieved using \texttt{deref}.

The new formations rules are given in table \ref{tab:ia_formrules}.

\begin{table}[htbp]
$$ \rulef{\Gamma \vdash M : \texttt{com} \quad \Gamma \vdash N :A}
    {\Gamma \vdash \texttt{seq}_A \ M\ N\ : A} \quad A \in \{ \texttt{com}, \texttt{exp}\}$$

$$ \rulef{\Gamma \vdash M : \texttt{var} \quad \Gamma \vdash N : \texttt{exp}}
    {\Gamma \vdash \texttt{assign}\ M\ N\ : \texttt{com}}
\qquad
 \rulef{\Gamma \vdash M : \texttt{var}}
    {\Gamma \vdash \texttt{deref}\ M\ : \texttt{exp}}$$

$$ \rulef{\Gamma, x : \texttt{var} \vdash M : A}
    {\Gamma \vdash \texttt{new } x \texttt{ in } M} \quad A \in \{ \texttt{com}, \texttt{exp}\}$$

$$ \rulef{\Gamma \vdash M_1 : \texttt{exp} \rightarrow \texttt{com} \quad \Gamma \vdash M_2 : \texttt{exp}}
    {\Gamma \vdash \texttt{mkvar } M_1\ M_2\ : \texttt{var}}$$

\caption{Formation rules for IA terms}
\label{tab:ia_formrules}
\end{table}

If $\vdash M : A$ (i.e. $M$ can be formed with an empty context), we say that $M$ is a close term.

\subsection{Operational semantics}

In IA the semantics is given in a slightly different form from PCF.
In PCF, the evaluation rules were given for closed terms only. Suppose that we
proceed the same way for IA and consider the evaluation rule for the $\texttt{new}$ construct:
the conclusion is $\texttt{new } x:=0 \texttt{ in } M$ and the premise
is an evaluation for a certain term constructed from $M$, more precisely the term $M$
where \emph{some} occurrences of $x$ are replaced by the value $0$.
Because of the presence of the \texttt{assign} operator, we cannot simply replace all
the occurrences of $x$ in $M$ (the required substitution is  more complicated
than the substitution used for beta-reduction).


Therefore, instead of giving the semantics for closed term we consider terms
whose free variables are all of type \texttt{var}. These free variables are ``closed'' by mean of
stores. A store is a function mapping free variables of type \texttt{var} to natural numbers.
Suppose $\Gamma$ is a context containing only variable of type \texttt{var}, then we say that
$\Gamma$ is a \texttt{var}-context. A store whose domain $\Gamma$ is called a $\Gamma$-store.

The notation $s\ |\ x \mapsto n$ refers to the store that maps $x$ to $n$
and otherwise maps variables according to the store $s$.


The canonical forms for IA are given by the grammar:
$$ V ::= n\ |\ \lambda x. M\ |\ x\ |\  \texttt{mkvar} M N$$

where $n \in \nat$ and $x:var$.


A program is now defined by a term together with a $\Gamma$-store such that $\Gamma \vdash M : A$.
The evaluation semantics is expressed by the judgment form
$$s,M \eval s', V$$
where $s$ and $s'$ are $\Gamma$-stores,
$\Gamma \vdash M : A$ and $\Gamma \vdash V : A$ where $V$ is in canonical form.

The operational semantics for IA is given by the rule of PCF (table \ref{tab:bigstep_pcf})
together with the rules of table \ref{tab:bigstep_ia} where the following abbreviation is used:
$$ \rulef{M_1 \eval V_1 \quad M_2 \eval V_2}{M \eval V} \qquad \mbox{for} \qquad
  \rulef{s,M_1 \eval s',V_1 \quad s', M_2 \eval s'',V_2 }{s,M \eval s'',V}
$$


\begin{table}[htbp]
$$\mbox{\textbf{Sequencing }}
    \rulef{M \eval \iaskip \quad N \eval V}{\texttt{seq } M\ N \eval V}
$$

$$\mbox{\textbf{Variables }}
    \rulef{s,N \eval s',n \quad s',M \eval s'',x}{s, \assign\ M\ N \eval (s''\ |\ x \mapsto n),\iaskip}
\qquad
    \rulef{s,M \eval s',x }{s, \deref\ M \eval s',s'(x)}$$

$$\mbox{\texttt{\textbf{mkvar}}}
    \rulef{N \eval n \quad M \eval \texttt{mkvar } M_1\ M_2 \quad M_1\ n \eval \iaskip}
    {\assign\ M\ N \eval \iaskip}
\qquad
    \rulef{N \eval \texttt{mkvar } M_1\ M_2 \quad M_2\ \eval n}
    {\deref\ M \eval n}
$$

$$\mbox{\textbf{Block}}
    \rulef{(s\ |\ x \mapsto 0),M \eval (s'\ |\ x \mapsto n),V }
    {s, \texttt{new } x \texttt{ in } M \eval s',V}
$$

\label{tab:bigstep_ia}
\caption{Big-step operational semantics of IA}
\end{table}

\subsection{Game semantics}

As we have seen in section \ref{sec:catgames}, games and strategies
form a cartesian closed category, therefore games can model the simply-typed $\lambda$-calculus. Let us first
explain how this is achieved before extending the model to PCF and IA.

\subsubsection{Simply typed $\lambda$-calculus}

In the cartesian closed category $\mathcal{C}$, the objects are the arenas and the morphisms are the strategies.

In the games that we describe here, the Opponent represents the environment while
the Proponent plays according to a strategy imposed by the program itself.


Given a simple type $A$, we will model it as an arena $\sem{A}$.
A context $\Gamma = x_1 :A_1, \ldots x_n:A_n$ will be mapped to the arena
$\sem{\Gamma} = \sem{A_1} \times \ldots \times \sem{A_n}$ and a term $\Gamma \vdash M : A$
will be modeled by a strategy on the arena $\sem{\Gamma} \rightarrow \sem{A}$.
Since $\mathcal{C}$ is cartesian closed, there is is a terminal object $\textbf{1}$ (the empty arena) that
models the empty context ($\sem{\Gamma} = \textbf{1}$).


The base type \texttt{exp} is interpreted by the following flat arena of natural numbers noted $\nat$:
$$  \pstree[levelsep=6ex]
    {\TR[name=R]{q}}
    { \TR{1} \TR{2} \TR{\ldots}
    }
$$
In this arena, there is only one question: the initial O-question, P can then answer it by playing a natural number $i \in \nat$.
There are only two kinds strategy on this arena:
\begin{itemize}
\item the empty strategy where P never answer the initial question. This corresponds to a non terminating computation;
\item the strategies where P answers by playing a number $n$. This models the constants of the language.
\end{itemize}

Given the interpretation of base types, we define the interpretation of $A\rightarrow B$ by induction:
$$\sem{A \rightarrow B} = \sem{A} \Rightarrow \sem{B}$$

where the operator $\Rightarrow$ denotes the arena construction $!A
\multimap B$ which exists because $\mathcal{C}$ is cartesian closed.

Graphically if we represent the arena $A$ and $B$ by two triangles, the arena for $A \rightarrow B$ would be represented by:
\begin{center}
\psset{xunit=.5pt,yunit=.5pt,runit=.5pt}
\begin{pspicture}(150,80)
\rput[tr](150,80){ \pnode(27,40){a} \pstribox{A} }
\rput[bl](0,0){ \pnode(27,40){b} \pstribox{B} }
\ncline{->}{a}{b}
\end{pspicture}
\end{center}


Variables are interpreted by projection:
$$\sem{x_1 : A_1, \ldots, x_n:A_n \vdash x_i : A_i} = \pi_i : \sem{A_i} \times \ldots \times \sem{A_i} \times \ldots \times \sem{A_n} \rightarrow  \sem{A_i}$$

The abstraction $\Gamma \vdash \lambda x :A.M : A \rightarrow B$ is modeled by a strategy on the arena
$\sem{\Gamma} \rightarrow (\sem{A}\Rightarrow\sem{B})$. This strategy is obtain by using the currying operator of the
cartesian closed category:
$$\sem{\Gamma \vdash \lambda x :A.M : A \rightarrow B} = \Lambda( \sem{\Gamma, x :A \vdash M : B})$$

The application $\Gamma \vdash M N$ is modeled using the evaluation map $ev_{A,B} : (A\Rightarrow B)\times A \rightarrow B$:

$$\sem{\Gamma \vdash M N} = \langle \sem{\Gamma \vdash M, \Gamma \vdash N} \rangle; ev_{A,B}$$


\subsubsection{PCF}

We now show how to model the PCF constructs in the game semantics setting.
In the following, the sub-arena of a game are tagged in order to distinguish identical arenas that are present in different components of the game.
Moves are also tagged in the exponent in order to identify the sub-arena in which moves are played. We will omit the pointers in the play
when they are not essential for the understanding of the model (moreover we will see later on that under certain assumptions
up to order 2, pointers can be recovered uniquely).

The successor arithmetic operator is modeled by the following strategy on the arena $\nat^1 \Rightarrow \nat^0$:
$$\sem{\texttt{succ}} = \{q^0 \cdot q^1 \cdot n^1 \cdot (n+1)^0\ |\ n \in \nat \}$$

The predecessor arithmetic operator is denoted by the strategy
$$\sem{\texttt{pred}} = \{q^0 \cdot q^1 \cdot n^1 \cdot (n-1)^0\ |\ n >0 \} \union \{ q^0 \cdot q^1 \cdot 0^1 \cdot 0^0 \} $$

Then given a term $\Gamma \vdash \texttt{succ} M : \texttt{exp}$ we define:
$$\sem{\Gamma \vdash \texttt{succ } M : \texttt{exp}} = \sem{\Gamma \vdash M} ; \sem{\texttt{succ}} $$
$$\sem{\Gamma \vdash \texttt{pred } M : \texttt{exp}} = \sem{\Gamma \vdash M} ; \sem{\texttt{pred}} $$


The conditional operator is denoted by the following strategy on the arena $\nat^3 \times \nat^2 \times \nat ^1 \Rightarrow \nat^0$:
$$\sem{\texttt{cond}} =
    \{ q^0 \cdot q^3 \cdot 0 \cdot q^2 \cdot n^2 \cdot n^0 \ | \ n \in \nat \}
    \union
    \{ q^0 \cdot q^3 \cdot m \cdot q^2 \cdot n^2 \cdot n^0 \ | \ m >0, n \in \nat \}
    $$


Given a term $\Gamma \vdash \texttt{cond} M\ N_1\ N_2$ we define:
$$\sem{\Gamma \vdash \texttt{cond} M\ N_1\ N_2} =
\langle \sem{\Gamma \vdash M}, \sem{\Gamma \vdash N_1}, \sem{\Gamma \vdash N_2} \rangle ; \sem{\texttt{cond}}$$


The interpretation of the \texttt{Y} combinator is a bit more complicated.

Consider the term $\Gamma \vdash M : A \rightarrow A$, its semantics $f$ is a strategy on $\sem{\Gamma} \times \sem{A} \rightarrow \sem{A}$.
We define the chain $g_n$ of strategies on the arena $\sem{\Gamma} \rightarrow \sem{A}$ as follows:
\begin{eqnarray*}
g_0 &=& \perp \\
g_{n+1} &=&  F(g_n) = \langle id_{\sem{\Gamma}}, g_n\rangle ; f
\end{eqnarray*}

where $\perp$ denotes the empty strategy $\{ \epsilon \}$.

It is easy to see that indeed the $g_n$ forms a chain.
We define $\sem{\texttt{Y } M}$ to be the least upper bound of the chain $g_n$
(i.e. the  least fixed point of $F$). Its existence is guaranteed by the fact that
the category of games is cpo-enriched.

\subsubsection{IA}

It is easy to check that all the strategies given until now are well-bracketed and innocent.
From now on, we will only require well-bracketing and we will introduce strategies that are
not innocent. This is a necessity if we want to give a model of memory cells that correspond
to variables. The intuition behind this fact is that a cell needs to remember what was the last value written in it
in order to be able to return it when it is read, and this can only be done by looking at the whole history of moves,
not only those present in the P-view.





\subsection{Full-abstraction}
In this section we recall the standard full abstraction result proved in  \cite{abramsky94full}
and \cite{hylandong_pcf}.

A context noted $C[-]$ is a term containing a hole denoted by $-$. If $C[-]$ is a context then $C[A]$ denotes the term obtained
after replacing the hole by the term $A$.

\begin{dfn}[Observational preorder]
Let $\vdash M : A$ and $\vdash N : A$ be two closed terms. We define the relation $\sqsubseteq$ as follows:


$M \sqsubseteq N$ if and only if for all context $C[-]$ such that $C[M]$ and $C[M]$ are well-formed terms if
$C[M] \eval$ then $C[N] \eval$.
\end{dfn}


\begin{lem}[Soundness for PCF terms] Let $M$ be a PCF term.
If $M \eval V$ then $\sem{M} = \sem{V}$.
\end{lem}

\begin{lem}[Soundness for IA terms] Let $\Gamma \vdash M : A$ be an IA term and a $\Gamma$ store $s$.
If $s,M \eval s',V$ then the plays of $\sem{s,M} : I \multimap A \otimes !\Gamma$ which begin
with a move of $A$ are identical to those of $\sem{s',V}$.
\end{lem}


\begin{lem}[Computational adequacy for PCF terms]
All PCF terms are computable. (i.e. $\sem{M} \neq \perp$ implies $M \eval$)
\end{lem}

\begin{lem}[Computational adequacy for IA terms]
All IA terms are computable. (i.e. $\sem{M} \neq \perp$ implies $M \eval$)
\end{lem}


The following result follows from soundness and computational adequacy of the model.
\begin{prop}[Inequational soundness]
\label{prop:ineqsoundness}
Let $M$ and $N$ be two closed terms then
$$\sem{M} \subseteq \sem{N} \implies  M \sqsubseteq N $$
\end{prop}

\begin{prop}[Definability]
\label{prop:definability}
Let $\sigma$ be a compact well-bracketed on a game $A$ denoting a IA type. Then there is
an IA-term $M$ such that $\sem{M} = \sigma$.
\end{prop}

The final standard result of game semantics can then be proved using proposition \ref{prop:ineqsoundness} and \ref{prop:definability}:
\begin{thm}[Full abstraction]
Let $M$ and $N$ be two closed IA-terms.
$$\sem{M} \precsim_b \sem{N} \ \iff \ M \sqsubseteq N$$
\end{thm}

where $\precsim_b$ denotes the intrinsic preorder of the category $\mathcal{C}_b$.


\subsection{Call-by-Value first-order Idealized Algol}

Game semantics for call-by-value programming Language.




\part{Contribution}


In the second chapter we present the \emph{safe $\lambda$-calculus}.
Originally, \emph{safety} has been introduced as a syntactical
restriction on higher-order grammars in order to show a decidability
result about MSO theory of infinite trees \citep{KNU02}. In
\cite{safety-mirlong2004}, Aehlig, de Miranda and Ong  proposed an
adaptation of the safety restriction to the $\lambda$-calculus. This
restriction gives rise to the safe $\lambda$-calculus. We first
present this calculus and then give a more general definition which
does not make any assumption on the types of the terms.

In the third chapter, following ideas described in
\cite{OngLics2006}, we introduce the notions of computation tree of
a simply-typed term and traversal over a computation tree. We prove
a theorem showing a correspondence between traversals of the
computation tree and the game semantics of a term. Based on that
correspondence, we give a characterisation of the game semantics of
safe terms by a property called ``P-incremental-justification''. In
P-incrementally-justified strategies, P-pointers are superfluous (i.e.
they can be recovered uniquely from the underlying sequence of
moves and from O-moves' pointers). This simplification of the game semantics suggests some potential applications in algorithmic game semantics. We finish the
chapter by extending the result to safe \pcf\ and by giving the key
elements for an extension to full Safe Idealized Algol.




\chapter{Safe Higher Order Functional Languages}
\label{chap:safelambda}
%    \section{Safe Lambda Calculus}
%        \subsection{Definition and properties}
%        \subsection{Expressivity of the calculus}
%        Result a la Schwichtenberg \cite{citeulike:622637}
%        Statman's result for Safe Lambda calculus?
    \section{Introduction}

\subsection*{Background}

The \emph{safety condition} was introduced by Knapik, Niwi{\'n}ski and
Urzyczyn at FoSSaCS 2002 \cite{KNU02} in a seminal study of the
algorithmics of infinite trees generated by higher-order grammars. The
idea, however, goes back some twenty years to Damm \cite{Dam82} who
introduced an essentially equivalent\footnote{See de Miranda's
 thesis \cite{demirandathesis} for a proof.} syntactic
restriction (for generators of word languages) in the form of
\emph{derived types}.
% Level-$n$ tree grammars as defined by Damm correspond exactly to a
% subset of safe level-$n$ grammars -- namely the safe complete grammars
% -- and every safe grammar corresponds to a safe complete one.
A higher-order grammar (that is assumed to be \emph{homogeneously
  typed}) is said to be \emph{safe} if it obeys certain syntactic
conditions that constrain the occurrences of variables in the
production (or rewrite) rules according to their type-theoretic
order. Though the formal definition of safety is somewhat intricate,
the condition itself is manifestly important. As we survey in the
following, higher-order \emph{safe} grammars capture fundamental
structures in computation, offer clear algorithmic advantages, and
lend themselves to a number of compelling characterizations:

\begin{itemize}
\item \emph{Word languages}. Damm and Goerdt \cite{DG86} have shown
  that the word languages generated by order-$n$ \emph{safe} grammars
  form an infinite hierarchy as $n$ varies over the natural numbers.
  The hierarchy gives an attractive classification of the
  semi-decidable languages: Levels 0, 1 and 2 of the hierarchy are
  respectively the regular, context-free, and indexed languages (in
  the sense of Aho \cite{Aho68}), although little is known about
  higher orders.

  Remarkably, for generating word languages, order-$n$ \emph{safe}
  grammars are equivalent to order-$n$ pushdown automata \cite{DG86},
  which are in turn equivalent to order-$n$ indexed grammars
  \cite{Mas74,Mas76}.

\item \emph{Trees}. Knapik \emph{et al.} have shown that the Monadic
  Second Order (MSO) theories of trees generated by \emph{safe}
  (deterministic) grammars of every finite order are
  decidable\footnote{It has recently been shown
    \cite{OngLics2006} that trees generated by \emph{unsafe}
    deterministic grammars (of every finite order) also have decidable
    MSO theories. More precisely, the MSO theory of trees generated by order-$n$
recursion schemes is $n$-EXPTIME complete.}.

  They have also generalized the equi-expressivity result due to Damm
  and Goerdt \cite{DG86} to an equivalence result with respect to
  generating trees: A ranked tree is generated by an order-$n$ \emph{safe}
  grammar if and only if it is generated by an order-$n$ pushdown
  automaton.

\item \emph{Graphs}. Caucal \cite{Cau02} has shown that the MSO
  theories of graphs generated\footnote{These are precisely the
    configuration graphs of higher-order pushdown systems.} by
  \emph{safe} grammars of every finite order are decidable. In a recent preprint \cite{hague-sto07}, however,
  Hague \emph{et al.} have
  shown that the MSO theories of graphs generated by order-$n$
  \emph{unsafe} grammars are undecidable, but deciding their modal
  mu-calculus theories is $n$-EXPTIME complete.
\end{itemize}

\subsection*{Overview}

In this paper, we aim to understand the safety condition in the
setting of the lambda calculus. Our first task is to transpose it to
the lambda calculus and pin it down as an appropriate sub-system of
the simply-typed theory. A first version of the \emph{safe lambda
  calculus} has appeared in an unpublished technical report
\cite{safety-mirlong2004}. Here we propose a more general and cleaner
version where terms are no longer required to be homogeneously typed
(see Section~\ref{sec:safe} for a definition). The formation rules of
the calculus are designed to maintain a simple invariant: Variables
that occur free in a safe $\lambda$-term have orders no smaller than
that of the term itself.  We can now explain the sense in which the
safe lambda calculus is safe by establishing its salient property: No
variable capture can ever occur when substituting a safe term into
another. In other words, in the safe lambda calculus, it is
\emph{safe} to use capture-\emph{permitting} substitution when
performing $\beta$-reduction.


There is no need for new names when computing $\beta$-reductions of
safe $\lambda$-terms, because one can safely ``reuse'' variable names
in the input term. Safe lambda calculus is thus cheaper to compute in
this na\"ive sense. Intuitively one would expect the safety constraint
to lower the expressivity of the simply-typed lambda calculus. Our
next contribution is to give a precise measure of the expressivity
deficit of the safe lambda calculus. An old result of Schwichtenberg
\cite{citeulike:622637} says that the numeric functions representable
in the simply-typed lambda calculus are exactly the multivariate
polynomials \emph{extended with the conditional function}.  In the
same vein, we show that the numeric functions representable in the
safe lambda calculus are exactly the multivariate polynomials.

Our last contribution is to give a game-semantic account of the safe
lambda calculus.
% Not much is known about the safe $\lambda$-calculus, and many problems
% remain to be studied concerning its computational power, the
% complexity classes that it characterizes, its interpretation under the
% Curry-Howard isomorphism and its game-semantic characterization. This
% paper is a contribution to the last problem.
%
% The difficulty in giving a game-semantic account of safety lies in the
% fact that it is a syntactic restriction whereas game semantics is
% syntax-independent. The solution consists in finding a particular
% syntactic representation of terms on which the plays of the game
% denotation can be represented.  To achieve this, we use ideas recently
% introduced by the second author \cite{OngLics2006}: a term is
% canonically represented by a certain abstract syntax tree of its
% $\eta$-long normal form referred as the \emph{computation tree}. This
% abstract syntax tree is specially designed to establish a
% correspondence with the game arena of the term. A computation is
% described by a justified sequence of nodes of the computation tree
% respecting some formation rules and called a
% \emph{traversal}. Traversals permit us to model $\beta$-reductions
% without altering the structure of the computation tree via
% substitution. A notable property is that \emph{P-views} (in the
% game-semantic sense) of traversals corresponds to paths in the
% computation tree.  We show that traversals are just representations of
% the uncovering of plays of the game-semantic denotation. We then
% define a \emph{reduction} operation which eliminates traversal nodes
% that are ``internal'' to the computation, this implements the
% counterpart of the hiding operation of game semantics. Thus, we obtain
% an isomorphism between the strategy denotation of a term and the set
% of reductions of traversals of its computation tree.
Using a correspondence result relating the game semantics of a
$\lambda$-term $M$ to a set of \emph{traversals} \cite{OngLics2006}
over a certain abstract syntax tree of the $\eta$-long form of $M$
(called \emph{computation tree}), we show that safe terms are denoted
by \emph{P-incrementally justified strategies}. In such a strategy,
pointers emanating from the P-moves of a play are uniquely
reconstructible from the underlying sequence of moves and the pointers
associated to the O-moves therein: Specifically, a P-question always
points to the last pending O-question (in the P-view) of a greater
order. Consequently pointers in the game semantics of safe
$\lambda$-terms are only necessary from order 4 onwards. Finally we
prove that a $\eta$-long $\beta$-normal $\lambda$-term is \emph{safe}
if and only if its strategy denotation is (innocent and)
\emph{P-incrementally justified}.



% \subsection*{Related work}

% \noindent\emph{The safety condition for higher-order grammars}

% \smallskip

% \noindent We have mentioned the result of Knapik \emph{et al.}~\cite{KNU02} that
% infinite trees generated by \emph{safe} higher-order grammars have
% decidable MSO theories.  A natural question to ask is whether the
% \emph{safety condition} is really necessary.  This has then been
% partially answered by Aehlig \emph{et al.}
% \cite{DBLP:conf/tlca/AehligMO05} where it was shown that safety is not
% a requirement at level $2$ to guarantee MSO decidability. Also, for
% the restricted case of word languages, the same authors have shown
% \cite{DBLP:conf/fossacs/AehligMO05} that level $2$ safe higher-order
% grammars are as powerful as (non-deterministic) unsafe ones.  De
% Miranda's thesis \cite{demirandathesis} proposes a unified framework
% for the study of higher-order grammars and gives a detailed analysis
% of the safety constraint at level 2.

% More recently, one of us obtained a more general result and showed
% that the MSO theory of infinite trees generated by higher-order
% grammars of any level, \emph{whether safe or not}, is decidable
% \cite{OngLics2006}.  Using an argument based on innocent
% game-semantics, he establishes a correspondence between the tree
% generated by a higher-order grammar called \emph{value tree} and a
% certain regular tree called \emph{computation tree}. Paths in the
% value tree correspond to traversals in the computation tree.
% Decidability is then obtain by reducing the problem to the acceptance
% of the (annotated) computation tree by a certain alternating parity
% tree automaton.  The approach that we follow in
% Sec. \ref{sec:correspondence} uses many ingredients introduced in this
% paper.


% The equivalence of \emph{safe} higher-order grammars and higher-order
% deterministic push-down automata for the purpose of generating
% infinite trees \cite{KNU02} has its counterpart in the general (not
% necessarily safe) case: the forthcoming paper \cite{hague-sto07}
% establishes the equivalence of order-$n$ higher-order grammars and
% order-$n$ \emph{collapsible pushdown automata}. Those automata form a
% new kind of pushdown systems in which every stack symbol has a link to
% a stack situated somewhere below it and with an additional stack
% operation whose effect is to ``collapse'' a stack $s$ to the state
% indicated by the link from the top stack symbol.

% \medskip

% \noindent\emph{Computation trees and traversals}

% \smallskip

% \noindent In \cite{DBLP:conf/lics/AspertiDLR94}, a notion of graph
% based on Lamping's graphs \cite{lamping} is introduced to represent
% $\lambda$-terms. The authors unify different notions of paths
% (regular, legal, consistent and persistent paths) that have appeared
% in the literature as ways to implement graph-based reduction of
% $\lambda$-expressions. We can regard a traversal as an alternative
% notion of path adapted to the graph representation of
% $\lambda$-expressions given by computation trees.

% The traversals of a computation tree provide a way to perform
% \emph{local computation} of $\beta$-reductions as opposed to a global
% approach where the $\beta$-reduction is implemented by performing
% substitutions. A notion of local computation of $\beta$-reduction has
% been investigated by Danos and Regnier
% \cite{DanosRegnier-Localandasynchronou} through the use of special
% graphs called ``virtual nets'' that embed the lambda-calculus.


\section{The safe lambda calculus}
\label{sec:safe}
\subsection*{Higher-order safe grammars}
We first present the safety restriction as it was originally defined
\cite{KNU02}. We consider simple types generated by the grammar $A \,
::= \, o \; | \; A \typear A$. By convention, $\rightarrow$ associates
to the right. Thus every type can be written as $A_1 \typear \cdots
\typear A_n \typear o$, which we shall abbreviate to $(A_1, \cdots,
A_n, o)$ (in case $n = 0$, we identify $(o)$ with $o$). The
\emph{order} of a type is given by $\ord{o} = 0$ and $\ord{A \typear
  B} = \max(\ord{A}+1, \ord{B})$. We assume an infinite set of typed
variables. The order of a typed term or symbol is defined to be the
order of its type.

A (higher-order) \defname{grammar} is a tuple $\langle
\Sigma, \mathcal{N}, \mathcal{R}, S \rangle$, where $\Sigma$ is a
ranked alphabet (in the sense that each symbol $f \in \Sigma$ has an
arity $\mathit{ar}(f) \geq 0$) of \emph{terminals}\footnote{Each $f \in
  \Sigma$ of arity $r \geq 0$ is assumed to have type $(\underbrace{o,
    \cdots, o}_r, o)$.}; $\mathcal{N}$ is a finite set of typed
\emph{non-terminals}; $S$ is a distinguished ground-type symbol of
$\mathcal{N}$, called the start symbol; $\mathcal{R}$ is a finite set
of production (or rewrite) rules, one for each non-terminal $F : (A_1,
\ldots, A_n, o) \in \mathcal{N}$, of the form $ F z_1 \ldots z_m
\rightarrow e$ where each $z_i$ (called \emph{parameter}) is a
variable of type $A_i$ and $e$ is an applicative term of type $o$
generated from the typed symbols in $\Sigma \union \mathcal{N} \union \{z_1,
\ldots, z_m \}$. We say that the grammar is \emph{order-$n$} just in
case the order of the highest-order non-terminal is $n$.

The \defname{tree generated by a recursion scheme} $G$ is a possibly
infinite applicative term, but viewed as a $\Sigma$-labelled tree;
it is \emph{constructed from the terminals in $\Sigma$}, and is obtained by
unfolding the rewrite rules of $G$ \emph{ad infinitum}, replacing
formal by actual parameters each time, starting from the start symbol
$S$. See e.g.~\cite{KNU02} for a formal definition.

\pssetcomptree
\parpic[r]{
$\tree[levelsep=3ex,nodesep=1pt,treesep=1cm,linewidth=0.5pt]{g}
{  \TR{a}
    \tree{g}{\TR{a} \tree{h}{\tree{h}{\vdots}}}
}$
}
\begin{example}\rm\label{eg:running}
  Let $G$ be the following order-2 recursion scheme:
\[\begin{array}{rll}
  S & \rightarrow & H \, a\\
  H \, z^o & \rightarrow & F \, (g \,
  z)\\
  F \, \phi^{(o, o)} & \rightarrow & \phi \, (\phi \, (F \, h))\\
\end{array}\]
where the arities of the terminals $g, h, a$ are $2, 1, 0$ respectively.
The tree generated by $G$ is defined by the infinite term $g \, a \, (g \, a \, (h \, (h \, (h \,
\cdots))))$.%  The only infinite \emph{path} in the
% tree is the node-sequence $\epsilon \cdot 2 \cdot 22 \cdot 221 \cdot
% 2211 \cdots$.

%(with the corresponding \textbfit{trace} $g \, g \, h \, h \, h \,
%\cdots \; \in \; \Sigma^\omega$).
\end{example}

A type $(A_1, \cdots, A_n, o)$ is said to be \defname{homogeneous} if
$\ord{A_1} \geq \ord{A_2}\geq \cdots \geq \ord{A_n}$, and each $A_1$,
\ldots, $A_n$ is homogeneous \cite{KNU02}.  We reproduce the following
definition from \cite{KNU02}.

\begin{definition}[Safe grammar]\rm
  (All types are assumed to be homogeneous.) A term of order $k > 0$
  is \emph{unsafe} if it contains an occurrence of a parameter of
  order strictly less than $k$, otherwise the term is \emph{safe}. An
  occurrence of an unsafe term $t$ as a subexpression of a term $t'$
  is \emph{safe} if it is in the context $\cdots (ts) \cdots$,
  otherwise the occurrence is \emph{unsafe}. A grammar is
  \defname{safe} if no unsafe term has an unsafe occurrence at a
  right-hand side of any production.
%   A rewrite rule $F z_1 \ldots z_m \rightarrow e$ is said to be
%   \defname{unsafe} if the righthand term $e$ has a subterm $t$ such
%   that
% \begin{enumerate}[(i)]
% \item $t$ occurs in an {\em operand} ({\it i.e.}~second) position of some
%   occurrence of the implicit application operator {\it i.e.}~$e$ has the
%   form $\cdots (s \, t) \cdots $ for some $s$
% \item $t$ contains an occurrence of a parameter $z_i$ (say) whose
%   order is less than that of $t$.
% \end{enumerate}
% A homogeneous grammar is said to be \defname{safe} if none of its
% rewrite rules is unsafe.
\end{definition}

\begin{example}\begin{inparaenum}[(i)] \item Take $\; H : ((o, o), o), \; f : (o, o, o)$; the
    following rewrite rules are unsafe (in each case we underline the
    unsafe subterm that occurs unsafely):
\[\begin{array}{rll}
G^{(o, o)} \, x & \quad \rightarrow \quad & H \, \underline{(f \, {x})} \\
F^{((o, o), o, o, o)} \, z \, x \, y & \quad \rightarrow \quad & f \, (F \, \underline{(F \, z
\, {y})} \, y \, (z \, x) ) \, x
\end{array}\]
\item The order-2 grammar defined in Example~\ref{eg:running} is
  unsafe.
\end{inparaenum}
% The
% reader is referred to the literature
% \cite{KNU02,demirandathesis,safety-mirlong2004}
% for details about the safety restriction for higher-order grammars.
\end{example}

\subsection*{Safety adapted to the lambda calculus}
We assume a set $\Xi$ of higher-order constants.
We use sequents of the form $\Gamma \vdash_\Xi M : A$ to represent
terms-in-context where $\Gamma$ is the context and $A$ is the type of
$M$. For simplicity
we write $(A_1, \cdots, A_n, B)$ to mean $A_1 \typear \cdots \typear
A_n \typear B$, where $B$ is not necessarily ground.

\begin{definition}\rm
\begin{inparaenum}[(i)]
\item The \defname{safe lambda calculus} is a sub-system of the
  simply-typed lambda calculus defined by induction over the
  following rules:
$$ \rulename{var} \ \rulef{}{x : A\vdash_\Xi x : A} \quad
\rulename{const} \ \rulef{}{\vdash_\Xi f : A} \quad f \in \Xi \quad
\rulename{wk} \ \rulef{\Gamma \vdash_\Xi s : A}{\Delta \vdash_\Xi s : A} \quad
\Gamma \subset \Delta$$
$$ \rulename{app} \ \rulef{\Gamma \vdash_\Xi s : (A_1,\ldots,A_n,B) \
  \Gamma \vdash_\Xi t_1 : A_1 \; \ldots \; \Gamma \vdash_\Xi t_n : A_n
} {\Gamma \vdash_\Xi s t_1 \ldots t_n : B} \ \ord{B} \sqsubseteq
\ord{\Gamma}$$
$$ \rulename{abs} \ \rulef{\Gamma, x_1 : A_1, \ldots, x_n : A_n
  \vdash_\Xi s : B} {\Gamma \vdash_\Xi \lambda x_1 \ldots x_n . s :
  (A_1, \ldots ,A_n,B)} \ \ord{A_1, \ldots ,A_n,B} \sqsubseteq
\ord{\Gamma}$$ where $\ord{\Gamma}$ denotes the set $\{ \ord{y} : y
\in \Gamma \}$ and ``$c \sqsubseteq S$'' means that $c$ is a
lower-bound of the set $S$. For convenience, we shall omit the
subscript from $\vdash_\Xi$ whenever the generator-set $\Xi$ is clear from
the context.

\noindent \item The sub-system that is defined by the same rules in
(i), such that all types that occur in them are homogeneous, is called
the \defname{homogeneous safe lambda calculus}.
\end{inparaenum}
\end{definition}

The safe lambda calculus deviates from the standard definition of the simply-typed lambda calculus in a number of ways. First the rules $\rulename{app}$ and $\rulename{abs}$
respectively can perform multiple applications and abstract several
variables at once. (Of course this feature alone does not alter
expressivity.) Crucially, the side-conditions in the application rule
and abstraction rules require that variables in the typing context
have order no smaller than that of the term being formed.  We do not
impose any constraint on types. In particular, type-homogeneity as
used originally to define safe grammars \cite{KNU02} is not required
here. Another difference is that we allow $\Xi$-constants to have
arbitrary higher-order types.  % Thus our formulation
% of the safe lambda calculus is more general than the one proposed in
% the technical report \cite{safety-mirlong2004}. (It is possible to
% reconcile the two definitions by adding the further constraint that
% each type occurring in our rules is homogeneous and by restricting
% constants to at most order 1.)

\begin{example}[Kierstead terms]
\label{ex:kierstead}
Consider the terms $M_1 = \lambda f . f (\lambda x . f (\lambda y . y
))$ and $M_2 = \lambda f . f (\lambda x . f (\lambda y .x ))$ where
$x,y:o$ and $f:((o,o),o)$. The term $M_2$ is not safe because in the
subterm $f (\lambda y . x)$, the free variable $x$ has order $0$ which
is smaller than $\ord{\lambda y . x} = 1$.  On the other hand, $M_1$
is safe.
%On the other hand, $M_1$ is safe as the following proof tree shows:
%$$
% \rulef{
%     \rulef{
%        \rulef{}{f \vdash f} {\sf(var)}
%        \
%        \rulef{
%             \rulef{
%                \rulef{
%                    \rulef{}{f \vdash f} {\sf(var)}
%                }
%                {f , x \vdash f } {\sf(wk)}
%                \
%                \rulef{
%                    \rulef{
%                        \rulef{}{y \vdash y} {\sf(var)}
%                    }
%                    {y \vdash \lambda y . y } \rulenamet{abs}
%                }
%                {f , x \vdash \lambda y .y } {\sf(wk)}
%             }
%             {f , x \vdash f (\lambda y .y )} {\sf(app)}
%        }
%        { f  \vdash \lambda x . f (\lambda y .y )} \rulenamet{abs}
%     }
%     {
%        f  \vdash f (\lambda x . f (\lambda y .y ))} {\sf(app)}
%     }
% { \vdash M_1 = \lambda f . f (\lambda x . f (\lambda y .y )) } \rulenamet{abs}
%$$
\end{example}

It is easy to see that valid typing judgements of the safe lambda
calculus satisfy the following simple invariant:
\begin{lemma}
\label{lem:ordfreevar}
If $\Gamma \vdash M : A$ then every variable in $\Gamma$ occurring
free in $M$ has order at least $ord(M)$.
\end{lemma}


When restricted to the homogeneously-typed
sub-system, the safe lambda calculus captures the original notion
of safety due to Knapik \emph{et al.} in the context of higher-order
grammars:

\begin{proposition} Let $G = \langle \Sigma, \mathcal{N}, \mathcal{R},
  S \rangle$ be a grammar and let $e$ be an applicative term generated
  from the symbols in $\mathcal{N} \cup \Sigma \cup \makeset{z_1^{A_1},
    \cdots, z_m^{A_m}}$.  A rule $F z_1 \ldots z_m \rightarrow e$ in
  $\mathcal{R}$ is safe if and only if $ z_1 : A_1, \cdots, z_m : A_m
  \vdash_{\Sigma \cup \mathcal{N}} e : o$ is a valid typing judgement
  of the \emph{homogeneous} safe lambda calculus.
\end{proposition}

\emph{In what sense is the safe lambda calculus safe?} A basic idea
in the lambda calculus is that when performing $\beta$-reduction, one
must use capture-\emph{avoiding} substitution, which is standardly
implemented by renaming bound variables afresh upon each substitution.
In the safe lambda calculus, however, variable capture can never
happen (as the following lemma shows). Substitution can therefore be
implemented simply by capture-\emph{permitting} replacement, without
any need for variable renaming. In the following, we write
$M\captsubst{N}{x}$ to denote the capture-\emph{permitting}
substitution\footnote{This substitution is done by
textually replacing all free occurrences of $x$ in $M$ by $N$ without performing variable renaming.  In particular for the abstraction
  case we have
$(\lambda y_1\ldots y_n . M)\captsubst{N}{x} = \lambda y_1\ldots y_n . M\captsubst{N}{x}$ when $x\not\in
  \{ y_1\ldots y_n \}$.}
%\footnote{This substitution is implemented by textually
%  replacing all free occurrences of $x$ in $M$ by $N$ without
%  performing variable renaming.  In particular for the abstraction
%  case $(\lambda \overline{y} . P)\captsubst{N}{x}$ is defined as
%  $\lambda \overline{y} . P\captsubst{N}{x}$ if $x\not\in
%  \overline{y}$ and $\lambda \overline{y} . P$ elsewhere.}
of $N$ for $x$ in $M$.

\begin{lemma}[No variable capture]\label{lem:nvc}
\label{lem:homog_nocapture} There is
no variable capture when performing capture-permitting
substitution of $N$ for $x$ in $M$
provided that $\Gamma, x:B \vdash M : A$ and $\Gamma \vdash  N : B$ are valid judgments of the safe lambda calculus.
\end{lemma}

\proof
  We proceed by structural induction. The variable, constant and
  application cases are trivial. For the abstraction case, suppose $M = \lambda \overline{y}. R$ where $\overline{y} = y_1
  \ldots y_p$. If $x \in \overline{y}$ then $M \captsubst{N}{x} = M$ and there is no variable capture.

 If $x \not\in \overline{y}$ then we have $M \captsubst{N}{x} = \lambda \overline{y} . R \captsubst{N}{x}$.  By the induction hypothesis there is no variable capture in $R \captsubst{N}{x}$.  Thus variable capture can only happen if the following two conditions are met: $x$ occurs freely in $R$, and some variable $y_i$ for $1 \leq i \leq p$ occurs freely in $N$. By Lemma \ref{lem:ordfreevar}, the latter condition  implies $\ord{y_i} \geq \ord{N} = \ord{x}$.  Since $x \not \in \overline{y}$, the former condition implies that $x$ occurs freely in the safe term $\lambda \overline{y}. R$
  therefore Lemma \ref{lem:ordfreevar} gives $ \ord{x} \geq
  \ord{\lambda \overline{y} . R} \geq 1+ \ord{y_i} > \ord{y_i}$ which  gives a contradiction.
\qed


\begin{remark}
  A version of the No-variable-capture Lemma also holds in safe
  grammars, as is implicit in (for example Lemma 3.2 of) the original
  paper \cite{KNU02}.
\end{remark}

\begin{example}
  In order to contract the $\beta$-redex in the term
\[f:(o,o,o),x:o
  \vdash (\lambda \varphi^{(o,o)} x^o . \varphi \, x) (\underline{f \,
    x}) : (o,o)\] one should rename the bound variable $x$ with a fresh name to
  prevent the capture of the free occurrence of $x$ in the underlined term during substitution. Consequently, by the previous lemma,
  the term is not safe. Indeed, it cannot be because $\ord{x} = 0 < 1
  = \ord{f x}$.
\end{example}

Note that it is not the case that $\lambda$-terms
that satisfy the No-variable-capture Lemma are necessarily safe. For instance the $\beta$-redex in $\lambda y^o
z^o. (\lambda x^o .y) z$ can be contracted using capture-permitting
substitution, even though the term is not safe.

\subsection*{Reductions and transformations preserving safety}

From now on we will use the standard notation $M\subst{N}{x}$ to
denote the substitution of $N$ for $x$ in $M$.  It is understood that,
provided that $M$ and $N$ are safe, this substitution is
capture-permitting.


\begin{lemma}[Substitution preserves safety]
\label{lem:subst_preserve_safety}
If $\Gamma, x :B \vdash M : A$ and $\Gamma \vdash N : B$ then $\Gamma \vdash M[N/x] : A$.
\end{lemma}
This is proved by an easy induction on the structure of the safe term $M$.


It is desirable to have an appropriate notion of reduction for our
calculus. However the standard $\beta$-reduction rule is not
adequate. Indeed, safety is not preserved by $\beta$-reduction as the
following example shows. Suppose that $w,x,y,z : o$ and $f : (o,o,o)
\in \Sigma$ then the safe term $(\lambda x y . f x y) z w$
$\beta$-reduces to $(\underline{\lambda y . f z y}) w$ which is unsafe
since the underlined order-1 subterm contains a free occurrence of the
ground-type $z$. However if we perform one more reduction we obtain
the safe term $f z w$. This suggests an alternative notion of
reduction that performs simultaneous reduction of ``consecutive''
$\beta$-redexes. In order to define this reduction we first introduce
the appropriate notion of redex.

In the simply-typed lambda calculus a redex is a term of the form
$(\lambda x . M) N$. In the safe lambda calculus, a redex is a
succession of several standard redexes:

\begin{definition}\rm
Let $l\geq 1$ and $n\geq 1$. We use the abbreviations $\overline{x}$ and $\overline{x}:\overline{A}$  for $x_1 \ldots x_n$ and $x_1:A_1, \ldots, x_n : A_n$ respectively.

A \defname{safe redex} is a safe term of the form $(\lambda
\overline{x} . M) N_1 \ldots N_l$ such that the variables
$\overline{x}$ are abstracted altogether by one instance of the
\rulenamet{abs} rule (possibly followed by \rulenamet{wk}) and the
term $(\lambda \overline{x}.M)$ is applied to $N_1$, \ldots, $N_l$
by one instance of the \rulenamet{app} rule. Thus $M$, the and the
$N_i$'s are also safe.
\end{definition}
For instance, in the case $n<l$, a safe redex has a derivation tree of the following  form:
$$   \rulef{
            \rulef{\rulef{\rulef{\ldots}{\Gamma', \overline{x}:\overline{A} \vdash M : (A_{n+1}, \ldots, A_l, B)}}{\Gamma' \vdash \lambda \overline{x} . M : (A_1, \ldots, A_l, B)} \rulename{abs}}{\Gamma \vdash \lambda \overline{x} . M : (A_1, \ldots, A_l, B)}\rulename{wk}
            \quad
            \rulef{\ldots}{\Gamma \vdash N_1 :A_1}  \ \ldots \  \rulef{\ldots}{\Gamma \vdash N_l :A_l}
    }
    {
       \Gamma \vdash (\lambda \overline{x} . M) N_1 \ldots N_l : B
    } \rulename{app}
$$


We are now in a position to define a notion of reduction for safe terms.
\begin{definition}\rm
\label{dfn:safereduction} We use the
abbreviations $\overline{x} = x_1 \ldots x_n$,
$\overline{N} = N_1 \ldots N_l$.
The relation $\beta_s$ is defined on the set of safe redexes as:
\begin{eqnarray*}
  \beta_s &=&
  \{  \ (\lambda \overline{x} . M) N_1 \ldots N_l \mapsto \lambda x_{l+1} \ldots x_n. M\subst{\overline{N}}{x_1 \ldots x_l} \mbox{, for $n> l$}
  \} \\
  &\cup&
  \{ \ (\lambda \overline{x}  . M) N_1 \ldots N_l \mapsto M\subst{N_1 \ldots N_n}{\overline{x}} N_{n+1} \ldots N_l
  \mbox{, for $n\leq l$} \} \ .
\end{eqnarray*}
where $M\subst{R_1 \ldots R_k}{z_1 \ldots z_k}$ denotes the simultaneous substitution in $M$ of $R_1$,\ldots,$R_k$ for $z_1, \ldots, z_k$.  The
\defname{safe $\beta$-reduction}, written $\betasred$, is the
compatible closure of the relation $\beta_s$ with respect to the
formation rules of the safe lambda calculus.
\end{definition}

\noindent \emph{Remark:} The $\beta_s$-reduction is a multi-step
$\beta$-reduction \ie it is a subset of the transitive closure of $\betared$.


\begin{lemma}[$\beta_s$-reduction preserves safety]
\label{lem:safered_preserve_safety}
If $\Gamma \vdash s :A$ and $s \betasred t$ then $\Gamma \vdash t :A$.
\end{lemma}

\proof
  It suffices to show that the relation $\beta_s$ preserves safety.
Suppose that $s\ \beta_s\ t$ where $s$ is the
safe-redex $(\lambda x_1 \ldots x_n . M) N_1
  \ldots N_l $ with $x_1 : B_1, \ldots, x_n: B_n$
and $M$ of type $C$.  W.l.o.g we can assume that the last rule used
to form the term $s$ is \rulenamet{app} (and not the weakening rule
\rulenamet{wk}, thus  we have $\Gamma = fv(s)$.

Suppose $n>l$ then $A = (B_{l+1}, \ldots, B_n, C)$. By Lemma \ref{lem:subst_preserve_safety} we can form the safe term %\begin{equation}
$\Gamma, x_{l+1}:B_{l+1}, \ldots x_n :B_{n}\vdash M\subst{\overline{N}}{x_1 \ldots x_l} : C$. %\label{jud:substsafe}\ .
%\end{equation}
By Lemma \ref{lem:ordfreevar}, since $s$ is safe, all the variables
in $\Gamma$ have order $\geq \ord{A}$. This ensures that the
side-condition of the \rulenamet{abs} rule is verified when
abstracting the variables $x_{l+1} \ldots x_n$, which gives us the
judgement $\Gamma \vdash t :A$.

Suppose $n \leq l$. The substitution lemma gives
$\Gamma \vdash M\subst{N_1 \ldots N_n}{\overline{x}} : C$ and using \rulenamet{app} we form $\Gamma \vdash t :A$.
  \qed


In general, safety is not preserved by $\eta$-expansion; for instance we have
% $f:o,o \vdash f$ but $f:o,o \not \vdash \lambda x^o . f x$.
%This remark remains true for closed terms, for instance
$\vdash \lambda y^o z^o . y : (o,o,o)$ but
$\not \vdash \lambda x^o . (\lambda y^o z^o . y) x : (o,o,o)$.
However safety is preserved by $\eta$-reduction:

\begin{lemma}[$\eta$-reduction preserves safety]
  $\Gamma \vdash \lambda \varphi . s \varphi :A $ with $\varphi$ not
  occurring free in $s$ implies $\Gamma \vdash s :A$.
\end{lemma}
\proof
  Suppose $\Gamma \vdash \lambda \varphi . s \varphi :A$. If $s$ is an  abstraction then by construction of the safe term $\lambda \varphi . s \varphi$, $s$ is necessarily safe.  If $s = N_0 \ldots N_p$ with
  $p\geq 1$ then again, since $\lambda \varphi . N_0 \ldots N_p
  \varphi$ is safe, each of the $N_i$ is safe for $0 \leq i \leq p$
  and for any $z\in fv(\lambda \varphi . s \varphi)$, $\ord{z} \geq
  \ord{\lambda \varphi . s \varphi} = \ord{s}$. Since  $\varphi$ does not occur free in $s$ we have $fv(s) = fv(\lambda \varphi . s \varphi)$, thus we can use the application rule to form $fv(s) \vdash N_0 \ldots N_p : A$. The weakening rules permits us to conclude $\Gamma \vdash s :A$. \qed



The $\eta$-long normal form (or simply $\eta$-long form) of a term
% (also called \emph{long reduced form}, \emph{$\eta$-normal form} and
% \emph{extensional form} in the literature
% \cite{DBLP:journals/tcs/JensenP76,DBLP:journals/tcs/Huet75,huet76})
is obtained by hereditarily $\eta$-expanding every subterm occurring
at an operand position. Formally the \defname{$\eta$-long form}
$\elnf{t}$ of a term $t: (A_1,\ldots,A_n,o)$ with $n \geq 0$ is
defined by cases according to the syntactic shape of $t$:
\begin{eqnarray*}
  \elnf{\lambda x . s } &=& \lambda x . \elnf{s} \\
  \elnf{x s_1 \ldots s_m } &=& \lambda \overline{\varphi} . x \elnf{s_1}\ldots \elnf{s_m} \elnf{\varphi_1} \ldots \elnf{\varphi_n} \\
  \elnf{(\lambda x . s) s_1 \ldots s_p } &=& \lambda \overline{\varphi} . (\lambda x . \elnf{s}) \elnf{s_1} \ldots \elnf{s_p} \elnf{\varphi_1} \ldots \elnf{\varphi_n}
\end{eqnarray*}
where $m \geq 0$, $p\geq 1$, $x$ is a  variable or constant, $\overline{\varphi} = \varphi_1 \ldots \varphi_n$ and each $\varphi_i : A_i$ is a fresh variable.

%\begin{remark}
%  Converting a term to its $\eta$-long normal form does not introduce
%  new redex therefore the $\eta$-long normal form of $\beta$-normal
%  term is a $\beta$-normal term.
%\end{remark}

\begin{lemma}[$\eta$-long normalization preserves safety]
\label{lem:elnf_preserves_safety}
If $\Gamma \vdash s :A$ then $\Gamma \vdash \elnf{s} :A$.
\end{lemma}
\proof

 First we observe that for any variable or constant $x:A$ we have $x:A \vdash \elnf{x} :A$. We show this by induction on $\ord{x}$.
It is verified for any ground type variable $x$
since $x = \elnf{x}$.
Step case: $x:A$ with $A=(A_1, \ldots, A_n,o)$ and $n>0$. Let $\varphi_i:A_i$ be fresh variables for $1\leq i\leq n$.
Since $\ord{A_i} < \ord{x}$ the induction hypothesis gives $\varphi_i :A_i \vdash \elnf{\varphi_i} : A_i$. Using \rulenamet{wk} we obtain $x:A, \overline{\varphi} : \overline{A}
  \vdash \elnf{\varphi_i} :A_i$.  The application rule gives $x :A, \overline{\varphi} : \overline{A} \vdash x \elnf{\varphi_1} \ldots \elnf{\varphi_n}
  : o$ and the abstraction rule gives $ x :A \vdash \lambda
  \overline{\varphi} . x \elnf{\varphi_1} \ldots \elnf{\varphi_n} =
  \elnf{x} :A$.


We now prove the lemma by induction on $s$.
The base case is covered by the previous observation.
\emph{Step case:}
\begin{compactitem}
\item $s = x s_1 \ldots s_m$ with $x: (B_1, \ldots, B_m, A)$, $A = (A_1, \ldots, A_n, o)$ for some $m\geq 0$, $n>0$ and $s_i : B_i$ for $1 \leq i \leq
  m$.  Let $\varphi_i: A_i$ be fresh variables for $1\leq i \leq
  n$. By the previous observation we have $\varphi_i :A_i \vdash \elnf{\varphi_i} :A_i$, the weakening rule then gives us $\Gamma , \overline{\varphi} : \overline{A}
  \vdash \elnf{\varphi_i} : A_i$.  Since the judgement
  $\Gamma \vdash x s_1 \ldots s_m : A$ is formed using the \rulenamet{app} rule, each $s_j$ must be safe for $1\leq j \leq m$, thus by the induction hypothesis we have $\Gamma \vdash \elnf{s_j} : B_j$ and by weakening we get $\Gamma, \overline{\varphi} :\overline{A} \vdash \elnf{s_j} : B_j$.  The \rulenamet{app}
  rule then gives $\Gamma, \overline{\varphi} :\overline{A} \vdash x \elnf{s_1} \ldots \elnf{s_m} \elnf{\varphi_1} \ldots \elnf{\varphi_n} : o$. Finally
  the \rulenamet{abs} rule gives $\Gamma \vdash \lambda \overline{\varphi} . x
  \elnf{s_1} \ldots \elnf{s_m} \elnf{\varphi_1} \ldots
  \elnf{\varphi_n} = \elnf{s} : A$, the side-condition of \rulenamet{abs} being verified since $\ord{\elnf{s}} = \ord{s}$.


\item $s = t s_0 \ldots s_m$ where $t$ is an abstraction.
For some fresh variables $\varphi_1$, \ldots, $\varphi_n$
we have $\elnf{s} = \lambda \overline{\varphi}. \elnf{t} \elnf{s_0} \ldots \elnf{s_m} \elnf{\varphi_1}
  \ldots \elnf{\varphi_n}$. Again, using the induction hypothesis we can easily derive $\Gamma \vdash
 \lambda \overline{\varphi}. \elnf{t} \elnf{s_0} \ldots \elnf{s_m} \elnf{\varphi_1} \ldots \elnf{\varphi_n} : A$.

\item $s = \lambda \overline{\eta} . t $ where
$\overline{\eta} : \overline{B}$ and $t:C$ is not an abstraction. The induction hypothesis gives $\Gamma,
  \overline{\eta} : \overline{B} \vdash \elnf{t} : C$ and using
\rulenamet{abs} we get $\Gamma \vdash \lambda \overline{\eta} . \elnf{t} = \elnf{s} : A$.  \qed
\end{compactitem}


Note that the converse does not hold in general, for instance $\lambda
x^o . f^{(o,o,o)} x^o$ is unsafe although $\elnf{\lambda x . f x} =
\lambda x^o y^o . f x y$ is safe.


%\notetoself{Check and prove the following lemma}
% For terms with homogeneous types however, the converse does hold:
%\begin{lemma}
%If $\Gamma \vdash \elnf{s} : T$ is homogeneously safe (i.e. it is a
%safe judgement of the safe $\lambda$-calculus and each sequent
%occurring at the nodes of the proof tree is homogeneously typed)
%then $\Gamma \vdash s :T$ is homogeneously safe.
%\end{lemma}

    \newcommand\bigo{\mathcal{O}} % big O notation
\newcommand\booltype{\mathbf{B}}

\section{Complexity of the Safe Lambda Calculus}
Here we study the problem of deciding beta-eta equivalence of two safe lambda terms.

\subsection{Statman's result}

A famous result by Statman  states that deciding the $\beta\eta$-equality of two first-order typable lambda terms is not elementary recursive \cite{Statman:1979:TLE}.
The idea of the proof is to encode the Henkin quantifier elimination of Type Theory into the simply-typed lambda calculus. The encoding relies on the fact that the function $\sf sg$ (conditional) can be encoded in the lambda-calculus. Hence the argument does not carry on   in the Safe Lambda Calculus since the conditional operator is not definable (\cite{blumong:safelambdacalculus}).

Mairson gave a simpler proof of Statman's theorem in \cite{mairson1992spt} which also proceeds by encoding the Henkin quantifier elimination procedure into the lambda-calculus but is much easier to understand as it makes use of list iteration to perform quantifier elimination.

It turns out that both encodings rely on the use of unsafe terms in order to implement the quantifier elimination procedure.

%Here we adapt Mairson's proof to produce a safe encoding of the quantifier elimination procedure, thus showing:
%\begin{theorem}
%The Safe Lambda Calculus is not elementary recursive.
%\end{theorem}

We recall the definition of the theory. Let $\mathcal{D}_0 = \{\mathbf{true},\mathbf{false}\}$ and $\mathcal{D}_{k+1} =powerset(\mathcal{D}_k)$.
For any $k\geq0$, we write $x^k$, $y^k$ and $z^k$ to denote variables ranging over $\mathcal{D}_k$. Prime formulas are $x^0$, $\mathbf{true}\in y^1$, $\mathbf{false}\in y^1$, and  $x^k \in y^{k+1}$. Formulae are built up from prime formulas using the logical connectives $\zand$,$\zor$,$\rightarrow$,$\neg$ and the quantifiers
$\forall$ and $\exists$. Meyer showed that deciding the truth of such formulae requires nonelementary time \cite{Meyer1974}.
\smallskip

In Mairson's encoding, all formula variables of a given order $k$ are encoded by terms of the same type $\Delta_k$. Using this encoding,
unsafety manifests itself in two different ways.
\begin{enumerate}[1.]
  \item
        First in the encoding of set membership. The prime formula $x^k \in y^{k+1}$ is encoded as \begin{equation} x^k : \Delta_k, y^{k+1}:\Delta_{k+1} \vdash y^{k+1} (\lambda y^k : \Delta_k . OR (eq_k~\underline{x^k}~y^k)~F : \Delta_k \typear \Delta_{k+1} \typear \Delta_0 \label{eqn:setmembership}\end{equation}
for some terms $OR$, $F$, $eq_k$.
This term is unsafe because of the underline occurrence of $x^k$ which is not abstracted together with $y^k$.

\item Secondly, quantifier elimination is performed by using a list iterator $\mathbf{D}_{k+1}$ which acts like the $fold\_left$ function from functional programming languages over the list of all elements of $\mathcal{D}_k$.
Thus for instance the formula $\forall x^0 . \exists y^0 . x^0 \zor y^0$
is encoded as $$\vdash \mathbf{D}_1 (\lambda x^0:\Delta_0. AND (\mathbf{D}_1 (\lambda y^0:\Delta_0. OR (\underline{x^0} \zor y^0)) F)) T$$ which is unsafe because of the underlined occurrence.

More generally, supposing that we find a way to encode set membership with a safe term, then the encoding of the formula will be safe if and only if for any variable $x$ in the formula, its binder is precisely the first quantifier $\exists z$ or $\forall z$ in the path to the root of the formula AST verifying $\ord{z} \geq \ord{x}$. For instance the formula $\forall x^k . \exists y^{k+1} . x^k \in y^{k+1}$ would be encoded by an unsafe term whereas the encoding of $\forall y^{k+1} . \exists x^k . x^k \in y^{k+1}$ would be safe.
\end{enumerate}

Surprisingly, the unsafety of the quantifier elimination procedure can be
easily overcome. The idea is as follows. We introduce multiple domains of representation for a given formula. An element of $\mathcal{D}_k$ is thereby represented by countably many terms of type $\Delta_k^n$ where $n\in\nat$ indicates the level of the representation. The type $\Delta_k^n$ is defined in such a way that its order strictly increases as $n$ grows. Moreover there exists a term that can reduce the level of representation of a given term. In the formula representation, each variable can now be encoded with a different level of representation. Since there are infinitely many levels, it is always possible to find an assignment of levels to variables such that the resulting encoding term is safe.

For set-membership, however, there is no obvious way to obtain a safe encoding. The set-member function from Eq.\ \ref{eqn:setmembership} can be turned into a safe term provided that we have access to a function permitting us to increase the representation level of term, but to our knowledge, such transformation cannot be expressed in the simply-typed lambda-calculus.



\subsubsection{Encoding basic boolean operations}

We assume a ground type $o$.
%For any type $\mu$ we define the type $\booltype_\mu \equiv (\mu\typear\mu)\typear \mu$.
%We abbreviate $\booltype_0$ into $\booltype$.
%We introduce the following hierarchy of types: $\sigma_0 \equiv o$, $\sigma_{n+1} \equiv \booltype_{\sigma_n}$ for $n\geq1$.
%Note that the order of $\sigma_n$ strictly increases as $n$ increases.
We introduce a parameterized type for encoding booleans defined by $\booltype_{-1} \equiv o$ and $\booltype_{n+1} \equiv \booltype_n\typear\booltype_n\typear\booltype_n$ for $n\geq0$.
We have $\ord{\booltype_n} = n+1$ for $n\geq-1$.


The representation of the truth values $\mathbf{true}$ and $\mathbf{false}$ will be parameterized by $n \in \nat$ as follows
\begin{align*}
  T^n &\equiv \lambda x^{\booltype_{n-1}} y^{\booltype_{n-1}} .x : \booltype_{n}\\
  F^n &\equiv \lambda x^{\booltype_{n-1}} y^{\booltype_{n-1}} .y : \booltype_{n}
\end{align*}
Clearly these terms as safe. Moreover the following relations hold for all $n$:
\begin{align*}
  T^{n+1}~T^n~F^n &\betared^*  T^n \\
  F^{n+1}~T^n~F^n &\betared^*  F^n
\end{align*}
Hence it is possible to lower the representation level of a term encoding a boolean value by applying the two terms $T^n$ and $F^n$ to it.
For $i\in\nat$, we define the function $\_ \downarrow_i$ that lowers the level-representation of a term, turning a term of type $\booltype_n$ for some $n\in\nat$ to a term of type $\booltype_{\min(i,l)}$:
$$ (M :\booltype_n)\downarrow_i = \left\{
  \begin{array}{ll}
    M~T^{n-1}~F^{n-1}~\ldots~T^{i+1}~F^{i+1}:\booltype_i, & \hbox{if $n>i$;} \\
M:\booltype_n, & \hbox{otherwise.}
  \end{array}
  \right.
$$


Boolean functions are encoded by the following level-parameterized terms:
\begin{align*}
AND^n &\equiv \lambda p : \booltype_n \lambda q : \booltype_n \lambda x:\booltype_{n-1} \lambda y:\booltype_{n-1} . p~(q~x~y)~y : \booltype_n\typear\booltype_n\typear\booltype_n \\
OR^n &\equiv \lambda p : \booltype_n \lambda q : \booltype_n \lambda x:\booltype_{n-1} \lambda y:\booltype_{n-1} . p~x~(q~x~y) : \booltype_n\typear\booltype_{n}\typear\booltype_n \\
NOT^n &\equiv \lambda p : \booltype_n \lambda q : \booltype_n \lambda x:\booltype_{n-1} \lambda y:\booltype_{n-1} . p~y~x : \booltype_n\typear\booltype_n\typear\booltype_n \\
IF^n &\equiv \lambda p : \booltype_n \lambda q : \booltype_n \lambda x:\booltype_{n-1} \lambda y:\booltype_{n-1} . OR^n (NOT^n p)~q : \booltype_n\typear\booltype_n\typear\booltype_n
\end{align*}
which are all safe terms.

\subsubsection{Coding elements of the type hierarchy}
For any $n\in\nat$ we define the hierarchy of type $\Delta_k^n$ as follows:
$\Delta_0^n \equiv \booltype_n$ and $\Delta_{k+1}^n \equiv {\Delta_k^n}^*$ where for any type $\alpha$, $\alpha^* = (\alpha \typear \tau \typear \tau)\typear \tau \typear \tau$.

An occurrence of a formula variable $x^k$ will be encoded as a term variable $x^k:\Delta_{k}^n$ for some level of representation $n\in\nat$.

Following Mairson's  proof, we encode the set $\mathcal{D}_0$ as the list $\mathbf{D}_0$ containing $\mathbf{true}$ and $\mathbf{false}$, and we parameterized this representation by $n\in \nat$:
$$\mathbf{D}_0^n \equiv \lambda c:\booltype_n \typear \tau \typear \tau . \lambda e : \tau . c~T^n~(c~F^n~e) : \Delta_1^n$$
and for $k\geq 0$, the higher-order set $\mathcal{D}_{k+1}$ is represented by the parameterized term:
$$\mathbf{D}_{k+1}^n \equiv powerset~\mathbf{D}_k^n : \Delta_{k+2}^n$$
where the term $powerset$ taken from \cite{mairson1992spt} is reproduced here:
\begin{align*}
  powerset &\equiv \lambda A^* :(\alpha \typear \alpha^{**} \typear \alpha^{**}) \typear \alpha^{**} \typear \alpha^{**}.\\
&\qquad  A^*~double~(\lambda c:\alpha^* \typear \tau\typear \tau.\lambda b:\tau . c ( \lambda c':\alpha\typear \tau\typear \tau. \lambda b':\tau.b') b)\\
powerset &: ((\alpha \typear \alpha^{**} \typear \alpha^{**}) \typear \alpha^{**} \typear \alpha^{**})\typear \alpha^{**}
\end{align*}
with
\begin{align*}
  double &\equiv \lambda x :\alpha.\lambda l : (\alpha^* \typear \tau\typear \tau)\typear \tau\typear \tau. \\
  & \qquad \lambda c:\alpha^*\typear \tau\typear \tau.\lambda b:\tau. \\
  & \qquad \qquad l(\lambda e:\alpha^*.c (\lambda c':\alpha\typear \tau\typear \tau.\lambda b':\tau.c'~x~(e~c'~b')))(l~c~b)\\
double &: \alpha \typear \alpha^{**} \typear \alpha^{**}
\end{align*}

It can be checked that these two terms are safe.

\subsubsection{Quantifier elimination}
Following \cite{mairson1992spt}, quantifier elimination interprets $\forall x^k.\Phi(x^k)$ as the iterated conjunction $\mathbf{D}_k(\lambda x^k:\Delta^k.AND(\hat\Phi~x^k))~T$ where $\hat\Phi$ is the interpretation of $\Phi$; similarly $\exists x^k.\Phi(x^k)$  is interpreted by the iterated disjunction $\mathbf{D}_k(\lambda x^k:\Delta^k.AND(\hat\Phi~x^k))~T$.

Let $x^{k_p}_p \ldots x^{k_1}_1$ for $p\geq1$ be the list of variables appearing in the formula. W.l.o.g.\ we can assume that they are given in the order of appearance of their binder in the formula \ie $x^{k_p}_p$ is bound by the leftmost binder. We assign representation levels to variables as follows. The right-most variable is assigned level $1$ \ie $x^{k_1}_1 : \Delta^1_{k_1}$; suppose that $x^{k_i}_i :\Delta^l_{k_l}$ for $1\leq i< p$ then the representation level of variable $x^{k_{i+1}}_{i+1}$ is defined as
the smallest $l'\in\nat$ such that $\ord{\Delta^{l'}_{k_i}} > \ord{\Delta^{l}_{k_{i-1}}}$.

This way, since variables that are bound first have higher order, the variables
 that are bound in the nested list-iterations (corresponding to the nested quantifiers in the formula) are necessarily safely bound.


\subsubsection{Coding set theory in the $\mathcal{D}_k$}
To complete the interpretation of prime formulas, we would need to show how to encode set membership. Unfortunately, this seems to be impossible in the safe lambda calculus. It would turn to be possible if we had at hand a function $\_ \uparrow^k$, counterpart of $\_ \downarrow_k$, that increases the representation level of a term to level $k$. Here is how we would proceed if such function were representable in the safe lambda calculus.

Firstly, the formulae ``$\mathbf{true} \in y^1$'' and ``$\mathbf{false} \in y^1$'' can be encoded by the safe terms $y^1 (\lambda x^0 . OR^0~x^0) F^0$ and $y^1 (\lambda x^0. OR^0(NOT^0~x^0)) F^0$ respectively.
For the general case ``$x^k\in y^{k+1}$''
we proceed as in \cite{mairson1992spt} by introducing lambda-terms encoding set equality, set membership and subset tests, and we further parameterize these encoding by $n\in\nat$.

Equality of booleans is encoded by:
$$ eq_0^n \equiv \lambda x^0 : \booltype_n .\lambda y^0 : \booltype_n. OR^n (AND^n~x^0~y^0) (AND^n (NOT^n~x^0)(NOT^n~y^0)) \ .$$

We now use variable of type $\Delta_{k+1}^n$ as iterators over list of elements of type $\Delta_k^n$ and we instantiate the type variable $\tau$ as $\booltype_n$ in order to iterate a level-$n$ Boolean function. We define the set membership function as follows. Note that
the level of representation differs from input to output: \begin{align*}
  member_{k+1}^{n+1} &\equiv \lambda x^k : \Delta_k^{n+1}.\lambda y^{k+1}:\Delta_{k+1}^{n+1}. \\
& \qquad (y^{k+1}\downarrow_n) (\lambda y^k : \Delta_k^n . OR^n (eq_k^{n+1}~x^k~(y^k\uparrow^{n+1})))~F^n \\
  & : \Delta_k^{n+1} \typear \Delta_{k+1}^{n+1} \typear \booltype_n
\\
  subset_{k+1}^{n+1} &\equiv \lambda x^{k+1} : \Delta_{k+1}^{n+1}.\lambda y^{k+1}:\Delta_{k+1}^{n+1}. \\
  & \qquad (x^{k+1}\downarrow_n) (\lambda x^k : \Delta_k^n . AND^n (member_{k+1}^{n+1}~x^k~y^{k+1}))~T^n \\
  & : \Delta_{k+1}^{n+1} \typear \Delta_{k+1}^{n+1} \typear\booltype_n
\\
  eq_{k+1}^{n+1} &\equiv \lambda x^{k+1} : \Delta_{k+1}^{n+1}.\lambda y^{k+1}:\Delta_{k+1}^{n+1}. \\
   & \qquad
   (\lambda op:\Delta_{k+1}^n\typear\Delta_{k+1}^n\typear\booltype_n. AND^n (op~x^{k+1}~y^{k+1})(op~y^{k+1}~x^{k+1}))~subset_{k+1}^{n+1} \\
  & : \Delta_{k+1}^{n+1} \typear \Delta_{k+1}^{n+1} \typear \booltype_n
\end{align*}
The terms $eq_{k+1}^n$ and $subset_{k+1}^n$ are safe, and so is $member_{k+1}^n$ thanks to the fact that $y^k$ has a lower representation level than $x^k$.

The formula $x^k\in y^{k+1}$ is then encoded by the term
$$x^k:\Delta_k^n, y^{k+1}:\Delta_{k+1}^{n'}\vdash \left(member_{k+1}^{\min(n,n')} (x^k\downarrow_{\min(n,n')})~(y^{k+1}\downarrow_{\min(n,n')})\right)\downarrow_0$$


\subsection{NP-hardness}
To show NP-hardness it suffices to observe that the encoding of SAT in the simply-typed lambda calculus from the paper\cite{asperti-np} relies only on safe terms.

\subsection{PSPACE-hardness}

We encode QBF into the calculus.
We assume that the quantified propositional formula is given in prenex form:
$$\$_{n-1} x_{n-1} \ldots \$_0 x_0 . \psi(x_0, \ldots, x_{n-1})$$
where $\$_i \in \{\exists,\forall\}$ for $0\leq i\leq n-1$.

The encoding is as follows:
\begin{align*}
\sem{1} &= T^0  : \booltype \\
\sem{0} &= F^0 : \booltype \\
\sem{x_i} &= x_i\downarrow_0 = x_i~T^{i-1}~F^{i-1}\ldots T^1~F^1: \booltype \qquad \hbox{where $x_i:\booltype_i$}\\
\sem{\psi_1\zand \psi_2} &= AND^0~\sem{\psi_1}~\sem{\psi_2}
:\booltype  \\
\sem{\psi_1\zor \psi_2} &= OR^0~\sem{\psi_1}~\sem{\psi_2}
:\booltype  \\
\sem{\neg \psi} &= NOT^0~\sem{\psi}
:\booltype  \\
\sem{\forall x_i.\psi(\ldots, x_i, \ldots)} & = \mathbf{D}_0^i(\lambda x^{\booltype_i} AND^0~\sem{\psi(\ldots, x_i, \ldots)})~T^0 :\booltype\\
\sem{\exists x_i.\psi(\ldots, x_i, \ldots)} & = \mathbf{D}_0^i(\lambda x^{\booltype_i}.OR^0~\sem{\psi(\ldots, x_i, \ldots)})~F^0 :\booltype
\end{align*}
The size of $\sem{\psi}$ is in $\bigo(|\psi|^2)$.

It is easy to check that this encoding is safe.
\begin{example}
  The formula $\forall x \exists y \exists z (x\zor y\zor z)\zand(\neg x\zor \neg y\zor \neg z)$ is represented by the safe term:
\begin{align*}
\vdash &\mathbf{D}_0^2(\lambda x^{\booltype_2}. AND^0\\
&\quad\quad (\mathbf{D}_0^1(\lambda x^{\booltype_1}.OR^0\\
&\quad\quad\quad (\mathbf{D}_0^0(\lambda x^{\booltype_0}.OR^0\\
&\quad\quad\quad\quad (AND^0 (OR^0(OR^0~(x~T^1 F^1 T^0 F^0)~(y~T^0 F^0))z) \\
&\quad\quad\quad\quad\quad (OR^0(OR^0(NEG^0 (x~T^1 F^1 T^0 F^0))(NEG^0 (y~T^0 F^0)))(NEG^0~z))) \\
&\quad\quad\quad )F^0)\\
&\quad\quad)F^0)\\
&\quad) T^0
\end{align*}
\end{example}
This gives us:
\begin{theorem}
  Deciding $\beta\eta$-equality of two terms of the Safe Lambda Calculus is PSPACE-hard.
\end{theorem}

% NP \subseteq PSPACE \subseteq EXP


    \section{Expressivity}
\subsection{Numeric functions representable in the safe lambda
calculus}

Natural numbers can be encoded into the simply-typed lambda calculus
using the Church Numerals: each $n\in\nat$ is encoded into the term
$\encode{n} = \lambda s z. s^n z$ of type $I = ((o,o),o,o)$ where
$o$ is a ground type. In 1976 Schwichtenberg \cite{citeulike:622637}
showed the following:


\begin{theorem}[Schwichtenberg 1976]
The numeric functions representable by simply-typed $\lambda$-terms
of type $I\rightarrow \ldots \rightarrow I$ using the Church Numeral
encoding are exactly the multivariate polynomials \emph{extended
with the conditional function}.
\end{theorem}

If we restrict ourselves to safe terms, the representable functions
are exactly the multivariate polynomials:
\begin{theorem}
\label{thm:polychar} The functions representable by safe
$\lambda$-expressions of type $I\rightarrow \ldots \rightarrow I$
are exactly the multivariate polynomials.
\end{theorem}

\begin{corollary}
The conditional operator $C:I\rightarrow I\rightarrow I \rightarrow
I$ verifying  $C t y z \rightarrow_\beta y$  if $t \rightarrow_\beta
\encode{0}$ and $C t y z \rightarrow_\beta z$ if $t
\rightarrow_\beta \encode{n+1}$ is not definable in the safe
simply-typed lambda calculus.
\end{corollary}
\proof
  Natural numbers are encoded using Church Numerals: $\encode{n} =
  \lambda s z. s^n z$.  Addition: For $n,m \in \nat$, $\encode{n+m} =
  \lambda \alpha^{(o,o)} x^o . (\encode{n} \alpha) (\encode{m} \alpha
  x)$. Multiplication: $\encode{n . m} = \lambda \alpha^{(o,o)}
  . \encode{n} (\encode{m} \alpha)$.  All these terms are safe and
  clearly any multivariate polynomial $P(n_1, \ldots, n_k)$ can be
  computed by composing the addition and multiplication terms as
  appropriate.

For the converse, let $U$ be a safe $\lambda$-term of type
$I\rightarrow I\rightarrow I$.  The generalization to terms of type
$I^n \rightarrow I$ for $n>2$ is immediate (they correspond to
polynomials with $n$ variables). W.l.o.g we can assume that $U =
\lambda x y \alpha z. u$ where $u$ is a safe term of ground type in
$\beta$-normal form with $fv(u) \subseteq \{ x, y : I, z :o, \alpha
: o\rightarrow o \}$.

\emph{Notation:} Let $T$ be a set of terms of type $\tau \rightarrow
\tau$ and $T'$ be a set of terms of type $\tau$ then $T \cdot T'$
denotes the set of terms $\{ s s' : \tau \ | \ s \in T \wedge s' \in
T' \}$. We also define $T^k \cdot T'$ recursively as follows:  $T^0
\cdot T' = T'$ and for $k\geq 0$, $T^{k+1} \cdot T' = T \cdot (T^k
\cdot T')$ ({\it i.e.}~$T^k \cdot T'$ denotes $\{ s_1( \ldots (s_k
s'))  \ | \ s_1, \ldots, s_k \in T \wedge s' \in T' \}$). We define
$T^+\cdot T' = \Union_{k > 0} T^k \cdot T'$ and $T^*\cdot T' =
(T^+\cdot T') \union T'$. For two sets of terms $T$ and $T'$, we
write $T =_\beta T'$ to express that any term of $T$ is
$\beta$-convertible to some term $t'$ of $T'$ and reciprocally.

Let us write $\mathcal{N}^\tau$ for the set of $\beta$-normal terms
of type $\tau$ where $\tau$ ranges in $\{ o, o\rightarrow o, I \}$
and with free variables in $\{ x,y:I, z:o, \alpha:o\rightarrow o\}$.
We write $\mathcal{A}^\tau$ for the subset of $\mathcal{N}^\tau$
consisting of applications only ({\it i.e.}~not abstractions). Let
$B$ be the set of terms of type $(o,o)$ defined by $B = \{ \alpha \}
\union \{ \lambda a.b \ | \ b \in \{a,z\}, a \neq z \}$. It is easy
to see that the following equations hold:
\begin{eqnarray*}
\mathcal{A}^I &=& \{ x,y \} \\
\mathcal{N}^{(o,o)} &=& B \union \mathcal{A}^I \cdot
\mathcal{N}^{(o,o)} = (\mathcal{A}^I)^* \cdot B \\
\mathcal{A}^{(o,o)} &=& \{ \alpha \} \union (\mathcal{A}^I)^+ \cdot B \\
\mathcal{A}^o = \mathcal{N}^o &=& \{ z \} \union \mathcal{A}^{(o,o)} \cdot \mathcal{N}^o = (\mathcal{A}^{(o,o)})^* \cdot \{ z \}
\end{eqnarray*}
Hence $\mathcal{A}^o = \left( \{\alpha \} \union \{x,y\}^+ \cdot
\left( \{\alpha \} \union \{\lambda a.b \ | \ b \in \{a,z\}, a \neq
z \} \right) \right)^* \cdot \{ z \}$. Since $u$ is safe, it cannot
contain terms of the form $\lambda a . z$ with $a \neq z$ occurring
at an operand position, therefore since $u$ belongs to
$\mathcal{A}^o$ we have:
\begin{equation}
u \in \left( \{\alpha\} \union \{x,y\}^+ \cdot \{\alpha,
\underline{i} \} \right)^* \cdot \{ z \} \label{eqn:u}
\end{equation}
where $\underline{i}$ is the identity term of type $o\rightarrow o$.


We observe that $\encode{k} \underline{i} =_\beta \underline{i}$ for
all $k \in \nat$ and for $l\geq 1$, for all $k_1, \ldots k_l \in
\nat$, $\encode{k_1}\ldots \encode{k_l} \alpha =_\beta
\encode{k_1\times \ldots \times k_l} \alpha$. Hence for all $m,n \in
\nat$ we have:
\begin{equation}
\begin{array}{llr}
\{\encode{m},\encode{n}\}^+ \cdot \{\alpha, \underline{i} \} &=_\beta
\{ \underline{i} \} \union
\{ \encode{m^i n^j} \alpha \ |\ i+j \geq 1 \} \nonumber \\
&= \{ \encode{m^i n^j} \alpha \ |\ i,j \geq 0 \} & ( \mbox{since } \underline{i} = \encode{0} \alpha) \end{array}
\label{eqn:intermediate}
\end{equation}
therefore:
$$\begin{array}{llr}
u[\encode{m}, \encode{n}/x,y] &\in \left( \{ \alpha \} \union \{\encode{m},\encode{n}\}^+ \cdot \{\alpha, \underline{i} \} \right)^* \cdot \{ z \}  & \mbox{(by Eq.\ \ref{eqn:u})} \\
&=_\beta \left( \{\alpha \} \union \{ \encode{m^i n^j}
\alpha \ | \ i,j \geq 0 \} \right)^* \cdot \{ z \} & \mbox{(by Eq.\ \ref{eqn:intermediate})}  \\
&=_\beta \left\{ \encode{m^i n^j}
\alpha \ | \ i,j \geq 0 \right\}^* \cdot \{ z \} & \mbox{($\alpha z =_\beta \encode{1} \alpha z$)}.
\end{array}$$

Furthermore, for all $m,n,r,i,j\in \nat$ we have $\encode{m^i n^j}
\alpha (\alpha^r z) =_\beta \alpha^{r + m^i n^j} z$, hence
$u[\encode{m} \encode{n}/x,y] =_\beta \alpha^{p(m,n)} z$ where
$p(m,n) = \sum_{0\leq k \leq d} m^{i_k} n^{j_k}$ for some $i_k,j_k
\geq 0$, $k \in\{ 0,..,d \}$ and $d\geq 0$. Thus $U \encode{m}
\encode{n} =_\beta \encode{p(m,n)}$. \qed


For instance, the term $ C = \lambda F G H \alpha x . H (
\underline{\lambda y . G \alpha x} ) (F \alpha x)$ used by
Schwichtenberg \cite{citeulike:622637} to define the conditional
operator is unsafe since the underlined subterm is of order $1$,
occurs at an operand position and contains an occurrence of $x$ of
order $0$.



    %% chapter from the transfer thesis
    \input{transfer_chap_safe_homog.texi}
    \section{Safe $\lambda$-Calculus without the Homogeneity Constraint}
\label{sec:safe_nonhomog}


In section \ref{sec:safe_homog}, we have presented a version of the
safe lambda calculus where types are required to be homogeneous. We
now give a more general version of the safe simply-typed
$\lambda$-calculus where type homogeneity is not required.

\subsection{Rules}

We use a set of sequents of the form $\Gamma \vdash M : A$ where
$\Gamma$ is the context of the term and $A$ is its type. Let
$\Sigma$ be a set of higher-order constants. We call safe terms any
simply-typed lambda term that is typable within the following system
of formation rules:
$$ \rulename{var} \   \rulef{}{x : A\vdash x : A}
\qquad  \rulename{const} \   \rulef{}{\vdash f : A} \quad f \in \Sigma
\qquad  \rulename{wk} \   \rulef{\Gamma \vdash M : A}{\Delta \vdash M : A} \quad \Gamma \subset \Delta$$

$$ \rulename{app} \  \rulef{\Gamma \vdash M : (A,\ldots,A_l,B)
                                        \qquad \Gamma \vdash N_1 : A_1
                                        \quad \ldots \quad \Gamma \vdash N_l : A_l  }
                                   {\Gamma  \vdash M N_1 \ldots N_l : B}
                                    \quad
                                   \forall y \in \Gamma : \ord{y} \geq \ord{B}$$

$$ \rulename{abs} \   \rulef{\Gamma \union \overline{x} : \overline{A} \vdash M : B}
                                   {\Gamma  \vdash \lambda \overline{x} : \overline{A} . M : (\overline{A},B)} \qquad
                                   \forall y \in \Gamma : \ord{y} \geq \ord{\overline{A},B}$$


Remark:
\begin{itemize}
\item $(\overline{A},B)$ denotes the type $(A_1,A_2, \ldots, A_n, B)$;
\item all the types appearing in the rule are not required to be homogeneous (for instance
it is possible to have $\ord{A_l} < \ord{B}$ in rule $\rulename{app}$) ;
\item the environment $\Gamma \union \overline{x}:\overline{A}$ is not stratified, in particular, variables in $\overline{x}$ do not necessarily have the same order;
\item in the abstraction rule, the side-condition imposes that at least all variables of the lowest order
in the context are abstracted. Variables of greater order can also be
abstracted together with the lowest order variables and, in contrast to
the homogeneous safe lambda calculus, there is no constraint on the
order in which these variables are abstracted;
\end{itemize}

\begin{example}
For $x:o$, $f:(o,o)$ and $\varphi:((o,o),o)$ the term $$\vdash \lambda x f \varphi .
\varphi : (o , (o, o) , ((o,o),o) , (o,o),o)$$ is
a valid safe term that is not homogeneously typed.
\end{example}

\begin{example}
For $x:o$, $g:(o,(o,o),o)$, the term $\vdash \lambda g x . g x$ is unsafe and not homogeneously typed
and the term $\lambda g x . g x (\lambda x . x)$ is safe and not homogeneously typed.
\end{example}

Side-remark: safety is preserved by full $\eta$-expansion. Indeed,
consider the safe term $\Gamma \vdash M:(A_1,\ldots,A_l,o)$ where
$(A_1,\ldots,A_l,o)$ is not necessarily homogeneous. Its full $\eta$-expansion
is $\lambda x_1 .. x_l . M x_1 \dots x_l$ for some variables
$x_1:A_1, \ldots, x_l:A_l$ fresh in $M$. For all $i \in 1..l$ we
have $\Gamma, \Sigma \vdash x_i :A_i$ where $\Sigma = \{ x_1:A_1,
\cdots x_l :A_l \}$. Applying $\rulename{app}$ we obtain $\Gamma,
\Sigma \vdash M x_1 \ldots x_l$ and by the (abs) rule we get
$$\Gamma \vdash \lambda x_1:A_1 \ldots x_l:A_l .M x_1 \ldots x_l.$$

\begin{lemma}[Context reduction]
\label{lem:nonhomosafe_basic_prop}
If $\Gamma \vdash M : B$ is a valid judgment then
\begin{enumerate}
\item $fv(M) \vdash M : B$
\item every variable in $\Gamma$ \emph{occurring free in $M$} has order at
least $ord(M)$.
\end{enumerate}
where $fv(M)$ denotes the context constituted of the free variables occurring in $M$.
\end{lemma}
\begin{proof}
(i) Suppose that some variable $x$ in $\Gamma$ does not occur free
in $M$, then necessarily $x$ has been introduced in the context
using the weakening rule. Hence $\Gamma\setminus \{ x \} \vdash M$
must also be typable. (ii) An easy structural induction.
\end{proof}

The converse of this lemma is not true: consider the simply-typed
term $\lambda y z. (\lambda x . y ) z$ with $x,y,z:o$. This term is
closed therefore it satisfies property (i) and (ii) of lemma
\ref{lem:nonhomosafe_basic_prop}. However it is not typable by the
rules of the safe lambda-calculus since the subterm $\lambda x .y$
is not safe.

\subsection{Substitution in the safe lambda calculus}

The traditional notion of substitution, on which the
$\lambda$-calculus is based, is defined as follows:
\begin{definition}[Substitution]
\label{dfn:subst}
\begin{eqnarray*}
c \subst{t}{x} &=& c \quad \mbox{where $c$ is a $\Sigma$-constant},\\
x \subst{t}{x} &=& t\\
 y\subst{t}{x} &=& y \quad \mbox{for } x \not \neq y,\\
(M_1 M_2) \subst{t}{x} &=& (M_1 \subst{t}{x}) (M_2 \subst{t}{x})\\
(\lambda x . M) \subst{t}{x} &=& \lambda x . M\\
(\lambda y . M) \subst{t}{x} &=& \lambda z . M \subst{z}{y}
\subst{t}{x} \mbox{where $z$ is a fresh variable and $x\not = y$}.
\end{eqnarray*}
\end{definition}

In the setting of the safe lambda calculus, the notion of
substitution can be simplified. Indeed, similarly to what we observe
in the homogeneous safe $\lambda$-calculus, we remark that for safe
$\lambda$-terms there is no need to rename variables when performing
substitution:

\begin{lemma}[No variable capture lemma]
\label{lem:noclash} There is no variable capture when performing
substitution on a safe term.
\end{lemma}

This is the counterpart of lemma \ref{lem:homog_nocapture}. The
proof (which does not rely on homogeneity) is the same.
Consequently, in the safe lambda calculus setting, we can omit to
rename variable when performing substitution. The equation
$$(\lambda x . M) \subst{t}{y} = \lambda z . M \subst{z}{x}
\subst{t}{y} \mbox{where $z$ is a fresh variable}$$ becomes
$$(\lambda x . M) \subst{t}{y} = \lambda x . M \subst{t}{y}.$$

Unfortunately, this notion of substitution is still not adequate for
the purpose of the safe simply-typed lambda calculus. The problem is
that performing a single $\beta$-reduction on a safe term will not
necessarily produce another safe term.

The solution consists in reducing several consecutive $\beta$-redex
at the same time until we obtain a safe term. To achieve this, we
introduce the \emph{simultaneous substitution}, a generalization of
the standard substitution given in definition \ref{dfn:subst}.

\begin{definition}[Simultaneous substitution]
\label{dnf:simsubst}
 The expression $\subst{\overline{N}}{\overline{x}}$ is an abbreviation for $\subst{N_1 \ldots N_n}{x_1
\ldots x_n}$:
\begin{eqnarray*}
c \subst{\overline{N}}{\overline{x}} &=& c \quad \mbox{where $c$ is a $\Sigma$-constant},\\
x_i \subst{\overline{N}}{\overline{x}} &=& N_i\\
 y \subst{\overline{N}}{\overline{x}} &=& y \quad \mbox{ if } y \not \neq x_i \mbox{ for all } i,\\
(M N) \subst{\overline{N}}{\overline{x}} &=& (M \subst{\overline{N}}{\overline{x}}) (N \subst{\overline{N}}{\overline{x}}) \\
(\lambda x_i . M) \subst{\overline{N}}{\overline{x}} &=& \lambda x_i
. M
\subst{N_1 \ldots N_{i-1} N_{i+1}\ldots N_n}{x_1 \ldots x_{i-1} x_{i+1}\ldots x_n} \\
(\lambda y . M)
\subst{\overline{N}}{\overline{x}} &=& \lambda z . M \subst{z}{y} \subst{\overline{N}}{\overline{x}} \\
&& \mbox{where $z$ is a fresh variables and } y \neq x_i \mbox{ for
all } i.
\end{eqnarray*}
\end{definition}

In general, variable capture should be avoided, this explains why
the definition of simultaneous substitution uses auxiliary fresh
variables. However in the current setting, lemma \ref{lem:noclash}
can clearly be transposed to the simultaneous substitution,
therefore there is no need to rename variables.

The notion of substitution that we need is therefore the
\emph{capture-permitting simultaneous substitution} defined as
follows:

\begin{definition}[Capture-permitting simultaneous substitution]
 We use the notation
$\subst{\overline{N}}{\overline{x}}$ for $\subst{N_1 \ldots N_n}{x_1
\ldots x_n}$:
\begin{eqnarray*}
c \subst{\overline{N}}{\overline{x}} &=& c \quad \mbox{where $c$ is a $\Sigma$-constant},\\
 x_i \subst{\overline{N}}{\overline{x}} &=& N_i\\
 y \subst{\overline{N}}{\overline{x}} &=& y \quad \mbox{where } x \not \neq y_i \mbox{ for all } i,\\
(M_1 M_2) \subst{\overline{N}}{\overline{x}} &=& (M_1 \subst{\overline{N}}{\overline{x}}) (M_2 \subst{\overline{N}}{\overline{x}})\\
(\lambda x_i . M) \subst{\overline{N}}{\overline{x}} &=& \lambda x_i
. M
\subst{N_1 \ldots N_{i-1} N_{i+1}\ldots N_n}{x_1 \ldots x_{i-1} x_{i+1}\ldots x_n} \\
(\lambda y . M) \subst{\overline{N}}{\overline{x}} &=& \lambda y . M
\subst{\overline{N}}{\overline{x}} \mbox{where $y \not = x_i$ for
all $i$}. \qquad \mathbf{(\star)}
\end{eqnarray*}
The symbol $\mathbf{(\star)}$ identifies the equation which has
changed compared to the previous definition.
\end{definition}

\begin{lemma}[Substitution preserves safety]
\label{lem:subst_preserve_i}
$$ \Gamma\union \overline{x} : \overline{A}\vdash M : T
\quad \mbox{and} \quad \Gamma \vdash N_k : B_k \mbox{, } k \in
1..n \qquad \mbox{ implies } \qquad \Gamma \vdash
M[\overline{N}/\overline{x}] : T$$
\end{lemma}

\begin{proof}
Suppose that $\Gamma \union \overline{x}: \overline{A} \vdash M :T$ and
$\Gamma \vdash N_k : B_k$ for $k \in 1..n$.

We prove $\Gamma \vdash M[\overline{N}/\overline{x}]$ by induction
on the size of the proof tree of $\Gamma\union
\overline{x}:\overline{A} \vdash M : T$ and by case analysis on the
last rule used. We only give the proof for the abstraction case. If
$\Gamma \union \overline{x}:\overline{A} \vdash \lambda \overline{y}
: \overline{C}. P : (\overline{C}|D)$ where $\Gamma\union
\overline{x}:\overline{A}\union \overline{y}:\overline{C} \vdash P :
D$, then by the induction hypothesis $\Gamma\union
\overline{y}:\overline{C} \vdash P\subst{\overline{N}}{\overline{x}}
: D$. Applying the rule $\rulename{abs}$ gives $\Gamma \vdash
\lambda \overline{y}:\overline{C} . P
\subst{\overline{N}}{\overline{x}}$.
\end{proof}

\subsection{Safe-redex}
In the simply-typed lambda calculus a redex is a term of the form
$(\lambda x . M) N$. We generalize this definition to the safe
lambda calculus:
\begin{definition}[Safe redex]
We call safe redex a term of the form $(\lambda \overline{x} . M)
N_1 \ldots N_l$ such that:
\begin{itemize}
\item $ \Gamma \vdash (\lambda \overline{x} . M) N_1 \ldots N_l $;
\item the variable $\overline{x}=x_1\ldots x_n$ are abstracted altogether by one occurrence of the rule $\rulename{abs}$ in the proof
tree;
\item the terms $(\lambda \overline{x} . M)$, $N_1$, $N_l$ are applied together at once using the $\rulename{app}$ rule :
$$   \rulef{
            \Sigma \vdash \lambda \overline{x} . M
            \quad
            \Sigma \vdash N_1         \quad \ldots \quad \Sigma \vdash N_l
    }
    {
       \Sigma \vdash (\lambda \overline{x} . L) N_1 \ldots N_l
    } (\mathbf{app})
$$
and consequently each $N_i$ is safe;
\end{itemize}
\end{definition}

The relation $\beta_s$ is defined exactly the same way as in the homogeneous safe $\lambda$-calculus. The safe $\beta$-reduction $\betasred$ is defined as the closure of $\beta_s$ by
compatibility with the formation rules of the safe
$\lambda$-calculus.  It is straightforward to show, as we did for the homogeneous safe $\lambda$-calculus, that $\betasred \subset \betaredtr$.


\begin{lemma}
\label{lem:safereduction} A safe redex reduces to a safe term.
\end{lemma}

This lemma, which is a consequence of lemma
\ref{lem:subst_preserve_i}, is the counterpart of lemma
\ref{lem:homoh_safered_preserve_safety} in the homogeneous safe
lambda calculus. Their proofs are identical.


\subsection{Particular case of homogeneously-safe lambda terms}

In this section, we derive a new set of rules by adding the type-homogeneity restriction to the non-homogenous safe lambda calculus.

We recall the definition of type-homogeneity from section
\ref{sec:safe_homog}: a type $(A_1, A_2, \ldots A_n, o)$ is said to
be homogeneous whenever $\ord{A_1} \geq \ord{A_2} \geq \ldots \geq
\ord{A_n}$ and each of the $A_i$ is homogeneous. A term is said to
be homogeneous if its type is homogeneous.

We now impose type-homogeneity to all the sequents present in the
rules of the safe $\lambda$-calculus: we say that a term is
\emph{homogeneously-safe} if there is a proof tree showing its
safety in which all sequents are of homogenous type. Consequently a
homogeneously-safe term is safe and has an homogenous type.

We say that $\Gamma \vdash M : A$ verifies $P_i$ for $i \in \zset$
if all the variables in $\Gamma$ have order at least $\ord{A}+i$.
Lemma \ref{lem:nonhomosafe_basic_prop} can then be restated as
follows:
\begin{lemma}[Context reduction]
\label{lem:context_reduction} If $\Gamma \vdash M : A$ then the sequent $fv(M) \vdash M : A$ is valid and satisfies $P_0$.
\end{lemma}


We now prove that if we impose the homogeneity of types, the set of
rules of the non-homogenous safe $\lambda$-calculus and the rules of
table \ref{tab:homosafelmd_rules_refined} are equivalent.  We recall
that in the system of rules of table
\ref{tab:homosafelmd_rules_refined}, if the sequent $\Gamma
\vdash^{i} M : A$ is valid for some $i \in \zset$ then all the
variables in $\Gamma$ have orders at least $\ord{A}+i$.

\begin{proposition}[Homogeneity restriction]
\label{prop:nonhomogsafe_homog_restriction}
Let $k \in \{ 0, -1 \}$. The sequent $\Gamma \vdash M : A$ is valid, homogeneously-safe and satisfies $P_k$
if and only if the sequent $\Gamma \vdash^k M : A$ is valid in the system of rules of table \ref{tab:homosafelmd_rules_refined}.
\end{proposition}

\begin{proof}
\emph{If}: An easy induction by case analysis on the last rule used to derive $\Gamma \vdash^0 M : A$.

\emph{Only if}:
Consider an homogeneously-safe term $\Gamma \vdash S : T$ satisfying $P_0$.
We proceed by induction and case analysis on the last rule used to derive $\Gamma \vdash S : T$.
We only give the details for the application and abstraction
case:
\begin{itemize}
\item \textbf{Abstraction.} We recall the abstraction rule:
$$ \rulename{abs} \quad  \rulef{\Gamma \union \overline{x} : \overline{A} \vdash M : B}
                                   {\Gamma  \vdash \lambda \overline{x} : \overline{A} . M : (\overline{A},B)} \qquad
                                   \forall y \in \Gamma : \ord{y} \geq \ord{\overline{A},B}$$

Type homogeneity requires that for all $i$: $\ord{x_i} = \ord{A_i} \geq
\ord{B} -1$. Therefore the premise of the rule verifies $P_{-1}$. Using the induction hypothesis we have:
\begin{equation}
\Gamma \union \overline{x} : \overline{A} \vdash^{-1} M : B. \label{eq:prop:nonhomogsafe_homog_restriction:abs1}
\end{equation}

We now partition the context $\Gamma$ according to the order of
the variables. The partitions are written in decreasing order of
type order. The notation $\Gamma | \overline{x}:\overline{A}$ means
that $\overline{x}:\overline{A}$ is the lowest partition of the
context.
We also use the notation $(\overline{A}|B)$ to denote the
homogeneous type $(A_1, A_2, \ldots A_n, B)$ where $\ord{A_1} =
\ord{A_2} =  \ldots \ord{A_n} \geq \ord{B} -1$.


Suppose that we abstract a single variable $x$, then in order to
respect the side condition, we need to abstract all variables of
order less or equal to $\ord{x}$. In particular we need to abstract
the partition of the order of $x$. Moreover to respect type
homogeneity, we need to abstract variables of the lowest order
first.

Hence $\overline{x}$ must contain at least the lowest variable
partition (all the variables of the lowest order). If $\overline{x}$
contains variables of different order, then the instance of the
abstraction rule can be replaced by consecutive instances of the
abstraction rule, one for each of the different variable order in
$\overline{x}$. Therefore, without loss of generality, we can assume
that $\overline{x}$ only contains the lowest partition, that is to
say, $\overline{x}$ \emph{is} the lowest partition.

The sequent \ref{eq:prop:nonhomogsafe_homog_restriction:abs1} therefore becomes:
$$\Gamma | \overline{x} : \overline{A} \vdash^{-1} M : B.$$

We conclude by applying the abstraction rule of table
\ref{tab:homosafelmd_rules_refined}:
$$ \rulename{abs} \quad  \rulef{\Gamma| \overline{x} : \overline{A} \vdash^{-1} M : B}
                                   {\Gamma  \vdash^{0} \lambda \overline{x} : \overline{A} . M : (\overline{A}|B)}$$



\item \textbf{Application.} We recall the application rule:
$$ \rulename{app} \  \rulef{\Gamma \vdash M : (A,\ldots,A_l,B)
                                        \qquad \Gamma \vdash N_1 : A_1
                                        \quad \ldots \quad \Gamma \vdash N_l : A_l  }
                                   {\Gamma  \vdash M N_1 \ldots N_l : B}
                                    \quad
                                   \forall y \in \Gamma : \ord{y} \geq \ord{B}$$

The term in the conclusion is homogeneously safe therefore the term in the first premise must be of homogeneous \
type. This implies that $\ord{A_1} \geq \ldots \geq \ord{A_l}
\geq \ord{B} - 1$.
Furthermore, we can make the assumption that $\ord{A_1} = \ldots = \ord{A_l} = \ord{\overline{A}}$
(it is always possible to replace an instance of the application rule
by several consecutive instances of this kind).

By lemma \ref{lem:context_reduction}, we have for all $i \in 1..l$:
$$fv(N_i) \vdash N_i : A_i \mbox{ is valid and satisfies } P_0.$$

Let $\Sigma = \Union_{i=1..p} fv(N_i)$. Since $\ord{A_1} = \ldots = \ord{A_l}$, by applying the weakening rule we get for all $i\in 1..p$:
$$\Sigma \vdash N_i : A_i \mbox{ is valid and satisfies } P_0.$$


Applying lemma \ref{lem:context_reduction} to the term $M$ we have:
$$fv(M) \vdash M : (A_1,\ldots,A_l,B) \mbox{ is valid and satisfies } P_0.$$

The weakening rule $\rulename{wk}$ then gives:
$fv(M) \union \Sigma \vdash M : (A_1,\ldots,A_l,B)$.
Since $\Sigma \vdash N_i : A_i$ satisfies $P_0$, for any
$z \in \Sigma$ we have $\ord{z} \geq \ord{A_i} = \ord{(A_1,\ldots,A_l,B)} - 1$.
Hence:
\begin{equation}
fv(M) \union \Sigma \vdash M : (A_1,\ldots,A_l,B) \mbox{ is valid
and satisfies } P_{-1}
\label{eq:prop:nonhomogsafe_homog_restriction:m}.
\end{equation}

Similarly, for all $i \in 1..p$, the weakening rule gives $fv(M) \union \Sigma \vdash N_i : A_i$.
Since $fv(M) \vdash M : (A_1,\ldots,A_l,B)$ satisfies $P_0$,
for any $z \in fv(M)$ we have $\ord{z} \geq \ord{M} \geq \ord{A_i}$. Hence:
\begin{equation}
fv(M) \union \Sigma \vdash N_i : A_i \mbox{ is valid and satisfies }
P_0 \label{eq:prop:nonhomogsafe_homog_restriction:ni}.
\end{equation}

Let us define the context $\Sigma' = fv(M) \union \Sigma$. Using the induction hypothesis on equation
\ref{eq:prop:nonhomogsafe_homog_restriction:m} and \ref{eq:prop:nonhomogsafe_homog_restriction:ni} we have:
$$
\Sigma' \vdash^{-1} M : (A_1,\ldots,A_l,B) \qquad \mbox{and} \qquad
\Sigma' \vdash^0 N_i : A_i \mbox{ for all } i \in 1..l.
$$


We consider the following two sub-cases:
\begin{itemize}
\item If $A_1, \ldots, A_l$ forms a type partition then we can apply
rule $\rulename{app}$ of table \ref{tab:homosafelmd_rules_refined}:

$$ \rulef{\Sigma' \vdash^{-1} M : \overline{A} | B
                                        \qquad \Sigma' \vdash^{0} N_1 :
                                        A_1
                                        \quad \ldots \quad \Sigma' \vdash^{0} N_l :
                                        A_l
                                        \quad l = |\overline{A}|
                                        }
                                   {\Sigma'  \vdash^{0} M N_1 \ldots N_l : B} \quad  \rulename{app}
$$
where $\overline{A} = A_1, \ldots, A_l$.

\item  Suppose that $A_1, \ldots, A_l$ does not form a type partition, then we
have $$\ord{A_1} = \ldots = \ord{A_l} = \ord{B} - 1.$$

The side condition in the original instance of the application rule
says that for any variable $y$ in $\Gamma$ we have
$$\ord{y} \geq \ord{B} = 1 + \ord{A_l} = \ord{(A_1,\ldots, A_l,B)} = \ord{M}.$$

In particular the variables in $\Sigma' \subseteq \Gamma$ are of order greater than $\ord{M}$ and consequently
the sequent $\Sigma' \vdash M : (A,\ldots,A_l,B)$ verifies $P_0$. The induction hypothesis then gives:
$$\Sigma' \vdash^0 M : (A,\ldots,A_l,B)$$

By using $l$ consecutive instances of the rules $\rulename{app^+}$ from table \ref{tab:homosafelmd_rules_refined} we get:
$$  \rulef{ \rulef{ \rulef{ \Sigma' \vdash^0 M : (A_1,\ldots, A_l,B)
                    \qquad \Sigma'\vdash^{0} N_1 : A_1
                    }{ \Sigma' \vdash^0 M N_1 : (A_2,\ldots, A_l,B)} \quad \rulename{app^+}
          }
          { \vdots
          }
          \quad \rulename{app^+}
       }
       { \Sigma'  \vdash^{0} M N_1 \ldots N_l : B } \quad \rulename{app^+}
$$
\end{itemize}

In both cases we have proved that $\Sigma'  \vdash^{0} M N_1 \ldots N_l : B$ is a valid sequent.

Clearly $\Sigma' \subseteq \Gamma$ since $fv(M) \subseteq \Gamma$ and $\Sigma' = \Union_{i\in1..l} fv(N_i) \subseteq \Gamma$.
Suppose that $\Sigma' = \Gamma$ then the proof is done.
Suppose that $\Sigma' \subset \Gamma$, then the side condition in the original instance of the application rule says that all
the variables in $\Gamma$ have order
greater or equal to $\ord{B}$, we can therefore apply the weakening rule $\rulename{wk^0}$
of table \ref{tab:homosafelmd_rules_refined} exactly $|\Gamma\setminus \Sigma'|$ times and get:
$$ \rulef{\Sigma'  \vdash^{0} M N_1 \ldots N_l : B}
                                   {\Gamma  \vdash^{0} M N_1 \ldots N_l : B} \quad
                                   \rulename{wk^0}.
$$


\end{itemize}
\end{proof}


\subsection{Examples}
\subsubsection{Example 1}
Let $f,g:o\rightarrow o$, $x,y:o\rightarrow o$, $\Gamma =
g:o\rightarrow o$ and $\Gamma' = g:o\rightarrow o, y:o$. The term
$(\lambda f x . x) g y $ is safe. One possible proof tree is:
$$ \rulef{
        \rulef{
            \rulef{
                \rulef{\vdots}{\Gamma \vdash \lambda f x. x}      \qquad \axiomf{\Gamma \vdash g} }
            {\Gamma \vdash (\lambda f x. x) g} \rulename{app}
        }
        { \Gamma' \vdash (\lambda f x. x) g } \rulename{wk}
        \qquad \axiomf{\Gamma' \vdash y}
    }
    { \Gamma' \vdash (\lambda f x. x) g y } \rulename{app}
$$
Here is another proof for the same judgment:
$$ \rulef{  \rulef{ \rulef{\vdots}{\Gamma \vdash \lambda f x. x} }{\Gamma' \vdash \lambda f x. x} \rulename{wk}    \qquad \rulef{}{\Gamma' \vdash g} \qquad \rulef{}{\Gamma' \vdash y}}
    {\Gamma' \vdash (\lambda f x. x) g y } \rulename{app}$$

We see on this particular example that there may exist different
proof trees deriving the same judgment.

\subsubsection{Example 2 - Damien Sereni's SCT counter-example}
In \cite{serenistypesct05}, the following counter-example is given
to show that not all simply-typed terms are size-change terminating
(see \cite{jones01} for a definition of size-change termination):

$$ E =  (\lambda a . a (\lambda b . a (\lambda c d .d))) (\lambda e . e (\lambda f .f))$$
where:
\begin{eqnarray*}
a &:& \sigma \typear \mu \typear \mu \\
b &:& \tau \typear \tau \\
c &:& \tau \typear \tau \\
d &:& \mu \\
e &:& \sigma = (\tau \typear \tau) \typear \mu \typear \mu \\
f &:& \tau
\end{eqnarray*}
and $\tau$, $\mu$ and $\sigma$ are type variables.

This example shows that the rules of the safe $\lambda$-calculus
without the homogeneity restriction generates a class of terms that
strictly contains the class of terms generated by the rules of the
homogeneous safe $\lambda$-calculus of section \ref{sec:safe_homog}.

Indeed, for $E$ to be an homogeneous safe lambda term, in the sense
of the rules of section \ref{sec:safe_homog}, $\tau$ and $\mu$ must
be homogeneous types and the variables $a,b,c,d,e,f$ must be
homogeneously typed. This implies that $ \ord{\tau} \geq
\ord{\mu}-1$. Conversely, if this condition is met then $\vdash E :
\mu \typear \mu$ is a valid judgement of the \emph{homogeneous} safe $\lambda$-calculus.

In the safe $\lambda$-calculus \emph{without} the homogeneity
constraint, however, the judgement $\vdash E : \mu \typear \mu$ is
always valid whatever the types $\mu$ and $\tau$ are.



    \section{Safe PCF}
        \subsection{Definition and properties}
        \subsection{Game-semantic analysis via a syntactic argument}

    %\section{Safe IA}
    \input{sec_safeia.texi}


\chapter{Local Computation of \texorpdfstring{$\beta$}{Beta}-Reduction}
    \label{chap:localbeta}
    We make an explicit correspondence between the game denotation of a
term and its syntax. Our approach follows ideas recently introduced
in \cite{OngLics2006}, mainly the notion of computation tree of a
simply-typed $\lambda$-term and traversals over the computation
tree. A computation tree can be regarded as an abstract syntax tree
(AST) of the $\eta$-long normal form of a term. A traversal is a
justified sequence of nodes of the computation tree respecting some
formation rules. Traversals are used to describe computations. An
interesting property is that the \emph{P-view} of a traversal
(computed in the same way as P-view of plays in Game Semantics) is a
path in the computation tree.

The main result of this paper is called the
\emph{Correspondence Theorem} (theorem \ref{thm:correspondence}). It
states that traversals over the computation tree are just
representations of the uncovering of plays in the
strategy-denotation of the term. Hence there is an isomorphism
between the strategy denotation of a term and its revealed game
denotation ({\it i.e.}~its strategy denotation where internal moves are
not hidden after composition). This theorem permits us to explore
the effect that a given syntactic restriction (such as the safety restriction) has on the strategy
denoting a term.

To really make use of the Correspondence Theorem, it will be
necessary to restate it in the standard game-semantic framework in
which internal moves are hidden. For that purpose, we will define a
\emph{reduction} operation on traversals responsible of eliminating
the ``internal nodes'' of the computation. This leads to a
correspondence between the standard game denotation of a term and
the set of reductions of traversals over its computation tree.
Fortunately, the reduction operation preserves the good properties
of traversals. This is guaranteed by the facts that the P-view of
the reduction of a traversal is equal to the reduction of the P-view
of the traversal, and the O-view of a traversal is the same as the
O-view of its reduction (lemma \ref{lem:redtrav_trav}). \vspace{8pt}

\emph{Related works}: Traversals of a computation tree provide a way
to perform \emph{local computation} of $\beta$-reductions as opposed
to a global approach where the $\beta$-reduction is implemented by
performing substitutions. A notion of local computation of
$\beta$-reduction has been investigated in
\cite{DanosRegnier-Localandasynchronou} through the use of special
graphs called ``virtual nets'' that embed the lambda-calculus.

In \cite{DBLP:conf/lics/AspertiDLR94}, a notion of graph based on
Lamping's graphs \citep{lamping} is introduced to represent
$\lambda$-terms. The authors unify different notions of paths
(regular, legal, consistent and persistent paths) that have appeared
in the literature as ways to implement graph-based reduction of
lambda-expressions. We can regard a traversal as an alternative
notion of path adapted to the graph representation of
$\lambda$-expressions given by computation trees.



%Is there any unsafe term whose game semantics is a strategy where
%pointers can be recovered?
%
%The answer is yes: take the term $T_i = (\lambda x y . y) M_i S$
%where $i =1..2$ and $\Gamma \vdash_s S : A$. $T_1$ and $T_2$ both
%$\beta$-reduce to the safe term $S$, therefore
%$\sem{T_1}=\sem{T_2}=\sem{S}$. But $T_1$ is safe whereas $T_2$ is
%unsafe. Since it is possible to recover the pointer from the game
%semantics of $S$, it is as well possible to recover the pointer from
%the semantics of $T_2$ which is unsafe.

\section{Computation tree}
We work in the general setting of the simply-typed
$\lambda$-calculus extended with a fixed set $\Sigma$ of
higher-order uninterpreted constants \footnote{A constant $f$ is
  \emph{uninterpreted} if the small-step semantics of the language
  does not contain any rule of the form $f \dots \rightarrow e$. $f$
  can be regarded as a data constructor.}

For the rest of the section we fix a simply-typed term $\Gamma \vdash M :T$.

\subsection{$\eta$-long normal form}

The $\eta$-long normal form appeared in
\citep{DBLP:journals/tcs/JensenP76} and
\citep{DBLP:journals/tcs/Huet75} under the names \emph{long reduced
form} and \emph{$\eta$-normal form} respectively. It was then
investigated in \citep{huet76} under the name \emph{extensional
form}.

The $\eta$-expansion of $M: A\typear B$ is defined to be the term
$\lambda x . M x : A\typear B$ where $x:A$ is a fresh variable. A
term $M : (A_1,\ldots,A_n,o)$ can be expanded in several steps into
$\lambda \varphi_1 \ldots \varphi_l . M \varphi_1 \ldots \varphi_l$
where the $\varphi_i:A_i$ are fresh variables. The $\eta$-normal
form of a term is obtained by hereditarily $\eta$-expanding every
subterm occurring at an operand position.

\begin{definition}[$\eta$-long normal form]
A simply-typed term is either an abstraction or it can be written uniquely as
$s_0 s_1 \ldots s_m$ where $m\geq0$ and $s_0$ is a variable, a $\Sigma$-constant or an abstraction.
The $\eta$-long normal form of a term $t$, written $\elnf{t}$ or sometimes $\etanf{t}$,
is defined as follows:
\begin{align*}
\elnf{\lambda x . s } &= \lambda x . \elnf{s} \\
\elnf{\alpha s_1 \ldots s_m : (A_1,\ldots,A_n,o)} &= \lambda \overline{\varphi} . \alpha \elnf{s_1}\ldots \elnf{s_m} \elnf{\varphi_1} \ldots \elnf{\varphi_n}
& \mbox{with $m,n\geq0$}\\
\elnf{(\lambda x . s) s_1 \ldots s_p : (A_1,\ldots,A_n,o) } &= \lambda \overline{\varphi} . (\lambda x . \elnf{s}) \elnf{s_1} \ldots \elnf{s_p} \elnf{\varphi_1} \ldots \elnf{\varphi_n}
& \mbox{with $p\geq 1,n\geq 0$}
\end{align*}
where $x$ and each $\varphi_i : A_i$ are variables and $\alpha$ is
either a variable or a constant.
\end{definition}

For $n=0$, the first clause in the definition becomes:
$$\elnf{x s_1 \ldots s_m : o} = \lambda . x \elnf{s_1} \elnf{s_2} \ldots \elnf{s_m},$$
and we deliberately keep the \textsl{dummy} lambda in the right-hand
side of the equation because it will play an important role in the
correspondence with game semantics.



Note that our version of the $\eta$-long normal form is defined not only for $\beta$-normal terms but also for any simply-typed term.
Moreover it is defined in such a way that $\beta$-normality is preserved:
\begin{lemma}
The $\eta$-long normal form of a term in $\beta$-normal form is also in $\beta$-normal form.
\end{lemma}
\begin{proof}
By induction on the structure of the term and the order of its type.
\emph{Base case}:
If $M=x:0$ then $\elnf{x} = \lambda . x$ is also in $\beta$-nf.
\emph{Step case}:
The case $M = (\lambda x . s) s_1 \ldots s_m : (A_1,\ldots,A_n,o)$ with $m>0$ is not possible since $M$ is in
$\beta$-normal form.
Suppose $M = \lambda x . s$ then $s$ is in $\beta$-nf. By the induction hypothesis $\elnf{s}$ is also in $\beta$-nf and therefore
so is $\elnf{M} = \lambda x . \elnf{s}$.

Suppose $M= \alpha s_1 \ldots s_m : (A_1,\ldots,A_n,o)$. Let $i,j$
range over $1..n$ and $1..m$ respectively. The $s_j$ are in
$\beta$-nf and the $\varphi_i$ are variables of order smaller than
$M$, therefore by the induction hypothesis the $\elnf{\varphi_i}$ and
the $\elnf{s_j}$ are in $\beta$-nf. Hence $\elnf{M}$ is also in
$\beta$-nf.
\end{proof}

\begin{lemma}[$\eta$-long normalisation preserves safety]
If $\Gamma \vdash s$ then $\Gamma \vdash \elnf{s}$.
\end{lemma}
\begin{proof}

First we observe that for any variable or constant $x$ we have $x \vdash \elnf{x}$. The proof is by induction on $\ord{x}$. Base case: $x$ is of ground type and we have $x \vdash x = \elnf{x}$. Step case:
$x:(A_1, \ldots, A_n,o)$ with $n>0$. Let $\varphi_i:A_i$ be fresh variables for $1\leq i\leq n$. The (var) rules gives $\varphi_i  \vdash \varphi_i$ and since $\ord{A_i} < \ord{x}$ the induction hypothesis gives $\varphi_i \vdash \elnf{\varphi_i}$. Using (wk) we obtain $x, \overline{\varphi} \vdash \elnf{\varphi_i}$.
The application rule gives $x, \overline{\varphi} \vdash x \elnf{\varphi_1} \ldots \elnf{\varphi_n} : o$ and the abstraction rule gives $ x \vdash \lambda \overline{\varphi} . x \elnf{\varphi_1} \ldots \elnf{\varphi_n} = \elnf{x}$.


We now prove the lemma by induction on the structure of $s$.
The base case (where $s$ is some variable $x$) is covered by the previous observation.
\emph{Step case:}
\begin{itemize}
\item $s = x s_1 \ldots s_m$ with $x: (B_1, \ldots, B_m, A_1, \ldots, A_n, o)$ with $m\geq 0$, $n>0$ and $s_i : B_i$ for $1 \leq i \leq m$.

Let $\varphi_i: A_i$ be fresh variables for $1\leq i \leq n$. By the previous observation we have $\varphi_i \vdash \elnf{\varphi_i}$ which in turn gives $\Gamma , \overline{\varphi} \vdash \elnf{\varphi_i}$ using the weakening rule.

The judgement $\Gamma \vdash x s_1 \ldots s_m$ is formed using the (app) rule therefore each $s_j$ is safe for $1\leq j \leq m$. By the induction hypothesis we have $\Gamma \vdash \elnf{s_j}$ and by weakening we get $\Gamma, \overline{\varphi} \vdash \elnf{s_j}$.

The application rule gives $\Gamma, \overline{\varphi} \vdash
x \elnf{s_1} \ldots \elnf{s_m} \elnf{\varphi_1} \ldots \elnf{\varphi_n} : o$. Finally the (abs) rule gives $\Gamma \vdash \lambda \overline{\varphi} . x \elnf{s_1} \ldots \elnf{s_m}  \elnf{\varphi_1} \ldots \elnf{\varphi_n} = \elnf{s}$, the side-condition of (abs) being met since $\ord{\elnf{s}} = \ord{s}$.


\item $s = t s_0 \ldots s_m$ where $t$ is an abstraction. Again, using the induction hypothesis it is easy to show that $\Gamma \vdash \elnf{s} = \elnf{t} \elnf{s_0} \ldots \elnf{s_m} \elnf{\varphi_1} \ldots \elnf{\varphi_n}$ holds for some fresh variables $\varphi_1$, \ldots, $\varphi_n$.

\item $s = \lambda \overline{\eta} . t$ where $t$ is not an abstraction. By the induction hypothesis we have $\Gamma, \overline{\eta} \vdash \elnf{t}$ and by the abstraction rule we have $\Gamma \vdash \lambda \overline{\eta} . \elnf{t} = \elnf{s}$.
\end{itemize}
\end{proof}

Note that in general the converse does not hold, for instance $\lambda x^o . f^{o,(o,o),o} x^o$ is unsafe although $\elnf{\lambda x . f x} = \lambda x^o \varphi^{o,o} . f x \varphi$ is safe (and not homogeneous). For terms with homogeneous types however, the converse does hold:
\begin{lemma}
If $\Gamma \vdash \elnf{s}$ is homogeneously safe (i.e. it is a safe judgement of the safe $\lambda$-calculus and each sequent occurring at the nodes of the proof tree is homogeneously typed) then
$\Gamma \vdash s$ is homogeneously safe.
\end{lemma}


\subsection{Computation tree}
The computation tree of a term is a certain tree representation of its
$\eta$-long normal form. It is defined as follows:

\begin{definition}
\label{dfn:comptree} Let $M$ be a simply-typed term in $\eta$-normal
form. Then $M$ is either an abstraction or it can be written
uniquely as $s_0 s_1 \ldots s_m : o$ for some $m\geq0$ where $s_0$
is a variable, a constant or an abstraction and each of the $s_j$
for $j\in 1..m$ is in $\eta$-normal form. The
\defname{computation tree} $\tau(M)$ of $M$ is defined by induction
on the structure of the term:
\begin{enumerate}[-]
\item If $n\geq0$ and $s$ is not an abstraction then:
$$ \tau(\lambda x_1 \ldots x_n . s) =
      \pstree[levelsep=3ex]
        { \TR{\lambda x_1 \ldots x_n} }
        { \SubTree{\tau(s)^{-}} }
$$
where $\tau(s)^{-}$ denotes the tree obtained after deleting the root of $\tau(s)$.

\item If $m\geq0$ and $\alpha$ is a variable or constant then:
$$ \tau( \alpha s_1 \ldots s_m : o) =
    \tree{\lambda}
    {
        \pstree[levelsep=3ex]
            { \TR{\alpha} }
            { \SubTree{\tau(s_1)} \SubTree[linestyle=none]{\ldots} \SubTree{\tau(s_m)}
            }
    }
$$

\item If $n \geq 1$ then:
$$ \tau((\lambda x.s) s_1 \ldots s_n : o) =
    \tree{\lambda}
    {
        \pstree[levelsep=3ex]
            { \TR{@} }
            {
            \SubTree{\tau(\lambda x.s)}    \SubTree{\tau(s_1)} \SubTree[linestyle=none]{\ldots} \SubTree{\tau(s_n)}
            }
    }
$$
\end{enumerate}

If $M$ is not in $\eta$-normal form then $\tau(M)$ is defined as the
computation tree of its $\eta$-normal form ($\tau(M) =
\tau(\etanf{M})$).
\end{definition}

The nodes (and leaves) of the tree are of three kinds:
\begin{itemize}
\item $\lambda$-nodes labelled $\lambda \overline{x}$ (note that a $\lambda$-node represents several consecutive variable abstractions),
\item application nodes labelled @,
\item variable or constant nodes labelled $\alpha$ for some constant or variable $\alpha$.
\end{itemize}
A node is said to be \defname{prime} if it is the 0$^{th}$ child of an @-node.

\emph{Notations:} We write $r$ for the root of $\tau(M)$. We write $E$ to denote the parent-child relation
of the tree, $N$ for the set of nodes of $\tau(M)$,
$N_\Sigma$ for the set of $\Sigma$-labelled nodes, $N_@$ for the set
of @-labelled nodes, $N_{\sf var}$ for the set of variable nodes,
$N_{\sf fv}$ for the subset of $N_{\sf var}$ constituted of free-variable
nodes, $N_{\sf spawn}$ for the set $N \inter E \relimg{N_@ \union N_\Sigma}$ constituted of children of constant-nodes and @-nodes and $N_{\sf prime}$ for the set of prime nodes.


Let $\mathcal{T}$ denote the set of $\lambda$-terms.
Each subtree of the computation tree $\tau(M)$ represents a subterm of $\elnf{M}$.
We define the function $\kappa : N \rightarrow \mathcal{T}$ that maps a node $n \in N$ to the subterm of $\elnf{M}$
corresponding to the subtree of $\tau(M)$ rooted at $n$.
In particular $\kappa(r) = \elnf{M}$.

\begin{definition}[Type and order of a node]
\label{def:nodeorder}
Suppose $\Gamma \vdash M : T$.
The \defname{type} of a node $n$ of $\tau(M)$ written $type(n)$ is defined as follows:
\begin{eqnarray*}
type(r) &=& \Gamma \rightarrow T \\
type(\alpha:A) &=& A, \mbox{ where $\alpha$ is a variable or constant} \\
type(n) &=& \hbox{ type of the term $\kappa(n)$ for $n \in (N_\lambda \union N_@) \setminus \{r \}$\ .}
\end{eqnarray*}
The order of a node $n$ written $\ord{n}$ is defined to be the order of the type of $n$.
\end{definition}

In particular, $\ord{@} = 0$, $\ord{\lambda \overline{\xi}} = 1+
\max_{z\in \overline{\xi}} \ord{z}$ for $\lambda \overline{\xi}\neq
r$ and if $r=\lambda \overline{\xi}$ then $\ord{r} = 1 + \max_{z\in
\overline{\xi}\union \Gamma} \ord{z}$ with the convention $\max
\emptyset = -1$.

\begin{remark} \hfill
\begin{itemize}
\item In a computation tree, nodes at even level are $\lambda$-nodes and nodes at odd level are either application nodes,
variable or constant nodes;

\item for any ground type variable or constant $\alpha$,
$\tau(\alpha) = \tau(\lambda . \alpha) =  \pstree[levelsep=3ex]
    { \TR{\lambda } }
    { \TR{\alpha}
    }$;

\item for any higher-order variable or constant $\alpha : (A_1,\ldots,A_p,o)$, the computation tree $\tau(\alpha)$ has the following form:
$ \pstree[levelsep=3ex]{\TR{\lambda}}
        {\pstree[levelsep=3ex]
                { \TR{\alpha} }
                { \tree{\lambda \overline{\xi_1}}{\TR{\ldots}} \TR{\ldots} \tree{\lambda \overline{\xi_p}}{\TR{\ldots}}
                }
        }
$;

\item for any tree of the form
        $ \pstree[levelsep=4ex]
            { \TR{\lambda \overline{\varphi}} }
            { \pstree[levelsep=3ex]
                {\TR{n}}
                {\TR{\lambda \overline{\xi_1}} \TR{\ldots} \TR{\lambda \overline{\xi_p}}}
            }
        $,
    we have $\ord{\kappa(n)}=0$.

\end{itemize}
\end{remark}


\subsection{Pointers and justified sequence of nodes}

\begin{definition}[Binder]
Let $n$ be a variable node of the computation tree labelled $x$. We
say that a node $n$ is bound by the node $m$, and $m$ is called the
binder of $n$, if $m$ is the closest node in the path from $n$ to
the root of the tree such that $m$ is labelled $\lambda
\overline{\xi}$ with $x\in \overline{\xi}$.
\end{definition}

\begin{definition}[Enabling]
The \defname{enabling relation} $\vdash$ is defined on the set of
nodes of the computation tree as follows. We write $m \vdash n$ and
we say that $m$ enables $n$ if and only if
\begin{itemize}
\item $n$ is a bound variable node and $m$ is the binder of $n$. We will write $m \vdash_i n$ to precise that $n$
is the $i^{\sf th}$ variable bound by $m$;
\item or $n$ is a free variable node and $m$ is the root of the computation
tree;
\item or $n$ is a $\lambda$-node and $m$ is the parent node of $n$.
\end{itemize}
\end{definition}

We say that a node $n_0$ of a justified sequence is
\defname{hereditarily justified} by $n_p$ if there are nodes $n_1,
\ldots, n_{p-1}$ in the sequence such that $n_i$ points to $n_{i+1}$
for all $i\in 0..p-1$.

For any set of nodes $S$ we write $S^{\upharpoonright r}$ for $\{ n \in S \ | \ r  \vdash^* n \}$ -- the subset of $S$ constituted of
nodes hereditarily enabled by $r$.
We call \defname{input-variables nodes} the elements of $N_{\sf var}^{\upharpoonright r}$ i.e.\
variables that are hereditarily enabled by the root. $N_{\sf var}^{\upharpoonright r}$ is also the set of nodes that are hereditarily enabled by a free variable or by a variable bound by the root.
\smallskip

We use the following numbering conventions:
the first child of a @-node is numbered $0$;
the first child of a variable or constant node is numbered $1$;
and variables in $\overline{\xi}$ are numbered from $1$ onward ($\overline{\xi} = \xi_1 \ldots \xi_n$).
We write $n.i$ to denote the $i$th child of node $n$.

\begin{definition}[Justified sequence of nodes]
A \defname{justified sequence of nodes} is a sequence of nodes of
the computation tree $\tau(M)$ with pointers such that each variable
or $\lambda$-node $n$ different from the root has a pointer to a
node $m$ occurring before it the sequence and such that $m \vdash
n$.

If $n$ points to $m$ then we say that $m$ \emph{justifies} $n$. We
represent the pointer in the sequence as follows \Pstr[0.4cm]{
(m){m} \ldots (n-m,45:i) n }. where the label indicates that either
$n$ is labelled with the $i$th variable abstracted by the
$\lambda$-node $m$ or that $n$ is the $i^{\sf th}$ child of $m$.
\end{definition}

Note that justified sequences are also defined for open terms:
occurrences of nodes in $N_{\sf fv}$ must point to an occurrence of the
root of the computation tree. Thus a pointer in a justified sequence of nodes has
one of the following forms:
$$
\Pstr[18pt]{ (m){r} \cdot \ldots \cdot (n-m,40){z} }
\hspace{1.5cm}
\Pstr{ (m){\lambda \overline{\xi}} \cdot \ldots \cdot (n-m,40:i){\xi_i} }
\hspace{1.5cm}
\Pstr{ (m){@} \cdot \ldots \cdot (n-m,40:j){\lambda \overline{\eta}} }
\hspace{1.5cm}
\Pstr{ (m){\alpha } \cdot \ldots \cdot (n-m,40:k){\lambda \overline{\eta}} }
$$
for some occurrences $r$ of $\tau(M)$'s root, $z \in N_{\sf fv}$,
bound variables $\xi_1,
\ldots \xi_n$, $\alpha \in N_{\Sigma} \union
N_{\sf var}$, $i \in 1..n$, $j$ ranges from $0$ to the number of
children nodes of @ minus 1 and $k \in 1 ..arity(\alpha)$.
\bigskip

\emph{Notations}: We write $s = t$ to denote that the justified sequences $t$ and $s$
have same nodes \emph{and} pointers. Justified sequence of nodes can
be ordered using the prefix ordering: $t \sqsubseteq t'$ if and only
if $t=t'$ or the sequence of nodes $t$ is a finite prefix of $t'$
(and the pointers of $t$ are the same as the pointers of the
corresponding prefix of $t'$). Note that with this definition,
infinite justified sequences can also be compared. This ordering
gives rise to a complete partial order.
We say that a node $n_0$ of a justified sequence is \defname{hereditarily justified} by $n_p$ if there are nodes $n_1, n_2, \ldots n_{p-1}$ in the sequence such that for all $i\in 0..p-1$, $n_i$ points to $n_{i+1}$.
We write $t^\omega$ to denote the last occurrence of $t$ and $\ip(t)$ for the immediate prefix of $t$ obtained by removing $t$'s last node.

We define a filtering operation on sequences of nodes:
\begin{definition}[Hereditary filtering]
Let $s$ be a justified  sequence of nodes from $\tau(M)$
and $n$ be an occurrence in $t$ of some node $n \in N_{\sf spawn}$.

We write $s \upharpoonright n$ to denote the subsequence of $s$ constituted of nodes that are hereditarily justified by $n$, where the pointer's target of all occurrences of free variable nodes in $t$ are set to $n$ (instead of $t$'s first node).

Thus $s \upharpoonright n$ is a valid justified sequence of nodes of the tree $\tau(\kappa(n))$.
\end{definition}


\begin{lemma}
\label{lem:filtercontinous}
The filtering function $\_ \upharpoonright n$ defined on the cpo of justified sequences ordered by the prefix ordering
is continuous.
\end{lemma}
\begin{proof}
Clearly $\_ \upharpoonright n$ is monotonous.
Suppose that $(t_i)_{i\in\omega}$ is a chain of justified sequence of nodes. Let $u$ be a finite prefix of $(\bigvee t_i) \filter n$.
Then $u = s \filter n$ for some finite prefix $s$ of $\bigvee t_i$. Since $s$ is finite we must have $s \sqsubseteq t_j$ for some $j\in\omega$.
Therefore $u \sqsubseteq t_j \filter n \sqsubseteq \bigvee (t_j \filter  n)$.
This is valid for any finite prefix $u$ therefore $(\bigvee t_i) \filter  n \sqsubseteq \bigvee (t_j \filter n)$.
\end{proof}



The notion of \defname{P-view} $\pview{t}$ of a justified sequence
of nodes $t$ is defined the same way as the P-view of a justified
sequences of moves in Game Semantics:

\begin{definition}[P-view of justified sequence of nodes]
The P-view of a justified sequence of nodes $t$ of $\tau(M)$, written $\pview{t}$, is defined as follows:
\begin{eqnarray*}
 \pview{\epsilon} &=&  \epsilon \\
 \pview{s \cdot n }  &=&  \pview{s} \cdot n \qquad \mbox{for $n \notin N_\lambda$, }\\
 \pview{\Pstr{ s \cdot (m){m} \cdot \ldots \cdot (lmd-m,25){\lambda \overline{\xi}}}} &=&
        \Pstr{ \pview{s} \cdot (m2){m} \cdot (lmd2-m2,60){\lambda \overline{\xi}} } \\
 \pview{s \cdot r }  &=&  r
\end{eqnarray*}
where $r$ is the root of the tree $\tau(M)$.

The equalities in the definition determine pointers implicitly. For
instance in the second clause, if in the left-hand side, $n$ points
to some node in $s$  that is also present in $\pview{s}$ then in the
right-hand side, $n$ points to that occurrence of the node in
$\pview{s}$.
\end{definition}

The O-view of $s$, written $\oview{s}$, is defined dually.
\begin{definition}[O-view of justified sequence of nodes]
The O-view of a justified sequence of nodes $t$ of $\tau(M)$, written $\oview{t}$, is defined as follows:
\begin{eqnarray*}
 \oview{\epsilon} &=&  \epsilon \\
 \oview{s \cdot \lambda \overline{\xi} }  &=&  \oview{s} \cdot \lambda \overline{\xi} \\
 \oview{\Pstr{s \cdot (m){m} \cdot \ldots \cdot (x-m,30){x}}} &=&
    \Pstr{ \oview{s} \cdot (m2){m} \cdot (n2-m2,60){x} } \qquad \mbox{ for $x \in N_{\sf var}$ }\\
 \oview{s \cdot n }  &=&  n \qquad \mbox{ for $x \in N_@ \union N_\Sigma$ }
\end{eqnarray*}
\end{definition}

We borrow some game semantic terminology:
\begin{definition} A justified sequence of nodes $s$ satisfies:
\begin{itemize}[-]
\item \defname{Alternation} if for any two consecutive nodes in $s$, one is a $\lambda$-node
and the other is not;
\item \defname{P-visibility} if every variable node in $s$ points to a node occurring in the P-view a that point;
\item  \defname{O-visibility} if every lambda node in $s$ points to a node occurring in the O-view a that point.
\end{itemize}
\end{definition}

\begin{property}
\label{proper:pview_visibility}
The P-view (resp. O-view) of a justified sequence verifying P-visibility (resp. O-visibility)
is a well-formed justified sequence verifying P-visibility (resp. P-visibility).
\end{property}
This is proved by an easy induction.

\subsection{Adding value-leaves to the computation tree}
\label{sec:adding_value_leaves}

We now add another ingredient to the computation tree defined in
the previous section. Let $\mathcal{D}$
denote the set of values of base type $o$.  We add
\defname{value-leaves} to $\tau(M)$ as follows: Every node $n \in \tau(M)$ has one child leaf labelled $v_n$ for every possible value $v \in \mathcal{D}$.
We write $V$ for the set of nodes and leaves of
the computation tree.  For $\$$ ranging in $\{@, \lambda, var \}$,
we write $V_\$$ to denote the set $N_\$ \union \{ v_n \ | \ n \in
N_\$, v \in \mathcal{D} \}$.

%If $n$ is a $\lambda$-node then its value-leaves are numbered from $1$ onwards.
%If $n$ is a variable or constant node then its children nodes are numbered from $1$ to $arity(n)$ and
%its value-leaves are numbered from $arity(n)+1$ onwards.
%If $n$ is an application node then its value-leaves are numbered from $1$ onwards.

Everything that we have defined for computation tree can be lifted
to this new version of computation tree. The node order of a
value-leaf is defined to be $0$. The enabling relation $\vdash$ is
extended so that every leaf is enabled by its parent node. The
definition of justified sequence does not change.
When representing a link in a justified sequence going from a value-leaf $v_n$ to a node $n$,
we label the link with $v$:
$$
\Pstr{ (n){n} \cdot \ldots \cdot (vn-n,40:v){v_n} }
$$

For the definition
of P-view, O-view and visibility, value-leaves are treated as
$\lambda$-nodes if they are at odd level in the computation tree and
as variable nodes if they are at an even level.

From now the term ``computation tree'' refers to this extended
definition.
\vspace{10pt}

We say that a node $n$ in of a justified sequence of nodes is
\defname{matched} by the value-leaf $v_n$ if there is an occurrence of $v_n$ for some value $v$ in the
sequence that points to $n$, otherwise we say that $n$ is
\defname{unmatched}. The last unmatched node is called the
\defname{pending node}.  A justified sequence of nodes is
\defname{well-bracketed} if each value-leaf occurring in it is justified by the pending node at that point.
If $t$ is a traversal then we write
$?(t)$ to denote the subsequence of $t$ consisting only of unmatched
nodes.

\subsection{Traversal of the computation tree}
\label{subsec:traversal}
A \emph{traversal} is a justified sequence of nodes of the computation tree where each node indicates a step that is taken during the evaluation of the term.

\subsubsection{Traversals for simply-typed $\lambda$-terms}

We first consider the simply-typed $\lambda$-calculus without interpreted constants.
Everything remains valid in the presence of \emph{uninterpreted} constants as we can just
consider them as free variables.

We define the notion of traversal over the computation tree $\tau(M)$.
We will then we show how to extend the notion of traversal to more general settings with interpreted constants.

\begin{definition}[Traversals for simply-typed $\lambda$-terms] \rm
\label{def:traversal} The set $\travset(M)$ of \defname{traversals}
over $\tau(M)$ is defined by induction over the following rules:

\noindent \emph{Initialization rules}
\begin{description}
\item[\rulenamet{Empty}] $\epsilon \in \travset(M)$.
\item[\rulenamet{Root}] The single-node sequence $r$, where $r$ denotes the root of $\tau(M)$, is a traversal.
%$ r \in \travset(M)$.
\end{description}

\noindent \emph{Structural rules}
\begin{description}
\item[\rulenamet{Lam}] If $t \cdot \lambda \overline{\xi}$ is a traversal then so is
$t \cdot \lambda \overline{\xi} \cdot n$ where $n$ denotes $\lambda
\overline{\xi}$'s child.

Moreover if $n$ is a variable node then it
points to the only
occurrence of its enabler that is still present in $\pview{t
\cdot \lambda \overline{\xi}}$.
In particular, if $n$ is a free variable node then $n$ points to the first node of $t$ (the root). (Prop. \ref{prop:pviewtrav_is_path} will show that indeed $n$'s enabler occurs exactly once in the P-view since P-views correspond to paths in the tree.)

\item[\rulenamet{App}] If $t \cdot @$ is a traversal then so is \Pstr[0.4cm]{t \cdot (m) @  \cdot (n-m,40:0) n}.
%{\em i.e.}~the next visited node is the $0^{th}$ child node of
%@: the node corresponding to the operator of the application.
\end{description}

\noindent \emph{Input-variable rules}
\begin{description}
\item[\rulenamet{InputVar$^{val}$}] If $t_1 \cdot x \cdot t_2$ is a traversal
with $x \in N_{\sf var}^{\upharpoonright r}$ and $?(t_1 \cdot x
\cdot t_2)=?(t_1) \cdot x$ then so is \Pstr[0.4cm]{t_1 \cdot
(x){x} \cdot t_2 \cdot (xv-x,38:v){v_x} } for all $v \in
\mathcal{D}$.

\item[\rulenamet{InputVar}] If $t_1 \cdot x \cdot t_2$ is a traversal with
  $x \in N_{\sf var}^{\upharpoonright r}$ and $x$ is the pending node in $t$ ($?(t_1 \cdot x \cdot
  t_2)=?(t_1) \cdot x$) then so is $t_1 \cdot x \cdot t_2 \cdot
  n$ for any $\lambda$-node $n$ whose parent occurs in
  $\oview{t_1 \cdot x}$, $n$ pointing to some occurrence of its
  parent node in $\oview{t_1 \cdot x}$.
\end{description}

\noindent \emph{Copy-cat answer rules}
\begin{description}
\item[\rulenamet{Answer-@-$\lambda$}]
  If \Pstr{t \cdot (app){@} \cdot (lz-app,60:0){\lambda
\overline{z}}  \ldots  (lzv-lz,60:v){v}_{\lambda \overline{z}} }
is a traversal then so is \Pstr[0.6cm]{t \cdot (app){@} \cdot
(lz-app,60){\lambda \overline{z}} \ldots
(lzv-lz,60:v){v}_{\lambda \overline{z}} \cdot
(appv-app,45:v){v}_@}.

\item[\rulenamet{Answer-$\lambda$-@}] If \Pstr[0.4cm]{t \cdot \lambda \overline{\xi} \cdot (x){@}  \ldots   (xv-x,50:v){v}_@}
is a traversal then so is \Pstr[0.5cm]{t \cdot (lmd){\lambda
\overline{\xi}} \cdot (x){@}  \ldots  (xv-x,50:v){v}_@  \cdot
(lmdv-lmd,30:v){v}_{\lambda \overline{\xi}} }.

\item[\rulenamet{Answer-var-$\lambda$}] If \Pstr[0.4cm]{t \cdot y \cdot (lmd){\lambda \overline{\xi}}
\ldots (lmdv-lmd,50:v){v}_{\lambda \overline{\xi}} } is a
traversal for some variable $y\not\in N_{\sf var}^{\upharpoonright
r}$ then so is \Pstr[0.7cm]{t \cdot (y){y} \cdot (lmd){\lambda
\overline{\xi}} \ldots (lmdv-lmd,30:v){v}_{\lambda
\overline{\xi}}  \cdot (vy-y,50:v){v}_y }.

\item[\rulenamet{Answer-$\lambda$-var}] If \Pstr[0.4cm]{t \cdot \lambda \overline{\xi} \cdot (x){x}  \ldots   (xv-x,50:v){v}_x}
is a traversal then so is \Pstr[0.5cm]{t \cdot (lmd){\lambda
\overline{\xi}} \cdot (x){x}  \ldots  (xv-x,50:v){v}_x  \cdot
(lmdv-lmd,30:v){v}_{\lambda \overline{\xi}} }.
\end{description}

\begin{description}
\item[\rulenamet{Var}]
If \Pstr[0.5cm]{t' \cdot (n){n} \cdot (lx){\lambda \overline{x}}
    \ldots (x-lx,50:i){x_i} } is a traversal for some variable
    $x_i$ not in $N_{\sf var}^{\upharpoonright r}$ then
so is \Pstr[0.6cm]{ t' \cdot (n){n} \cdot
    (lx){\lambda \overline{x}}  \ldots (x-lx,30:i){x_i}  \cdot
    (letai-n,40:i){\lambda \overline{\eta_i}}
     }.
\end{description}
A traversal that cannot be extended by any rule is said to be \emph{maximal}.
\end{definition}


A traversal always starts by visiting the root. Then it mainly
follows the structure of the tree.

The \rulenamet{Var} rule is particular and needs further explanation.
This rule permits the traversal to jump across the computation tree. The idea is that after visiting a
non-input variable node $x$, a jump can be made to the node corresponding to
the subterm that would be substituted for $x$ if all the
$\beta$-redexes occurring in the term were to be reduced.


Let $\lambda \overline{x}$ be $x$'s binder and suppose $x$ is the $i$th variable in $\overline{x}$.
The binding node necessarily occurs previously in the traversal (this will be proved in Prop. \ref{prop:pviewtrav_is_path}). Since $x$ is not hereditarily justified by the root, $\lambda \overline{x}$ is not the root of the tree and therefore it is not the first node of the traversal.
We do a case analysis on the node preceding $\lambda \overline{x}$:
    \begin{itemize}[-]
    \item If it is an @-node then $\lambda \overline{x}$ is necessarily the first child node of that node
    and it has has exactly $|\overline{x}|$ siblings:
    $$\pstree[levelsep=7ex]{\TR{\stackrel{\vdots}{@}}}
    {   \pstree[linestyle=dotted,levelsep=4ex]{\TR{\lambda \overline{x}}\treelabel{0}}
            {\TR{x }}
        \tree{\lambda \overline{\eta_1}}{\vdots}\treelabel{1}
        \TR[edge=\dotedge]{}
        \tree{\lambda \overline{\eta_i}}{\vdots}\treelabel{i}
        \TR[edge=\dotedge]{}
        \tree{\lambda \overline{\eta_{|x|}}}{\vdots}\treelabel{|x|}
    }
    $$
    In that case, the next step of the traversal is a jump to $\lambda \overline{\eta_i}$ -- the $i$th child of
    @ -- which corresponds to the subterm that would be substituted for $x$ if the $\beta$-reduction was
    performed:
    $$\Pstr[19pt]{ t' \cdot
            (n){@} \cdot
            (lx){\lambda \overline{x}} \cdot \ldots \cdot
            (x-lx,40:i){x} \cdot
            (mi-n,40:i){\lambda \overline{\eta_i}} \cdot \ldots
            \in {\travset(M)}   }
    $$

    \item If it is a variable node $y$, then
    the node $\lambda \overline{x}$ was necessarily added to the traversal $t_{\leq y}$ using the \rulenamet{Var} rule (see proposition \ref{prop:pviewtrav_is_path}(i)).
    Therefore $y$ is substituted by the term $\kappa(\lambda \overline{x})$ during the evaluation of the term.

    Consequently, during reduction, the variable $x$ will be substituted by the subterm represented by
    the $i$th child node of $y$. Hence the following justified sequence is also a traversal:
    $$\Pstr[18pt]{ t' \cdot
            (y){y} \cdot
            (lx){\lambda \overline{x}} \cdot \ldots \cdot
            (x-lx,40:i){x} \cdot
            (mi-y,40:i){\lambda \overline{\eta_i}} \cdot \ldots
    }
    $$
    \end{itemize}

\begin{remark}
Our notions of computation tree and traversal differ slightly from \cite{OngLics2006}:
\begin{itemize}[-]
    \item In \cite{OngLics2006} computation trees can have uninterpreted first-order constants. But as we have already observed, uninterpreted constants can be just regarded as free variables thus we do not lose any expressivity here.

    \item In \cite{OngLics2006}, constants are restricted to order one at most since computation tree
    are used to model computation of tree structures. Here we don't need this restriction (as long as constants are uninterpreted - so we can regard them as free variables).


    \item In our setting, we have to deal with \emph{free} variables.
    To model free variables we need the traversal rules \rulenamet{InputVar$^{val}$}, \rulenamet{InputVar}
    as well as the copy-cat answer rules. Whereas in \cite{OngLics2006}, the rule called \rulenamet{Sig} suffices to model the first-order constants necessary to construct tree structures.

    \item In our setting, the introduction of value-leaves
    is necessary in order to model free variables as well as interpreted constants. (We will use them to model the constants of \pcf\ and \ialgol).
    \end{itemize}
\end{remark}

\begin{example}
Consider the following computation tree:
$$\tree{\lambda}
{
    \tree{@}
    {
        \pstree[levelsep=8ex,linestyle=dotted]{\TR{\lambda y}\treelabel{0} }
        {
            \pstree[levelsep=8ex]{\TR{y}}
            {
                \tree{\lambda \overline{\eta_1}}{\vdots} \treelabel{1}
                \TR[edge=\dotedge]{}
                \tree{\lambda \overline{\eta_i}}{\vdots}\treelabel{i}
                \TR[edge=\dotedge]{}
                \tree{\lambda \overline{\eta_n}}{\vdots}\treelabel{n}
            }
        }
        \pstree[levelsep=6ex,linestyle=dotted]{\TR{\lambda \overline{x}}\treelabel{1}}{ \tree{x_i}{\TR{} \TR{} } }
    }
}
$$
An example of traversal of this tree is:
\vspace{0.3cm}
$$ \Pstr{ \lambda \cdot
            (app){@}  \cdot
            (ly){\lambda y} \cdot \ldots \cdot
            (y-ly,40:1){y} \cdot
            (lx-app,50:1){\lambda \overline{x}} \cdot \ldots \cdot
            (x-lx,40:i){x_i} \cdot
            (leta-y,50:i){\lambda \overline{\eta_i} } \cdot \ldots
        }$$
\end{example}

\subsubsection{Traversals for interpreted constants}

\begin{definition}[Well-behaved traversal rule]
\label{def:wellbehaved_traversal} A traversal rule is
\defname{well-behaved} if it can be stated under the following form:
$$\rulef{t = t_1\cdot n \cdot t_2 \in \travset \quad ?(t) = ?(t_1) \cdot n \quad P(t)}
  { \stackrel{  \rule{0pt}{3pt} }{\Pstr[5pt]{ t' = t_1\cdot (n){n} \cdot t_2 \cdot (m-n,35){m} \in \travset}}
   }
    \ m\in S(t)
   $$
such that:
\begin{enumerate}[i.]
  \item $n$ is a variable or a constant node ($n \in N_{\Sigma}\union N_{\sf var}$);
  \item $P$ expresses some condition on $t$;
  \item for every traversal $t$, $S(t)$ is some subset of $E(n)$, the set of children $\lambda$-nodes and value-leaves of $n$.
  If $S(t)$ has more than one element then the rule is non-deterministic.
\end{enumerate}
\end{definition}
Note that if $t$ is well-bracketed then $t'$ is also well-bracketed
and if $?(t)$ satisfies alternation and visibility then so does
$?(t')$.


\begin{example} The rule (InputVar$^{val}$) is an example of non-deterministic well-behaved traversal rule for which $S(t)$ is exactly the set of all children value-leaves of $n$:
$S(t) = \{ v_n \ | \ v \in \mathcal{D} \} $.
However (InputVar) is not well-behaved since it can jump to any node in the O-view at that point and not necessarily to a children node of the last pending node.
\end{example}

In the presence of higher-order interpreted constants, additional rules must be specified to indicate how
the constant nodes should be traversed in the computation tree. These rules
are specific to the language that is being studied.
In the last section of this chapter we will define such traversals for the interpreted constants of
\pcf\ and \ialgol.

From now on, we consider a simply-typed $\lambda$-calculus language extended with
higher-order interpreted constants for which some constant traversal rules have been defined
and we take the following condition as a prerequisite:
\begin{center}
  \textbf{(Condition WB)} The constant traversal rules are well-behaved.
\end{center}


\subsubsection{Some properties of traversals}

\begin{proposition}[counterpart of proposition 6 from \cite{OngHoMchecking2006}]
\label{prop:pviewtrav_is_path}
Let $t$ be a traversal. Then:
\begin{itemize}
\item[(i)] $t$ is a well-defined and well-bracketed justified sequence;
\item[(ii)] $t$ is a well-defined justified sequence verifying alternation, P-visibility and O-visibility;
\item[(iii)] If $t^\omega \in N$ {\it i.e.}~$t$'s last node is not a value-leaf, then $\pview{t}$ is the path in the computation tree going from the root to the node $t^\omega$.
\end{itemize}
\end{proposition}

This is the counterpart of proposition 6 from
\cite{OngHoMchecking2006} which is proved by induction on the
traversal rules. This proof can be easily adapted to take into
account the constant rules (using the assumption that constants
rules are well-behaved) and the presence of value-leaves in the
traversal.
\begin{proof}
The proof of (i), (ii) and (iii) is done simultaneously by induction on the traversal rules. We consider the rules \rulenamet{Var} and \rulenamet{Lam} only.

Rule \rulenamet{Var}: we just give a partial proof of (i). See proposition 6 from \cite{OngHoMchecking2006} for the details of (i), (ii) and (iii). We have to show that in the second case of the \rulenamet{Var} rule, where $p$ is a variable node $y$, the node $\lambda \overline{x}$ has necessarily been added to the traversal $t_{\leq y}$ using the \rulenamet{Var} rule. This is immediate since if the rule \rulenamet{InputVar} was used to produce $t_{<y} \cdot y \cdot \lambda \overline{x}$ this would imply that $\lambda \overline{x}$ is hereditarily justified by the root which in turn implies that $x_i$ is an input-variable which contradicts \rulenamet{Var}'s hypothesis.

Rule \rulenamet{Lam}: we need to show that $n$'s enabler occurs only once in the P-view at that point. By the induction hypothesis we have (by (iii)) that $\pview{t \cdot \lambda \overline{\xi}}$ is a path in the computation tree from the root to $\lambda \overline{\xi}$. $n$'s enabler occurs only once in this path: it is precisely it's binding node. Therefore the traversal $t \cdot \lambda \overline{\xi} \cdot n$ is well-defined and $t \cdot \lambda \overline{\xi} \cdot n$ satisfies P-visibility. Thus (i) and (ii) are verified. Furthermore $n$ is a child of $\lambda \overline{\xi}$ therefore (iii) also holds.
\end{proof}

%In particular to prove that the copy-cat rules are well-defined, one needs to ensure that
%if the last two unmatched nodes are $y$ and $\lambda \overline{\xi}$ in that order, for some non input-variable node $y$ then necessary
%      $y$ and $\lambda \overline{\xi}$ are consecutive nodes in the traversal.
%    This is because in a traversal, a non input-variable $y$ is always followed by a lambda node and whenever this lambda node is answered
%    there is only one way to extend the traversal : by using the copy cat rule to answer the $y$ node.

\begin{definition}
The \defname{reduction of a traversal} $t$ is define as the subsequence $ t
\filter r$ where $r$ denotes the first node in $t$ (which is necessarily $\tau(M)$'s root).
\end{definition}
The effect of this transformation is the elimination of the
``internal nodes'' of the computation. Since @-nodes and $\Sigma$-constants do not have pointers, the
reduction of traversal contains only nodes in $N_\lambda \union
N_{\sf var}$.

We define the set
$$\travset(M)^{\upharpoonright r} = \{ t  \upharpoonright r \ | \  t  \in \travset(M) \} \ . $$




\begin{lemma}
\label{lem:var_followedby_child} Suppose $M$ is in $\beta$-normal
form. Let $t \travset(M)$. If
$\Pstr{ t = u_1 \cdot (m){m} \cdot u_2 \cdot (n-m,30){n} }$
 where $m \in (N_{\sf var} \union N_{\Sigma}) \setminus (N^{\upharpoonright r}_{var} \union N^{\upharpoonright r}_{\Sigma})$
then $u_2 = \epsilon$.
\end{lemma}
\begin{proof}
By case analysis on the rule used to visit the node
$n$ in $t$. The only relevant rules are (Var), (Answer-var), (InputVar$^{val}$), (InputVar)
and the constant rules.
Since the term is in $\beta$-normal form, there is no @-node in $\tau(M)$ and therefore (Var) cannot be used.
Since $m$ is not hereditarily justified by the root, it is not an input-variable and therefore the rules
(InputVar$^{val}$) and (InputVar) cannot be used.
For the rule (Answer-var) the result follows from the well-bracketedness of traversals.
For constant rules, the result follows from the well-behaviour of constant rules (condition WB).
\end{proof}

\begin{lemma}[View of a traversal reduction]
\label{lem:redtrav_trav} Suppose that $M$ is a $\beta$-normal term and let $t$ be a traversal of $\tau(M)$ then
\begin{itemize}
\item[(i)] $ \pview{t \upharpoonright  r } = \pview{t} \upharpoonright r$\ ;
\item[(ii)] if $t^\omega \in N^{\filter r}$ {\it i.e.}~$t$'s last node is hereditarily justified by $r$, then
    $\oview{t \upharpoonright r } = \oview{t}$\ .
\end{itemize}
\end{lemma}
In the safe lambda calculus without interpreted constants this lemma
follows immediately from the fact that $\travset(M) =
\travset(M)^{\upharpoonright r }$. Here we prove the result in a
more general setting of a calculus extended with interpreted
constants whose corresponding traversal rules are
\emph{well-behaved}.


\begin{proof}
(i) By induction. It is trivially true for the empty
traversal and for the traversal $t = r$. Step case: consider a traversal $t$ and
suppose that the property (i) is verified for all traversal shorter
than $t$:
\begin{itemize}[-]
\item If $t = t' \cdot n$ with $n \in N_{\sf var} \union N_{\Sigma}$ then:
    \begin{align*}
    \pview{t} \upharpoonright  r
&= \pview{t' \cdot n} \upharpoonright  r & (\mbox{definition of } t)\\
        &= (\pview{t'} \cdot n) \upharpoonright  r  & (\mbox{P-view computation}) \\
        &= \pview{t'} \upharpoonright  r  \cdot (n \upharpoonright  r)            & (\mbox{def. of filtering $\upharpoonright$}) \\
        &= \pview{t' \upharpoonright  r } \cdot (n \upharpoonright  r)           & (\mbox{induction hypothesis}) \\
        &= \pview{t' \upharpoonright  r \cdot (n \upharpoonright  r) } & (\mbox{P-view computation, $n \in N_{\sf var} \union N_{\Sigma}$}) \\
        &= \pview{(t' \cdot n ) \upharpoonright  r  }           & (\mbox{def. of filtering $\upharpoonright$}) \\
        &= \pview{t \upharpoonright  r  }
 & (\mbox{definition of } t).
    \end{align*}


\item If $\Pstr{ t =  t' \cdot (m){m} \cdot  u \cdot (lmd-m,30){n}}$ with $n\in N_\lambda \setminus N^{\upharpoonright r}_\lambda$ then we have $u = \epsilon$ by lemma
    \ref{lem:var_followedby_child} and:
        \begin{align*}
        \pview{t} \upharpoonright  r
        &= \pview{\Pstr{t' \cdot (m){m} \cdot (n-m,60){n}}} \upharpoonright  r
                                                        & (u=\epsilon)\\
        &= (\Pstr{\pview{t'} \cdot (m){m} \cdot (lmd-m,60){n}} ) \upharpoonright  r
                                                        & (\mbox{P-view computation}) \\
        &= \pview{t'} \upharpoonright  r                & (m, n \not\in N^{\upharpoonright r}) \\
        &= \pview{t' \upharpoonright  r }               & \mbox{(induction hypothesis)} \\
        &= \pview{ (\Pstr{t' \cdot (m){m} \cdot (lmd-m,40){n}}) \upharpoonright r }
                                                        & (m, n \not\in N^{\upharpoonright r}) \\
        &= \pview{ t \upharpoonright r }             & \mbox{(def. of $t$ \& $u = \epsilon$).}
        \end{align*}

\item If $\Pstr{ t =  t' \cdot (m){m} \cdot u \cdot (lmd-m,30){n} }$ with $n\in N^{\upharpoonright r}_\lambda$ then:
        \begin{align*}
        \pview{t} \upharpoonright  r
        &= \pview{\Pstr{t' \cdot (m){m} \cdot u \cdot (n-m,40){n}}} \upharpoonright  r
                                                              & (\mbox{definition of } t)\\
        &= (\Pstr{\pview{t'} \cdot (m){m} \cdot  (lmd-m,60){n}}) \upharpoonright  r
                                                              & (\mbox{P-view computation}) \\
        &= \Pstr{ \pview{t'} \upharpoonright  r \cdot (m){m} \cdot  (lmd-m,60){n} }
                                                              & (m, n \in N^{\upharpoonright r}) \\
        &= \Pstr{ \pview{t'\upharpoonright r}  \cdot (m){m} \cdot  (lmd-m,60){n} }
                                                              & \mbox{(induction hypothesis)} \\
        &= \pview{ \Pstr{t' \upharpoonright r \cdot (m){m} \cdot {(u \upharpoonright r)} \cdot (lmd-m,35){n}}}
                                                           & (\mbox{P-view computation}) \\
        &= \pview{ (\Pstr{t' \cdot (m){m} \cdot u \cdot (lmd-m,35){n}}) \upharpoonright r }
                                                           & (m, n \in N^{\upharpoonright r}) \\
        &= \pview{ t \upharpoonright r }                & \mbox{(def. of $t$).}
        \end{align*}
\end{itemize}
(ii) By a straightforward induction similar to (i).
\end{proof}

\begin{remark}
\label{rem:inputvar}
Using the previous lemma we observe that in the definition of the rule \rulenamet{InputVar} we have
$n \in N_\lambda^{\filter r}$. Indeed,
$\oview{t_1 \cdot x } = \oview{ (t_1 \cdot x) \filter r}$ therefore $n$
is hereditarily enabled by $r$.
\end{remark}

\begin{lemma}[Traversal of $\beta$-normal terms]
\label{lem:betaeta_trav}
Let $M$ be a $\beta$-normal term, $r$ be the root of the tree $\tau(M)$ and
$t$ be a traversal of $\tau(M)$.
For any node $n$ occurring in $t$:
\begin{eqnarray*}
r \mbox{ does not hereditarily justify } n  \  \iff \   n \mbox{ is
hereditarily justified by some node in } N_\Sigma.
\end{eqnarray*}
\end{lemma}
\begin{proof}
 In a computation tree, the only nodes that do not have justification pointer are:
the root $r$, @-nodes and $\Sigma$-constant nodes. But since $M$ is
in $\beta$-normal form, there is no @-node in the computation tree.
Hence nodes are either hereditarily justified by $r$ or hereditarily
justified by a node in $N_\Sigma$. Moreover $r$ is not in $N_\Sigma$
therefore the ``or'' is exclusive : a node cannot be hereditarily
justified at the same time by $r$ and by some node in $N_\Sigma$.
\end{proof}


\section{Game semantics correspondence}
\label{sec:gamesemcorresp}

 We are working in the general setting of an applied
simply-typed $\lambda$-calculus with a given set of higher-order
constants $\Sigma$. The operational semantics of these constants is
given by certain reduction rules. We assume that a fully abstract
model of the calculus is provided by means of a category of
well-bracketed games. For instance, if $\Sigma$ consists
of the \pcf\ constants then we consider the traditional
category of games and innocent well-bracketed strategies
\cite{hylandong_pcf,abramsky94full}.


In the literature, a strategy is commonly defined as a set of plays closed by
even-length prefixing. However, for our purpose here, it is more convenient to represent strategies using \emph{prefix-closed} set of plays. This saves us from considerations on the parity of traversal length when
showing the correspondence between traversals and game semantics.
 For the rest of the section we fix a simply-typed term $\Gamma \vdash M :T$. We write $\sem{\Gamma \vdash M : T}$ for its strategy denotation (in the standard cartesian closed category of games and innocent strategies \cite{abramsky94full, hylandong_pcf}). We use the notation $\prefset(S)$ to denote the prefix-closure of the set $S$.

\subsection{Relating computation trees and games}
Let us first study an example:
\subsubsection{Example}
Consider the following term $M \equiv \lambda f z . (\lambda g x . f (f x)) (\lambda y. y) z$ of type $(o \typear o) \typear o \typear o$.
Its $\eta$-long normal form is $\lambda f z . (\lambda g x . f (f x)) (\lambda y. y) (\lambda .z)$.
The computation tree is:

$$
\tree{\lambda f z}
{ \tree{@}
    {
        \tree{\lambda g x}
            { \tree{f}{   \tree{\lambda}{ \tree{f}{  \tree{\lambda}{\TR{x}}} }  }
            }
        \tree{\lambda y}{\TR{y}}
        \tree{\lambda}{\TR{z}}
    }
}
$$

The arena for the type $(o \typear o) \typear o \typear o$ is:
$$\tree{q^1}
{
    \tree{q^3}
        {  \tree{q^4}
                {\TR{a^4_1} \TR{\ldots}}
            \TR{a^3_1} \TR{\ldots} }
    \tree{q^2}
    { \TR{a^2_1} \TR{a^2_2}\TR{\ldots} }
    \TR{a_1} \TR{a_2}\TR{\ldots}
}
$$

\newlength{\yNull}
\def\bow{\quad\psarc{->}(0,\yNull){1.5ex}{90}{270}}

The figure below represents the computation tree (left) and the
arena (right). The dashed line defines a partial function $\psi$
from the set of nodes in the computation tree to the set of moves.
For simplicity, we now omit answers moves when representing arenas.
$$
\tree{ \Rnode{root} {\lambda f z}^{[1]} }
     {  \tree{@^{[2]}}
        {   \tree{\lambda g x ^{[3]}}
                { \tree{\Rnode{f}{f^{[6]}}}{  \tree{\Rnode{lmd}\lambda^{[7]}}{ \tree{\Rnode{f2}{f^{[8]}}} {\tree{\Rnode{lmd2}\lambda^{[9]}}{\TR{x^{[10]}}}}}  }
                }
            \tree{\lambda y ^{[4]}}{\TR{y}}
            \tree{\lambda ^{[5]}}{\TR{\Rnode{z}z}}
        }
    }
\hspace{3cm}
  \tree[levelsep=12ex]{ \Rnode{q1}q^1 }
    {   \pstree[levelsep=4ex]{\TR{\Rnode{q3}q^3}}{\TR{\Rnode{q4}q^4}}
        \TR{\Rnode{q2}q^2}
        \TR{\Rnode{q5}q^5}
    }
\psset{nodesep=1pt,arrows=->,arcangle=-20,arrowsize=2pt 1,linestyle=dashed,linewidth=0.3pt}
\ncline{->}{root}{q1} \aput*{:U}{\varphi}
\ncarc{->}{z}{q2}
\ncline{->}{f}{q3}
\ncline{->}{lmd}{q4}
\ncline{->}{f2}{q3}
\ncline{->}{lmd2}{q4}
$$

Consider the justified sequence of moves $s \in \sem{M}$:
 $$s = \Pstr[0.6cm][5pt]{(q1){q}^1\ (q3-q1,60){q}^3\ (q4-q3,60){q}^4\ (q3b-q1){q}^3\ (q4b-q3b,60){q}^4\ (q2-q1,30){q}^2 }
\in \sem{M}$$

There is a corresponding justified sequence of nodes in the computation tree:
$$r = \Pstr[0.8cm]{
        (q1){\lambda f z} \cdot
        (q3-q1,60){f}^{[6]} \cdot
        (q4-q3,60){\lambda^{[7]}} \cdot
        (q3b-q1,60){f}^{[8]} \cdot
        (q4b-q3b,50){\lambda^{[9]}} \cdot
        (q2-q1,50){z} }$$
such that $s_i = \psi(r_i)$ for all $i < |s|$.

The sequence $r$ is in fact the reduction of the following
traversal:
$$t = \Pstr[1.1cm]{ (q1){\lambda f z} \cdot
            (n2){@^{[2]}} \cdot (n3-n2,60){\lambda g x^{[3]}} \cdot
            (q3-q1,60){f}^{[6]} \cdot (q4-q3,60){\lambda^{[7]}} \cdot
            (q3b-q1,40){f}^{[8]} \cdot (q4b-q3b,70){\lambda^{[9]}} \cdot
            (n8-n3,35){x^{[10]}} \cdot
            (n9-n2,30){\lambda^{[5]}} \cdot
            (q2-q1,35){z} }
$$

By representing side-by-side the computation tree and the type arena of a term in $\eta$-normal form we have observed
that some nodes of the computation tree can be mapped to question moves of the arena.
In the next section, we show how to define this mapping in a systematic manner.

\subsubsection{Formal definition}

We now establish formally the relationship between games and computation trees. Suppose $\Gamma \vdash M : T$
is in $\eta$-long normal form. We suppose that computation tree $\tau(M)$
is given by a pair $(V,E)$ where $V$ is the set of vertices of
and $E \subseteq V \times V$ is the parent-child relation. We have $V = N \union VL$ where $N$
and $VL$ are the set of nodes and value-leaves respectively.

\emph{Notations:}
We write $V_\$$ for $N_\$ \union (E(N_\$) \inter VL)$ where $\$$ ranges over $\{@, {\sf var}, \Sigma, {\sf fv} \}$.
Let $\mathcal{D}$ be the set of values of the base type $o$. If $n$ is a node in $N$ then the value-leaves attached to the node $n$ are written $v_n$ where $v$ ranges in $\mathcal{D}$.
Similarly, if $q$ is a question in $\sem{A}$ then the answer moves enabled by $q$ are written $v_q$ where $v$ ranges in $\mathcal{D}$.

\begin{definition}[Mapping from nodes to moves]\hfill
\label{def:phi_psi mapping}

    \begin{itemize}[-]
    \item Let $n$ be a node in $N_\lambda \union N_{\sf var}$ and $q$ be a question move of some game $A$
such that $n$ and $q$ are of type $(A_1,\ldots,A_p,o)$ for some $p\geq 0$. The function $\psi^{n,q}_A$ from $V^{\upharpoonright n}$ to $\sem{A}$ is defined as:
        \begin{eqnarray*}
        \psi^{n,q}_A &=& \{ n \mapsto q \} \union  \{ v_n \mapsto v_q \ | \ v \in \mathcal{D} \}\\
         &&\union \left\{
                        \begin{array}{ll}
                          \emptyset, & \hbox{if $p=0$\ ;} \\
                          \Union_{m \in N | n \vdash_i m} \psi^{m, q^i}_A, & \hbox{if $p\geq1$ and $n\in N_{\lambda}$\ ;} \\
                          \Union_{i=1..p} \psi^{n.i, q^i}_A, & \hbox{if $p\geq1$ and $n\in N_{\sf var}$\ .}
                        \end{array}
                      \right.
        \end{eqnarray*}
        where $\{ q^1, \ldots, q^p \} \union \{ v_q \ | \ v \in \mathcal{D} \}$ is the set of moves enabled by $q$ in $A$ (each $q^i$ being of type $A_i$).

    \item We use the abbreviation $\psi_n$
    for $\psi^{n,m}_{T(n)} : V^{\upharpoonright n} \rightarrow \sem{T(n)}$
    where $m$ denotes $\sem{T(n)}$'s initial move.

    \item Similarly we write $\psi_M$ (or just $\psi$ if this does not cause any ambiguity)
    for $\psi^{r,m}_{\Gamma\rightarrow T}$ where $m$ denote $\sem{\Gamma\rightarrow T}$'s initial move.\footnote{Arenas involved in the game semantics of simply-typed $\lambda$-calculus are all trees: they have a single initial move.}
    \end{itemize}
\end{definition}

It can easily be checked that the domain of definition of $\psi_n$ is indeed the set of nodes that are hereditarily enabled by $n$.

Let us detail a little the definition of $\psi_n$:
\begin{itemize}
\item If $p=0$ then $n$ is a dummy $\lambda$-node or a ground type variable: $\psi_n$ maps $n$ to the initial move $q$.

\item  If $p\geq 1$ and $n \in N_{\lambda}$ with $n$ labelled $\lambda \overline{\xi} = \lambda \xi_1 \ldots \xi_p$ then the sub-computation tree rooted at $n$ and the arena $\sem{T(n)}$ have the following forms (value-leaves and answer moves are not represented for simplicity):
    $$ \tree{ \Rnode{r}\lambda \overline{\xi}  ^{[n]}}
        {
            \tree[levelsep=6ex]{\alpha}
            {   \TR{\ldots} \TR{\ldots} \TR{\ldots}
            }
        }
    \hspace{3cm}
    \tree{ \Rnode{q0}m_n }
        {
            \tree[linestyle=dotted]{q^1}{\TR{} \TR{} }
            \tree[linestyle=dotted]{q^2}{\TR{} \TR{} }
            \TR{\ldots}
            \tree[linestyle=dotted]{q^p}{\TR{} \TR{} }
        }
    \psset{nodesep=1pt,arrows=->,arcangle=-20,arrowsize=2pt 1,linestyle=dashed,linewidth=0.3pt}
    \ncline{->}{r}{q0}
    \ncarc{->}{q2}{z}
    \ncline{->}{q3}{f}
    \ncline{->}{q4}{lmd}
    \ncline{->}{q3}{f2}
    \ncline{->}{q4}{lmd2}
    $$

    For each abstracted variable $\xi_i$ there exists a corresponding question move $q^i$ of the same order in the arena. $\psi_n$ maps each free occurrence of $\xi_i$ in the computation tree to the move $q^i$.

\item If $p\geq 1$ and $n\in N_{\sf var}$ then $n$ is labelled with a variable $x:(A_1,\ldots,A_p,o)$
with children nodes $\lambda \overline{\eta}_1$, \ldots, $\lambda \overline{\eta}_p$. The computation tree $\tau(M)$ rooted at $n$ and the arena $\sem{T(n)}$ have the following forms:
    $$\tree{\Rnode{r}{x^{[n]}}}
        {   \tree{\TR{\lambda \overline{\eta}_1}}{\vdots} \TR{\ldots}
        \tree{\TR{\lambda \overline{\eta}_p }}{\vdots}
        }
    \hspace{3cm}
    \tree{ \Rnode{q0}m_n }
        {
\tree[linestyle=dotted]{\Rnode{q1}{q^1}}{\TR{} \TR{} }
            \tree[linestyle=dotted]{\Rnode{q2}{q^2}}{\TR{} \TR{} }
            \TR{\ldots}
            \tree[linestyle=dotted]{\Rnode{qp}{q^p}}{\TR{} \TR{} }
        }
    \psset{nodesep=1pt,arrows=->,arcangle=-20,arrowsize=2pt 1,linestyle=dashed,linewidth=0.3pt}
    \ncline{->}{r}{q0}
    \ncarc{->}{q2}{z}
    \ncline{->}{q3}{f}
    \ncline{->}{q4}{lmd}
    \ncline{->}{q3}{f2}
    \ncline{->}{q4}{lmd2}
    $$

    and $\psi_n$ maps each node $\lambda \overline{\eta}_i$ to the question move $q^i$.
\end{itemize}

\begin{example}
Take $M = \lambda x . (\lambda g . g x) (\lambda y . y)$ with $x,y:o$
and $g:(o,o)$. The diagram below represents the computation tree
(middle), the arenas $\sem{(o,o), o}$ (left), $\sem{o , o}$ (right),
$\sem{o\rightarrow o}$ (rightmost), $\psi_{\lambda x}$,
$\psi_{\lambda g}$ and $\psi_{\lambda y}$ (dashed-lines).
$$\psset{levelsep=3.5ex}
\pstree{\TR[name=root]{\lambda x}}
{
    \pstree{\TR[name=App]{@}}
    {
            \pstree{\TR[name=lg]{\lambda g}}
                { \pstree{\TR[name=lgg]{g}}{
                        \pstree{\TR[name=lgg1]{\lambda}}
                        { \TR[name=lgg1x]{x}  } } }
            \pstree{\TR[name=ly]{\lambda y}}
                    {\TR[name=lyy]{y}}
    }
}
\rput(4.5cm,-1cm){
  \pstree{\TR[name=A1lx]{q_{\lambda x}}}
        { \TR[name=A1x]{q_x} }
}
\rput(-6cm,-1.5cm){
    \pstree{\TR[name=A2lg]{q_{\lambda g}}}
    {
        \pstree{\TR[name=A2g]{q_g}}
        {  \TR[name=A2g1]{q_{g_1}}   }
    }}
\rput(2.5cm,-1.5cm){
    \pstree{\TR[name=A3ly]{q_{\lambda y}}}
        { \TR[name=A3y]{q_y}
        }
}
\psset{nodesep=1pt,arrows=->,arcangle=-20,arrowsize=2pt 1,linestyle=dashed,linewidth=0.3pt}
\ncline{->}{root}{A1lx} \mput*{\psi_{\lambda x}}
\ncarc{->}{lgg1x}{A1x}
\ncline{->}{lg}{A2lg} \mput*{\psi_{\lambda g}}
\ncline{->}{lgg}{A2g}
\ncline{->}{lgg1}{A2g1}
\ncline{->}{ly}{A3ly} \mput*{\psi_{\lambda y}}
\ncline{->}{lyy}{A3y}
$$
\end{example}

\begin{property} \
\label{proper:psi_properties}
\begin{enumerate}[(i)]
\item $\psi$ maps $\lambda$-nodes to O-questions, variable nodes to
P-questions, value-leaves of $\lambda$-nodes to P-answers and
value-leaves of variable nodes to O-answers;
\item $\psi$ maps a node of a given order to a move of the same order;
\item Let $s \in \travset(M)^{\filter r}$. The P-view (resp. O-view) of $\psi(s)$ and $s$ are computed
identically {\it i.e.}~the set of occurrence positions that must be removed
from each sequences in order to obtain their respective P-view (resp. O-view) is the same for both sequence.
\end{enumerate}
\end{property}
\begin{proof}
(i) and (ii) are direct consequences of the definition.

(iii) Because of (i) and since $t$ and $\psi(t)$ have the
same pointers, the computations of the P-view (resp. O-view) of the
sequence of moves and the P-view (resp. O-view) of the sequence of
nodes follow the same steps.
\end{proof}
The fact that we have defined the order of the root node differently from the order of other $\lambda$-nodes
(Def. \ref{def:nodeorder}) should now make more sense to the reader: this definition permits us to state property (ii).
\smallskip

By extension, we can define the function $\psi_M$ on $\travset(M)^{\filter r}$, the set of justified
sequences of nodes that are hereditarily justified by (the only occurrence of) the root $r$:
\begin{definition}[Mapping sequences of nodes to sequences of moves]
We define the function $\psi_M : \travset(M)^{\filter r} \rightarrow \sem{\Gamma \rightarrow T}$ as follows.
If $s = s_0 s_1 \ldots \in \travset(M)^{\filter r}$ then:
$$\psi_M(s) = \psi_M(s_0)\ \psi_M(s_1)\  \psi_M(s_2) \ldots$$
where $\psi_M(s)$ is equipped with $s$'s pointers.

Thus the pointer-free version of this function is a monoid homomorphism.
\end{definition}


\subsection{Interaction games}
\label{sec:interaction_semantics}

In game semantics, strategy composition is achieved by performing a
CSP-like ``composition + hiding''. It is possible to define an
alternative semantics where the internal moves are not hidden when
performing composition. This semantics is named \emph{revealed
semantics} in \cite{willgreenlandthesis} and \emph{interaction}
semantics in \cite{DBLP:conf/sas/DimovskiGL05}.

In addition to the moves of the standard semantics, the interaction
semantics contains certain internal moves of the computation.
Consequently, the interaction semantics depends on the syntactical
structure of the term and therefore cannot lead to a full
abstraction result. However this semantics will prove to be useful
to identify a correspondence between the game semantics of a term
and the traversals of its computation tree.

Our interaction semantics will be calculated from the $\eta$-normal
form of a term. However we do not want to keep all the internal
moves: we only keep the internal moves that are produced when
composing two subterms of the computation tree joint by an @-node.
This means that when computing the strategy denoting $y N_1 \ldots
N_p$ where $y$ is a variable, we preserve the internal moves of
$N_1$, \ldots, $N_p$ while omitting the internal moves produced by
the copy-cat projection strategy denoting $y$.


\begin{definition} \hfill
\begin{itemize}
\item We call \defname{interaction type tree} or just \defname{interaction type},
a tree whose leaves are labelled with linear simple types and
nodes are labelled with symbol in $\{ ;, \langle \_\ ,\_
\rangle, \otimes, \dagger, \Lambda \}$.


Nodes labelled $;$, $\langle \_\ ,\_ \rangle$ or $\otimes$ are
binary nodes and nodes labelled $\dagger$ or $\Lambda$ are unary
nodes. If $T_1$ and $T_2$ are interaction types we write
$\langle T_1, T_2 \rangle$ to denote the interaction type
obtained by attaching $T_1$ and $T_2$ to a $\langle \_\ ,\_
\rangle$-node. Similarly we use the notations $T_1 \otimes T_2$,
$T_1 ; T_2$, $\Lambda(T_1)$ and $T_1^\dagger$.

\item To every node or leaf we can associate a linear type. We write
    $type(T)$ to denote the type associated to the root node. We
    sometime write the type in exponent {\it e.g.}
    $T^{A\rightarrow B}$ if $type(T) =A\rightarrow B$. This type
    is determined by the structure of the tree as follows:
    \begin{itemize}
    \item If $T$ is a leaf then $type(T)$ is define as the type that labels the leaf;

    \item $type\ (T^{!A \multimap B})^\dagger = !A \multimap !B$;

    \item $type\ \Lambda(T_1^{A \otimes B \multimap C}) = A \multimap (B \multimap C)$

    \item $type\ \langle T_1^{C \multimap A} , T_2^{C \multimap B} \rangle =
    C \multimap A \times B$;

    \item $type\ T_1^{A \multimap B} \otimes T_2^{C \multimap D} = (A \otimes C) \multimap (B \otimes D)$;

    \item $type\ T_1^{A \multimap B};T_2^{B \multimap C} = A \multimap C$.
    \end{itemize}

\end{itemize}

For the interaction type tree to be well-defined, it is required
that types of children nodes are consistent with the meaning of the
parent node; for instance the two children nodes of a ;-node must be
of type $A\multimap B$ and $B\multimap C$.

\end{definition}


Let $T$ be an interaction type tree. Each leaf or node of type $A$
in $T$ can be mapped to the (standard) game $\sem{A}$. By taking the
image of $T$ across this mapping we obtain a tree whose leaves and
nodes are labelled by games. This tree, written $\intersem{T}$, is
called an \defname{interaction game}.

A \defname{revealed strategy} $\Sigma$ on the interaction game $\intersem{T}$ is a compositions of several standard strategies in which certain internal moves are not hidden. Formally:
\begin{definition}[Revealed strategy]
A revealed strategy $\Sigma$ on an interaction game $\intersem{T}$,
written $\Sigma: \intersem{T}$, is an annotated interaction type
tree $T$ where
\begin{itemize}
\item each leaf $\sem{A}$ of $T$ is annotated with a (standard) strategy $\sigma$ on the game
$\sem{A}$;
\item each $;$-node is annotated with a set of indices $U \subseteq \nat$.
\end{itemize}
\end{definition}

The intuition behind this definition is that each $;$-node with children of type $A\multimap B$ and $B\multimap C$ is annotated with a set of indices $U$ indicating which components of $B$ should be uncovered when performing composition.
More precisely, if $B = B_0 \times \ldots \times B_l$ then the revealed strategy built by connecting two revealed strategies $\Sigma_1 : \intersem{A\multimap B}$ and $\Sigma_2 : \intersem{B\multimap C}$
using a $;$-node annotated with $U$ represents the
set of uncovered plays obtained
by performing the usual composition while ignoring and copying the internal moves already in $\Sigma_1$ and $\Sigma_2$ and preserving any internal
move produced by the composition in some component $B_k$ for $k \in U$.

\begin{example}
The diagrams below represent an interaction type tree $T$ (left),
the corresponding interaction game $\intersem{T}$ (middle) and a
revealed strategy $\Sigma$ (right):
$$
\pstree[levelsep=6ex]{\TR{;}}
        {
            \pstree[levelsep=6ex]{\TR{;}}
            { \TR{A\multimap B}
              \TR{B\multimap C}
            }
            \TR{C\multimap D}
        }
\hspace{1cm}
\pstree[levelsep=6ex]{\TR{;}}
        {
            \pstree[levelsep=6ex]{\TR{;}}
            { \TR{\sem{A\multimap B}}
              \TR{\sem{B\multimap C}}
            }
            \TR{\sem{C\multimap D}}
        }
\hspace{1cm}
\pstree[levelsep=6ex]{\TR{;^{\{0\}}}}
        {
            \pstree[levelsep=6ex]{\TR{;^{\{0\}}}}
            { \TR{A\multimap B^{\sigma_1}}
              \TR{B\multimap C^{\sigma_2}}
            }
            \TR{C\multimap D^{\sigma_3}}
        }
$$
\end{example}
A revealed strategy can also be written as an expression, for
instance the strategy represented above is given by the expression
$\Sigma = (\sigma_1 ;^{\{0\}} \sigma_2) ;^{\{0\}} \sigma_3$. We will
use the abbreviation $\Sigma_1 \fatsemi^U \Sigma_2$ for
$\Sigma_1^\dagger ; ^U \Sigma_2$.


\subsubsection{Uncovered play}

The analogous of a play in the interaction semantics is called an
\emph{uncovered play}, it is a play containing internal moves. The
moves are implicitly tagged so that it is possible to retrieve in
which component of the arena of which node/leaf-game the move
belongs to. A given move may belong to several games from different
nodes/leaves of the interaction game.

\begin{definition}
The \defname{set of possible moves} $M_T$ of an interaction game
$\intersem{T}$ is defined as $\mathcal{M}_T/\hspace{-0.5em}\sim_T$,
the quotient of the set $\mathcal{M}_T$ by the equivalence relation
$\sim_T \subseteq \mathcal{M}_T \times \mathcal{M}_T$ defined as follows:
For a single leaf tree $T$ labelled by a type $A$ we define
$\mathcal{M}_T = M_A$ and $\sim_T = id_{M_A}$. For other cases:
    \begin{align*}
        \mathcal{M}_{T^\dagger} &= \mathcal{M}_{T} + M_{type(T^\dagger)}
    &
        \mathcal{M}_{\Lambda(T)} &= \mathcal{M}_{T} + M_{type(\Lambda(T))}
    \\
        \sim_{T^\dagger} &= \left( \sim_{T}
        \union \left(type\ T^\dagger \leftrightarrow  type\ T\right)
        \right)^\star
    &
        \sim_{\Lambda(T)} &= \left( \sim_{T}
        \union \left(type\ \Lambda(T) \leftrightarrow type\ T\right)
        \right)^\star
    \end{align*}
    \begin{align*}
        \mathcal{M}_{\langle T_1^{C^1 \multimap A^1}, T_2^{C^2 \multimap B^2}\rangle}
        &= \mathcal{M}_{T_1} + \mathcal{M}_{T_2} + M_{C \multimap (A \otimes B)}
    \\
         \sim_{\langle T_1^{C^1 \multimap A^1}, T_2^{C^2 \multimap B^2}\rangle} &= \left( \sim_{T_1}
        \union \sim_{T_2} \union (C^1 \leftrightarrow C) \union (C^2 \leftrightarrow C)
        \union (A^1 \leftrightarrow A) \union (B^2 \leftrightarrow B)
        \right)^\star
    \\
    \\
        \mathcal{M}_{T_1^{A^1 \multimap B^1}\otimes T_2^{C^2 \multimap D^2}} &= \mathcal{M}_{T_1} +  \mathcal{M}_{T_2} + M_{A \otimes C \multimap B \otimes D }
        \\
         \sim_{T_1^{A^1 \multimap B^1}\otimes T_2^{C^2 \multimap D^2}} &= \left( \sim_{T_1}
        \union \sim_{T_2} \union (A^1 \leftrightarrow A)
        \union (B^1 \leftrightarrow B) \union (C^2 \leftrightarrow C)\union (D^2 \leftrightarrow D)
        \right)^\star
    \\
    \\
        \mathcal{M}_{T_1^{A \multimap B};T_2^{B \multimap C}} &=
            \mathcal{M}_{T_1} + \mathcal{M}_{T_2} + M_{A\multimap C}
        \\
         \sim_{T_1^{A^1 \multimap B^1};T_2^{B^2 \multimap C^2}} &= \left( \sim_{T_1}
        \union \sim_{T_2} \union (A^1 \leftrightarrow A)
        \union (B^1 \leftrightarrow B^2) \union (C \leftrightarrow C^2)
        \right)^\star
    \end{align*}
    where $A\leftrightarrow B$ denotes the implicit bijection between
    two isomorphic arenas $\sem{A}$ and $\sem{B}$; $R^\star$
    denotes the smallest superset of the relation $R$ complete
    by transitivity, reflexivity and symmetry.
\end{definition}

We call \defname{internal move} of the game $\intersem{T}$, any move
from $M_T$ which is not $\sim$-equivalent to any move in
$M_{type(T)}$.


A \defname{justified interaction sequence} of moves on the
interaction game $\intersem{T}$ is a sequence of moves from $M_T$
together with pointers. In contrast to the standard notion of
justified sequence, to each move in the sequence can be attached
several pointers. More precisely, if the equivalence class $m$ is
$\{m_1, \ldots, m_l \}$ then $m$ has one pointer for each
non-initial move $m_i$ in the equivalence class.

\begin{definition}[Filtering] We define several filtering operations
over justified interaction sequences. Let $s$ be a justified
sequence of moves on the interaction game $\intersem{T}$.
\begin{itemize}
\item  Let $T'$ be a subtree of $T$. We define the
filtering operator $s\upharpoonright T'$ to be the subsequence
of $s$ consisting of moves $\sim$-equivalent to some move in
$M_{T'}$. This operation causes some move to ``lose'' some of
their attached pointers: a given move $m$ with equivalence class
$\{m_1, \ldots, m_l \}$ may have up to $l$ pointers, but in
$s\upharpoonright T'$, only pointers associated to a $m_i$
belonging to $\mathcal{M}_{T'}$ are preserved.

Note that since $M_T$ is a set of equivalence classes with
respect to $\sim$, the filtering operator $\_ \filter T'$
implicitly performs the ``retagging'' of the moves to the
appropriate components of each game of the interaction game
$\intersem{T'}$.

\item  For any sub-game $A$ of the standard game $\sem{type(T')}$ we
define the filtering operator $s\upharpoonright A$ to be the
subsequence of $s$ consisting of moves from $A$ where at most
one pointer is kept for each move in the sequence: the one
corresponding to the class citizen from $A$.

\item For any initial move $m$ of the game $\sem{type(T)}$ occurring in $s$, $s
\hjfilter m$ is the subsequence of $s$ consisting of moves
that are \emph{hereditarily justified} by that particular occurrence of $m$ in $s \filter type(T)$.
% NOTE: it is important to precise ``in $s \filter type(T)$'' because $s$'justification
% pointers differs depending on the sub-interaction game considered.

%\item For any initial move $m$ of the game $\sem{type(T)}$, $s
%\hefilter m$ is the subsequence of $s$ consisting of moves
%that are \emph{hereditarily enabled} by $m$ in the game $\sem{type(T)}$.
\end{itemize}
By extension, we also define these operations on sets of justified
interaction sequences.
\end{definition}

Allowing moves to have multiple pointers complicates slightly the
presentation here, but this capability is necessary to model
strategy composition. Indeed, in game semantics after composing
strategies, the pointers from some moves may change! (See definition
of $\filter A,C$ in \cite{abramsky:game-semantics-tutorial}.)
However, for all the other operations on strategies that we will
used, the pointers will just be preserved. Formally we define this
property as follows: Let $s$ be an  interaction sequence on a game
$\intersem{T}$, $T'$ a direct subtree $T$ ({\it i.e.}~a subtree of
$T$ whose root is a child of $T$'s root), $A$ be a sub-game of
$\sem{type(T)}$ and $A'$ be a sub-game of $\sem{type(T')}$, then we
define the predicate $A'\stackrel{s}\hookrightarrow A$ as:
\begin{align*}
 A'\stackrel{s}\hookrightarrow A \mbox{ holds iff } &
 \Pstr{s_1\ (n){n'}\ s_2\ (m-n){m'}\ s_3 } = s\filter A'  \\
 & \implies \exists! m,n \in A | m \sim m' \zand n \sim n' \zand \Pstr{s_1\
(n){n}\  s_2\ (m-n){m}\ s_3} = s\filter A
\end{align*}

and we say that $s$'s justification is preserved from $A'$ to $A$
with respect to $\sim$.



\begin{definition}[Legal uncovered positions] We recall
that in the standard game semantics, the set of legal positions
$L_A$ of a game $A$ is the set of justified sequences of moves from
$M_A$ respecting visibility and alternation. We define the set of
\defname{legal uncovered position} $L_T$ of an interaction game $\intersem{T}$ as
follows:
    \begin{itemize}
    \item If $T$ is a leaf annotated by a type $A$ then $L_T =
    L_A$;
    \item If $T$ is a unary node with child node $T'$ then:
    $$L_T = \{ s \in JustSeq(T) \ | \ s \filter type(T) \in L_{type(T)} \zand  s \filter T' \in L_{T'} \} \ ;$$
    \item If $T$ is a binary node with children nodes $T_1$ and $T_2$ then:
    $$L_T = \{ s \in JustSeq(T) \ | \ s \filter type(T) \in L_{type(T)} \zand  s \filter T_1 \in L_{T_1}
    \zand  s \filter T_2 \in L_{T_2} \} \ .$$
    \end{itemize}
    where $JustSeq(T)$ denotes the set of justified interaction sequences on
    $\intersem{T}$.
\end{definition}

Revealed strategies can alternatively be represented as by means
of sets of uncovered positions:
\begin{definition}[Revealed strategies as set of uncovered positions]
\label{dfn:revealedstrat}
The set of uncovered positions of a revealed strategy is defined inductively on the
structure of the annotated interaction type tree underlying the
interaction strategy:
\begin{itemize}[-]
\item Leaf labelled with type $A$ and annotated by the strategy $\sigma$: The set of positions of the revealed strategy is precisely the set of positions of the standard strategy $\sigma$.

\item Tensor product, pairing, promotion, currying:
\begin{eqnarray*}
(\Sigma_1 : \intersem{T_1}) \otimes (\Sigma_2 : \intersem{T_2}) : \intersem{T} &=\{ s \in L_T \ | \  &s \filter T_1 \in \Sigma_1 \zand\ s \filter T_2 \in \Sigma_2 \\
&& \zand\ type(T_1)\stackrel{s}\hookrightarrow type(T) \\
&& \zand\ type(T_2)\stackrel{s}\hookrightarrow type(T) \}
\\ \\
\langle \Sigma_1 : \intersem{T_1}, \Sigma_2 : \intersem{T_2} \rangle : \intersem{T} &= \{ s \in L_T \ | &
   ( (s \filter T_1 \in \Sigma_1 \zand\ s \filter T_2 = \epsilon) \\
&&  \   \zor ( s \filter T_1 = \epsilon \zand s \filter T_2 \in \Sigma_2)) \\
&& \zand\ type(T_1)\stackrel{s}\hookrightarrow type(T) \\
&& \zand\ type(T_2)\stackrel{s}\hookrightarrow type(T) \}
\\ \\
(\Sigma' : \intersem{T'})^\dagger : \intersem{T} &= \{ s \in L_T \ | \ &
\mbox {for all occurrence $m$ in $s$ of an initial  }\\
&& \mbox{ $\sem{type(T)}$-move, $(s \filter m) \filter T' \in \Sigma'$} \\
&& \zand\ type(T')\stackrel{s}\hookrightarrow type(T) \}
\\ \\
\Lambda(\Sigma' : \intersem{T'}) : \intersem{T} &= \{ s \in L_T \ | & s \filter T' \in \Sigma' \ \zand\ type(T')\stackrel{s}\hookrightarrow type(T) \}
\end{eqnarray*}

\item Uncovered composition $(\Sigma_1 : \intersem{T_1})\ ;^U\ (\Sigma_2
:\intersem{T_2})$ defined on the game $\intersem{T}$ where
$type(T) = A \multimap C$, $type(T_1) = A^1 \multimap B_0 \times
\ldots \times B_l$ and $type(T_2) = B_0 \times \ldots \times B_l
\multimap C^2$. We first define
\begin{eqnarray*}
\Sigma_1 \| \Sigma_2 &= \{ u \in L_T  \ | \ & u \upharpoonright T_1 \in \Sigma_1 \mbox{ and } u \upharpoonright T_2 \in \Sigma_2 \\
&& \zand\ C^2\stackrel{u}\hookrightarrow C\ \zand\ (A^1)^-\stackrel{u}\hookrightarrow A^-  \\
&& \zand\ \parbox[t]{8cm}{for any initial $m$ in $A^1$, if $m$ is justified in $u \filter type(T_1)$ by $b\in B_j$,
itself justified by $c \in C^2$ in $u \filter type(T_2)$ then $m$ justified by $c$ in $u \filter type(T)$ \} }
\end{eqnarray*}
where $A^-$ denotes the set of non-initial moves of the game $A$. We can now define composition as:
$$ \Sigma_1 ;^U \Sigma_2 = \{ cover(u,(0..l)\setminus U) \ | \ u \in \Sigma_1 \| \Sigma_2 \}$$
where $cover(u,C) = u \filter \left( M_T \setminus \Union_{j\in
C} B_j \right)$ {\it i.e.}~the subsequence of $u$ obtained by
removing moves in $\Union_{j\in C} B_j$. Hence
$\Sigma_1;^{\{0..l\}} \Sigma_2 = \Sigma_1 \| \Sigma_2$.

In other words $\Sigma_1 ;^U \Sigma_1$ is the set of uncovered
plays obtained by performing the usual composition while
ignoring and copying the internal moves from arenas in
$\intersem{T_1}$ or $\intersem{T_2}$ and preserving any internal
move produced by the composition in some component $B_k$ for $k
\in U$.
\end{itemize}
\end{definition}

\begin{remark} \hfill
\label{rem:interstrat}
\begin{enumerate}[i.]
\item We observe that for all strategy operator
except composition, pointers associated to moves are preserved.
For strategy composition, additional pointers are
``created'' only for initial $A$-moves.
\item It is straightforward to generalize the pairing operator $\langle \Sigma_1, \Sigma_2 \rangle$ to more than two parameters: an interaction strategy $\langle \Sigma_1, \ldots, \Sigma_p \rangle$ for $p\geq2$
is defined on an interaction game whose root node has $p$ children.
\end{enumerate}
\end{remark}

We write $\mathcal{I}$ for the set of all revealed strategies. Note
that $\mathcal{I}$ is not a category since composition is not
associative and there is no identity interaction strategy.


\begin{lemma}[Complete interaction sequence]
\label{lem:inter_complete}
Let $u$ be an interaction sequence of some interaction strategy $\Sigma : \intersem{T}$
and suppose that the standard strategy denoting the leaves of $\Sigma$ are all well-bracketed.

Then for any node/leaf game $A$ of $T$ and interaction sequence $u\in \Sigma$ we have:
\begin{itemize}[i.]
\item $u \filter A$ is well-bracketed;

\item If $u \filter type(T)$ is complete (all question moves answered) then
    $u \filter A$ is complete.
\end{itemize}
\end{lemma}
\begin{proof}
By induction on the structure of the interaction game $\intersem{T}$. The base case is
trivial. We only treat composition, the other cases being trivial: Let $ u \in \Sigma_1 ; ^U \Sigma_2$ for some $U \subseteq \nat$ with
$\Sigma_1 : \intersem{T_1^{A\multimap B}}$ and $\Sigma_2 : \intersem{T_2^{B\multimap C}}$.

i. During composition, pointers attached to answer moves are preserved with respect to $\sim$
thus non-well-bracketing of $u\filter A\multimap C$ implies
either non-well-bracketing of $u\filter A\multimap B$ or $u\filter B\multimap C$.

For ii., suppose $u \filter type(T) = \Pstr{(q)q\ u'\ (a-q)a }$.
By well-bracketing (i.) and since $q$ and $a$ belong to $C$ we must have
$u \filter B\multimap C = \Pstr{(q)q \ldots (a-q)a}$ thus $u \filter B\multimap C$ is complete.
Suppose that $u \filter A\multimap B$ is not complete, then its first move is unanswered,
but since this is a $B$-move, it must also occur unanswered in $u \filter B\multimap C$ which is a contradiction
since we have just prove that $u \filter B\multimap C$ is complete. Thus $u \filter A\multimap B$  is also complete.

The induction hypothesis permits to conclude.
\end{proof}
Consequently if $u\filter type(T)$ is complete then $u$ is maximal {\em i.e.~no move (and in particular no internal move) can be played after $u$}.

\subsubsection{Modeling the $\lambda$-calculus in $\mathcal{I}$}

We would like to use revealed strategies from $\mathcal{I}$ to model terms of
the simply-typed lambda calculus.
Depending on the internal moves that we wish to hide, we obtain different possible interaction strategies for a given term.
The following definition fixes a unique strategy denotation which is computed from the $\eta$-normal form of the term.

\begin{definition}[Revealed denotation of a term]
\label{dfn:interactionstrategy_ofterms}
Let $\pi_i$ denote the $i^{th}$ projection copycat strategy $\pi_i : \sem{X_1 \times \ldots \times X_l} \rightarrow \sem{X_i}$.

The \defname{revealed game denotation} or \emph{revealed strategy} of
$M$ written $\intersem{\Gamma \vdash M : A}$ is defined as
$\sem{\Gamma \vdash M : A}$ if $M$ is in $\beta$-normal form, otherwise
it is defined by structural induction on the \emph{$\eta$-long normal form of $M$}:
\begin{eqnarray*}
\intersem{\Gamma \vdash \lambda \overline{\xi} . M  : A} &=& \Lambda^{|\overline{\xi}|}(\intersem{\Gamma, \overline{\xi} \vdash M : o })  \\
\intersem{\Gamma  \vdash x_i N_1 \ldots N_p :o} &=& \langle \pi_i, \intersem{\Gamma \vdash N_1 : A_1}, \ldots, \intersem{\Gamma \vdash N_p : A_p}  \rangle \fatsemi ^{\{1..p\}} ev^p \\
\intersem{\Gamma \vdash f N_1 \ldots N_p : o} &=& \langle \intersem{\Gamma \vdash N_1 : A_1}, \ldots, \intersem{\Gamma \vdash N_p : A_p} \rangle^\dagger\  \|\ \sem{f} \\
\intersem{\Gamma \vdash N_0 \ldots N_p : o} &=& \langle \intersem{\Gamma \vdash N_0 : A_0}, \ldots, \intersem{\Gamma \vdash N_p : A_p}  \rangle^\dagger\ \|\ ev^p
\end{eqnarray*}
where $\Gamma = x_1 : X_1 \ldots x_l : X_l$, $f : A_0$ is a $\Sigma$-constants, $p\geq 1$, $A_0 =
(A_1,\ldots,A_p,o)$, $ev^p$ denotes the evaluation strategy with
$p$ parameters and $X_i = A_0$ in the second equation.
\end{definition}

Figure \ref{fig:interaction_strategy_denotations} contains tree representations of the interaction games of the revealed strategy $\intersem{\Gamma \vdash M : A}$ for the application cases. These tree tell us all the information that we need about the strategy involved in $\intersem{M}$. For instance the revealed strategy $\Sigma$ is defined on the interaction arena $\intersem{T^{00}}$ whose root is $!A^0 \multimap B^0$; the strategy $ev$ is defined on the interaction arena $\intersem{T^1}$ with a single arena-node $!B^1 \multimap C^1$; thus plays of $ev$ do not contain uncovered moves.


    \begin{figure}[htbp]
        $$
        \tree[levelsep=6ex,thistreesep=3cm]{\TR{\intersem{N_0 N_1 \ldots N_p :o}:T [!A\multimap C]}}
                {   \tree[levelsep=6ex]{\TR{\Sigma^\dagger:T^0[!A^0\multimap !B_0^0\otimes \ldots \otimes !B_p^0 ]}}
                        {
                            \tree[levelsep=6ex,thistreesep=3cm]{\TR{\Sigma:T^{00}[!A^{00}\multimap B_0^{00}\times \ldots \times B_p^{00}]}}
                            {
                                \tree[levelsep=6ex]{\TR{\intersem{N_0}:T^{000}[!A^{000}\multimap B_0]}}{\Tfan[fansize=10ex]}
                                \TR{\ldots}
                                \tree[levelsep=6ex]{\TR{\intersem{N_p}:T^{00p}[!A^{00p}\multimap B_p]}}{\Tfan[fansize=10ex]}
                            }
                        }
                    \TR{ ev:T^1[!B_0^1 \otimes \ldots \otimes !B_p^1 \multimap C] }
                }
       $$
       \begin{center}
       \emph{Tree-representation of the revealed strategy $\intersem{\Gamma \vdash N_0 N_1 \ldots N_p :o}$.}
       \end{center}

        $$
        \tree[levelsep=6ex,thistreesep=3cm]{\TR{\intersem{x_i N_1 \ldots N_p :o}:T [!A\multimap C]}}
                {   \tree[levelsep=6ex]{\TR{\Sigma^\dagger:T^0[!A^0\multimap !B_0^0\otimes \ldots \otimes !B_p^0 ]}}
                        {
                            \tree[levelsep=6ex,thistreesep=3cm]{\TR{\Sigma:T^{00}[!A^{00}\multimap B_0^{00}\times \ldots \times B_p^{00}]}}
                            {
                                \TR{\pi_i:T^{000}[!A^{000}\multimap B_0]}
                                \tree[levelsep=6ex]{\TR{\intersem{N_1}:T^{001}[!A^{001}\multimap B_1]}}{\Tfan[fansize=10ex]}
                                \TR{\ldots}
                                \tree[levelsep=6ex]{\TR{\intersem{N_p}:T^{00p}[!A^{00p}\multimap B_p]}}{\Tfan[fansize=10ex]}
                            }
                        }
                    \TR{ ev:T^1[!B_0^1 \otimes \ldots \otimes !B_p^1 \multimap C] }
                }
        $$
       \begin{center}\emph{Tree-representation of the revealed strategy $\intersem{\overline{x}:\overline{X}\vdash x_i N_1 \ldots N_p :o}$}
       \end{center}
    \bigskip
    {\small
     Node labels are of the form $\Pi : T' [A]$ where $\Pi$ is a strategy, $T'$ is the corresponding interaction game and $A$ is the standard game lying at the root of the interaction game $T$. The games $A$, $B$ and $C$ are defined as follows:
    \begin{eqnarray*}
        A &=& \Gamma = X_1 \times \ldots \times X_n\\
        B &=& \underbrace{((B_1' \times \ldots \times B_p') \rightarrow o')}_{B_0} \times B_1 \times \ldots \times B_p\\
        C &=& o \ .
    \end{eqnarray*}
    Games are annotated with string  $s \in \{ 0..p \}^*$ in the exponent to indicate the path from the root to the corresponding node in the tree (each number in $s$ indicates which direction to take at the corresponding branch point).
   }
        \smallskip
       \caption{Tree-representation of the revealed strategy in the application case.}
      \label{fig:interaction_strategy_denotations}
    \end{figure}


\begin{remark}
When computing an interaction strategy of the form
$\intersem{y_i N_1 \ldots N_p}$ for some variable $y_i$, the
internal moves of $N_1$, \ldots, $N_p$ are preserved however the
internal moves produced by the copy-cat projection strategy denoting
$y_i$ are omitted.
\end{remark}

\begin{example}
Take the term $\lambda x . (\lambda f . f x) (\lambda y . y)$.
%Its computation tree is:
%$$
%\tree{\lambda x} {
%    \pstree[levelsep=4ex]{\TR{@}}
%    {       \pstree[levelsep=4ex]{\TR{\lambda f}}
%                { \tree{f}{  \tree{\lambda}{ \TR{x}  } } }
%            \pstree[levelsep=4ex]{\TR{\lambda y}}
%                    {\TR{y}}
%    } }
%$$
Its revealed strategy is $$\Lambda ( \langle \sem{ x:X \vdash \lambda f . f
x : (o\rightarrow o) \rightarrow o} , \sem{ x:X \vdash \lambda y . y
: o \rightarrow o} \rangle \| ev_2 ) \ .$$
\end{example}


\subsubsection{From interaction semantics to standard semantics and vice-versa}

In the standard semantics, given two strategies $\sigma : A
\rightarrow B$, $\tau : B \rightarrow C$ and a sequence $s \in
\sigma \fatsemi \tau$, it is possible to (uniquely) recover the
internal moves. The uncovered sequence is written ${\bf u}(s,
\sigma, \tau)$. The algorithm to obtain this unique uncovering is
given in part II of \cite{hylandong_pcf}. Therefore given a term
$M$, we can completely uncover the internal moves of a sequence
$s\in\sem{M}$ by performing the uncovering operation recursively at
every @-node of the computation tree.

Conversely, the standard semantics can be recovered from the
interaction semantics by filtering the moves, keeping only those
played in the root arena:
\begin{eqnarray}
 \sem{\Gamma \vdash M : T} = \intersem{\Gamma \vdash M : T} \upharpoonright \sem{\Gamma \rightarrow T} \label{eqn:int_std_gamsem}
\end{eqnarray}

\subsection{The correspondence theorem for the simply-typed $\lambda$-calculus without interpreted constants}
In this section, we establish a connection between the interaction
semantics of a simply-typed term without constants ($\Sigma =
\emptyset$) and the traversals of its computation tree: we show that
the set $\travset(M)$ of traversals of the computation tree is
isomorphic to the set of uncovered plays of the strategy denotation
(this is the counterpart of the ``Path-Traversal Correspondence'' of
\cite{OngLics2006}), and that the set of traversal reductions is
isomorphic to the strategy denotation.

\subsubsection{@-free traversals}

When defining computation trees, it was necessary to introduce
application nodes (labelled @) in order to connect the operator and
the operand of an application. The presence of @-nodes has also
another advantage: it ensures that the lambda-nodes are all at even
level in the computation tree, and thus a traversal respects a certain form of
alternation.

Application nodes are however redundant in the sense that they do
not play any role in the computation of the term. In fact it is
necessary to filter them out if we want to establish the
correspondence with the interaction game semantics.

\begin{definition}[@-free traversal]
\label{dfn:appnode_filter}
Let $t$ be a traversal of $\tau(M)$.
We write $t-@$ for the sequence of nodes-with-pointers obtained by
\begin{itemize}
\item removing from $t$ all @-nodes and value-leaves of some @-node;
\item replacing any link pointing to an @-node by a link pointing to the immediate predecessor of @ in $t$.
\end{itemize}

Suppose $u = t-@$ is a sequence of nodes obtained by applying the
previously defined transformation on the traversal $t$, then $t$ can
be partially recovered from $u$ by reinserting the @-nodes as
follows. For each @-node @ in the computation tree with parent node
denoted by $p$, we perform the following operations:
\begin{enumerate}
\item replace every occurrence of the pattern $p \cdot n$, where $n$ is a $\lambda$-nodes,
by $p \cdot @ \cdot n$;
\item replace any link in $u$ starting from a $\lambda$-node and pointing to $p$ by a link pointing to the inserted @-node;
\item if there is an occurrence in $u$ of a value-leaf $v_p$ pointing to $p$ then insert a value-leaf $v_@$
immediately before $v_p$ and make it point to the node immediately
following $p$ (which is also the $@$-node that we inserted in 1).
\end{enumerate}
We write $u+@$ for this second transformation.
\end{definition}
These transformations are well-defined because in a traversal, an @-node
always occurs in-between two nodes $n_1$ and $n_2$ such that  $n_1$ is the parent node of @
and $n_2$ is the first child node of @ in the computation tree:
$$      \pstree[levelsep=4ex]{\TR{n_1}\treelabel{0} }
        {
            \pstree[levelsep=3ex]{\TR{@}}
            {
                \tree{n_2}{\vdots}
                \TR[edge=\dedge]{}
                \TR[edge=\dedge]{}
            }
        }
$$
\begin{remark}
Justified sequences of nodes of the form $t-@$ for some traversal $t$ are not, strictly speaking, proper justified sequences of nodes since they do not respect alternation (two $\lambda$-nodes may become adjacent after removing a @-node)
and since any $\lambda$-node justified by @ becomes justified by @'s parent which is also a $\lambda$-node. However we will treat them just as justified sequence.
\end{remark}

\begin{lemma} \label{lem:minus_at_plus_at}
$$\forall t \in \travset(M), \quad (t-@)+@ = \left\{
            \begin{array}{ll}
              t, & \hbox{if $t^\omega \neq @$\ ;} \\
              \ip\ t, & \hbox{if $t^\omega = @$\ .}
            \end{array}
          \right.
$$
\end{lemma}
\proof
The result follows immediately from the definition of the operation -@ and +@.
\qed
\smallskip

We introduce the following notation:
$$
\travset(M)^{-@} = \{ t - @ \ | \  t \in \travset(M) \}
$$

\begin{remark}
If $M$ is $\beta$-normal then $\tau(M)$ does not contain any
@-node therefore all nodes are hereditarily justified by $r$ and we
have $\travset(M)^{-@} = \travset(M) = \travset(M)^{\upharpoonright
r }$.
\end{remark}

\paragraph{Mapping @-free traversals to interaction plays}
\hfill

\notetoself{
\begin{definition}[Mapping from nodes to moves]\hfill
    \label{def:theta mapping}
    Let $T$ be the interaction game of the interaction strategy $\intersem{M}$ and
    $M_T$ be the set of equivalence class of moves from $\mathcal{M}$.


    For $n \in N_{\sf prime}$, let $\Gamma(n) \vdash \kappa(n) : T(n)$ denote the subterm of $\elnf{M}$ rooted at $n$.
    We define the disjoint union of games:
    $$\mathcal{G}_M = \sem{\Gamma\rightarrow T} \quad \uplus \quad  \biguplus_{n \in N_{\sf prime} } \sem{T(n)}.$$
    $$\mathcal{G}_M = \sem{\Gamma\rightarrow T} \quad \uplus \quad  \biguplus_{n \in N_{\sf spawn} } \sem{T(n)}.$$

    We define the function $\varphi_M: V_\lambda \union V_{\sf var} \rightarrow M_T$
    as:
    \begin{equation*}
        \varphi_M = \psi_{M} \quad \union \Union_{n \in N_{\sf prime}} \psi_{n}
    \end{equation*}
    where $q_0$ denotes $\sem{\Gamma\rightarrow T}$'s initial move.

    We omit the subscript in $\varphi_M$ if it does not cause any ambiguity.
\end{definition}


$\varphi_M$ is indeed totally defined on $V_\lambda \union V_{\sf var} = V\setminus (V_@ \union V_\Sigma)$ (since a node is either hereditarily justified by the root, by a @-node or by a $\Sigma$-node).

\begin{remark}
\label{rem:phi_preserves_her_enabling}
$\varphi_M$ \defname{preserves hereditary enabling}: a node $n$ is hereditarily
 enabled by some node $n' \in N \inter E \relimg{N_@ \union N_\Sigma}$ in $\tau(M)$ if and only if
 $\varphi(n)$ and $\varphi(n')$  are both played in the same game $A \in \mathcal{G}$ and
the move $\varphi_M(n)$ is hereditarily enabled by $\varphi_M(n')$ in $A$.
\end{remark}

%If $t$ is a justified sequence of nodes in $V_\lambda \union V_{\sf var}$ then $?(\varphi(t)) =
%\varphi(?(t))$.
%where $?(\varphi(t))$ denotes the subsequence of $\varphi(t)$ consisting of the unanswered questions
%and $?(t)$ denotes the subsequence of $t$ consisting of the unmatched nodes (see the
%definition in section \ref{sec:adding_value_leaves}).

}

As we observed in a previous remark, sequences from $\travset(M)^{-@}$ are not, strictly speaking, proper justified sequences. Consequently the filtering operators introduced up to now are undefined on $\travset(M)^{-@}$. We now introduce a new filtering operation on $\travset(M)^{-@}$:
\begin{definition}
Let $\Delta \vdash \kappa(n) : A$ be some subterm of $\elnf{M}$ for some $n\in N_\lambda$.
We define the \defname{subterm filtering} operator on sequences of the form $t-@$ for some traversal $t$ of $M$ as follows:
$$ (t - @) \subtermfilter \kappa(n) = (t-@)\hefilter n = t\hefilter n \ . $$
\end{definition}
Note that this is well-defined because $t-@ = t'-@$ implies $t\hefilter n = t'\hefilter n$ (since @-nodes have no justifier).
In particular we have:
$$ (t - @) \subtermfilter M = t \hefilter r  = t \hjfilter r \ .$$
(Here hereditary justification and hereditarily enabling coincide because the root node can appear at most once in a traversal.)

\begin{lemma}[Filtering lemma]
\label{lem:varphi_filter}
Let $t$ be a traversal of $M$, $\Delta \vdash N : A$ be some subterm of $\elnf{M}$ and $m$ be an occurrence of an initial $A$-move in $\varphi(t-@)$ then:
$$(i) \quad \varphi_M((t-@)\subtermfilter N) = \varphi_M(t-@) \filter \sem{\Delta\rightarrow A} \ .$$
$$(ii) \quad \varphi_M((t-@)\subtermfilter N) \hjfilter m = \varphi_M(t\hjfilter n) \ .$$
where $n$ denotes the occurrence of $\tau(N)$'s root in $t$ whose image
by $\varphi_M$ is the occurrence $m$.

Consequently:
$$(iii) \quad  \varphi_M(\travset^{-@}(M)) \filter \sem{\Gamma \rightarrow T} = \psi_M(\travset^{\filter r}(M))\ .$$
\end{lemma}
\proof Let $t$ be a traversal of $M$:
$$\begin{array}{lrclr}
\mbox{i.}& \varphi( (t-@) \subtermfilter N ) &=& \varphi_M((t-@) \hefilter n ) & \mbox{(Def. subterm filtering)}\\
          &&=& \varphi_M(t-@) \filter \sem{\Delta \rightarrow A}  & \parbox[t]{5.5cm}{(By remark \ref{rem:phi_preserves_her_enabling}, $\varphi_M$ preserves hereditary enabling,  and  moves in $\sem{\Delta \rightarrow T}$ are all hereditarily enabled by the initial move $m = \varphi_M(n)$).} \\
\\
\mbox{ii.}& \varphi_M((t-@)\subtermfilter N) \hjfilter m
  &=& \varphi_M(t\hefilter n) \hjfilter m & \mbox{(Def. subterm filtering)}\\
  &&=& ( \varphi_M(t) \hefilter \varphi_M(n) ) \hjfilter m & \parbox[t]{5.5cm}{($\varphi_M$ maps the set of nodes hered. \emph{enabled} by $n$ to the set of moves hered. \emph{enabled} by $\varphi_M(n)$)} \\
  &&=& \varphi_M(t) \hefilter  m \hjfilter m & \mbox{($m = \varphi_M(n)$)} \\
  &&=& \varphi_M(t) \hjfilter m \ . \\
  \\
\mbox{iii.} & \varphi(t-@) \filter \sem{\Gamma \rightarrow T}
             &=& \varphi( (t-@) \subtermfilter M ) & \mbox{(by i.)} \\
           &&=& \varphi( (t-@) \hefilter r ) & \mbox{(Def. subterm filtering)} \\
           &&=& \varphi( t \hefilter r ) & \mbox{(@-node are not justified).} \qed
\end{array}$$

The function $\varphi$ regarded as a function from the set of vertices $V_\lambda \union V_{\sf var}$ of the computation tree to moves in arenas is not injective.
For instance the two occurrences of $x$ in the computation tree of the term $\lambda f x. f x x$ are mapped to the same question. However
the function $\varphi$ defined on the set of traversals to interaction plays of game semantics is injective:
\begin{lemma}[$\psi$ and $\varphi$ are injective]
\label{lem:varphiinjective}
For any two traversals $t_1$ and $t_2$:
\begin{itemize}
\item[(i)] If $\varphi (t_1 - @ ) = \varphi (t_2 - @ )$ then $t_1-@ =t_2 -@$\ ;
\item[(ii)] if $\psi (t_1 \upharpoonright r ) = \psi (t_2 \upharpoonright r )$ then $t_1\upharpoonright r = t_2\upharpoonright r$\ .
\end{itemize}
\end{lemma}
\begin{proof}
For any node $n$ of a traversal $t$ let us write $ptr(n)$ to denote the distance between $n$ and its justifier node in $t$. If $n$ has not link then we set $ptr(n)=0$. We also use the same notation for sequences of moves.

\begin{lemma}[Preleminary lemma]
\label{lem:varphiinjective:prelem}
\begin{equation}
\left(
  \begin{array}{ll}
    t \cdot n_1, t \cdot n_2 \in \travset \\
    \zand\ n_1 \neq n_2
  \end{array}
\right)
 \mbox{ implies } n_1,n_2 \in N^{\upharpoonright r}_{\lambda} \zand ( \varphi(n_1) \neq \varphi(n_2) \zor ptr(n_1) \neq ptr(n_2) ) \ . \end{equation}
\end{lemma}
\begin{proof}
Let $t \cdot n_1, t \cdot n_2 \in \travset$.
First we remark that the traversal rules have a weak form of determinism which ensures that $n_1$ and $n_2$ belong to the same category of node i.e.\ they must be both in $N_{\sf var}$, $N_@$ or $N_\lambda$.

Suppose that $n_1, n_2 \in N_@$ then $t \cdot n_1$ and $t \cdot n_2$ were formed using the (App) rule. Since this rule is deterministic we must have $n_1=n_2$ which violates the second hypothesis.


Suppose that $n_1,n_2\in N_{\sf var}$. The traversals $t \cdot n_1$ and $t \cdot n_2$ must have been formed using either rule (Lam) or (App). But these two rules are deterministic and their domains of definition are disjoint. Hence again the second hypothesis is violated.

Suppose that $n_1,n_2\in N_\lambda$ then
the traversals $t \cdot n_1$ and $t \cdot n_2$ must have been formed using either rule (Root), (App), (Var) or (InputVar). Since all these rules have disjoint domains of definition, the same rule must have been use to form $t \cdot n_1$ and $t \cdot n_2$. Supposed that one of the rules (Root), (App) and (Var) has been used then since they are all deterministic we have $n_1=n_2$ which violates the second hypothesis. Consequently, the rule (InputVar) must have been used and therefore $n_1,n_2 \in N_\lambda^{\upharpoonright r}$. By definition of (InputVar), in order to have $n_1\neq n_2$ and $\varphi(n_1) = \varphi(n_2)$, the parent node of the last node in $t$ must occurs at more than one position in $\oview{t}$ and $n_1,n_2$ correspond to the child node of two different occurrences of that parent node in $\oview{t}$. But then the links associated to $n_1$ and $n_2$ will point to their respective occurrence of that parent node in $\oview{t}$ hence $ptr(n_1) \neq ptr(n_2)$.
\end{proof}

\noindent {\it (continuation of the proof of Lemma \ref{lem:varphiinjective})}

(i) The result is trivial is either $t_1$ or $t_2$ is empty.
Suppose that $t_1-@\neq t_2-@$ then necessarily $t_1 \neq t_2$, thus there are some sequences $t'$, $u_1$, $u_2$ and some nodes $n_1,n_2$ such that
 $t_1 = t' \cdot n_1 \cdot u_1$, $t_2 = t' \cdot n_2 \cdot u_2$ with either $n_1\neq n_2$ or $ptr(n_1) \neq ptr(n_2)$.

If $n_1 = n_2$ then $ptr(n_1) \neq ptr(n_2)$ therefore $n_1,n_2 \not\in N_@$ (otherwise $ptr(n_1) = 0 = ptr(n_2)$). Since $ptr(\varphi(n_1)) = ptr(n_1)$ and  $ptr(\varphi(n_2)) = ptr(n_2)$ we must have $\varphi(t' \cdot n_1) \neq \varphi(t' \cdot n_2)$. Since $n_1,n_2 \not\in N_@$ we also have $\varphi((t' \cdot n_1)-@) \neq \varphi((t' \cdot n_2)-@)$. Hence $\varphi(t_1-@) \neq \varphi(t_2-@)$.

If $n_1 \neq n_2$ then by Lemma \ref{lem:varphiinjective:prelem} we have $n_1,n_2 \not\in N_@$ and $\varphi(n_1) \neq \varphi(n_2)$ or $ptr(n_1) \neq ptr(n_2)$ which again implies $\varphi(t_1-@) \neq \varphi(t_2-@)$.


(ii) Suppose that $t \upharpoonright r \neq t' \upharpoonright r$ then necessarily $t \neq t'$ which in turn implies that for some sequences $t_1'$, $t_2'$, $u_1$, $u_2$ and some nodes $n_1 \neq n_2$
we have $t_1 = t' \cdot n_1 \cdot u_1$, $t_2 = t' \cdot n_2 \cdot u_2$ and either $n_1\neq n_2$ or $ptr(n_1) \neq ptr(n_2)$.

If $n_1 = n_2$ then $ptr(n_1) \neq ptr(n_2)$. An   analysis of the traversal rules shows that the rule (InputVar) is the only rule which can visit the same node with two different pointers. Hence $n_1,n_2 \in N_\lambda^{\upharpoonright r}$.
Therefore $\psi( (t'\cdot n_1) \upharpoonright r ) = \psi( (t'\upharpoonright r) \cdot n_1 )  \neq \psi( (t'\upharpoonright r) \cdot n_2 )$. Hence    $\psi( t_1\upharpoonright r ) \neq \psi( t_2\upharpoonright r )$.

If $n_1 \neq n_2$ then we can use Lemma \ref{lem:varphiinjective:prelem}
to obtain $\psi( t_1\upharpoonright r ) \neq \psi( t_2\upharpoonright r )$.
\end{proof}

\begin{corollary} \
\label{cor:varphi_bij}
\begin{itemize}
\item[(i)] $\varphi$ defines a bijection from $\travset(M)^{-@}$
to $\varphi(\travset(M)^{-@})$\ ;
\item[(ii)] $\psi$ defines a bijection from $\travset(M)^{\upharpoonright r}$ to
$\psi(\travset(M)^{\upharpoonright r})$\ .
\end{itemize}
\end{corollary}

\subsubsection{The correspondence theorem}
We now state and prove the correspondence theorem for the
simply-typed $\lambda$-calculus without interpreted constants
($\Sigma = \emptyset$). The result extends immediately to the
simply-typed $\lambda$-calculus with \emph{uninterpreted} constants
since we can regard constants as being free variables.

\begin{lemma}[Local Traversal Extension]
\label{lem:local_traversal_progression}
Let $M'$ be a subterm of $M$, $t \in \travset(M)$,
$t' \in \travset(M')$ such that $t' \neq \epsilon$ and $t\subseqof t'$. If the
traversal $t' \cdot n$ of $\tau(M')$ can be formed using a rule different from \rulenamet{InputVar}
and $\rulename{InputVar^{val}}$ then either $t' \cdot n \subseqof t $ or
$ t' \cdot n \in \travset(M)$.
where $n$'s link in $t \cdot n$ points to the same node occurrence as in $t' \cdot n$.
\end{lemma}
\proof
By Case analysis on the traversal rule used to form $t'\cdot n$.
\qed

This lemma says that extending a traversal locally also extends the traversal globally: the traversal $t$ of $M$ can be extended by extending a ``sub-traversal'' $t'$ of some sub-term $M'$.
This is not obvious since $t'$ is a subsequence of $t$ which means that
the nodes in $t'$ are also present in $t$ with the same pointers but with some other nodes interleaved in between. However these interleaved nodes are inserted in a preservative way which allows us to apply the rule used to extend $t'$ on $t$.

The following theorem establishes a correspondence between the
game-denotation of a term and the set of traversals of its
computation tree:
\begin{theorem}[The Correspondence Theorem]
\label{thm:correspondence}
 For any simply-typed term $\Gamma \vdash M :T$,
the function $\varphi_M$ defines a bijection from $\travset(M)^{\upharpoonright
r}$ to $\sem{\Gamma \vdash M : T}$ and a bijection from
$\travset(M)^{-@}$ to $\intersem{\Gamma \vdash M : T}$:
\begin{eqnarray*}
 \varphi_M  &:& \travset(\Gamma \vdash M : T)^{-@} \stackrel{\cong}{\longrightarrow} \intersem{\Gamma \vdash M :T} \\
 \psi_M  &:& \travset(\Gamma \vdash M : T)^{\upharpoonright r} \stackrel{\cong}{\longrightarrow} \sem{\Gamma \vdash M :T} \ .
\end{eqnarray*}

\end{theorem}

%\begin{proposition}
%\label{prop:rel_gamesem_trav} Let $\Gamma \vdash M : T$ be a
%simply-typed $\lambda$-term and $r$ be the root of $\tau(M)$. Then:
%\begin{itemize}
%\item[(i)]  $\varphi_M(\travset(M)^{-@}) = \intersem{\Gamma \vdash M : T}$ \ ;
%\item[(ii)] $\varphi_M(\travset(M)^{\upharpoonright r}) = \sem{\Gamma \vdash M : T}$ \ .
%\end{itemize}
%\end{proposition}

\begin{remark}
\label{rem:corresp_proofreduction}
    By corollary \ref{cor:varphi_bij}, we just need to show that
    $\varphi_M$ defines \emph{surjections}, that is to
    say:
    \begin{eqnarray*}
    \varphi_M(\travset(M)^{-@}) &=& \intersem{\Gamma \vdash M : T} \\
    \psi_M(\travset(M)^{\upharpoonright r}) &=& \sem{\Gamma \vdash M :
    T}
    \end{eqnarray*}
    The first equation implies the second one, indeed:
    \begin{align*}
    \sem{\Gamma \vdash M : T} &= \intersem{\Gamma \vdash M : T} \upharpoonright \sem{\Gamma \rightarrow T} & \mbox{(eq. \ref{eqn:int_std_gamsem})} \\
            &= \varphi_M(\travset^{-@}(M)) \upharpoonright \sem{\Gamma \rightarrow T} & \mbox{(by (i))}\\
            &= \psi_M(\travset^{\upharpoonright r}(M)) & \mbox{(lemma \ref{lem:varphi_filter})}
    \end{align*}
    therefore we just need to prove the first equation.
\end{remark}

    Let us give a brief overview of the proof before giving it in full details.
    It proceeds by induction on the structure of the computation tree.
    The only non-trivial case is the application: the computation tree
    $\tau(M)$ has the following form:
        $$ \tree[levelsep=4ex]{\lambda \overline{\xi}}
            { \tree[levelsep=4ex]{@}
                {   \TR{\tau(N_0)} \TR{\ldots} \TR{\tau(N_p)}}}
        $$

    A traversal of $\tau(M)$ proceeds as follows: it starts at the root $\lambda \overline{\xi}$ of the tree $\tau(M)$ (rule
    (Root)), it then passes the node @ (rule (Lam)).
    After this initialization part, it proceeds by traversing the term $N_0$ (rule (App)).
    At some point, while traversing $N_0$, some variable $y_i$ bound by the root of $N_0$ is visited. The traversal
    of $N_0$ is interrupted and jumps (rule (Var)) to the root of $\tau(N_i)$. The process then goes on with $\tau(N_i)$.
    When traversing $N_i$, if the traversal encounters a variable bound by the root of $\tau(N_i)$ then the traversal of $N_i$
    is interrupted and
    the traversal of $N_0$ resumes.  This schema is repeated until the traversal of $\tau(N_0)$ is completed\footnote{Since we are considering
    simply-typed terms, the traversal does indeed terminate. However this will not be true anymore in the \pcf\ case.}.

    The traversal of $M$ is therefore made of an initialization part followed by an interleaving of a traversal of $N_0$ and
    several traversals of $N_i$ for $i=1..p$. This schema is reminiscent of the way the evaluation copycat map $ev$ works in game semantics.

    The key idea is that every time the traversal pauses the traversal of a subterm and switches to another one,
    the jump is permitted by one of the four ``copycat'' rules (Var), (Answer-@-$\lambda$), (Answer-$\lambda$-var) or (Answer-var).
    We show by (a second) induction that these copycat rules define precisely what the copycat strategy $ev$ performs on sets of plays.

%    In the game semantics, the evaluation map (a copy-cat strategy) copies this opening move to an initial move $m_0$ in the game
%    $B_0$ and the game continues in $B_0$. We reflect this in the traversal : we make $t$ follow
%    the ``script'' given by the traversal $t^0_{m_0}$.
%    The rule (App) allow us to initiate this simulation  by visiting the  first move in $t^0_{m_0}$: the root of $\tau(N_0)$.
%
%    This simulation continues until it reaches a node $\alpha_0$ which is hereditarily justified by the root
%    $\tau(N_0)$: $\alpha_0$ is present in the reduction of traversal of $t^0_{m_0}$ therefore $\varphi_{N_0}(\alpha_0)$ is an un-hidden move played in $A_0$.
%
%    In the game semantics this corresponds to a move played in a component $A_k$ for some $k\in 1..p$ of
%    of the game $B_0$ in which case the evaluation map copies the move to an initial move $m_1$ in the corresponding component $B_k$.
%
%    To reflect this the traversal now opens up a new thread and simulates the traversal $t^k_{m_1}$.  Again, this simulation stops when we reach a node
%    $\alpha_1$ in $t^k_{m_1}$ which is hereditarily justified by the root of $\tau(N_k)$: $\alpha_1$ must be present in the reduction of traversal
%    of $t^k_{m_1}$ therefore $\varphi_{N_k}(\alpha_1)$ is an un-hidden move played in $A_k$.
%    In the game semantics, this move $\alpha$ is copied back to the component $B_k$ of the game $B_0$.
%
%    The traversal now resumes the simulation of $t^0_{m_0}$. And the process goes continuously.
\smallskip

\begin{proof}
Let $\Gamma \vdash M : T$ be a simply-typed term where $\Gamma =
x_1:X_1,\ldots x_n:X_n$. We assume that $M$ is already in
$\eta$-long normal form. By remark \ref{rem:corresp_proofreduction} we just need to
show that $\varphi_M(\travset(M)^{-@}) = \intersem{\Gamma \vdash M : T}$.
We proceed by induction on the structure of $M$:
\begin{enumerate}[$\bullet$]
    \item (abstraction) $M \equiv \lambda \overline{\xi}. N : \overline{Y} \rightarrow B$ where $\overline{\xi} = \xi_1:Y_1,\ldots \xi_n:Y_n$. On the first hand we have:
\begin{eqnarray*}
\intersem{\Gamma \vdash \lambda \overline{\xi}. N:T} &=& \Lambda^n( \intersem{\overline{\xi}, \Gamma \vdash N: B } ) \\
        &\simeq& \intersem{\overline{\xi}, \Gamma \vdash N: B } \ .
\end{eqnarray*}
On the other hand, the computation tree $\tau(N)$ is isomorphic to
$\tau(\lambda \xi_1\ldots \xi_n . N)$ (up to a renaming of the root
of the computation tree) and $\travset(N)$ is isomorphic to
$\travset(\lambda \xi_1\ldots \xi_n . N)$.
Hence we can conclude using the induction hypothesis.

  \item (variable) $M \equiv x_i$. Since $M$ is in $\eta$-long normal form, $x$ must be of ground
      type. The computation tree $\tau(M)$ and the arena $\intersem{\Gamma \rightarrow o}$ are represented below
      (value leaves and answer moves are not represented):
        $$ \tree[levelsep=6ex]{ \lambda }{\TR{x_i}} \hspace{2cm}
        \tree{ q_0 }
        {   \tree[linestyle=dotted]{q^1}{\TR{} \TR{} }
            \tree[linestyle=dotted]{q^2}{\TR{} \TR{} }
            \TR{\ldots}
            \tree[linestyle=dotted]{q^n}{\TR{} \TR{} }
        }
        $$

        Let $\pi_i$ denote the $i$th projection of the interaction game
        semantics. We have:
        \begin{align*}
        \intersem{M} &= \pi_i = \prefset(\{ \Pstr{(q0){q_0} \cdot (qi){q^i} \cdot (vqi-qi){v_{q^i}} \cdot (vq0-q0){v_{q_0}} } \ | \ v\in \mathcal{D} \})\ .
        \end{align*}

        It is easy to see that traversals of $M$ are precisely
        the prefixes of $ \Pstr{ (lmd)\lambda \cdot (xi){x_i}
        \cdot (vxi-xi){v_{x_i}} \cdot (vlmd-lmd){v_{\lambda}}}$.
        $M$ is in $\beta$-normal therefore $\travset(M)^{-@} =
        \travset(M)$ and since $\varphi_M(\lambda) =
        q_0$ and $\varphi_M(x_i) = q^i$, we have:
        $$ \varphi_M(\travset^{-@}(M)) = \varphi_M(\travset(M)) = \varphi_M(\prefset( \lambda \cdot x_i \cdot v_{x_i} \cdot v_{\lambda}))
         = \intersem{M} \ .
        $$


    \item (application) $M = N_0 N_1 \ldots N_p :o$ where $N_0$ is not a variable.
    We have the typing judgments $\Gamma \vdash N_0 N_1 \ldots
    N_p : o$ and $\Gamma \vdash N_i : B_i$ for $i\in 0..p$ where
    $B_0 = (B_1,\ldots,B_p,o)$ and $p\geq 1$.

    The tree $\tau(M)$ has the following form:
    $$ \tree[levelsep=6ex]{\lambda^{[r]}}
        { \tree[levelsep=6ex]{@}
            {
            \tree[levelsep=3mm,edge=\noedge]{\TR{{\lambda y_1 \ldots y_p}^{[r_0]}}}{\Tr[ref=t]{\pstribox{\tau(N_0)}}}
            \tree[levelsep=3mm,edge=\noedge]{\TR{[r_1]}}{\Tr[ref=t]{\pstribox{\tau(N_1)}}}
             \TR{\ldots}
            \tree[levelsep=3mm,edge=\noedge]{\TR{[r_p]}}{\Tr[ref=t]{\pstribox{\tau(N_p)}}}
        }}
    $$
    where $r_j$ denote the root of $\tau(N_j)$ for $j\in \{0..p\}$.

    We have:
    $$
    \intersem{\Gamma \vdash M : o}
            =  \underbrace{\langle \intersem{\Gamma \vdash N_0 : B_0}, \ldots \intersem{\Gamma \vdash N_p : B_p} \rangle}_{\Sigma} \,^\dagger\ \| \ ev
    $$

    We define the games $A$, $B$ and $C$ are defined as follows:
    \begin{eqnarray*}
        A &=& \Gamma = X_1 \times \ldots \times X_n\\
        B &=& \underbrace{((B_1' \times \ldots \times B_p') \rightarrow o')}_{B_0} \times B_1 \times \ldots \times B_p\\
        C &=& o \ .
    \end{eqnarray*}

    Figure \ref{fig:interaction_strategy_denotations} shows
    a tree-representation of $\intersem{M}$ which fixes the names of the different games involved in the interaction strategy.

%    Since $\varphi_M = \psi_M \union \varphi_{N_0} \union
%    \varphi_{N_1}$ the induction hypothesis gives us:
%    \begin{align}
%    \varphi_{M} (\travset^{-@}(N_0)) &= \intersem{\Gamma \vdash N_0 : B_0} \label{eqn:ih_1} \\
%    \varphi_{M}(\travset^{-@}(N_1)) &= \intersem{\Gamma \vdash N_1 : B_1} \label{eqn:ih_2}
%    \end{align}
\begin{enumerate}
\item[$\subseteq$]
    We first prove that $\intersem{\Gamma \vdash M : T}
    \subseteq \varphi_{M}( \travset^{-@}(M) )$. Suppose $u \in
    \intersem{\Gamma \vdash M : T}$. We give a constructive
    proof that there is a traversal $t$ of $M$ such
    that $\varphi_M(t-@) = u$ by induction on the length of $u$.
    Let $q_o$ and $q_0'$ be the initial question of $C$
    and $B_0$ respectively.

    \emph{Base cases}:
    \begin{compactitem}[-]
    \item If $u=\epsilon$ then we take the empty traversal $t=\epsilon$ formed
with \rulenamet{Empty}. Clearly $\varphi(t) = u$.
    \item If $|u|=1$ then $u=q_0$ is the initial move in $C$. The traversal $t=\lambda$ formed with the rule \rulenamet{Root} verifies $\varphi(t) = u$.
    \item If $|u|=2$ then necessarily $u = q_0 \cdot q_0'$. The rules \rulenamet{Root}, \rulenamet{App}
and \rulenamet{Lam} permit us to build the traversal $t = \lambda^{[r]} \cdot @ \cdot \lambda \overline{y}^{[r_0]}$ which clearly verifies $\varphi_M(t-@) = u$.
    \end{compactitem}

    \emph{Step cases}: Suppose that $u = w \cdot m \in \intersem{\Gamma \vdash M : T}$
    for some move $m \in M_T$ where
    $w = \varphi_M(t-@)$ for some traversal $t$ of $\tau(M)$
    and $|w|>1$.

    By unraveling the definition of $u \in \intersem{\Gamma \vdash M : T}$ we have:
    \begin{eqnarray*}
      &&      \left\{
            \begin{array}{ll}
                u \in L_T\\
                u \upharpoonright T^0  \in \Sigma^\dagger \\
                u \upharpoonright T^1  \in  ev
            \end{array}
            \right. \\
    & \mbox{or equivalently} & \left\{
    \begin{array}{ll}
        u \in L_T \\
        \hbox{for any initial $m$ in $!B_0^0 \otimes \ldots \otimes !B_p^0$ there is $j \in \{0..p\}$ such that } \\
        \left\{\begin{array}{ll}
            u \filter m \filter T^{00j} \in \intersem{N_j} \label{eq:def_z} \\
            u \filter m \filter T^{00k} = \epsilon \quad \mbox{ for every } k\in \{1..p\}\setminus\{j\} \label{eq:b}
        \end{array}
        \right. \\
        u \upharpoonright T^1  \in  ev
    \end{array}
    \right.
    \end{eqnarray*}

We recall that $m \in M_T$ is an equivalence class of moves from $\mathcal{M}_T$. For any game $A$ appearing in the interaction game $T$ we will write ``$m \in A$'' to mean that some citizen of the class $m$ belongs to the set of moves $M_A$. Similarly, for any sub-interaction game $T'$ of $T$, we write ``$m \in T'$'' to mean that some citizen of the class $m$ belongs to the set of moves $\mathcal{M}_A$.

We do a case analysis on $m$: we either have $m\in C$ or $m\in T^0$:
    \begin{enumerate}[-]
    \item Suppose $m \in C$. $m$ is played by the strategy $ev$ whose plays do not contain any internal move. Hence $m$ is either $q_0$ or $v_{q_0}$ for some
    $v\in\mathcal{D}$. But since $q_0$ can occur only once in
    $u$ and $|u|>1$, $m$ must be $v_{q_0}$ for some
    $v\in \mathcal{D}$.  Moreover $m$ is a P-move played by the
    copy-cat strategy $ev$ in $B,C$ therefore it is the copy
    of the some move $v_{q_0'}$ answering the question $q_0'$ in the sub-game $o'$.

    In fact this move $v_{q_0'}$ is precisely $w$'s last move. Indeed
    suppose that $w = \ldots v_{q_0'} \cdot w'$. The play
    $w_{\prefixof v_{q_0'}}\filter A,B$ is complete since its
    first move $q_0'$ is answered by $v_{q_0'}$. Therefore by
    Lemma \ref{lem:inter_complete}(ii), $w_{\prefixof
    v_{q_0'}}\filter T^0$ is maximal. Thus moves in $w'$ must
    be played in $T^1$ by $ev$, but since $ev$ does not play internal
    moves, $w'$ is necessarily empty.

    Consequently, by the induction hypothesis, the last move in $t$ is $\varphi(v_{q_0'}) = v_{\lambda y_1}$.
    The rules \rulenamet{Answer-@-$\lambda$} and \rulenamet{Answer-$\lambda$-@} permits us to extend
    the traversal $t$ into $t \cdot v_@ \cdot v_{\lambda \overline{\xi}}$ where $v_@$ and $v_{\lambda
    \overline{\xi}}$ point to the second and first node of $t$ respectively. Clearly we have $\varphi_M((t\cdot v_@ \cdot v_{\lambda \overline{\xi}})-@) = u$.

    \item Suppose $m\in T^0$. Then $m$ is hereditarily justified by some initial move $b$ in $B_j$ for some $j\in \{0..p\}$.

        Since $u \filter b \filter T^{00j} \in \intersem{N_j}$, the outermost induction hypothesis gives us:
        \begin{equation}
        u \filter b \filter T^{00j} = \varphi_{N_j}(t_j-@)  \label{eqn:corresp_outmost_ih}
        \end{equation}
          for some traversal $t_j \in \travset(N_j)$. W.l.o.g we can assume that $t_j^\omega \neq @$.
        By Corollary \ref{cor:varphi_bij}, $\varphi_{N_j}$ is a bijection from $\travset(N_j)^{-@}$ to
        $\varphi_{N_j}( \travset(N_j)^{-@})$ therefore $t_j-@ = \varphi^{-1}_{N_j}( (u \filter b) \filter T^{00j} )$
        and we have:
        \begin{align}
         t_j - @ &= \varphi^{-1}_{N_j}(u \filter b \filter T^{00j}) &  \nonumber \\
                        &\in \varphi^{-1}_{M}(u \filter b \filter T^{00j}) & \parbox[t]{7cm}{($\varphi_M = \psi_M \union \Union_{k\in \{0..p\}} \varphi_{N_k}$ by definition.)} \label{eqn:proof_corres_1}
        \end{align}

        Note that $\varphi^{-1}_{M}(\ip(u \filter b \filter T^{00j}))$ is not a traversal but \emph{a set of} traversals since $\varphi_{M}^{-1}$ is not necessarily bijective on $\intersem{N_j}$. Thus we have to use set-membership in the equation instead of traversal equality.

        Since $\varphi^{-1}_{M}$ is monotonous and $u \filter b \filter T^{00j} \subseqof u \subseqof w$, all the traversals in $\varphi^{-1}_{M}(u \filter b \filter T^{00j})$ are subsequences of $\varphi^{-1}_{M}( w )$ thus:
        \begin{align*}
        t_j - @ &\subseqof \varphi^{-1}_{M}( w ) & \mbox{(by Eq. \ref{eqn:proof_corres_1})}\\
                &= t -@ & \mbox{($\varphi_{M}$ is bijective on $\travset(M)$ by Cor. \ref{cor:varphi_bij}).}
        \end{align*}

      Thus by Lemma \ref{lem:minus_at_plus_at}(ii) we have
        $\ip( t_j) \subseqof t$.

    Furthermore:
     \begin{align*}
    t_j\subseqof t
        &\implies \varphi_M (t_j) \subseqof \varphi_M (t) \\
        &\iff (w \cdot m) \filter b \filter T^{00j} \subseqof  w \\
        &\iff (w \filter b \filter T^{00j}) \cdot m \subseqof  w  & \mbox{ ($m$ is h.j. by $b$ and belongs to $T^{00j}$)} \\
        &\implies \left( \left(w \filter b \filter T^{00j} \cdot m \right) \filter b \filter T^{00j} \right) \cdot m \subseqof w & \mbox{ (by iterating the previous equation)} \\
        &\iff (w \filter b \filter T^{00j}) \cdot m  \cdot m \subseqof  w \ .
    \end{align*}
    The last equation is false since a given move cannot occur twice consecutively in a legal interaction play! Hence $t_j\not\subseqof t$.

    \begin{enumerate}[(a)]
    \item  Suppose $t_j$'s last move is \emph{not} visited by the rule \rulenamet{InputVar} nor
        $\rulename{InputVar^{val}}$. Since $\ip( t_j) \subseqof t$ and $t_j$ is a traversal of the subterm $N_j$ of $M$, by Lemma \ref{lem:local_traversal_progression}
        we have either $t_j\subseqof t$ or $t \cdot t_j^\omega$ is a traversal of $M$
        where $t_j^\omega$'s pointer is the same as in $t_j$. Hence, since $t_j\not\subseqof t$, $t \cdot t_j^\omega$ is a traversal of $M$.

        Furthermore we have
        \begin{align*}
            \varphi_M (t_j^\omega) &= (\varphi_M (t_j-@))^\omega & \mbox{($t_j^\omega \neq @$ by assumption)}\\
                                   &= ((w \cdot m) \filter b\filter T^{00j})^\omega & \mbox{(by Eq. \ref{eqn:corresp_outmost_ih})}\\
                                   &= ((w \filter b\filter T^{00j}) \cdot m))^\omega & \mbox{($m$ is h.j. by $b$ and belongs to $T^{00j}$)}\\
                                   &= m
        \end{align*}
        and therefore
        \begin{align*}
          \varphi_{M}((t \cdot t_j^\omega)-@)  &=  \varphi_{M}(t -@)  \cdot \varphi_{M}(t_j^\omega-@)\\
                &=   w \cdot \varphi_{M}(t_j^\omega-@) & \mbox{(by the innermost induction hypothesis)}\\
                &=   w \cdot m & \mbox{(by the previous equation).}
        \end{align*}
        Hence the traversal $t \cdot t_j^\omega$ meets the requirement.

    \item Suppose $t_j$'s last move is visited with the rule \rulenamet{InputVar}.

    Then $t_j$ is of the form
    $$\Pstr[18pt]{ t_j = t' (z)z \cdot t'' \cdot (n-z){t_j^\omega}}$$
for some $z \in N_\lambda^{\filter r_j}$ (see remark \ref{rem:inputvar})
and some input-variable $x \in N^{\filter r_j}_{var}$ occurs in $z\cdot t'$ such that $x$ is the pending node in $\ip t_j = t' \cdot z \cdot t''$ ({\it i.e.}~ with $?(t' \cdot z \cdot t'')^\omega = x$).

Suppose that $z\in N^{\filter r}$ then $z$ is a free variable of $M$ (and $N_j$).
Since the O-view of $t_j$ coincides with the O-view of $t$

    \item Suppose $t_j$'s last move is visited with the rule $\rulename{InputVar^{val}}$.
    This case is similar to the previous one but the rule $\rulename{InputVar^{val}}$ is used instead
    of $\rulename{InputVar}$.
    \end{enumerate}



\notetoself{PIECE OF OLD PROOF
%   \item Suppose that $m,m^1 \in T^{000}$.
%    The strategy $ev$ is responsible for switching of thread
%    in $B_0$ therefore, in the interaction semantics, there
%    must be a copycat move in-between two moves belonging to
%    two different threads. Since $m$ and $m^1$ are
%    consecutive moves in the sequence $u$, they must belong
%    to the same thread i.e. there are hereditarily justified
%    by the same initial $m_0$ in $B_0$.

Suppose that $m \in T^{000}$ and $m^1 \in T^{001}$.

    $t$ is obtained from $t-@$ by applying the
    transformation $+@$. We apply the same transformation to
    $u$ in order to make $O$-questions and $P$-questions in
    $u$ match with $\lambda$-nodes and variable nodes in
    $t'$ respectively. We write this sequence $u+@$. The
    $+@$ operation inserts nodes in the sequence but not at
    the end, therefore $m^1$, the last move in $u'$, is also
    the last move in $u'+@$. Let us note $n^1$ for the last
    move in $t'$.


        $n^1$ is a variable node then $m^1$ is a P-move and $m$ is an O-move
            and therefore $m$ is the copy of $m^1$ duplicated in $B_1$ by the evaluation strategy.
            Consequently, $m^1$ points to some $m^2$ and $m$ points to the node preceding $m^2$ denoted by $m^3$.
            The diagram below shows an example of such sequence:
                $$
                \begin{array}{ccccccccccc}
                  & A & \longrightarrow & ( (B_1' &\rightarrow & o') & \times & B_1 ) & \longrightarrow & o' \\
                  &&& &&&&&& \rnode{q0}{q_0 (\lambda \overline{\xi})} & O\\
                  &&& &&&&&  \\
                O &&& && \rnode{q1}{q_0' (\lambda \overline{y})} &&&&& P \\
                P &&& \rnode{m3}{m^3 (y_1)} &&&&&&& O \\
                O &&& &&&& \rnode{m2}{m^2 (\lambda \overline{z}^{[r_1]})} &&& P \\
                P &&& &&&& \rnode{m1}{m^1 (z)} &&& O \\
                O &&& \rnode{m}{m} &&&&&&& P \\
                \end{array}
                \ncline[nodesep=3pt]{->}{q1}{q0} \mput*{@}
                \nccurve[nodesep=3pt,ncurv=2,angleA=180,angleB=180]{->}{m1}{m2}
                \ncarc[nodesep=3pt,ncurv=1,angleA=90,angleB=180]{->}{m3}{q1}
                \ncarc[nodesep=3pt,ncurv=1,angleA=90,angleB=180]{->}{m}{m3}
                \ncline[nodesep=3pt]{->}{m2}{q0}
                $$

        $t'$  and $u+@$ have the following forms:
        \begin{eqnarray*}
                t'&=& \Pstr{ \ldots \cdot n^3 \cdot (n2){n^2} \cdot \ldots \cdot (n1-n2,30){n^1} } \\ \\
                u+@ &=& \Pstr{ \ldots \cdot (m3){m^3} \cdot (m2){m^2} \cdot \ldots \cdot (m1-m2,30){m^1} \cdot (m-m3,30){m} }
        \end{eqnarray*}

        Since $n^1$ is a variable node, $n^2$ must be a $\lambda$-node.
        $n^3$ is either a variable node or an @-node. In fact $n^3$ is necessarily a variable node. Indeed,
        $n^3$ is mapped to $m^3$ by $\varphi_{N_0}$ and $m^3$ belongs to $B_i'$ (i.e. it is not
        an internal move of $T^0$). The function $\varphi_{N_0}$ is defined in such a way that
        only nodes which are hereditarily justified by $r_0$ are mapped to nodes in $B_j'$.
        Consequently, since @-node don't have justifier, $n^3$ cannot be an @-node.

        Hence $n^1$ is a variable node, $n^2$ is a $\lambda$-node and $n^3$ is a variable node.


        We  can therefore apply the (Var) rule to $t'$ and we obtain a traversal of the following form:

        \begin{eqnarray*}
            t&=& \Pstr{ \ldots \cdot (n3){n^3} \cdot (n2){n^2} \cdot \ldots \cdot (n1-n2,30){n^1} \cdot (n-n3,30){n} }
        \end{eqnarray*}

        We have $\varphi(t'-@) = u'$ by the induction hypothesis and $\varphi(n) = m$ by definition of $\varphi$.
        Therefore since $m$ and $n$ point to the same position we have $\varphi(t-@) = u$.
}

    \end{enumerate}

\item[$\supseteq$]
  For the converse, $\varphi_{M}( \travset^{-@}(M) ) \subseteq \intersem{M}$, it is an easy induction
  on the traversal rules. We omit the details here.


\end{enumerate}


    \item (application') $M = x_i N_1 \ldots N_p :o$ with $X_i = B_0 = (B_1' \times \ldots \times B_p') \rightarrow o'$. The tree $\tau(M)$ has the following form:
    $$ \tree[levelsep=6ex]{\lambda^{[r]}}
        { \tree[levelsep=6ex]{x_i}
            {
            \tree[levelsep=3mm,edge=\noedge]{\TR{[r_1]}}{\Tr[ref=t]{\pstribox{\tau(N_1)}}}
             \TR{\ldots}
            \tree[levelsep=3mm,edge=\noedge]{\TR{[r_p]}}{\Tr[ref=t]{\pstribox{\tau(N_p)}}}
        }}
    $$
    The interaction strategy
    $\intersem{\Gamma \vdash M : o}
            =  \underbrace{\langle \pi_i, \intersem{\Gamma \vdash N_1 : B_1}, \ldots \intersem{\Gamma \vdash N_p : B_p} \rangle}_{\Sigma} \,^\dagger\ ;^{\{1..p\}} \ ev$
    is represented on Figure \ref{fig:interaction_strategy_denotations}.

    The proof is identical to the previous case except that in the $\subseteq$ part of the proof:
    \begin{itemize}
        \item In the base case of the induction where $|u|=2$,
        the rule \rulenamet{InputVar} is used instead of \rulenamet{App} to visit the node $x$ instead of $@$;
        \item in the step case of the induction, for the subcase $m\in C$, the rules \rulenamet{Answer-var-$\lambda$} and \rulenamet{Answer-$\lambda$-var} are used instead of \rulenamet{Answer-@-$\lambda$} and \rulenamet{Answer-$\lambda$-@} respectively;
        \item in the step case $m\in T^0$, when $m$ is hereditarily justified by a move $b \in B_j$ for
         $j\in \{1 .. p\}$ the proof remains unchanged. The case where $m$ is hereditarily justified by a move $b \in B_0$ is treated as follows: $m$ is played by the projection strategy $\pi$ denoting $x$.
         Since $m$ is played in $B_0' = B_1' \times \ldots \times B_p'$ it must be also hereditarily justified by some initial move $b'$ of $B_k'$ for some $k \in \{1.. p\}$ or by an initial move in $o'$. But moves of $B_k'$ are $\sim$-equivalent to the corresponding move in $B_k$ and similarly $o'$ is $\sim$-equivalent to $o$, therefore we fall back to the previous case of the induction where $m$ is hereditarily justified by some initial move $b\in B_k$ for $k\in \{1..p\}$ or some initial move in $o$!
    \end{itemize}


\end{enumerate}


\end{proof}


\begin{corollary} \hfill
\begin{enumerate}[i.]
\item Let $\tau(M')$ be a subtree of $\tau(M)$ for some subterm $M'$ of $M$, and  $N'$ denote the set
of nodes of $\tau(M')$. Then
$$t \in \travset(M) \implies t\filter N' \in \travset(M') \ .$$

\item If $M$ is in $\beta$-normal form then for any traversal $t$,
$\varphi_M(t)$ is a maximal play if and only if $t$ is a maximal
traversal.
\end{enumerate}
\end{corollary}
\begin{proof}
\begin{enumerate}[i.]
\item
 \todo
\item If $M$ is in $\beta$-normal form then
$\travset(M)^{\upharpoonright r} = \travset(M)$ therefore
$\varphi$ defines a bijection on $\travset(M)$. Let $t$ be a
traversal such that $\varphi(t)$ is a maximal play. Let $t'$ be
a traversal such that $t \sqsubseteq t'$. By monotonicity of
$\varphi$ we have $\varphi(t) \sqsubseteq \varphi(t')$ which
implies $\varphi(t) = \varphi(t')$ by maximality of $\varphi(t)$
which in turn implies $t'=t$ by injectivity of $\varphi$. The
other direction is proved identically using injectivity and
monotonicity of $\varphi^-1$.
\end{enumerate}
\end{proof}
\smallskip The following diagram recapitulates the main results of
this section:
$$
\xymatrix @C=6pc{
                                           & \travset(M)^{-@} \ar@/_/[dl]_{+@}  \ar[r]^{\varphi_M}_\cong & \intersem{M} \ar@/_/[dd]_{\_ \upharpoonright \sem{\Gamma\rightarrow T}} \\
\travset(M) \ar@/_/[ur]_{-@}^{} \ar[dr]^{\_ \upharpoonright r}  \\
                                           & \travset(M)^{\upharpoonright r} \ar[r]^{\varphi_M}_\cong & \sem{M} \ar@/_/[uu]^{\cong}_{\mbox{full uncovering}}
}
$$


\begin{example}
Take $M = \lambda f z . (\lambda g x . f x) (\lambda y. y) (f z) :
((o,o),o, o)$.  The figure below represents the computation tree
(left tree), the arena $\sem{((o,o),o, o)}$ (right tree) and
$\psi_M$ (dashed line). (Only question moves are shown for clarity.)
The justified sequence of nodes $t$ defined hereunder is an example
of traversal:

\begin{tabular}{lp{6.3cm}}
$\tree[levelsep=2.5ex,treesep=0.3cm]{ \Rnode{root}{\lambda f z} }
     {  \tree{@}
        {   \tree{\lambda g x}{
                  \tree{\Rnode{f}{f^{[1]}}}{
                            \tree{\Rnode{lmd}{\lambda^{[2]}}}
                            {\TR{x}}
                  }
                }
            \tree{ \lambda y }{\TR{y}}
            \tree{\lambda ^{[3]}}{
                \tree{\Rnode{f2}{f^{[4]}}} {
                \tree{\Rnode{lmd2}{\lambda^{[5]}}}{\TR{\Rnode{z}{z}}}
                }
            }
        }
     }
\hspace{1cm}
  \tree[levelsep=8ex,treesep=0.3cm]{ \Rnode{q0}q^0 }
    {   \pstree[levelsep=4ex]{\TR{\Rnode{q1}{q^1}}}{\TR{\Rnode{q2}{q^2}}}
        \TR{\Rnode{q3}q^3}
        \TR{\Rnode{q4}q^4}
    }
\psset{nodesep=1pt,arrows=->,arrowsize=2pt 1,linestyle=dashed,linewidth=0.3pt}
\ncline{->}{root}{q0} \mput*{\psi_M}
\ncarc[arcangle=-25]{->}{z}{q3}
\ncarc[arcangle=10]{->}{f}{q1}
\ncarc[arcangle=10]{->}{lmd}{q2}
\ncline{->}{f2}{q1}
\ncline{->}{lmd2}{q2}$
\hspace{2cm}
&
\begin{asparablank}
  \item  \Pstr[0.8cm]{
t = (n){\lambda f z} \
(n2){@} \
(n3-n2,60){\lambda g x} \
(n4-n,45){f^{[1]}} \
(n5-n4,45){\lambda^{[2]}} \
(n6-n3,45){x} \
(n7-n2,35){\lambda^{[3]}} \
(n8-n,35){f^{[4]}} \
(n9-n8,45){\lambda^{[5]}} \
(n10-n,35){z}
}

\item \Pstr[0.9cm]{
t\upharpoonright r = (n){\lambda f z} \ (n4-n,50){f}^{[1]} \
(n5-n4,60){\lambda}^{[2]} \ (n8-n,45){f}^{[4]} \
(n9-n8,60){\lambda}^{[5]} \ (n10-n,40){z}}
\item
\Pstr[0.8cm]{ {\psi_M(t\upharpoonright r) =\ } (n){q^0}\
(n4-n,60){q^1}\ (n5-n4,60){q^2}\ (n8-n,45){q^1}\ (n9-n8,60){q^2}\
(n10-n,38){q^3} \in \sem{M}\ .}
\end{asparablank}
\end{tabular}
\end{example}


    \section{Extension to PCF and IA terms}
         \input{corresp_pcf_ia.texi}


    \section{Applications}


\chapter{Game-Semantic Analysis via a Syntactic Argument}
    \label{chap:syntactic_gamesem}
    
\section{A game-semantic account of safety}
\label{sec:gamesemaccount} Our aim is to characterize safety by game
semantics. We shall assume that the reader is familiar with the
basics of game semantics; For an introduction, we recommend
\cite{abramsky:game-semantics-tutorial}. Recall that a
\emph{justified sequence} over an arena is an alternating sequence
of O-moves and P-moves such that every move $m$, except the opening
move, has a pointer to some earlier occurrence of the move $m_0$
such that $m_0$ enables $m$ in the arena. A \emph{play} is just a
justified sequence that satisfies Visibility and Well-Bracketing. A
basic result in game semantics is that $\lambda$-terms are denoted
by \emph{innocent strategies}, which are strategies that depend only
on the \emph{P-view} of a play. The main result
(Theorem~\ref{thm:safeincrejust}) of this section is that if a
$\lambda$-term is safe, then its game semantics (is an innocent
strategy that) is, what we call, \emph{P-incrementally justified}. In such a
strategy, pointers emanating from the P-moves of a play are uniquely
reconstructible from the underlying sequence of moves and pointers
from the O-moves therein: specifically a P-question always points to
the last pending O-question (in the P-view) of a greater order.

The proof of Theorem~\ref{thm:safeincrejust} depends on a
Correspondence Theorem (see the Appendix) that relates the strategy
denotation of a $\lambda$-term $M$ to the set of \emph{traversals}
over a souped-up abstract syntax tree of the $\eta$-long form of $M$.
In the language of game semantics, traversals are just (concrete
representations of) the \emph{uncovering} (in the sense of Hyland
and Ong \cite{hylandong_pcf}) of plays in the strategy denotation.

The useful transference technique between plays and traversals was
originally introduced by one of us \cite{OngLics2006} for studying
the decidability of monadic second-order theories of infinite structures generated by
higher-order grammars (in which the $\Sigma$-constants or terminal symbols are at most
order 1, and \emph{uninterpreted}).
% In this setting, free variables are interpreted
% as constructors and therefore they do not have the ``full power'' of
% true free variables and are limited to order $1$ at most. Also,
% although the grammar can perform higher-order computations, the
% structure being studied is itself of ground type.
In the Appendix, we present an extension of this framework to the
general case of the simply-typed lambda calculus with free variables
of any order. A new traversal rule is introduced to handle nodes
labelled with free variables. Also new nodes are added to the
computation tree to account for the answer moves of the game
semantics, thus enabling the framework to model languages with
interpreted constants such as \pcf~(by adding traversal rules to
handle constant nodes).

\subsection*{Incrementally-bound computation tree}
 In \cite{OngLics2006} the computation tree of a grammar is
defined as the unravelling of a finite graph representing the \emph{long
transform} of a grammar. Similarly we define the computation tree of
a $\lambda$-term as an abstract syntax tree of its $\eta$-long
normal form.  We write $l\langle t_1, \ldots, t_n \rangle$ with $n
\geq 0$ to denote the ordered tree with a root labelled $l$ with $n$
child-subtrees $t_1$, \ldots, $t_n$. In the following we consider arbitrary
simply-typed terms.

\begin{definition}\rm
\label{dfn:comptree}
  The \defname{computation tree} $\tau(M)$ of a simply-typed term
  $\Gamma \stentail M:T$ with variable names in a countable set
  $\mathcal{V}$ is a tree with labels in $$ \{ @ \} \union \mathcal{V}
  \union \{ \lambda x_1 \ldots x_n \ | \ x_1 ,\ldots, x_n \in
  \mathcal{V}, n\in\nat \}$$ defined from its $\eta$-long form as follows. Suppose $\overline{x} = x_1 \ldots x_n$ for $n\geq 0$ then
\begin{eqnarray*}
  \mbox{for $m\geq 0$, $z \in \mathcal{V}$: } \tau(\lambda \overline{x} . z s_1 \ldots s_m : o) &=& \lambda \overline{x} \langle z \langle\tau(s_1),\ldots,\tau(s_m)\rangle\rangle \\
  \mbox{for $m \geq 1$: } \tau(\lambda \overline{x} . (\lambda y.t) s_1 \ldots s_m :o) &=& \lambda \overline{x} \langle @ \langle \tau(\lambda y.t),\tau(s_1),\ldots,\tau(s_m) \rangle \rangle \ .
\end{eqnarray*}
\end{definition}

\begin{example}
\label{examp:comptree}
  Take $\stentail \lambda f^{o \typear o} .
(\lambda u^{o \typear o} . u) f : (o \typear o) \typear
o \typear o$.
\bigskip

\noindent
\begin{tabular}{cc}
Its $\eta$-long normal form is: & Its computation tree is:\\[8pt]
\begin{minipage}{0.45\textwidth}
\centering
$\begin{array}{ll}
 &\stentail  \lambda f^{o \typear o} z^o . \\
&\qquad(\lambda u^{o \typear o} v^o . u (\lambda.v)) \\
&\qquad(\lambda y^o. f y) \\
&\qquad(\lambda.z) \\
&: (o \typear o) \typear o \typear o
\end{array}$
\end{minipage}
&
\begin{minipage}{0.45\textwidth}
\centering
\psset{levelsep=5ex,linewidth=0.5pt,nodesep=1pt,arcangle=-20,arrowsize=2pt 1}
${\pstree{\TR{\lambda f z}}{\pstree{\TR{@}}{\pstree{\TR{\lambda u v}}{\pstree{\TR{u}}{\pstree{\TR{\lambda }}{\TR{v}}}}\pstree{\TR{\lambda y}}{\pstree{\TR{f}}{\pstree{\TR{\lambda }}{\TR{y}}}} \pstree{\TR{\lambda }}{\TR{z}}}}
}$
\end{minipage}
\end{tabular}
\end{example}

\begin{example}
  Take $\stentail \lambda u^o v^{((o \typear o) \typear o)} . (\lambda x^o . v (\lambda z^o . x)) u : o \typear ((o \typear o) \typear o) \typear o$.
  \bigskip

\noindent
\begin{tabular}{cc}
Its $\eta$-long normal form is: & Its computation tree is:\\[8pt]
\begin{minipage}{0.45\textwidth}
\centering
$\begin{array}{ll}
 &\stentail  \lambda u^o v^{((o \typear o) \typear o)} . \\
&\qquad(\lambda x^o . v (\lambda z^o . x)) u \\
&: o \typear ((o \typear o) \typear o) \typear o
\end{array}$
\end{minipage}
&
\begin{minipage}{0.45\textwidth}
\centering
$\pstree{\TR{\lambda u v}}{\pstree{\TR{@}}{\pstree{\TR{\lambda x}}{\pstree{\TR{v}}{\pstree{\TR{\lambda z}}{\TR{x}}}}\pstree{\TR{\lambda }}{\TR{u}}}}
$
\end{minipage}
\end{tabular}
\end{example}

Even-level nodes are $\lambda$-nodes (the root is on level 0). A
single $\lambda$-node can represent several consecutive variable
abstractions or it can just be a \emph{dummy lambda} if the
corresponding subterm is of ground type.  Odd-level nodes are
variable or application nodes.

The \defname{order} of a node $n$, written $\ord{n}$, is defined as
follows: @-nodes have order $0$. The order of a variable-node is the
type-order of the variable labelling it. The order of the root node
is the type-order of $(A_1,\ldots,A_p, T)$ where $A_1,\ldots, A_p$
are the types of the variables in the context $\Gamma$. Finally, the
order of a lambda node different from the root is the type-order of
the term represented by the sub-tree rooted at that node.

We say that a variable node $n$ labelled $x$ is \defname{bound} by a
node $m$, and $m$ is called the \defname{binder} of $n$, if $m$ is
the closest node in the path from $n$ to the root such that $m$ is
labelled $\lambda \overline{\xi}$ with $x\in \overline{\xi}$.


We introduce a class of computation trees in which the binder node
is uniquely determined by the nodes' orders:
\begin{definition}\rm
  A computation tree is \defname{incrementally-bound} if for all
  variable node $x$, either $x$ is \emph{bound} by the first
  $\lambda$-node in the path to the root with order $> \ord{x}$, or $x$
  is a \emph{free variable} and all the $\lambda$-nodes in the path to
  the root except the root have order $\leq \ord{x}$.
\end{definition}

\begin{proposition}[Safety and incremental-binding] \hfill
\label{prop:safe_imp_incrbound}
\begin{enumerate}[(i)]
\item If $M$ is safe then $\tau(M)$ is incrementally-bound.
\item Conversely, if $M$ is a \emph{closed} simply-typed term and $\tau(M)$
is incrementally-bound then $M$ is safe.
\end{enumerate}
\end{proposition}
\proof
  (i) Suppose that $M$ is safe. By Lemma
  \ref{prop:safe_iff_elnfsafe} the $\eta$-long form of $M$ is safe
  therefore $\tau(M)$ is the tree representation of a safe term.

In the safe lambda calculus, the variables in the context with the
lowest order must be all abstracted at once when using the
abstraction rule. Since the computation tree merges consecutive
abstractions into a single node, any variable $x$ occurring free in
the subtree rooted at a node $\lambda \overline{\xi}$ different from
the root must have order greater or equal to $\ord{\lambda
  \overline{\xi}}$. Conversely, if a lambda node $\lambda
\overline{\xi}$ binds a variable node $x$ then $\ord{\lambda
  \overline{\xi}} = 1+\max_{z\in\overline{\xi}} \ord{z} > \ord{x}$.

Let $x$ be a bound variable node. Its binder occurs in the path from
$x$ to the root, therefore, according to the previous observation,
$x$ must be bound by the first $\lambda$-node occurring in this path
with order $>\ord{x}$. Let $x$ be a free variable node then $x$ is
not bound by any of the $\lambda$-nodes occurring in the path to the
root. Once again, by the previous observation, all these
$\lambda$-nodes except the root have order smaller than $\ord{x}$.
Hence $\tau$ is incrementally-bound.

(ii) Let $M$ be a closed term such that $\tau(M)$ is
incrementally-bound.  W.l.o.g. we can assume that $M$ is in $\eta$-long
form.  We prove that $M$ is safe by induction on its structure. The
base case $M = \lambda \overline{\xi} . x$ for some variable $x$ is
trivial.  \emph{Step case:} If $M = \lambda \overline{\xi} . N_1
\ldots N_p$.  Let $i$ range over $1..p$. We have $N_i \equiv \lambda
\overline{\eta_i} . N'_i$ for some non-abstraction term $N'_i$. By
the induction hypothesis, $\lambda \overline{\xi} . N_i = \lambda
\overline{\xi} \overline{\eta_i} . N'_i$ is a safe closed term, and
consequently $N'_i$ is necessarily safe. Let $z$ be a free variable
of $N'_i$ not bound by $\lambda \overline{\eta_i}$ in $N_i$. Since
$\tau(M)$ is incrementally-bound we have $\ord{z} \geq \ord{\lambda
  \overline{\eta_1}} = \ord{N_i}$, thus we can abstract the variables $\overline{\eta_1}$ using \rulenamet{abs} which shows that $N_i$ is safe.  Finally
we conclude $\sentail M = \lambda \overline{\xi} . N_1 \ldots N_p :
T$ using the rules \rulenamet{app} and \rulenamet{abs}.  \qed



The assumption that $M$ is closed is necessary. For instance for
$x,y:o$, the computation trees $\tau(\lambda x y .x)$ and
$\tau(\lambda y . x)$ are both incrementally-bound but $\lambda x y
.x$ is safe and $\lambda y . x$ is not.

\subsection*{P-incrementally justified strategy}

We now consider the game-semantic model of the simply-typed lambda
calculus. The strategy denotation of a term-in-context $\Gamma
\stentail M : T$ is written $\sem{\Gamma
\stentail M : T}$. We define the \defname{order} of a move $m$,
written $\ord{m}$, to be the length of the path from $m$ to its
furthest leaf in the arena minus 1. (There are several ways to
define the order of a move; the definition chosen here is sound in
the current setting where each question move in the arena enables at
least one answer move.)
%{\it i.e.}~height of the subarena rooted at $q$ minus 2.

\begin{definition}\rm
  A strategy $\sigma$ is said to be \defname{P-incrementally
    justified} if for any play $s \, q \in \sigma$ where $q$ is a
  P-question, $q$ points to the last unanswered O-question in $\pview{s}$ with
  order strictly greater than $\ord{q}$.
\end{definition}
Note that although the pointer is determined by the P-view, the
choice of the move itself can be based on the whole history of the
play. Thus P-incremental justification does not imply innocence.

The definition suggests an algorithm that, given a play of a
P-incrementally justified denotation, uniquely recovers the pointers
from the underlying sequence of moves and from the pointers
associated to the O-moves therein. Hence:
\begin{lemma}
\label{lem:incrjustified_pointers_uniqu_recover} In P-incrementally
justified strategies, pointers emanating from P-moves are
superfluous.
\end{lemma}

\begin{example}
Copycat strategies, such as the identity strategy $id_A$ on game $A$
or the evaluation map $ev_{A,B}$ of type $(A \Rightarrow B) \times A
\typear B$, are all P-incrementally justified.\footnote{In such
strategies, a P-move $m$ is justified as follows: either $m$ points
to the preceding move in the P-view or the preceding move is of
smaller order and $m$ is justified by the second last O-move in the
P-view.}
\end{example}
%%%% the following example is wrong : ev is P-ij.
%
%\begin{example}
%Take the evaluation map $ev : (o^1 \Rightarrow o^2) \times o^3 \rightarrow o^4$ and the play $s = q^4 q^2 q^1 q^3 \in \sem{ev}$. We have $\ord{q^2} = 1 > \ord{q^1} = \ord{q^3} = 0$. Now $q^3$ points to $q^4$ but $q^2$ is the last unanswered O-question in $\pview{s}= s$ with order $>\ord{q^3}$, hence $\sem{ev}$ is not P-incrementally justified.
%\end{example}



The Correspondence Theorem~\ref{thm:correspondence}
% and Lemma \ref{lem:betanf_wellbehavedconst_trav_pview_red}
gives us the following equivalence:
\begin{proposition} % [Incremental-binding vs P-incremental justification]
\label{prop:Nher_incrbound_and_incrjustified} Let $\Gamma \stentail
M : T$ be a $\beta$-normal term. The computation tree $\tau(M)$ is
incrementally-bound if and only if $\sem{\Gamma \stentail M : T}$ is
P-incrementally justified.
\end{proposition}


\parpic[r]{
\pssetcomptree
\raisebox{-12pt}
{$\tree{\lambda^3}{\tree{f^2}{ \tree{\lambda y^1}{\TR{x^0} }}}$}
}
%\noindent \emph{Example:}
\begin{example}
Consider the $\beta$-normal term $\Gamma\stentail f (\lambda y .x) :
o$ where $y:o$ and $\Gamma =f:((o,o),o),~x:o$. The figure on the
right represents its computation tree with the node orders given as
superscripts.  The node $x$ is not incrementally-bound therefore $\tau(f
(\lambda y .x))$ is not incrementally-bound and by Proposition
\ref{prop:Nher_incrbound_and_incrjustified}, $\sem{\Gamma \stentail
f (\lambda y .x) : o}$ is not incrementally-justified (although
$\sem{\Gamma \stentail f : ((o,o),o)}$ and $\sem{\Gamma \stentail
\lambda
  y. x : (o,o)}$ are).
\end{example}
\smallskip

Propositions \ref{prop:safe_imp_incrbound} and
\ref{prop:Nher_incrbound_and_incrjustified} allow us to show the
following:
\begin{theorem}[Safety and P-incremental justification]
\label{thm:safeincrejust} \hfill
\begin{enumerate}[(i)]
\item If $\Gamma \sentail M : T$ then $\sem{\Gamma \sentail M : T}$ is P-incrementally justified.
\item If $\stentail M : T$ is a closed simply-typed term and $\sem{\stentail M : T}$ is P-incrementally justified then the $\beta$-normal form of $M$ is safe.
\end{enumerate}
\end{theorem}
\proof (i) Let $M$ be a safe simply-typed term. By Lemma
\ref{lem:safered_preserves_safety}, its $\beta$-normal form $M'$ is
also safe. By Proposition \ref{prop:safe_imp_incrbound}(i),
$\tau(M')$ is incrementally-bound and by Proposition
\ref{prop:Nher_incrbound_and_incrjustified}, $\sem{M'}$ is an
incrementally-justified. Finally the soundness of the game model
gives $\sem{M} = \sem{M'}$.  (ii) is a consequence of Lemma
\ref{lem:safered_preserves_safety}, Proposition
\ref{prop:Nher_incrbound_and_incrjustified} and
\ref{prop:safe_imp_incrbound}(ii) and soundness of the game model.
\qed



Putting Theorem \ref{thm:safeincrejust}(i) and Lemma
\ref{lem:incrjustified_pointers_uniqu_recover} together gives:
\begin{proposition}
  \label{prop:safe_ptr_recoverable} In the game semantics of safe
  $\lambda$-terms, pointers emanating from P-moves are unnecessary
  {\it i.e.}~they are uniquely recoverable from the underlying sequences of
  moves and from O-moves' pointers.
\end{proposition}

 \begin{example} If justification pointers are omitted then the denotations of the
   two Kierstead terms from Example~\ref{ex:kierstead} are not distinguishable.
   In the safe lambda calculus this ambiguity disappears
   since $M_1$ is safe whereas $M_2$ is not.
 \end{example}

In fact, as the last example highlights, pointers are superfluous at
order $3$ for safe terms whether from P-moves or O-moves. This is
because for question moves in the first two levels of an arena
(initial moves being at level $0$), the associated pointers are
uniquely recoverable thanks to the visibility condition. At the
third level, the question moves are all P-moves therefore their
associated pointers are uniquely recoverable by P-incremental
justification. This is not true anymore at order $4$: Take the safe
term $\psi:(((o^4,o^3),o^2),o^1) \sentail \psi (\lambda \varphi .
\varphi a) : o^0$ for some constant $a:o$, where $\varphi:(o,o)$.
Its strategy denotation contains plays whose underlying sequence of
moves is $q_0 \, q_1 \, q_2 \, q_3 \, q_2 \, q_3 \, q_4$. Since
$q_4$ is an O-move, it is not constrained by P-incremental
justification and thus it can point to any of the two occurrences of
$q_3$.\footnote{More generally, a P-incrementally justified strategy
can contain plays that are not ``O-incrementally justified'' since
it must take into account any possible strategy incarnating its
context, including those that are not P-incrementally justified. For
instance in the given example, there is one version of the play that
is not O-incrementally justified (the one where $q_4$ points to the
first occurrence of $q_3$). This play is involved in the strategy
composition $\sem{ \stentail M_2 : (((o,o),o),o)} ; \sem{
\psi:(((o,o),o),o) \stentail \psi (\lambda \varphi . \varphi a):o}$
where $M_2$ denotes the unsafe Kierstead term.}


\subsection*{Towards a fully abstract game model}\hfill

The standard game models which have been shown to be fully abstract
for PCF \cite{abramsky94full,hylandong_pcf} are of course also fully
abstract for the restricted language safe PCF. One may ask, however,
whether there exists a fully abstract model with respect to safe
context only.

Such model may be obtain by considering P-incrementally justified strategies
- which have been shown to compose in \cite{Blumphd}. Its is reasonable to think that
 O-moves also needs to be constrained by the symmetrical O-incremental justification, which corresponds to the requirement that contexts are safe. This line of work is still in progress.


\subsection*{Safe PCF and safe Idealised Algol}

\pcf\ is the simply-typed lambda calculus augmented with basic
arithmetic operators, if-then-else branching and a family of
recursion combinator $Y_A : ((A,A),A)$ for any type $A$.  We define
\emph{safe} \pcf\ to be \pcf\ where the application and abstraction
rules are constrained in the same way as the safe lambda calculus.
This language inherits the good properties of the safe lambda
calculus: No variable capture occurs when performing substitution
and safety is preserved by the reduction rules of the small-step
semantics of \pcf.

\subsubsection{Correspondence}

The computation tree of a \pcf\ term is defined as the least
upper-bound of the chain of computation trees of its \emph{syntactic
approximants} \cite{abramsky:game-semantics-tutorial}.  It is
obtained by infinitely expanding the $Y$ combinator, for instance
$\tau(Y (\lambda f x. f x))$ is the tree representation of the
$\eta$-long form of the infinite term $(\lambda f x. f x)
 ((\lambda f x. f x) ((\lambda f x. f x) ( \ldots$

It is straightforward to define the traversal rules modeling the
arithmetic constants of \pcf. Just as in the safe lambda calculus we
had to remove @-nodes in order to reveal the game-semantic
correspondence, in safe \pcf\ it is necessary to filter out the
constant nodes from the traversals. The Correspondence Theorem for
\pcf\ says that the interaction game semantics is isomorphic to the
set of traversals disposed of these superfluous nodes. This can
easily be shown for term approximants. It is then lifted to full
\pcf\ using the continuity of the function $\travset(\_)^{\filter
\theroot}$ from the set of computation trees (ordered by the
approximation ordering) to the set of sets of justified sequences of
nodes (ordered by subset inclusion). Finally computation trees of
safe \pcf\ terms are incrementally-bound thus we have
%Computation trees of safe \pcf\ terms are incrementally-bound.
%Moreover since \pcf\ constant are of order $1$ at most, the constant
%traversal rules are all \emph{well-behaved} (Lemma
%\ref{lem:sigma_order1_are_wellbehaved}) hence Lemma
%\ref{lem:betanf_wellbehavedconst_trav_pview_red} (from the Appendix)
%still holds and the game-semantic analysis of safety remains valid
%for \pcf. Hence we have:
\begin{theorem}
\label{thm:safepcfpincr} Safe PCF terms have P-incrementally
justified denotations. \qed
\end{theorem}


Similarly, we can define safe \ialgol\ to be safe \pcf\ augmented
with the imperative features of Idealized Algol (\ialgol\ for short)
\cite{Reynolds81}.  Adapting the game-semantic correspondence and
safety characterization to \ialgol\ seems feasible although the
presence of the base type \iavar, whose game arena $\iacom^{\nat}
\times \iaexp$ has infinitely many initial moves, causes a mismatch
between the simple tree representation of the term and its game
arena. It may be possible to overcome this problem by replacing the
notion of computation tree by a ``computation directed acyclic
graph''.

The possibility of representing plays \emph{without some or all of
  their pointers} under the safety assumption suggests potential
applications in algorithmic game semantics. Ghica and McCusker
\cite{ghicamccusker00} were the first to observe that pointers are
unnecessary for representing plays in the game semantics of the
second-order finitary fragment of Idealized Algol ($\ialgol_2$ for
short). Consequently observational equivalence for this fragment can
be reduced to the problem of equivalence of regular expressions.  At
order $3$, although pointers are necessary, deciding observational
equivalence of $\ialgol_3$ is EXPTIME-complete
\cite{DBLP:journals/apal/Ong04,DBLP:conf/fossacs/MurawskiW05}.
Restricting the problem to the safe fragment of $\ialgol_3$ may lead
to a lower complexity.

% (note that it is unlikely to obtain the complexity PSPACE because the
% set of complete plays of the safe term $\lambda f^{(o,o),o} . f
% (\lambda x^o . x)$ is not regular \cite{DBLP:journals/apal/Ong04}).

% Murawski showed the undecidability of program equivalence in
% $\ialgol_i$ for $i\geq4$ by encoding Turing machine computations
% into a finitary $IA_4$ term \cite{murawski03program}. The term
% constructed being not safe, the proof cannot be transposed to the
% safe fragments. Hence the question remains of whether observational
% equivalence is decidable for the \emph{safe} fragments of these
% language.

%In \cite{Ong02}, one of us showed that observational equivalence for
% finitary second-order \ialgol\ with recursion ($\ialgol_2 + Y_1$) is
% undecidable. The proof consists in reducing the Queue-Halting
% problem to the observational equivalence of two $\ialgol_2 + Y_1$
% terms. The same reduction is still valid in the safe fragment of
% $\ialgol_2 + Y_1$.  Consequently, observational equivalence of safe
% $\ialgol_2 + Y_1$ is also undecidable.

    \notetoself{This chapter is taken from my transfer report. I need to
rework it to integrate it correctly within the present thesis.}


Safety has been defined as a syntactical constraint. Since Game
Semantics is by essence syntax-independent, it seems difficult at
first sight to give a game-semantic characterization of a syntactic restriction such as the Safety Condition.
In fact, the Correspondence Theorem makes such analysis possible since it allows us to regard the plays of a strategy
as sequences of nodes of some AST of the term.


The main theorem of this chapter (theorem
\ref{thm:safe_ptr_recoverable}) states that pointers in a play of
the strategy denotation of a safe term can be uniquely recovered
from O-questions' pointers and from the underlying sequence of
moves. The proof is in several steps. We start by introducing the
notion of \emph{P-incrementally-justified strategies} and prove that
for plays of such strategies, pointers emanating from P-moves can be
reconstructed uniquely from the underlying sequences of moves and
from O-moves' pointers. We then introduce the notion of
\emph{incrementally-bound computation trees} and prove that
incremental-binding coincides with P-incremental-justification
(proposition \ref{prop:Nher_incrbound_iff_incrjustified}).


Finally, we show that safe simply-typed terms in $\beta$-normal form
have incrementally\--bound computation trees, consequently their
game denotation is P-incrementally-justified.


The first section of this chapter is concerned only with the safe $\lambda$-calculus without interpreted constants. In the next
section we extend the result by taking into account the interpreted
constants of \pcf\ and \ialgol. We define the language safe \ialgol\
(resp. safe \pcf) to be the fragment of \ialgol\ (resp. \pcf) where
the application and abstraction rules are constrained the same way
as in the safe $\lambda$-calculus. We show that safe \pcf\ terms are
denoted by P-incrementally-justified strategies and we give the key
elements for a possible extension of the result to Safe Idealized
Algol.

\section{Preliminaries}

In this section, we assume that we work in a general setting of a
language extending the simply-typed lambda calculus with new
constants and respecting the following prerequisites:
\begin{itemize}
\item A fully-abstract game-semantic model of the language is
defined;
\item A notion of safety is defined for the language such that the
restriction of the language to the safe pure simply-typed
fragment coincides with the definition of the Safe Lambda
Calculus and such that for any typable term $\Gamma \vdash M :
T$ we have $\forall z \in \Gamma . \ord{z} \geq \ord{T}$ ;
\item The small-step reduction semantics of the language preserves safety;
%\item Substitution preserves safety.
\item New traversal rules are defined to take into account the constants of the language.
\item Constant traversal rules are well-behaved (see Def.\
\ref{def:wellbehaved_traversal});
\item Constant traversal rules correctly model the behaviour of the constants in such a way
that the game-semantic correspondence (Theorem
\ref{thm:correspondence}) still holds.
\end{itemize}

The simply-typed lambda calculus is of course such a language, but
we will show that \pcf\ also lends itself into this setting.

For the rest of this section we fix a term $\Gamma \vdash M : T$
from this generic language. We will explicitly specify when a result
holds only in the pure (\ie no constants) simply-typed calculus
fragment of the language.

\subsection{Incremental binding}

In a computation tree, a binder node always occurs in the path from
the bound node to the root. We now introduce a class of computation
trees in which binder nodes can be uniquely recovered from the order
of the nodes. We call path any sequence of nodes such that for any
two consecutive nodes $a \cdot b$ in the sequence, $a$ is the parent
of $b$. We write $[n_1,n_2]$ to denote the path going from node
$n_1$ to node $n_2$ equipped with the justification pointers induced
by the enabling relation $\vdash$ (each node of the tree has a
unique enabler in the path to the root thus for each occurrence in
$[n_1,n_2]$ there is at most one occurrence of its enabler in
$[n_1,n_2]$). We write $]n_1,n_2]$ for the sub-sequence of
$[n_1,n_2]$ obtained by removing $n_1$ as welle as all the
associated pointers.

We recall that $\theroot$ denotes the root of the computation tree
$\tau(M)$ and $N^{\theroot\vdash}$ denotes the subset of $N$
consisting of nodes that are hereditarily enabled by $\theroot$.



\begin{definition}[Incrementally-bound computation tree]
Let $A$ be a subset of nodes of the computation tree. A variable
node $x$ of a computation tree is said to be
\defname{$A$-incrementally-bound} if its enabler is the first
$\lambda$-node from $A$ in the path to the root that has order
strictly greater than $\ord{x}$. Formally:
\begin{align*}
x \mbox{ is $A$-incrementally-bound} \  \iff \  \left\{
                                                  \begin{array}{ll}
                                                    x \hbox{ is enabled by } b \in [\theroot,x]\inter A \ ; \\
                                                    \ord{b} > \ord{x} \;\\
                                                    \forall \lambda\mbox{-node } n' \in ]n,x]\inter A  . \ord{n'} \leq \ord{x} \ .
                                                  \end{array}
                                                \right.
\end{align*}

This definition can be split into two cases:
\begin{enumerate}
\item $x$ is \emph{bound} by the first $\lambda$-node from $A$ occurring in the path to the root that has
order strictly greater than $\ord{x}$.
\item or $x$ is a \emph{free variable} and all the $\lambda$-nodes from from $A$ occurring in the path to the root except the root have order
 smaller or equal to $\ord{x}$.
\end{enumerate}

A computation tree is said to be \defname{$A$-incrementally-bound},
also abbreviated $A$-i.b., if all the variable nodes from $A$ are
$A$-incrementally-bound.

We say that a node (resp.\ a tree) is
\defname{incrementally-bound} if it is
\defname{$N$-incrementally-bound} where $N$ is the entire set of nodes of the computation tree.
\end{definition}

Clearly for any two sets of nodes $A$ and $B$ verifying $A\subseteq
B$ we have that $B$-incremental-binding implies
$A$-incremental-binding.


\smallskip

Let $\closure{M}$ denote the function that converts $M$ into the
closed term obtained from $M$ by abstracting all its free variables
(in order of appearance in the term). From the previous definition,
if $\tau(M)$ is $A$-i.b.\ then so is $\tau(\closure{M})$.

\smallskip

A node of the computation tree is said to be \defname{reachable} if
there is some traversal of the computation tree that visits it.


\begin{lemma}[Safe terms have incrementally-bound computation trees]
\label{lem:incrbound_iff_etanf_safe} Suppose that  $\Gamma \vdash M
:T$ is a simply-typed term.
\begin{itemize}
\item[(i)] If $M$ is a safe term then $\tau(M)$ is incrementally-bound ;
\item[(ii)] conversely, if $M$ is \emph{closed} and $\tau(M)$ is i.b.\ then the $\eta$-long normal form of $M$ is safe.
\end{itemize}
\end{lemma}
\begin{proof}
(i) Suppose that $M$ is safe. The safety property is preserved after
taking the $\eta$-long normal form, therefore $\tau(M)$ is the tree
representation of a safe term.

In the safe $\lambda$-calculus, the variables in the context with
the the lowest order must be all abstracted at once when using the
abstraction rule. Since the computation tree merges consecutive
abstractions into a single node, any variable $x$ occurring free in
the subtree rooted at a $\lambda$-node $\lambda \overline{\xi}$
different from the root must have order greater or equal to
$\ord{\lambda \overline{\xi}}$. Reciprocally, if a lambda node
$\lambda \overline{\xi}$ binds a variable node $x$ then
$\ord{\lambda \overline{\xi}} = 1+\max_{z\in\overline{\xi}} \ord{z}
> \ord{x}$.

Let $x$ be a bound variable node. Its binder occurs in the path from
$x$ to the root, therefore, according to the previous observation,
$x$ must be bound by the first $\lambda$-node occurring in $[r,x]$
with order strictly greater than $\ord{x}$. Let $x$ be a free
variable node then $x$ is not bound by any of the $\lambda$-nodes
occurring in $[\theroot,x]$. Once again, by the previous
observation, all these $\lambda$-nodes except $\theroot$ have order
smaller than $\ord{x}$. Hence $\tau$ is incrementally-bound.

(ii) Let $M$ be a closed term such that $\tau(M)$ is
incrementally-bound. We assume that $M$ is already in $\eta$-normal
form. We prove that $M$ is safe by induction on its structure. The
base case $M = \lambda \overline{\xi} . \alpha$ for some variable or
constant $\alpha$ is trivial. \emph{Step case:} If $M = \lambda
\overline{\xi} . N_1 \ldots N_p$. Let $i$ range over $1..p$. $N_i$
can be written $\lambda \overline{\eta_i} . N'_i$ where $N'_i$ is
not an abstraction. By the induction hypothesis, $\lambda
\overline{\xi} . N_i = \lambda \overline{\xi} \overline{\eta_i} .
N'_i$ is safe. Hence $\vdash \lambda \overline{\xi}
\overline{\eta_i} . N'_i$ is a valid judgment of safe
$\lambda$-calculus. But this judgment can only be derived using the
\rulenamet{Abs} rule on the term $N'_i$. Hence $N'_i$ is necessarily
safe. Let $z$ be a variable occurring free in $N'_i$. Since $M$ is
closed, $z$ is either bound by $\lambda \overline{\eta_1}$ or
$\lambda \overline{\xi}$. In the latter case, since $\tau(M)$ is
i.b., $\ord{z}$ is smaller than $\ord{\lambda
\overline{\eta_1}}=\ord{N_i}$ thus in both case we are allowed to
abstract the variables $\overline{\eta_1}$ using the rule
\rulenamet{Abs}. This shows that $N_i$ is safe.

Each of the $N_i$s is safe and $N_1 \ldots N_p$ is of type $o$
therefore by the rule \rulenamet{App} rule we have $\overline{\xi}
\vdash N_1 \ldots N_p$. Finally, \rulenamet{Abs} gives us the
judgement $\vdash M = \lambda \overline{\xi} . N_1 \ldots N_p$.
\end{proof}

Note that the hypothesis that $M$ is closed in (ii) is necessary.
For instance, the two terms $\lambda x y .x$ and $\lambda y . x$,
where $x,y:o$, have (isomorphic) incrementally-bound computation
trees. However $\lambda x y .x$ is safe whereas $\lambda y . x$ is
not.

\begin{corollary}
\label{cor:betared_preserve_incrbound} Suppose $M$ is a closed term
in $\eta$-long normal form. If $\tau(M)$ is incrementally-bound and
$M \betared N$ then $\tau(N)$ is incrementally-bound.
\end{corollary}
\proof Suppose that $\tau(M)$ is i.b. Then by Lemma
\ref{lem:incrbound_iff_etanf_safe}(ii), $M$ is safe and since safety
is preserved by $\beta$-reduction, so is $N$. Thus by Lemma
\ref{lem:incrbound_iff_etanf_safe}(i), $\tau(N)$ is
incrementally-bound. \qed
\smallskip

Note that this corollary  cannot be generalized to
$A$-incremental-binding for any set of node $A$. Take for instance
the eta-normal term $M = \lambda u^{o} v^{((o,o),o)} . (\lambda x^o
. v (\lambda z^o . x)) u$ which beta-reduces to $N = \lambda u v . v
(\lambda z . u)$. The computation trees are:
$$\pssetcomptree
\tau(M) = \pstree{\TR{\underline{\lambda u v}}}{\pstree{\TR{@}}{\pstree{\TR{\lambda
x}}{\pstree{\TR{\underline{v}}}{\pstree{\TR{\underline{\lambda
z}}}{\TR{x}}}}\pstree{\TR{\lambda }}{\TR{\underline{u}}}}} \hspace{2cm}
\tau(N) = \pstree{\TR{\underline{\lambda u v}}}{\pstree{\TR{\underline{v}}}{\pstree{\TR{\underline{\lambda z}}}{\TR{\underline{u}}}}}
$$
Take $A$ to be the set of nodes that are hereditarily justified by
the root (the nodes underlined in the above figure). Then $\tau(M)$
is $A$-incrementally-bound but $\tau(N)$ is not.


\subsection{P-incremental-justified strategies}
\begin{definition}[P-incremental-justification]
A strategy $\sigma$ on a game $A$ is
\emph{P-incrementally\-justified} if and only if for any sequence of
moves $s q \in P_A$ we have:
\begin{eqnarray*}
s q \in \sigma \wedge q \mbox{ is a P-question } &\implies&
\parbox[t]{9cm}{$q$  points to the last O-move in $\pview{s}$
with order strictly greater than $\ord{q}$.}
\end{eqnarray*}
\end{definition}




\begin{lemma}
\label{lem:incrjustified_pointers_uniqu_recover} Pointers emanating
from P-moves are superfluous for P-incrementally-justified
strategies.
\end{lemma}
\begin{proof}
Suppose $\sigma$ is a P-incrementally-justified strategy. We prove
that pointers attached to P-moves in a play $s\in \sigma$ are
uniquely recoverable by induction on the length of $s$. \noindent
\emph{Base case}: if $|s| \leq 1$ then there is no pointer to
recover. \noindent \emph{Step case}: suppose $s m \in \sigma$. If
$m$ is an answer move then by the well-bracketing condition $m$
points to the last unanswered question in $s$. If $m$ is a
P-question then by  P-incremental-justification of $\sigma$, $m$
points to the last O-move in $\pview{s}$ with order strictly greater
than $\ord{q}$. Since we have access to O-moves' pointers, we can
compute the P-view $\pview{s}$. Hence $m$'s pointer is uniquely
recoverable.
\end{proof}

%\begin{example}
%The denotation of the evaluation map $ev$ is
%P-incrementally-justified since it is the uncurrying of the identity
% map on the game A=>B.
%\end{example}



\begin{proposition}[Incremental-binding and P-incremental-justification]
\hfill

 \label{prop:Nher_incrbound_iff_incrjustified}

\begin{enumerate}[(i)]
\item Suppose $M$ is $\beta$-normal. Then if all the \emph{reachable} input-variable nodes of the computation tree
$\tau(\Gamma \vdash M : T)$ are
$N^{\theroot\vdash}$-incrementally-bound then $\sem{\Gamma
\vdash M : T}$ is P-incrementally-justified.

\item If $\sem{\Gamma \vdash M : T}$ is
P-incrementally-justified then all the \emph{reachable}
input-variable nodes of the computation tree $\tau(\Gamma \vdash
M : T)$ are $N^{\theroot\vdash}$-incrementally-bound.
\end{enumerate}
\end{proposition}

\begin{proof}
\noindent (i) Suppose that $\tau(M)$ is
$N^{\theroot\vdash}$-incrementally-bound, then so is
$\tau(\etalnf{\closure{M}})$. Thus by Corollary
\ref{cor:betared_preserve_incrbound} $\etalnf{\closure{M}}$ is safe
and since safety is preserved by $\beta$-reduction, so is its
beta-normal form. Thus by Lemma
\ref{lem:incrbound_iff_etanf_safe}(i),
$\tau(\betanf{\etalnf{\closure{M}}})$ is incrementally-bound. Hence
we can assume without loss of generality that $M$ is a closed term
in beta-normal form and prove that $\sem{M}$ is
P-incrementally-justified (This will imply that
$\sem{\betanf{\etalnf{\closure{M}}}}$ is P-i.j.\ since the two game
denotations are isomorphic).

Take a play $s \in \sem{\Gamma \vdash M : T}$ ending with a question
P-move $q$. By the Correspondence Theorem \ref{thm:correspondence},
there is a traversal $t$ of $\tau(M)$ starting with an occurrence
$r$ of the root $\theroot$ such that $\psi_M (t\filter r) = s$. We
assume $t$ to be the shortest such traversal, thus the last
occurrence of $t$ - let us name it $n$ - is hereditarily justified
by $r$ and is by definition an occurrence of a reachable node.
Moreover since $\psi_M$ maps $n$ to $q$, $n$ is necessarily an
occurrence of a variable node $x$. There are two cases:
\begin{itemize}
\item Suppose $x$ is bound variable. Let $m$ denote its justifier
in $t$ (which is an occurrence of $x$'s binder in $\tau(M)$). By
assumption $\tau(M)$ is $N^{\theroot\vdash}$-incrementally-bound
therefore since $n$ belongs to $N^{\theroot\vdash}$, $m$ must be
the last $\lambda$-node in $[\theroot,n]\ \inter
N^{\theroot\vdash}$ of order strictly greater than $\ord{n}$.

By the Path--P-view correspondence (Prop.\
\ref{prop:pviewtrav_is_path}) we have $[\theroot,n]\ \inter
N^{\theroot\vdash} = \pview{t} \filter r$. This is in turn is
equal to $\pview{?(t \filter r)}$ (by Lemma
\ref{lem:betanf_wellbehavedconst_trav_pview_red}, since $M$ is
in $\beta$-normal form).


By property \ref{proper:psi_properties} (iv), the P-view of
$?(s)$ and the P-view of $?(t \filter r)$ are computed similarly
and have the same pointers, therefore node $n$ and move $q$ both
point to the same position in the justified sequence
$\pview{?(t\filter r)}$ and $\pview{?(s)}$ respectively.
Moreover since $\psi_M$ maps nodes of a given order to moves of
the same order (property \ref{proper:psi_properties}) this means
that $q$ points to the last O-move in $\pview{?(s)}$ with order
$>\ord{q}$.

Finally Lemma \ref{lem:views_and_questionmarkfilter} gives us
$?(\pview{s}) = \pview{?(s)}$, and since $s$'s last move is a
question, $\pview{s}$ contains only question moves and therefore
$\pview{?(s)} = \pview{s}$. Thus $q$ points to the last O-move
in $\pview{s}$ with order is strictly greater than $\ord{q}$.


\item  Second case: $n$ is a free input-variable $x$.
Thus $n$ is justified by $r$, the first occurrence in $t$. By
definition of $\psi$, $x = \psi(n)$ must be a move enabled by
the initial move $q_0 = \psi(\theroot)$ in the arena
$\sem{\Gamma \rightarrow A}$, therefore we have $\ord{q_0} >
\ord{x}$. Furthermore since  $x$ is
$N^{\theroot\vdash}$-incrementally-bound all the $\lambda$-nodes
in $]\theroot,n]$ have order smaller than $\ord{n}$, thus by the
Correspondence Theorem, all the O-moves in $\pview{s}$ have
order smaller than $\ord{x}$.
\end{itemize}

\noindent (ii) Suppose $\sem{M}$ is P-incrementally-justified. Let
$x$ be a reachable input-variable node of $\tau(M)$: there exists a
traversal of the form $t \cdot x$ in $\travset(M)$ such that $x$ is
hereditarily justified by the first occurrence $r$ of $\tau(M)$'s
root in $t$.

The correspondence theorem tells us that $\varphi((t \cdot x)
\filter r) = \varphi((t \filter r) \cdot x)$ belongs to $\sem{M}$.
Since $\sem{M}$ is P-incrementally-justified, $\varphi(x)$ points to
the last O-move in $\pview{\varphi(t \filter r)}$ with order
strictly greater than $\ord{\varphi(x)}$. Consequently $x$ points to
the last $\lambda$-node in $\pview{t \filter r}$ with order strictly
greater than $\ord{x}$.

But by Lemma \ref{lem:pviewproj_wrt_theroot}, $\pview{t \filter r}$
contains $\pview{t} \filter r$ as a subsequence. Thus since by
P-visibility $m$ occurs in this subsequence, we have that $m$ is
also the last $\lambda$-node in $\pview{t} \filter r$ with order
strictly greater than $\ord{x}$. By the path-P-view correspondence
(Prop.\ \ref{prop:pviewtrav_is_path}) this can in turn be restated
as: $m$ is the last $\lambda$-node in $[\theroot,x[\  \inter\
N^{\theroot \vdash}$ with order strictly greater than $\ord{x}$.
Hence $\tau(M)$ is $N^{\vdash \theroot}$-incrementally-bound.
\end{proof}

\section{Safe $\lambda$-Calculus}

We now consider the special case of the Safe $\lambda$-Calculus
without interpreted constants. We show that pointers in the game
denotation of safe terms can be uniquely recovered. The example of
section \ref{subsec:pointer_necessary} gives a good intuition: in
order to distinguish the terms $M_1 = \lambda f . f (\lambda x . f
(\lambda y .y ))$ and $M_2 = \lambda f . f (\lambda x . f (\lambda y
.x ))$ it is necessary to keep pointers in the strategy plays. In
the Safe $\lambda$-Calculus, however, the ambiguity disappears since
$M_1$ is safe whereas $M_2$ is not (in the subterm $f (\lambda y .
x)$, the free variable $x$ has the same order as $y$ but it is not
abstracted together with $y$).



\begin{corollary}[of Proposition \ref{prop:Nher_incrbound_and_incrjustified}]
\label{cor:Nher_incrbound_iff_incrjustified}
  Suppose $\Gamma \vdash M : T$ is a pure (\ie with no interpreted constants) simply-typed term
  in $\beta$-normal form. Then $\sem{M}$ is P-incrementally-justified if and only if $\tau(M)$ is incrementally-bound.
\end{corollary}
\proof We first observe that all the variable nodes are
input-variable nodes. Indeed, let $x$ be a variable node of
$\tau(M)$. Since $M$ is $\beta$-normal, by lemma
\ref{lem:betanorm_enabling}, $x$ is either hereditarily enabled by
the root or by a constant in $N_\Sigma$. But the pure simply-typed
$\lambda$-calculus does not have constants thus $N_\Sigma =
\emptyset$ and $x$ is hereditarily enabled by the root, \ie it is an
input-variable node. Consequently, incremental-binding coincides
with $N^{\vdash \theroot}$-incremental-binding.

Furthermore, since all the input-variables are reachable, every node
of the computation tree can be reached by the traversal consisting
of the path from the root to that node, the \rulenamet{InputVar}
permitting us to visit the children of the input-variable nodes
occurring in the path.\qed
\smallskip

\parpic[r]{
    \pssetcomptree
     \tree[levelsep=4ex]{$\lambda x^3$}{\tree{$f^2$}{ \tree{$\lambda y^1$}{ \TR{$x^0$} }}}
} \noindent \emph{Examples:} Consider the $\beta$-normal term
$\lambda x . f (\lambda y .x)$ where $x,y:o$ and $f:(o,o),o$. The
figure on the right represents the computation tree with the order
of each node in the exponent part. Since node $x$ of order $0$ is
not bound by the order 1 node $\lambda y$, $\tau(M)$ is not
incrementally-bound and by proposition
\ref{prop:Nher_incrbound_and_incrjustified} $\sem{\lambda x . f
(\lambda y .x)}$ is not P-incrementally-justified. Similarly we can
check that $\sem{f (\lambda y .x)}$ is not P-incrementally-justified
whereas $\sem{\lambda y. x}$ is. Also, for any higher-order variable
$x:A$ the computation tree $\tau(x)$ is incrementally-bound
therefore the projection strategies $\pi_i$ are
P-incrementally-justified. From these examples we observe that
application does not preserve P-incremental-justification ($\sem{f}$
and $\sem{\lambda y. x}$ are P-incrementally-justified whereas
$\sem{f (\lambda y .x)}$ is not).
\smallskip

These examples suggest that P-incremental-justification is not a
compositional property. In Chapter \ref{chap:pincrjust} we will
identify a sufficient condition guaranteeing that the composition of
two P-incrementally-justified strategies gives a
P-incrementally-justified strategy. \smallskip


Putting Corollary \ref{cor:Nher_incrbound_iff_incrjustified} and
Lemma \ref{lem:incrbound_iff_etanf_safe} together gives us a
game-semantic characterization of safe terms:
\begin{corollary}[P-incrementally-justified strategies characterize safe closed $\eta\beta$-normal terms]
Let $\Gamma \vdash M : T$ be a simply-typed term (without
interpreted constants). Then:
$$ \sem{\Gamma \vdash M : T} \mbox{ is P-incrementally-justified if and only if $\etabetalnf{M}$ is safe,} $$
where $\etabetalnf{M}$ denotes the $\eta$-long normal form of the
$\beta$-normal form of $M$.
\end{corollary}



\begin{theorem}[P's pointers are superfluous for safe terms]
\label{thm:safe_ptr_recoverable} Pointers emanating from P-moves in the game semantics of
safe terms are uniquely recoverable.
\end{theorem}
\begin{proof}
Let $M$ be a safe simply-typed term. Then the $\beta$-normal form of
$M$ is also safe, thus by lemma \ref{lem:incrbound_iff_etanf_safe}
(i), $\tau(\betanf{M})$ is incrementally-bound and by proposition
\ref{prop:Nher_incrbound_and_incrjustified}, $\sem{\Gamma \vdash
\betanf{M} :T}$ is a P-incrementally-justified strategy. By lemma
\ref{lem:incrjustified_pointers_uniqu_recover}, P's pointers in
$\sem{\Gamma \vdash \betanf{M} :T}$ are uniquely recoverable.
Finally, the soundness of the game model gives $\sem{\Gamma \vdash
M:T} = \sem{\Gamma \vdash \betanf{M} : T}$.
\end{proof}


\section{Safe PCF and Safe Idealized Algol}

Safe Idealized Algol, or safe \ialgol\ for short, is Idealized Algol
where the application and abstraction rules are restricted the same
way as in the safe $\lambda$-calculus (see rules of section
\ref{sec:safe_nonhomog}).

The properties of the safe $\lambda$-calculus can be transposed
straightforwardly to safe \ialgol. In particular, it can be shown
that safety is preserved by $\beta$-reduction and that no variable
capture occurs when performing substitution on a safe term.

A natural question to ask is whether we can extend the result about
game semantics of safe $\lambda$-terms to safe \ialgol-terms. In
this section we lay out the key elements permitting to prove that
the pointers in the game semantics of safe IA terms can be recovered
uniquely.

Such result has potential applications in algorithmic game semantics.
For instance, by following the framework of \cite{ghicamccusker00},
it may be possible to give a characterisation of the game semantics
of some higher-order fragments of safe \ialgol\ using extended
regular expressions. Subsequently, this would lead to the
decidability of program equivalence for the considered fragment.


\subsection{Formation rules of Safe \ialgol}
We call safe \ialgol\ term any term that is typable within the
following system of formation rules:
$$ \rulename{var} \   \rulef{}{x : A\vdash x : A}
%\qquad  \rulename{const} \   \rulef{}{\vdash f : A} \quad f \in \Sigma
\qquad  \rulename{wk} \   \rulef{\Gamma \vdash M : A}{\Delta \vdash
M : A} \quad  \Gamma \subset \Delta$$

$$ \rulename{app} \  \rulef{\Gamma \vdash M : (A,\ldots,A_l,B)
                                        \qquad \Gamma \vdash N_1 : A_1
                                        \quad \ldots \quad \Gamma \vdash N_l : A_l  }
                                   {\Gamma  \vdash M N_1 \ldots N_l : B}
                                    \quad
\mbox{\fbox{$\forall y \in \Gamma : \ord{y} \geq \ord{B}$}}$$

$$ \rulename{abs} \   \rulef{\Gamma \union \overline{x} : \overline{A} \vdash M : B}
                                   {\Gamma  \vdash \lambda \overline{x} : \overline{A} . M : (\overline{A},B)} \quad
\mbox{\fbox{$\forall y \in \Gamma : \ord{y} \geq \ord{\overline{A},B}$}}$$

$$ \rulename{num} \rulef{}{\Gamma \vdash n :\texttt{exp}}
\qquad \rulename{succ} \rulef{\Gamma \vdash M:\texttt{exp} }{\Gamma
\vdash \texttt{succ}\ M:\texttt{exp}} \qquad \rulename{pred}
\rulef{\Gamma \vdash M:\texttt{exp} }{\Gamma \vdash \texttt{pred}\
M:\texttt{exp}}$$

$$
\rulename{cond} \rulef{\Gamma \vdash M : \texttt{exp} \qquad \Gamma
\vdash N_1 : \texttt{exp} \qquad \Gamma \vdash N_2 : \texttt{exp}
}{\Gamma \vdash \texttt{cond}\ M\ N_1\ N_2} \qquad  \rulename{rec}
\rulef{\Gamma \vdash M : A\rightarrow A }{ \Gamma \vdash Y_A M :
A}$$

$$ \rulename{seq} \rulef{\Gamma \vdash M : \texttt{com} \quad \Gamma \vdash N :A}
    {\Gamma \vdash \texttt{seq}_A \ M\ N\ : A} \quad A \in \{ \texttt{com}, \texttt{exp}\}$$

$$ \rulename{assign} \rulef{\Gamma \vdash M : \texttt{var} \quad \Gamma \vdash N : \texttt{exp}}
    {\Gamma \vdash \texttt{assign}\ M\ N\ : \texttt{com}}
\qquad
 \rulename{deref} \rulef{\Gamma \vdash M : \texttt{var}}
    {\Gamma \vdash \texttt{deref}\ M\ : \texttt{exp}}$$

$$ \rulename{new} \rulef{\Gamma, x : \texttt{var} \vdash M : A}
    {\Gamma \vdash \texttt{new } x \texttt{ in } M} \quad A \in \{ \texttt{com}, \texttt{exp}\}$$

$$ \rulename{mkvar} \rulef{\Gamma \vdash M_1 : \texttt{exp} \rightarrow \texttt{com} \quad \Gamma \vdash M_2 : \texttt{exp}}
    {\Gamma \vdash \texttt{mkvar } M_1\ M_2\ : \texttt{var}}$$

\subsection{Small-step semantics of Safe \ialgol}
In the first chapter we defined the operational semantics of
\ialgol\ using a big step semantics. The operational semantics of
\ialgol\ can be defined equivalently using a small-step semantics.
The reduction rules of the small-step semantics are of the form $s,e
\rightarrow s',e'$ where $s$ and $s'$ denotes the stores and $e$ and
$e'$ denotes \ialgol\ expressions.

Let us give the rules that tell how to reduce redexes:
\begin{itemize}
\item the reduction of safe-redex (relation $\beta_s$ from definition \ref{dfn:safereduction});
\item reduction rules for \pcf\ constants:
\begin{eqnarray*}
\pcfsucc\ n &\rightarrow& n+1 \\
\pcfpred\ n+1 &\rightarrow& n \\
\pcfpred\ 0 &\rightarrow& 0 \\
\pcfcond\ 0\ N_1 N_2 &\rightarrow& N_1 \\
\pcfcond\ n+1\ N_1 N_2 &\rightarrow& N_2 \\
Y\ M &\rightarrow& M (Y M)
\end{eqnarray*}
\item reduction rules for \ialgol\ constants:
\begin{eqnarray*}
\iaseq\ \iaskip\  M &\rightarrow& M \\
s, \ianewin{x}\ M &\rightarrow& (s|x\mapsto 0), M \\
s, \iaassign\ x\ n &\rightarrow& (s|x\mapsto n), \iaskip \\
s, \iaderef\ x &\rightarrow& s, s(x) \\
\iaassign\ (\iamkvar M N)\ n &\rightarrow& M n \\
\iaderef\ (\iamkvar M N) &\rightarrow& N
\end{eqnarray*}
\end{itemize}

Redex can also be reduced when they occur as subexpressions within a
larger expression. We make use of evaluation contexts to indicate
when such reduction can happen. Evaluation contexts are given by the
following grammar:
\begin{eqnarray*}
E[-] &::=& - |\ E N\ |\ \pcfsucc\ E\ |\ \pcfpred\ E\ |\ \pcfcond\ E\ N_1\ N_2\ |\ \\
&&    \iaseq\ E\ N\ |\ \iaderef\ E\ |\ \iaassign\ E\ n\ |\ \iaassign\ M\ E \ |\ \\
&&    \iamkvar\ M\ E\ |\ \iamkvar\ E\ M\ |\ \ianewin{x}\ E  .
\end{eqnarray*}

The small-step semantics is completed with following rule:
$$ \rulef{M \rightarrow N}{E[M] \rightarrow E[N]} $$

\begin{lemma}[Reduction preserves safety]
\label{lem:ia_safety_preserved} Let $M$ be a safe IA term. If
$M \rightarrow N$ then $N$ is also a safe term.
\end{lemma}
This can be proved easily by induction on the structure of M.


\subsection{Safe \pcf\ fragment}
In this section, we show how to extend the results obtained for the
safe $\lambda$-calculus to the \pcf\ fragment of safe \ialgol.

The $Y$ combinator needs a special treatment. In order to deal with
it, we follow the idea of \cite{abramsky:game-semantics-tutorial}:
we consider the sublanguage $\pcf_1$ of \pcf\ in which the only
allowed use of the $Y$ combinator is in terms of the form $Y(
\lambda x:A .x )$ for some type $A$. We will write $\Omega_A$ to
denote the non-terminating term $Y(\lambda x:A .x)$ for a given type
$A$.

We introduce the \emph{syntactic approximants} to $Y_A M$:
\begin{eqnarray*}
Y^0_A M &=& \Gamma \vdash \Omega_A : A\\
Y^{n+1}_A M &=& M( Y^n M )
\end{eqnarray*}
For any \pcf\ term $M$ and natural number $n$, we define $M_n$ to be
the $\pcf_1$ term obtained from $M$ by replacing each subterm of the
form $Y N$ with $Y^n N_n$. We have $\sem{M} = \Union_{n\in\omega}
\sem{M_n}$ (\cite{abramsky:game-semantics-tutorial}, lemma 16).


\subsubsection{Computation tree}

We would like to define a unique computation tree for terms that use
the $Y$ combinator.

Let us first define the computation tree for $\pcf_1$ terms. We
introduce a special $\Sigma$-constant $\bot$ representing the
non-terminating computation of ground type $\Omega_o$. Given any
type $A = (A_1, \ldots, A_n, o)$, the computation tree
$\tau(\Omega_A)$ is defined to be the tree representation of
$\lambda x_1:A_1 \ldots x_n:A_n . \bot$. The computation tree of a
$\pcf_1$ term is then computed inductively in the standard way.

We now introduce a partial order on the set of computation trees.

A \emph{tree} $t$ is a labelling function $t:T\rightarrow L$ where
$T$, called the domain of $t$ and written $dom(t)$, is a non-empty
prefix-closed subset of some free monoid $X^*$ and $L$ denotes the
set of possible labels. Intuitively, $T$ represents the structure of
the tree (the set of all paths) and $t$ is the labelling function
mapping paths to labels. Trees can be ordered using the
\emph{approximation ordering} defined in \cite{KNU02}, section 1: we
write $t' \sqsubseteq t$ if the tree $t'$ is obtained from $t$ by
replacing some of its subtrees by $\bot$. Formally:
$$t' \sqsubseteq t \quad \iff dom(t') \subseteq dom(t) \wedge \forall  w \in dom(t'). (t'(w) = t(w) \vee t'(w) = \bot).$$
The set of all trees together with the approximation ordering is a
complete partial order.

We now consider a strict subset of the set of all trees: the set of
computation trees. A computation tree is a tree which represents the
$\eta$-normal form of some (potentially infinite) \pcf\ term. In
other words a tree is a computation tree if it can be written
$\tau(M)$ for some infinite \pcf\ term $M$. The set $L$ of labels is
constituted of the $\Sigma$-constants, @, the special constant
$\bot$, variables and abstractions of any sequence of variables. We
will write $(CT, \sqsubseteq)$ to denote the set of computation
trees ordered by the approximation ordering $\sqsubseteq$ defined
above. $(CT, \sqsubseteq)$ is also a complete partial order.

It is easy to check that the sequence of computation trees
$(\tau(M_n))_{n\in\omega}$ is a chain. We can therefore define the
computation tree of a \pcf\ term $M$ to be the least upper-bound of
the chain of computation trees of its approximants:
$$\tau(M) = \Union_{n\in\omega}(\tau(M_n))_{n\in\omega}.$$

In other words, we construct the computation tree by expanding
infinitely any subterm of the form $Y M$. For instance consider the
term $M = Y (\lambda f x. f x)$ where $f:(o,o)$ and $x:o$. Its
computation tree $\tau(M)$, represented below, is a tree
representation of the $\eta$-normal form of the infinite term
$(\lambda f x. f x) ((\lambda f x. f x) ((\lambda f x. f x)  (
\ldots$.
$$\tau(M) = \pssetcomptree\tree{\lambda y}{
                \tree{@}{
                        \tree{\lambda f x} { \tree{f}{\tree{\lambda}{\TR{x}} }}
                        \TR{\tau(M)}
                        \tree{\lambda}{\TR{y}}
                }
            }
$$

The remaining operators of \ialgol\ are treated as standard
constants and the corresponding computation tree is constructed from
the $\eta$-normal form of the term in the standard way. For instance
the diagram below shows the computation tree for $\pcfcond\ b\ x\ y$
(left) and $\lambda x . 5$ (right):
$$
\pssetcomptree\tree{\lambda b x y}
     {  \tree{\pcfcond}
        {   \tree{\lambda} {\TR{b}}
            \tree{\lambda} {\TR{x}}
            \tree{\lambda} {\TR{y}}
        }
    }
\hspace{2cm} \tree{\lambda x}{  \TR{5} }
$$
The node labelled $5$ has, like any other node, children
value-leaves which are not represented on the diagram above for
simplicity.

\subsubsection{Traversal}

New traversal rules accompany the additional constants of \ialgol.
There is one additional rule for natural number constants:
\begin{itemize}
\item (Nat) If $t \cdot n$ is a traversal where $n$ denotes a node labelled with some numeral constant $i\in \nat$ then
            $\Pstr{t \cdot (n){n} \cdot (in-n){i_n}}$
            is also a traversal where $i_n$ denotes the value-leaf of $m$ corresponding to the value $i\in \nat$.
\end{itemize}

\noindent The traversals rules for \pcfpred\ and \pcfsucc\ are
defined similarly. For instance, the rules for \pcfsucc\ are:
\begin{itemize}
\item (Succ) If $t \cdot \pcfsucc$ is a traversal and $\lambda$ denotes the only child node of \pcfsucc\ then
$\Pstr{t \cdot (succ){\pcfsucc} \cdot (l-succ,35:1){\lambda}}$ is also a traversal.

\item (Succ') If
$\Pstr{ t_1 \cdot (succ){\pcfsucc} \cdot (l-succ,35:1){\lambda} \cdot t_2
\cdot (lv-l){i_{\lambda}}} $ is a traversal for some
$i \in \nat$ then $\Pstr{t_1 \cdot (succ){\pcfsucc} \cdot
(l-succ,35:1){\lambda} \cdot t_2 \cdot (lv-l){i_{\lambda}} \cdot
(succv-succ,25){(i+1)_{\pcfsucc}}}$ is also a traversal.
\end{itemize}

\noindent In the computation tree, nodes labelled with \pcfcond\
have three children nodes numbered from $1$ to $3$ corresponding to
the three parameters of the operator \pcfcond. The traversal rules
are:
\begin{itemize}
\item (Cond-If) If $t_1 \cdot \pcfcond$ is a traversal and $\lambda$ denotes the first child of \pcfcond\ then
$\Pstr{ t_1 \cdot (cond){{\pcfcond}} \cdot (l-cond,30:1){\lambda}}$
 is also a traversal.

\item (Cond-ThenElse) If
$\Pstr{t_1 \cdot (cond){\pcfcond} \cdot (l-cond,35:1){\lambda} \cdot t_2
\cdot (lv-l){i_{\lambda}}} $
then $\Pstr{t_1 \cdot
(cond){\pcfcond} \cdot (l-cond,35:1){\lambda} \cdot t_2 \cdot
(lv-l){i_{\lambda}} \cdot (condthenelse-cond,35:{2+[i>0]}){\lambda} }
$
is also a traversal.



\item (Cond') If
$\Pstr{t_1 \cdot (cond){\pcfcond} \cdot t_2 \cdot (l-cond,35:k){\lambda}
\cdot t_3 \cdot (lv-l){i_{\lambda}}}$
 for $k=2$ or $k=3$ then  $\Pstr{ t_1 \cdot
(cond){\pcfcond} \cdot t_2 \cdot (l-cond,35:k){\lambda} \cdot t_3
\cdot (lv-l){i_{\lambda}} \cdot (condv-cond,25){i_{\pcfcond}}}$
 is also a traversal.
\end{itemize}
It is easy to verify that these traversal rules are all
well-behaved. This completes the definition of traversal for the
\pcf\ subset of \ialgol.

\subsubsection{Interaction semantics}
We recall that the interaction semantics defined in section
\ref{sec:interaction_semantics} takes into account the constants
of the language. For any higher-order constant $f : (A_1,\ldots,A_p,B) \in \Sigma$, definition \ref{dfn:canonical_revealed_semantics} gives the  revealed strategy of a term of the form $\lambda \overline{\xi}. f N_1 \ldots
N_p$ as follows:
$$ \revsem{\lambda \overline{\xi}. f N_1 \ldots N_p} = \langle \revsem{N_1}, \ldots, \revsem{N_p} \rangle \fatsemi^{0..p-1} \sem{f}.$$
where $\sem{f}$ is the standard strategy denotation of the constant $f$.


\subsubsection{Removing $\Sigma$-nodes from the traversals}


\notetoself{Need to rework the following lemma}

\begin{lemma}[Projection lemma]
\label{lem:SIGMACONST:varphi_projection} Let $\Gamma \vdash M :T$ be
a term and $r$ be the root of $\tau(M)$. For any traversal $t$ of
the computation tree we have $ \varphi(\travset(M)^*) \filter
\sem{\Gamma \rightarrow T} = \varphi(\travset(M)^{\filter r}) $.
 Consequently,
$$\varphi(t^*) \filter \sem{\Gamma \rightarrow T} = \varphi(t\filter r).$$
\end{lemma}
\begin{proof}
    From the definition of $\varphi$, the nodes of the computation tree that $\varphi$ maps
    to moves in the arena $\sem{\Gamma \rightarrow T}$ are exactly the nodes that are hereditarily justified by $r$.
    The result follows from the fact that @-nodes, constant nodes and value-leaves of constant nodes
    are not hereditarily justified by the root.
\end{proof}


The following lemma is the counterpart of lemma
\ref{lem:varphiinjective} and it is proved identically.
\begin{lemma}[$\varphi$ is injective]
\label{lem:SIGMACONST:varphiinjective} $\varphi$ regarded as a
function defined on the set of sequences of nodes is injective in
the sense that for any two traversals $t_1$ and $t_2$:
\begin{itemize}
\item[(i)] if $\varphi (t_1^* ) = \varphi (t_2^* )$ then $t_1^* =t_2^*$;
\item[(ii)] if $\varphi (t_1 \filter r ) = \varphi (t_2 \filter r )$ then $t_1\filter r = t_2\filter r$.
\end{itemize}
\end{lemma}

\begin{corollary} \
\label{cor:SIGMACONST:varphi_bij}
\begin{itemize}
\item[(i)] $\varphi$ defines a bijection from $\travset(M)^*$
to $\varphi(\travset(M)^*)$;
\item[(ii)] $\varphi$ defines a bijection from $\travset(M)^{\filter r}$ to
$\varphi(\travset(M)^{\filter r})$.
\end{itemize}
\end{corollary}


\subsubsection{Correspondence theorem}
We would like to prove the counterpart of proposition
\ref{prop:rel_gamesem_trav} in the context of the simply-typed
$\lambda$-calculus \emph{with interpreted PCF constants}. The game
model of the language \pcf\ is given by the category $\mathcal{C}_b$
of well-bracketed strategies. Hence the well-bracketing assumption
stated at the beginning of section \ref{sec:gamesemcorresp} is
satisfied.

We first prove that $\travset(\_)^{\filter r}$ is continuous.
\begin{lemma}
\label{lem:travred_continuous} Let $(S,\subseteq)$ denote the set of
sets of justified sequences of nodes ordered by subset inclusion.
The function $\travset(\_)^{\filter r} : (CT,\sqsubseteq)
\rightarrow (S,\subseteq)$ is continuous.
\end{lemma}
\begin{proof} \
    \begin{description}
    \item[Monotonicity:] Let $T$ and $T'$ be two computation trees such that $T \sqsubseteq T'$
    and let $t$ be some traversal of $T$.
    Traversals ending with a node labelled $\bot$ are maximal therefore $\bot$ can only occur
    at the last position in a traversal. Let us prove the following two properties:
        \begin{itemize}
            \item[(i)]  If $t = t \cdot n$ with $n\neq \bot$ then $t$ is a traversal of $T'$;
            \item[(ii)] if $t= t_1 \cdot \bot$ then $t_1\in \travset(T')$.
        \end{itemize}

        (i) By induction on the length of $t$. It is trivial for the empty traversal.
            Suppose that $t = t_1 \cdot n$ is a traversal with $n \neq \bot$.
            By the induction hypothesis, $t_1$ is a traversal of $T'$.

            We observe that for all traversal rules, the traversal produced is of the form $t_1 \cdot n$ where
            $n$ is defined to be a child node or value-leaf of some node $m$ occurring in $t_1$.
            Moreover, the choice of the node $n$ only depends on the traversal $t_1$
            (for the constant rules, this is guaranteed by assumption (WB)).

            Since $T \sqsubseteq T'$, any node $m$ occurring in $t_1$ belongs
            to $T'$ and the children nodes and leaves of $m$ in $T$ also belong to the tree $T'$.
            Hence $n$ is also present in $T'$ and the rule used to produce the traversal $t$ of $T$
            can be used to produce the traversal $t$ of $T'$.

        (ii) $\bot$ can only occur at the last position in a traversal
        therefore $t_1$ does not end with $\bot$ and by (i) we have $t_1\in \travset(T')$.
\vspace{6pt}

        Hence we have:
        \begin{align*}
        \travset(T)^{\filter r} &= \{ t \filter r \ | \ t \in \travset(T)     \} \\
        & = \{ (t\cdot n) \filter r \ | \ t\cdot n \in \travset(T) \wedge n \neq \bot \}
            \union \{ (t \cdot \bot ) \filter r \ | \ t \cdot \bot \in \travset(T)  \} \\
\mbox{(by (i) and (ii))} \quad        & \subseteq  \{ (t\cdot n)
\filter r \ | \ t\cdot n \in \travset(T') \wedge n \neq \bot
\}
            \union \{ t \filter r \ | \ t \in \travset(T')  \} \\
        & = \travset(T')^{\filter r}
        \end{align*}

        \item[Continuity:] Let $t \in \travset \left( \Union_{n\in\omega} T_n \right)$.
        We write $t_i$ for the finite prefix of $t$ of length $i$.
        The set of traversals is prefix-closed therefore $t_i \in \travset \left( \Union_{n\in\omega} T_n \right)$ for any $i$.
        Since $t_i$ has finite length we have $t_i \in \travset(T_{j_i})$ for some $j_i \in \omega$.
        Therefore we have:
        \begin{align*}
          t \filter r &= (\bigvee_{i\in\omega} t_i ) \filter r   & (\mbox{the sequence $(t_i)_{i\in\omega}$ converges to $t$}) \\
          &= \Union_{i\in\omega} ( t_i \filter r )   & (\_ \filter r \mbox{ is continuous, lemma \ref{lem:projection_continuous}}) \\
          &\in \Union_{i\in\omega} \travset(T_{j_i})^{\filter r}   & (t_i \in \travset(T_{j_i})) \\
          &\subseteq \Union_{i\in\omega} \travset(T_i)^{\filter r}   & (\mbox{since } \{ j_i \sthat i \in \omega \} \subseteq \omega)
        \end{align*}

        Hence $\travset(\Union_{n\in\omega} T_n )^{\filter r} \subseteq \Union_{n\in\omega} \travset(T_n)^{\filter r}.$

    \end{description}
\end{proof}

\begin{proposition}
Let $\Gamma \vdash M : T$ be a PCF term and $r$ be the root of
$\tau(M)$. Then:
\begin{align*}
(i)  \quad\varphi_M(\travset(M)^*) = \revsem{M},  \\
(ii) \quad \varphi_M(\travset(M)^{\filter r}) = \sem{M}.
\end{align*}
\end{proposition}
\begin{proof}
We first prove the result for $\pcf_1$: (i) The proof is an
induction identical to the proof of proposition
\ref{prop:rel_gamesem_trav}. However we need to complete the case
analysis with the $\Sigma$-constant cases:
\begin{itemize}
\item The cases \pcfsucc, \pcfpred, \pcfcond\ and numeral constants are straightforward.

\item Suppose $M = \Omega_o$ then $\travset(\Omega_o) = \prefset ( \{ \lambda \cdot \bot \} )$ therefore
$\travset(\Omega_o)^{\filter r} = \prefset( \{ \lambda \} )$
and $\sem{\Omega_o} = \prefset( \{ q \})$ with $\varphi(\lambda) =
q$. Hence $\sem{\Omega_o} = \varphi
(\travset(\Omega_o)^{\filter r})$.
\end{itemize}
(ii) is a direct consequence of (i) and the Projection Lemma (Lemma
\ref{lem:SIGMACONST:varphi_projection}). \vspace{10pt}

\noindent We now extend the result to \pcf. Let $M$ be a \pcf\ term,
we have:
\begin{align*}
\sem{M} &= \Union_{n\in\omega} \sem{M_n} & (\mbox{\cite{abramsky:game-semantics-tutorial}, lemma 16})\\
&= \Union_{n\in\omega} \travset(\tau(M_n))^{\filter r} & (M_n \mbox{ is a $\pcf_1$ term}) \\
&= \travset(\Union_{n\in\omega} \tau(M_n) )^{\filter r} & (\mbox{by continuity of $\travset(\_)^{\filter r}$, lemma \ref{lem:travred_continuous}}) \\
&= \travset(\tau(M))^{\filter r} & (\mbox{by definition of } \tau(M)) \\
&= \travset(M)^{\filter r} & (\mbox{abbreviation}).
\end{align*}
\end{proof}

Hence by corollary \ref{cor:SIGMACONST:varphi_bij}, $\varphi$
defines a bijection from $\travset(M)^{\filter r}$ to
$\sem{M}$:
$$\varphi : \travset(M)^{\filter r} \stackrel{\cong}{\longrightarrow} \sem{M}.$$

\subsubsection{Example: \pcfsucc}

Consider the term $M = \pcfsucc\ 5$ whose computation tree is
represented below. The value-leaves are also represented on the
diagram, they are the vertices attached to their parent node with a
dashed line.
$$
\psmatrix[colsep=3ex,rowsep=2ex]
\lambda^0 \\
\pcfsucc & 0 & 1 & \ldots \\
\lambda^1 & 0 & 1 & \ldots \\
5 & 0 & 1 & \ldots \\
  & 0 & 1 & \ldots
\endpsmatrix
\ncline{1,1}{2,1} \ncline{2,1}{3,1} \ncline{3,1}{4,1}
\valueedge{1,1}{2,2} \valueedge{1,1}{2,3} \valueedge{1,1}{2,4}
\valueedge{2,1}{3,2} \valueedge{2,1}{3,3} \valueedge{2,1}{3,4}
\valueedge{3,1}{4,2} \valueedge{3,1}{4,3} \valueedge{3,1}{4,4}
\valueedge{4,1}{5,2} \valueedge{4,1}{5,3} \valueedge{4,1}{5,4}
$$

The following sequence of nodes is a traversal of $\tau(M)$:
$$ \Pstr[20pt]{ t = (l0){\lambda^0} \cdot (succ){\pcfsucc} \cdot (l1){\lambda^1} \cdot (c5){5} \cdot (v55-c5){5_5} \cdot (5l1-l1){5_{\lambda^1}} \cdot (6succ-succ){6_\pcfsucc} \cdot (6l0-l0,35){6_{\lambda^0}}}.
$$

The subsequences $t^*$ and $t \filter r$ are given by:
$$
\Pstr[17pt]{ t^* = (l0){\lambda^0} \cdot (l1-l0){\lambda^1} \cdot
(5l1-l1){5_{\lambda^1}} \cdot (6l0-l0){6_{\lambda^0}}.
\qquad  \mbox{ and } \qquad t
\filter r = (l0){\lambda^0} \cdot
(6l0-l0){6_{\lambda^0}}. }
$$
We have $\varphi(t^*) = q_0 \cdot q_5 \cdot 5_{q_5} \cdot 5_{q_0}$
and $\varphi(t\filter r) = q_0 \cdot 5_{q_0}$ where $q_0$
and $q_5$ denote the roots of two flat arenas over $\nat$. These two
sequences of moves correspond to some play of the interaction
semantics and the standard semantics respectively. The interaction
play is represented below:
$$\begin{array}{ccccc}
  \textbf{1} & \stackrel{5}{\multimap} & !\nat & \stackrel{\pcfsucc}{\multimap} & \nat \\
&&&&  \rnode{q0}{q_0} \\
&&  \rnode{q5}{q_5} \\
&&  \rnode{a5}{5_{q_5}} \\
&&&&  \rnode{a6}{6_{q_0}}
\end{array}
\nccurve[nodesep=2pt,ncurv=0.9,angleA=180,angleB=180]{->}{a5}{q5}
\nccurve[nodesep=2pt,ncurv=0.9,angleA=180,angleB=210]{->}{a6}{q0}
\ncarc[nodesep=2pt,ncurv=0.9,angleA=180,angleB=180]{->}{q5}{q0}
$$

\subsubsection{Another example : \pcfcond}

Consider the term $M = \lambda x y . \pcfcond\ 1\ x\ y$. Its
computation tree is represented below (without the value-leaves):
    $$ \pssetcomptree\tree{\lambda x y}
       {
          \tree{\pcfcond}
          {
            \tree{\lambda^1}{ \TR{1} }
            \tree{\lambda^2}{ \TR{x} }
            \tree{\lambda^3}{ \TR{y} }
          }
      }
    $$
For any value $v \in\mathcal{D}$ the following sequence of nodes is
a traversal of $\tau(M)$:
$$\Pstr[27pt]{ t = (lxy){\lambda x y} \cdot (cond){\pcfcond} \cdot (l1-cond){\lambda^1} \cdot (1){1} \cdot (v11-1){1_1}
    \cdot (l3){\lambda^3} \cdot (y-vxy){y} \cdot (vy-y){v_y}  \cdot (vl3-l3){v_{\lambda^3}} \cdot (vcond-cond,30){v_{\pcfcond}}
    \cdot (vlxy-lxy,30){v_{\lambda x y}}.
}
$$
The subsequences $t^*$ and $t \filter r$ are given by:
$$
\Pstr[17pt]{ t^* =  t = (lxy){\lambda x y} \cdot
        (l1-lxy){\lambda^1} \cdot
        (l3-lxy){\lambda^3} \cdot
        (y-vxy){y} \cdot
        (vy-y){v_y}  \cdot
        (vl3-l3){v_{\lambda^3}} \cdot
        (vlxy-lxy,35){v_{\lambda x y}}
\qquad  \mbox{ and } \qquad t \filter r =
(lxy){\lambda x y} \cdot (y-vxy){y} \cdot (vy-y){v_y}
\cdot (vlxy-lxy){v_{\lambda x y}}.
}
$$
The sequence of moves $\varphi(t^*)$ corresponds to some play of the
interaction semantics and the sequence $\varphi(t\filter r)$
is a play of the standard semantics obtained by hiding the internal
moves of $\varphi(t^*)$. The interaction play $\varphi(t^*)$ is
represented below:
$$\begin{array}{ccccccccccc}
!\nat & \otimes & !\nat & \stackrel{ \langle \sem{1}, \pi_1,
\pi_2\rangle }{\multimap} & !\nat & \otimes & !\nat & \otimes &
!\nat
& \stackrel{ \pcfcond}{\multimap} & \nat \\
&&&&&&&&&&  \rnode{q0}{q_0^{(\lambda x y)}} \\
&&&&  \rnode{qa}{q_a^{(\lambda^1)}} \\
&&&&  \rnode{1}{1} \\
&&&&&&  \rnode{qb}{q_b^{(\lambda^2)}} \\
&&  \rnode{qy}{q_y^{(y)}} \\
&&  \rnode{vqy}{v_{q_y}} \\
&&&&&&  \rnode{vqb}{v_{q_b}} \\
&&&&&&&&&& \rnode{vq0}{v_{q_0}}
\end{array}
\ncarc[nodesep=2pt,ncurv=0.9,angleA=180,angleB=180]{->}{vq0}{q0}
\ncarc[nodesep=2pt,ncurv=0.9,angleA=180,angleB=180]{->}{vqb}{qb}
\nccurve[nodesep=2pt,ncurv=0.9,angleA=180,angleB=180]{->}{vqy}{qy}
\ncarc[nodesep=2pt,ncurv=0.9,angleA=180,angleB=180]{->}{qy}{qb}
\ncarc[nodesep=2pt,ncurv=0.9,angleA=90,angleB=180]{->}{qb}{q0}
\nccurve[nodesepB=2pt,nodesepA=6pt,ncurv=0.9,angleA=180,angleB=180]{->}{1}{qa}
\ncarc[nodesep=2pt,ncurv=0.9,angleA=90,angleB=180]{->}{qa}{q0}
$$


\subsubsection{Game characterisation of safe terms}

A difficulty arises because of the presence of the Y combinator :
computation trees of \pcf\ terms are potentially infinite. Despite
this particularity, lemma \ref{lem:incrbound_iff_etanf_safe} still
holds in the \pcf\ setting:
\begin{lemma} \label{lem:pcf_safe_imp_incrbound} If $M$ is a safe
PCF term then $\tau(M)$ is incrementally-bound.
\end{lemma}
\begin{proof}
Let $i$ denote the number of occurrences of the Y combinator in $M$.
We first prove by induction on $i$ that $M_k$ is safe for any $k\in
\omega$. \emph{Base case:} $i=0$ then $M_k = M$. \emph{Step case:}
$i>0$. Let $Y_A N$ be a subterm of $M$. Since $M$ is safe, $N$ is
also safe. The number of occurrences of the Y combinator in $N$ is
smaller than $i$ therefore by the induction hypothesis $N_k$ is
safe. Consequently the term $Y_A^k N_k = \underbrace{N_k ( \ldots (
N_k}_{k \mbox{ times}} \Omega ) \ldots )$ is also safe and by
compositionality so is $M_k$.

Clearly, lemma \ref{lem:incrbound_iff_etanf_safe}(i) is remains
valid for infinite $\pcf_1$ terms (the subterms of the form $\Omega$
are just represented by the constant $\bot$ in the computation
tree), thus since $M_k$ is a safe $\pcf_1$ term, $\tau(M_k)$ is
incrementally-bound. Now let $z$ be a variable node in $\tau(M) =
\Union_{k\in\omega} \tau(M_k)$. There exists $k\in \omega$ such that
$z$ belongs to $\tau(M_k) \sqsubseteq \tau(M)$. If we write $r_k$ to
denote the root of the tree $\tau(M_k)$ then the path $[r_k,z]$ in
$\tau(M_k)$ is equal to the path $[r,z]$ in $\tau(M)$. Hence, since
the node $z$ is incrementally-bound in $\tau(M_k)$, it is also
incrementally-bound in $\tau(M)$.
\end{proof}


\begin{theorem}
Safe PCF terms are denoted by P-incrementally-justified strategies.
\end{theorem}
\begin{proof}
Let $M^{\infty}$ be the $\beta$-normal form of $M$ (i.e. the possibly infinite term obtained by reducing all the redexes in $M$). By lemma \ref{lem:ia_safety_preserved}, safety is preserved by small-step reduction therefore, by lemma \ref{lem:pcf_safe_imp_incrbound}, if $M$ is a \pcf\ term then $\tau(M^{\infty})$ is also
incrementally-bound.

Since \pcf\ constant rules are well-behaved (by Lemma
\ref{lem:sigma_order1_are_wellbehaved}), the result from Lemma
\ref{lem:betanf_wellbehavedconst_trav_pview_red} is also true for
Safe \pcf. Thus proposition
\ref{prop:Nher_incrbound_and_incrjustified}(i) remains valid for the
infinite computation trees of \pcf: infinite terms in $\beta$-nf
with an incrementally-bound computation tree are denoted by
P-incrementally-justified strategies. Consequently,
$\sem{M^{\infty}}$ is P-incrementally-justified. By soundness of the
game denotation, $\sem{M^{\infty}} = \sem{M}$, thus $\sem{M}$ is
P-incrementally-justified.
\end{proof}

Consequently, P-pointers are superfluous in the game denotation of safe \pcf\ terms {\it i.e.} pointers emanating from P-moves are uniquely recoverable.

\subsection{Safe \ialgol}

We are now in a position to consider the full safe Idealized Algol
language. The general idea is the same as for safe \pcf, however
there are some difficulties caused by the presence of the two new
base types \iavar\ and \iacom. We just give indications on how to
adapt our framework to the particular case of safe \ialgol\ without
giving the complete proofs. However we believe that enough
indications are given to convince the reader that the argument used
in the \pcf\ case can be easily adapted to \ialgol.

\subsubsection{Computation DAG}
In \pcf, arenas have a single initial move, therefore they can be
regarded as trees. In \ialgol, on the other hand, the base type
\iavar\ is represented by the infinite product of games
$\iacom^{\nat} \times \iaexp$ which has an infinite number of
initial moves. In order to preserve the relationship established
between arenas and computation trees, we need to accommodate the
definition of computation tree to reflect this property. The
consequence is that in \ialgol, ``computation trees'' become
``computation directed acyclic graphs (DAG)'': a computation DAG may
have (possibly infinitely) many roots and two nodes of a given level
can share children at the next level.


We use the notations $\mathcal{D}_{\iaexp} = \nat$ and
$\mathcal{D}_{\iacom} = \{ \iadone \}$ to denote the set of value
leaves of type \iaexp\ and \iacom\ respectively. There are two types
of value-leaves in the computation DAG: the value-leaf \iadone\ of
type \iacom\ and the value-leaves labelled in $\mathcal{D}_{\iaexp}$
of type \iaexp.

Let $n$ be a node. If $\kappa(n)$ is of type $(A_1,\ldots A_n,B)$,
we call $B$ the \emph{return type of $n$}. The set of value-leaves
of a node $n$ is given by $\mathcal{D}_{\iaexp}$ if the return type
of $n$ is \iaexp, by $\mathcal{D}_{\iacom}$ if its return type is
\iacom, and by $\mathcal{D}_{\iaexp} \union \{ \iadone \}$ if its
return type is \iavar.


Table \ref{tab:ia_computationdag} shows the computation DAG for each
construct of \ialgol. The value-leaves are represented in the DAGs
using the following abbreviations:
$$ \pssetcomptree\tree{n}{ \TRV{\mathcal{D}_\iaexp} }  \quad \mbox{ for }\quad
 \tree{n}{ \TRV{0} \TRV{1} \TRV{2} \TRV{\ldots} }
 \qquad \mbox{ and } \qquad
 \tree{n}{ \TRV{\mathcal{D}_\iadone} }  \quad \mbox{ for }\quad
 \tree{n}{ \TRV{\iadone }}.
$$

A term of type \iavar\ has a computation DAG with an infinite number
of root $\lambda$-nodes. Suppose that $M$ is a term of type \iavar,
then the computation DAG for $\lambda \overline{\xi} . M$ is
obtained by relabelling the root $\lambda$-nodes $\lambda^r$,
$\lambda^{w_0}$, $\lambda^{w_1}$, $\lambda^{w_2}$, \ldots into
$\lambda^r \overline{\xi}$, $\lambda^{w_0} \overline{\xi}$,
$\lambda^{w_1} \overline{\xi}$, $\lambda^{w_2} \overline{\xi}$,
\ldots. For a term $M$  of type \iaexp\ or \iacom, the computation
DAG for $\lambda \overline{\xi} . M$ is computed in the same way as
in the safe $\lambda$-calculus.

\begin{table}
\begin{center}
\begin{tabular}{cc}
$M$ & $\tau(M)$ \\ \hline \hline \\
x $: A \in \{ \iacom, \iaexp \}$ &
    $\psmatrix[colsep=3ex,rowsep=2ex] \lambda \\ x & \mathcal{D}_A \\  & \mathcal{D}_A \endpsmatrix
    \ncline{1,1}{2,1} \valueedge{1,1}{2,2} \valueedge{2,1}{3,2} $
\\ \\
x : \iavar &
    $\psmatrix[colsep=3ex,rowsep=3ex]
    \lambda^r & \lambda^{w_0} & \lambda^{w_1}  & \lambda^{w_2} & \lambda^{w_{\ldots}} \\
    \mathcal{D}_\iaexp &  & x & & \iadone \\
    &  &  & \mathcal{D}_\iaexp & \iadone
    \endpsmatrix
    \ncline{1,1}{2,3} \ncline{1,2}{2,3} \ncline{1,3}{2,3} \ncline{1,4}{2,3} \ncline{1,5}{2,3}
    \valueedge{2,3}{3,4} \valueedge{2,3}{3,5}
    \valueedge{1,1}{2,1}
    \valueedge{1,5}{2,5} \valueedge{1,4}{2,5} \valueedge{1,3}{2,5} \valueedge{1,2}{2,5}
    $
\\ \\
\iaskip : \iacom &
    $\psmatrix[colsep=3ex,rowsep=3ex] \lambda \\ \iaskip & \iadone \\  & \iadone \endpsmatrix
    \ncline{1,1}{2,1} \valueedge{1,1}{2,2} \valueedge{2,1}{3,2} $
\\ \\
$\iaassign\ L\ N :\iacom$ &
    $\psmatrix[colsep=3ex,rowsep=3ex] & \lambda \\ & \iaassign & \iadone \\ \tau(N:\iaexp)  & \tau(L:\iavar) & \iadone \endpsmatrix
    \ncline{1,2}{2,2} \ncline{2,2}{3,2} \ncline{2,2}{3,1}
    \valueedge{1,2}{2,3} \valueedge{2,2}{3,3} $
\\ \\
$\iaderef\ L :\iaexp$ &
    $\psmatrix[colsep=3ex,rowsep=3ex] \lambda \\ \iaderef & \iadone \\ \tau(L:\iavar) & \iadone \endpsmatrix
    \ncline{1,1}{2,1} \ncline{2,1}{3,1} \valueedge{1,1}{2,2} \valueedge{2,1}{3,2} $
\\ \\
$\iaseq_{\iaexp}\ N_1\ N_2 :\iacom$ &
    $\psmatrix[colsep=3ex,rowsep=3ex] & \lambda \\ & \iaseq_{\iaexp} & \mathcal{D}_\iaexp \\ \tau(N_1:\iacom)  & \tau(N_2:\iaexp) & \iadone \endpsmatrix
    \ncline{1,2}{2,2} \ncline{2,2}{3,2} \ncline{2,2}{3,1}
    \valueedge{1,2}{2,3} \valueedge{2,2}{3,3} $
\\ \\
$\iamkvar\ N_w\ N_r :\iavar$ &
    $\psmatrix[colsep=3ex,rowsep=3ex]
    \lambda^r & \lambda^{w_0} & \lambda^{w_1}  & \lambda^{w_2} & \lambda^{w_{\ldots}} \\
    \mathcal{D}_\iaexp &  & \iamkvar & & \iadone \\
    & \tau(N_r) & \tau(N_w) & \mathcal{D}_\iaexp & \iadone
    \endpsmatrix
    \ncline{1,1}{2,3} \ncline{1,2}{2,3} \ncline{1,3}{2,3} \ncline{1,4}{2,3} \ncline{1,5}{2,3}
    \ncline{2,3}{3,2} \ncline{2,3}{3,3}
    \valueedge{2,3}{3,4} \valueedge{2,3}{3,5}
    \valueedge{1,1}{2,1}
    \valueedge{1,5}{2,5} \valueedge{1,4}{2,5} \valueedge{1,3}{2,5} \valueedge{1,2}{2,5}
    $
\\ \\
$\ianewin{x}\ N : A \in \{ \iacom, \iaexp \} $ &
   $\psmatrix[colsep=3ex,rowsep=3ex] \lambda \\ \ianewin{x} & \mathcal{D}_A \\ \tau(N:A) & \mathcal{D}_A \endpsmatrix
    \ncline{1,1}{2,1} \ncline{2,1}{3,1} \valueedge{1,1}{2,2} \valueedge{2,1}{3,2} $
\end{tabular}
\end{center}
  \caption{Computation DAGs for the constructs of \ialgol.}
  \label{tab:ia_computationdag}
\end{table}


\subsubsection{Traversals}
Let $p$ be a node and suppose that its $i$th child $n$ has the
return type \iavar. Then $n$ is in fact constituted of several
$\lambda$-nodes : $\lambda^r \overline{\xi}$, $\lambda^{w_0}
\overline{\xi}$, \ldots. From $p$'s point of view, these nodes are
referenced as follows: $i.r$ refers to $\lambda^r \overline{\xi}$
and  $i.w_k$ refers to $\lambda^{w_k} \overline{\xi}$ for $k \in
\omega$.

\begin{itemize}
\item \emph{The application rule}

There are two rules (app$_{\iaexp}$) and (app$_{\iacom}$)
corresponding to traversals ending with an @-node of return type
\iaexp\ and \iacom\ respectively. These rules are identical to the
rule \iaexp\ of section \ref{subsec:traversal}.

The application rule for $@$-nodes with return type \iavar\ is:
$$(\mbox{app}_{\iavar})
\rulef{ \Pstr{t \cdot (lHyp){\lambda^k \overline{\xi}} \cdot
(appHyp-lHyp,35:0){@} \in \travset }
 }{\Pstr[18pt] {t \cdot (l){\lambda^k
\overline{\xi}} \cdot (app-l,35:0){@} \cdot (l2-app,35:0.k){\lambda^k
\overline{\eta}} \in \travset }}
 \ k \in \{ r, w_0, w_1, \ldots \}
$$


\item \emph{Input-variable rules}

There are two rules (InputVar$^{\iaexp}$) and (InputVar$^{\iacom}$)
which are the counterparts of rule (InputVar$^0$) of section
\ref{subsec:traversal} and are defined identically.

Let $x$ be an input-variable of type \iavar:
$$ (\mbox{InputVar}^{\iavar})
\rulef{t \cdot \lambda^r \overline{\xi} \cdot x \in \travset}
    {t \cdot \lambda^r \overline{\xi} \cdot \rnode{x}{x} \cdot v_x \in \travset }
\hspace{2cm} (\mbox{InputVar}^{' \iavar}) \rulef{t \cdot
\lambda^{w_i} \overline{\xi} \cdot x \in \travset}
    {t \cdot \lambda^{w_i} \overline{\xi} \cdot \rnode{x}{x} \cdot \iadone_x \in \travset }
$$

\item \emph{IA constants rules}

The rules for \ianew\ are purely structural, they are defined the
same way as the rules (app$_{\iaexp}$), (app$_{\iacom}$) and
(app$_{\iadone}$).

The rules for \iaderef\ are:
$$(\mbox{deref}) \rulef{t \cdot \iaderef \in \travset}{\Pstr[15pt]{t \cdot (d){\iaderef} \cdot (n-d,35:1.r){n} \in \travset }}
 \hspace{1.6cm} (\mbox{deref'})
\rulef{t \cdot \iaderef \cdot n \cdot t_2 \cdot v_n \in \travset} {t
\cdot \iaderef \cdot n \cdot t_2 \cdot v_n \cdot v_{\iaderef}\in
\travset }
$$


The rules for \iaassign\ are:
$$(\mbox{assign}) \rulef{t \cdot \iaassign \in \travset}{\Pstr[15pt]{t \cdot (ass){\iaassign} \cdot (n-ass,35:1){n} \in \travset} }
\hspace{1.6cm}
(\mbox{assign'})
\rulef{t \cdot \iaassign \cdot n \cdot t_2 \cdot v_n \in
\travset} {\Pstr[18pt]{t \cdot (ass){\iaassign} \cdot (n){n} \cdot
t_2 \cdot v_n \cdot (m-ass,15:2.w_n){m} \in \travset } }
$$
$$(\mbox{assign''})  \rulef{\Pstr{t \cdot (assHyp){\iaassign} \cdot t_2 \cdot (mHyp-assHyp,35:2.w_k){m} \cdot t_3 \cdot \iadone_m \in \travset}}
{t \cdot \iaassign \cdot t_2 \cdot m \cdot t_3 \cdot \iadone_m \cdot
\iadone_{\iaassign} \in \travset }
$$

The rules for $\iaseq_{\iaexp}$ are:
$$(\mbox{seq}) \rulef{t \cdot \iaseq \in \travset}{\Pstr[13pt]{t \cdot (seq){\iaseq} \cdot (n-seq,35:1){n} \in \travset } }
\hspace{1.6cm} (\mbox{seq'})
\rulef{t \cdot \iaseq \cdot n \cdot t_2 \cdot v_n \in
\travset} {\Pstr[18pt]{ t \cdot (seq){\iaseq} \cdot (n){n} \cdot t_2
\cdot v_n \cdot (m-seq,25:2){m} \in \travset }}
$$
$$(\mbox{seq''})  \rulef{\Pstr{t \cdot (seqHyp){\iaseq} \cdot t_2 \cdot (mHyp-seqHyp,35:2){m} \cdot t_3 \cdot v_m \in \travset}}
{t \cdot \iaseq \cdot t_2 \cdot m \cdot t_3 \cdot v_m \cdot
v_{\iaseq} \in \travset }$$




The rules for \iamkvar\ are:
$$(\mbox{mkvar}_r) \rulef{t \cdot \lambda^r \overline{\xi} \cdot \iamkvar \in \travset}{\Pstr[14pt]{t \cdot \lambda^r \overline{\xi} \cdot (d){\iamkvar} \cdot (n-d,35:1){n} \in \travset} }
\hspace{1cm} (\mbox{mkvar}_r')
\rulef{t \cdot \iamkvar \cdot n \cdot t_2 \cdot v_n \in \travset} {t
\cdot \iamkvar \cdot n \cdot t_2 \cdot v_n \cdot v_{\iamkvar}\in
\travset } $$
$$(\mbox{mkvar}_w) \rulef{t \cdot \lambda^{w_k} \overline{\xi} \cdot \iamkvar \in \travset}{\Pstr[15pt]{t \cdot \lambda^{w_k} \overline{\xi} \cdot (mk){\iamkvar} \cdot (n-mk,35:2){n} \in \travset} }$$
$$ (\mbox{mkvar}_w'')  \rulef{t \cdot \lambda^{w_k} \overline{\xi} \cdot \iamkvar \cdot n \cdot t_2 \cdot \iadone_n \in \travset}
{t \cdot \lambda^{w_k} \overline{\xi} \cdot \iamkvar \cdot n \cdot
t_2 \cdot \iadone_n \cdot \iadone_{\iamkvar} \in \travset }
$$
These four rules are not sufficient to model the constant \iamkvar.
Indeed, consider the term $\iaassign\ (\iamkvar\ (\lambda x . M) N)
7$. The rule (\mbox{mkvar}$_w''$) permits to traverse the node
\iamkvar\ and to go on by traversing the computation tree of
$\lambda x . M$. The problem is that when traversing $\tau(M)$, if
we reach a variable $x$, we are not able to relate $x$ to the value
$7$ that is assigned to the variable.

To overcome this problem, we need to define traversal rules for
variable in such a way that a variable node bound by the second
child of a $\iamkvar$-node is treated differently from other
variables.

\item \emph{Variable rules}
Let $x$ be a non input-variable node. It either corresponds to a $\lambda$-abstracted variable or
a block-allocated variable declared by the $\ianewin{x}$ construct.

\begin{itemize}
\item Suppose that $x$ is $\lambda$-abstracted and let $\lambda \overline{x}$ be its binder.
In \ialgol, the only constant nodes of order greater than 1 is
\iamkvar, therefore there are two cases: $\lambda \overline{x}$ is
either the child of a node in $N_@ \union N_{\sf var}$ or it is the
second child of a \iamkvar-node.

To handle the first case, we define a rule similar to the (Var) rule
of section \ref{subsec:traversal} with some modification to take
into account variables $x$ of type \iavar (in which case $x$ has
multiple parent $\lambda$-nodes). We do not give the details here
but it is easy to see how to redefine this rule.

To handle the case where $\lambda \overline{x}$ is the child of a
\iamkvar-node, we define the following rule:
$$ (\mbox{Var}_{\iamkvar})  \rulef{t \cdot \lambda^{w_k} \overline{\xi} \cdot \iamkvar \cdot \lambda \overline{x} \cdot t_2 \cdot x \in \travset}
{t \cdot \lambda^{w_k} \overline{\xi} \cdot \iamkvar \cdot \lambda
\overline{x} \cdot t_2 \cdot x \cdot k_{x} \in \travset }
$$

\item Suppose that $x$ is block-allocated with $\ianewin{x}$.

We call \emph{overwrite of $x$ relatively to an occurrence of a} ``\ianewin{x}''\emph{-node}, any sequence of nodes of the form
$\Pstr[17pt]{(decl){\ianewin{x}}\cdot \ldots \cdot \lambda^{w_k}\overline{\xi} \cdot (x-decl,25){x}}$ for some $k\in \mathcal{D}_{\iaexp}$ and node $\lambda^{w_k}\overline{\xi}$ parent
of $x$.
$$(\mbox{Var}_w)
    \rulef{
        t \cdot \lambda^{w_k} \overline{\xi} \cdot x \in \travset
    }
    {   t \cdot \lambda^{w_k} \overline{\xi} \cdot x \cdot \iadone_x \in
        \travset
    },
$$

$$(\mbox{Var}_r)
    \rulef{
        \Pstr[17pt]{t_1 \cdot (decl){\ianewin{x}} \cdot t_2 \cdot \lambda^r \overline{\xi} \cdot (x-decl,25){x} \in \travset}
    }
    {   t_1 \cdot \ianewin{x} \cdot t_2 \cdot \lambda^r \overline{\xi}
        \cdot x \cdot 0_x \in \travset
    }
    \mbox{ if $t_2$ contains no overwrite of $x$},
$$

$$(\mbox{Var'}_r)
    \rulef{
        \Pstr[15pt]{
            t_1 \cdot (decl){\ianewin{x}} \cdot t_2 \cdot \lambda^r \overline{\xi} \cdot (x-decl,25){x} \in \travset
        }
    }
    {
        t_1 \cdot \ianewin{x} \cdot t_2 \cdot \lambda^r \overline{\xi} \cdot x \cdot k_x \in \travset
    }
    \mbox{ if $\lambda^{w_k} \cdot x$ is the last overwrite of $x$ in } t_2. $$
\end{itemize}
\end{itemize}
 
\subsubsection{Game semantics correspondence}
The properties that we proved for computation trees and traversals
of the safe $\lambda$-calculus with constants can easily be lifted
to computation DAGs of \ialgol. In particular:
\begin{itemize}
\item constant traversal rules are well-behaved (for order-$0$ and order-$1$ constants, this is a consequence
of Lemma \ref{lem:sigma_order1_are_wellbehaved}; for $\iamkvar$
however it needs to be proved separately);
\item P-view of traversals are paths in the computation DAG;
\item the P-view of the reduction of a traversal is the reduction of the P-view,
and the O-view of a traversal is the O-view of its reduction
(Lemma \ref{lem:pview_trav_projection} and
\ref{lem:oview_trav_projection});
\item there is a mapping from vertices of the computation DAG to moves in the interaction game semantics;
\item there is a correspondence between traversals of the computation tree and plays in interaction game semantics;
\item consequently, there is a correspondence between the standard game semantics and
the set of justified sequences of nodes $\travset(M)^{\filter r}$.
\end{itemize}

\subsubsection{Game-semantic characterisation of safe terms}
Clearly, the computation DAG of a safe term is incrementally-bound.
By using the correspondence between traversals and plays, it is easy
to prove that incrementally-bound computation trees are denoted by
P-incrementally-justified strategies. Consequently, by lemma
\ref{lem:incrjustified_pointers_uniqu_recover}, P's pointers are superfluous in the
game semantics of safe \ialgol\ terms.

Since the game denotation of an \ialgol\ term is fully determined by
the set of complete plays, this pointer economy suggests that the
game denotation of a safe \ialgol\ can be represented in a compact
way. This raises the question of the decidability of observational
equivalence for safe \ialgol.



%%%%%%%%%%%%%%%%%%%%%%%%%%%%
%%%%%%%%%%%%%%%%%%%%%%%%%%%%
\notetoself{the following section needs to be integrate into the previous chapter.}


\section{Game-semantic of Safe PCF}
In this section will give a game-semantic characterization of Safe
PCF based on syntactical arguments.

\begin{definition}
We say that a PCF term is \defname{semi-safe} if it is of the form
$N_0 N_1 \ldots N_k$ for $k\geq 1$ where each of the $N_i$ is a Safe
PCF term or if it can be written $\lambda \overline{x} . N$ for some
safe PCF term $N$.
\end{definition}
Semi-safe terms are either safe or ``almost safe'' in the sense that
they can be turned into an equivalent (i.e.~with isomorphic game
semantics) safe term  by performing $\eta$-expansions. Indeed, let
$M$ be an semi-safe term that is unsafe. If $M$ is of the first form
$N_0 N_1 \ldots N_k : (A_1,\ldots,A_n)$ with $k\geq 1$ then let
$\varphi_i:A_i$ for $i\in\{1..n\}$ be fresh variables, using the
(app) and (abs) rules we can build the safe term $\lambda \varphi_1
\ldots \varphi_n . N_0 N_1 \ldots N_k \varphi_1 \ldots \varphi_n$.
If $M$ is of the second form $\lambda \overline{x} . N$ then using
the abstraction rule we can build the equivalent safe term $\lambda
\overline{y} \overline{x}. N$  where $\overline{y} = fv(\lambda
\overline{x}. N)$.

The $\beta$-normal form of a \pcf\ term is the possibly infinite
term obtained by reducing all the redexes in $M$.

\subsubsection{Safe terms vs P-i.j.\ strategies}

In the context of the simply typed lambda calculus, the
correspondence between safety and P-incremental justification was
first shown in \cite[Theorem 3(ii)]{blumong:safelambdacalculus}
using a syntactic argument:
\begin{theorem}[\cite{blumong:safelambdacalculus},Theorem 3(ii)]
\label{thm:safeincrejust}
 In the simply typed lambda calculus:
\begin{enumerate}[(i)]
\item If $M$ is safe then $\sem{M}$ is P-incrementally justified.
\item If $M$ is a closed term and $\sem{M}$ is
  P-incrementally justified then the $\eta$-long form of the
  $\beta$-normal form of $M$ is safe.
\end{enumerate}
\end{theorem}
In fact the following more precise result holds (the proof of the
previous theorem can be easily adapted to this one):
\begin{theorem}[Semi-safety and P-incremental justification]
\label{thm:semisafeincrejust} Let $\Gamma \vdash M : A$ be a simply typed term. Then:
\begin{enumerate}[(i)]
\item If $\Gamma \vdash M : A$ is semi-safe then $\sem{\Gamma \vdash M : A}$ is P-incrementally justified.
\item If $\sem{\Gamma \vdash M : A}$ is
  P-incrementally justified then $\etalnf{\betanf{M}}$ is
semi-safe if $M$ is open and safe if $M$ is closed.
\end{enumerate}
\end{theorem}



In the context of \pcf\ however, only the first part of the theorem
holds (see \cite{blumtransfer} for the proof). However (ii) does not
hold. Indeed, take the closed \pcf\ term $M = \lambda f x y. f
(\lambda z. \pcfcond (\pcfsucc\ x) y z )$ where $x,y,z:o$ and
$f:((o,o),o)$. $M$ is in normal form (conditional cannot be reduced
since the value of $x$ is undetermined). The $\eta$-long form of the
$\beta$-normal form of $M$ is therefore $M$ itself which is unsafe.
But clearly we have $\sem{M} = \sem{\lambda f x y. f (\lambda z.
z)}$, and since $\lambda f x y. f (\lambda z. z)$ is safe, by (i),
$\sem{M}$ is P-incrementally justified.

Such counter-example arises because the conditional operator of
\pcf\ permits us to construct terms in normal form that contain
``dead code'' {\it i.e.}~some subterm that will never be evaluated
for any value of M's parameters. In the example above, the dead code
consists of the subterm $y$. In general, if the dead code part of
the computation tree contains a variable that is not incrementally
bound then the resulting term will be unsafe even if the rest of the
tree is incrementally bound. In the example above, it was possible
to turn $M$ into the equivalent safe term $\lambda f x y. f (\lambda
z. z)$ by eliminating the dead code from $M$. In fact we can
generalise this method to any \pcf\ term with a P-incrementally
justified denotation.
\smallskip

Dead code elimination can be difficult to achieve in practice but it
is easy to define it formally: We say that a subterm $N$ occurring
in a context $C[-]$ in $M : (A_1, \ldots, A_n,o)$ is part of the
\defname{dead code} of $M$ if for any term $T_0$ of the form $M M_1
\ldots M_n$, any reduction sequence starting from $T_0$ does not
involve a reduction of the subterm $N$ {\it i.e.}~for any reduction
sequence $T_0 \redar T_1 \redar \ldots \redar T_k$, there is no
$j\in \{0.. k-1\}$ such that $T_j = C[N]$ and $T_{j+1} = C[N']$ for
some term $N'$.


Let $M$  be a \pcf\ term in $\eta$-nf. An occurrence of a variable
$x$ in $M$ is said to be a \defname{dead occurrence} if it occurs in
the dead code of $M$. In other words, it is a dead occurrence of $x$
if the corresponding node in the computation tree does not appear in
any traversal of $\travset(M)$. Equivalently, thanks to the
Correspondence Theorem, an occurrence of $x:B$ is dead if and only
if the initial move of the arena $\sem{B}$ does not appear in any
play of $\sem{M}$.


We define $M^*$ as the term obtained from $M$ after substituting all
subterms of the form  $x N_1 \dots N_k$ for some dead variable
occurrence $x:(B_1,\ldots, B_k, o)$ by the constant $0$. This
process is called \defname{dead variable elimination}. Note that if
$M$ is in $\eta\beta$-nf then so is $M^*$. We also write $\tau(M)^*$
to denote the equivalent transformation on the computation tree.
Since the computation tree is constructed from the $\eta$-nf of $M$,
we will use this notation even when $M$ is not in $\eta$-nf.



\begin{proposition}[Incremental-binding and P-incremental justification coincide] \
\label{prop:Nher_incrbound_and_incrjustified_pcf} Let $\Gamma \vdash
M : A$ be a PCF term in $\beta$-normal form.
\begin{enumerate}[(i)]
\item  If $\tau(\Gamma \vdash M : A)$ is incrementally-bound then $\sem{\Gamma \vdash M : A}$ is P-incrementally justified,
\item  if $\sem{\Gamma \vdash M : A}$ is P-incrementally justified
then $\tau(\Gamma \vdash M : A)^*$ is incrementally-bound.
\end{enumerate}
\end{proposition}
\begin{proof}
(i) The proof is exactly the same as in the simply typed lambda calculus case,
see \cite[Proposition 4.1.5(i)]{blumtransfer}.

\noindent (ii)
Take $\Gamma \vdash M : A$ a \pcf\ term in $\beta$-normal form denoted by $\sem{\Gamma \vdash M : A}$ P-incrementally justified. Let $r$ denote the root of $\tau(M)^*$.
Let $n$ be a node of $\tau(M)^*$ labelled by the variable $x$.
$\tau(M)^*$ is free from dead code therefore $n$ is not a dead occurrence of $x$ and there exists a traversal of $\tau(M)^*$ of the form $t \cdot x$.

\pcf\ constants are of order $1$ at most therefore they cannot
hereditarily justify a variable node, thus $x$ is necessarily
hereditarily justified by the only occurrence $r$ of the root of the
computation tree.

By considering $t\cdot x$ as a traversal of $\tau(M)$,  the
correspondence theorem gives $\varphi((t \cdot x) \filter r) =
\varphi((t \filter r) \cdot x) \in \sem{M}$. Since $\sem{M}$ is
P-incrementally justified, $\varphi(x)$ must point to the last
O-move in $\pview{\varphi(t \filter r)}$ with order strictly greater
than $\ord{\varphi(x)}$. Consequently $x$ points to the last node in
$\pview{t \filter r} \filter N^{\lambda}$ with order strictly
greater than $\ord{x}$. We have:
\begin{align*}
\pview{t \filter r} &= \pview{t} \filter N^{r \vdash} & (\mbox{by Lemma \ref{lem:betanf_wellbehavedconst_trav_pview_red}}) \\
& = [r,x[ \ \filter N^{r \vdash} & (\mbox{by Prop.\ \ref{prop:pviewtrav_is_path}})
\end{align*}
\notetoself{review use of Lemma
\ref{lem:betanf_wellbehavedconst_trav_pview_red}}

Since $M$ is in $\beta$-nf, the set of nodes not hereditarily
enabled by $r$ is exactly the set of nodes hereditarily enabled by
$N_{\Sigma}$ thus $[r,x[ \ \filter N^{r \vdash} = [r,x[\ \setminus\
N^{\filter \Sigma}$. Moreover \pcf\ constants are of order $1$ at
most therefore $N^{\filter \Sigma} = N_{\Sigma} \union N^c_{\Sigma}$
where $N^c_{\Sigma}$ is the set of children nodes of $N_{\Sigma}$.
Thus $\pview{t \filter r} \filter N^{\lambda} = ([r,x[\ \setminus\
N_{\Sigma} \setminus N^c_{\Sigma} ) \filter N^{\lambda} = ([r,x[\
\setminus\  N^c_{\Sigma} )  \filter N^{\lambda}$, and since
$N^c_{\Sigma}$ is constituted of order $0$ lambda-nodes only, $x$
must point to the last node in $[r,x[ \filter N^{\lambda}$ with
order strictly greater than $\ord{x}$.

Hence if $x$ is a bound variable node then it is bound by the
last $\lambda$-node in $[r,x[$ with order strictly greater than
$\ord{x}$ and if $x$ is a free variable then it points to $r$ and
therefore all the $\lambda$-node in $]r,x[$ have order smaller than
$\ord{x}$. Thus $\tau(M)^*$ is incrementally-bound.
\end{proof}

The counterpart of Lemma 4.1.6 from
\cite{blumtransfer} can be stated as follows in the context of PCF:
\begin{lemma}[Semi-safety and incrementally-binding]
\label{lem:incrbound_iff_etanf_safe_pcf} Let $\Gamma \vdash M : A$
be a PCF term.
\begin{itemize}
\item[(i)] If $\Gamma \vdash M : A$ is a semi-safe term then $\tau(\Gamma \vdash M : A)$ is incrementally-bound ;
\item[(ii)] conversely, if $\tau(\Gamma \vdash M : A)$ is incrementally-bound then the $\eta$-normal form of $\Gamma \vdash M : A$ is semi-safe if $M$ is open and safe if $M$ is closed.
\end{itemize}
\end{lemma}
The proof can be obtained by adapting the proof
of Lemma 4.1.6 from \cite{blumtransfer}.

\begin{theorem}[Semi-safety and P-incremental justification]
\label{thm:semisafeincrejust_pcf} Let $\Gamma \vdash M : A$ be a PCF term. Then:
\begin{enumerate}[(i)]
\item If $\Gamma \vdash M : A$ is semi-safe then $\sem{\Gamma \vdash M : A}$ is P-incrementally justified.
\item If $\sem{\Gamma \vdash M : A}$ is
  P-incrementally justified then $\etalnf{\betanf{M}}^*$ is
  semi-safe  if $M$ is open, and safe if $M$ is closed.
\end{enumerate}
\end{theorem}

\begin{proof}
\noindent(i)
A proof of this is given in the proof of Theorem 4.2.10 in \cite{blumtransfer}.

\noindent(ii) Suppose $M$ is a \pcf\ term with a P-incrementally
justified strategy denotation. By Proposition
\ref{prop:Nher_incrbound_and_incrjustified_pcf}(ii),
$\tau(\betanf{M})^* = \tau(\etalnf{\betanf{M}}^*)$ is
incrementally-bound. If $M$ is closed then so is
$\etalnf{\betanf{M}}^*$ therefore by Lemma
\ref{lem:incrbound_iff_etanf_safe_pcf},
$\etalnf{\etalnf{\betanf{M}}^*} = \etalnf{\betanf{M}}^*$ is safe. If
$M$ is open then so is $\etalnf{\betanf{M}}^*$ and by Lemma
\ref{lem:incrbound_iff_etanf_safe_pcf},
$\etalnf{\etalnf{\betanf{M}}^*} = \etalnf{\betanf{M}}^*$ is
semi-safe.
\end{proof}


We write \pcf' to denote the language obtained by extending \pcf\
with the $\pcfcase_k$ construct (see \cite{Abr02}).
The $\pcfcase_k$ construct is the obvious generalisation of the
conditional operator \pcfcond\ to $k$ branches instead of $2$. All the results obtained so far concerning Safe \pcf\ (including those
cited from \cite{blumtransfer}) can clearly be transposed to \pcf'.

\subsubsection{Definability result}

The previous theorem leads to the following definability result for safe \pcf':
\begin{proposition}[Definability for safe \pcf' terms]
\label{prop:safetydefinability} Let $\overline{A}=(A_1,\ldots, A_i)$
and $B =(B_1, \ldots, B_l,o)$ be two PCF types for some $i,l\geq 0$
and $\sigma$ be a well-bracketed innocent P-i.j.\ strategy with
finite view function defined on the game $!A_1 \otimes \ldots
\otimes !A_i \lingamear (!B_1 \lingamear \ldots \lingamear !B_l
\lingamear o) $. There exists a \emph{semi-safe} PCF' term
$\overline{x} : \overline{A} \vdash M : B$ in $\eta$-long normal
form such that:
$$ \sem{\overline{x} : \overline{A} \vdash M_\sigma : B} = \sigma $$
and a safe closed PCF' term $\vdash_s M'_\sigma : (\overline{A},B)$ in $\eta$-long normal form such that:
$$ \sem{\vdash M'_\sigma : (\overline{A},B)} \cong \sigma \ .$$
\end{proposition}
\begin{proof}
By the standard definability result for PCF', there is a term
$\overline{x} : \overline{A} \vdash N : B$ such that
$\sem{\overline{x} :\overline{A} \vdash N : B} = \sigma$. Take
$M_\sigma$ to be $\etalnf{\betanf{N}}^* $. We have
$\sem{\overline{x} : \overline{A} \vdash M_\sigma : B} =
\sem{\overline{x} :\overline{A} \vdash N : B} = \sigma$ and by
Theorem  \ref{thm:semisafeincrejust_pcf}(ii), $M_\sigma$ is
semi-safe. For the second part we just need to take $M'_\sigma =
\lambda \overline{x}. M_\sigma$.
\end{proof}



\subsubsection{Application of the definability result: a syntactic
argument showing compositionality of P-i.j.\ strategies}


We have already shown in Sec. \ref{sec:closedpij} that under certain
conditions, P-i.j.\ strategies compose. Here we will obtain a
slightly weaker version of this result using a much simpler argument
which exploits the definability result from the previous section.


 Let $\overline{A} = (A_1, \ldots, A_i)$, $B = (B_1, \ldots,
B_l,o)$ and $C=(C_1,\ldots,C_k,o)$ be three PCF types for some
$i\geq 1,l,k\geq 0$. Let $f:\ !A_1 \otimes \ldots \otimes !A_i
\lingamear B$ and $g:\ !B\lingamear C$ be two innocent
well-bracketed and P-incrementally justified strategies with finite
view function. We would like to find under which conditions the
composition $f\fatcompos g$ is also P-incrementally justified.

By the definability result, there are two closed safe terms (in $\eta$-nf) $\vdash M_f :(\overline{A},B)$  and $\vdash M_g :B \typear C$ such that $\sem{M_f} = f$
and $\sem{M_f} = g$.
We define the term $M_{f\fatcompos g} = \lambda \overline{x} . M_g (M_f \overline{x})$ for some fresh variables $\overline{x} : \overline{A}$. Clearly we have $\sem{M_{f\fatcompos g}} = \sem{M_f} \fatcompos \sem{M_g} = f\fatcompos g$.

\paragraph{Sufficient conditions}

By Theorem \ref{thm:semisafeincrejust_pcf}, we know that
$f\fatcompos g$ is P-incrementally justified just when
$\etalnf{\betanf{M_{f\fatcompos g}}}^*$ is safe. We will now exploit
this fact to extract a sufficient condition on the types $A$ and $B$
for the composition of $f$ and $g$ to be P-incrementally justified.

The term $M_f$ and $M_g$, being in $\eta$-nf, are of the following forms:
\begin{eqnarray*}
\vdash M_f &=& \lambda x_1^{A_1} \ldots x_i^{A_i} \varphi_1^{B_1} \ldots \varphi_l^{B_l} . N_f^o\\
\vdash  M_g &=& \lambda y^{ (B_1, \ldots, B_l,o)} \phi_1^{C_1} \ldots \phi_k^{C_k} . N_g^o
\end{eqnarray*}
for some distinct variables $x_1, \ldots, x_i$, $y$, $\varphi_1, \dots \varphi_l$, $\phi_1, \dots \phi_k$  and $\eta$-normal terms $N_f$ and $N_g$:
\begin{eqnarray*}
x_1:A, \ldots, x_i:A_i, \varphi_1:B_1, \dots, \varphi_l:B_l &\vdash& N_f :o \\
y: (B_1, \ldots, B_l,o), \phi_1:C_1, \dots, \phi_l:C_l &\vdash& N_g :o
\end{eqnarray*}



The fact that $M_f$ and $M_g$ are safe does not imply that $M_{f\fatcompos g}$ is: take $M_f = \lambda x^o z^o.x$ and $M_g = \lambda y^{(o,o)} . y a$ for some constant $a\in \Sigma$, then $\lambda x:A . M_g (M_f x) = \lambda x . (\lambda y . y a) ( \underline{(\lambda x z.x) x} )$ is unsafe because of the underlined subterm. However we have:
\begin{align*}
f\fatcompos g &= \sem{\lambda \overline{x} . M_g (M_f  \overline{x})} \\
 &= \sem{\lambda \overline{x} . (\lambda \phi_1\ldots \phi_k . N_g) [(M_f \overline{x}) / y]} \\
&= \sem{\lambda \overline{x} \phi_1 \dots \phi_k. N_g [(M_f  \overline{x}) / y]}
& \mbox{(the $x_j$'s and $\phi_j$'s are disjoint)}.
\end{align*}

We now concentrate on the term  $\lambda \overline{x} \phi_1 \dots
\phi_k. N_g [(M_f  \overline{x}) / y]$ and try to find a sufficient
condition guaranteeing its safety.

\subparagraph{A sufficient condition}
\begin{lemma}
Suppose that $\Gamma,y:B \vdash M$ is a safe term in $\eta$-nf and $\Gamma \vdash R : B$ is an almost safe application. Let $N$ denote the set of nodes of the computation tree $\tau(M)$. We have:
\begin{align*}
\Gamma \vdash M[R/y] :A \mbox{ safe }
\iff&  \forall x \in fv(R) . \\
    & \forall n_y \in N_{\sf fv} \mbox{ labelled $y$}.
      \forall m \in N_{\lambda} \inter ]r,n_y] : \ord{m} \leq \ord{x}
\end{align*}
\end{lemma}
\begin{proof}
Since $M$ is in $\eta$-nf, all the application to the variable $y$ are total (i.e.~of the form $y P_1 \ldots P_l :o$). Hence after substituting the safe term $N$ for $y$ in $M$, the only possible cause of unsafety is when
some variable free in $N$ becomes not safely bound in $\tau(M)$.
\end{proof}

Applying this lemma with $R= M_f \overline{x}$ gives us a sufficient
condition -- the right-hand side of the equivalence -- for $\lambda
x \phi_1 \dots \phi_k. N_g [(M_f \overline{x}) / y]$ to be safe, and
hence for $f\fatcompos g$ to be P-incrementally justified. Of course
it is not a necessary condition since $N_g[(M_f \overline{x}) /y]$
can be unsafe while its eta-beta normal form is safe.

\subparagraph{A simpler sufficient condition}
\begin{lemma}
If $y:B, \Sigma \vdash N : T$ and $\vdash M : (\overline{A}, B)$
are safe terms with $\ord{A_i} \geq \ord{B}$ for all $i\in 1..n$
then $\overline{x}:\overline{A}, \Sigma \vdash N[(M \overline{x})/y] :T$ is also safe.
\end{lemma}
\begin{proof}
Since $\ord{x_i} = \ord{A_i} \geq \ord{B} = \ord{M \overline{x}}$, we can use the application
rule of the safe lambda calculus to form the safe term $\overline{x}:\overline{A} \vdash M \overline{x}$.
Using the substitution lemma we have that $N[(M \overline{x})/y]$ is safe.
\end{proof}

Hence we obtain the following sufficient condition for $f\fatcompos
g$ to be P-incrementally justified:
$$\ord{A_i}\geq\ord{B} \mbox{ for all } 1 \leq i \leq n$$


Indeed the lemma gives that $\vdash \lambda \overline{x} \phi_1
\dots \phi_k. N_g [(M_f \overline{x}) / y]$ is safe and therefore
its denotation $\sem{\vdash \lambda \overline{x} \phi_1 \dots
\phi_k. N_g [(M_f \overline{x}) / y]} = f\fatcompos g$ is
P-incrementally justified.

Note that this condition is not necessary: Take $A=o$, $B=(o,o)$,
$C=(o,o)$ and consider the two safe terms $M_f = \lambda x^A u^o.u$
and $M_g = \lambda y^B . y a$ for  some constant $a:o$. Then we have
$M_{f\fatcompos g} = \lambda x . a$ which is safe hence $f\fatcompos
g$ is P-incrementally justified although $\ord{A} < \ord{B}$.

\begin{remark}
This result corroborates what we already know about compositionality
of P-i.j.\ strategies (see Sec. \ref{sec:closedpij}). Indeed, the
condition given hereinbefore implies that the strategy $f$ is
\emph{closed} P-i.j.\ (the $A_i$s are prime because we are working
with PCF types) and therefore by Prop.\ \ref{prop:closedpijcompose},
$f \fatcompos g$ must also be P-i.j.
\end{remark}




\paragraph{Counter-example: two P-i.j.\ strategies whose composition is not
P-i.j.}

We now give counter-example to show that P-i.j.\ strategies do not
compose in general.

\subparagraph{First attempt}

Take the types $A=o$, $B=(o,o)$, $C=o$, the variables
$x,u,v:o$, $y:B$ and $\varphi:((o,o),o)$ and $\Sigma$-constant $a:o$.
Consider the two safe terms $\vdash_s  M_f = \lambda xv.x : A\typear B$ and $\vdash_s M_g = \lambda y . \varphi (\lambda u . y a) : B\typear C$.
The $\eta\beta$-nf of $M_{f\fatcompos g}$ is $\vdash \lambda x . \varphi (\underline{\lambda u . x})$ which is unsafe because of the underlined term. It is then tempting to use
Theorem \ref{thm:safeincrejust}(ii) to conclude that
$\sem{M_{f\fatcompos g}}$ is not P-incrementally justified. However this theorem cannot be used here because $M_g$ contains an order $2$ constants ($\varphi$) therefore
$M_{f\fatcompos g}$ is not a valid simply typed $\lambda$-term (nor a \pcf-term).

\subparagraph{Second attempt} The previous example can be easily
changed into a working counter-example: we just need to elevate
$\varphi$ from the status of constant to variable.

Take $A=o$, $B=(o,o)$, $C=(((o,o),o),o)$, the variables
$x,u,v:o$, $y:B$ and $\varphi:((o,o),o)$ and the $\Sigma$-constant $a:o$. Consider the two safe terms $\vdash_s  M_f = \lambda xv.x : A\typear B$ and  $\vdash_s M_g = \lambda y \varphi. \varphi (\lambda u . y a) : B\typear C$.
The $\eta\beta$-nf of $M_{f\fatcompos g}$ is $\vdash \lambda x \varphi. \varphi (\underline{\lambda u . x})$ which is unsafe because of the underlined term, thus by Theorem \ref{thm:safeincrejust}(ii), $\sem{M_{f\fatcompos g}}=\sem{M_f} \fatcompos
\sem{M_g}$ is not P-incrementally justified. The following diagram illustrates a play that is not P-i.j.:
\begingroup
\def\sigcol#1{{\color{gray} #1}}
\def\mucol#1{{\color{red} #1}}
$$\begin{array}{ccccccccc}
A &  & \multicolumn{2}{c}{B} && \multicolumn{4}{c}{C}\\
\cline{1-1} \cline{3-4} \cline{6-9}
o & \stackrel{\sigcol{\sem{M_f}}}\longrightarrow & o, & o & \stackrel{\mucol{\sem{M_g}}}\longrightarrow & ((o, &o),& o),& o \\ \\
&&&&&&&&\rnode{n0}{\lambda x \varphi \omove  \mucol {\lambda y \varphi}}\\
&&&&&&&\rnode{n1}{\varphi  \pmove \mucol \varphi}\\
&&&&&&\rnode{n2}{\lambda u \omove  \mucol {\lambda u}} \\
&&&  \rnode{n3}{\omove \sigcol {\lambda x v} \pmove \mucol y} \\
\rnode{n4}{x \pmove \sigcol x}
\end{array}
\ncarc[arcangleA=20,arcangleB=20,linecolor=black]{->}{n4}{n0}
\ncarc[arcangleA=30,arcangleB=20,linecolor=red]{->}{n2}{n1}
\ncarc[arcangleA=30,arcangleB=20,linecolor=red]{->}{n1}{n0}
\ncarc[arcangleA=20,arcangleB=20,linecolor=red]{->}{n3}{n0}
\ncarc[arcangleA=20,arcangleB=20,linecolor=gray]{->}{n4}{n3}
$$
\endgroup

\subparagraph{Another counter-example with $\ord{B} = \ord{C}$.}

Let $A=o$, $B=C=(((o,o),o),o)$ and let $x:A$, $y:B$, $u:o$, $v,\varphi:((o,o),o)$
and $g:(o,o)$ be variables and  $a:o$ be a $\Sigma$-constant. Take the two safe terms $\vdash  M_f = \lambda x v.x$ and $\vdash M_g = \lambda y \varphi. \varphi (\lambda u . y (\lambda g. a))$.
The $\eta\beta$-nf of $M_{f\fatcompos g}$ is $\vdash \lambda x \varphi. \varphi (\underline{\lambda u . x})$ which is unsafe because of the underlined term, so
$f\fatcompos g$ is not P-incrementally justified.



\chapter{Game-Semantic Models of Safe Languages}
    \label{chap:model}
    \psset{linecolor=darkGreen,linewidth=0.5pt}

\section{Preliminaries}

We consider an arena $A$ and make the following two assumptions on it:
\begin{itemize}
\item (A1) For $A \neq \bot$ (the arena with a single initial question), each question move in the arena enables at least one answer move.
\item (A2) Answer moves do not enable any other move.
\end{itemize}

An arena is said to be \defname{prime} if it has a single initial move; a type is prime if its arena denotation is prime.

\subsection{Node-order}

\subsubsection{Definition}

We define the \defname{order of a move} $m$ in the arena $A$, written $\ord_A{m}$ (or just $\ord{m}$ where there is no ambiguity), as the length of the path from $m$ to its furthest leaf in $A$ minus 1
({\it i.e.}~the height of the subarena rooted at $m$ minus 2.). Because of assumptions (A1) and (A2),
for any move $m$ of $A \neq \bot$, $m$ is a question move if and only if $\ord{m} \geq 0$, and $m$ is an answer move if and only if $\ord{m} = -1$.

The \defname{order of an arena} $A$ is defined to be the maximal order of its initial moves. The order of a (simple, PCF or IA) type is defined as the order of the arena denoting it - or equivalently as 0 for ground type, $\ord{A\rightarrow B} = \max(1+\ord{A},\ord{B})$ and $\ord(A\times B) = \max(\ord A, \ord B)$. The order of a term is the order of its type.


\subsubsection{Node-order after composition}

Consider the arena $X\lingamear Y$ and let $m$ be a move of
$X\lingamear Y$. We write $\ord_{X\lingamear Y}{m}$ to denote the
order of $m$ in the arena ${X\lingamear Y}$. If $m$ belongs to $X$
(resp.~$Y$) then we write $\ord_X{m}$ (resp.~$\ord_Y{m}$) to denote
the order of the move $m$ in the arena $X$ (resp.~$Y$).

\begin{lemma}
\label{lem:compositionorder} Let $A$, $B$ and $C$ be three arenas.
We have:
$$\begin{array}{lll}
\forall m \in A:
    &  \ord_{A\lingamear B}{m} = \ord_{A\lingamear C}{m} \ ,\\
\forall m \in B:
    & \ord_{A\lingamear B}{m} \geq \ord_{B\lingamear C}{m}  & \mbox{for $m$ initial,}\\
    & \ord_{A\lingamear B}{m} = \ord_{B\lingamear C}{m} & \mbox{for $m$ non initial,} \\
\forall m \in C:
    & \ord_{A\lingamear C}{m} \geq \ord_{B\lingamear C}{m} \iff
\ord{A} \geq \ord{B}\ & \mbox{for $m$ initial,}\\
    & \ord_{A\lingamear C}{m} = \ord_{B\lingamear C}{m}   & \mbox{for $m$ non initial.}
\end{array}
$$
\end{lemma}

\subsection{Well-bracketing}

We call \defname{pending question} of a sequence of moves $s \in L_A$ the last unanswered question in $s$.

\begin{definition}\rm
A strategy $\sigma$ is said to be \defname{P-well-bracketed} if for any play $s \, a \in \sigma$ where $a$ is a  P-answer, $a$ points to the pending question in $s$.
\end{definition}



P-well-bracketing can be restated differently as the following proposition shows:
\begin{proposition}
\label{prop:char_wellbrack}
\rm We make assumption (A1) and (A2).
Let $\sigma$ be a strategy on an arena $A\neq \bot$.
The following statements are equivalent:
\begin{enumerate}
\item[(i)] $\sigma$ is P-well-bracketed,
\item[(ii)] for $s \, a \in \sigma$ with $a$ a P-answer, $a$ points to the pending question in $\pview{s}$,
\item[(iii)] for $s \, a \in \sigma$ with $a$ a P-answer, $a$ points to the last O-question in $\pview{s}$,
\item[(iv)] for $s \, a \in \sigma$ with $a$ a P-answer, $a$ points to the last O-move in $\pview{s}$ with order $>\ord{a}$.
\end{enumerate}
\end{proposition}
\begin{proof}
$(i)\iff(ii)$: \cite[Lemma 2.1]{McC96b} states that if P is to move then the pending question in $s$ is the same as that of $\pview{s}$.

$(ii)\iff(iii)$: Assumption (A2) implies that the pending question in $\pview{s}$ is also the last O-question occurring in $\pview{s}$.

$(iii)\iff(iv)$: Because of assumption (A1) and (A2),
for any move $m$, we have $m$ is a question move
if and only if $\ord{m} \geq 0$ if and only if $\ord{m} > \ord{a} = -1$.
\end{proof}




\begin{lemma}
\label{lem:justfied_by_unanswered}
Under assumption (A2), if $s$ be a justified sequence of moves satisfying alternation and visibility then any O-move (resp. P-move) in $s$ points to an \emph{unanswered} P question (resp. O-question).
\end{lemma}
\begin{proof}
Suppose that an O-move $c$ points to a P-move $d$ that has already been answered by the O-move $a$. The sequence $s$ as the following form:
$$ s= \ldots \Pstr{(d){d}  \ldots  (a-d,20){a}  \ldots  (c-d,20){c}}$$

By O-visibility, $d$ must belong to $\oview{s_{<c}}$. But since $a$ is an answer, by assumption (A2), it cannot justify any P-move, therefore
$\oview{s_{<q}}$ must contain an OP-arc ``hoping'' over $a$. We name the nodes of this arc $d^1$ and $c^1$:
$$ s = \ldots \Pstr[0.7cm]{(d){d}  \ldots  (d1){d^1} \ldots (a-d,20){a} \ldots
 (c1-d1,20){c^1} \ldots (c-d,25){c}}$$

By P-visibility, $d^1$ must belong to $\pview{s_{<c^1}}$. Consequently, $a$ does not belong to $\pview{s_{<c^1}}$ (otherwise the PO-arc $\Pstr[0.5cm]{(d){d} \quad (a-d,45){a}}$ would cause the P-view to jump over $d^1$).
Therefore there must be a PO-arc $\Pstr[0.5cm]{(d2){d^2} \quad (c2-d2,45){c^2}}$ in $\pview{s_{<c^1}}$ hoping over $a$:
$$ s = \ldots \Pstr[0.7cm]{(d){d}  \ldots
(d1){d^1} \ldots (d2){c^2} \ldots
(a-d,20){a} \ldots
 (c2-d2,20){d^2} \ldots (c1-d1,20){c^1} \ldots (c-d,25){c}}$$

This process can be repeated infinitely often by using alternatively O-visibility and P-visibility. This gives a contradiction since the sequence of moves $s_{<c}$ has finite length.
Hence $d$ cannot point to a question that has already been answered. Since, by assumption (A2), a question is enabled by another question, $d$ is necessarily justified by an unanswered question.
\end{proof}


\begin{lemma}
\label{lem:oq_in_pview_unanswered}
Under assumption (A2), if $s$ is a P-well-bracketed justified sequence of moves of odd length satisfying alternation and visibility then  all O-questions occurring in $\pview{s}$ are unanswered in $s$.
\end{lemma}
\begin{proof}
We proof the first part by induction on $s$.
The base case ($s = q$ with $q$ initial O-move) is trivial.

Suppose $\Pstr[0.4cm]{ s = s' \cdot (n)n \cdot u \cdot (m-n,45){m} }$.
Let $r$ be an O-question in $\pview{s} = \pview{s'} \cdot n \cdot m$.
If $r$ is the last move $m$ then it is necessarily unanswered.
If $r \in \pview{s'}$ then by the induction hypothesis, $r$ is unanswered in $s'$.
Suppose that $r$ is answered in $s$. This implies that some answer move $a$ in $u$ points to $r$:
$$\pstr[0.7cm][5pt]{ s = \underbrace{\cdots\ \nd(r){r}^O \cdots }_{s'} \
\nd(n){n}^P \ \underbrace{\cdots\ \nd(a-r,35){a}^P \cdots }_{u} \
\nd(m-n,30){m}^O } \ .$$

Since $m$ points to $n$, by lemma \ref{lem:justfied_by_unanswered}, $n$ is still unanswered at $s_{\prefixof a}$. Therefore the pending
question at $s_{\prefixof a}$ cannot be $r$. But $a$ is justified by $r$, therefore the well-bracketing condition is violated. Hence $r$ is
unanswered in $s$.
\end{proof}








\subsection{Interaction sequences} Let us first recall the
definition of an interaction sequence. Let $A$,$B$ and $C$ be three
games. We say that $u$  is an
\defname{interaction sequence} of $A$,$B$ and $C$ whenever $u\filter
A,B$ is a valid position of the game $A\lingamear B$ (i.e.~$u\filter
A,B \in P_{A\lingamear B}$) and  $u\filter B,C$ is a valid position
of the game $B\lingamear C$. We write $Int(A,B,C)$ to denote the set
of all such interaction sequences.

Let $\sigma:A\lingamear B$ and $\mu:B\lingamear C$ be two
strategies. We write $\sigma \parallel \mu$ to denote the set of
interaction sequences that unfold according to the strategy $\sigma$
in the $A,B$-projection of the game and to $\mu$ in the
$B,C$-projection:
$$ \sigma \parallel \mu = \{ u \in Int(A,B,C) \ | \ u\filter A,B \in \sigma \wedge u \filter B,C \in \mu \} \ .$$
The composite of $\sigma$ and $\mu$ is then defined as $\sigma ; \mu
= \{ u \filter A,C \ | \ u \in \sigma \parallel \tau \}$.

The diagram below shows the structure of an interaction sequence
from $\sigma \parallel \mu$. There are four states represented by
the rectangular boxes. The content of the state shows who is to play
in each of the game $A\lingamear B$, $B\lingamear C$ and
$A\lingamear C$. For instance in state $OPP$, it is O's turn to play
in $A\lingamear B$ and P's turn to play in $B\lingamear C$ and
$A\lingamear C$. Arrows represent the moves. When specifying
interaction sequence, the following bullet symbols are used to
represent moves: $\pmove$ for P-moves, $\omove$ for O-moves,
$\pomove$ for a move playing the role of P in $A\lingamear B$ and O
in $B\lingamear C$ and $\opmove$ for the symmetric of $\pomove$. We
sometimes add a subscript to the symbols $\pmove$ and $\omove$ to
denote the component in which the moves is played ($A$ or $C$).


\tikzstyle{state}=[rectangle,draw=blue!50,fill=blue!20,thick,minimum
height = 4ex, text width=4cm] \tikzstyle{move}=[->,shorten
<=1pt,>=latex',line width=1pt] \tikzstyle{intmove}=[dashed]
\tikzstyle{extomove}=[color=\extomovecolor]
\tikzstyle{genomove}=[]%[dashed]
\tikzstyle{genpmove}=[color=\genpmovecolor]
\def\sep{1.5cm}
\begin{figure}[htbp]
\begin{center}
\begin{tikzpicture}[node distance=1.7cm]

% the four states
\path
 node(oooT)  [state] {}
 node(opp)   [state, below of=oooT] {}
 node(pop)   [state, below of=opp]  {}
 node(oooB)  [state, below of=pop] {}
 node(title) [anchor=south, at=(oooT.north), minimum height = 4ex, text width=4cm] { };

\path
% text in the title centered in 3 columns
  ([xshift=-\sep]title) node {$A\lingamear B$}
        (title) node {$B\lingamear C$}
        ([xshift=\sep]title) node {$A\lingamear C$}

% text in the states centered in 3 columns
  ([xshift=-\sep]oooT) node {O}
        (oooT) node {O}
        ([xshift=\sep]oooT) node {O}
  ([xshift=-\sep]opp) node {O}
        (opp) node {P}
        ([xshift=\sep]opp) node {P}
  ([xshift=-\sep]pop) node {P}
        (pop) node {O}
        ([xshift=\sep]pop) node {P}
  ([xshift=-\sep]oooB) node {O}
        (oooB) node {O}
        ([xshift=\sep]oooB) node {O}

% text in between two arrows giving the arena of the move
  (oooT) to node {\bf C} (opp)
  (opp) to node {\bf B} (pop)
  (pop) to node {\bf A} (oooB)

% arrows representing the moves
  (opp.20)    edge[move, genpmove]
        node[right] {$\mu$}
        node[left]{$\pmove$} (oooT.-20)
  (oooT.-160) edge[move, extomove, genomove]
        node[left] {$env_\mu$}
        node[right]{$\omove$} (opp.160)
  (pop.20)    edge[move, genomove,genpmove,intmove]
        node[right] {$\sigma$}
        node[left]{$\pomove$} (opp.-20)
  (opp.-160)  edge[move, genomove, genpmove,intmove]
        node[left] {$\mu$}
        node[right]{$\opmove$}  (pop.160)
  (oooB.20)   edge[move, extomove,genomove]
        node[right] {$env_\sigma$}
        node[left]{$\omove$} (pop.-20)
  (pop.-160)  edge[move, genpmove]
        node[left] {$\sigma$}
        node[right]{$\pmove$} (oooB.160);

%\draw[move, genpmove] (3.5cm,-1cm) -- +(1,0) node[right] {Generalised P-move \& External P-move };
%\draw[move, genomove,genpmove] (3.5cm,-2cm) -- +(1,0) node[right] {Generalised O-move \& Generalised P-move};
%\draw[move, genomove,extomove] (3.5cm,-3cm) -- +(1,0) node[right] {Generalised O-move \& External O-move};
\draw[move] (3.5cm,-1cm) -- +(1cm,0cm) node[right] {External move};
\draw[move,intmove] (3.5cm,-2cm) -- +(1cm,0cm) node[right] {Internal
move}; \draw (3.5cm,-3cm) node[anchor=west]
{\textcolor{\extomovecolor}{External O-moves: $\omove$}}; \draw
(3.5cm,-4cm) node[anchor=west]
{\textcolor{\genpmovecolor}Generalised P-move: $\opmove, \pomove,
\pmove$};
\end{tikzpicture}
\end{center}
\caption{Structure of an interaction sequence.} \label{fig:interseq}
\end{figure}

Note that in state OPP, the alternation condition (for each of the
three games involved) prevents the players from playing in A.
Indeed, the O-moves in component $A$ of $A\lingamear B$ are also
$O$-moves in component $A$ of $A\lingamear C$ however the state name
indicates that the next move in $A\lingamear B$ must be an O-move
and the next move in $A\lingamear C$ must be a P-move.

Similarly, in the top state OOO, the players cannot make move in B
since the O-moves in component B of the game $B\lingamear C$
correspond to P-moves in the component B of $A\lingamear B$. However
the state name indicates that the next move in $A\lingamear B$ and
the next move in $B\lingamear C$ must be played by O.


Let $u \in Int(A,B,C)$ and $m$ be a move of $u$. The
\defname{component} of $m$ is $A,B$ if after playing $m$ the game is
under the control of the strategy $\sigma$ and $B,C$ otherwise (if
$\mu$ has control). In other words, the moves $\omove, \pmove \in A$
and $\opmove \in B$ shown on the diagram of Figure
\ref{fig:interseq} have component $A,B$ and $\omove, \pmove \in C$
and $\pomove \in B$ have component $B,C$.


Also we call \defname{generalized O-move in component $A,B$} moves
that play the role of O in the game $A\lingamear B$, that is to say
moves represented by $\opmove$ and $\omove_A$. Similarly $\pomove$
and $\pmove_A$ moves are the \defname{generalized P-moves in
component $A,B$}, $\omove_C$ and $\pomove$ moves are the
\defname{generalized O-moves in component $B,C$} and  $\pmove_C$ and
$\opmove$ moves are the \defname{generalized P-moves in component
$B,C$}.

The P-view (also called \emph{core} in
\cite{McCusker-GamesandFullAbstrac}) of an interaction sequence $u
\in Int(A,B,C)$, written $\overline{u}$ or $\pview{u}$ is defined
as:
\begin{align*}
\pview{u\cdot \extomove{n}} &= \extomove{n} &
\mbox{ if \extomove{$m$} is an \extomove{external O-move} initial in C,}\\
\pview{\Pstr{u\cdot (m)m\cdot v \cdot (n-m,45){\extomove{n}} }} &= \extomove{n} &\mbox{ if \extomove{$m$} is an \extomove{external O-move} non initial in C,}\\
\pview{u \cdot \genpmove{m}} &= \pview{u}\cdot \genpmove{m}  & \mbox{ if \genpmove{$m$} is a \genpmove{generalised P-move}.}\\
\end{align*}

We can show the following property by an easy induction :
\begin{lemma}
\label{lem:pviewAC_eq_ACpview}
 Let $u$ be an interaction sequence in $Int(A,B,C)$ then
$$\pview{u} \filter A,C = \pview{u \filter A,C} \ .$$
\end{lemma}
\begin{proof}
  By induction on $u$. It is trivial for the empty sequence.
Let $b$ be a move in $B$. We have $\pview{u b} \filter A,C =
\pview{u} \filter A,C$. By the I.H.\ this is equal to $\pview{u
\filter A,C} = \pview{u b\filter A,C}$. Let $m$ be a P-move in $A$
or $C$ then $\pview{u m} \filter A,C = (\pview{u} \filter A,C) m$
and by the I.H.\ this is equal to $\pview{u \filter A,C} m =
\pview{(u \filter A,C) m} = \pview{u m \filter A,C}$. Let $c$ be an
initial move in $C$. We have $\pview{u c \filter A,C}  = \pview{(u
\filter A,C) c} = c =  c \filter A,C = \pview{u c} \filter A,C$. Let
$u = \Pstr{u_1 (m){m} u_2 (n-m){n}}$ with $n$ an O-move in
$A\rightarrow C$. Then necessarily $m\in A,C$ and $ \pview{u\filter
A,C} = \pview{\Pstr[0.5cm]{u_1\filter A,C \cdot (m){m} \cdot
u_2\filter A,C \cdot (n-m,30){n}}} =
 \pview{u_1 \filter A,C} \Pstr{(m){m} (n-m){n}}$. By the I.H.\ this is equal to
$(\pview{u_1}\filter A,C) \Pstr{(m){m} (n-m){n}} = (\pview{u_1}
\Pstr{(m){m} (n-m){n}} ) \filter A,C  = \pview{u_1 \Pstr{(m){m} u_2
(n-m){n}}} \filter A,C$
\end{proof}


\subsection{P-incremental justification}


\begin{definition}\rm
A play $s m$ of even length is said to be \defname{P-incrementally
justified}, or \emph{P-i.j.} for short, if $m$ points to the last
unanswered O-question in $\pview{s}$ with order strictly greater
than $\ord{m}$.

 A strategy $\sigma$ is said to be \defname{P-incrementally justified}, if all plays in $\sigma$ ending with a P-question are
P-incrementally justified.
\end{definition}
Let $\sigma$ be a strategy. We write $Pij(\sigma)$ to denote the set of plays of $\sigma$ that are P-i.j.
We can define equivalently P-i.j.\ strategies as those verifying the relation $\sigma = Pij(\sigma)$.
\begin{proposition}
\label{prop:char_pincr}
\rm We make assumption (A1) and (A2).
Let $\sigma$ be a \emph{P-well-bracketed} strategy on an arena $A\neq \bot$.
The following statements are equivalent:
\begin{enumerate}
\item[(i)] $\sigma$ is P-incrementally justified,
\item[(ii)] for $s \, q \in \sigma$ with $q$ a P-question, $q$ points to the last O-question in $\pview{s}$ with order $>\ord{q}$,
\item[(iii)] for $s \, q \in \sigma$ with $q$ a P-question, $q$ points to the last O-move in $\pview{s}$ with order $>\ord{q}$.
\end{enumerate}
\end{proposition}
\begin{proof}
$(i)\iff(ii)$: By lemma \ref{lem:oq_in_pview_unanswered}, O-question occurring in $\pview{s}$ are all unanswered.

$(ii)\iff(iii)$: Because of (A1) and (A2), $\ord{q} \geq 0$ thus an O-move with order $>\ord{q}$ is necessarily an O-question.
\end{proof}

Putting proposition \ref{prop:char_pincr} and
\ref{prop:char_wellbrack} together we obtain:
\begin{proposition}
Under assumption (A1) and (A2).
A strategy $\sigma$ on $A\neq \bot$
is \emph{P-well-bracketed} and
 \emph{P-incrementally justified} if and only if
for $s \, m \in \sigma$, $m$ points to the last O-move in $\pview{s}$ with order $>\ord{m}$.
\end{proposition}




\section{Closed P-i.j.\ strategies}
\label{sec:closedpij}

\subsection{Definition}

\begin{definition}
\label{def:closedpij} Let $s m$ be an even-length play on some game
$A \rightarrow B$. $s m$ is said to be
\defname{closed P-incrementally justified} (closed P-i.j.\ for short)
just if
\begin{itemize}
\item $s m$ is P-incrementally justified;
\item and if $m$ is an initial move in $A$ then its justifier $n$ (initial in
$B$) verifies $\ord_A m \geq \ord_B n$.
\end{itemize}

\noindent A strategy $\sigma$ is \defname{closed P-i.j.} just if all
plays in $\sigma$ ending with a P-questions are closed P-i.j.
\end{definition}
An example of closed P-i.j.\ strategy is the identity strategy $id_A$
for any game $A$.

\begin{lemma}
\label{lem:closedpij_singleBinitmove} Let $\sigma : A \lingamear B$
be a P-i.j.\ strategy.
\begin{enumerate}[i.]
\item If for each initial move $m$ of $A$ occurring in some play of $\sigma$ we have $\ord_A m \geq \ord{B}$, then $\sigma$ is closed P-i.j.
\item Suppose that $A=A_1\times \ldots \times A_n$ where each of the $A_i$ are prime arenas. If for each initial move $m_i$ of $A_i$, for $i \in \{1..n\}$, occurring in some play of $\sigma$ we have $\ord A_i \geq \ord{B}$, then $\sigma$ is closed P-i.j.
\end{enumerate}
\end{lemma}
\begin{proof}
(i) This is a direct consequence of the definition since $\ord B \geq \ord_B b$ for every move $b$ initial in $B$.

(ii) Take an initial move $m$ of $A$. It is necessary an initial move of $A_i$ for some $i$ hence $\ord_A m = \ord_{A_i} m$ which is equal to $\ord A_i$ since $A_i$ is prime. By hypothesis this is in turn greater than $\ord{B}$ hence we can conclude using (i).
\end{proof}



We observe that every P-i.j.\ strategy $\sigma$ on the game $I
\lingamear A$ is closed P-i.j.\ while $\sigma : A$ is not
necessarily closed P-i.j.\footnote{In particular, every P-i.j.\
strategy $\sigma$ on the game $!A_1 \otimes \ldots \otimes !A_n
\lingamear B$, is isomorphic, up to arena-tagging of the moves, to
the closed P-i.j.\ strategy $\Lambda^n(\sigma)$ on the game $I
\lingamear (A_1,\ldots,A_n,B)$, where $\Lambda$ denotes the usual
{\it currying} isomorphism.}; hence the distinction between $I
\lingamear A$ and $A$ matters. This is because the definition of
closed P-i.j.\ strategy specifically refers to the moves of  the
arena in the left-hand side of the function space arrow
$\lingamear$, therefore the property is not valid up to an
isomorphism that retags the moves such as {\it currying}.

Consequently, it is possible to have two isomorphic strategies $\sigma$ and
$\mu$ such that one is closed P-i.j.\ but not the other. In contrast, the ``ordinary'' P-incremental
justification condition is preserved across the  {\it curry} isomorphism. A consequence of this remark is that the category of closed P-i.j.\ strategies
that we will introduce later on, is not closed (neither monoidal closed nor cartesian closed) and
that it only admits a weak form of {\it curry} isomorphism.

\subsection{Compositionality - A semantic proof}

{\bf Notation} In plays representations, the symbol $\omove$ stands
for an O-move and $\pmove$ for a P-move. Suppose the game considered
is $L\lingamear R$ for some game $L$ and $R$ then whenever the
sub-arena in which the move is played is known, it is specified in
subscripts ($\omove_L$, $\pmove_L$, $\omove_R$ or $\pmove_R$). For
interaction sequences in $Int(A,B,C)$ we use the symbols $\omove_A$,
$\pmove_A$, $\omove_C$, $\pmove_C$, $\opmove$ and $\pomove$ as
defined in Figure \ref{fig:interseq}. We use the variable $X$ to
denote one of the component $A,B$ or $B,C$, the variable  $Y$ then
denotes the other component. We write $s \subseqof t$ to say that
$s$ is a subsequence (with pointers) of $t$, $s \prefixof t$ to say
that $s$ is a prefix (with pointers) of $t$ and  $s \suffixof t$ to
say that $s$ is a suffix of $t$.

We now prove several useful lemmas which will become useful when studying compositionality of P-i.j.\ strategies.

\begin{lemma}
\label{lem:interjump}
Let $X$ be a component (either  $A,B$ or  $B,C$).
Let $u$ be an interaction sequence of the form
$ u =
\Pstr[0.5cm][2pt]{ \ldots (b){\stk \beta \pmove}  \ldots
 {n}  \ldots  (a-b,30){\stk \alpha\omove}
\ldots m}$ where:
\begin{itemize}[-]
\item $\alpha,\beta$ are external moves in component $X$ (necessarily both played in $A$ or in $C$),
\item  $m$ is either played in $B$ or an external P-move in $X$,
\item  $\alpha$ is visible at $m$ in $X$ \emph{i.e.}~$\alpha\in \pview{u \filter X}$ (consequently $\beta$ is also visible).
\end{itemize}
Then $n \not\in \pview{u \filter A, C}$.
\end{lemma}
\begin{proof}
Since $\alpha$ is an O-move, $\alpha$ and $\beta$ are necessarily
played in the same arena ($A$ or $C$). Take $v=u$ if $m$ is a
generalized O-move in $X$ and $v=u_{<z}$ otherwise (if $m$ is a
generalized P-move in $X$). The third assumption implies
$\alpha,\beta\in \pview{v}$. The last move in $v$ is necessarily a
generalized O-move in component $X$ (see diagram of Figure
\ref{fig:interseq}) therefore by \cite[Lemma 3.3.1]{Harmer2005} we
have $\pview{v \filter X} = \pview{\overline{v} \filter X} \subseqof
\overline{v} \subseqof \overline{u}$. Thus $\alpha,\beta \in
\overline{u}$ and since $\alpha,\beta$ are played in $A,C$ we have
$\alpha,\beta  \in \overline{u} \filter A,C = \pview{u
\filter A,C}$ (Lemma \ref{lem:pviewAC_eq_ACpview}). Finally
since $n$ lies underneath the $\beta$-$\alpha$ PO-arc it cannot
appear in the P-view  $\pview{u \filter A,C}$.
\end{proof}

\begin{lemma}
\label{lem:in_pviewAC_imp_in_pviewX}
Let $u$ be an interaction sequence in $Int(A,B,C)$ and
$n$ be a move of $u$ such that $n\in\pview{u \filter A,C}$:
\begin{enumerate}[i.]
\item
if all the moves in $u_{\suffixof n}$
are played in $C$  then $n \in \pview{u \filter B,C}$;
\item
if all the moves in $u_{\suffixof n}$ are played in $A$ then $n \in \pview{u \filter A,B}$.
\end{enumerate}
\end{lemma}
\begin{proof}
\begin{enumerate}[(i)]
\item
We show the contrapositive. Suppose that $n \not\in\pview{u \filter B,C}$. This must be due to one of the following  two
reasons:
\begin{itemize}[-]
\item $\pview{u \filter B,C}$ contains an initial move $c_0 \in C$
occurring after $n$ in $u$.


By \cite[Lemma 3.3.1]{Harmer2005}
we have $\pview{u \filter B,C} = \pview{\overline{u} \filter B,C} \subseqof \pview{u}$, thus $c_0$ also occurs in $\pview{u}$.
Since $c_0$ belongs to $C$ we have
$c_0 \in \pview{u} \filter A,C=
\pview{u \filter A,C}$ (Lemma \ref{lem:pviewAC_eq_ACpview}).
Thus the P-view $\pview{u \filter A,C}$
starts with the initial move $c_0$ and
since $n$ occurs before $c_0$, $n$ does not occur in the P-view.

\item $n$ lies underneath a PO-arc $\beta$-$\alpha$ visible
at $ u \filter B,C$.
By assumption, since $\alpha$ occurs after $n$ in $u$, it must belong to $C$. We can therefore apply Lemma \ref{lem:interjump}
with $X\assignar B,C$ which gives
$n \not\in\pview{u \filter A,C}$.
\end{itemize}

\item Suppose that $n \not\in\pview{u \filter A,B}$ then either:
\begin{itemize}[-]
\item $\pview{u \filter A,B}$ contains an initial move $b_0 \in B$
occurring after $n$ in $u$. But this is impossible since by assumption all the moves occurring after $n$ in $u$ belong to $A$.

\item or $n$ lies underneath a PO-arc $\beta$-$\alpha$ in $A,B$.
By assumption, since $\alpha$ occurs after $n$ it must belong to $A$. We can then conclude using
Lemma \ref{lem:interjump} with $X\assignar A,B$.
\end{itemize}
\end{enumerate}
\end{proof}

Note that we cannot completely relax the assumption
which says that moves in $u_{\suffixof n}$ are all in the same component.
For instance take $u = \Pstr[0.5cm]{(co){\omove_C}\thinspace
(b0-co){\opmove} \thinspace
(n){\stk{\pmove_A}{n}} \thinspace
(b1-co){\opmove}}$ then we have $n\in\pview{u\filter A,C}$ but $n\notin\pview{u\filter A,B}$.


%%%%%%%%%%%
% This commented Lemma could be useful be we did not make use of it eventually.
%
% \begin{lemma}
%\label{lem:oviewsegmentinB}
%For any legal sequence $s = \ldots x \cdot r \cdot y$ of a game $A\lingamear B$ if $x, y \in A$ and $x$ is O-visible from $y$ then any move in $r$ occurring in $\oview{s}$ belongs to $A$.
%\end{lemma}
%\begin{proof}
%We proceed by induction on the length of the segment $r$.
%Base case $r=\epsilon$ is trivial. Suppose $r = r' \cdot m$.
%If $y$ is an O-move then by the Switching Condition
%$m$ is necessarily in $A$. Clearly $x$ is O-visible from $m$ thus  by the I.H.\ any move from $r$ occurring in the O-view is in $A$.
%
%If $y$ is a P-move then it cannot point to an initial move in $B$. Indeed, suppose that it points to an initial O-move $b_0 \in B$ then
%we have $\oview{s} = b_0 \cdot y$ which contradicts the fact that $x\in \oview{s}$.
%Thus $y$ points to a move in $A$ and again we can conclude using the induction hypothesis.
%\end{proof}


\begin{lemma}[P-visibility decomposition (from $C$)]
\label{lem:middlepomove}
Let $u = \ldots n' \cdot r \cdot m \in Int(A,B,C)$ where
$n'$ is a $\omove_A$-move verifying $n' \in \pview{u\filter A,C}$ and $m$ is in $\{ \pmove_C, \opmove, \pomove \}$. Then there is a $\pomove$-move $\gamma$ in $r \cdot m$ such that $\gamma \in \pview{u\filter B,C}$ , $n' \in \pview{u_{\leq \gamma} \filter A,B}$ and $\gamma$ is justified by a move occurring before $n'$.
\end{lemma}
\begin{proof}
By induction on $|r|$.
If $r=\epsilon$ then necessarily $u = \ldots \stk{\omove_A}{n'} \thinspace\stk \pomove m$ where $m$ points before $n'$ ($n'$ being played in $A$ cannot justify $m$ played in $B$) so we just need to take $\gamma = m$.
If $|r|=1$ then either
$u = \ldots \stk{\omove_A}{n'} \pomove\thinspace\stk {\pmove_C} m$
or $u = \ldots \stk{\omove_A}{n'} \pomove\thinspace\stk \opmove m$.
In both cases we can take $\gamma$ to be the $\pomove$-move between $n'$ and $m$.
Suppose $|r|>1$. Let $m^-$ denote the move preceding $m$ in $u$.
We proceed by case analysis:
\begin{enumerate}[i.]
\item Suppose $m = \pmove_C$ and $m^- = \omove_C$.
Let $q$ be the external P-move that justifies $m^-$.
Since $n' \in \pview{u\filter A,C}$, $q$ must occur after $n'$ in $u$:
$$
\begin{array}{ccccl}
A & \stackrel\sigma{\longrightarrow} & B & \stackrel\mu{\longrightarrow} & C \\
&\vdots&&\vdots\\
n' \omove\\
&\vdots&&\vdots  \\
&& & &  \rnode{q}{\pmove}q  \\
&\vdots&&\vdots  \\
&& & &  \rnode{mp}{\omove}m^-  \\
&& & &  \rnode{m}{\pmove}m  \\
\end{array}
\ncarc[arcangleA=60,arcangleB=60]{->}{mp}{q}
 $$
Thus we can use the induction hypothesis (with $u\assignar u_{\prefixof q}$): there is a $\pomove$-move $\gamma$
in $u_{]n',q]}$ pointing before $n'$ such that $\gamma \in \pview{u_{\prefixof q} \filter B,C}$, $n' \in \pview{u_{\prefixof \gamma} \filter A,B}$.
Moreover $\pview{u_{\prefixof q} \filter B,C} \prefixof \pview{u_{\prefixof m} \filter B,C}$ (since $q$ is visible from $m$ in $B,C$) thus we have $\gamma \in \pview{u_{\prefixof m} \filter B,C}$ as required.

\item Suppose $m = \pmove_C$ and $m^- = \pomove \in B$.
Again we can conclude using
the induction hypothesis with $u \assignar u_{\prefixof m^-}$.

\item Suppose $m = \pomove \in B$.

Suppose that all the moves in $r$ are in $A$.
Then $r$ is of the form $(\pmove_A \omove_A)^*$ (where $(\cdot)^*$ denotes the Kleenee star operator).
We just need to take $\gamma = m$.
Indeed, moves in $u_{\suffixof m}$ are all in $A$
and by assumption $n'\in\pview{u\filter A,C}$  therefore
Lemma \ref{lem:in_pviewAC_imp_in_pviewX}(ii) gives
$n'\in\pview{u\filter A,B}$.
Also, since $m$ is a $\pomove$-move,
its justifier is a $\opmove$-move but $r$ contains only $\omove$ and $\pmove$ moves hence $m$'s justifier must occur before $n'$.

Suppose that $r$ contains at least one move in $B$. Let $b$ be the last such move, then $u$ is of the form $\ldots n' \cdot \ldots \cdot \stk\opmove  b \cdot (\pmove_A \omove_A)^* \cdot\thinspace\stk\pomove m $. We then have
$u\filter B,C = \ldots n' \cdot \ldots \cdot
\thinspace\stk\opmove b \thinspace\cdot \stk\pomove m $ thus $b \in \pview{u\filter B,C}$. We can then conclude by applying the induction hypothesis with $u \assignar u_{\prefixof b}$.

\item Suppose $m = \pomove \in B$.
If $m^- = \opmove \in B$ then the I.H.\ with $u \assignar u_{\prefixof m^-}$ permits us to conclude.
If $m^- = \omove \in C$ then we conlude by applying  the I.H.\ on $u \assignar u_{\prefixof q}$ where $q$ is the external P-move in $C$ justifying
$m^-$.
\end{enumerate}
\end{proof}

We now show the lemma symmetric to the previous one:
\begin{lemma}[P-visibility decomposition (from $A$)]
\label{lem:middleopmove}
Let $u = \ldots n' \cdot r \cdot m \in Int(A,B,C)$ where
$n'$ is an O-move \emph{non initial} in $C$ verifying $n' \in \pview{u\filter A,C}$ and $m$ is in $\{\pmove_A, \opmove, \pomove\}$. Then there is a $\opmove$-move $\gamma$ in $r \cdot m$ such that $\gamma \in \pview{u\filter A,B}$ , $n' \in \pview{u_{\leq \gamma} \filter B,C}$ and $\gamma$ is justified by a move occurring before $n'$.
\end{lemma}
\begin{proof}
The proof is almost symmetrical to the previous one (Lemma \ref{lem:middlepomove}). We proceed by induction on $|r|$.
If $r=\epsilon$ then necessarily $u = \ldots \stk {\omove_C} {n'} \thinspace\stk \opmove m$ where $m$ points before $n'$ (it cannot point to $n'$
since $n'$ is not initial in $C$). Thus we just need to take $\gamma = m$.

If $|r|=1$ then either
$u = \ldots \stk {\omove_C} {n'} \thinspace\opmove\thinspace\thinspace\stk{\pmove_A} m$
or $u = \ldots \stk {\omove_C} {n'} \thinspace\opmove\thinspace\thinspace\stk \pomove m$.
In both cases we can take $\gamma$ to be the $\opmove$-move between $n'$ and $m$.
Suppose $|r|>1$. Let $m^-$ denote the move preceding $m$ in $u$.
We do a case analysis:
\begin{enumerate}[i.]
\item Suppose $m = \pmove_A$ and $m^- = \omove_A$.
Let $q$ be the external P-move that justifies $m^-$.
Since $n' \in \pview{u\filter A,C}$, $q$ must occur after $n'$ in $u$:
$$
\begin{array}{rcccl}
A & \stackrel\sigma{\longrightarrow} & B & \stackrel\mu{\longrightarrow} & C \\
&\vdots&&\vdots\\
&&&& \omove\ n'\\
&\vdots&&\vdots  \\
q\rnode{q}{\pmove}  \\
&\vdots&&\vdots  \\
m^- \rnode{mp}{\omove}  \\
m \rnode{m}{\pmove}  \\
\end{array}
\ncarc[arcangleA=-45,arcangleB=-45]{->}{mp}{q}
 $$
Thus we can use the induction hypothesis (with $u\assignar u_{\prefixof q}$): there is a $\opmove$-move $\gamma$
in $u_{]n',q]}$ pointing before $n'$ such that $\gamma \in \pview{u_{\prefixof q} \filter A,B}$, $n' \in \pview{u_{\prefixof \gamma} \filter B,C}$.
Moreover $\pview{u_{\prefixof q} \filter A,B} \prefixof \pview{u_{\prefixof m} \filter A,B}$ (since $q$ is visible from $m$ in $A,B$) thus we have $\gamma \in \pview{u_{\prefixof m} \filter A,B}$ as required.

\item Suppose $m = \pmove_A$ and $m^- = \pomove$ then again we can conclude using the I.H.\ with $u \assignar u_{\prefixof m^-}$.

\item Suppose $m = \opmove$.
\begin{itemize}[-]
\item Suppose that $r$ does not contain any move in $B$  then $r$ is of the form $(\pmove_C \omove_C)^*$.

We just need to take $\gamma = m$.
Indeed:
\begin{enumerate}
\item By lemmma \ref{lem:in_pviewAC_imp_in_pviewX}(i)
we have $n'\in \pview{u\filter B,C}$.

\item  $m$ is justified by a move occurring before $n'$.
Indeed, if $m$ is justified by a $\pomove$-move then since $n' \cdot r$ contains only $\omove$ and $\pmove$ moves, $m$'s justifier must occur before $n'$.
If $m$'s justifier is an initial $\omove_C$-move $c_i$, then
by P-visibility we have $c_i \in \pview{u\filter B,C}$
but since the P-view computation ``stops'' when reaching an initial moves, in order to guarantee that $n'$ also belongs to the P-view (as shown in (a)) it must
occurs after $c_i$.
\end{enumerate}


\item Suppose that $r$ contains some move in $B$. Let $b$ be the last such move. Then $u$ is of the form $u = \ldots n' \cdot \ldots \cdot \stk\opmove  b \cdot (\pmove_A \omove_A)^* \cdot\ \stk\pomove m $.
So we have
$u\filter B,C = \ldots n' \cdot \ldots \cdot \stk\opmove  b \cdot \stk\pomove m $ hence $b \in \pview{u\filter B,C}$. We can now
conclude by applying the I.H.\ with $u \assignar u_{\prefixof b}$.
\end{itemize}

\item Suppose $m = \pomove \in B$.
If $m^- = \pomove \in B$ then the I.H.\ with $u \assignar u_{\prefixof m^-}$ permits us to conclude.
If $m^- = \omove \in A$ then we conclude by applying the I.H.\ on $u \assignar u_{\prefixof q}$ where $q$ is the external P-move in $A$ justifying $m^-$.
\end{enumerate}
\end{proof}

We now use the two preceding Lemmas to show
the following useful result:
\begin{lemma}[Increasing order lemma]
\label{lem:increasing_order}
Let $u = \ldots n' \cdot r \cdot m \in Int(A,B,C)$ where
\begin{enumerate}
\item
$n'$ is an external O-move in compoment $X$
($n'=\omove_A$ and $X=A,B$, or $n'=\omove_C$ and $X=B,C$)  non initial in $C$,
\item $n' \in \pview{u\filter A,C}$,
\item $m$ is either played in $B$
($\opmove$ or $\pomove$) or is an external
 P-move in $Y$
($\pmove_C$ if $n'=\omove_A$ and
$\pmove_A$ if $n'=\omove_C$),
\item $m$'s justifier occurs before $n'$,
\item $u\filter X$ is P-i.j.,
\item $u_{\prefixof b}\filter Y$ is P-i.j.\ for all non-initial B-move $b$ occurring in $u$.
\end{enumerate}
Then:
$$ \ord_{Y} m \geq \ord_{A\lingamear C} n' \ .$$
\end{lemma}
\begin{proof}
If $n' =\omove_C$ (resp.~if $n'=\omove_A$)
then by Lemma \ref{lem:middleopmove}
(resp.~Lemma \ref{lem:middlepomove})
there is an occurrence in $r \cdot m$ of a non-initial B-move $\gamma$ of type $\opmove$
(resp.~$\pomove$) such that $\gamma \in \pview{u\filter Y}$ , $n' \in \pview{u_{\leq \gamma} \filter X}$ and $\gamma$ is justified by a move occurring before $n'$. By the $6^{th}$ hypothesis, $u_{\prefixof \gamma}\filter Y$ is P-i.j.

There are six possible cases depending on
the type of the moves $n'$ and $m$:
$(n',m) \in \{ \omove_A \} \times \{\pmove_C,\opmove,\pomove \}
\union \{ \omove_C \} \times \{\pmove_A,\opmove,\pomove \} $).
The following diagram illustrates the cases $(n',m)
 = (\omove_A,\pmove_C)$ (left)
and  $(n',m)
 = (\omove_C,\pmove_A)$  (right):
$$
\begin{array}{ccccc}
A & \longrightarrow & B &
 \longrightarrow & C \\
&\vdots&&\vdots\\
&&&& \rnode{n}{\omove} \\
&\vdots&\rnode{gj}{\opmove}&\vdots\\
n' \omove \\
&\vdots&&\vdots  \\
&&\rnode{g}{\gamma} \pomove \\
&\vdots&&\vdots  \\
&&&&\rnode{m}{m} \pmove \\
\end{array}
\ncarc[arcangleA=30,arcangleB=30]{->}{m}{n}
\ncarc[arcangleA=30,arcangleB=30]{->}{g}{gj}
\hspace{2cm} \begin{array}{ccccc}
A & \longrightarrow & B & \longrightarrow & C \\
&\vdots&&\vdots\\
& \rnode{n}{\omove} \\
&\vdots& &\rnode{gj}\vdots\\
&&&&n' \omove \\
&\vdots&&\vdots  \\
&&\rnode{g}{\gamma} \opmove \\
&\vdots&&\vdots  \\
\rnode{m}{m} \pmove \\
\end{array}
\ncarc[arcangleA=30,arcangleB=30]{->}{m}{n}
\ncarc[arcangleA=30,arcangleB=30]{->}{g}{gj}
 $$

We have:
\begin{equation}
\ord_Y \gamma \geq \ord_X \gamma \label{eqn:gammaorderXY}
\end{equation}
Indeed, if $n' =\omove_C$ then $X=B,C$ and $Y=A,B$ and by Lemma
\ref{lem:compositionorder} we have $\ord_{A\lingamear B} \gamma \geq
\ord_{B\lingamear C} \gamma$. If $n=\omove_A$ then $\gamma$ is a
$\pomove$-move therefore it is not initial in $B$ and Lemma
\ref{lem:compositionorder} gives $\ord_{A\lingamear B} \gamma =
\ord_{B\lingamear C} \gamma$.

Hence:
\begin{align*}
\ord_{A\lingamear C} n'
& = \ord_{X} n' & \mbox{(n' non initial in $C$ \& Lemma \ref{lem:compositionorder})} \\
& \leq \ord_{X} \gamma & \mbox{($u_{\prefixof \gamma}\filter Y$ is P-i.j. \& $\gamma$'s justifier occurs before $n'$)} \\
& \leq \ord_{Y} \gamma & \mbox{(By Eq.\ \ref{eqn:gammaorderXY})} \\
& \leq \ord_{Y} m & \mbox{($u\filter X$ is P-i.j. \&
4$^{th}$ assumption: $m$'s justifier occurs before $\gamma$)}.
\end{align*}
\end{proof}


\begin{lemma}
\label{lem:visibleatprefixofu}
Let $u\in Int(A,B,C)$ such that
$u = \ldots \gamma \ldots \delta \ldots m$
where $m$ is a generalized P-move in $X$,
$\gamma \in \pview{u\filter A,C}$  and $\delta \in \pview{u\filter X}$. Then $\gamma \in \pview{u_{\prefixof \delta} \filter A,C}$.
\end{lemma}
\begin{proof}
First we remark than $\delta$ must occur in $\pview{u}$.
Indeed, $\delta \in \pview{u\filter X} = \pview{u_{< m} \filter X} \cdot m$ therefore $\delta \in \pview{u_{< m} \filter X}$ and since the move preceding $m$ in $u$ is necessarily a generalized O-move in $X$, we can use Lemma 3.3.1 from \cite{Harmer2005}:
\begin{align*}
\delta \in \pview{u_{< m} \filter X}
&= \pview{\pview{u_{<m}}\filter X} & \mbox{(Lemma 3.3.1 from \cite{Harmer2005})}\\
&\subseqof \pview{u_{<m}} \\
&\subseqof \pview{u} \ .
\end{align*}

Clearly, $\pview{u_{\prefixof \delta} \filter A,C}$ is a prefix of $\pview{u \filter A,C}$, indeed:
\begin{align*}
\pview{u_{\prefixof \delta} \filter A,C}
& = \pview{u_{\prefixof \delta}}\filter A,C
  & \mbox{(Lemma \ref{lem:pviewAC_eq_ACpview})}  \\
& \prefixof \pview{u}\filter A,C
  & \mbox{($\delta \in \pview{u}$)} \\
& = \pview{u\filter A,C}
  & \mbox{(Lemma \ref{lem:pviewAC_eq_ACpview})} \ .
\end{align*}

Finally since $\gamma \in \pview{u\filter A,C}$ and $\gamma$ occurs before $\delta$ in $u$, we necessarily have $\gamma \in \pview{u_{\prefixof \delta}\filter A,C}$.
\end{proof}

\begin{lemma}
\label{lem:compos_auxiliary_lemma}
Let $X$ be a component and $u \in Int(A,B,C)$ such that
the projection of $u$ on the component $X$ has the form:
$$ u \filter X =
\Pstr[0.5cm][2pt]{ \ldots (n){n}  \ldots
 {\stk {n'}{\omove}}  \ldots  (m-n,30){\stk m {\pmove}}
}$$
and
\begin{enumerate}
  \item $m$ and $n'$ are external move in $X$ ({\it i.e.}~in $A$ if $X =A,B$ and in $C$ if $X=B,C$);
  \item $u\filter X$ is P-i.j.;
  \item $u_{\prefixof b}\filter Y$ is P-i.j.\ for all non-initial B-move $b$ occurring in $u$.
\end{enumerate}
Then either $\ord_{A\lingamear C} n' \leq \ord_{A\lingamear C} m$ or
$n' \not \in \pview{u\filter A,C}$.
\end{lemma}
\begin{proof}

Suppose that $n'$ occurs in the P-view $\pview{u\filter X}$. Then we have
\begin{equation}
\ord_{A\lingamear C} n'  = \ord_{B\lingamear C} n' \ . \label{eqn:ordnp}
\end{equation}
Indeed, if $X$ is the component $B,C$ then necessarily $n'$ is not initial in $C$ (otherwise it would be the first move in $\pview{u \filter B,C}$, which is not the case since by visibility $n$ must occur before $n'$ in the P-view) and
if $X=A,B$ then $n'$ is in $A$. Thus in both cases, Lemma \ref{lem:compositionorder} gives us the claimed equality.

Hence we have
\begin{align*}
\ord_{A\lingamear C} n'
& = \ord_{X} n' & \mbox{(Eq.\
\ref{eqn:ordnp})} \\
& \leq \ord_{X} m & \mbox{($u\filter X$ is P-i.j.)} \\
& = \ord_{A\lingamear C} m & \mbox{(Lemma \ref{lem:compositionorder} \& $m$ is not initial in $C$)} \ .
\end{align*}

Suppose that $n'$ does not occur in the P-view $\pview{u \filter X}$, then $n'$ lies underneath a PO arc occurring in $\pview{u \filter X}$. Let us denote this arc by $\beta$-$\alpha$ where $\beta$ and $\alpha$ denote the arc's nodes. We have:
$$ u \filter X = \ldots
\Pstr[0.5cm]{
 (n){n} \ldots (b){\stk\beta \pmove} \ldots \stk{n'} {\omove}
\ldots (a-b){\stk\alpha \omove}  \ldots (m-n){\stk m {\pmove} }
} $$
with $\ord_X \alpha \leq \ord_X m$ (by P-i.j.\ of $u \filter X$).

\begin{enumerate}[i.]
\item Suppose $\alpha$ is an external move then so is $\beta$. Indeed, if $X=B,C$ and $\alpha = \omove_C$ then $\alpha$ can only point to another move in $C$ and
if $X=A,B$ and $\alpha = \omove_A$ then since $\alpha$ is an O-move in $A,B$, it is not initial in $A$ and therefore its justifier must also be in $A$.

Then instancing Lemma \ref{lem:interjump} with
$n \assignar n'$ gives us $n' \not\in\pview{u \filter A,C}$.

\item Suppose $\alpha$ is a $B$-move then necessarily so is $\beta$. Indeed, if $X=A,B$ then $\alpha \in B$
can only point to a move in $B$, and if $X=B,C$ then
since $\alpha$ is an O-move in the game $B,C$ it is not initial in $B$ and therefore its justfier must also be in $B$.

Now suppose that $n' \in \pview{u\filter A,C}$,
then by Lemma \ref{lem:visibleatprefixofu}
(with $\delta,\gamma \assignar \alpha,n'$)
we have $n' \in \pview{u_{\prefixof \alpha}\filter A,C}$,
and $u_{\prefixof \alpha}\filter Y$ is P-i.j.\ by hypothesis 3. This permits us to apply Lemma \ref{lem:increasing_order} on $u_{\prefixof \alpha}$:
\begin{align*}
\ord_{A\lingamear C} n'
& \leq \ord_{Y} \alpha & \mbox{(Lemma \ref{lem:increasing_order} with $u\assignar u_{\prefixof \alpha}$)} \\
& = \ord_{X} \alpha & \mbox{(Lemma \ref{lem:compositionorder} \& $\alpha$ non initial in $B$)} \\
& \leq \ord_{X} m & \mbox{($u \filter X$ is P-i.j.)} \\
& = \ord_{A\lingamear C} m & \mbox{(Lemma \ref{lem:compositionorder} \& $m$ is not initial in $C$)} \ .
\end{align*}
\end{enumerate}
\end{proof}


\begin{proposition}
\label{prop:closedpijcompose} Let $\sigma : A \lingamear B$ and $\mu
: B \lingamear C$ be two well-bracketed (P-visible) strategies then
\begin{enumerate}[(I)]
\item $\sigma$ closed P-i.j.\ $\wedge$ $\mu$ P-i.j.
$\implies$ $\sigma ; \mu$  P-i.j.;
\item $\sigma, \mu$ closed P-i.j.
$\implies$ $\sigma ; \mu$ closed P-i.j.
\end{enumerate}
\end{proposition}

\begin{proof}
Well-bracketing is preserved by strategy composition (see \cite[Proposition 2.5]{abramsky94full}) thus
$\sigma ; \mu$ is well-bracketed so we can use the definition of P-i.j.\ from Proposition \ref{prop:char_wellbrack}.

\noindent (I) Let us prove that $\sigma ; \mu$ is P-i.j..
Let $u$ be a play of the interaction $\sigma\ \|\ \mu$ between $\sigma$ and $\mu$
ending with an external P-move $m$
justified by $n$ in $\pview{u \filter A , C}$.
Let $n'$ be an external O-move occurring betweeen $n$ and $m$:
$$ u \filter A,C =
\Pstr[0.5cm][2pt]{ \ldots (n){\stk {n} \omove}  \ldots
 {\stk {n'} \omove}  \ldots  (m-n,30){\stk m \pmove}
}
$$
To show that $u \filter A,C$ is P-incrementally justified, we just
need to prove that either $n'\not\in \pview{u \filter A,C}$ or
$\ord_{A\lingamear C} n' \leq \ord_{A\lingamear C} m$. Note that if
$n'\in \pview{u \filter A,C}$ then necessarily $n'$ is not initial
in $C$ because $n$ occurs before $n'$ in $\pview{u \filter A,C}$.

Let $E$ denote one of the two external arenas ($A$ or $C$), $X$ be
the corresponding component ({\it i.e.}~$X=A,B$ if $E=A$ and $X=B,C$
if $E=C$) and $Y$ denote the other component.
    \begin{enumerate}[1)]
    \item Suppose $m$ and $n$ are two external moves in $E$.

        \begin{enumerate}[{1}.a)]
        \item Suppose $n' \in E$.

        This case corresponds to the situation handled by
        Lemma \ref{lem:compos_auxiliary_lemma}: we have
        either $\ord_{A\lingamear C} n' \leq
        \ord_{A\lingamear C} m$ or $n' \not\in \pview{u
        \filter A,C}$.

        \item Suppose $n' \not\in E$.

        Suppose that $n' \in \pview{u\filter A,C}$, then by
        Lemma \ref{lem:increasing_order} with $X\assignar Y$
        we have $ \ord_{A\lingamear C} n'  \leq \ord_X m$
        and since $m$ is not initial in $C$, Lemma
        \ref{lem:compositionorder} gives $\ord_X m =
        \ord_{A\lingamear C} m$, thus $\ord_{A\lingamear C}
        n' \leq \ord_{A\lingamear C} m$.
        \end{enumerate}

        \item \label{case:mA} Suppose $m \in A$ and $n \in C$.

        Then $m$ is an initial move in $A$
        pointing to a $\opmove$-move
        $b_0$ initial in $B$ which in turn points to the $\omove_C$-move $n$ initial in $C$.

        This situation cannot be handled similarly as the
        previous case. Indeed the pointer associated to the move
        $m$ in the game $A,C$ is not the same as the one
        attached to the corresponding move in the game $A,B$
        (see in \cite{Abr02} for the definition of the
        projection operation over the overall component A,C),
        hence we cannot use Lemma \ref{lem:increasing_order}
        since the condition requiring that $m$ points before
        $n'$ is not necessarily met. A more detailed analysis is
        therefore required.

        Let us assume that $n'\in \pview{u\filter A,C}$ and
        prove that we necessarily have $\ord_{A\lingamear C} n'
        \leq \ord_{A\lingamear C} m$. We do a case analysis:
        \begin{itemize}[-]
        \item Suppose $n'$ occurs before $b_0$.
        Note that we cannot apply Lemma \ref{lem:increasing_order} on $u$
        since $m$ does not point before $b_0$.
        Up to now we have only used the fact that $\sigma$ and $\mu$ are P-i.j. The assumption that $\sigma$ is  \emph{closed} P-i.j.\ now becomes crucial.

        Since $n' \in \pview{u\filter A,C}$ and
        $b_0 \in \pview{u\filter B,C}$, applying Lemma \ref{lem:visibleatprefixofu}
        with $X\assignar B,C$ and $\delta,\gamma \assignar b_0,n'$ gives
        $n' \in \pview{u_{\prefixof b_0}\filter A,C}$. This allows us to apply Lemma \ref{lem:increasing_order} on $u_{\prefixof b_0}$:
            \begin{align*}
            \ord_{A\lingamear C} m
            = \ord_A m
            & \geq \ord_B b_0 & \mbox{($u \filter A,B$ is closed P-i.j., $m$ is initial in $A$)} \\
            & = \ord_{B\lingamear C} b_0  \\
            & \geq \ord_{A\lingamear C} n' & \mbox{(Lemma \ref{lem:increasing_order} on $u_{\prefixof b_0}$ with $X\assignar A,B$)} \ .
            \end{align*}

        \item Suppose $n'$ occurs after $b_0$ (and necessarily before $m$).

            \begin{enumerate}[a.]
            \item Suppose $n'\in C$. Since $m$'s justifier occurs before $n'$ in $u$, we can use Lemma \ref{lem:increasing_order} which gives $\ord_{A\lingamear C} n' \leq \ord_{A\lingamear B} m
                = \ord_{A\lingamear C} m$.

            \item Suppose $n'\in A$.
By Lemma \ref{lem:compos_auxiliary_lemma} with $X
\assignar A,B$, since $n' \in \pview{u \filter A,C}$, we
have $\ord_{A\lingamear C} n' \leq \ord_{A\lingamear C}
m$.
\smallskip

        Note that we could not use Lemma
        \ref{lem:increasing_order} on $u$ directly since
        both $m$ and $n'$ are played in $A$. Also, in
        the ideal case where $n'$ is hereditarily
        enabled by the initial move $m$, we can
        immediately conclude $\ord_{A\lingamear C} n'
        \leq \ord_{A\lingamear C} m$; however this
        argument does not work in general: there may be
        more than one initial move in $A$ in which case
        $n'$ can be hereditarily enabled by an initial
        $A$-move distinct from $m$.
            \end{enumerate}
        \end{itemize}

    \end{enumerate}

\noindent (II) We now show that $\sigma;\mu$ is closed P-i.j.\
provided that both $\sigma$ and $\mu$ are. Take a play $s m \in
\sigma ; \mu$ such that $m$ is initial in $A$ and let $n$ be the
initial move of $C$ justifying $m$. Let $u \in \sigma \ \|\ \mu$ be
the uncovering of $s m$ ($s m = u \filter A,C$) and $b_0$ be the
initial $B$-move justifying $m$ in $u$.
 We have:
\begin{align*}
\ord_A m & \geq \ord_B b_0 & \mbox{($u \filter A,B \in \sigma$ is closed P-i.j.)} \\
 & \geq \ord_C n & \mbox{($u_{\prefixof b_0} \filter B,C \in \mu $ is closed P-i.j.)}.
\end{align*}
\end{proof}

{\it Remark:} The second part of the proposition only gives a
\emph{sufficient} condition for $\sigma ; \mu$ to be closed P-i.j.
In fact it is possible to have that $\sigma ; \mu$ is closed P-i.j.\
although $\mu$ is not.


\subsection{Tensor product}

 Given two strategies $\sigma :\ A
\lingamear B$  and $\tau :\ C\lingamear D$, their tensor product
$\sigma \otimes \tau :\ A\otimes B \lingamear C\otimes D$ is defined
as
$$\sigma \otimes \tau = \{ s \in L_{A\otimes C \lingamear B\otimes
D} \ | \ s \filter A,B \in \sigma \wedge s \filter C,D \in \tau \} $$
 where $A\otimes B$ denotes the tensor product of the games $A$ and $B$ (see \cite{abramsky:game-semantics-tutorial}).
\begin{proposition}
Let $\sigma :\ A \lingamear B$  and $\tau :\ C\lingamear D$.
\begin{enumerate}
\item If $\sigma$ and $\tau$ are P-i.j.\
then so is $\sigma \otimes \tau$;
\item If $\sigma$ and $\tau$ are closed P-i.j.\ then so is $\sigma \otimes \tau$.
\end{enumerate}
\end{proposition}

\begin{proof}
By establishing the state diagram of the game $A\otimes C \lingamear
B\otimes D$ one can show easily that only player O can switch
between the subgames $A\lingamear B$ and $C\lingamear D$.
Consequently, in the P-view of a play of the game $A\otimes C
\lingamear B\otimes D$, all the moves are played in the same subgame
({\it i.e.~} all in $A\lingamear B$ or all in $C\lingamear D$).
Hence if the last move of a play $m$ is played in $A\lingamear B$
then $\pview{s\filter A,B} = \pview{s} \filter A,B = \pview{s}$ (and
reciprocally if $m$ is played in $C\lingamear D$). The first part of
the proposition then follows immediately. The second part is also
straightforward.
\end{proof}


\subsection{Pairing} Given two strategies $\sigma :\ C \lingamear A$
and $\tau :\ C\lingamear B$, the pairing $\langle \sigma , \tau
\rangle :\ C \lingamear A\& B$ is defined as
\begin{align*}
\langle \sigma , \tau \rangle
    &= \{ s \in L_{C \lingamear A\& B} \ | \ s \filter C,A \in \sigma \wedge s \filter B = \epsilon \} \\
    & \union \{ s \in L_{C \lingamear A\& B} \ | \ s \filter C,B \in \tau \wedge s \filter A = \epsilon \}
\ .
\end{align*}
 where $A\& B$ denotes the product of the games $A$ and $B$ (see \cite{abramsky:game-semantics-tutorial}).

\begin{proposition}
\label{prop:pij_paring} Let $\sigma :\ C \lingamear A$  and $\tau :\
C\lingamear B$.
\begin{enumerate}
\item If $\sigma$ and $\tau$ are P-i.j.\ then so is $\langle \sigma , \tau \rangle$;
\item If $\sigma$ and $\tau$ are closed P-i.j.\ then so is $\langle \sigma , \tau \rangle$.
\end{enumerate}
\end{proposition}
The proof is immediate.


\subsection{Promotion} \emph{Notation:} Let $s$ be a play. We call
\defname{thread} a maximal subsequence of $s$ constituted of moves
that are hereditarily justified by the same occurrence of an initial
move. Let $m$ be a move occurring in $s$. We call thread of $m$ the
only thread in $s$ containing $m$.


We recall some definitions. Let $A$ and $B$ be two well-opened
games. Given a strategy  $\sigma :\ !A \lingamear B$, its promotion
$\sigma^\dag :\ !A\lingamear !B$ is defined as
$$ \sigma^\dag = \{ s \in L_{!A\lingamear !B}\ |\ \mbox{for all inital $m$ in $B$, } s\filter m \in \sigma \}$$
and for $\mu :\ !B\lingamear C$ the composite strategy $\sigma
\fatcompos \mu$ is defined as:
$$ \sigma \fatcompos \mu = \sigma^\dag ; \mu \ .$$

Since $B$ is well-opened, plays of $\sigma$ are constituted of a
single thread initiated by some initial $B$-move. Plays of
$\sigma^\dag$ however, are interleaves of potentially infinitely many single-threaded
plays of $\sigma$. One can show easily, using the visibility condition, that the thread of a $P$-move
is always the same as the thread of the preceding $O$-move. Consequently, the P-view of a play is equal to the P-view of the current thread:
if the current thread of a play $s$ is opened by an initial move $b \in B$ then
$\pview{s} = \pview{s \filter b} = \pview{s} \filter b$.


The state of the game is given by an infinite sequence of symbols in $\{O, P\}$, each element of the
sequence indicating who is to play in the corresponding thread.
The diagram on Figure \ref{fig:promotion_state_diagram} illustrates
how the state changes as a play of $\sigma^\dag$ unfolds.
The initial state of the game is $O^\omega$ - an infinite
sequence of O's -- which indicates that O is to play in all the
threads. When O plays an initial move in $B$, it ``opens'' a new
thread so the state of the game becomes $O^k P O^\omega$ where $k$
is the index of the thread being opened. By alternation, $P$ now has to play. His move must be played in a thread
already opened by $O$ and in which $P$ is to play; only one thread is in such state: the $k$th one. Hence after P's move
we are back to state $O^\omega$.

\tikzstyle{state}=[rectangle,draw=blue!50,fill=blue!20,thick,minimum
height = 4ex, text width=1.2cm, text centered]
\tikzstyle{state_nobg}=[thick,minimum
height = 4ex, text width=1.2cm, text centered]
\tikzstyle{omove}=[->,shorten <=1pt,>=latex',line width=0.5pt,bend left=10]
\tikzstyle{pmove}=[->,shorten <=1pt,>=latex',line width=0.5pt,bend left=10, draw=blue!50]
\begin{figure}[htbp]
\begin{center}
\begin{tikzpicture}[node distance=2cm]
% the states
\path
 node(init)  [state, text width=4cm] {$O^\omega$}
 (init)+(-2.8cm,-3cm)
 node(p)     [state, anchor=east,] {$PO^\omega$}
 node(p1)    [state, right of=p]  {$OPO^\omega$}
 node(p2)    [state_nobg, right of=p1] {\ldots}
 node(p3)    [state, right of=p2] {$O^kPO^\omega$}
 node(p4)    [state_nobg, right of=p3] {\ldots} ;
\path
% arrows representing the moves
  ([xshift=-1.4cm]init.south)  edge[omove] node[right]{O} ([xshift=0.2cm]p.north)
  (p.north)    edge[pmove] node[left]{P} ([xshift=-1.5cm]init.south)
  ([xshift=-0cm]init.south)   edge[omove] node[right]{O} ([xshift=0.2cm]p1.north)
  (p1.north)   edge[pmove] node[left]{P} ([xshift=-0.2cm]init.south)
  ([xshift=1cm]init.south)   edge[omove] node[right]{O} ([xshift=0cm]p3.north)
  ([xshift=-0.2cm]p3.north)   edge[pmove] node[left]{P} ([xshift=0.8cm]init.south);
\end{tikzpicture}
\end{center}
\caption{State diagram for plays of $\sigma^\dag$.}
\label{fig:promotion_state_diagram}
\end{figure}



\begin{proposition}
\label{prop:fatcompos_pij} If $A$ and $B$ are two well-opened games
and $\sigma :\ !A \lingamear B$ is a well-bracketed P-i.j.\ strategy
then $\sigma^\dag$ is also well-bracketed and P-i.j. Furthermore if
$\sigma$ is closed P-i.j.\ then so is $\sigma^\dagger$.
\end{proposition}
\begin{proof}
$\sigma^\dag$ is well-bracketed by \cite[Proposition
2.10.]{abramsky94full}. For P-incremental justification, the result is a direct consequence of the
fact that the P-view of a play in $\sigma^\dag$ is equal to the P-view of the current thread.
For closed P-incremental justification, the result is immediate.
\end{proof}

From propositions \ref{prop:closedpijcompose} and
\ref{prop:fatcompos_pij} we obtain:
\begin{corollary}
Let $A$ and $B$ be two well-opened games. Let $\sigma :\ !A
\lingamear B$ and $\mu :\ !B\lingamear C$ be two well-bracketed
strategies then:
\begin{enumerate}
\item If $\sigma$ is closed P-i.j.\ and $\mu$ is P-i.j.\ then $\sigma \fatcompos \mu :\ !A \lingamear
C$ is also P-i.j.;
\item If $\sigma$ and $\mu$ are closed P-i.j.\ then so is $\sigma \fatcompos \mu :\ !A \lingamear C$.
\end{enumerate}
\end{corollary}

\subsection{The category $\mathcal{G}_{Pij}$ of closed P-i.j.\ strategies}

We define the category $\mathcal{G}_{Pij}$ as follows:
\begin{itemize}
\item the objects are games (as defined in \cite{abramsky:game-semantics-tutorial}),
\item the morphisms from $A$ to $B$ are the closed P-incrementally-justified strategies
on the game $A\rightarrow B$,
\item morphisms are composed using the standard game-semantic strategy composition.
\end{itemize}
This indeed defines a category. Indeed we have shown in the previous
section that closed P-i.j.\ compose, strategy composition is
associative (\cite{abramsky94full,hylandong_pcf}) and finally the
identity strategy $id_A$ for any game $A$ is closed P-i.j.

We mentioned before that such category cannot be cartesian closed. Indeed, remember that
a P-i.j.\ strategy from $A$ to $B$ is said to be \emph{closed} P-i.j.\ provided that some condition
on the arena $A\rightarrow B$ holds. This condition refers precisely to the structure of the arenas $A$ and $B$ and consequently relies on the fact the arena considered is exactly $A\rightarrow B$ (and not any other isomorphic arena).



\section{Modeling the Safe Lambda Calculus in $\mathcal{G}_{Pij}^{inn}$}

Consider the category $\mathcal{G}_{Pij}$ defined in section
\ref{sec:closedpij}. In this section we show how the safe lambda
calculus can be modeled in the sub-category
$\mathcal{G}_{Pij}^{inn}$ of innocent P-ij strategies.

\subsection{The language}
We recall the definition of the safe simply-typed lambda calculus
\cite{blumong:safelambdacalculus}.  We use sequents of the form
$\Gamma \vdash_s M : A$ to represent terms-in-context where $\Gamma$
is the context and $A$ is the type of $M$. For simplicity we write
$(A_1, \cdots, A_n, B)$ to mean $A_1 \typear \cdots \typear A_n
\typear B$, where $B$ is not necessarily ground.

\begin{definition}\rm
The \defname{safe lambda calculus}, or Safe $\Lambda^{\rightarrow}$ for short, is a sub-system of the
  simply-typed lambda calculus defined by induction over the
  following rules:
$$ \rulename{var} \ \rulef{}{x : A\vdash_s x : A} \quad
\rulename{wk} \ \rulef{\Gamma \vdash_s s : A}{\Delta \vdash_s s : A} \quad
\Gamma \subset \Delta$$
$$ \rulename{app} \ \rulef{\Gamma \vdash_s s : (A_1,\ldots,A_n,B) \
  \Gamma \vdash_s t_1 : A_1 \; \ldots \; \Gamma \vdash_s t_n : A_n
} {\Gamma \vdash_s s t_1 \ldots t_n : B} \ \min_{y:Y \in \Gamma} \ord Y \geq \ord B$$
$$ \rulename{abs} \ \rulef{\Gamma, x_1 : A_1, \ldots, x_n : A_n
  \vdash_s s : B} {\Gamma \vdash_s \lambda x_1 \ldots x_n . s :
  (A_1, \ldots ,A_n,B)} \ \min_{y:Y \in \Gamma} \ord Y \geq \ord (A_1, \ldots ,A_n,B)$$
%  where $\ord{\Gamma}$ denotes the set $\{ \ord{y} : y
%\in \Gamma \}$ and ``$c \sqsubseteq S$'' means that $c$ is a
%lower-bound of the set $S$.
\end{definition}


\subsection{Game-semantic denotation}

In \cite{blumong:safelambdacalculus} we showed that in the game
semantic model safe lambda terms are denoted by P-i.j.\ strategies.
The argument was syntactic: it is based on the analysis of a special
kind of abstract syntax tree of a term called computation tree
\cite{OngLics2006}. Here we give another proof based on a semantic
argument that uses the results of section \ref{sec:closedpij}.


\begin{proposition}
\label{prop:safe_closepij_sem}
  Safe simply-typed terms are denoted by closed P-i.j.\ strategies.
\end{proposition}
\begin{proof}
  By induction on the formation rules.
  \begin{enumerate}
    \item (var) $\sem{x:A \vdash_s x:A } = id_A$. Clearly the
    identity strategy is closed P-i.j.

    \item (wk) Take $\Gamma \subset \Delta $ and suppose $\sem{\Gamma \vdash_s
    s : A}$ is closed P-i.j. Up to an appropriate retagging of
    the moves the two strategies $\sem{\Delta \vdash_s s : A}$
    and $\sem{\Gamma \vdash_s s : A}$ are isomorphic. Hence
    $\sem{\Delta \vdash_s s : A}$ is P-i.j. It is also closed
    P-i.j.\ since none of the new initial moves introduced by
    $\Delta$ occurs in any play of the strategy.

    \item (app) Suppose that $\sem{\Gamma \vdash_s s :
    (A_1,\ldots,A_n,B)}$ and $\sem{\Gamma \vdash_s t_i : A_i}$
    for $i \in \{1..n\}$ are closed P-i.j.\ and $\ord{B}
    \sqsubseteq \ord{\Gamma}$. We have $\sem{ \Gamma \vdash_s s
    t_1 \ldots t_n : B} = \langle \sem{\Gamma \vdash s},
    \sem{\Gamma \vdash t_1}, \ldots, \sem{\Gamma \vdash t_n}
    \rangle \fatsemi ev_n$ where $ev_n$ is the $n$-parameter
    evaluation strategy. By Proposition \ref{prop:pij_paring},
    $\langle \sem{\Gamma \vdash s}, \sem{\Gamma \vdash t_1} ,
    \ldots, \sem{\Gamma \vdash t_n} \rangle$ is closed P-i.j.
    The evaluation map $ev_n$ is P-i.j.\ (but not necessarily
    closed P-i.j.) therefore by Proposition
    \ref{prop:closedpijcompose} I. $\sem{ \Gamma \vdash_s s t_1
    \ldots t_n : B}$ is P-i.j. The arena of the game
    $\sem{\Gamma}$ is of type $\sem{Y_1} \times \ldots \times
    \sem{Y_n}$ where the $Y_i$s are the types of the variables
    in the context $\Gamma$. The $\sem{Y_i}$s are all prime
    since we work with pure simple types without product.
    Moreover the side-condition of the rule gives $\ord Y_i \geq
    \ord B$ for all $i \in \{1..n\}$. Hence by Lemma
    \ref{lem:closedpij_singleBinitmove}(ii), $\sem{ \Gamma
    \vdash_s s t_1 \ldots t_n : B}$ is closed P-i.j.

    \item (abs) Suppose that $\sem{\Gamma, x_1 : A_1, \ldots, x_n : A_n \vdash_s
    s : B}$ is closed P-i.j. Then the isomorphic strategy $\sigma = \sem{\Gamma \vdash_s \lambda
    x_1 \ldots x_n . s : (A_1,\ldots,A_n,B)}$ is also P-i.j.
    Again, using the side-condition, Lemma \ref{lem:closedpij_singleBinitmove}(ii)
    implies that $\sigma$ is closed P-i.j.
  \end{enumerate}
\end{proof}

\subsection{The safe lambda calculus with product ($\Lambda^{\rightarrow}_\times$)}
We will now show how product types and pairing can be added to the safe lambda calculus.
This can be done trivially as follows. Types are now given by the following grammar:
\begin{align*}
T ::=& \ B,  \quad \mbox{for some base type $B$} \\
   &| \ T \rightarrow T \\
   &| \ T \times T
\end{align*}
and the typing system is extended with the following three rules:
$$ \rulename{\times} \ \rulef{\Gamma \vdash_s s : A \qquad \Gamma \vdash_s t : B}
{\Gamma \vdash_s \langle s, t \rangle : A \times B}
\qquad \rulename{\pi_1} \ \rulef{\Gamma \vdash_s s : A \times B}
{\Gamma \vdash_s \pi_1 s : A} \qquad
 \rulename{\pi_2} \ \rulef{\Gamma \vdash_s s : A \times B}
{\Gamma \vdash_s \pi_2 s : B}$$

One can easily check that most of the good properties of the safe
lambda calculus remains in this extended calculus: the free
variables of a term have order greater than the order of the term
itself; the no-variable-renaming Lemma holds and terms are denoted
by P-i.j.\ strategies. However in general, terms are not denoted by
\emph{closed} P-i.j.\ strategies. Indeed, in the safe lambda calculus
without product - as defined in the previous section - all the
arenas involved are prime {\it i.e.}~they have a single initial
move. In the present case however, since types can be constructed
using the cartesian product, the corresponding arenas can have many
initial moves. Consequently Lemma
\ref{lem:closedpij_singleBinitmove}(ii) cannot be used anymore! Here
is a counter-example: Take the term $x : (o^1\rightarrow o^2)\times
o^3 \vdash \lambda y^o . \pi_2 x : o^4 \rightarrow o^5$ denoted by
some P-i.j.\ strategy $\sigma$ containing the play $q^5 q^3$. We have
$\ord_{(o^1\rightarrow o^2)\times o^3} q^3 = 0 < 1 = \ord_{o^4
\rightarrow o^5} q^5$ therefore $\sigma$ is not closed P-i.j.


There are two different approaches to overcome this problem. The
first one consists in restricting the types of the variables
appearing in the context of a term. More precisely we require that
whenever a variable has product type $A \times B$ we have $\ord A =
\ord B$. One can easily check that this rules out the problem
underlined in the previous counter-example and that it guarantees
that terms are then indeed denoted by closed P-i.j.\ strategies.

The other approach is less restrictive but requires us to modify
slightly the side-conditions of the application and abstraction
rules: instead of requiring that all variables in the context have
order greater than the order of the term, we require that the order
of \emph{any prime sub-type of any variable} in the context has
order greater that the order of the term. The set $Pr(A)$ of prime
sub-types of a type $A$ being defined as follows:
\begin{align*}
Pr(B) &= \{ B \} \qquad \mbox{ for some base type } B \\
Pr(A\rightarrow B) &= \{ A\rightarrow B \} \\
Pr(A\times B) &= Pr(A) \union Pr(B)
\end{align*}

This gives rise the following calculus:
\begin{definition}\rm
The \defname{safe lambda calculus with product}, or Safe
$\Lambda^{\rightarrow}_\times$ for short, is given by induction over
the following rules:
$$ \rulename{var} \ \rulef{}{x : A\vdash_s x : A} \quad
\rulename{wk} \ \rulef{\Gamma \vdash_s s : A}{\Delta \vdash_s s : A} \quad
\Gamma \subset \Delta$$
$$ \rulename{\times} \ \rulef{\Gamma \vdash_s s : A \qquad \Gamma \vdash_s t : B}
{\Gamma \vdash_s \langle s, t \rangle : A \times B}
\qquad \rulename{\pi_1} \ \rulef{\Gamma \vdash_s s : A \times B}
{\Gamma \vdash_s \pi_1 s : A} \qquad
 \rulename{\pi_2} \ \rulef{\Gamma \vdash_s s : A \times B}
{\Gamma \vdash_s \pi_2 s : B}$$
$$ \rulename{app} \ \rulef{\Gamma \vdash_s s : (A_1,\ldots,A_n,B) \
  \Gamma \vdash_s t_1 : A_1 \; \ldots \; \Gamma \vdash_s t_n : A_n
} {\Gamma \vdash_s s t_1 \ldots t_n : B} \ C(\Gamma ; B)$$
$$ \rulename{abs} \ \rulef{\Gamma, x_1 : A_1, \ldots, x_n : A_n
  \vdash_s s : B} {\Gamma \vdash_s \lambda x_1 \ldots x_n . s :
  (A_1, \ldots ,A_n,B)} \ C(\Gamma ; (A_1, \ldots ,A_n,B) )$$

where the side-condition $C(\Gamma ; B)$ expresses that $\forall y:Y
\in \Gamma. \forall Y' \in Pr(Y) . \ord Y' \geq \ord B$.
\end{definition}

One can show by induction on the rules that the game denotations of
terms of this calculus are closed P-i.j.\ (the argument is similar to
the one used in the proof of Proposition
\ref{prop:safe_closepij_sem}). However this syntax does not
completely capture all the closed P-i.j.\ strategies. Take for
instance the simply-typed term $x:(o\rightarrow o)\times o \vdash \lambda z^o. \
(\pi_1 x) : o \rightarrow (o \rightarrow o)$; Its denotation is
closed P-i.j.\ but it is not typable in
$\Lambda^{\rightarrow}_\times$.


\section{Modeling Safe PCF in $\mathcal{G}_{Pij}^{inn}$}

\subsection{Safe PCF}

\notetoself{Insert here the definition of Safe PCF from transfer
thesis or refer to a definition given in a previous chapter.}

\subsection{Game-semantic denotation (semantic argument)}

\begin{proposition}
\label{prop:safepcf_closedpij} Safe PCF terms are denoted by closed
P-incrementally justified strategies.
\end{proposition}
\begin{proof}
We first prove the result for $\pcf_1$ - the fragment of \pcf\
containing terms of the form $\Omega_A = Y (\lambda x : A.x)$ but
where no other use of Y is allowed (see
\cite{abramsky:game-semantics-tutorial}). The proof is by structural
induction over the structure of the term.
\begin{itemize}
\item The strategy $\sem{\Omega_A} = \bot$ is
clearly closed P-i.j.

\item The functional rules are treated the same way as in the
corresponding proof for the safe lambda calculus.

\item For the arithmetic rules, we observe that the strategies
$succ$, $pred$ and $cond$ are all closed P-i.j. The fact that
pairing and strategy composition preserve closed P-incremental
justification permits us to conclude.
\end{itemize}

We now lift the result to full PCF using the technique of
\emph{syntactic approximant} (see
\cite{abramsky:game-semantics-tutorial}). By \cite[lemma
16]{abramsky:game-semantics-tutorial} we have
$$ \sem{M} = \Union_{n\in\omega} \sem{M_n}$$
where $M_n$ is the $\pcf_1$ term obtained from $M$ by replacing each
subterm of the form $Y N$ with $Y^n N_n$, and $Y^n F$ denotes the
$n$th approximant of $Y F$. Since the $M_n$s are $\pcf_1$ terms, by
the previous result each $\sem{M_n}$ is closed P-i.j.\ and since
closed P-incremental justification is clearly a continuous property,
$\sem{M}$ is also closed P-i.j.
\end{proof}


\subsection{Full abstraction}

\subsubsection{O-incremental justification}
 
\defname{O-incremental justification} is the counterpart of P-incremental justification ({\it i.e.}~the role of O and P is exchanged in the definition).

O-incremental justification relates to P-incremental justification very much like O-visibility relates to P-visibility
(see \cite[Sec.~3.6]{Harmer2005}).

Let $\sigma : A$ and $\mu : A \rightarrow o$ be two strategies and $q$ be the initial move of the game $A \rightarrow o$. Then P-views of plays in $A$ correspond to O-views
in the game $A \rightarrow o$. Indeed, for $s\in L_A$ we have $q s \in L_{A \rightarrow o}$ and
due to alternation, $q \pview{s}^A = \oview{q s }_{A \rightarrow o}$.

Consequently, if $\sigma$ is P-i.j.\ then the play involved in the interaction between $\sigma$ and $\mu$
are all O-i.j.\ from $\mu$'s perspective. Indeed, let $u \in \sigma \| \mu$ with $|u|\geq1$. Then $u=q v$
and $u\filter A = v \filter A$ is P-i.j. By the previous remark, this implies that $q (v\filter A) = (q v)\filter (A \rightarrow o) = u \filter (A \rightarrow o)$ is O-i.j.
\smallskip

Now if we regard $\sigma$ as the denotation of some closed term $\vdash M:A$ and $\mu$ as the
denotation of some context $x:A \vdash C[x]:o$ then what the previous remark says is that
non O-i.j.\ plays are useless for the purpose of studying observational equivalence!
This suggests that it is not necessary to include non O-i.j.\ plays in the game denotation of safe terms. However before removing completely those plays from the game model, we have to ensure that this does not prevent us from constructing a category:
\begin{lemma}
\label{lem:oij_decomp}
Let $\sigma : A\rightarrow B$ and $\tau : A\rightarrow B$ be closed P-i.j.\ strategies and suppose
that $u\in \sigma \| \tau$ such that for all external O-moves $o$ of $u$, we have that $u_{\prefixof o} \filter A,C$ satisfies
O-incremental justification. Then, for any generalized O-move $m$ of $u$ in component $X$, we have that
$u_{\prefixof m} \filter X$ satisfies O-incremental justification
\end{lemma}

This lemma states that O-i.j plays cannot be obtained from the interaction of plays that are not O-i.j. In other words, if we write $\mathcal{O}(\sigma)$ for the set of O-i.j.\ plays of $\sigma$, then the Lemma can be restated equivalently as:
\begin{eqnarray}
     \forall \sigma, \tau\ \mbox{closed P-i.j.}: \mathcal{O}(\sigma) ; \mathcal{O}(\tau) \supseteq \mathcal{O}(\sigma ; \tau)
     \label{eqn:oijdecomp_1}
\end{eqnarray}
which in turn is equivalent to
\begin{eqnarray}
    \forall \sigma, \tau\ \mbox{closed P-i.j.}: \mathcal{O}( \mathcal{O}(\sigma) ; \mathcal{O}(\tau) ) = \mathcal{O}(\sigma ; \tau)
    \label{eqn:oijdecomp_2}
\end{eqnarray}
Indeed, Eq.~\ref{eqn:oijdecomp_1} implies the right-to-left inclusion and the other inclusion
is given by the fact that $\mathcal{O}(\sigma) ; \mathcal{O}(\tau) \subseteq \sigma;\tau$.


In some sense, Lemma \ref{lem:oij_decomp} is the dual of the proposition stating that closed P-i.j.\
strategies compose, since  the latter can be reexpressed more succinctly with the relation:
\begin{eqnarray}
     \forall \sigma, \tau .\, \mathcal{P}(\sigma) ; \mathcal{P}(\tau) \subseteq \mathcal{P}(\sigma ; \tau)
     \label{eqn:pijcomp_1}
\end{eqnarray}
where $\mathcal{P}(\sigma)$ is define as be the largest even-length-prefix-closed subset of $\sigma$ consisting of closed P-i.j.\ plays.

\subsubsection{A category of incremental strategies}

\notetoself{
-Incremental strategies means O-i.j.\ and closed P-i.j.

}

\subsubsection{Full abstraction}

The fully-abstract game-model of PCF is also fully-abstract for the
safe fragment of PCF when observational equivalence is defined with
respect to unrestricted ({\it i.e.}~possibly unsafe) PCF contexts.
However one may ask what is a fully abstract model of Safe PCF with
respect to \emph{safe} contexts.




\notetoself{
By the definability results for Safe PCF, it should be possible to
prove that the category $\mathcal{C}^{inn}_{OP-incr}$ of
OP-incrementally justified and innocent strategies is fully abstract
for Safe PCF. We can use the same proof as in the PCF case: we have
a compact test strategy $\alpha:A\rightarrow N$ and by definability,
there must be some context $x:A \vdash C[x] : N$ such that $\sem{x:A
\vdash C[x] : N} = \alpha$. The definability result for Safe PCF
gives us that $\lambda x . C[x] : A \rightarrow N$ is safe which in
turns implies that $x:A \vdash C[x] : N$ is safe since $\ord{N} =
0$.
}

\subsubsection{Algorithmic game semantics}
We recall that Strongly Safe IA $\subseteq$ Safe IA $\subseteq$ IA.
Up to order $3$, it is conservative, with respect to observational equivalence, to add unsafe context to safe ones.
At order $4$, it is not conservative anymore.

\paragraph{Observational equivalence}
\begin{table}
\begin{tabular}{|c|c||c|c|c|c|c|}
    \cline{3-7}
  \multicolumn{2}{c|}{}  & \multicolumn{5}{c|}{Finitary fragments} \\ \hline
  \multirow{2}{*}{$L$} & \multirow{2}{*}{$C[\_]$} &   order 2          &  order 2       & order 3     & order 3 & \multirow{2}{*}{order 4}  \\
                       &                          &    + while         &   + $Y_1$      & + while     & +$Y_0$  &          \\ \hline \hline

  \multirow{4}{*}{IA}  & \multirow{2}{*}{IA}      & \multirow{4}{2cm}{PSPACE$^{(1)}$ \\ {\small $\preccurlyeq$ DFA}} & \multirow{4}{*}{U$^{(2)}$} & \multirow{4}{2.8cm}{EXP-complete$^{(3)}$ \\ {\small $\preccurlyeq$ VPA} }  & \multirow{4}{2cm}{D$^{(4)}$ \\ {\small $\preccurlyeq_{exp}$ DPDA\\ $\succcurlyeq$ DPDA} } & \multirow{2}{*}{U$^{(5)}$}\\
                       &                          &                    &                    &  & & \\
\cline{2-2}\cline{7-7} & \multirow{2}{*}{Safe IA} &                    &                    &  & & \multirow{2}{*}{?} \\
                       &                          &                    &                    &  & & \\ \hline

  \multirow{4}{*}{Safe IA} & \multirow{2}{*}{IA}      & \multirow{4}{2cm}{PSPACE \\ {\small $\preccurlyeq$ DFA}} & \multirow{4}{*}{U} & \multirow{4}{2.3cm}{EXP-complete \\ {\small $\preccurlyeq$ VPA}} & \multirow{4}{2cm}{D \\ {\small $\preccurlyeq_{exp}$ DPDA\\ $\succcurlyeq$ DPDA} } & \multirow{2}{*}{U} \\
                           &                          &                    &                & & & \\
\cline{2-2}\cline{7-7}     & \multirow{2}{*}{Safe IA} &                    &                & & & \multirow{2}{*}{?} \\
                           &                          &                    &                & & & \\ \hline

  \multirow{4}{*}{St. Safe IA} & \multirow{2}{*}{IA}           & \multirow{4}{*}{D} & \multirow{4}{*}{?} & \multirow{4}{*}{D} & \multirow{4}{*}{D} & \multirow{2}{*}{?} \\
                               &                               &                    &                    &                    &                    & \\
\cline{2-2} \cline{7-7}        &  \multirow{2}{*}{St. Safe IA} &                    &                    &                    &                    & \multirow{2}{*}{?} \\
                               &                               &                    &                    &                    &                    & \\ \hline
\end{tabular}
\caption{Decidability (and complexity) of observational equivalence for some finitary fragments of IA}

U stands for Undecidable and D stands for decidable with unknown complexity, $\preccurlyeq P$ means ``reducible to problem $P$''
and $\succcurlyeq P$ means ``at least as hard as problem $P$''.
\begin{asparaenum}
\item[1.] See \cite{ghicamccusker00}.
\item[2.] Showed by Ong in \cite{OngLics2006}.
\item[3.] See \cite{DBLP:conf/fossacs/MurawskiW05}.
\item[4.] See \cite{DBLP:conf/icalp/MurawskiOW05}.
\item[5.] By encoding of $\Sigma$-machine (turing complete) into IA$_4$, see \cite{murawski03program}.

\end{asparaenum}
\end{table}

\paragraph{Observational approximation}

Observational approximation has been shown to be undecidable at order $1$ already, for the fragment $IA_1 + Y_0$ (\cite{DBLP:conf/fossacs/MurawskiW05}).


\notetoself{
- Characterization of the set of complete plays for Safe IA.
(Easy adaptation of the corresponding result for IA. In the present case however, the proof relies
on the fact that plays of the strategy are O-i.j. (in order for $\alpha$ to be P-i.j.)
}


\subsection{What is a model of Safe PCF/Safe IA?}

\notetoself{
- Define the notion of incremental category.

- Show that any incremental category is a model of Safe PCF and that any model of Safe PCF
is an incremental category.

- Show that the category of games and OP-i.j. strategies is an incremental category.

}



\section{Modeling Safe IA in $\mathcal{G}_{Pij}$}
\notetoself{
- I need to merge this section with the other note on Safe IA.

- $\iavar =  \iacom^{\omega}\times \iaexp$

- Any strategy on the game $I \lingamear\ !\iavar$ is P-i.j.\ (and
thus closed P-i.j.) since there is no P-question in the arena
\iavar. Hence the strategy $cell$ is P-i.j. }

\subsubsection{Game-semantic denotation}

In this section, our aim is to extend the game-semantic characterization of safety to Safe IA.

We first observe that the result extends extends trivially from Safe PCF to Strongly Safe IA:
\begin{proposition}
  Strongly Safe IA terms are denoted by closed P-i.j. strategies.
\end{proposition}
\begin{proof}
The proof is an adaptation of the proof for Safe PCF. We first show that the result holds for the
fragment of Strongly Safe IA in which the only allowed uses of $Y$ are in terms of the form $\Omega$.
This is done by induction over the structure of the term:
The functional rules and the arithmetic rules are treated
the same way as in the proof for Safe PCF. For the imperative rules, we
observe that the strategies $assign$, $deref$, $mkvar$, $seq$ and
$cell$ are all closed P-i.j. The fact that pairing, tensor product
and strategy composition all preserve closed P-incremental
justification permits us to conclude.

The result is then lifted to the whole of Strongly Safe IA using syntactic approximants as in the PCF case.
\end{proof}

Now we would like to extend this result to full Safe IA:
\begin{proposition}
\label{prop:safeia_closedpij} Safe IA terms are denoted by closed
P-incrementally justified strategies.
\end{proposition}

This happens to be less trivial than for the previous restricted fragment of Safe IA. We first introduce some definitions:

\begin{definition}[P-i.j. modulo $\mathfrak{M}$]
\label{def:pij_modulo} Let $\sigma$ be a strategy on some game $A$
and $\mathfrak{M}$ be a set of moves. We say that $\sigma$ is P-i.j.
modulo $\mathfrak{M}$ iff for all $s m \in \sigma$ with $m \not\in
\mathfrak{M}$, the play $s m$ is P-i.j.

Similarly we say that $\sigma$ is \emph{closed} P-i.j. modulo
$\mathfrak{M}$ iff for all $s m \in \sigma$ with $m \not\in
\mathfrak{M}$ the play $s m$ is \emph{closed} P-i.j.

Hence a strategy is P-i.j. if and only if it is P-i.j. modulo
$\emptyset$.
\end{definition}

Given a term $\Gamma | \Gamma^{\ianew} \safeentail M : A$, we write
$\sem{\Gamma | \Gamma^{\ianew} \safeentail M : A}$ to denote the
game denotation of the corresponding IA term {\it i.e.}
$\sem{\Gamma, \Gamma^{\ianew} \vdash M : A}$. Instead of showing
Proposition \ref{prop:safeia_closedpij} we will prove the following
more general result:
\begin{proposition}
\label{prop:safeia_closedpijmodulo} Let $\Gamma | \Gamma^{\ianew}
\safeentail M : A $ be a Safe IA term. Its denotation $\sem{\Gamma |
\Gamma^{\ianew} \safeentail M : A}$ is closed P-i.j. modulo
$\mathfrak{M}_{\Gamma^{\ianew}}$ where
$\mathfrak{M}_{\Gamma^{\ianew}}$ is the set of initial moves in
$\Gamma^{\ianew}$.
\end{proposition}

\begin{remark}
Since the context $\Gamma^{\ianew}$ contains variable of type
\iavar\ only, $\mathfrak{M}_{\Gamma^{\ianew}}$ contains only moves
of the form `$read$' or `$write_i$' for some $i\in \nat$.
\end{remark}

\begin{lemma}
\label{lem:leftcompos_preserv_pijmodulo}
 Let $\sigma : A \rightarrow
B$ and $\mu : B \rightarrow C$.
  Let $\mathfrak{M}$ be any set of moves initial in $A$.
  If $\sigma$ is closed  P-i.j. modulo $\mathfrak{M}$ and $\mu$ is
  P-i.j. (resp. closed P-i.j.) then $\sigma \fatsemi \mu$ is P-i.j. (resp. closed P-i.j.) modulo $\mathfrak{M}$.
\end{lemma}
\notetoself{
\begin{proof}
Let us analyze the proof of compositionality for closed P-i.j.
strategies.

\end{proof}
}


\begin{lemma}
\label{lem:cellcomposition_preserve_pijmodulo} Let $\tau : I
\rightarrow C_2$, $\sigma : C_1 \otimes C_2 \rightarrow B$  and
$\mathfrak{M}$ be any set of moves initial in $C_1 \otimes
C_2$.
  If $\tau$ is P-i.j. and
  $\sigma$ is P-i.j. (resp. closed P-i.j.) modulo $\mathfrak{M}$
  then $(id_{C_1} \otimes \tau) \fatsemi \sigma$ is P-i.j. (resp. closed P-i.j.) modulo $\mathfrak{M} \inter C_1$.
\end{lemma}
\notetoself{
\begin{proof}
\end{proof}
}


\notetoself{
\begin{proof}[Proof of Prop.\ \ref{prop:safeia_closedpijmodulo}]
We first prove the result for the fragment of Safe IA where the only allowed uses of the $Y$ combinator is in terms of the form $\Omega$. By induction and and case analysis on the structure of Safe IA terms:
\begin{itemize}
  \item[$\rulename{var}$, $\rulename{var^\ianew}$
  , $\rulename{wk}$, $\rulename{wk^\ianew}$]

  These cases are treated the same way as in the corresponding
  proof for the Safe Lambda Calculus.

  \item[$\rulename{app}$]

  \item[$\rulename{abs}$]


  \item[$\rulename{const}$]
  \item[$\rulename{succ}$, $\rulename{pred}$,$\rulename{cond}$]

  \item[$\rulename{seq}$]
  \item[$\rulename{assign}$]
  \item[$\rulename{deref}$]
  \item[$\rulename{new}$] $\Gamma | \Gamma^\ianew \safeentail \ianewin{x}\ M : B$.

Let $cell : I \rightarrow !\iavar$ denotes the ``storage cell''
strategy (see \cite{abramsky:game-semantics-tutorial}).  Let
$\sigma = \sem{\Gamma | \Gamma^\ianew, x : \iavar \safeentail M
: B}$.  We have $\sem{\Gamma | \Gamma^\ianew \safeentail
\ianewin{x}\  M : B} = (id_{\Gamma,\Gamma^\iavar} \otimes cell)
\fatcompos \sigma$. By induction hypothesis $\sigma$ is closed
P-i.j. modulo $\mathfrak{M}_{\Gamma^{\ianew} \otimes !\iavar}$
and one can easily check that $cell$ is P-incrementally
justified. Instancing Lemma
\ref{lem:cellcomposition_preserve_pijmodulo} with $\tau
\leftarrow cell$, $C_1 \leftarrow \Gamma \otimes  \Gamma^\ianew$
and $C_2\leftarrow !\iavar$ gives us the desired result.
  \item[$\rulename{mkvar}$]

\end{itemize}
The result is then lifted to the whole of Safe IA using the technique of syntactic approximants and using the fact that the ``closed P-i.j.'' property is continuous.
\end{proof}
}


\subsection{Algorithmic game semantics}

There is an important theorem (\cite{AM97a}) in game semantics
which states that two IA terms are equivalent if and only if the set
of complete plays of their game denotations are equal. This result was used in \cite{ghicamccusker00} to show that observational
equivalence for the $IA_2$ fragment of IA is decidable -- the set of
complete plays being representable by regular expressions. In
\cite{Ong02} it was shown that it is still decidable
 for $IA_3+Y_0$. Indeed, for this fragment, the set of complete plays becomes context-free
therefore the problem reduces to the DPDA equivalence problem which
is itself decidable (with an unknown complexity).

Imposing the safety condition should lead to some improvement in
complexity. The complexity of  Safe $IA_3$ (resp. Safe $IA'_3$) for
instance, must be lower than the complexity of the DPDA equivalence
problem. Moreover the fact that Safe $IA_3$ (resp. Safe $IA'_3$)
contains terms whose denotation is context free -- e.g. $\lambda f .
f (\lambda x .x )$ -- strongly suggests that its complexity is
strictly higher than the complexity of regular language equivalence.

Murawski \cite{Murawski2003} has shown that observational
equivalence for $\ialgol_4$ is undecidable. The proofs proceeds by
showing that the computations of $\Gamma$-machine -- some variation
of queue machines that are Turing complete -- are representable
using IA terms.

Does this result extend to the safe fragments? For Safe IA, it does,
simply because the $IA_4$ term exhibited in \cite{Murawski2003} to
represent computations of the $\Gamma$-machine is also a Safe IA
term. The same argument does not carry over to Very Safe IA since
the term is not typable in this language. We do not know whether
observational equivalence is decidable for Strongly Safe $IA_4$.



\subsection{Expressivity of Safe IA/Strongly Safe IA}

Murawski representability : Safe IA representable languages are
exactly the context free languages. For Strongly Safe IA however, we
believe that the representable languages are a proper subclass of
the context free languages.













\section{Remarks}
\subsection{Homogeneity constraint}

Type homogeneity is not preserved after composition. Indeed the
types  $o \typear (o \typear o)$ and $(o \typear o) \typear \left((o
\typear o) \typear o \right)$ are homogeneous but $o \typear
\left((o \typear o) \typear o\right)$ is not.

If $A\typear B$ and $B \typear C$ are homogeneous types then  a
sufficient condition for $A\typear C$ to be homogeneous is
``$\ord{A} \geq \ord{B}$''.




\chapter{Conclusion}
    \label{chap:conclusion}
    \chapter{Further possible developments}

In the previous chapter, we have given an account of the game
semantics of Safe $\lambda$-Calculus. However the nature of this
calculus is still not well known. We propose the following possible
roadmap for further research:
\begin{enumerate}
\item prove or disprove that observational equivalence is decidable for Safe \ialgol;
\item find a categorical interpretation of the Safe $\lambda$-Calculus;
\item study the proof theory obtained by the Curry-Howard isomorphism and determine whether it has nice properties that can be helpful in theorem proving;
\item In \cite{DBLP:conf/tlca/LeivantM93}, the $\lambda$-calculus is used to
give several characterisations of the complexity class P. We would
like to investigate whether, by following similar techniques, we can
obtain a characterisation of a different complexity class using the
Safe $\lambda$-Calculus.
\end{enumerate}


In a more general direction of research, we would like to study the
class of languages for which pointers are uniquely recoverable. We
name this class PUR for ``Pointer Uniquely Recoverable''.

We proved that Safe $\lambda$-Calculus is a PUR-language. Another
example is the Serially Re-entrant Idealized Algol (SRIA) proposed
by Abramsky  in \cite{abramsky:mchecking_ia}. This language allows
multiple occurrences or uses of arguments, as long as they do not
overlap in time. In the game semantics denotation of a SRIA term
there is at most one pending occurrence of a question at any time.
Each move has therefore a unique justifier and consequently
justification pointers may be ignored. Safe \ialgol\ is not a
sublanguage of SRIA. One reason for this is that none of the two
Kierstead terms $\lambda f . f (\lambda x . f (\lambda y .y ))$ and
$\lambda f . f (\lambda x . f (\lambda y .x ))$ are Serially
Re-entrant whereas the first one is safe. Conversely, SRIA is not a
sublanguage of Safe \ialgol\ since the term $\lambda f g. f (\lambda
x . g (\lambda y .x ))$ where $f,g:((o,o),o)$ belongs to SRIA but
not to Safe \ialgol. SRIA and Safe \ialgol\ are therefore two
different examples of languages with pointer-less game semantics.

Finitary $\ialgol_2$ is also an example of PUR-language for which
observational equivalence is decidable. As we indicated in the first
chapter, decidability of observational equivalence is a very
appealing property which has immediate applications in the domain of
program verification. Intuitively, PUR-languages seem to be good
candidates of languages for which observational equivalence is
decidable. It would be interesting to discover classes of PUR
languages having this appealing property.

Another possible way to generate PUR-languages might be to constrain
the types of an existing language. In \cite{DBLP:conf/tlca/Joly01},
a notion of ``complexity'' is defined for $\lambda$-terms. It is
proved that a type $T$ can be generated from a finite set of
combinators if and only if there is a constant bounding the
complexity of every closed normal $\lambda$-term of type $T$;
consequently, the only inhabited finitely generated types are the
type of rank $\leq 2$ and the types $(A_1, A_2, \ldots, A_n, o)$
such that for all $i = 1..n$: $A_i = o$ , $A_i = o \rightarrow o$ or
$A_i = o^k \rightarrow o \rightarrow o$.

We know that imposing the first of these two type restrictions to
Finitary \ialgol\ leads to a PUR language. Is is also the case when
imposing the second type restriction?



\bibliographystyle{plain}
\bibliography{../bib/dphil-all}

\printindex

    %adds the bibliography to the table of contents
    \addcontentsline{toc}{chapter}
         {\protect\numberline{Bibliography\hspace{-96pt}}}


\end{document}



\makeindex

%\includeonly{chap_gamesem,chap_pincrjust,sec_safeia,transfer_chap_gsemsafety,../lmcs/safelambda,../corresp/corresp}

%\includeonly{corresp}
%\includeonly{fromlmcs/safelambda}

\author{William Blum}
\title{The safe lambda calculus  \\{\small DPhil thesis}}
\college{Linacre College}
\degree{Doctor of Philosophy}
\degreedate{?}
\renewcommand{\crest}{\beltcrest}

%\institution{Oxford University Computing Laboratory}
\date{Draft of \today}

%set the number of sectioning levels that get number and appear in the contents
\setcounter{secnumdepth}{3}
\setcounter{tocdepth}{3}

\begin{document}
\maketitle

%\setcounter{chapter}{0}
%\chapapp{Chapter}
\begin{abstract}
We consider a syntactic restriction for higher-order grammars called \emph{safety}  that  constrains occurrences of variables in the production rules according to their type-theoretic order. We transpose and generalize this restriction to the setting of the simply-typed lambda-calculus, giving us what we call the \emph{safe lambda calculus}. We study this language under different angles. First we give an account of its game semantic model. For that purpose, we introduce a new concrete presentation of game semantics based on the theory of \emph{traversals}: We show that the \emph{revealed game denotation} of a term can be computed by traversing some souped-up version of the abstract syntax tree of the term using adequately defined traversal rules. This result was presented at the Galop workshop at ETAPS 2008. This allows us to give a game-semantic analysis of safety via syntactic reasoning: We show that  safe lambda-terms are denoted by what we call \emph{P-incrementally justified strategies}. This result was presented at TLCA 2007.

We study the expressivity of the calculus and show a result in the
same vein as Schwichtenberg's 1976 characterization of the
simply-typed lambda calculus, we show that the numeric functions
representable in the safe lambda calculus are exactly the
multivariate polynomials; thus conditional is not definable. We
also give a characterization of representable word functions.
We then study the complexity of deciding beta-eta equality of two safe simply-typed terms and show that this problem is PSPACE-hard.

Finally we consider extension of the safety restriction to functional languages with recursion and references such as Idealized Algol.

\end{abstract}

\begin{romanpages}
\tableofcontents
\listoffigures
\end{romanpages}

    \chapter*{Acknowledgment}
    %\input{acknowledgment.texi}

    \chapter{Introduction}
    \input{chap_introduction.texi}


\part{Background}
    \chapter{Higher-Order Grammars and the Safety Restriction}
    \newcommand\lcalculrec{\Lambda^{\rightarrow}_\Sigma+Y}

\begin{proposition}
Higher-order recursion schemes are equivalent to the simply-typed lambda calculus extended with recursion and $\Sigma$-constants.
\end{proposition}
This is shown straightforwardly by showing that every higher-order recursion scheme can be converted into an equivalent lambda-term and conversely.

Let $\lcalculrec$ denotes the simply-typed lambda calculus extended with the typed-constants $\Sigma$ and the recursion combinator $Y$.


\begin{itemize}
\item First direction: Take a recursion scheme $\mathcal{R} = \langle \Sigma, \mathcal{N}, \mathcal{R}, S \rangle$.
We can construct an equivalent lambda term over the constant $\Sigma$ by induction on the rewriting rules as follows: we define a function $\Pi : \mathcal{A}(\Sigma,\mathcal{N}) \funto $ 

\end{itemize}

    \chapter{Lambda Calculus, PCF, Idealized Algol}
    \input{chap_languages.texi}

    % chapter presenting game semantics
    \chapter{Game Semantics}
    \chapter{Game semantics}

The aim of this chapter is to introduce game semantics. It starts
with a history of game semantics and a presentation of the full
abstraction problem for PCF which has been solved using game
semantics. It then goes on by introducing the basic notions of game
semantics and by giving a categorical interpretation of games.
Finally we show how games are used to define a syntax-independent
model of programming languages like PCF and Idealized Algol (IA).

This chapter is largely based on the tutorial by Samson Abramsky tutorial on Game Semantics \cite{AM98a}.
Most of the proof will be omitted and we refer the reader to
\cite{hylandong_pcf, abramsky94full} for a deeper description
of game semantics with complete proofs.

\section{History}

\subsection{Game semantics}

In the 1950s, Paul Lorenzen invented Game semantics as a tool to
study semantics of intuitionistic logic \citep{lor61}.

Four decade later, Abramsky proved the full completeness of
Multiplicative Linear Logic (MLL) using game semantics
\citep{abramsky92games}. Shortly after, game semantics has been used
as tool to study models of programming languages. In game semantics,
the meaning of a program is given by a strategy in a two-player
game. One player, the Opponent, represents the environment while the
other, the Proponent, represents the system.


\subsection{Model of programming languages}

Before the 1980s, there were many approaches to define models for
programming languages. Among the successful ones, there were the
axiomatic, operational and denotational semantics:
\begin{itemize}
\item Operational semantics gives a meaning to a program by describing the
behaviour of a machine executing the program. It is defined formally
by giving a state transition system.
\item Axiomatic semantics defined the behaviour of the program
with axioms and is used to prove program correctness by static
analysis of the code of the program.
\item The denotational semantics approach consists in mapping a program to a mathematical structure
having good properties such as compositionality. This mapping is
achieved by structural induction on the syntax of the program.
\end{itemize}

In the 1990s, three different independent research groups: Samson
Abramsky, Radhakrishnan Jagadeesan and Pasquale Malacaria
\citep{abramsky94full}, Martin Hyland and Luke Ong
\citep{hylandong_pcf} and Nickau \citep{Nickau:lfcs94} have
introduced game semantics, a new kind of semantics, in order to
solve a long standing problem in the semanticists community :
finding a fully abstract model for PCF.

\subsection{The problem of full abstraction for PCF}

PCF is a simple programming language introduced in a classical paper
by Plotkin ``LCF considered as a programming language''
(\cite{DBLP:journals/tcs/Plotkin77}). PCF is based on LCF, the Logic
of Computable Functions devised by Dana Scott in \cite{scott_lcf}.
It is a simply typed lambda calculus extended with arithmetic
operators, conditional and recursion.

The problem of the Full Abstraction for PCF goes back to the 1970s.
In \citep{scott93}, Scott gave a model for PCF based on domain
theory. This model gives a sound interpretation of observational
equivalence: if two terms have the same domain theoretic
interpretation then they are observationally equivalent. However the
converse is not true: there exist two PCF terms which are
observationally equivalent but have different domain theoretic
denotation. We say that the model is not fully abstract.

The key reason why the domain theoretic model of PCF is not fully
abstract is that the parallel-or operator defined by the following
truth table
\begin{center}
\begin{tabular}{l|lll}
p-or  & $\bot$ & tt & ff \\ \hline
$\bot$ & $\bot$ & tt & $\bot$\\
tt & tt & tt & tt\\
ff & $\bot$ & tt & ff\\
\end{tabular}
\end{center}
is not definable as a PCF term! It is possible to create two
different PCF terms that always behave the same except when they are
apply to a term computing p-or. Since p-or is not definable in PCF,
these two terms will have the same denotation. This implies that the
model is not fully abstract.

One can patch PCF by adding the operator $p-or$, the resulting
language ``PCF+p-or'' now becomes fully-abstracted by Scott domain
theoretic model \citep{DBLP:journals/tcs/Plotkin77}. However the
language we are now dealing with is strictly more powerful than PCF,
it allows parallel execution of commands whereas PCF only permits
sequential execution.

Another approach consists in getting rid of the undefinable elements
(like p-or) by strengthening the conditions on the function used in
the model (a condition stronger than strictness and continuity) but
unfortunately this approach did not succeed.

The only successful approaches to obtain a fully abstract model for
PCF were the ones taken by Ambramsky, Jagadeesan and Malacaria
\citep{abramsky94full}, Hyland and Ong \citep{hylandong_pcf} and
Nickau \citep{Nickau:lfcs94}, all based on game semantics.

This result has then been adapted to other varieties of programming
paradigm including languages with stores (Idealized Algol),
call-by-value \citep{honda99gametheoretic, abramsky98callbyvalue}
and call-by-name, general referencees
\citep{DBLP:conf/lics/AbramskyHM98}, polymorphism
\citep{DBLP:journals/apal/AbramskyJ05}, control features
(continuation and exception), non determinism, concurrency. In all
these cases, the game semantics model led to a syntax-independent
fully abstract model of the corresponding language.

\section{Games}
\label{sec:catgames}

We now introduce formally the notion of game that will be used in
the following section to give a model of the programming languages
PCF and Idealized Algol. The definitions are taken from
\cite{abramsky:game-semantics, hylandong_pcf, abramsky94full}.


\subsection{Arenas and Games}

The games we are interested in are two-players games. The players are named O for Opponent and P for Proponent.

The game played by O and P is constraint by something called
\emph{arena}. The arena defines the possible moves of the game. By
analogy with real board games, the arena represents the board
together with the rules that tell how players can make their moves
on the board. In fact the analogy with board game stops here. Our
games can be thought as dialog games: one person O interviews
another person P, P tries to answer the initial O-question by
possibly asking O some precisions about its initial question.
Moreover, the notion of winner and winning strategy will not be
relevant in our setting.


More formally, the arena can be seen as a forest of trees whose nodes are possible questions and leaves are possible answers.
The arena is partitioned into two kinds of moves: the moves that can be played by P and the ones that can be played by O.
A move is either a question to the other player or an answer to a question previously asked by the other player.

Each move of the game must be justified by another move that has already been played by the other player. This justification relation
is induced by the edges of the forest arena. Moreover, an answer must always be justified by the question that it answers and a question
is always justified by another question.

\begin{dfn}[Arena]
An arena is a structure $\langle M, \lambda, \vdash \rangle$ where:
\begin{itemize}
\item $M$ is the set of possible moves;
\item $(M,\vdash)$ is a forest of trees;

\item $\lambda : M \rightarrow \{ O, P\} \times \{Q, A\}$ is a labeling functions indicating whether a given move
    is a question or an answer and whether it can be played by O or by P.

    $\lambda = [\lambda^{OP},\lambda^{QA}]$ where $\lambda^{OP} : M \rightarrow  \{ O, P\}$
    and $\lambda^{QA} : M \rightarrow  \{ Q, A\}$.

    \begin{itemize}
    \item If $\lambda^{OP} (m) = O$, we call $m$ and O-move otherwise $m$ is a P-move.
    $\lambda^{QA} (m) = Q$ indicates that $m$ is a question otherwise $m$ is an answer.

    \item For any leaf $l$ of the tree $(M,\vdash)$, $\lambda^{QA} (l) = A$ and for any node
    $n \in (M,\vdash)$, $\lambda^{QA} (n) = Q$.
    \end{itemize}

\item The forest of tree $(M,\vdash)$ respect the following condition:
    \begin{itemize}
    \item[(e1)] The roots are O-moves: for any root $r$ of $(M,\vdash)$, $\lambda^{OP} (r) = O$.
    \item[(e2)] Answers are enabled by questions: $m \vdash n  \zand \lambda^{QA}(n) = A \imp \lambda^{QA}(m) = Q$.
    % Or more succinctly, if we write $\dashv$ the relation $\vdash^-1$: $\lambda^{QA} \left( \dashv( (\lambda^{QA})^{-1}(\{A\}) ) \right) = \{ O \}$
    \item[(e3)] A player move must be justified by a move played by the other player:
         $m\vdash n \imp \lambda^{OP}(m) \neq \lambda^{OP}(n)$.
    \end{itemize}
\end{itemize}
\end{dfn}

For commodity we write the set $\{O,P\} \times \{Q,A\}$ as $\{OQ,OA,PQ,PA\}$.
$\overline{\lambda}$ denotes the labeling function $\lambda$ with the question and answer swapped. For instance:
$$\overline{\lambda(m)} = OQ \iff \lambda(m) = PQ$$

The roots of the forest of tree $(M,\vdash)$ are the \emph{initial moves}.

For example, the simplest possible arena is written $\mathbf{1}$ and
denotes the arena which set of moves $M$ is empty.

\begin{exmp}[The flat arena]
\label{exmp:flatarena}

 Let $A$ be any countable set then the flat arena over $A$
is defined to be the arena $\langle M, \lambda, \vdash \rangle$ such
that $M$ has one move $q$ with $\lambda(q) = OQ$ and for each
element in $A$, there is a corresponding move $a_i$ in $M$ with
$\lambda(a_i) = PA$ for some $i \in \nat$. The enabling relation
$\vdash$ is defined to be $\{ q \vdash a_i \ | i \in \nat \}$.

This arena is represented by the following tree:
\begin{center}
  \pstree[levelsep=6ex]
    { \TR{$q$} }
    {    \TR{$a_1$} \TR{$a_2$} \TR{\ldots} }
\end{center}
The vertices represent the moves and the edges represent the
enabling relation.

The flat arena over $\nat$ and $\mathbb{B}$ is written
$\mathbf{int}$ and  $\mathbf{bool}$ respectively.

\end{exmp}

Once the arena has been defined, the bases of the game are set and the players have something to play with.
We now need to describe the state of the game, for that purpose
we introduced \emph{justified sequences of moves}. Sequence of moves are used to record the history of all the moves that have been
played.

\begin{dfn}[Justified sequence of moves]
A justified sequence is a sequence of moves $s$ together with an associated sequence of pointers. Any
move $m$ in the sequence that is not initial has as pointer that points to a previous move $n$ that justifies it (i.e. $n \vdash m$).
\end{dfn}

The pointers of a justified sequences are represented with arrows.
This is an example of justified sequence of moves:
$$\rnode{q4}{q}^4
\rnode{q3}{q}^3 \rnode{q2}{q}^2 \rnode{q3b}{q}^3 \rnode{q2b}{q}^2
\rnode{q1}{q}^1 \bkptrc{q3}{q4} \bkptrc{q2}{q3}
\bkptrc[ncurv=0.6]{q3b}{q4} \bkptrc{q2b}{q3b}$$

The first move of a justified sequence must be an O-move since
initial moves are all O-moves.

Notation: we write $s t$ or sometimes $s \cdot t$ do denote the
sequences obtain by concatenating $s$ and $t$. The empty sequence is
written $\epsilon$.

 A justified sequence has two particular subsequences which
will be of particular interest later on when we introduce
strategies. These subsequences are called the P-view and the O-view
of the sequence. The idea is that a view describes the local context
of the game. Here is the formal definition:

\begin{dfn}[View]
Given a justified sequence of moves $s$. We define the proponent view (P-view) noted $\pview{s}$ by induction:
\begin{align*}
\pview{\epsilon} &= \epsilon \\
\pview{s \cdot m} &= \pview{s} \cdot \ m && \mbox{ if $m$ is a P-move} \\
\pview{s \cdot m} &= m && \mbox{ if $m$ is initial (O-move) } \\
\pview{ s \cdot \rnode{m}{m} \cdot t \cdot \rnode{n}{n} \bkptra{50}{n}{m} } &=
 \pview{s} \cdot \rnode{mm}{m} \cdot \rnode{nn}{n} \bkptra{70}{nn}{mm} && \mbox{ if $n$ is a non initial O-move }
\end{align*}
The O-view $\oview{s}$ is defined similarly:
\begin{align*}
\oview{\epsilon} &= \epsilon \\
\oview{s \cdot m} &= \oview{s} \cdot \ m && \mbox{ if $m$ is a O-move} \\
\oview{ s \cdot \rnode{m}{m} \cdot t \cdot \rnode{n}{n} \bkptra{50}{n}{m} } &=
 \pview{s} \cdot \rnode{mm}{m} \cdot \rnode{nn}{n} \bkptra{70}{nn}{mm} && \mbox{ if $n$ is a P-move }
\end{align*}
\end{dfn}

In fact not all justified sequences will be of interest for the
games that we will use. We call \emph{legal position} any justified
sequence verifying two additional conditions: alternation and
visibility. Alternation says that players O and P plays
alternatively. Visibility expresses that each non-initial move is
justified by a move situated in the local context at that point.
Intuitively, the visibility condition gives some coherence to the
justification pointers of the sequence.

\begin{dfn}[Legal position]
A legal position is a justified sequence of move $s$ respecting the following constraint:
\begin{itemize}
\item Alternation: For any subsequence $m \cdot n$ of $s$, $\lambda^{OP}(m) \neq \lambda^{OP}(n)$.
\item Visibility: For any subsequence $t m$ of $s$ where $m$ is not initial, if $m$ is a P-move then $m$ points to a move in $\pview{s}$
and if $m$ is a O-move then $m$ points to a move in $\oview{s}$.
\end{itemize}

The set of legal position of an arena $A$ is noted $L_A$.
\end{dfn}

We say that a move $n$ is hereditarily justified by a move $m$ if there is a sequence of move
$m_1, \ldots, m_q$ such that:
$$ m \vdash m_1 \vdash m_2 \vdash \ldots m_q \vdash n$$
If a move has no justification pointer, we says that it is an
\emph{initial move} (in that case it must be a root of the forest
arena).

Suppose that $n$ is an occurrence of a move in the sequence $s$ then
$s \upharpoonright n$ denotes the subsequence of $s$ containing all the moves hereditarily justified by $n$.
Similarly, $s \upharpoonright I$ denotes the
subsequence of $s$ containing all the moves hereditarily justified by the moves in $I$.

\begin{dfn}[Game]
A game is a structure $\langle M, \lambda, \vdash, P \rangle$ such that
\begin{itemize}
\item $ \langle M, \lambda, \vdash \rangle$ is an arena.
\item $P$ is called the set of valid positions, it is:
    \begin{itemize}
    \item a non-empty prefix closed subset of the set of legal position
    \item closed by initial hereditary filtering: if $s$ is a valid position then for any set $I$ of occurrences of initial moves
    in $s$, $s\upharpoonright I$ is also a valid position.
    \end{itemize}
\end{itemize}
\end{dfn}

\begin{exmp}  Consider the flat arena  $\mathbf{int}$.
The set of valid position $P = \{ \epsilon, q \} \union \{ q \cdot
a_i \ | i \in \nat \}$ defines a game on the arena $\mathbf{int}$.
\end{exmp}

\subsection{Constructions on games}
\label{sec:gameconstruction}

We now define game constructors that will be useful later on.

Consider the two functions $f : A \rightarrow C$ and $g : B
\rightarrow C$, we write $[f,g]$ to denote the pairing of $f$ and
$g$ defined on the direct sum $A + B$. Given a game $A$ with a set
of moves $M_A$, we use the filtering operator $s \upharpoonright A$
do denote the subsequence of $s$ consisting of all moves in $M_A$.
Although this notation conflicts with the hereditarily filtering
operator, it should not cause any confusion.

\subsubsection{Tensor product}
Given two games $A$ and $B$ we define the tensor product constructor
$A \otimes B$ as follows:
\begin{eqnarray*}
  M_{A \otimes B} &=& M_A + M_B \\
  \lambda_{A\otimes B} &=& [\lambda_A,\lambda_B] \\
  \vdash_{A\otimes B} & = & \vdash_{A}\ \union\ \vdash_{B} \\
  P_{A\otimes B} & = & \{ s \in L_{A\otimes B} | s \upharpoonright A \in P_A \wedge s \ \upharpoonright B \in P_B  \}.
\end{eqnarray*}

In particular,  $n$ is initial in $A\otimes B$ if and only if $n$ is
initial in A or B. And $m \vdash_{A\otimes B} n$  holds if and only if $m
\vdash_{A} n$ or $m \vdash_{B} n$ holds.

\subsubsection{Function space}
The game $A \otimes B$ is defined as follows:
\begin{eqnarray*}
  M_{A \multimap B} &=& M_A + M_B \\
  \lambda_{A\multimap B} &=& [\overline{\lambda_A},\lambda_B] \\
  \vdash_{A\multimap B} & = & \vdash_{A}\ \union\ \vdash_{B}\ \union\  \{ (m,n) \ |\ m \mbox{ initial in } B \wedge n \mbox{ initial in } A \} \\
  P_{A\otimes B} & = & \{ s \in L_{A\otimes B} | s \upharpoonright A \in P_A \wedge s \ \upharpoonright B \in P_B  \}.
\end{eqnarray*}

\subsubsection{Cartesian product}
The game $A \& B$ is defined as follows:
\begin{eqnarray*}
  M_{A \& B} &=& M_A + M_B \\
  \lambda_{A\& B} &=& [\lambda_A,\lambda_B] \\
  \vdash_{A\& B} & = & \vdash_{A}\ \union\ \vdash_{B} \\
  P_{A\& B} & = & \{ s \in L_{A\otimes B} | s \upharpoonright A \in P_A \wedge s \ \upharpoonright B = \epsilon  \} \\
        &&   \union \{ s \in L_{A\otimes B} | s \upharpoonright A \in P_B \wedge s \ \upharpoonright A = \epsilon  \}.
\end{eqnarray*}

A play of the game $A \& B$ is either a play of $A$ or a play of $B$ whether a play
of the game $A \otimes B$ may be an interleaving of plays on $A$ and plays on $B$.

\subsection{Representation of plays}

Plays of the game are usually represented in a table diagram. The
columns of the table correspond to the different components of the
arena and each row corresponds to one move in the play. The first
row always represents an O-move, this is because O is the only
player who can open a game (since roots of the arena are O-moves).

As an example the play
$$\rnode{q1}{q}\
 \rnode{q2}{q}
 \ \rnode{a2}{8}
\  \rnode{a1}{12}
  \bkptrc{a1}{q1}
\bkptrc{a2}{q2} $$
on the
game $\textbf{int} \multimap \textbf{int} $ can be represented by
the following diagram:

\begin{center}
\begin{tabular}{cccc}
\textbf{int} & $\imp$ & \textbf{int} & \\
&& q & O\\
q  &&& P\\
8  &&& O\\
&& 12 & P
\end{tabular}
\end{center}

When it is necessary, the justification pointers of the play can also
be shown on the diagram.


\subsection{Strategy}

\subsubsection{Definition}

During a game, the player who has to play may have several choices
for his next move. The move that he makes is chosen according to a
given strategy.

A strategy is a rule telling the player which move to make when the
game is in a given position. More abstractly, a strategy is a
partial function mapping legal position where Proponent has to move
to P-moves.

\begin{dfn}[Strategy]
A strategy for player P on a given game $\langle M, \lambda, \vdash, P \rangle$ is a
non-empty set of even-length positions from $P$ such that:
\begin{enumerate}
\item (\emph{no unreachable position}) $sab \in \sigma \imp s \in \sigma$
\item (\emph{determinacy}) $sab, sac \in \sigma \quad \imp \quad  b = c$  and $b$ has the same justifier as
$c$.
\end{enumerate}
\end{dfn}

The idea is that the presence of the even-length sequence $s a b$ in
$\sigma$ tells the player P that whenever the game is in position
$s$ and player O plays the move $a$ then it must respond by playing
the move $b$.

The first condition ensures that the strategy $\sigma$ only
considers positions that the strategy itself could have led to in a
previous move. The second condition in the definition requires that
this choice of move is deterministic (i.e. there is a function $f$
from the set of odd length position to the set of moves $M$ such
that $f(s a) = b$).


For any game $A$, the smallest possible strategy is the strategy
that never respond given by $\{ \epsilon \}$. It is called the
\emph{empty strategy} and denoted $\bot$.

\subsubsection{Copy-cat strategy}

For any arena $A$ there is a strategy on the game $A \multimap A$
called the \emph{copy-cat strategy}. We write $A_1$ and $A_2$ to
denote the first and second copy of the arena $A$ in the game $A
\multimap A$. If $A$ is the arena $A_1$ then $A^\perp$ denotes the
arena $A_2$ and reciprocally.

Let $A$ be one of the arena $A_1$ or $A_2$. The copy-cat strategy
operates as follows: whenever P has to respond to an O-move played
in $A$, it replicates the move played by O in the arena $A^{\perp}$
after that $O$ has to respond in $A^{\perp}$ and $P$ replicates this
response in $(A^\perp)^\perp = A$ and so on and so forth.


More formally, the copy-cat strategy is defined by:
$$ \textsf{id}_A = \{ s \in P^{\textsf{even}}_{A \multimap A} \ | \ \forall t \sqsubseteq^{\textsf{even}} s\ .\ t \upharpoonright A_1 = t \upharpoonright A_2 \}$$
where $P^{\textsf{even}}_A$ denotes the valid position of even
length in the game $A$ and $t \sqsubseteq^{\textsf{even}} s$ denotes
that $t$ is an even length prefix of $s$.

The copy-cat strategy is also called \emph{identity strategy} since
it is the identity for strategy composition as we will see in the
next paragraph.

\begin{exmp} The copy-cat strategy on $\textbf{int}$ is:
$$\begin{array}{ccc}
\textbf{int} & \imp & \textbf{int} \\
&& q\\
q \\
n \\
&& n
\end{array}
$$
Note that we introduced this type of diagram to represent plays of
games but, as we can see here, the same diagrams can be used to
represent strategies when the play represented is general enough.

The copy-cat strategy on $\textbf{int} \typar \textbf{int}$ is given
by the following diagram:
$$\begin{array}{ccccccc}
(\textbf{int} & \imp & \textbf{int}) & \imp & (\textbf{int} & \imp & \textbf{int}) \\
&&&& && q\\
&& q\\
q \\
&&&& q \\
&&&& m \\
m\\
&& n \\
&&&& && n
\end{array}$$
\end{exmp}

\subsubsection{Composition}

It is well-known that any model of the simply typed lambda-calculus
is a cartesian closed category \citep{CroleRL:catt}. Games are used
to give a fully-abstract model of PCF, an extended simply typed
lambda calculus, therefore the game model should fit into a
cartesian closed category. This category will have games as objects
and strategies as morphisms. In a category, morphisms should be able
to compose together, therefore there should be an appropriate notion
of strategy composition.

Composition of strategies is an essential feature of game semantics.
As we will see in the following section, in the game model of PCF,
strategies represent programs. Therefore, strategy composition will
prove to be very useful : obtaining the model of a composed program
boils down to composing the strategies of the composing programs.

The way composition is defined for strategies is similar to
``parallel composition plus hiding'' in the trace semantics of CSP
\citep{hoare_csp}. Consider two strategies $\sigma : A \multimap B$
and $\tau : B \multimap C$ that we wish to compose.

For any sequence of moves $u$ on three arenas $A$, $B$, $C$, we call
projection of $s$ on the game $A \multimap B$ and we note $u
\upharpoonright A,B$ the subsequence of $s$ obtained by removing
from $u$ the moves in $C$ and pointers to moves in $C$. The
projection on $B \multimap C$ is defined similarly.

The definition of the projection on $A \multimap B$ differs
slightly: $u \upharpoonright A,C$ is the subsequence of $u$
consisting of the moves from $A$ and $C$ with some additional
pointers: we add a pointer from $a \in A$ to $c\in C$ whenever $a$
points to some move $b \in B$ itself pointing to $c$. All the
pointers to moves in $B$ are removed.


First we remark that for a given legal position $s$ in the game $A
\multimap C$, there is what is called an \emph{uncovering} of $s$.
The uncovering of $s$ is the maximal justified sequence of moves $u$
from the games $A$, $B$ and $C$ such that:
\begin{itemize}
\item The sequence $s$, considered as a pointer-less sequence, is a subsequence of
$u$;
\item the projection of $u$ on the game $A \multimap B$ lies in the
strategy $\sigma$;
\item the projection of $u$ on the game $B \multimap C$
lies in the strategy $\tau$;
\item and the projection of $u$ on the game $A \multimap C$ is a subsequence of $s$ (here the term ``subsequence'' refers to the sequence of nodes together with the auxiliary sequence of pointers).
\end{itemize}
This uncovering, noted $uncover(s, \sigma, \tau)$, is
defined uniquely for given strategies $\sigma$, $\tau$ and legal
position $s$ (this is proved in part II of \cite{hylandong_pcf}).

We define $\sigma \| \tau $ to be the set of uncovering of legal
positions in $A \multimap C$:
$$ \sigma \| \tau = \{ uncover(s, \sigma, \tau) \ | \ s \mbox{ is a legal position in } A \multimap C \}$$

The composition of $\sigma$, $\tau$ is defined to be the set of
projections of uncovering of legal positions in $A \multimap C$:

\begin{dfn}[Strategy composition]
Consider $\sigma : A \multimap B$ and  $\tau : B \multimap C$ two
strategies. We define $\sigma ; \tau$ to be:
$$ \sigma ; \tau = \{ u \upharpoonright A,C \ | \ u \in \sigma \|
\tau \}$$
\end{dfn}

It can be verified that composition is well-defined and associative
\citep{hylandong_pcf} and that the copy-cat strategy $\textsf{id}_A$ is the identity for composition.

\subsubsection{Constraint on strategies}

Different classes of strategies will be considered depending on the
features of the language that we want to model. Here is a list of
common restrictions that we will consider:
\begin{itemize}
\item \emph{Well-bracketing:} In a well-bracketed strategies the players always answer the last unanswered question (called the pending question) first.
If we represent Opponent's question as ``['', Proponent's answer as
``]'', Proponent's question as ``('' and Opponent's answers as ``)''
then requiring that the last pending question is answered first is
the same as requiring that the string representing the play is a
prefix of a well-bracketed sequence.

\item \emph{History-free strategies:} A strategy is history-free if the Proponent's move at any position of the game where he has to play
is determined by the last move of the Opponent. In other words, the
history prior to the last move is ignored by the Proponent when
deciding how to respond.

\item \emph{History-sensitive strategies:} The Proponent follows a history-sensitive strategy if he needs to have access to the full
history of the moves in order to decide which move to make.

\item \emph{Innocence:} a strategy is innocent if it determines Proponent's moves based on a restricted view of the history of the play, mainly the P-view
at that point. Such strategies can be specified by a partial
function mapping P-views to P-moves. However not every partial
function from P-views to P-moves gives rise to an innocent strategy
(a sufficient condition is given in \cite{hylandong_pcf}).
\end{itemize}

The formal definition of innocence follows:
\begin{dfn}[Innocence]
Given positions $sab, ta \in L_A$ where $sab$ has even length and
$\pview{sa} = \pview{ta}$, there is a unique extension of $ta$ by
the move $b$ together with a justification pointer such that
$\pview{sab} = \pview{sa}$. We write this extension
$\textsf{match}(sab,ta)$.

The strategy $\sigma:A$ is \emph{innocent} if and only if:
$$ \left(
     \begin{array}{c}
       \pview{sa} = \pview{ta} \\
       sab \in \sigma \\
       t\in \sigma \wedge ta \in P_A \\
     \end{array}
   \right)
\quad \imp\quad  \textsf{match}(sab,ta) \in \sigma$$

\end{dfn}


\subsection{Categorical interpretation of games}

In this section we recall some results about the categorical representation of Games.
These results with complete details and proofs can be found in \cite{McC96b,hylandong_pcf,abramsky94full}.
We refer the reader to \cite{CroleRL:catt} for more information about category theory.

We consider the category $\mathcal{G}$ whose objects are games and morphisms are
strategies. A morphism from $A$ to $B$ is a strategy on the game $A \multimap B$.

Three other sub-categories of $\mathcal{G}$ are considered: each of them correspond to some restriction on strategies:
$\mathcal{G}_i$ is the sub-category
of $\mathcal{G}$ whose morphisms are the innocent strategies,
$\mathcal{G}_b$ has only the well-bracketed strategies and $\mathcal{G}_{ib}$ has the innocent and well-bracketed strategies.

\begin{prop}
$\mathcal{G}$, $\mathcal{G}_i$, $\mathcal{G}_b$ and $\mathcal{G}_{ib}$ are categories.
\end{prop}

Proving this requires to prove that composition of strategies is well-defined, associative, has a unit (the copy-cat strategy), preserves innocence and
well-bracketedness. See \cite{hylandong_pcf,abramsky94full} for a proof.


\subsubsection{Monoidal structure}

We have already defined the tensor product on games in section \ref{sec:gameconstruction}.
We now define the corresponding transformation on morphisms:
given two strategies $\sigma : A \multimap B$ and $\tau : C \multimap D$ the strategy
$\sigma \otimes \tau : (A \otimes C) \multimap (B\otimes D)$ is defined by:
$$ \sigma \otimes \tau = \{ s \in L_{A \otimes C \multimap B\otimes D} \ s \upharpoonright A,B \in \sigma
\wedge s \upharpoonright C,D \in \tau \}$$

It can be shown that the tensor product is associative, commutative and has
$I = \langle \emptyset, \emptyset,\emptyset, \{ \epsilon \} \rangle $ as identity.
Hence the game categories $\mathcal{G}$ is a symmetric monoidal categories. Moreover
$\mathcal{G}_i$ and  $\mathcal{G}_b$ are sub-symmetric monoidal categories of $\mathcal{G}$,
and $\mathcal{G}_{ib}$ is a sub-symmetric monoidal category of $\mathcal{G}_i$, $\mathcal{G}_b$ and
$\mathcal{G}$.

\subsubsection{Closed structure}

For any game $A$, $B$ and $C$,
to any strategy $\sigma : A\otimes B \multimap C$, there is a corresponding strategy
$\tau : A\otimes B \multimap C$ obtained by relabeling the moves in $\sigma$. This transformation
is in fact an isomorphism: the hom-set $\mathcal{G}(A\otimes B, C)$ is isomorphic to the hom-set
$\mathcal{G}(A,B\multimap C)$. Hence $\mathcal{G}$ is an autonomous (i.e. symmetric monoidal closed) category.

$\mathcal{G}_i$ and  $\mathcal{G}_b$ are sub-autonomous categories of $\mathcal{G}$,
and $\mathcal{G}_{ib}$ is a sub-autonomous category of $\mathcal{G}_i$, $\mathcal{G}_b$ and
$\mathcal{G}$.

\subsubsection{Cartesian product}
The cartesian product defined in section \ref{sec:gameconstruction} is indeed a cartesian product in the category
$\mathcal{G}$, $\mathcal{G}_i$, $\mathcal{G}_b$ and $\mathcal{G}_{ib}$.

The projections $\pi_1:A \& B \rightarrow A$ and $\pi_1:A \& B \rightarrow B$ are given by the obvious copy-cat strategies.
Given two category morphisms $\sigma :C \rightarrow A$ and $\tau : C \rightarrow B$ the pairing function
$\langle \sigma, \tau \rangle : C \rightarrow A \& B$ is given by:
\begin{eqnarray*}
\langle \sigma, \tau \rangle &=& \{ s \in L_{C\multimap A\&B} \ | \ s \upharpoonright C,A \in \sigma \wedge s \upharpoonright B = \epsilon  \} \\
&\union& \{ s \in L_{C\multimap A\&B} \ | \ s \upharpoonright C,A \in \sigma \wedge s \upharpoonright B = \epsilon  \}
\end{eqnarray*}

\subsubsection{Cartesian closed structure}
Having defined the cartesian product is not enough to turn $\mathcal{G}$ into a cartesian closed category :
we also need to define a terminal object $I$ and the exponential construct $A \imp B$ for any two games $A$ and $B$.
In fact, this cannot be done in the current categories $\mathcal{G}$ and we have to move on to another category
of games noted $\mathcal{C}$ whose objects and morphisms are certain sub-classes of games and strategies.

Before introducing the category $\mathcal{C}$ we need some new definitions:


For any game $A$ we define the exponential game noted $!A$.
The game $!A$ corresponds to a repeated version of the game $A$. Plays of $!A$ are interleaving of plays of
$A$. It is defined as follows:
\begin{eqnarray*}
  M_{!A} &=& M_A \\
  \lambda_{!A} &=& \lambda_A \\
  \vdash_{!A} & = & \vdash_{A} \\
  P_{!A} & = & \{ s \in L_{!A} | \mbox{ for each initial move $m$, } s \upharpoonright m \in P_A \}
\end{eqnarray*}
The following equalities hold:
\begin{eqnarray*}
  !(A \& B) &=& !A \otimes !B\\
  I &=& !I
\end{eqnarray*}

\begin{dfn}[Well-opened games]
A game $A$ is well-opened if for any position $s \in P_A$ the only initial move is the first
one.
\end{dfn}

Well-opened games have single thread of dialog. Then can be turned into games with multiple-thread of dialog
using the promotion operator:

\begin{dfn}[Promotion]
Consider a well-opened game $B$.
Given a strategy on ${!A} \multimap B$, we define it promotion $\sigma^\dagger : {!A} \multimap {!B}$ to be the
strategy which plays several copies of $\sigma$. It is formally defined by:
$$ \sigma^\dagger = \{ s \in L_{{!A} \multimap !B} \ | \ \mbox{ for all initial $m$, } s \upharpoonright m \in \sigma  \}.$$
\end{dfn}

It can be shown that promotion is well-defined (it is indeed a strategy) and that it preserves innocence and
well-bracketedness.


We now introduce the category of well-opened games:
\begin{dfn}[Category of well-opened games]
The category $\mathcal{C}$ of well-opened games is defined as follow:
\begin{enumerate}
\item The objects are the well-opened games,
\item a morphism $\sigma : A \rightarrow B$ is a strategy for the game $!A \multimap B$,
\item the identity map for $A$ is the copy-cat strategy on $!A \multimap A$ (which is well-defined for well-opened games).
It is called dereliction, noted
$\textsf{der}_A$ and defined formally by:
$$ \textsf{der}_A = \{ s \in P^{\textsf{even}}_{{!A} \multimap A} \ | \ \forall t \sqsubseteq^{\textsf{even}} s \ . \ t \upharpoonright {!A} = t \upharpoonright A \},$$
\item composition of morphisms $\sigma : {!A} \multimap B$ and $\tau : {!B} \multimap C$ is defined to be
the strategy $\sigma^\dagger;\tau$ on the game ${!A} \multimap C$.
\end{enumerate}
\end{dfn}
$\mathcal{C}$ is a well-defined category and the three sub-categories
$\mathcal{C}_i$, $\mathcal{C}_b$, $\mathcal{C}_{ib}$ corresponding to sub-category
with innocent strategies, well-bracketed strategies and innocent and well-bracketed strategies respectively.


The category $\mathcal{C}$ has a terminal object $I$, for any two games $A$ and $B$ a product $A \& B$ and
an exponential $A \imp B$. Moreover the hom-sets $\mathcal{C}(A \& B,C)$ and
$\mathcal{C}(A,!B \multimap C)$ are isomorphic. Indeed:
\begin{eqnarray*}
\mathcal{C}(A\& B,C) &=& \mathcal{G}(!(A\& B),C) \\
&=& \mathcal{G}({!A}\otimes {!B}),C) \\
&\cong& \mathcal{G}({!A}, {!B} \multimap C) \qquad  \mbox{($\mathcal{G}$ is a closed monoidal category)}\\
&=& \mathcal{C}(A, {!B} \multimap C)
\end{eqnarray*}
Hence $\mathcal{C}$ is a cartesian closed category. Moreover $\mathcal{C}_i$ and $\mathcal{C}_b$
are sub-cartesian closed caterogies of $\mathcal{C}$ and $\mathcal{C}_{ib}$ is as sub-cartesian closed category
of each of $\mathcal{C}$, $\mathcal{C}_i$ and $\mathcal{C}_b$.



\subsubsection{Order enrichment}

Strategies can be ordered using the inclusion ordering.
The set of strategies on a given game $A$ is a pointed directed complete partial order under this ordering: the
least upper bounds is the union of two strategies and the least element is the empty strategy $\{ \epsilon \}$.

The category  $\mathcal{C}$ and  $\mathcal{G}$ are cpo-enriched.





directe It is possible to define an order on strategies


\subsection{Arena of order at most 2}
In this section, we consider a restricted class of arena and prove a
property on the games played on these arenas.

The height of the arena is the length of the longest sequence of moves
$m_1 \ldots m_h$ in $M$ such that $m_1 \vdash m_2 \vdash \ldots \vdash m_h$.

The order of an arena $\langle M, \lambda, \vdash \rangle$ is defined to be
$h-2$ where $h$ is the height of the forest of trees $(M, \vdash)$.


\begin{lem}[Pointers are superfluous up to order 2]
Let $A$ be the arena of order at most 2. Let $s$ be a justified sequence of moves in the arena $A$ satisfying
 alternation, visibility and well-bracketing then
the pointers of the sequence $s$ can be reconstructed uniquely.
\end{lem}



\begin{proof}
In the graphic representation of the arena, we display the sub-arena by decreasing order of sub-arena order.
It is safe to do so since in the definition of the forest of tree of an arena, the children nodes
are not ordered.

Let $A$ be an arena of order 2. We assume that $A$ has only one root. The arena $A$ has therefore the following shape:
\begin{center}
\
  \pstree[levelsep=6ex]
    { \TR{$q$} }
    {
\SubTree{$T_1$} \SubTree[linestyle=none]{$\ldots$} \SubTree{$T_n$}
    \TR{$a_1$} \TR{$a_2$} \TR{\ldots} }
\end{center}

where each triangle $T_i$ represents an arena of order 0 or 1.

We will see that the following proof can easily be adapted to take into account the general case of forest arenas (multiple roots).

We write $I_k$, for $k=0$ or $1$, the set of indices $i$ such that the arena $T_i$ has order $k$:
$$I_k = \{ i \in 1.. n\ |\ \order{T_i} = k \}$$

Here is a graphic representation of the arenas $T_i$ for $i \in I_0$ and $T_j$ for $j \in I_1$:
\begin{center}
\
  \pstree[levelsep=6ex]
    {\TR{$q^i$}}
    { \TR{$a_1^i$} \TR{$a_2^i$} \TR{\ldots} }
\hspace{2cm}
  \pstree[levelsep=6ex]
    { \TR{$p^j$} }
    {
      \pstree[levelsep=6ex]
        { \TR{$q^j$} }
        { \TR{$a_1^j$} \TR{$a_2^j$} \TR{\ldots} }
      \TR{$b_1^j$} \TR{$b_2^j$} \TR{\ldots}
    }
\end{center}



For any justified sequence of moves $u$, we write $?(u)$ for the
subsequence of $u$ consisting of the questions in the sequence $u$
that are still pending at the end of the sequence.

Let $L$ be the following language $L = \{\ p^i q^i\ | \ i \in I_1
\}$. We consider the following cases:

\begin{center}
\begin{tabular}{c|c|l|l}
Case & $\lambda_{OP}(m)$ & $?(u) \in$ & condition \\ \hline
0 & O & $\{ \epsilon \}$ \\
A & P & $q$ \\
B & O & $q \cdot L^* \cdot p^i$     & $i \in I_1$ \\
C & P & $q \cdot L^* \cdot p^i q^i$ & $i \in I_1$ \\
D & O & $q \cdot L^* \cdot q^i$      & $i \in I_0$ \\
\end{tabular}
\end{center}

We use the notation $\hat{s}$ to denote a legal and well-bracketed
\emph{justified} sequence of moves and $s$ to denote the same
sequence of moves with pointers removed.

Note that the well-bracketing condition already tells us how to
uniquely recover the pointers for P answer moves: a P-answers points
to the last pending question having the same tag. However for O
answers, we will see that the visibility condition already ensures
the unique recoverability of the pointer and that the
well-bracketing condition is not needed.


We prove by induction on the sequence of moves $u$ that $?(u)$
corresponds to either case 0, A, B, C or D and that the pointers in
$u$ can be recovered uniquely.

\textbf{Base cases:}

If $u$ is the empty sequence $\epsilon$ then there is no pointer to
recover and it corresponds to case 0.

If $u$ is a singleton then it must be the initial question $q$ and
there is not pointer to recover. This corresponds to case A.

\textbf{Step case:}

Consider a legal well-bracketed justified sequence $\hat{s}$ where
$s = u \cdot m$ and $m \in M_A$. The induction hypothesis tells us
that the pointers of $u$ can be recovered (and therefore the P-view
or O-view at that point can be computed) and that $u$ corresponds to
one of the cases 0,A,B,C or D.

We proceed by case analysis on $u$:

\begin{description}

\item[case 0] This case cannot happen because $?(u) = \epsilon$ ($u$ is a complete play) implies that there cannot be any further move $m$.

Indeed the visibility condition implies that $m$ must point to a
P-question in the O-view at that point. But since $u$ is a complete
play, the O-view is $\oview{\hat{u}} = q a$ which does not contain
any P-question. Hence the move $m$ cannot be justified and is not
valid.


\item[case A] $?(u) = q$ and the last move $m$ is played by P.
    There are several cases:
    \begin{itemize}
    \item $m$ is an answer $a_k$ (to the initial question
    $q$) for some $k$, then $m$ points to $q$:

    $\hat{s} = \justseq{ q & \ldots & m \pointto{ll}}$

    and $?(s) = \epsilon$ therefore $s$ correspond to the case 0 (complete play).

    \item $m = q^i$ where $q^i$ is an order 0 question ($i \in I_0$).
    Then $q^i$ points to the initial question $q$ and $s$ falls into category D.

    \item $m = p^i$, a first order question, then $p^i$ points to $q$,

    $?(s)= q p^i$ and it is O's turn after $s$ therefore $s$ falls into category B.

    \end{itemize}


\item[case B] $?(u) \in q \cdot L^* \cdot p^i$ where $i \in I_1$ and O plays the move $m$.

We now analyse the different possible O-moves:
\begin{itemize}
\item Suppose that O gives the (tagged) answer $b^j$ for some $j \in I_1$ then
the visibility condition constraints it to point to a question in
the O-view at that point.

We remark that the last move in $\hat{u}$ must be $p^i$. Indeed,
suppose that there is a move $x \in M_A$ such that $\hat{u} =
\justseq{q & \ldots & p^i\ x \pointto{ll}}$ then by visibility, the
O-move $x$ should points to a move in the O-view a that point. The
O-view is $q p^i$, therefore $x$ can only points to $p^i$. But then,
$p^i$ is not a pending question in $s$ which is a contradiction.


Therefore $\oview{\hat{u}} = \oview{ \justseq{ q & \ldots & p^i
\pointto{ll}} } = q p^i$.

Hence $b^j$ can only point to $p^i$ (and therefore $i=j$).

We then have $?(s) = ?(u \cdot b^i) \in  q \cdot L^*$ which is
covered by case A and C.

\item The only other possible O-move is $q^i$ which, again by the visibility condition, points necessarily
to the previous move $p^i$. We then have $?(s) = ?(u \cdot q^i) \in
q \cdot L^* \cdot p^i q^i$. This falls into category C.

\end{itemize}

\item[case C] $?(u) \in q \cdot L^* \cdot p^i q^i$ where $i \in I_1$ and the move $m$ is played by $P$.

Suppose $m$ is an answer, then the well-bracketing condition imposes
to answer to $q^i$ first. The move $m$ is therefore an integer $a^i$
pointing to $q^i$. We then have $?(s) = ?(u \cdot a^i) \in  q \cdot
L^* \cdot p^i$. This correspond to case B.


Suppose $m$ is a question then there are two cases:
\begin{itemize}
\item $m = q^j$ with $j \in I_0$, the pointer goes to the initial question $q$ and $s$ falls into category D.
\item $m = p^j$ with $j \in I_1$, the pointer goes to the initial question $q$ and $s$ falls into category B.
\end{itemize}

\item[case D] $?(u) \in q \cdot L^* \cdot q^i$ where $i \in I_0$ and the move $m$ is played by $O$.

    The same argument as in case B holds. However there is now another possible move:
    the answer $m = a^i_k$ for some $k$.  This moves can only points to
    $q^i$ (this is the only pending question tagged by $i \in I_0$).

    Then $?(\hat{s}) = ?(\hat{u}\cdot a^i_k) = ?(\justseq{ q & \ldots & q^i \pointto{ll} & \ldots & a^i_k \pointto{ll}}) \in q \cdot L^* $ therefore $s$ falls either into category A or C.

\end{description}

This completes the induction.

How to generalize the proof to arenas that have multiple roots
(forest arenas)? In fact there is no ambiguity since all the moves
are implicitly tagged according to the arena that they belong to.
Therefore in the induction, it suffices to ignore the moves that
belong to another tree (as if they were part of a different game
played in parallel).


\end{proof}


\subsection{Pointer-less strategies}
\label{subsec:ptrless_strat}

Up to order 2, the semantics of PCF terms is entirely defined by
pointer-less strategies. In other words, the pointers can be
uniquely reconstructed from any non justified sequence of moves
satisfying the visibility and well-bracketing condition.

At level 3 however, pointers cannot be omitted in general. Here is an example
taken from \cite{abramsky:game-semantics} illustrating this. Consider the
following two terms of type $((\nat \typar \nat) \typar \nat) \typar
\nat$:

$$M_1 = \lambda f . f (\lambda x . f (\lambda y .y ))$$
$$M_2 = \lambda f . f (\lambda x . f (\lambda y .x ))$$

We assign tags to the types in order to identify in which arena the
questions are asked: $((\nat^1 \typar \nat^2) \typar \nat^3) \typar
\nat^4$. Consider now the following pointer-less sequence of moves
$s = q^4 q^3 q^2 q^3 q^2 q^1$. It is possible to retrieve the
pointers of the first five moves but there is an ambiguity for the
last move: does it point to the first or second occurrence of $q^2$
in the sequence $s$?

Note that the visibility condition does not eliminate the ambiguity,
since the two occurrences of $q^2$ both appear in the P-view at that
point (after recovering the pointers of $s$ up to the second last
move we get:
$$s = \rnode{q4}{q}^4
\rnode{q3}{q}^3
\rnode{q2}{q}^2
\rnode{q3b}{q}^3
\rnode{q2b}{q}^2
\rnode{q1}{q}^1
\bkptrc{q3}{q4}
\bkptrc{q2}{q3}
\bkptrc[ncurv=0.6]{q3b}{q4}
\bkptrc{q2b}{q3b}$$

 therefore the P-view of $s$ is $s$ itself.)

In fact these two different possibilities correspond to two
different strategies. Suppose that the link goes to the first
occurrence of $q^2$ then it means that the proponent is requesting
the value of the variable $x$ bound in the subterm $\lambda x . f (
\lambda y. ... )$. If P needs to know the value of $x$, this is
because P is in fact following the strategy of the subterm $\lambda
y . x$. And the entire play is part of the strategy $\sem{M_2}$.

Similarly, if the link points to the second occurrence of $q^2$ then
the play belongs to the strategy $\sem{M_1}$.

\section{Game model for PCF}
\subsection{Syntax of the PCF language}
PCF is a simply-type $\lambda$-calculus with the following
additions: integer constants  (of ground type), first-order
arithmetic operators, if-then-else branching, and the recursion
combinator $Y_A : (A\rightarrow A)\rightarrow A$ for any type $A$.

The types of PCF are given by the following grammar:
$$ T ::= \texttt{exp}\ |\ T \rightarrow T$$

The following grammar gives the structure of terms:
\begin{eqnarray*}
 M ::= x\ |\ \lambda x :A . M \ |\ M M \ |\ \\
\ |\ n \ |\ \texttt{succ } M \ |\  \texttt{pred } M \\
\ |\ \texttt{cond } M M M \ |\ \texttt{Y}_A\ M
\end{eqnarray*}

where $x$ ranges over a set of countably many variables and $n$
ranges over the set of natural numbers.

Terms are generated according to the formation rules given in table
\ref{tab:pcf_formrules} where the judgement is of the form $ \Gamma  \vdash M : A$.

\begin{table}[htbp]
$$ (var) \rulef{}{x_1:A_1, x_2:A_2, \ldots x_n : A_n  \vdash x_i : A_i}\ i \in 1..n$$
$$ (app) \rulef{\Gamma \vdash M : A\rightarrow B \qquad \Gamma \vdash N:A}{\Gamma \vdash M\ N : B}
\qquad (abs) \rulef{\Gamma, x:A \vdash M : B}{\Gamma \vdash \lambda x :A . M : A\rightarrow B}$$

$$ (const) \rulef{}{\Gamma \vdash n :\texttt{exp}}
\qquad (succ) \rulef{\Gamma \vdash M:\texttt{exp} }{\Gamma \vdash \texttt{succ}\ M:\texttt{exp}}
\qquad (pred) \rulef{\Gamma \vdash M:\texttt{exp} }{\Gamma \vdash \texttt{pred}\ M:\texttt{exp}}$$

$$
(cond) \rulef{\Gamma \vdash M : exp \qquad \Gamma \vdash N_1 : exp \qquad \Gamma \vdash N_2 : exp }{\Gamma \vdash \texttt{cond}\ M\ N_1\ N_2}
\qquad  (rec) \rulef{\Gamma \vdash M : A\rightarrow A }{ \Gamma \vdash Y_A M : A}$$

\caption{Formation rules for PCF terms}
\label{tab:pcf_formrules}
\end{table}

\subsection{Operational semantics of PCF}

We give the big-step operational semantics of PCF. The notation $M \eval V$ means
that the closed term $M$ evaluates to the canonical form $V$. The canonical forms are given by the following
grammar:
$$V ::= n\ |\ \lambda x. M$$
In other word, a canonical form is either a number or a function.

The operational semantics is given for closed terms therefore the context $\Gamma$ is not present in
the evaluation rules.

The full operational semantics is given in table \ref{tab:bigstep_pcf}.

\begin{table}[htbp]
$$\rulef{}{V \eval V} \quad \mbox{ provided that $V$ is in canonical form.} $$

$$ \rulef{M \eval \lambda x. M' \quad M'\subst{x}{N}}{M N \eval V}$$

$$\rulef{M \eval n}{\texttt{succ}\ M \eval n+1}
\qquad \rulef{M \eval n+1}{\texttt{pred}\ M \eval n}
\qquad \rulef{M \eval 0}{\texttt{pred}\ M \eval 0}$$

$$\rulef{M \eval 0 \quad N_1 \eval V}{\texttt{cond}\ M N_1 N_2  \eval V}
\qquad
 \rulef{M \eval n+1 \quad N_2 \eval V}{\texttt{cond}\ M N_1 N_2  \eval V}$$

$$\rulef{M (\mathrm{Y} M) \eval V }{\texttt{Y} M \eval V}$$
\label{tab:bigstep_pcf}
\caption{Big-step operational semantics of PCF}
\end{table}



\section{Idealized Algol (IA)}
\label{sec:ia}

\subsection{The syntax of IA}
IA is an extension of PCF introduced by J.C. Reynold in
\cite{Reynolds81}. It adds imperative features such as local variables and sequential composition.

The description of the language that we give here follows the one of \cite{abramsky:game-semantics}.

On top of \texttt{exp}, PCF has the following two new types:
 \texttt{com} for commands and \texttt{var} for variables.

There is a constant \texttt{skip} of type \texttt{com} which corresponds to the command that do
nothing. Commands can be composed using the sequential composition operator \texttt{seq}.
Local variable are declared using the \texttt{new} operator, variable content is written
using \texttt{assign} and retrieved using \texttt{deref}.

The new formations rules are given in table \ref{tab:ia_formrules}.

\begin{table}[htbp]
$$ \rulef{\Gamma \vdash M : \texttt{com} \quad \Gamma \vdash N :A}
    {\Gamma \vdash \texttt{seq}_A \ M\ N\ : A} \quad A \in \{ \texttt{com}, \texttt{exp}\}$$

$$ \rulef{\Gamma \vdash M : \texttt{var} \quad \Gamma \vdash N : \texttt{exp}}
    {\Gamma \vdash \texttt{assign}\ M\ N\ : \texttt{com}}
\qquad
 \rulef{\Gamma \vdash M : \texttt{var}}
    {\Gamma \vdash \texttt{deref}\ M\ : \texttt{exp}}$$

$$ \rulef{\Gamma, x : \texttt{var} \vdash M : A}
    {\Gamma \vdash \texttt{new } x \texttt{ in } M} \quad A \in \{ \texttt{com}, \texttt{exp}\}$$

$$ \rulef{\Gamma \vdash M_1 : \texttt{exp} \rightarrow \texttt{com} \quad \Gamma \vdash M_2 : \texttt{exp}}
    {\Gamma \vdash \texttt{mkvar } M_1\ M_2\ : \texttt{var}}$$

\caption{Formation rules for IA terms}
\label{tab:ia_formrules}
\end{table}

If $\vdash M : A$ (i.e. $M$ can be formed with an empty context), we say that $M$ is a close term.

\subsection{Operational semantics}

In IA the semantics is given in a slightly different form from PCF.
In PCF, the evaluation rules were given for closed terms only. Suppose that we
proceed the same way for IA and consider the evaluation rule for the $\texttt{new}$ construct:
the conclusion is $\texttt{new } x:=0 \texttt{ in } M$ and the premise
is an evaluation for a certain term constructed from $M$, more precisely the term $M$
where \emph{some} occurrences of $x$ are replaced by the value $0$.
Because of the presence of the \texttt{assign} operator, we cannot simply replace all
the occurrences of $x$ in $M$ (the required substitution is  more complicated
than the substitution used for beta-reduction).


Therefore, instead of giving the semantics for closed term we consider terms
whose free variables are all of type \texttt{var}. These free variables are ``closed'' by mean of
stores. A store is a function mapping free variables of type \texttt{var} to natural numbers.
Suppose $\Gamma$ is a context containing only variable of type \texttt{var}, then we say that
$\Gamma$ is a \texttt{var}-context. A store whose domain $\Gamma$ is called a $\Gamma$-store.

The notation $s\ |\ x \mapsto n$ refers to the store that maps $x$ to $n$
and otherwise maps variables according to the store $s$.


The canonical forms for IA are given by the grammar:
$$ V ::= n\ |\ \lambda x. M\ |\ x\ |\  \texttt{mkvar} M N$$

where $n \in \nat$ and $x:var$.


A program is now defined by a term together with a $\Gamma$-store such that $\Gamma \vdash M : A$.
The evaluation semantics is expressed by the judgment form
$$s,M \eval s', V$$
where $s$ and $s'$ are $\Gamma$-stores,
$\Gamma \vdash M : A$ and $\Gamma \vdash V : A$ where $V$ is in canonical form.

The operational semantics for IA is given by the rule of PCF (table \ref{tab:bigstep_pcf})
together with the rules of table \ref{tab:bigstep_ia} where the following abbreviation is used:
$$ \rulef{M_1 \eval V_1 \quad M_2 \eval V_2}{M \eval V} \qquad \mbox{for} \qquad
  \rulef{s,M_1 \eval s',V_1 \quad s', M_2 \eval s'',V_2 }{s,M \eval s'',V}
$$


\begin{table}[htbp]
$$\mbox{\textbf{Sequencing }}
    \rulef{M \eval \iaskip \quad N \eval V}{\texttt{seq } M\ N \eval V}
$$

$$\mbox{\textbf{Variables }}
    \rulef{s,N \eval s',n \quad s',M \eval s'',x}{s, \assign\ M\ N \eval (s''\ |\ x \mapsto n),\iaskip}
\qquad
    \rulef{s,M \eval s',x }{s, \deref\ M \eval s',s'(x)}$$

$$\mbox{\texttt{\textbf{mkvar}}}
    \rulef{N \eval n \quad M \eval \texttt{mkvar } M_1\ M_2 \quad M_1\ n \eval \iaskip}
    {\assign\ M\ N \eval \iaskip}
\qquad
    \rulef{N \eval \texttt{mkvar } M_1\ M_2 \quad M_2\ \eval n}
    {\deref\ M \eval n}
$$

$$\mbox{\textbf{Block}}
    \rulef{(s\ |\ x \mapsto 0),M \eval (s'\ |\ x \mapsto n),V }
    {s, \texttt{new } x \texttt{ in } M \eval s',V}
$$

\label{tab:bigstep_ia}
\caption{Big-step operational semantics of IA}
\end{table}

\subsection{Game semantics}

As we have seen in section \ref{sec:catgames}, games and strategies
form a cartesian closed category, therefore games can model the simply-typed $\lambda$-calculus. Let us first
explain how this is achieved before extending the model to PCF and IA.

\subsubsection{Simply typed $\lambda$-calculus}

In the cartesian closed category $\mathcal{C}$, the objects are the arenas and the morphisms are the strategies.

In the games that we describe here, the Opponent represents the environment while
the Proponent plays according to a strategy imposed by the program itself.


Given a simple type $A$, we will model it as an arena $\sem{A}$.
A context $\Gamma = x_1 :A_1, \ldots x_n:A_n$ will be mapped to the arena
$\sem{\Gamma} = \sem{A_1} \times \ldots \times \sem{A_n}$ and a term $\Gamma \vdash M : A$
will be modeled by a strategy on the arena $\sem{\Gamma} \rightarrow \sem{A}$.
Since $\mathcal{C}$ is cartesian closed, there is is a terminal object $\textbf{1}$ (the empty arena) that
models the empty context ($\sem{\Gamma} = \textbf{1}$).


The base type \texttt{exp} is interpreted by the following flat arena of natural numbers noted $\nat$:
$$  \pstree[levelsep=6ex]
    {\TR[name=R]{q}}
    { \TR{1} \TR{2} \TR{\ldots}
    }
$$
In this arena, there is only one question: the initial O-question, P can then answer it by playing a natural number $i \in \nat$.
There are only two kinds strategy on this arena:
\begin{itemize}
\item the empty strategy where P never answer the initial question. This corresponds to a non terminating computation;
\item the strategies where P answers by playing a number $n$. This models the constants of the language.
\end{itemize}

Given the interpretation of base types, we define the interpretation of $A\rightarrow B$ by induction:
$$\sem{A \rightarrow B} = \sem{A} \Rightarrow \sem{B}$$

where the operator $\Rightarrow$ denotes the arena construction $!A
\multimap B$ which exists because $\mathcal{C}$ is cartesian closed.

Graphically if we represent the arena $A$ and $B$ by two triangles, the arena for $A \rightarrow B$ would be represented by:
\begin{center}
\psset{xunit=.5pt,yunit=.5pt,runit=.5pt}
\begin{pspicture}(150,80)
\rput[tr](150,80){ \pnode(27,40){a} \pstribox{A} }
\rput[bl](0,0){ \pnode(27,40){b} \pstribox{B} }
\ncline{->}{a}{b}
\end{pspicture}
\end{center}


Variables are interpreted by projection:
$$\sem{x_1 : A_1, \ldots, x_n:A_n \vdash x_i : A_i} = \pi_i : \sem{A_i} \times \ldots \times \sem{A_i} \times \ldots \times \sem{A_n} \rightarrow  \sem{A_i}$$

The abstraction $\Gamma \vdash \lambda x :A.M : A \rightarrow B$ is modeled by a strategy on the arena
$\sem{\Gamma} \rightarrow (\sem{A}\Rightarrow\sem{B})$. This strategy is obtain by using the currying operator of the
cartesian closed category:
$$\sem{\Gamma \vdash \lambda x :A.M : A \rightarrow B} = \Lambda( \sem{\Gamma, x :A \vdash M : B})$$

The application $\Gamma \vdash M N$ is modeled using the evaluation map $ev_{A,B} : (A\Rightarrow B)\times A \rightarrow B$:

$$\sem{\Gamma \vdash M N} = \langle \sem{\Gamma \vdash M, \Gamma \vdash N} \rangle; ev_{A,B}$$


\subsubsection{PCF}

We now show how to model the PCF constructs in the game semantics setting.
In the following, the sub-arena of a game are tagged in order to distinguish identical arenas that are present in different components of the game.
Moves are also tagged in the exponent in order to identify the sub-arena in which moves are played. We will omit the pointers in the play
when they are not essential for the understanding of the model (moreover we will see later on that under certain assumptions
up to order 2, pointers can be recovered uniquely).

The successor arithmetic operator is modeled by the following strategy on the arena $\nat^1 \Rightarrow \nat^0$:
$$\sem{\texttt{succ}} = \{q^0 \cdot q^1 \cdot n^1 \cdot (n+1)^0\ |\ n \in \nat \}$$

The predecessor arithmetic operator is denoted by the strategy
$$\sem{\texttt{pred}} = \{q^0 \cdot q^1 \cdot n^1 \cdot (n-1)^0\ |\ n >0 \} \union \{ q^0 \cdot q^1 \cdot 0^1 \cdot 0^0 \} $$

Then given a term $\Gamma \vdash \texttt{succ} M : \texttt{exp}$ we define:
$$\sem{\Gamma \vdash \texttt{succ } M : \texttt{exp}} = \sem{\Gamma \vdash M} ; \sem{\texttt{succ}} $$
$$\sem{\Gamma \vdash \texttt{pred } M : \texttt{exp}} = \sem{\Gamma \vdash M} ; \sem{\texttt{pred}} $$


The conditional operator is denoted by the following strategy on the arena $\nat^3 \times \nat^2 \times \nat ^1 \Rightarrow \nat^0$:
$$\sem{\texttt{cond}} =
    \{ q^0 \cdot q^3 \cdot 0 \cdot q^2 \cdot n^2 \cdot n^0 \ | \ n \in \nat \}
    \union
    \{ q^0 \cdot q^3 \cdot m \cdot q^2 \cdot n^2 \cdot n^0 \ | \ m >0, n \in \nat \}
    $$


Given a term $\Gamma \vdash \texttt{cond} M\ N_1\ N_2$ we define:
$$\sem{\Gamma \vdash \texttt{cond} M\ N_1\ N_2} =
\langle \sem{\Gamma \vdash M}, \sem{\Gamma \vdash N_1}, \sem{\Gamma \vdash N_2} \rangle ; \sem{\texttt{cond}}$$


The interpretation of the \texttt{Y} combinator is a bit more complicated.

Consider the term $\Gamma \vdash M : A \rightarrow A$, its semantics $f$ is a strategy on $\sem{\Gamma} \times \sem{A} \rightarrow \sem{A}$.
We define the chain $g_n$ of strategies on the arena $\sem{\Gamma} \rightarrow \sem{A}$ as follows:
\begin{eqnarray*}
g_0 &=& \perp \\
g_{n+1} &=&  F(g_n) = \langle id_{\sem{\Gamma}}, g_n\rangle ; f
\end{eqnarray*}

where $\perp$ denotes the empty strategy $\{ \epsilon \}$.

It is easy to see that indeed the $g_n$ forms a chain.
We define $\sem{\texttt{Y } M}$ to be the least upper bound of the chain $g_n$
(i.e. the  least fixed point of $F$). Its existence is guaranteed by the fact that
the category of games is cpo-enriched.

\subsubsection{IA}

It is easy to check that all the strategies given until now are well-bracketed and innocent.
From now on, we will only require well-bracketing and we will introduce strategies that are
not innocent. This is a necessity if we want to give a model of memory cells that correspond
to variables. The intuition behind this fact is that a cell needs to remember what was the last value written in it
in order to be able to return it when it is read, and this can only be done by looking at the whole history of moves,
not only those present in the P-view.





\subsection{Full-abstraction}
In this section we recall the standard full abstraction result proved in  \cite{abramsky94full}
and \cite{hylandong_pcf}.

A context noted $C[-]$ is a term containing a hole denoted by $-$. If $C[-]$ is a context then $C[A]$ denotes the term obtained
after replacing the hole by the term $A$.

\begin{dfn}[Observational preorder]
Let $\vdash M : A$ and $\vdash N : A$ be two closed terms. We define the relation $\sqsubseteq$ as follows:


$M \sqsubseteq N$ if and only if for all context $C[-]$ such that $C[M]$ and $C[M]$ are well-formed terms if
$C[M] \eval$ then $C[N] \eval$.
\end{dfn}


\begin{lem}[Soundness for PCF terms] Let $M$ be a PCF term.
If $M \eval V$ then $\sem{M} = \sem{V}$.
\end{lem}

\begin{lem}[Soundness for IA terms] Let $\Gamma \vdash M : A$ be an IA term and a $\Gamma$ store $s$.
If $s,M \eval s',V$ then the plays of $\sem{s,M} : I \multimap A \otimes !\Gamma$ which begin
with a move of $A$ are identical to those of $\sem{s',V}$.
\end{lem}


\begin{lem}[Computational adequacy for PCF terms]
All PCF terms are computable. (i.e. $\sem{M} \neq \perp$ implies $M \eval$)
\end{lem}

\begin{lem}[Computational adequacy for IA terms]
All IA terms are computable. (i.e. $\sem{M} \neq \perp$ implies $M \eval$)
\end{lem}


The following result follows from soundness and computational adequacy of the model.
\begin{prop}[Inequational soundness]
\label{prop:ineqsoundness}
Let $M$ and $N$ be two closed terms then
$$\sem{M} \subseteq \sem{N} \implies  M \sqsubseteq N $$
\end{prop}

\begin{prop}[Definability]
\label{prop:definability}
Let $\sigma$ be a compact well-bracketed on a game $A$ denoting a IA type. Then there is
an IA-term $M$ such that $\sem{M} = \sigma$.
\end{prop}

The final standard result of game semantics can then be proved using proposition \ref{prop:ineqsoundness} and \ref{prop:definability}:
\begin{thm}[Full abstraction]
Let $M$ and $N$ be two closed IA-terms.
$$\sem{M} \precsim_b \sem{N} \ \iff \ M \sqsubseteq N$$
\end{thm}

where $\precsim_b$ denotes the intrinsic preorder of the category $\mathcal{C}_b$.


\subsection{Call-by-Value first-order Idealized Algol}

Game semantics for call-by-value programming Language.




\part{Contribution}


In the second chapter we present the \emph{safe $\lambda$-calculus}.
Originally, \emph{safety} has been introduced as a syntactical
restriction on higher-order grammars in order to show a decidability
result about MSO theory of infinite trees \citep{KNU02}. In
\cite{safety-mirlong2004}, Aehlig, de Miranda and Ong  proposed an
adaptation of the safety restriction to the $\lambda$-calculus. This
restriction gives rise to the safe $\lambda$-calculus. We first
present this calculus and then give a more general definition which
does not make any assumption on the types of the terms.

In the third chapter, following ideas described in
\cite{OngLics2006}, we introduce the notions of computation tree of
a simply-typed term and traversal over a computation tree. We prove
a theorem showing a correspondence between traversals of the
computation tree and the game semantics of a term. Based on that
correspondence, we give a characterisation of the game semantics of
safe terms by a property called ``P-incremental-justification''. In
P-incrementally-justified strategies, P-pointers are superfluous (i.e.
they can be recovered uniquely from the underlying sequence of
moves and from O-moves' pointers). This simplification of the game semantics suggests some potential applications in algorithmic game semantics. We finish the
chapter by extending the result to safe \pcf\ and by giving the key
elements for an extension to full Safe Idealized Algol.




\chapter{Safe Higher Order Functional Languages}
\label{chap:safelambda}
%    \section{Safe Lambda Calculus}
%        \subsection{Definition and properties}
%        \subsection{Expressivity of the calculus}
%        Result a la Schwichtenberg \cite{citeulike:622637}
%        Statman's result for Safe Lambda calculus?
    \section{Introduction}

\subsection*{Background}

The \emph{safety condition} was introduced by Knapik, Niwi{\'n}ski and
Urzyczyn at FoSSaCS 2002 \cite{KNU02} in a seminal study of the
algorithmics of infinite trees generated by higher-order grammars. The
idea, however, goes back some twenty years to Damm \cite{Dam82} who
introduced an essentially equivalent\footnote{See de Miranda's
 thesis \cite{demirandathesis} for a proof.} syntactic
restriction (for generators of word languages) in the form of
\emph{derived types}.
% Level-$n$ tree grammars as defined by Damm correspond exactly to a
% subset of safe level-$n$ grammars -- namely the safe complete grammars
% -- and every safe grammar corresponds to a safe complete one.
A higher-order grammar (that is assumed to be \emph{homogeneously
  typed}) is said to be \emph{safe} if it obeys certain syntactic
conditions that constrain the occurrences of variables in the
production (or rewrite) rules according to their type-theoretic
order. Though the formal definition of safety is somewhat intricate,
the condition itself is manifestly important. As we survey in the
following, higher-order \emph{safe} grammars capture fundamental
structures in computation, offer clear algorithmic advantages, and
lend themselves to a number of compelling characterizations:

\begin{itemize}
\item \emph{Word languages}. Damm and Goerdt \cite{DG86} have shown
  that the word languages generated by order-$n$ \emph{safe} grammars
  form an infinite hierarchy as $n$ varies over the natural numbers.
  The hierarchy gives an attractive classification of the
  semi-decidable languages: Levels 0, 1 and 2 of the hierarchy are
  respectively the regular, context-free, and indexed languages (in
  the sense of Aho \cite{Aho68}), although little is known about
  higher orders.

  Remarkably, for generating word languages, order-$n$ \emph{safe}
  grammars are equivalent to order-$n$ pushdown automata \cite{DG86},
  which are in turn equivalent to order-$n$ indexed grammars
  \cite{Mas74,Mas76}.

\item \emph{Trees}. Knapik \emph{et al.} have shown that the Monadic
  Second Order (MSO) theories of trees generated by \emph{safe}
  (deterministic) grammars of every finite order are
  decidable\footnote{It has recently been shown
    \cite{OngLics2006} that trees generated by \emph{unsafe}
    deterministic grammars (of every finite order) also have decidable
    MSO theories. More precisely, the MSO theory of trees generated by order-$n$
recursion schemes is $n$-EXPTIME complete.}.

  They have also generalized the equi-expressivity result due to Damm
  and Goerdt \cite{DG86} to an equivalence result with respect to
  generating trees: A ranked tree is generated by an order-$n$ \emph{safe}
  grammar if and only if it is generated by an order-$n$ pushdown
  automaton.

\item \emph{Graphs}. Caucal \cite{Cau02} has shown that the MSO
  theories of graphs generated\footnote{These are precisely the
    configuration graphs of higher-order pushdown systems.} by
  \emph{safe} grammars of every finite order are decidable. In a recent preprint \cite{hague-sto07}, however,
  Hague \emph{et al.} have
  shown that the MSO theories of graphs generated by order-$n$
  \emph{unsafe} grammars are undecidable, but deciding their modal
  mu-calculus theories is $n$-EXPTIME complete.
\end{itemize}

\subsection*{Overview}

In this paper, we aim to understand the safety condition in the
setting of the lambda calculus. Our first task is to transpose it to
the lambda calculus and pin it down as an appropriate sub-system of
the simply-typed theory. A first version of the \emph{safe lambda
  calculus} has appeared in an unpublished technical report
\cite{safety-mirlong2004}. Here we propose a more general and cleaner
version where terms are no longer required to be homogeneously typed
(see Section~\ref{sec:safe} for a definition). The formation rules of
the calculus are designed to maintain a simple invariant: Variables
that occur free in a safe $\lambda$-term have orders no smaller than
that of the term itself.  We can now explain the sense in which the
safe lambda calculus is safe by establishing its salient property: No
variable capture can ever occur when substituting a safe term into
another. In other words, in the safe lambda calculus, it is
\emph{safe} to use capture-\emph{permitting} substitution when
performing $\beta$-reduction.


There is no need for new names when computing $\beta$-reductions of
safe $\lambda$-terms, because one can safely ``reuse'' variable names
in the input term. Safe lambda calculus is thus cheaper to compute in
this na\"ive sense. Intuitively one would expect the safety constraint
to lower the expressivity of the simply-typed lambda calculus. Our
next contribution is to give a precise measure of the expressivity
deficit of the safe lambda calculus. An old result of Schwichtenberg
\cite{citeulike:622637} says that the numeric functions representable
in the simply-typed lambda calculus are exactly the multivariate
polynomials \emph{extended with the conditional function}.  In the
same vein, we show that the numeric functions representable in the
safe lambda calculus are exactly the multivariate polynomials.

Our last contribution is to give a game-semantic account of the safe
lambda calculus.
% Not much is known about the safe $\lambda$-calculus, and many problems
% remain to be studied concerning its computational power, the
% complexity classes that it characterizes, its interpretation under the
% Curry-Howard isomorphism and its game-semantic characterization. This
% paper is a contribution to the last problem.
%
% The difficulty in giving a game-semantic account of safety lies in the
% fact that it is a syntactic restriction whereas game semantics is
% syntax-independent. The solution consists in finding a particular
% syntactic representation of terms on which the plays of the game
% denotation can be represented.  To achieve this, we use ideas recently
% introduced by the second author \cite{OngLics2006}: a term is
% canonically represented by a certain abstract syntax tree of its
% $\eta$-long normal form referred as the \emph{computation tree}. This
% abstract syntax tree is specially designed to establish a
% correspondence with the game arena of the term. A computation is
% described by a justified sequence of nodes of the computation tree
% respecting some formation rules and called a
% \emph{traversal}. Traversals permit us to model $\beta$-reductions
% without altering the structure of the computation tree via
% substitution. A notable property is that \emph{P-views} (in the
% game-semantic sense) of traversals corresponds to paths in the
% computation tree.  We show that traversals are just representations of
% the uncovering of plays of the game-semantic denotation. We then
% define a \emph{reduction} operation which eliminates traversal nodes
% that are ``internal'' to the computation, this implements the
% counterpart of the hiding operation of game semantics. Thus, we obtain
% an isomorphism between the strategy denotation of a term and the set
% of reductions of traversals of its computation tree.
Using a correspondence result relating the game semantics of a
$\lambda$-term $M$ to a set of \emph{traversals} \cite{OngLics2006}
over a certain abstract syntax tree of the $\eta$-long form of $M$
(called \emph{computation tree}), we show that safe terms are denoted
by \emph{P-incrementally justified strategies}. In such a strategy,
pointers emanating from the P-moves of a play are uniquely
reconstructible from the underlying sequence of moves and the pointers
associated to the O-moves therein: Specifically, a P-question always
points to the last pending O-question (in the P-view) of a greater
order. Consequently pointers in the game semantics of safe
$\lambda$-terms are only necessary from order 4 onwards. Finally we
prove that a $\eta$-long $\beta$-normal $\lambda$-term is \emph{safe}
if and only if its strategy denotation is (innocent and)
\emph{P-incrementally justified}.



% \subsection*{Related work}

% \noindent\emph{The safety condition for higher-order grammars}

% \smallskip

% \noindent We have mentioned the result of Knapik \emph{et al.}~\cite{KNU02} that
% infinite trees generated by \emph{safe} higher-order grammars have
% decidable MSO theories.  A natural question to ask is whether the
% \emph{safety condition} is really necessary.  This has then been
% partially answered by Aehlig \emph{et al.}
% \cite{DBLP:conf/tlca/AehligMO05} where it was shown that safety is not
% a requirement at level $2$ to guarantee MSO decidability. Also, for
% the restricted case of word languages, the same authors have shown
% \cite{DBLP:conf/fossacs/AehligMO05} that level $2$ safe higher-order
% grammars are as powerful as (non-deterministic) unsafe ones.  De
% Miranda's thesis \cite{demirandathesis} proposes a unified framework
% for the study of higher-order grammars and gives a detailed analysis
% of the safety constraint at level 2.

% More recently, one of us obtained a more general result and showed
% that the MSO theory of infinite trees generated by higher-order
% grammars of any level, \emph{whether safe or not}, is decidable
% \cite{OngLics2006}.  Using an argument based on innocent
% game-semantics, he establishes a correspondence between the tree
% generated by a higher-order grammar called \emph{value tree} and a
% certain regular tree called \emph{computation tree}. Paths in the
% value tree correspond to traversals in the computation tree.
% Decidability is then obtain by reducing the problem to the acceptance
% of the (annotated) computation tree by a certain alternating parity
% tree automaton.  The approach that we follow in
% Sec. \ref{sec:correspondence} uses many ingredients introduced in this
% paper.


% The equivalence of \emph{safe} higher-order grammars and higher-order
% deterministic push-down automata for the purpose of generating
% infinite trees \cite{KNU02} has its counterpart in the general (not
% necessarily safe) case: the forthcoming paper \cite{hague-sto07}
% establishes the equivalence of order-$n$ higher-order grammars and
% order-$n$ \emph{collapsible pushdown automata}. Those automata form a
% new kind of pushdown systems in which every stack symbol has a link to
% a stack situated somewhere below it and with an additional stack
% operation whose effect is to ``collapse'' a stack $s$ to the state
% indicated by the link from the top stack symbol.

% \medskip

% \noindent\emph{Computation trees and traversals}

% \smallskip

% \noindent In \cite{DBLP:conf/lics/AspertiDLR94}, a notion of graph
% based on Lamping's graphs \cite{lamping} is introduced to represent
% $\lambda$-terms. The authors unify different notions of paths
% (regular, legal, consistent and persistent paths) that have appeared
% in the literature as ways to implement graph-based reduction of
% $\lambda$-expressions. We can regard a traversal as an alternative
% notion of path adapted to the graph representation of
% $\lambda$-expressions given by computation trees.

% The traversals of a computation tree provide a way to perform
% \emph{local computation} of $\beta$-reductions as opposed to a global
% approach where the $\beta$-reduction is implemented by performing
% substitutions. A notion of local computation of $\beta$-reduction has
% been investigated by Danos and Regnier
% \cite{DanosRegnier-Localandasynchronou} through the use of special
% graphs called ``virtual nets'' that embed the lambda-calculus.


\section{The safe lambda calculus}
\label{sec:safe}
\subsection*{Higher-order safe grammars}
We first present the safety restriction as it was originally defined
\cite{KNU02}. We consider simple types generated by the grammar $A \,
::= \, o \; | \; A \typear A$. By convention, $\rightarrow$ associates
to the right. Thus every type can be written as $A_1 \typear \cdots
\typear A_n \typear o$, which we shall abbreviate to $(A_1, \cdots,
A_n, o)$ (in case $n = 0$, we identify $(o)$ with $o$). The
\emph{order} of a type is given by $\ord{o} = 0$ and $\ord{A \typear
  B} = \max(\ord{A}+1, \ord{B})$. We assume an infinite set of typed
variables. The order of a typed term or symbol is defined to be the
order of its type.

A (higher-order) \defname{grammar} is a tuple $\langle
\Sigma, \mathcal{N}, \mathcal{R}, S \rangle$, where $\Sigma$ is a
ranked alphabet (in the sense that each symbol $f \in \Sigma$ has an
arity $\mathit{ar}(f) \geq 0$) of \emph{terminals}\footnote{Each $f \in
  \Sigma$ of arity $r \geq 0$ is assumed to have type $(\underbrace{o,
    \cdots, o}_r, o)$.}; $\mathcal{N}$ is a finite set of typed
\emph{non-terminals}; $S$ is a distinguished ground-type symbol of
$\mathcal{N}$, called the start symbol; $\mathcal{R}$ is a finite set
of production (or rewrite) rules, one for each non-terminal $F : (A_1,
\ldots, A_n, o) \in \mathcal{N}$, of the form $ F z_1 \ldots z_m
\rightarrow e$ where each $z_i$ (called \emph{parameter}) is a
variable of type $A_i$ and $e$ is an applicative term of type $o$
generated from the typed symbols in $\Sigma \union \mathcal{N} \union \{z_1,
\ldots, z_m \}$. We say that the grammar is \emph{order-$n$} just in
case the order of the highest-order non-terminal is $n$.

The \defname{tree generated by a recursion scheme} $G$ is a possibly
infinite applicative term, but viewed as a $\Sigma$-labelled tree;
it is \emph{constructed from the terminals in $\Sigma$}, and is obtained by
unfolding the rewrite rules of $G$ \emph{ad infinitum}, replacing
formal by actual parameters each time, starting from the start symbol
$S$. See e.g.~\cite{KNU02} for a formal definition.

\pssetcomptree
\parpic[r]{
$\tree[levelsep=3ex,nodesep=1pt,treesep=1cm,linewidth=0.5pt]{g}
{  \TR{a}
    \tree{g}{\TR{a} \tree{h}{\tree{h}{\vdots}}}
}$
}
\begin{example}\rm\label{eg:running}
  Let $G$ be the following order-2 recursion scheme:
\[\begin{array}{rll}
  S & \rightarrow & H \, a\\
  H \, z^o & \rightarrow & F \, (g \,
  z)\\
  F \, \phi^{(o, o)} & \rightarrow & \phi \, (\phi \, (F \, h))\\
\end{array}\]
where the arities of the terminals $g, h, a$ are $2, 1, 0$ respectively.
The tree generated by $G$ is defined by the infinite term $g \, a \, (g \, a \, (h \, (h \, (h \,
\cdots))))$.%  The only infinite \emph{path} in the
% tree is the node-sequence $\epsilon \cdot 2 \cdot 22 \cdot 221 \cdot
% 2211 \cdots$.

%(with the corresponding \textbfit{trace} $g \, g \, h \, h \, h \,
%\cdots \; \in \; \Sigma^\omega$).
\end{example}

A type $(A_1, \cdots, A_n, o)$ is said to be \defname{homogeneous} if
$\ord{A_1} \geq \ord{A_2}\geq \cdots \geq \ord{A_n}$, and each $A_1$,
\ldots, $A_n$ is homogeneous \cite{KNU02}.  We reproduce the following
definition from \cite{KNU02}.

\begin{definition}[Safe grammar]\rm
  (All types are assumed to be homogeneous.) A term of order $k > 0$
  is \emph{unsafe} if it contains an occurrence of a parameter of
  order strictly less than $k$, otherwise the term is \emph{safe}. An
  occurrence of an unsafe term $t$ as a subexpression of a term $t'$
  is \emph{safe} if it is in the context $\cdots (ts) \cdots$,
  otherwise the occurrence is \emph{unsafe}. A grammar is
  \defname{safe} if no unsafe term has an unsafe occurrence at a
  right-hand side of any production.
%   A rewrite rule $F z_1 \ldots z_m \rightarrow e$ is said to be
%   \defname{unsafe} if the righthand term $e$ has a subterm $t$ such
%   that
% \begin{enumerate}[(i)]
% \item $t$ occurs in an {\em operand} ({\it i.e.}~second) position of some
%   occurrence of the implicit application operator {\it i.e.}~$e$ has the
%   form $\cdots (s \, t) \cdots $ for some $s$
% \item $t$ contains an occurrence of a parameter $z_i$ (say) whose
%   order is less than that of $t$.
% \end{enumerate}
% A homogeneous grammar is said to be \defname{safe} if none of its
% rewrite rules is unsafe.
\end{definition}

\begin{example}\begin{inparaenum}[(i)] \item Take $\; H : ((o, o), o), \; f : (o, o, o)$; the
    following rewrite rules are unsafe (in each case we underline the
    unsafe subterm that occurs unsafely):
\[\begin{array}{rll}
G^{(o, o)} \, x & \quad \rightarrow \quad & H \, \underline{(f \, {x})} \\
F^{((o, o), o, o, o)} \, z \, x \, y & \quad \rightarrow \quad & f \, (F \, \underline{(F \, z
\, {y})} \, y \, (z \, x) ) \, x
\end{array}\]
\item The order-2 grammar defined in Example~\ref{eg:running} is
  unsafe.
\end{inparaenum}
% The
% reader is referred to the literature
% \cite{KNU02,demirandathesis,safety-mirlong2004}
% for details about the safety restriction for higher-order grammars.
\end{example}

\subsection*{Safety adapted to the lambda calculus}
We assume a set $\Xi$ of higher-order constants.
We use sequents of the form $\Gamma \vdash_\Xi M : A$ to represent
terms-in-context where $\Gamma$ is the context and $A$ is the type of
$M$. For simplicity
we write $(A_1, \cdots, A_n, B)$ to mean $A_1 \typear \cdots \typear
A_n \typear B$, where $B$ is not necessarily ground.

\begin{definition}\rm
\begin{inparaenum}[(i)]
\item The \defname{safe lambda calculus} is a sub-system of the
  simply-typed lambda calculus defined by induction over the
  following rules:
$$ \rulename{var} \ \rulef{}{x : A\vdash_\Xi x : A} \quad
\rulename{const} \ \rulef{}{\vdash_\Xi f : A} \quad f \in \Xi \quad
\rulename{wk} \ \rulef{\Gamma \vdash_\Xi s : A}{\Delta \vdash_\Xi s : A} \quad
\Gamma \subset \Delta$$
$$ \rulename{app} \ \rulef{\Gamma \vdash_\Xi s : (A_1,\ldots,A_n,B) \
  \Gamma \vdash_\Xi t_1 : A_1 \; \ldots \; \Gamma \vdash_\Xi t_n : A_n
} {\Gamma \vdash_\Xi s t_1 \ldots t_n : B} \ \ord{B} \sqsubseteq
\ord{\Gamma}$$
$$ \rulename{abs} \ \rulef{\Gamma, x_1 : A_1, \ldots, x_n : A_n
  \vdash_\Xi s : B} {\Gamma \vdash_\Xi \lambda x_1 \ldots x_n . s :
  (A_1, \ldots ,A_n,B)} \ \ord{A_1, \ldots ,A_n,B} \sqsubseteq
\ord{\Gamma}$$ where $\ord{\Gamma}$ denotes the set $\{ \ord{y} : y
\in \Gamma \}$ and ``$c \sqsubseteq S$'' means that $c$ is a
lower-bound of the set $S$. For convenience, we shall omit the
subscript from $\vdash_\Xi$ whenever the generator-set $\Xi$ is clear from
the context.

\noindent \item The sub-system that is defined by the same rules in
(i), such that all types that occur in them are homogeneous, is called
the \defname{homogeneous safe lambda calculus}.
\end{inparaenum}
\end{definition}

The safe lambda calculus deviates from the standard definition of the simply-typed lambda calculus in a number of ways. First the rules $\rulename{app}$ and $\rulename{abs}$
respectively can perform multiple applications and abstract several
variables at once. (Of course this feature alone does not alter
expressivity.) Crucially, the side-conditions in the application rule
and abstraction rules require that variables in the typing context
have order no smaller than that of the term being formed.  We do not
impose any constraint on types. In particular, type-homogeneity as
used originally to define safe grammars \cite{KNU02} is not required
here. Another difference is that we allow $\Xi$-constants to have
arbitrary higher-order types.  % Thus our formulation
% of the safe lambda calculus is more general than the one proposed in
% the technical report \cite{safety-mirlong2004}. (It is possible to
% reconcile the two definitions by adding the further constraint that
% each type occurring in our rules is homogeneous and by restricting
% constants to at most order 1.)

\begin{example}[Kierstead terms]
\label{ex:kierstead}
Consider the terms $M_1 = \lambda f . f (\lambda x . f (\lambda y . y
))$ and $M_2 = \lambda f . f (\lambda x . f (\lambda y .x ))$ where
$x,y:o$ and $f:((o,o),o)$. The term $M_2$ is not safe because in the
subterm $f (\lambda y . x)$, the free variable $x$ has order $0$ which
is smaller than $\ord{\lambda y . x} = 1$.  On the other hand, $M_1$
is safe.
%On the other hand, $M_1$ is safe as the following proof tree shows:
%$$
% \rulef{
%     \rulef{
%        \rulef{}{f \vdash f} {\sf(var)}
%        \
%        \rulef{
%             \rulef{
%                \rulef{
%                    \rulef{}{f \vdash f} {\sf(var)}
%                }
%                {f , x \vdash f } {\sf(wk)}
%                \
%                \rulef{
%                    \rulef{
%                        \rulef{}{y \vdash y} {\sf(var)}
%                    }
%                    {y \vdash \lambda y . y } \rulenamet{abs}
%                }
%                {f , x \vdash \lambda y .y } {\sf(wk)}
%             }
%             {f , x \vdash f (\lambda y .y )} {\sf(app)}
%        }
%        { f  \vdash \lambda x . f (\lambda y .y )} \rulenamet{abs}
%     }
%     {
%        f  \vdash f (\lambda x . f (\lambda y .y ))} {\sf(app)}
%     }
% { \vdash M_1 = \lambda f . f (\lambda x . f (\lambda y .y )) } \rulenamet{abs}
%$$
\end{example}

It is easy to see that valid typing judgements of the safe lambda
calculus satisfy the following simple invariant:
\begin{lemma}
\label{lem:ordfreevar}
If $\Gamma \vdash M : A$ then every variable in $\Gamma$ occurring
free in $M$ has order at least $ord(M)$.
\end{lemma}


When restricted to the homogeneously-typed
sub-system, the safe lambda calculus captures the original notion
of safety due to Knapik \emph{et al.} in the context of higher-order
grammars:

\begin{proposition} Let $G = \langle \Sigma, \mathcal{N}, \mathcal{R},
  S \rangle$ be a grammar and let $e$ be an applicative term generated
  from the symbols in $\mathcal{N} \cup \Sigma \cup \makeset{z_1^{A_1},
    \cdots, z_m^{A_m}}$.  A rule $F z_1 \ldots z_m \rightarrow e$ in
  $\mathcal{R}$ is safe if and only if $ z_1 : A_1, \cdots, z_m : A_m
  \vdash_{\Sigma \cup \mathcal{N}} e : o$ is a valid typing judgement
  of the \emph{homogeneous} safe lambda calculus.
\end{proposition}

\emph{In what sense is the safe lambda calculus safe?} A basic idea
in the lambda calculus is that when performing $\beta$-reduction, one
must use capture-\emph{avoiding} substitution, which is standardly
implemented by renaming bound variables afresh upon each substitution.
In the safe lambda calculus, however, variable capture can never
happen (as the following lemma shows). Substitution can therefore be
implemented simply by capture-\emph{permitting} replacement, without
any need for variable renaming. In the following, we write
$M\captsubst{N}{x}$ to denote the capture-\emph{permitting}
substitution\footnote{This substitution is done by
textually replacing all free occurrences of $x$ in $M$ by $N$ without performing variable renaming.  In particular for the abstraction
  case we have
$(\lambda y_1\ldots y_n . M)\captsubst{N}{x} = \lambda y_1\ldots y_n . M\captsubst{N}{x}$ when $x\not\in
  \{ y_1\ldots y_n \}$.}
%\footnote{This substitution is implemented by textually
%  replacing all free occurrences of $x$ in $M$ by $N$ without
%  performing variable renaming.  In particular for the abstraction
%  case $(\lambda \overline{y} . P)\captsubst{N}{x}$ is defined as
%  $\lambda \overline{y} . P\captsubst{N}{x}$ if $x\not\in
%  \overline{y}$ and $\lambda \overline{y} . P$ elsewhere.}
of $N$ for $x$ in $M$.

\begin{lemma}[No variable capture]\label{lem:nvc}
\label{lem:homog_nocapture} There is
no variable capture when performing capture-permitting
substitution of $N$ for $x$ in $M$
provided that $\Gamma, x:B \vdash M : A$ and $\Gamma \vdash  N : B$ are valid judgments of the safe lambda calculus.
\end{lemma}

\proof
  We proceed by structural induction. The variable, constant and
  application cases are trivial. For the abstraction case, suppose $M = \lambda \overline{y}. R$ where $\overline{y} = y_1
  \ldots y_p$. If $x \in \overline{y}$ then $M \captsubst{N}{x} = M$ and there is no variable capture.

 If $x \not\in \overline{y}$ then we have $M \captsubst{N}{x} = \lambda \overline{y} . R \captsubst{N}{x}$.  By the induction hypothesis there is no variable capture in $R \captsubst{N}{x}$.  Thus variable capture can only happen if the following two conditions are met: $x$ occurs freely in $R$, and some variable $y_i$ for $1 \leq i \leq p$ occurs freely in $N$. By Lemma \ref{lem:ordfreevar}, the latter condition  implies $\ord{y_i} \geq \ord{N} = \ord{x}$.  Since $x \not \in \overline{y}$, the former condition implies that $x$ occurs freely in the safe term $\lambda \overline{y}. R$
  therefore Lemma \ref{lem:ordfreevar} gives $ \ord{x} \geq
  \ord{\lambda \overline{y} . R} \geq 1+ \ord{y_i} > \ord{y_i}$ which  gives a contradiction.
\qed


\begin{remark}
  A version of the No-variable-capture Lemma also holds in safe
  grammars, as is implicit in (for example Lemma 3.2 of) the original
  paper \cite{KNU02}.
\end{remark}

\begin{example}
  In order to contract the $\beta$-redex in the term
\[f:(o,o,o),x:o
  \vdash (\lambda \varphi^{(o,o)} x^o . \varphi \, x) (\underline{f \,
    x}) : (o,o)\] one should rename the bound variable $x$ with a fresh name to
  prevent the capture of the free occurrence of $x$ in the underlined term during substitution. Consequently, by the previous lemma,
  the term is not safe. Indeed, it cannot be because $\ord{x} = 0 < 1
  = \ord{f x}$.
\end{example}

Note that it is not the case that $\lambda$-terms
that satisfy the No-variable-capture Lemma are necessarily safe. For instance the $\beta$-redex in $\lambda y^o
z^o. (\lambda x^o .y) z$ can be contracted using capture-permitting
substitution, even though the term is not safe.

\subsection*{Reductions and transformations preserving safety}

From now on we will use the standard notation $M\subst{N}{x}$ to
denote the substitution of $N$ for $x$ in $M$.  It is understood that,
provided that $M$ and $N$ are safe, this substitution is
capture-permitting.


\begin{lemma}[Substitution preserves safety]
\label{lem:subst_preserve_safety}
If $\Gamma, x :B \vdash M : A$ and $\Gamma \vdash N : B$ then $\Gamma \vdash M[N/x] : A$.
\end{lemma}
This is proved by an easy induction on the structure of the safe term $M$.


It is desirable to have an appropriate notion of reduction for our
calculus. However the standard $\beta$-reduction rule is not
adequate. Indeed, safety is not preserved by $\beta$-reduction as the
following example shows. Suppose that $w,x,y,z : o$ and $f : (o,o,o)
\in \Sigma$ then the safe term $(\lambda x y . f x y) z w$
$\beta$-reduces to $(\underline{\lambda y . f z y}) w$ which is unsafe
since the underlined order-1 subterm contains a free occurrence of the
ground-type $z$. However if we perform one more reduction we obtain
the safe term $f z w$. This suggests an alternative notion of
reduction that performs simultaneous reduction of ``consecutive''
$\beta$-redexes. In order to define this reduction we first introduce
the appropriate notion of redex.

In the simply-typed lambda calculus a redex is a term of the form
$(\lambda x . M) N$. In the safe lambda calculus, a redex is a
succession of several standard redexes:

\begin{definition}\rm
Let $l\geq 1$ and $n\geq 1$. We use the abbreviations $\overline{x}$ and $\overline{x}:\overline{A}$  for $x_1 \ldots x_n$ and $x_1:A_1, \ldots, x_n : A_n$ respectively.

A \defname{safe redex} is a safe term of the form $(\lambda
\overline{x} . M) N_1 \ldots N_l$ such that the variables
$\overline{x}$ are abstracted altogether by one instance of the
\rulenamet{abs} rule (possibly followed by \rulenamet{wk}) and the
term $(\lambda \overline{x}.M)$ is applied to $N_1$, \ldots, $N_l$
by one instance of the \rulenamet{app} rule. Thus $M$, the and the
$N_i$'s are also safe.
\end{definition}
For instance, in the case $n<l$, a safe redex has a derivation tree of the following  form:
$$   \rulef{
            \rulef{\rulef{\rulef{\ldots}{\Gamma', \overline{x}:\overline{A} \vdash M : (A_{n+1}, \ldots, A_l, B)}}{\Gamma' \vdash \lambda \overline{x} . M : (A_1, \ldots, A_l, B)} \rulename{abs}}{\Gamma \vdash \lambda \overline{x} . M : (A_1, \ldots, A_l, B)}\rulename{wk}
            \quad
            \rulef{\ldots}{\Gamma \vdash N_1 :A_1}  \ \ldots \  \rulef{\ldots}{\Gamma \vdash N_l :A_l}
    }
    {
       \Gamma \vdash (\lambda \overline{x} . M) N_1 \ldots N_l : B
    } \rulename{app}
$$


We are now in a position to define a notion of reduction for safe terms.
\begin{definition}\rm
\label{dfn:safereduction} We use the
abbreviations $\overline{x} = x_1 \ldots x_n$,
$\overline{N} = N_1 \ldots N_l$.
The relation $\beta_s$ is defined on the set of safe redexes as:
\begin{eqnarray*}
  \beta_s &=&
  \{  \ (\lambda \overline{x} . M) N_1 \ldots N_l \mapsto \lambda x_{l+1} \ldots x_n. M\subst{\overline{N}}{x_1 \ldots x_l} \mbox{, for $n> l$}
  \} \\
  &\cup&
  \{ \ (\lambda \overline{x}  . M) N_1 \ldots N_l \mapsto M\subst{N_1 \ldots N_n}{\overline{x}} N_{n+1} \ldots N_l
  \mbox{, for $n\leq l$} \} \ .
\end{eqnarray*}
where $M\subst{R_1 \ldots R_k}{z_1 \ldots z_k}$ denotes the simultaneous substitution in $M$ of $R_1$,\ldots,$R_k$ for $z_1, \ldots, z_k$.  The
\defname{safe $\beta$-reduction}, written $\betasred$, is the
compatible closure of the relation $\beta_s$ with respect to the
formation rules of the safe lambda calculus.
\end{definition}

\noindent \emph{Remark:} The $\beta_s$-reduction is a multi-step
$\beta$-reduction \ie it is a subset of the transitive closure of $\betared$.


\begin{lemma}[$\beta_s$-reduction preserves safety]
\label{lem:safered_preserve_safety}
If $\Gamma \vdash s :A$ and $s \betasred t$ then $\Gamma \vdash t :A$.
\end{lemma}

\proof
  It suffices to show that the relation $\beta_s$ preserves safety.
Suppose that $s\ \beta_s\ t$ where $s$ is the
safe-redex $(\lambda x_1 \ldots x_n . M) N_1
  \ldots N_l $ with $x_1 : B_1, \ldots, x_n: B_n$
and $M$ of type $C$.  W.l.o.g we can assume that the last rule used
to form the term $s$ is \rulenamet{app} (and not the weakening rule
\rulenamet{wk}, thus  we have $\Gamma = fv(s)$.

Suppose $n>l$ then $A = (B_{l+1}, \ldots, B_n, C)$. By Lemma \ref{lem:subst_preserve_safety} we can form the safe term %\begin{equation}
$\Gamma, x_{l+1}:B_{l+1}, \ldots x_n :B_{n}\vdash M\subst{\overline{N}}{x_1 \ldots x_l} : C$. %\label{jud:substsafe}\ .
%\end{equation}
By Lemma \ref{lem:ordfreevar}, since $s$ is safe, all the variables
in $\Gamma$ have order $\geq \ord{A}$. This ensures that the
side-condition of the \rulenamet{abs} rule is verified when
abstracting the variables $x_{l+1} \ldots x_n$, which gives us the
judgement $\Gamma \vdash t :A$.

Suppose $n \leq l$. The substitution lemma gives
$\Gamma \vdash M\subst{N_1 \ldots N_n}{\overline{x}} : C$ and using \rulenamet{app} we form $\Gamma \vdash t :A$.
  \qed


In general, safety is not preserved by $\eta$-expansion; for instance we have
% $f:o,o \vdash f$ but $f:o,o \not \vdash \lambda x^o . f x$.
%This remark remains true for closed terms, for instance
$\vdash \lambda y^o z^o . y : (o,o,o)$ but
$\not \vdash \lambda x^o . (\lambda y^o z^o . y) x : (o,o,o)$.
However safety is preserved by $\eta$-reduction:

\begin{lemma}[$\eta$-reduction preserves safety]
  $\Gamma \vdash \lambda \varphi . s \varphi :A $ with $\varphi$ not
  occurring free in $s$ implies $\Gamma \vdash s :A$.
\end{lemma}
\proof
  Suppose $\Gamma \vdash \lambda \varphi . s \varphi :A$. If $s$ is an  abstraction then by construction of the safe term $\lambda \varphi . s \varphi$, $s$ is necessarily safe.  If $s = N_0 \ldots N_p$ with
  $p\geq 1$ then again, since $\lambda \varphi . N_0 \ldots N_p
  \varphi$ is safe, each of the $N_i$ is safe for $0 \leq i \leq p$
  and for any $z\in fv(\lambda \varphi . s \varphi)$, $\ord{z} \geq
  \ord{\lambda \varphi . s \varphi} = \ord{s}$. Since  $\varphi$ does not occur free in $s$ we have $fv(s) = fv(\lambda \varphi . s \varphi)$, thus we can use the application rule to form $fv(s) \vdash N_0 \ldots N_p : A$. The weakening rules permits us to conclude $\Gamma \vdash s :A$. \qed



The $\eta$-long normal form (or simply $\eta$-long form) of a term
% (also called \emph{long reduced form}, \emph{$\eta$-normal form} and
% \emph{extensional form} in the literature
% \cite{DBLP:journals/tcs/JensenP76,DBLP:journals/tcs/Huet75,huet76})
is obtained by hereditarily $\eta$-expanding every subterm occurring
at an operand position. Formally the \defname{$\eta$-long form}
$\elnf{t}$ of a term $t: (A_1,\ldots,A_n,o)$ with $n \geq 0$ is
defined by cases according to the syntactic shape of $t$:
\begin{eqnarray*}
  \elnf{\lambda x . s } &=& \lambda x . \elnf{s} \\
  \elnf{x s_1 \ldots s_m } &=& \lambda \overline{\varphi} . x \elnf{s_1}\ldots \elnf{s_m} \elnf{\varphi_1} \ldots \elnf{\varphi_n} \\
  \elnf{(\lambda x . s) s_1 \ldots s_p } &=& \lambda \overline{\varphi} . (\lambda x . \elnf{s}) \elnf{s_1} \ldots \elnf{s_p} \elnf{\varphi_1} \ldots \elnf{\varphi_n}
\end{eqnarray*}
where $m \geq 0$, $p\geq 1$, $x$ is a  variable or constant, $\overline{\varphi} = \varphi_1 \ldots \varphi_n$ and each $\varphi_i : A_i$ is a fresh variable.

%\begin{remark}
%  Converting a term to its $\eta$-long normal form does not introduce
%  new redex therefore the $\eta$-long normal form of $\beta$-normal
%  term is a $\beta$-normal term.
%\end{remark}

\begin{lemma}[$\eta$-long normalization preserves safety]
\label{lem:elnf_preserves_safety}
If $\Gamma \vdash s :A$ then $\Gamma \vdash \elnf{s} :A$.
\end{lemma}
\proof

 First we observe that for any variable or constant $x:A$ we have $x:A \vdash \elnf{x} :A$. We show this by induction on $\ord{x}$.
It is verified for any ground type variable $x$
since $x = \elnf{x}$.
Step case: $x:A$ with $A=(A_1, \ldots, A_n,o)$ and $n>0$. Let $\varphi_i:A_i$ be fresh variables for $1\leq i\leq n$.
Since $\ord{A_i} < \ord{x}$ the induction hypothesis gives $\varphi_i :A_i \vdash \elnf{\varphi_i} : A_i$. Using \rulenamet{wk} we obtain $x:A, \overline{\varphi} : \overline{A}
  \vdash \elnf{\varphi_i} :A_i$.  The application rule gives $x :A, \overline{\varphi} : \overline{A} \vdash x \elnf{\varphi_1} \ldots \elnf{\varphi_n}
  : o$ and the abstraction rule gives $ x :A \vdash \lambda
  \overline{\varphi} . x \elnf{\varphi_1} \ldots \elnf{\varphi_n} =
  \elnf{x} :A$.


We now prove the lemma by induction on $s$.
The base case is covered by the previous observation.
\emph{Step case:}
\begin{compactitem}
\item $s = x s_1 \ldots s_m$ with $x: (B_1, \ldots, B_m, A)$, $A = (A_1, \ldots, A_n, o)$ for some $m\geq 0$, $n>0$ and $s_i : B_i$ for $1 \leq i \leq
  m$.  Let $\varphi_i: A_i$ be fresh variables for $1\leq i \leq
  n$. By the previous observation we have $\varphi_i :A_i \vdash \elnf{\varphi_i} :A_i$, the weakening rule then gives us $\Gamma , \overline{\varphi} : \overline{A}
  \vdash \elnf{\varphi_i} : A_i$.  Since the judgement
  $\Gamma \vdash x s_1 \ldots s_m : A$ is formed using the \rulenamet{app} rule, each $s_j$ must be safe for $1\leq j \leq m$, thus by the induction hypothesis we have $\Gamma \vdash \elnf{s_j} : B_j$ and by weakening we get $\Gamma, \overline{\varphi} :\overline{A} \vdash \elnf{s_j} : B_j$.  The \rulenamet{app}
  rule then gives $\Gamma, \overline{\varphi} :\overline{A} \vdash x \elnf{s_1} \ldots \elnf{s_m} \elnf{\varphi_1} \ldots \elnf{\varphi_n} : o$. Finally
  the \rulenamet{abs} rule gives $\Gamma \vdash \lambda \overline{\varphi} . x
  \elnf{s_1} \ldots \elnf{s_m} \elnf{\varphi_1} \ldots
  \elnf{\varphi_n} = \elnf{s} : A$, the side-condition of \rulenamet{abs} being verified since $\ord{\elnf{s}} = \ord{s}$.


\item $s = t s_0 \ldots s_m$ where $t$ is an abstraction.
For some fresh variables $\varphi_1$, \ldots, $\varphi_n$
we have $\elnf{s} = \lambda \overline{\varphi}. \elnf{t} \elnf{s_0} \ldots \elnf{s_m} \elnf{\varphi_1}
  \ldots \elnf{\varphi_n}$. Again, using the induction hypothesis we can easily derive $\Gamma \vdash
 \lambda \overline{\varphi}. \elnf{t} \elnf{s_0} \ldots \elnf{s_m} \elnf{\varphi_1} \ldots \elnf{\varphi_n} : A$.

\item $s = \lambda \overline{\eta} . t $ where
$\overline{\eta} : \overline{B}$ and $t:C$ is not an abstraction. The induction hypothesis gives $\Gamma,
  \overline{\eta} : \overline{B} \vdash \elnf{t} : C$ and using
\rulenamet{abs} we get $\Gamma \vdash \lambda \overline{\eta} . \elnf{t} = \elnf{s} : A$.  \qed
\end{compactitem}


Note that the converse does not hold in general, for instance $\lambda
x^o . f^{(o,o,o)} x^o$ is unsafe although $\elnf{\lambda x . f x} =
\lambda x^o y^o . f x y$ is safe.


%\notetoself{Check and prove the following lemma}
% For terms with homogeneous types however, the converse does hold:
%\begin{lemma}
%If $\Gamma \vdash \elnf{s} : T$ is homogeneously safe (i.e. it is a
%safe judgement of the safe $\lambda$-calculus and each sequent
%occurring at the nodes of the proof tree is homogeneously typed)
%then $\Gamma \vdash s :T$ is homogeneously safe.
%\end{lemma}

    \newcommand\bigo{\mathcal{O}} % big O notation
\newcommand\booltype{\mathbf{B}}

\section{Complexity of the Safe Lambda Calculus}
Here we study the problem of deciding beta-eta equivalence of two safe lambda terms.

\subsection{Statman's result}

A famous result by Statman  states that deciding the $\beta\eta$-equality of two first-order typable lambda terms is not elementary recursive \cite{Statman:1979:TLE}.
The idea of the proof is to encode the Henkin quantifier elimination of Type Theory into the simply-typed lambda calculus. The encoding relies on the fact that the function $\sf sg$ (conditional) can be encoded in the lambda-calculus. Hence the argument does not carry on   in the Safe Lambda Calculus since the conditional operator is not definable (\cite{blumong:safelambdacalculus}).

Mairson gave a simpler proof of Statman's theorem in \cite{mairson1992spt} which also proceeds by encoding the Henkin quantifier elimination procedure into the lambda-calculus but is much easier to understand as it makes use of list iteration to perform quantifier elimination.

It turns out that both encodings rely on the use of unsafe terms in order to implement the quantifier elimination procedure.

%Here we adapt Mairson's proof to produce a safe encoding of the quantifier elimination procedure, thus showing:
%\begin{theorem}
%The Safe Lambda Calculus is not elementary recursive.
%\end{theorem}

We recall the definition of the theory. Let $\mathcal{D}_0 = \{\mathbf{true},\mathbf{false}\}$ and $\mathcal{D}_{k+1} =powerset(\mathcal{D}_k)$.
For any $k\geq0$, we write $x^k$, $y^k$ and $z^k$ to denote variables ranging over $\mathcal{D}_k$. Prime formulas are $x^0$, $\mathbf{true}\in y^1$, $\mathbf{false}\in y^1$, and  $x^k \in y^{k+1}$. Formulae are built up from prime formulas using the logical connectives $\zand$,$\zor$,$\rightarrow$,$\neg$ and the quantifiers
$\forall$ and $\exists$. Meyer showed that deciding the truth of such formulae requires nonelementary time \cite{Meyer1974}.
\smallskip

In Mairson's encoding, all formula variables of a given order $k$ are encoded by terms of the same type $\Delta_k$. Using this encoding,
unsafety manifests itself in two different ways.
\begin{enumerate}[1.]
  \item
        First in the encoding of set membership. The prime formula $x^k \in y^{k+1}$ is encoded as \begin{equation} x^k : \Delta_k, y^{k+1}:\Delta_{k+1} \vdash y^{k+1} (\lambda y^k : \Delta_k . OR (eq_k~\underline{x^k}~y^k)~F : \Delta_k \typear \Delta_{k+1} \typear \Delta_0 \label{eqn:setmembership}\end{equation}
for some terms $OR$, $F$, $eq_k$.
This term is unsafe because of the underline occurrence of $x^k$ which is not abstracted together with $y^k$.

\item Secondly, quantifier elimination is performed by using a list iterator $\mathbf{D}_{k+1}$ which acts like the $fold\_left$ function from functional programming languages over the list of all elements of $\mathcal{D}_k$.
Thus for instance the formula $\forall x^0 . \exists y^0 . x^0 \zor y^0$
is encoded as $$\vdash \mathbf{D}_1 (\lambda x^0:\Delta_0. AND (\mathbf{D}_1 (\lambda y^0:\Delta_0. OR (\underline{x^0} \zor y^0)) F)) T$$ which is unsafe because of the underlined occurrence.

More generally, supposing that we find a way to encode set membership with a safe term, then the encoding of the formula will be safe if and only if for any variable $x$ in the formula, its binder is precisely the first quantifier $\exists z$ or $\forall z$ in the path to the root of the formula AST verifying $\ord{z} \geq \ord{x}$. For instance the formula $\forall x^k . \exists y^{k+1} . x^k \in y^{k+1}$ would be encoded by an unsafe term whereas the encoding of $\forall y^{k+1} . \exists x^k . x^k \in y^{k+1}$ would be safe.
\end{enumerate}

Surprisingly, the unsafety of the quantifier elimination procedure can be
easily overcome. The idea is as follows. We introduce multiple domains of representation for a given formula. An element of $\mathcal{D}_k$ is thereby represented by countably many terms of type $\Delta_k^n$ where $n\in\nat$ indicates the level of the representation. The type $\Delta_k^n$ is defined in such a way that its order strictly increases as $n$ grows. Moreover there exists a term that can reduce the level of representation of a given term. In the formula representation, each variable can now be encoded with a different level of representation. Since there are infinitely many levels, it is always possible to find an assignment of levels to variables such that the resulting encoding term is safe.

For set-membership, however, there is no obvious way to obtain a safe encoding. The set-member function from Eq.\ \ref{eqn:setmembership} can be turned into a safe term provided that we have access to a function permitting us to increase the representation level of term, but to our knowledge, such transformation cannot be expressed in the simply-typed lambda-calculus.



\subsubsection{Encoding basic boolean operations}

We assume a ground type $o$.
%For any type $\mu$ we define the type $\booltype_\mu \equiv (\mu\typear\mu)\typear \mu$.
%We abbreviate $\booltype_0$ into $\booltype$.
%We introduce the following hierarchy of types: $\sigma_0 \equiv o$, $\sigma_{n+1} \equiv \booltype_{\sigma_n}$ for $n\geq1$.
%Note that the order of $\sigma_n$ strictly increases as $n$ increases.
We introduce a parameterized type for encoding booleans defined by $\booltype_{-1} \equiv o$ and $\booltype_{n+1} \equiv \booltype_n\typear\booltype_n\typear\booltype_n$ for $n\geq0$.
We have $\ord{\booltype_n} = n+1$ for $n\geq-1$.


The representation of the truth values $\mathbf{true}$ and $\mathbf{false}$ will be parameterized by $n \in \nat$ as follows
\begin{align*}
  T^n &\equiv \lambda x^{\booltype_{n-1}} y^{\booltype_{n-1}} .x : \booltype_{n}\\
  F^n &\equiv \lambda x^{\booltype_{n-1}} y^{\booltype_{n-1}} .y : \booltype_{n}
\end{align*}
Clearly these terms as safe. Moreover the following relations hold for all $n$:
\begin{align*}
  T^{n+1}~T^n~F^n &\betared^*  T^n \\
  F^{n+1}~T^n~F^n &\betared^*  F^n
\end{align*}
Hence it is possible to lower the representation level of a term encoding a boolean value by applying the two terms $T^n$ and $F^n$ to it.
For $i\in\nat$, we define the function $\_ \downarrow_i$ that lowers the level-representation of a term, turning a term of type $\booltype_n$ for some $n\in\nat$ to a term of type $\booltype_{\min(i,l)}$:
$$ (M :\booltype_n)\downarrow_i = \left\{
  \begin{array}{ll}
    M~T^{n-1}~F^{n-1}~\ldots~T^{i+1}~F^{i+1}:\booltype_i, & \hbox{if $n>i$;} \\
M:\booltype_n, & \hbox{otherwise.}
  \end{array}
  \right.
$$


Boolean functions are encoded by the following level-parameterized terms:
\begin{align*}
AND^n &\equiv \lambda p : \booltype_n \lambda q : \booltype_n \lambda x:\booltype_{n-1} \lambda y:\booltype_{n-1} . p~(q~x~y)~y : \booltype_n\typear\booltype_n\typear\booltype_n \\
OR^n &\equiv \lambda p : \booltype_n \lambda q : \booltype_n \lambda x:\booltype_{n-1} \lambda y:\booltype_{n-1} . p~x~(q~x~y) : \booltype_n\typear\booltype_{n}\typear\booltype_n \\
NOT^n &\equiv \lambda p : \booltype_n \lambda q : \booltype_n \lambda x:\booltype_{n-1} \lambda y:\booltype_{n-1} . p~y~x : \booltype_n\typear\booltype_n\typear\booltype_n \\
IF^n &\equiv \lambda p : \booltype_n \lambda q : \booltype_n \lambda x:\booltype_{n-1} \lambda y:\booltype_{n-1} . OR^n (NOT^n p)~q : \booltype_n\typear\booltype_n\typear\booltype_n
\end{align*}
which are all safe terms.

\subsubsection{Coding elements of the type hierarchy}
For any $n\in\nat$ we define the hierarchy of type $\Delta_k^n$ as follows:
$\Delta_0^n \equiv \booltype_n$ and $\Delta_{k+1}^n \equiv {\Delta_k^n}^*$ where for any type $\alpha$, $\alpha^* = (\alpha \typear \tau \typear \tau)\typear \tau \typear \tau$.

An occurrence of a formula variable $x^k$ will be encoded as a term variable $x^k:\Delta_{k}^n$ for some level of representation $n\in\nat$.

Following Mairson's  proof, we encode the set $\mathcal{D}_0$ as the list $\mathbf{D}_0$ containing $\mathbf{true}$ and $\mathbf{false}$, and we parameterized this representation by $n\in \nat$:
$$\mathbf{D}_0^n \equiv \lambda c:\booltype_n \typear \tau \typear \tau . \lambda e : \tau . c~T^n~(c~F^n~e) : \Delta_1^n$$
and for $k\geq 0$, the higher-order set $\mathcal{D}_{k+1}$ is represented by the parameterized term:
$$\mathbf{D}_{k+1}^n \equiv powerset~\mathbf{D}_k^n : \Delta_{k+2}^n$$
where the term $powerset$ taken from \cite{mairson1992spt} is reproduced here:
\begin{align*}
  powerset &\equiv \lambda A^* :(\alpha \typear \alpha^{**} \typear \alpha^{**}) \typear \alpha^{**} \typear \alpha^{**}.\\
&\qquad  A^*~double~(\lambda c:\alpha^* \typear \tau\typear \tau.\lambda b:\tau . c ( \lambda c':\alpha\typear \tau\typear \tau. \lambda b':\tau.b') b)\\
powerset &: ((\alpha \typear \alpha^{**} \typear \alpha^{**}) \typear \alpha^{**} \typear \alpha^{**})\typear \alpha^{**}
\end{align*}
with
\begin{align*}
  double &\equiv \lambda x :\alpha.\lambda l : (\alpha^* \typear \tau\typear \tau)\typear \tau\typear \tau. \\
  & \qquad \lambda c:\alpha^*\typear \tau\typear \tau.\lambda b:\tau. \\
  & \qquad \qquad l(\lambda e:\alpha^*.c (\lambda c':\alpha\typear \tau\typear \tau.\lambda b':\tau.c'~x~(e~c'~b')))(l~c~b)\\
double &: \alpha \typear \alpha^{**} \typear \alpha^{**}
\end{align*}

It can be checked that these two terms are safe.

\subsubsection{Quantifier elimination}
Following \cite{mairson1992spt}, quantifier elimination interprets $\forall x^k.\Phi(x^k)$ as the iterated conjunction $\mathbf{D}_k(\lambda x^k:\Delta^k.AND(\hat\Phi~x^k))~T$ where $\hat\Phi$ is the interpretation of $\Phi$; similarly $\exists x^k.\Phi(x^k)$  is interpreted by the iterated disjunction $\mathbf{D}_k(\lambda x^k:\Delta^k.AND(\hat\Phi~x^k))~T$.

Let $x^{k_p}_p \ldots x^{k_1}_1$ for $p\geq1$ be the list of variables appearing in the formula. W.l.o.g.\ we can assume that they are given in the order of appearance of their binder in the formula \ie $x^{k_p}_p$ is bound by the leftmost binder. We assign representation levels to variables as follows. The right-most variable is assigned level $1$ \ie $x^{k_1}_1 : \Delta^1_{k_1}$; suppose that $x^{k_i}_i :\Delta^l_{k_l}$ for $1\leq i< p$ then the representation level of variable $x^{k_{i+1}}_{i+1}$ is defined as
the smallest $l'\in\nat$ such that $\ord{\Delta^{l'}_{k_i}} > \ord{\Delta^{l}_{k_{i-1}}}$.

This way, since variables that are bound first have higher order, the variables
 that are bound in the nested list-iterations (corresponding to the nested quantifiers in the formula) are necessarily safely bound.


\subsubsection{Coding set theory in the $\mathcal{D}_k$}
To complete the interpretation of prime formulas, we would need to show how to encode set membership. Unfortunately, this seems to be impossible in the safe lambda calculus. It would turn to be possible if we had at hand a function $\_ \uparrow^k$, counterpart of $\_ \downarrow_k$, that increases the representation level of a term to level $k$. Here is how we would proceed if such function were representable in the safe lambda calculus.

Firstly, the formulae ``$\mathbf{true} \in y^1$'' and ``$\mathbf{false} \in y^1$'' can be encoded by the safe terms $y^1 (\lambda x^0 . OR^0~x^0) F^0$ and $y^1 (\lambda x^0. OR^0(NOT^0~x^0)) F^0$ respectively.
For the general case ``$x^k\in y^{k+1}$''
we proceed as in \cite{mairson1992spt} by introducing lambda-terms encoding set equality, set membership and subset tests, and we further parameterize these encoding by $n\in\nat$.

Equality of booleans is encoded by:
$$ eq_0^n \equiv \lambda x^0 : \booltype_n .\lambda y^0 : \booltype_n. OR^n (AND^n~x^0~y^0) (AND^n (NOT^n~x^0)(NOT^n~y^0)) \ .$$

We now use variable of type $\Delta_{k+1}^n$ as iterators over list of elements of type $\Delta_k^n$ and we instantiate the type variable $\tau$ as $\booltype_n$ in order to iterate a level-$n$ Boolean function. We define the set membership function as follows. Note that
the level of representation differs from input to output: \begin{align*}
  member_{k+1}^{n+1} &\equiv \lambda x^k : \Delta_k^{n+1}.\lambda y^{k+1}:\Delta_{k+1}^{n+1}. \\
& \qquad (y^{k+1}\downarrow_n) (\lambda y^k : \Delta_k^n . OR^n (eq_k^{n+1}~x^k~(y^k\uparrow^{n+1})))~F^n \\
  & : \Delta_k^{n+1} \typear \Delta_{k+1}^{n+1} \typear \booltype_n
\\
  subset_{k+1}^{n+1} &\equiv \lambda x^{k+1} : \Delta_{k+1}^{n+1}.\lambda y^{k+1}:\Delta_{k+1}^{n+1}. \\
  & \qquad (x^{k+1}\downarrow_n) (\lambda x^k : \Delta_k^n . AND^n (member_{k+1}^{n+1}~x^k~y^{k+1}))~T^n \\
  & : \Delta_{k+1}^{n+1} \typear \Delta_{k+1}^{n+1} \typear\booltype_n
\\
  eq_{k+1}^{n+1} &\equiv \lambda x^{k+1} : \Delta_{k+1}^{n+1}.\lambda y^{k+1}:\Delta_{k+1}^{n+1}. \\
   & \qquad
   (\lambda op:\Delta_{k+1}^n\typear\Delta_{k+1}^n\typear\booltype_n. AND^n (op~x^{k+1}~y^{k+1})(op~y^{k+1}~x^{k+1}))~subset_{k+1}^{n+1} \\
  & : \Delta_{k+1}^{n+1} \typear \Delta_{k+1}^{n+1} \typear \booltype_n
\end{align*}
The terms $eq_{k+1}^n$ and $subset_{k+1}^n$ are safe, and so is $member_{k+1}^n$ thanks to the fact that $y^k$ has a lower representation level than $x^k$.

The formula $x^k\in y^{k+1}$ is then encoded by the term
$$x^k:\Delta_k^n, y^{k+1}:\Delta_{k+1}^{n'}\vdash \left(member_{k+1}^{\min(n,n')} (x^k\downarrow_{\min(n,n')})~(y^{k+1}\downarrow_{\min(n,n')})\right)\downarrow_0$$


\subsection{NP-hardness}
To show NP-hardness it suffices to observe that the encoding of SAT in the simply-typed lambda calculus from the paper\cite{asperti-np} relies only on safe terms.

\subsection{PSPACE-hardness}

We encode QBF into the calculus.
We assume that the quantified propositional formula is given in prenex form:
$$\$_{n-1} x_{n-1} \ldots \$_0 x_0 . \psi(x_0, \ldots, x_{n-1})$$
where $\$_i \in \{\exists,\forall\}$ for $0\leq i\leq n-1$.

The encoding is as follows:
\begin{align*}
\sem{1} &= T^0  : \booltype \\
\sem{0} &= F^0 : \booltype \\
\sem{x_i} &= x_i\downarrow_0 = x_i~T^{i-1}~F^{i-1}\ldots T^1~F^1: \booltype \qquad \hbox{where $x_i:\booltype_i$}\\
\sem{\psi_1\zand \psi_2} &= AND^0~\sem{\psi_1}~\sem{\psi_2}
:\booltype  \\
\sem{\psi_1\zor \psi_2} &= OR^0~\sem{\psi_1}~\sem{\psi_2}
:\booltype  \\
\sem{\neg \psi} &= NOT^0~\sem{\psi}
:\booltype  \\
\sem{\forall x_i.\psi(\ldots, x_i, \ldots)} & = \mathbf{D}_0^i(\lambda x^{\booltype_i} AND^0~\sem{\psi(\ldots, x_i, \ldots)})~T^0 :\booltype\\
\sem{\exists x_i.\psi(\ldots, x_i, \ldots)} & = \mathbf{D}_0^i(\lambda x^{\booltype_i}.OR^0~\sem{\psi(\ldots, x_i, \ldots)})~F^0 :\booltype
\end{align*}
The size of $\sem{\psi}$ is in $\bigo(|\psi|^2)$.

It is easy to check that this encoding is safe.
\begin{example}
  The formula $\forall x \exists y \exists z (x\zor y\zor z)\zand(\neg x\zor \neg y\zor \neg z)$ is represented by the safe term:
\begin{align*}
\vdash &\mathbf{D}_0^2(\lambda x^{\booltype_2}. AND^0\\
&\quad\quad (\mathbf{D}_0^1(\lambda x^{\booltype_1}.OR^0\\
&\quad\quad\quad (\mathbf{D}_0^0(\lambda x^{\booltype_0}.OR^0\\
&\quad\quad\quad\quad (AND^0 (OR^0(OR^0~(x~T^1 F^1 T^0 F^0)~(y~T^0 F^0))z) \\
&\quad\quad\quad\quad\quad (OR^0(OR^0(NEG^0 (x~T^1 F^1 T^0 F^0))(NEG^0 (y~T^0 F^0)))(NEG^0~z))) \\
&\quad\quad\quad )F^0)\\
&\quad\quad)F^0)\\
&\quad) T^0
\end{align*}
\end{example}
This gives us:
\begin{theorem}
  Deciding $\beta\eta$-equality of two terms of the Safe Lambda Calculus is PSPACE-hard.
\end{theorem}

% NP \subseteq PSPACE \subseteq EXP


    \section{Expressivity}
\subsection{Numeric functions representable in the safe lambda
calculus}

Natural numbers can be encoded into the simply-typed lambda calculus
using the Church Numerals: each $n\in\nat$ is encoded into the term
$\encode{n} = \lambda s z. s^n z$ of type $I = ((o,o),o,o)$ where
$o$ is a ground type. In 1976 Schwichtenberg \cite{citeulike:622637}
showed the following:


\begin{theorem}[Schwichtenberg 1976]
The numeric functions representable by simply-typed $\lambda$-terms
of type $I\rightarrow \ldots \rightarrow I$ using the Church Numeral
encoding are exactly the multivariate polynomials \emph{extended
with the conditional function}.
\end{theorem}

If we restrict ourselves to safe terms, the representable functions
are exactly the multivariate polynomials:
\begin{theorem}
\label{thm:polychar} The functions representable by safe
$\lambda$-expressions of type $I\rightarrow \ldots \rightarrow I$
are exactly the multivariate polynomials.
\end{theorem}

\begin{corollary}
The conditional operator $C:I\rightarrow I\rightarrow I \rightarrow
I$ verifying  $C t y z \rightarrow_\beta y$  if $t \rightarrow_\beta
\encode{0}$ and $C t y z \rightarrow_\beta z$ if $t
\rightarrow_\beta \encode{n+1}$ is not definable in the safe
simply-typed lambda calculus.
\end{corollary}
\proof
  Natural numbers are encoded using Church Numerals: $\encode{n} =
  \lambda s z. s^n z$.  Addition: For $n,m \in \nat$, $\encode{n+m} =
  \lambda \alpha^{(o,o)} x^o . (\encode{n} \alpha) (\encode{m} \alpha
  x)$. Multiplication: $\encode{n . m} = \lambda \alpha^{(o,o)}
  . \encode{n} (\encode{m} \alpha)$.  All these terms are safe and
  clearly any multivariate polynomial $P(n_1, \ldots, n_k)$ can be
  computed by composing the addition and multiplication terms as
  appropriate.

For the converse, let $U$ be a safe $\lambda$-term of type
$I\rightarrow I\rightarrow I$.  The generalization to terms of type
$I^n \rightarrow I$ for $n>2$ is immediate (they correspond to
polynomials with $n$ variables). W.l.o.g we can assume that $U =
\lambda x y \alpha z. u$ where $u$ is a safe term of ground type in
$\beta$-normal form with $fv(u) \subseteq \{ x, y : I, z :o, \alpha
: o\rightarrow o \}$.

\emph{Notation:} Let $T$ be a set of terms of type $\tau \rightarrow
\tau$ and $T'$ be a set of terms of type $\tau$ then $T \cdot T'$
denotes the set of terms $\{ s s' : \tau \ | \ s \in T \wedge s' \in
T' \}$. We also define $T^k \cdot T'$ recursively as follows:  $T^0
\cdot T' = T'$ and for $k\geq 0$, $T^{k+1} \cdot T' = T \cdot (T^k
\cdot T')$ ({\it i.e.}~$T^k \cdot T'$ denotes $\{ s_1( \ldots (s_k
s'))  \ | \ s_1, \ldots, s_k \in T \wedge s' \in T' \}$). We define
$T^+\cdot T' = \Union_{k > 0} T^k \cdot T'$ and $T^*\cdot T' =
(T^+\cdot T') \union T'$. For two sets of terms $T$ and $T'$, we
write $T =_\beta T'$ to express that any term of $T$ is
$\beta$-convertible to some term $t'$ of $T'$ and reciprocally.

Let us write $\mathcal{N}^\tau$ for the set of $\beta$-normal terms
of type $\tau$ where $\tau$ ranges in $\{ o, o\rightarrow o, I \}$
and with free variables in $\{ x,y:I, z:o, \alpha:o\rightarrow o\}$.
We write $\mathcal{A}^\tau$ for the subset of $\mathcal{N}^\tau$
consisting of applications only ({\it i.e.}~not abstractions). Let
$B$ be the set of terms of type $(o,o)$ defined by $B = \{ \alpha \}
\union \{ \lambda a.b \ | \ b \in \{a,z\}, a \neq z \}$. It is easy
to see that the following equations hold:
\begin{eqnarray*}
\mathcal{A}^I &=& \{ x,y \} \\
\mathcal{N}^{(o,o)} &=& B \union \mathcal{A}^I \cdot
\mathcal{N}^{(o,o)} = (\mathcal{A}^I)^* \cdot B \\
\mathcal{A}^{(o,o)} &=& \{ \alpha \} \union (\mathcal{A}^I)^+ \cdot B \\
\mathcal{A}^o = \mathcal{N}^o &=& \{ z \} \union \mathcal{A}^{(o,o)} \cdot \mathcal{N}^o = (\mathcal{A}^{(o,o)})^* \cdot \{ z \}
\end{eqnarray*}
Hence $\mathcal{A}^o = \left( \{\alpha \} \union \{x,y\}^+ \cdot
\left( \{\alpha \} \union \{\lambda a.b \ | \ b \in \{a,z\}, a \neq
z \} \right) \right)^* \cdot \{ z \}$. Since $u$ is safe, it cannot
contain terms of the form $\lambda a . z$ with $a \neq z$ occurring
at an operand position, therefore since $u$ belongs to
$\mathcal{A}^o$ we have:
\begin{equation}
u \in \left( \{\alpha\} \union \{x,y\}^+ \cdot \{\alpha,
\underline{i} \} \right)^* \cdot \{ z \} \label{eqn:u}
\end{equation}
where $\underline{i}$ is the identity term of type $o\rightarrow o$.


We observe that $\encode{k} \underline{i} =_\beta \underline{i}$ for
all $k \in \nat$ and for $l\geq 1$, for all $k_1, \ldots k_l \in
\nat$, $\encode{k_1}\ldots \encode{k_l} \alpha =_\beta
\encode{k_1\times \ldots \times k_l} \alpha$. Hence for all $m,n \in
\nat$ we have:
\begin{equation}
\begin{array}{llr}
\{\encode{m},\encode{n}\}^+ \cdot \{\alpha, \underline{i} \} &=_\beta
\{ \underline{i} \} \union
\{ \encode{m^i n^j} \alpha \ |\ i+j \geq 1 \} \nonumber \\
&= \{ \encode{m^i n^j} \alpha \ |\ i,j \geq 0 \} & ( \mbox{since } \underline{i} = \encode{0} \alpha) \end{array}
\label{eqn:intermediate}
\end{equation}
therefore:
$$\begin{array}{llr}
u[\encode{m}, \encode{n}/x,y] &\in \left( \{ \alpha \} \union \{\encode{m},\encode{n}\}^+ \cdot \{\alpha, \underline{i} \} \right)^* \cdot \{ z \}  & \mbox{(by Eq.\ \ref{eqn:u})} \\
&=_\beta \left( \{\alpha \} \union \{ \encode{m^i n^j}
\alpha \ | \ i,j \geq 0 \} \right)^* \cdot \{ z \} & \mbox{(by Eq.\ \ref{eqn:intermediate})}  \\
&=_\beta \left\{ \encode{m^i n^j}
\alpha \ | \ i,j \geq 0 \right\}^* \cdot \{ z \} & \mbox{($\alpha z =_\beta \encode{1} \alpha z$)}.
\end{array}$$

Furthermore, for all $m,n,r,i,j\in \nat$ we have $\encode{m^i n^j}
\alpha (\alpha^r z) =_\beta \alpha^{r + m^i n^j} z$, hence
$u[\encode{m} \encode{n}/x,y] =_\beta \alpha^{p(m,n)} z$ where
$p(m,n) = \sum_{0\leq k \leq d} m^{i_k} n^{j_k}$ for some $i_k,j_k
\geq 0$, $k \in\{ 0,..,d \}$ and $d\geq 0$. Thus $U \encode{m}
\encode{n} =_\beta \encode{p(m,n)}$. \qed


For instance, the term $ C = \lambda F G H \alpha x . H (
\underline{\lambda y . G \alpha x} ) (F \alpha x)$ used by
Schwichtenberg \cite{citeulike:622637} to define the conditional
operator is unsafe since the underlined subterm is of order $1$,
occurs at an operand position and contains an occurrence of $x$ of
order $0$.



    %% chapter from the transfer thesis
    \input{transfer_chap_safe_homog.texi}
    \section{Safe $\lambda$-Calculus without the Homogeneity Constraint}
\label{sec:safe_nonhomog}


In section \ref{sec:safe_homog}, we have presented a version of the
safe lambda calculus where types are required to be homogeneous. We
now give a more general version of the safe simply-typed
$\lambda$-calculus where type homogeneity is not required.

\subsection{Rules}

We use a set of sequents of the form $\Gamma \vdash M : A$ where
$\Gamma$ is the context of the term and $A$ is its type. Let
$\Sigma$ be a set of higher-order constants. We call safe terms any
simply-typed lambda term that is typable within the following system
of formation rules:
$$ \rulename{var} \   \rulef{}{x : A\vdash x : A}
\qquad  \rulename{const} \   \rulef{}{\vdash f : A} \quad f \in \Sigma
\qquad  \rulename{wk} \   \rulef{\Gamma \vdash M : A}{\Delta \vdash M : A} \quad \Gamma \subset \Delta$$

$$ \rulename{app} \  \rulef{\Gamma \vdash M : (A,\ldots,A_l,B)
                                        \qquad \Gamma \vdash N_1 : A_1
                                        \quad \ldots \quad \Gamma \vdash N_l : A_l  }
                                   {\Gamma  \vdash M N_1 \ldots N_l : B}
                                    \quad
                                   \forall y \in \Gamma : \ord{y} \geq \ord{B}$$

$$ \rulename{abs} \   \rulef{\Gamma \union \overline{x} : \overline{A} \vdash M : B}
                                   {\Gamma  \vdash \lambda \overline{x} : \overline{A} . M : (\overline{A},B)} \qquad
                                   \forall y \in \Gamma : \ord{y} \geq \ord{\overline{A},B}$$


Remark:
\begin{itemize}
\item $(\overline{A},B)$ denotes the type $(A_1,A_2, \ldots, A_n, B)$;
\item all the types appearing in the rule are not required to be homogeneous (for instance
it is possible to have $\ord{A_l} < \ord{B}$ in rule $\rulename{app}$) ;
\item the environment $\Gamma \union \overline{x}:\overline{A}$ is not stratified, in particular, variables in $\overline{x}$ do not necessarily have the same order;
\item in the abstraction rule, the side-condition imposes that at least all variables of the lowest order
in the context are abstracted. Variables of greater order can also be
abstracted together with the lowest order variables and, in contrast to
the homogeneous safe lambda calculus, there is no constraint on the
order in which these variables are abstracted;
\end{itemize}

\begin{example}
For $x:o$, $f:(o,o)$ and $\varphi:((o,o),o)$ the term $$\vdash \lambda x f \varphi .
\varphi : (o , (o, o) , ((o,o),o) , (o,o),o)$$ is
a valid safe term that is not homogeneously typed.
\end{example}

\begin{example}
For $x:o$, $g:(o,(o,o),o)$, the term $\vdash \lambda g x . g x$ is unsafe and not homogeneously typed
and the term $\lambda g x . g x (\lambda x . x)$ is safe and not homogeneously typed.
\end{example}

Side-remark: safety is preserved by full $\eta$-expansion. Indeed,
consider the safe term $\Gamma \vdash M:(A_1,\ldots,A_l,o)$ where
$(A_1,\ldots,A_l,o)$ is not necessarily homogeneous. Its full $\eta$-expansion
is $\lambda x_1 .. x_l . M x_1 \dots x_l$ for some variables
$x_1:A_1, \ldots, x_l:A_l$ fresh in $M$. For all $i \in 1..l$ we
have $\Gamma, \Sigma \vdash x_i :A_i$ where $\Sigma = \{ x_1:A_1,
\cdots x_l :A_l \}$. Applying $\rulename{app}$ we obtain $\Gamma,
\Sigma \vdash M x_1 \ldots x_l$ and by the (abs) rule we get
$$\Gamma \vdash \lambda x_1:A_1 \ldots x_l:A_l .M x_1 \ldots x_l.$$

\begin{lemma}[Context reduction]
\label{lem:nonhomosafe_basic_prop}
If $\Gamma \vdash M : B$ is a valid judgment then
\begin{enumerate}
\item $fv(M) \vdash M : B$
\item every variable in $\Gamma$ \emph{occurring free in $M$} has order at
least $ord(M)$.
\end{enumerate}
where $fv(M)$ denotes the context constituted of the free variables occurring in $M$.
\end{lemma}
\begin{proof}
(i) Suppose that some variable $x$ in $\Gamma$ does not occur free
in $M$, then necessarily $x$ has been introduced in the context
using the weakening rule. Hence $\Gamma\setminus \{ x \} \vdash M$
must also be typable. (ii) An easy structural induction.
\end{proof}

The converse of this lemma is not true: consider the simply-typed
term $\lambda y z. (\lambda x . y ) z$ with $x,y,z:o$. This term is
closed therefore it satisfies property (i) and (ii) of lemma
\ref{lem:nonhomosafe_basic_prop}. However it is not typable by the
rules of the safe lambda-calculus since the subterm $\lambda x .y$
is not safe.

\subsection{Substitution in the safe lambda calculus}

The traditional notion of substitution, on which the
$\lambda$-calculus is based, is defined as follows:
\begin{definition}[Substitution]
\label{dfn:subst}
\begin{eqnarray*}
c \subst{t}{x} &=& c \quad \mbox{where $c$ is a $\Sigma$-constant},\\
x \subst{t}{x} &=& t\\
 y\subst{t}{x} &=& y \quad \mbox{for } x \not \neq y,\\
(M_1 M_2) \subst{t}{x} &=& (M_1 \subst{t}{x}) (M_2 \subst{t}{x})\\
(\lambda x . M) \subst{t}{x} &=& \lambda x . M\\
(\lambda y . M) \subst{t}{x} &=& \lambda z . M \subst{z}{y}
\subst{t}{x} \mbox{where $z$ is a fresh variable and $x\not = y$}.
\end{eqnarray*}
\end{definition}

In the setting of the safe lambda calculus, the notion of
substitution can be simplified. Indeed, similarly to what we observe
in the homogeneous safe $\lambda$-calculus, we remark that for safe
$\lambda$-terms there is no need to rename variables when performing
substitution:

\begin{lemma}[No variable capture lemma]
\label{lem:noclash} There is no variable capture when performing
substitution on a safe term.
\end{lemma}

This is the counterpart of lemma \ref{lem:homog_nocapture}. The
proof (which does not rely on homogeneity) is the same.
Consequently, in the safe lambda calculus setting, we can omit to
rename variable when performing substitution. The equation
$$(\lambda x . M) \subst{t}{y} = \lambda z . M \subst{z}{x}
\subst{t}{y} \mbox{where $z$ is a fresh variable}$$ becomes
$$(\lambda x . M) \subst{t}{y} = \lambda x . M \subst{t}{y}.$$

Unfortunately, this notion of substitution is still not adequate for
the purpose of the safe simply-typed lambda calculus. The problem is
that performing a single $\beta$-reduction on a safe term will not
necessarily produce another safe term.

The solution consists in reducing several consecutive $\beta$-redex
at the same time until we obtain a safe term. To achieve this, we
introduce the \emph{simultaneous substitution}, a generalization of
the standard substitution given in definition \ref{dfn:subst}.

\begin{definition}[Simultaneous substitution]
\label{dnf:simsubst}
 The expression $\subst{\overline{N}}{\overline{x}}$ is an abbreviation for $\subst{N_1 \ldots N_n}{x_1
\ldots x_n}$:
\begin{eqnarray*}
c \subst{\overline{N}}{\overline{x}} &=& c \quad \mbox{where $c$ is a $\Sigma$-constant},\\
x_i \subst{\overline{N}}{\overline{x}} &=& N_i\\
 y \subst{\overline{N}}{\overline{x}} &=& y \quad \mbox{ if } y \not \neq x_i \mbox{ for all } i,\\
(M N) \subst{\overline{N}}{\overline{x}} &=& (M \subst{\overline{N}}{\overline{x}}) (N \subst{\overline{N}}{\overline{x}}) \\
(\lambda x_i . M) \subst{\overline{N}}{\overline{x}} &=& \lambda x_i
. M
\subst{N_1 \ldots N_{i-1} N_{i+1}\ldots N_n}{x_1 \ldots x_{i-1} x_{i+1}\ldots x_n} \\
(\lambda y . M)
\subst{\overline{N}}{\overline{x}} &=& \lambda z . M \subst{z}{y} \subst{\overline{N}}{\overline{x}} \\
&& \mbox{where $z$ is a fresh variables and } y \neq x_i \mbox{ for
all } i.
\end{eqnarray*}
\end{definition}

In general, variable capture should be avoided, this explains why
the definition of simultaneous substitution uses auxiliary fresh
variables. However in the current setting, lemma \ref{lem:noclash}
can clearly be transposed to the simultaneous substitution,
therefore there is no need to rename variables.

The notion of substitution that we need is therefore the
\emph{capture-permitting simultaneous substitution} defined as
follows:

\begin{definition}[Capture-permitting simultaneous substitution]
 We use the notation
$\subst{\overline{N}}{\overline{x}}$ for $\subst{N_1 \ldots N_n}{x_1
\ldots x_n}$:
\begin{eqnarray*}
c \subst{\overline{N}}{\overline{x}} &=& c \quad \mbox{where $c$ is a $\Sigma$-constant},\\
 x_i \subst{\overline{N}}{\overline{x}} &=& N_i\\
 y \subst{\overline{N}}{\overline{x}} &=& y \quad \mbox{where } x \not \neq y_i \mbox{ for all } i,\\
(M_1 M_2) \subst{\overline{N}}{\overline{x}} &=& (M_1 \subst{\overline{N}}{\overline{x}}) (M_2 \subst{\overline{N}}{\overline{x}})\\
(\lambda x_i . M) \subst{\overline{N}}{\overline{x}} &=& \lambda x_i
. M
\subst{N_1 \ldots N_{i-1} N_{i+1}\ldots N_n}{x_1 \ldots x_{i-1} x_{i+1}\ldots x_n} \\
(\lambda y . M) \subst{\overline{N}}{\overline{x}} &=& \lambda y . M
\subst{\overline{N}}{\overline{x}} \mbox{where $y \not = x_i$ for
all $i$}. \qquad \mathbf{(\star)}
\end{eqnarray*}
The symbol $\mathbf{(\star)}$ identifies the equation which has
changed compared to the previous definition.
\end{definition}

\begin{lemma}[Substitution preserves safety]
\label{lem:subst_preserve_i}
$$ \Gamma\union \overline{x} : \overline{A}\vdash M : T
\quad \mbox{and} \quad \Gamma \vdash N_k : B_k \mbox{, } k \in
1..n \qquad \mbox{ implies } \qquad \Gamma \vdash
M[\overline{N}/\overline{x}] : T$$
\end{lemma}

\begin{proof}
Suppose that $\Gamma \union \overline{x}: \overline{A} \vdash M :T$ and
$\Gamma \vdash N_k : B_k$ for $k \in 1..n$.

We prove $\Gamma \vdash M[\overline{N}/\overline{x}]$ by induction
on the size of the proof tree of $\Gamma\union
\overline{x}:\overline{A} \vdash M : T$ and by case analysis on the
last rule used. We only give the proof for the abstraction case. If
$\Gamma \union \overline{x}:\overline{A} \vdash \lambda \overline{y}
: \overline{C}. P : (\overline{C}|D)$ where $\Gamma\union
\overline{x}:\overline{A}\union \overline{y}:\overline{C} \vdash P :
D$, then by the induction hypothesis $\Gamma\union
\overline{y}:\overline{C} \vdash P\subst{\overline{N}}{\overline{x}}
: D$. Applying the rule $\rulename{abs}$ gives $\Gamma \vdash
\lambda \overline{y}:\overline{C} . P
\subst{\overline{N}}{\overline{x}}$.
\end{proof}

\subsection{Safe-redex}
In the simply-typed lambda calculus a redex is a term of the form
$(\lambda x . M) N$. We generalize this definition to the safe
lambda calculus:
\begin{definition}[Safe redex]
We call safe redex a term of the form $(\lambda \overline{x} . M)
N_1 \ldots N_l$ such that:
\begin{itemize}
\item $ \Gamma \vdash (\lambda \overline{x} . M) N_1 \ldots N_l $;
\item the variable $\overline{x}=x_1\ldots x_n$ are abstracted altogether by one occurrence of the rule $\rulename{abs}$ in the proof
tree;
\item the terms $(\lambda \overline{x} . M)$, $N_1$, $N_l$ are applied together at once using the $\rulename{app}$ rule :
$$   \rulef{
            \Sigma \vdash \lambda \overline{x} . M
            \quad
            \Sigma \vdash N_1         \quad \ldots \quad \Sigma \vdash N_l
    }
    {
       \Sigma \vdash (\lambda \overline{x} . L) N_1 \ldots N_l
    } (\mathbf{app})
$$
and consequently each $N_i$ is safe;
\end{itemize}
\end{definition}

The relation $\beta_s$ is defined exactly the same way as in the homogeneous safe $\lambda$-calculus. The safe $\beta$-reduction $\betasred$ is defined as the closure of $\beta_s$ by
compatibility with the formation rules of the safe
$\lambda$-calculus.  It is straightforward to show, as we did for the homogeneous safe $\lambda$-calculus, that $\betasred \subset \betaredtr$.


\begin{lemma}
\label{lem:safereduction} A safe redex reduces to a safe term.
\end{lemma}

This lemma, which is a consequence of lemma
\ref{lem:subst_preserve_i}, is the counterpart of lemma
\ref{lem:homoh_safered_preserve_safety} in the homogeneous safe
lambda calculus. Their proofs are identical.


\subsection{Particular case of homogeneously-safe lambda terms}

In this section, we derive a new set of rules by adding the type-homogeneity restriction to the non-homogenous safe lambda calculus.

We recall the definition of type-homogeneity from section
\ref{sec:safe_homog}: a type $(A_1, A_2, \ldots A_n, o)$ is said to
be homogeneous whenever $\ord{A_1} \geq \ord{A_2} \geq \ldots \geq
\ord{A_n}$ and each of the $A_i$ is homogeneous. A term is said to
be homogeneous if its type is homogeneous.

We now impose type-homogeneity to all the sequents present in the
rules of the safe $\lambda$-calculus: we say that a term is
\emph{homogeneously-safe} if there is a proof tree showing its
safety in which all sequents are of homogenous type. Consequently a
homogeneously-safe term is safe and has an homogenous type.

We say that $\Gamma \vdash M : A$ verifies $P_i$ for $i \in \zset$
if all the variables in $\Gamma$ have order at least $\ord{A}+i$.
Lemma \ref{lem:nonhomosafe_basic_prop} can then be restated as
follows:
\begin{lemma}[Context reduction]
\label{lem:context_reduction} If $\Gamma \vdash M : A$ then the sequent $fv(M) \vdash M : A$ is valid and satisfies $P_0$.
\end{lemma}


We now prove that if we impose the homogeneity of types, the set of
rules of the non-homogenous safe $\lambda$-calculus and the rules of
table \ref{tab:homosafelmd_rules_refined} are equivalent.  We recall
that in the system of rules of table
\ref{tab:homosafelmd_rules_refined}, if the sequent $\Gamma
\vdash^{i} M : A$ is valid for some $i \in \zset$ then all the
variables in $\Gamma$ have orders at least $\ord{A}+i$.

\begin{proposition}[Homogeneity restriction]
\label{prop:nonhomogsafe_homog_restriction}
Let $k \in \{ 0, -1 \}$. The sequent $\Gamma \vdash M : A$ is valid, homogeneously-safe and satisfies $P_k$
if and only if the sequent $\Gamma \vdash^k M : A$ is valid in the system of rules of table \ref{tab:homosafelmd_rules_refined}.
\end{proposition}

\begin{proof}
\emph{If}: An easy induction by case analysis on the last rule used to derive $\Gamma \vdash^0 M : A$.

\emph{Only if}:
Consider an homogeneously-safe term $\Gamma \vdash S : T$ satisfying $P_0$.
We proceed by induction and case analysis on the last rule used to derive $\Gamma \vdash S : T$.
We only give the details for the application and abstraction
case:
\begin{itemize}
\item \textbf{Abstraction.} We recall the abstraction rule:
$$ \rulename{abs} \quad  \rulef{\Gamma \union \overline{x} : \overline{A} \vdash M : B}
                                   {\Gamma  \vdash \lambda \overline{x} : \overline{A} . M : (\overline{A},B)} \qquad
                                   \forall y \in \Gamma : \ord{y} \geq \ord{\overline{A},B}$$

Type homogeneity requires that for all $i$: $\ord{x_i} = \ord{A_i} \geq
\ord{B} -1$. Therefore the premise of the rule verifies $P_{-1}$. Using the induction hypothesis we have:
\begin{equation}
\Gamma \union \overline{x} : \overline{A} \vdash^{-1} M : B. \label{eq:prop:nonhomogsafe_homog_restriction:abs1}
\end{equation}

We now partition the context $\Gamma$ according to the order of
the variables. The partitions are written in decreasing order of
type order. The notation $\Gamma | \overline{x}:\overline{A}$ means
that $\overline{x}:\overline{A}$ is the lowest partition of the
context.
We also use the notation $(\overline{A}|B)$ to denote the
homogeneous type $(A_1, A_2, \ldots A_n, B)$ where $\ord{A_1} =
\ord{A_2} =  \ldots \ord{A_n} \geq \ord{B} -1$.


Suppose that we abstract a single variable $x$, then in order to
respect the side condition, we need to abstract all variables of
order less or equal to $\ord{x}$. In particular we need to abstract
the partition of the order of $x$. Moreover to respect type
homogeneity, we need to abstract variables of the lowest order
first.

Hence $\overline{x}$ must contain at least the lowest variable
partition (all the variables of the lowest order). If $\overline{x}$
contains variables of different order, then the instance of the
abstraction rule can be replaced by consecutive instances of the
abstraction rule, one for each of the different variable order in
$\overline{x}$. Therefore, without loss of generality, we can assume
that $\overline{x}$ only contains the lowest partition, that is to
say, $\overline{x}$ \emph{is} the lowest partition.

The sequent \ref{eq:prop:nonhomogsafe_homog_restriction:abs1} therefore becomes:
$$\Gamma | \overline{x} : \overline{A} \vdash^{-1} M : B.$$

We conclude by applying the abstraction rule of table
\ref{tab:homosafelmd_rules_refined}:
$$ \rulename{abs} \quad  \rulef{\Gamma| \overline{x} : \overline{A} \vdash^{-1} M : B}
                                   {\Gamma  \vdash^{0} \lambda \overline{x} : \overline{A} . M : (\overline{A}|B)}$$



\item \textbf{Application.} We recall the application rule:
$$ \rulename{app} \  \rulef{\Gamma \vdash M : (A,\ldots,A_l,B)
                                        \qquad \Gamma \vdash N_1 : A_1
                                        \quad \ldots \quad \Gamma \vdash N_l : A_l  }
                                   {\Gamma  \vdash M N_1 \ldots N_l : B}
                                    \quad
                                   \forall y \in \Gamma : \ord{y} \geq \ord{B}$$

The term in the conclusion is homogeneously safe therefore the term in the first premise must be of homogeneous \
type. This implies that $\ord{A_1} \geq \ldots \geq \ord{A_l}
\geq \ord{B} - 1$.
Furthermore, we can make the assumption that $\ord{A_1} = \ldots = \ord{A_l} = \ord{\overline{A}}$
(it is always possible to replace an instance of the application rule
by several consecutive instances of this kind).

By lemma \ref{lem:context_reduction}, we have for all $i \in 1..l$:
$$fv(N_i) \vdash N_i : A_i \mbox{ is valid and satisfies } P_0.$$

Let $\Sigma = \Union_{i=1..p} fv(N_i)$. Since $\ord{A_1} = \ldots = \ord{A_l}$, by applying the weakening rule we get for all $i\in 1..p$:
$$\Sigma \vdash N_i : A_i \mbox{ is valid and satisfies } P_0.$$


Applying lemma \ref{lem:context_reduction} to the term $M$ we have:
$$fv(M) \vdash M : (A_1,\ldots,A_l,B) \mbox{ is valid and satisfies } P_0.$$

The weakening rule $\rulename{wk}$ then gives:
$fv(M) \union \Sigma \vdash M : (A_1,\ldots,A_l,B)$.
Since $\Sigma \vdash N_i : A_i$ satisfies $P_0$, for any
$z \in \Sigma$ we have $\ord{z} \geq \ord{A_i} = \ord{(A_1,\ldots,A_l,B)} - 1$.
Hence:
\begin{equation}
fv(M) \union \Sigma \vdash M : (A_1,\ldots,A_l,B) \mbox{ is valid
and satisfies } P_{-1}
\label{eq:prop:nonhomogsafe_homog_restriction:m}.
\end{equation}

Similarly, for all $i \in 1..p$, the weakening rule gives $fv(M) \union \Sigma \vdash N_i : A_i$.
Since $fv(M) \vdash M : (A_1,\ldots,A_l,B)$ satisfies $P_0$,
for any $z \in fv(M)$ we have $\ord{z} \geq \ord{M} \geq \ord{A_i}$. Hence:
\begin{equation}
fv(M) \union \Sigma \vdash N_i : A_i \mbox{ is valid and satisfies }
P_0 \label{eq:prop:nonhomogsafe_homog_restriction:ni}.
\end{equation}

Let us define the context $\Sigma' = fv(M) \union \Sigma$. Using the induction hypothesis on equation
\ref{eq:prop:nonhomogsafe_homog_restriction:m} and \ref{eq:prop:nonhomogsafe_homog_restriction:ni} we have:
$$
\Sigma' \vdash^{-1} M : (A_1,\ldots,A_l,B) \qquad \mbox{and} \qquad
\Sigma' \vdash^0 N_i : A_i \mbox{ for all } i \in 1..l.
$$


We consider the following two sub-cases:
\begin{itemize}
\item If $A_1, \ldots, A_l$ forms a type partition then we can apply
rule $\rulename{app}$ of table \ref{tab:homosafelmd_rules_refined}:

$$ \rulef{\Sigma' \vdash^{-1} M : \overline{A} | B
                                        \qquad \Sigma' \vdash^{0} N_1 :
                                        A_1
                                        \quad \ldots \quad \Sigma' \vdash^{0} N_l :
                                        A_l
                                        \quad l = |\overline{A}|
                                        }
                                   {\Sigma'  \vdash^{0} M N_1 \ldots N_l : B} \quad  \rulename{app}
$$
where $\overline{A} = A_1, \ldots, A_l$.

\item  Suppose that $A_1, \ldots, A_l$ does not form a type partition, then we
have $$\ord{A_1} = \ldots = \ord{A_l} = \ord{B} - 1.$$

The side condition in the original instance of the application rule
says that for any variable $y$ in $\Gamma$ we have
$$\ord{y} \geq \ord{B} = 1 + \ord{A_l} = \ord{(A_1,\ldots, A_l,B)} = \ord{M}.$$

In particular the variables in $\Sigma' \subseteq \Gamma$ are of order greater than $\ord{M}$ and consequently
the sequent $\Sigma' \vdash M : (A,\ldots,A_l,B)$ verifies $P_0$. The induction hypothesis then gives:
$$\Sigma' \vdash^0 M : (A,\ldots,A_l,B)$$

By using $l$ consecutive instances of the rules $\rulename{app^+}$ from table \ref{tab:homosafelmd_rules_refined} we get:
$$  \rulef{ \rulef{ \rulef{ \Sigma' \vdash^0 M : (A_1,\ldots, A_l,B)
                    \qquad \Sigma'\vdash^{0} N_1 : A_1
                    }{ \Sigma' \vdash^0 M N_1 : (A_2,\ldots, A_l,B)} \quad \rulename{app^+}
          }
          { \vdots
          }
          \quad \rulename{app^+}
       }
       { \Sigma'  \vdash^{0} M N_1 \ldots N_l : B } \quad \rulename{app^+}
$$
\end{itemize}

In both cases we have proved that $\Sigma'  \vdash^{0} M N_1 \ldots N_l : B$ is a valid sequent.

Clearly $\Sigma' \subseteq \Gamma$ since $fv(M) \subseteq \Gamma$ and $\Sigma' = \Union_{i\in1..l} fv(N_i) \subseteq \Gamma$.
Suppose that $\Sigma' = \Gamma$ then the proof is done.
Suppose that $\Sigma' \subset \Gamma$, then the side condition in the original instance of the application rule says that all
the variables in $\Gamma$ have order
greater or equal to $\ord{B}$, we can therefore apply the weakening rule $\rulename{wk^0}$
of table \ref{tab:homosafelmd_rules_refined} exactly $|\Gamma\setminus \Sigma'|$ times and get:
$$ \rulef{\Sigma'  \vdash^{0} M N_1 \ldots N_l : B}
                                   {\Gamma  \vdash^{0} M N_1 \ldots N_l : B} \quad
                                   \rulename{wk^0}.
$$


\end{itemize}
\end{proof}


\subsection{Examples}
\subsubsection{Example 1}
Let $f,g:o\rightarrow o$, $x,y:o\rightarrow o$, $\Gamma =
g:o\rightarrow o$ and $\Gamma' = g:o\rightarrow o, y:o$. The term
$(\lambda f x . x) g y $ is safe. One possible proof tree is:
$$ \rulef{
        \rulef{
            \rulef{
                \rulef{\vdots}{\Gamma \vdash \lambda f x. x}      \qquad \axiomf{\Gamma \vdash g} }
            {\Gamma \vdash (\lambda f x. x) g} \rulename{app}
        }
        { \Gamma' \vdash (\lambda f x. x) g } \rulename{wk}
        \qquad \axiomf{\Gamma' \vdash y}
    }
    { \Gamma' \vdash (\lambda f x. x) g y } \rulename{app}
$$
Here is another proof for the same judgment:
$$ \rulef{  \rulef{ \rulef{\vdots}{\Gamma \vdash \lambda f x. x} }{\Gamma' \vdash \lambda f x. x} \rulename{wk}    \qquad \rulef{}{\Gamma' \vdash g} \qquad \rulef{}{\Gamma' \vdash y}}
    {\Gamma' \vdash (\lambda f x. x) g y } \rulename{app}$$

We see on this particular example that there may exist different
proof trees deriving the same judgment.

\subsubsection{Example 2 - Damien Sereni's SCT counter-example}
In \cite{serenistypesct05}, the following counter-example is given
to show that not all simply-typed terms are size-change terminating
(see \cite{jones01} for a definition of size-change termination):

$$ E =  (\lambda a . a (\lambda b . a (\lambda c d .d))) (\lambda e . e (\lambda f .f))$$
where:
\begin{eqnarray*}
a &:& \sigma \typear \mu \typear \mu \\
b &:& \tau \typear \tau \\
c &:& \tau \typear \tau \\
d &:& \mu \\
e &:& \sigma = (\tau \typear \tau) \typear \mu \typear \mu \\
f &:& \tau
\end{eqnarray*}
and $\tau$, $\mu$ and $\sigma$ are type variables.

This example shows that the rules of the safe $\lambda$-calculus
without the homogeneity restriction generates a class of terms that
strictly contains the class of terms generated by the rules of the
homogeneous safe $\lambda$-calculus of section \ref{sec:safe_homog}.

Indeed, for $E$ to be an homogeneous safe lambda term, in the sense
of the rules of section \ref{sec:safe_homog}, $\tau$ and $\mu$ must
be homogeneous types and the variables $a,b,c,d,e,f$ must be
homogeneously typed. This implies that $ \ord{\tau} \geq
\ord{\mu}-1$. Conversely, if this condition is met then $\vdash E :
\mu \typear \mu$ is a valid judgement of the \emph{homogeneous} safe $\lambda$-calculus.

In the safe $\lambda$-calculus \emph{without} the homogeneity
constraint, however, the judgement $\vdash E : \mu \typear \mu$ is
always valid whatever the types $\mu$ and $\tau$ are.



    \section{Safe PCF}
        \subsection{Definition and properties}
        \subsection{Game-semantic analysis via a syntactic argument}

    %\section{Safe IA}
    \input{sec_safeia.texi}


\chapter{Local Computation of \texorpdfstring{$\beta$}{Beta}-Reduction}
    \label{chap:localbeta}
    We make an explicit correspondence between the game denotation of a
term and its syntax. Our approach follows ideas recently introduced
in \cite{OngLics2006}, mainly the notion of computation tree of a
simply-typed $\lambda$-term and traversals over the computation
tree. A computation tree can be regarded as an abstract syntax tree
(AST) of the $\eta$-long normal form of a term. A traversal is a
justified sequence of nodes of the computation tree respecting some
formation rules. Traversals are used to describe computations. An
interesting property is that the \emph{P-view} of a traversal
(computed in the same way as P-view of plays in Game Semantics) is a
path in the computation tree.

The main result of this paper is called the
\emph{Correspondence Theorem} (theorem \ref{thm:correspondence}). It
states that traversals over the computation tree are just
representations of the uncovering of plays in the
strategy-denotation of the term. Hence there is an isomorphism
between the strategy denotation of a term and its revealed game
denotation ({\it i.e.}~its strategy denotation where internal moves are
not hidden after composition). This theorem permits us to explore
the effect that a given syntactic restriction (such as the safety restriction) has on the strategy
denoting a term.

To really make use of the Correspondence Theorem, it will be
necessary to restate it in the standard game-semantic framework in
which internal moves are hidden. For that purpose, we will define a
\emph{reduction} operation on traversals responsible of eliminating
the ``internal nodes'' of the computation. This leads to a
correspondence between the standard game denotation of a term and
the set of reductions of traversals over its computation tree.
Fortunately, the reduction operation preserves the good properties
of traversals. This is guaranteed by the facts that the P-view of
the reduction of a traversal is equal to the reduction of the P-view
of the traversal, and the O-view of a traversal is the same as the
O-view of its reduction (lemma \ref{lem:redtrav_trav}). \vspace{8pt}

\emph{Related works}: Traversals of a computation tree provide a way
to perform \emph{local computation} of $\beta$-reductions as opposed
to a global approach where the $\beta$-reduction is implemented by
performing substitutions. A notion of local computation of
$\beta$-reduction has been investigated in
\cite{DanosRegnier-Localandasynchronou} through the use of special
graphs called ``virtual nets'' that embed the lambda-calculus.

In \cite{DBLP:conf/lics/AspertiDLR94}, a notion of graph based on
Lamping's graphs \citep{lamping} is introduced to represent
$\lambda$-terms. The authors unify different notions of paths
(regular, legal, consistent and persistent paths) that have appeared
in the literature as ways to implement graph-based reduction of
lambda-expressions. We can regard a traversal as an alternative
notion of path adapted to the graph representation of
$\lambda$-expressions given by computation trees.



%Is there any unsafe term whose game semantics is a strategy where
%pointers can be recovered?
%
%The answer is yes: take the term $T_i = (\lambda x y . y) M_i S$
%where $i =1..2$ and $\Gamma \vdash_s S : A$. $T_1$ and $T_2$ both
%$\beta$-reduce to the safe term $S$, therefore
%$\sem{T_1}=\sem{T_2}=\sem{S}$. But $T_1$ is safe whereas $T_2$ is
%unsafe. Since it is possible to recover the pointer from the game
%semantics of $S$, it is as well possible to recover the pointer from
%the semantics of $T_2$ which is unsafe.

\section{Computation tree}
We work in the general setting of the simply-typed
$\lambda$-calculus extended with a fixed set $\Sigma$ of
higher-order uninterpreted constants \footnote{A constant $f$ is
  \emph{uninterpreted} if the small-step semantics of the language
  does not contain any rule of the form $f \dots \rightarrow e$. $f$
  can be regarded as a data constructor.}

For the rest of the section we fix a simply-typed term $\Gamma \vdash M :T$.

\subsection{$\eta$-long normal form}

The $\eta$-long normal form appeared in
\citep{DBLP:journals/tcs/JensenP76} and
\citep{DBLP:journals/tcs/Huet75} under the names \emph{long reduced
form} and \emph{$\eta$-normal form} respectively. It was then
investigated in \citep{huet76} under the name \emph{extensional
form}.

The $\eta$-expansion of $M: A\typear B$ is defined to be the term
$\lambda x . M x : A\typear B$ where $x:A$ is a fresh variable. A
term $M : (A_1,\ldots,A_n,o)$ can be expanded in several steps into
$\lambda \varphi_1 \ldots \varphi_l . M \varphi_1 \ldots \varphi_l$
where the $\varphi_i:A_i$ are fresh variables. The $\eta$-normal
form of a term is obtained by hereditarily $\eta$-expanding every
subterm occurring at an operand position.

\begin{definition}[$\eta$-long normal form]
A simply-typed term is either an abstraction or it can be written uniquely as
$s_0 s_1 \ldots s_m$ where $m\geq0$ and $s_0$ is a variable, a $\Sigma$-constant or an abstraction.
The $\eta$-long normal form of a term $t$, written $\elnf{t}$ or sometimes $\etanf{t}$,
is defined as follows:
\begin{align*}
\elnf{\lambda x . s } &= \lambda x . \elnf{s} \\
\elnf{\alpha s_1 \ldots s_m : (A_1,\ldots,A_n,o)} &= \lambda \overline{\varphi} . \alpha \elnf{s_1}\ldots \elnf{s_m} \elnf{\varphi_1} \ldots \elnf{\varphi_n}
& \mbox{with $m,n\geq0$}\\
\elnf{(\lambda x . s) s_1 \ldots s_p : (A_1,\ldots,A_n,o) } &= \lambda \overline{\varphi} . (\lambda x . \elnf{s}) \elnf{s_1} \ldots \elnf{s_p} \elnf{\varphi_1} \ldots \elnf{\varphi_n}
& \mbox{with $p\geq 1,n\geq 0$}
\end{align*}
where $x$ and each $\varphi_i : A_i$ are variables and $\alpha$ is
either a variable or a constant.
\end{definition}

For $n=0$, the first clause in the definition becomes:
$$\elnf{x s_1 \ldots s_m : o} = \lambda . x \elnf{s_1} \elnf{s_2} \ldots \elnf{s_m},$$
and we deliberately keep the \textsl{dummy} lambda in the right-hand
side of the equation because it will play an important role in the
correspondence with game semantics.



Note that our version of the $\eta$-long normal form is defined not only for $\beta$-normal terms but also for any simply-typed term.
Moreover it is defined in such a way that $\beta$-normality is preserved:
\begin{lemma}
The $\eta$-long normal form of a term in $\beta$-normal form is also in $\beta$-normal form.
\end{lemma}
\begin{proof}
By induction on the structure of the term and the order of its type.
\emph{Base case}:
If $M=x:0$ then $\elnf{x} = \lambda . x$ is also in $\beta$-nf.
\emph{Step case}:
The case $M = (\lambda x . s) s_1 \ldots s_m : (A_1,\ldots,A_n,o)$ with $m>0$ is not possible since $M$ is in
$\beta$-normal form.
Suppose $M = \lambda x . s$ then $s$ is in $\beta$-nf. By the induction hypothesis $\elnf{s}$ is also in $\beta$-nf and therefore
so is $\elnf{M} = \lambda x . \elnf{s}$.

Suppose $M= \alpha s_1 \ldots s_m : (A_1,\ldots,A_n,o)$. Let $i,j$
range over $1..n$ and $1..m$ respectively. The $s_j$ are in
$\beta$-nf and the $\varphi_i$ are variables of order smaller than
$M$, therefore by the induction hypothesis the $\elnf{\varphi_i}$ and
the $\elnf{s_j}$ are in $\beta$-nf. Hence $\elnf{M}$ is also in
$\beta$-nf.
\end{proof}

\begin{lemma}[$\eta$-long normalisation preserves safety]
If $\Gamma \vdash s$ then $\Gamma \vdash \elnf{s}$.
\end{lemma}
\begin{proof}

First we observe that for any variable or constant $x$ we have $x \vdash \elnf{x}$. The proof is by induction on $\ord{x}$. Base case: $x$ is of ground type and we have $x \vdash x = \elnf{x}$. Step case:
$x:(A_1, \ldots, A_n,o)$ with $n>0$. Let $\varphi_i:A_i$ be fresh variables for $1\leq i\leq n$. The (var) rules gives $\varphi_i  \vdash \varphi_i$ and since $\ord{A_i} < \ord{x}$ the induction hypothesis gives $\varphi_i \vdash \elnf{\varphi_i}$. Using (wk) we obtain $x, \overline{\varphi} \vdash \elnf{\varphi_i}$.
The application rule gives $x, \overline{\varphi} \vdash x \elnf{\varphi_1} \ldots \elnf{\varphi_n} : o$ and the abstraction rule gives $ x \vdash \lambda \overline{\varphi} . x \elnf{\varphi_1} \ldots \elnf{\varphi_n} = \elnf{x}$.


We now prove the lemma by induction on the structure of $s$.
The base case (where $s$ is some variable $x$) is covered by the previous observation.
\emph{Step case:}
\begin{itemize}
\item $s = x s_1 \ldots s_m$ with $x: (B_1, \ldots, B_m, A_1, \ldots, A_n, o)$ with $m\geq 0$, $n>0$ and $s_i : B_i$ for $1 \leq i \leq m$.

Let $\varphi_i: A_i$ be fresh variables for $1\leq i \leq n$. By the previous observation we have $\varphi_i \vdash \elnf{\varphi_i}$ which in turn gives $\Gamma , \overline{\varphi} \vdash \elnf{\varphi_i}$ using the weakening rule.

The judgement $\Gamma \vdash x s_1 \ldots s_m$ is formed using the (app) rule therefore each $s_j$ is safe for $1\leq j \leq m$. By the induction hypothesis we have $\Gamma \vdash \elnf{s_j}$ and by weakening we get $\Gamma, \overline{\varphi} \vdash \elnf{s_j}$.

The application rule gives $\Gamma, \overline{\varphi} \vdash
x \elnf{s_1} \ldots \elnf{s_m} \elnf{\varphi_1} \ldots \elnf{\varphi_n} : o$. Finally the (abs) rule gives $\Gamma \vdash \lambda \overline{\varphi} . x \elnf{s_1} \ldots \elnf{s_m}  \elnf{\varphi_1} \ldots \elnf{\varphi_n} = \elnf{s}$, the side-condition of (abs) being met since $\ord{\elnf{s}} = \ord{s}$.


\item $s = t s_0 \ldots s_m$ where $t$ is an abstraction. Again, using the induction hypothesis it is easy to show that $\Gamma \vdash \elnf{s} = \elnf{t} \elnf{s_0} \ldots \elnf{s_m} \elnf{\varphi_1} \ldots \elnf{\varphi_n}$ holds for some fresh variables $\varphi_1$, \ldots, $\varphi_n$.

\item $s = \lambda \overline{\eta} . t$ where $t$ is not an abstraction. By the induction hypothesis we have $\Gamma, \overline{\eta} \vdash \elnf{t}$ and by the abstraction rule we have $\Gamma \vdash \lambda \overline{\eta} . \elnf{t} = \elnf{s}$.
\end{itemize}
\end{proof}

Note that in general the converse does not hold, for instance $\lambda x^o . f^{o,(o,o),o} x^o$ is unsafe although $\elnf{\lambda x . f x} = \lambda x^o \varphi^{o,o} . f x \varphi$ is safe (and not homogeneous). For terms with homogeneous types however, the converse does hold:
\begin{lemma}
If $\Gamma \vdash \elnf{s}$ is homogeneously safe (i.e. it is a safe judgement of the safe $\lambda$-calculus and each sequent occurring at the nodes of the proof tree is homogeneously typed) then
$\Gamma \vdash s$ is homogeneously safe.
\end{lemma}


\subsection{Computation tree}
The computation tree of a term is a certain tree representation of its
$\eta$-long normal form. It is defined as follows:

\begin{definition}
\label{dfn:comptree} Let $M$ be a simply-typed term in $\eta$-normal
form. Then $M$ is either an abstraction or it can be written
uniquely as $s_0 s_1 \ldots s_m : o$ for some $m\geq0$ where $s_0$
is a variable, a constant or an abstraction and each of the $s_j$
for $j\in 1..m$ is in $\eta$-normal form. The
\defname{computation tree} $\tau(M)$ of $M$ is defined by induction
on the structure of the term:
\begin{enumerate}[-]
\item If $n\geq0$ and $s$ is not an abstraction then:
$$ \tau(\lambda x_1 \ldots x_n . s) =
      \pstree[levelsep=3ex]
        { \TR{\lambda x_1 \ldots x_n} }
        { \SubTree{\tau(s)^{-}} }
$$
where $\tau(s)^{-}$ denotes the tree obtained after deleting the root of $\tau(s)$.

\item If $m\geq0$ and $\alpha$ is a variable or constant then:
$$ \tau( \alpha s_1 \ldots s_m : o) =
    \tree{\lambda}
    {
        \pstree[levelsep=3ex]
            { \TR{\alpha} }
            { \SubTree{\tau(s_1)} \SubTree[linestyle=none]{\ldots} \SubTree{\tau(s_m)}
            }
    }
$$

\item If $n \geq 1$ then:
$$ \tau((\lambda x.s) s_1 \ldots s_n : o) =
    \tree{\lambda}
    {
        \pstree[levelsep=3ex]
            { \TR{@} }
            {
            \SubTree{\tau(\lambda x.s)}    \SubTree{\tau(s_1)} \SubTree[linestyle=none]{\ldots} \SubTree{\tau(s_n)}
            }
    }
$$
\end{enumerate}

If $M$ is not in $\eta$-normal form then $\tau(M)$ is defined as the
computation tree of its $\eta$-normal form ($\tau(M) =
\tau(\etanf{M})$).
\end{definition}

The nodes (and leaves) of the tree are of three kinds:
\begin{itemize}
\item $\lambda$-nodes labelled $\lambda \overline{x}$ (note that a $\lambda$-node represents several consecutive variable abstractions),
\item application nodes labelled @,
\item variable or constant nodes labelled $\alpha$ for some constant or variable $\alpha$.
\end{itemize}
A node is said to be \defname{prime} if it is the 0$^{th}$ child of an @-node.

\emph{Notations:} We write $r$ for the root of $\tau(M)$. We write $E$ to denote the parent-child relation
of the tree, $N$ for the set of nodes of $\tau(M)$,
$N_\Sigma$ for the set of $\Sigma$-labelled nodes, $N_@$ for the set
of @-labelled nodes, $N_{\sf var}$ for the set of variable nodes,
$N_{\sf fv}$ for the subset of $N_{\sf var}$ constituted of free-variable
nodes, $N_{\sf spawn}$ for the set $N \inter E \relimg{N_@ \union N_\Sigma}$ constituted of children of constant-nodes and @-nodes and $N_{\sf prime}$ for the set of prime nodes.


Let $\mathcal{T}$ denote the set of $\lambda$-terms.
Each subtree of the computation tree $\tau(M)$ represents a subterm of $\elnf{M}$.
We define the function $\kappa : N \rightarrow \mathcal{T}$ that maps a node $n \in N$ to the subterm of $\elnf{M}$
corresponding to the subtree of $\tau(M)$ rooted at $n$.
In particular $\kappa(r) = \elnf{M}$.

\begin{definition}[Type and order of a node]
\label{def:nodeorder}
Suppose $\Gamma \vdash M : T$.
The \defname{type} of a node $n$ of $\tau(M)$ written $type(n)$ is defined as follows:
\begin{eqnarray*}
type(r) &=& \Gamma \rightarrow T \\
type(\alpha:A) &=& A, \mbox{ where $\alpha$ is a variable or constant} \\
type(n) &=& \hbox{ type of the term $\kappa(n)$ for $n \in (N_\lambda \union N_@) \setminus \{r \}$\ .}
\end{eqnarray*}
The order of a node $n$ written $\ord{n}$ is defined to be the order of the type of $n$.
\end{definition}

In particular, $\ord{@} = 0$, $\ord{\lambda \overline{\xi}} = 1+
\max_{z\in \overline{\xi}} \ord{z}$ for $\lambda \overline{\xi}\neq
r$ and if $r=\lambda \overline{\xi}$ then $\ord{r} = 1 + \max_{z\in
\overline{\xi}\union \Gamma} \ord{z}$ with the convention $\max
\emptyset = -1$.

\begin{remark} \hfill
\begin{itemize}
\item In a computation tree, nodes at even level are $\lambda$-nodes and nodes at odd level are either application nodes,
variable or constant nodes;

\item for any ground type variable or constant $\alpha$,
$\tau(\alpha) = \tau(\lambda . \alpha) =  \pstree[levelsep=3ex]
    { \TR{\lambda } }
    { \TR{\alpha}
    }$;

\item for any higher-order variable or constant $\alpha : (A_1,\ldots,A_p,o)$, the computation tree $\tau(\alpha)$ has the following form:
$ \pstree[levelsep=3ex]{\TR{\lambda}}
        {\pstree[levelsep=3ex]
                { \TR{\alpha} }
                { \tree{\lambda \overline{\xi_1}}{\TR{\ldots}} \TR{\ldots} \tree{\lambda \overline{\xi_p}}{\TR{\ldots}}
                }
        }
$;

\item for any tree of the form
        $ \pstree[levelsep=4ex]
            { \TR{\lambda \overline{\varphi}} }
            { \pstree[levelsep=3ex]
                {\TR{n}}
                {\TR{\lambda \overline{\xi_1}} \TR{\ldots} \TR{\lambda \overline{\xi_p}}}
            }
        $,
    we have $\ord{\kappa(n)}=0$.

\end{itemize}
\end{remark}


\subsection{Pointers and justified sequence of nodes}

\begin{definition}[Binder]
Let $n$ be a variable node of the computation tree labelled $x$. We
say that a node $n$ is bound by the node $m$, and $m$ is called the
binder of $n$, if $m$ is the closest node in the path from $n$ to
the root of the tree such that $m$ is labelled $\lambda
\overline{\xi}$ with $x\in \overline{\xi}$.
\end{definition}

\begin{definition}[Enabling]
The \defname{enabling relation} $\vdash$ is defined on the set of
nodes of the computation tree as follows. We write $m \vdash n$ and
we say that $m$ enables $n$ if and only if
\begin{itemize}
\item $n$ is a bound variable node and $m$ is the binder of $n$. We will write $m \vdash_i n$ to precise that $n$
is the $i^{\sf th}$ variable bound by $m$;
\item or $n$ is a free variable node and $m$ is the root of the computation
tree;
\item or $n$ is a $\lambda$-node and $m$ is the parent node of $n$.
\end{itemize}
\end{definition}

We say that a node $n_0$ of a justified sequence is
\defname{hereditarily justified} by $n_p$ if there are nodes $n_1,
\ldots, n_{p-1}$ in the sequence such that $n_i$ points to $n_{i+1}$
for all $i\in 0..p-1$.

For any set of nodes $S$ we write $S^{\upharpoonright r}$ for $\{ n \in S \ | \ r  \vdash^* n \}$ -- the subset of $S$ constituted of
nodes hereditarily enabled by $r$.
We call \defname{input-variables nodes} the elements of $N_{\sf var}^{\upharpoonright r}$ i.e.\
variables that are hereditarily enabled by the root. $N_{\sf var}^{\upharpoonright r}$ is also the set of nodes that are hereditarily enabled by a free variable or by a variable bound by the root.
\smallskip

We use the following numbering conventions:
the first child of a @-node is numbered $0$;
the first child of a variable or constant node is numbered $1$;
and variables in $\overline{\xi}$ are numbered from $1$ onward ($\overline{\xi} = \xi_1 \ldots \xi_n$).
We write $n.i$ to denote the $i$th child of node $n$.

\begin{definition}[Justified sequence of nodes]
A \defname{justified sequence of nodes} is a sequence of nodes of
the computation tree $\tau(M)$ with pointers such that each variable
or $\lambda$-node $n$ different from the root has a pointer to a
node $m$ occurring before it the sequence and such that $m \vdash
n$.

If $n$ points to $m$ then we say that $m$ \emph{justifies} $n$. We
represent the pointer in the sequence as follows \Pstr[0.4cm]{
(m){m} \ldots (n-m,45:i) n }. where the label indicates that either
$n$ is labelled with the $i$th variable abstracted by the
$\lambda$-node $m$ or that $n$ is the $i^{\sf th}$ child of $m$.
\end{definition}

Note that justified sequences are also defined for open terms:
occurrences of nodes in $N_{\sf fv}$ must point to an occurrence of the
root of the computation tree. Thus a pointer in a justified sequence of nodes has
one of the following forms:
$$
\Pstr[18pt]{ (m){r} \cdot \ldots \cdot (n-m,40){z} }
\hspace{1.5cm}
\Pstr{ (m){\lambda \overline{\xi}} \cdot \ldots \cdot (n-m,40:i){\xi_i} }
\hspace{1.5cm}
\Pstr{ (m){@} \cdot \ldots \cdot (n-m,40:j){\lambda \overline{\eta}} }
\hspace{1.5cm}
\Pstr{ (m){\alpha } \cdot \ldots \cdot (n-m,40:k){\lambda \overline{\eta}} }
$$
for some occurrences $r$ of $\tau(M)$'s root, $z \in N_{\sf fv}$,
bound variables $\xi_1,
\ldots \xi_n$, $\alpha \in N_{\Sigma} \union
N_{\sf var}$, $i \in 1..n$, $j$ ranges from $0$ to the number of
children nodes of @ minus 1 and $k \in 1 ..arity(\alpha)$.
\bigskip

\emph{Notations}: We write $s = t$ to denote that the justified sequences $t$ and $s$
have same nodes \emph{and} pointers. Justified sequence of nodes can
be ordered using the prefix ordering: $t \sqsubseteq t'$ if and only
if $t=t'$ or the sequence of nodes $t$ is a finite prefix of $t'$
(and the pointers of $t$ are the same as the pointers of the
corresponding prefix of $t'$). Note that with this definition,
infinite justified sequences can also be compared. This ordering
gives rise to a complete partial order.
We say that a node $n_0$ of a justified sequence is \defname{hereditarily justified} by $n_p$ if there are nodes $n_1, n_2, \ldots n_{p-1}$ in the sequence such that for all $i\in 0..p-1$, $n_i$ points to $n_{i+1}$.
We write $t^\omega$ to denote the last occurrence of $t$ and $\ip(t)$ for the immediate prefix of $t$ obtained by removing $t$'s last node.

We define a filtering operation on sequences of nodes:
\begin{definition}[Hereditary filtering]
Let $s$ be a justified  sequence of nodes from $\tau(M)$
and $n$ be an occurrence in $t$ of some node $n \in N_{\sf spawn}$.

We write $s \upharpoonright n$ to denote the subsequence of $s$ constituted of nodes that are hereditarily justified by $n$, where the pointer's target of all occurrences of free variable nodes in $t$ are set to $n$ (instead of $t$'s first node).

Thus $s \upharpoonright n$ is a valid justified sequence of nodes of the tree $\tau(\kappa(n))$.
\end{definition}


\begin{lemma}
\label{lem:filtercontinous}
The filtering function $\_ \upharpoonright n$ defined on the cpo of justified sequences ordered by the prefix ordering
is continuous.
\end{lemma}
\begin{proof}
Clearly $\_ \upharpoonright n$ is monotonous.
Suppose that $(t_i)_{i\in\omega}$ is a chain of justified sequence of nodes. Let $u$ be a finite prefix of $(\bigvee t_i) \filter n$.
Then $u = s \filter n$ for some finite prefix $s$ of $\bigvee t_i$. Since $s$ is finite we must have $s \sqsubseteq t_j$ for some $j\in\omega$.
Therefore $u \sqsubseteq t_j \filter n \sqsubseteq \bigvee (t_j \filter  n)$.
This is valid for any finite prefix $u$ therefore $(\bigvee t_i) \filter  n \sqsubseteq \bigvee (t_j \filter n)$.
\end{proof}



The notion of \defname{P-view} $\pview{t}$ of a justified sequence
of nodes $t$ is defined the same way as the P-view of a justified
sequences of moves in Game Semantics:

\begin{definition}[P-view of justified sequence of nodes]
The P-view of a justified sequence of nodes $t$ of $\tau(M)$, written $\pview{t}$, is defined as follows:
\begin{eqnarray*}
 \pview{\epsilon} &=&  \epsilon \\
 \pview{s \cdot n }  &=&  \pview{s} \cdot n \qquad \mbox{for $n \notin N_\lambda$, }\\
 \pview{\Pstr{ s \cdot (m){m} \cdot \ldots \cdot (lmd-m,25){\lambda \overline{\xi}}}} &=&
        \Pstr{ \pview{s} \cdot (m2){m} \cdot (lmd2-m2,60){\lambda \overline{\xi}} } \\
 \pview{s \cdot r }  &=&  r
\end{eqnarray*}
where $r$ is the root of the tree $\tau(M)$.

The equalities in the definition determine pointers implicitly. For
instance in the second clause, if in the left-hand side, $n$ points
to some node in $s$  that is also present in $\pview{s}$ then in the
right-hand side, $n$ points to that occurrence of the node in
$\pview{s}$.
\end{definition}

The O-view of $s$, written $\oview{s}$, is defined dually.
\begin{definition}[O-view of justified sequence of nodes]
The O-view of a justified sequence of nodes $t$ of $\tau(M)$, written $\oview{t}$, is defined as follows:
\begin{eqnarray*}
 \oview{\epsilon} &=&  \epsilon \\
 \oview{s \cdot \lambda \overline{\xi} }  &=&  \oview{s} \cdot \lambda \overline{\xi} \\
 \oview{\Pstr{s \cdot (m){m} \cdot \ldots \cdot (x-m,30){x}}} &=&
    \Pstr{ \oview{s} \cdot (m2){m} \cdot (n2-m2,60){x} } \qquad \mbox{ for $x \in N_{\sf var}$ }\\
 \oview{s \cdot n }  &=&  n \qquad \mbox{ for $x \in N_@ \union N_\Sigma$ }
\end{eqnarray*}
\end{definition}

We borrow some game semantic terminology:
\begin{definition} A justified sequence of nodes $s$ satisfies:
\begin{itemize}[-]
\item \defname{Alternation} if for any two consecutive nodes in $s$, one is a $\lambda$-node
and the other is not;
\item \defname{P-visibility} if every variable node in $s$ points to a node occurring in the P-view a that point;
\item  \defname{O-visibility} if every lambda node in $s$ points to a node occurring in the O-view a that point.
\end{itemize}
\end{definition}

\begin{property}
\label{proper:pview_visibility}
The P-view (resp. O-view) of a justified sequence verifying P-visibility (resp. O-visibility)
is a well-formed justified sequence verifying P-visibility (resp. P-visibility).
\end{property}
This is proved by an easy induction.

\subsection{Adding value-leaves to the computation tree}
\label{sec:adding_value_leaves}

We now add another ingredient to the computation tree defined in
the previous section. Let $\mathcal{D}$
denote the set of values of base type $o$.  We add
\defname{value-leaves} to $\tau(M)$ as follows: Every node $n \in \tau(M)$ has one child leaf labelled $v_n$ for every possible value $v \in \mathcal{D}$.
We write $V$ for the set of nodes and leaves of
the computation tree.  For $\$$ ranging in $\{@, \lambda, var \}$,
we write $V_\$$ to denote the set $N_\$ \union \{ v_n \ | \ n \in
N_\$, v \in \mathcal{D} \}$.

%If $n$ is a $\lambda$-node then its value-leaves are numbered from $1$ onwards.
%If $n$ is a variable or constant node then its children nodes are numbered from $1$ to $arity(n)$ and
%its value-leaves are numbered from $arity(n)+1$ onwards.
%If $n$ is an application node then its value-leaves are numbered from $1$ onwards.

Everything that we have defined for computation tree can be lifted
to this new version of computation tree. The node order of a
value-leaf is defined to be $0$. The enabling relation $\vdash$ is
extended so that every leaf is enabled by its parent node. The
definition of justified sequence does not change.
When representing a link in a justified sequence going from a value-leaf $v_n$ to a node $n$,
we label the link with $v$:
$$
\Pstr{ (n){n} \cdot \ldots \cdot (vn-n,40:v){v_n} }
$$

For the definition
of P-view, O-view and visibility, value-leaves are treated as
$\lambda$-nodes if they are at odd level in the computation tree and
as variable nodes if they are at an even level.

From now the term ``computation tree'' refers to this extended
definition.
\vspace{10pt}

We say that a node $n$ in of a justified sequence of nodes is
\defname{matched} by the value-leaf $v_n$ if there is an occurrence of $v_n$ for some value $v$ in the
sequence that points to $n$, otherwise we say that $n$ is
\defname{unmatched}. The last unmatched node is called the
\defname{pending node}.  A justified sequence of nodes is
\defname{well-bracketed} if each value-leaf occurring in it is justified by the pending node at that point.
If $t$ is a traversal then we write
$?(t)$ to denote the subsequence of $t$ consisting only of unmatched
nodes.

\subsection{Traversal of the computation tree}
\label{subsec:traversal}
A \emph{traversal} is a justified sequence of nodes of the computation tree where each node indicates a step that is taken during the evaluation of the term.

\subsubsection{Traversals for simply-typed $\lambda$-terms}

We first consider the simply-typed $\lambda$-calculus without interpreted constants.
Everything remains valid in the presence of \emph{uninterpreted} constants as we can just
consider them as free variables.

We define the notion of traversal over the computation tree $\tau(M)$.
We will then we show how to extend the notion of traversal to more general settings with interpreted constants.

\begin{definition}[Traversals for simply-typed $\lambda$-terms] \rm
\label{def:traversal} The set $\travset(M)$ of \defname{traversals}
over $\tau(M)$ is defined by induction over the following rules:

\noindent \emph{Initialization rules}
\begin{description}
\item[\rulenamet{Empty}] $\epsilon \in \travset(M)$.
\item[\rulenamet{Root}] The single-node sequence $r$, where $r$ denotes the root of $\tau(M)$, is a traversal.
%$ r \in \travset(M)$.
\end{description}

\noindent \emph{Structural rules}
\begin{description}
\item[\rulenamet{Lam}] If $t \cdot \lambda \overline{\xi}$ is a traversal then so is
$t \cdot \lambda \overline{\xi} \cdot n$ where $n$ denotes $\lambda
\overline{\xi}$'s child.

Moreover if $n$ is a variable node then it
points to the only
occurrence of its enabler that is still present in $\pview{t
\cdot \lambda \overline{\xi}}$.
In particular, if $n$ is a free variable node then $n$ points to the first node of $t$ (the root). (Prop. \ref{prop:pviewtrav_is_path} will show that indeed $n$'s enabler occurs exactly once in the P-view since P-views correspond to paths in the tree.)

\item[\rulenamet{App}] If $t \cdot @$ is a traversal then so is \Pstr[0.4cm]{t \cdot (m) @  \cdot (n-m,40:0) n}.
%{\em i.e.}~the next visited node is the $0^{th}$ child node of
%@: the node corresponding to the operator of the application.
\end{description}

\noindent \emph{Input-variable rules}
\begin{description}
\item[\rulenamet{InputVar$^{val}$}] If $t_1 \cdot x \cdot t_2$ is a traversal
with $x \in N_{\sf var}^{\upharpoonright r}$ and $?(t_1 \cdot x
\cdot t_2)=?(t_1) \cdot x$ then so is \Pstr[0.4cm]{t_1 \cdot
(x){x} \cdot t_2 \cdot (xv-x,38:v){v_x} } for all $v \in
\mathcal{D}$.

\item[\rulenamet{InputVar}] If $t_1 \cdot x \cdot t_2$ is a traversal with
  $x \in N_{\sf var}^{\upharpoonright r}$ and $x$ is the pending node in $t$ ($?(t_1 \cdot x \cdot
  t_2)=?(t_1) \cdot x$) then so is $t_1 \cdot x \cdot t_2 \cdot
  n$ for any $\lambda$-node $n$ whose parent occurs in
  $\oview{t_1 \cdot x}$, $n$ pointing to some occurrence of its
  parent node in $\oview{t_1 \cdot x}$.
\end{description}

\noindent \emph{Copy-cat answer rules}
\begin{description}
\item[\rulenamet{Answer-@-$\lambda$}]
  If \Pstr{t \cdot (app){@} \cdot (lz-app,60:0){\lambda
\overline{z}}  \ldots  (lzv-lz,60:v){v}_{\lambda \overline{z}} }
is a traversal then so is \Pstr[0.6cm]{t \cdot (app){@} \cdot
(lz-app,60){\lambda \overline{z}} \ldots
(lzv-lz,60:v){v}_{\lambda \overline{z}} \cdot
(appv-app,45:v){v}_@}.

\item[\rulenamet{Answer-$\lambda$-@}] If \Pstr[0.4cm]{t \cdot \lambda \overline{\xi} \cdot (x){@}  \ldots   (xv-x,50:v){v}_@}
is a traversal then so is \Pstr[0.5cm]{t \cdot (lmd){\lambda
\overline{\xi}} \cdot (x){@}  \ldots  (xv-x,50:v){v}_@  \cdot
(lmdv-lmd,30:v){v}_{\lambda \overline{\xi}} }.

\item[\rulenamet{Answer-var-$\lambda$}] If \Pstr[0.4cm]{t \cdot y \cdot (lmd){\lambda \overline{\xi}}
\ldots (lmdv-lmd,50:v){v}_{\lambda \overline{\xi}} } is a
traversal for some variable $y\not\in N_{\sf var}^{\upharpoonright
r}$ then so is \Pstr[0.7cm]{t \cdot (y){y} \cdot (lmd){\lambda
\overline{\xi}} \ldots (lmdv-lmd,30:v){v}_{\lambda
\overline{\xi}}  \cdot (vy-y,50:v){v}_y }.

\item[\rulenamet{Answer-$\lambda$-var}] If \Pstr[0.4cm]{t \cdot \lambda \overline{\xi} \cdot (x){x}  \ldots   (xv-x,50:v){v}_x}
is a traversal then so is \Pstr[0.5cm]{t \cdot (lmd){\lambda
\overline{\xi}} \cdot (x){x}  \ldots  (xv-x,50:v){v}_x  \cdot
(lmdv-lmd,30:v){v}_{\lambda \overline{\xi}} }.
\end{description}

\begin{description}
\item[\rulenamet{Var}]
If \Pstr[0.5cm]{t' \cdot (n){n} \cdot (lx){\lambda \overline{x}}
    \ldots (x-lx,50:i){x_i} } is a traversal for some variable
    $x_i$ not in $N_{\sf var}^{\upharpoonright r}$ then
so is \Pstr[0.6cm]{ t' \cdot (n){n} \cdot
    (lx){\lambda \overline{x}}  \ldots (x-lx,30:i){x_i}  \cdot
    (letai-n,40:i){\lambda \overline{\eta_i}}
     }.
\end{description}
A traversal that cannot be extended by any rule is said to be \emph{maximal}.
\end{definition}


A traversal always starts by visiting the root. Then it mainly
follows the structure of the tree.

The \rulenamet{Var} rule is particular and needs further explanation.
This rule permits the traversal to jump across the computation tree. The idea is that after visiting a
non-input variable node $x$, a jump can be made to the node corresponding to
the subterm that would be substituted for $x$ if all the
$\beta$-redexes occurring in the term were to be reduced.


Let $\lambda \overline{x}$ be $x$'s binder and suppose $x$ is the $i$th variable in $\overline{x}$.
The binding node necessarily occurs previously in the traversal (this will be proved in Prop. \ref{prop:pviewtrav_is_path}). Since $x$ is not hereditarily justified by the root, $\lambda \overline{x}$ is not the root of the tree and therefore it is not the first node of the traversal.
We do a case analysis on the node preceding $\lambda \overline{x}$:
    \begin{itemize}[-]
    \item If it is an @-node then $\lambda \overline{x}$ is necessarily the first child node of that node
    and it has has exactly $|\overline{x}|$ siblings:
    $$\pstree[levelsep=7ex]{\TR{\stackrel{\vdots}{@}}}
    {   \pstree[linestyle=dotted,levelsep=4ex]{\TR{\lambda \overline{x}}\treelabel{0}}
            {\TR{x }}
        \tree{\lambda \overline{\eta_1}}{\vdots}\treelabel{1}
        \TR[edge=\dotedge]{}
        \tree{\lambda \overline{\eta_i}}{\vdots}\treelabel{i}
        \TR[edge=\dotedge]{}
        \tree{\lambda \overline{\eta_{|x|}}}{\vdots}\treelabel{|x|}
    }
    $$
    In that case, the next step of the traversal is a jump to $\lambda \overline{\eta_i}$ -- the $i$th child of
    @ -- which corresponds to the subterm that would be substituted for $x$ if the $\beta$-reduction was
    performed:
    $$\Pstr[19pt]{ t' \cdot
            (n){@} \cdot
            (lx){\lambda \overline{x}} \cdot \ldots \cdot
            (x-lx,40:i){x} \cdot
            (mi-n,40:i){\lambda \overline{\eta_i}} \cdot \ldots
            \in {\travset(M)}   }
    $$

    \item If it is a variable node $y$, then
    the node $\lambda \overline{x}$ was necessarily added to the traversal $t_{\leq y}$ using the \rulenamet{Var} rule (see proposition \ref{prop:pviewtrav_is_path}(i)).
    Therefore $y$ is substituted by the term $\kappa(\lambda \overline{x})$ during the evaluation of the term.

    Consequently, during reduction, the variable $x$ will be substituted by the subterm represented by
    the $i$th child node of $y$. Hence the following justified sequence is also a traversal:
    $$\Pstr[18pt]{ t' \cdot
            (y){y} \cdot
            (lx){\lambda \overline{x}} \cdot \ldots \cdot
            (x-lx,40:i){x} \cdot
            (mi-y,40:i){\lambda \overline{\eta_i}} \cdot \ldots
    }
    $$
    \end{itemize}

\begin{remark}
Our notions of computation tree and traversal differ slightly from \cite{OngLics2006}:
\begin{itemize}[-]
    \item In \cite{OngLics2006} computation trees can have uninterpreted first-order constants. But as we have already observed, uninterpreted constants can be just regarded as free variables thus we do not lose any expressivity here.

    \item In \cite{OngLics2006}, constants are restricted to order one at most since computation tree
    are used to model computation of tree structures. Here we don't need this restriction (as long as constants are uninterpreted - so we can regard them as free variables).


    \item In our setting, we have to deal with \emph{free} variables.
    To model free variables we need the traversal rules \rulenamet{InputVar$^{val}$}, \rulenamet{InputVar}
    as well as the copy-cat answer rules. Whereas in \cite{OngLics2006}, the rule called \rulenamet{Sig} suffices to model the first-order constants necessary to construct tree structures.

    \item In our setting, the introduction of value-leaves
    is necessary in order to model free variables as well as interpreted constants. (We will use them to model the constants of \pcf\ and \ialgol).
    \end{itemize}
\end{remark}

\begin{example}
Consider the following computation tree:
$$\tree{\lambda}
{
    \tree{@}
    {
        \pstree[levelsep=8ex,linestyle=dotted]{\TR{\lambda y}\treelabel{0} }
        {
            \pstree[levelsep=8ex]{\TR{y}}
            {
                \tree{\lambda \overline{\eta_1}}{\vdots} \treelabel{1}
                \TR[edge=\dotedge]{}
                \tree{\lambda \overline{\eta_i}}{\vdots}\treelabel{i}
                \TR[edge=\dotedge]{}
                \tree{\lambda \overline{\eta_n}}{\vdots}\treelabel{n}
            }
        }
        \pstree[levelsep=6ex,linestyle=dotted]{\TR{\lambda \overline{x}}\treelabel{1}}{ \tree{x_i}{\TR{} \TR{} } }
    }
}
$$
An example of traversal of this tree is:
\vspace{0.3cm}
$$ \Pstr{ \lambda \cdot
            (app){@}  \cdot
            (ly){\lambda y} \cdot \ldots \cdot
            (y-ly,40:1){y} \cdot
            (lx-app,50:1){\lambda \overline{x}} \cdot \ldots \cdot
            (x-lx,40:i){x_i} \cdot
            (leta-y,50:i){\lambda \overline{\eta_i} } \cdot \ldots
        }$$
\end{example}

\subsubsection{Traversals for interpreted constants}

\begin{definition}[Well-behaved traversal rule]
\label{def:wellbehaved_traversal} A traversal rule is
\defname{well-behaved} if it can be stated under the following form:
$$\rulef{t = t_1\cdot n \cdot t_2 \in \travset \quad ?(t) = ?(t_1) \cdot n \quad P(t)}
  { \stackrel{  \rule{0pt}{3pt} }{\Pstr[5pt]{ t' = t_1\cdot (n){n} \cdot t_2 \cdot (m-n,35){m} \in \travset}}
   }
    \ m\in S(t)
   $$
such that:
\begin{enumerate}[i.]
  \item $n$ is a variable or a constant node ($n \in N_{\Sigma}\union N_{\sf var}$);
  \item $P$ expresses some condition on $t$;
  \item for every traversal $t$, $S(t)$ is some subset of $E(n)$, the set of children $\lambda$-nodes and value-leaves of $n$.
  If $S(t)$ has more than one element then the rule is non-deterministic.
\end{enumerate}
\end{definition}
Note that if $t$ is well-bracketed then $t'$ is also well-bracketed
and if $?(t)$ satisfies alternation and visibility then so does
$?(t')$.


\begin{example} The rule (InputVar$^{val}$) is an example of non-deterministic well-behaved traversal rule for which $S(t)$ is exactly the set of all children value-leaves of $n$:
$S(t) = \{ v_n \ | \ v \in \mathcal{D} \} $.
However (InputVar) is not well-behaved since it can jump to any node in the O-view at that point and not necessarily to a children node of the last pending node.
\end{example}

In the presence of higher-order interpreted constants, additional rules must be specified to indicate how
the constant nodes should be traversed in the computation tree. These rules
are specific to the language that is being studied.
In the last section of this chapter we will define such traversals for the interpreted constants of
\pcf\ and \ialgol.

From now on, we consider a simply-typed $\lambda$-calculus language extended with
higher-order interpreted constants for which some constant traversal rules have been defined
and we take the following condition as a prerequisite:
\begin{center}
  \textbf{(Condition WB)} The constant traversal rules are well-behaved.
\end{center}


\subsubsection{Some properties of traversals}

\begin{proposition}[counterpart of proposition 6 from \cite{OngHoMchecking2006}]
\label{prop:pviewtrav_is_path}
Let $t$ be a traversal. Then:
\begin{itemize}
\item[(i)] $t$ is a well-defined and well-bracketed justified sequence;
\item[(ii)] $t$ is a well-defined justified sequence verifying alternation, P-visibility and O-visibility;
\item[(iii)] If $t^\omega \in N$ {\it i.e.}~$t$'s last node is not a value-leaf, then $\pview{t}$ is the path in the computation tree going from the root to the node $t^\omega$.
\end{itemize}
\end{proposition}

This is the counterpart of proposition 6 from
\cite{OngHoMchecking2006} which is proved by induction on the
traversal rules. This proof can be easily adapted to take into
account the constant rules (using the assumption that constants
rules are well-behaved) and the presence of value-leaves in the
traversal.
\begin{proof}
The proof of (i), (ii) and (iii) is done simultaneously by induction on the traversal rules. We consider the rules \rulenamet{Var} and \rulenamet{Lam} only.

Rule \rulenamet{Var}: we just give a partial proof of (i). See proposition 6 from \cite{OngHoMchecking2006} for the details of (i), (ii) and (iii). We have to show that in the second case of the \rulenamet{Var} rule, where $p$ is a variable node $y$, the node $\lambda \overline{x}$ has necessarily been added to the traversal $t_{\leq y}$ using the \rulenamet{Var} rule. This is immediate since if the rule \rulenamet{InputVar} was used to produce $t_{<y} \cdot y \cdot \lambda \overline{x}$ this would imply that $\lambda \overline{x}$ is hereditarily justified by the root which in turn implies that $x_i$ is an input-variable which contradicts \rulenamet{Var}'s hypothesis.

Rule \rulenamet{Lam}: we need to show that $n$'s enabler occurs only once in the P-view at that point. By the induction hypothesis we have (by (iii)) that $\pview{t \cdot \lambda \overline{\xi}}$ is a path in the computation tree from the root to $\lambda \overline{\xi}$. $n$'s enabler occurs only once in this path: it is precisely it's binding node. Therefore the traversal $t \cdot \lambda \overline{\xi} \cdot n$ is well-defined and $t \cdot \lambda \overline{\xi} \cdot n$ satisfies P-visibility. Thus (i) and (ii) are verified. Furthermore $n$ is a child of $\lambda \overline{\xi}$ therefore (iii) also holds.
\end{proof}

%In particular to prove that the copy-cat rules are well-defined, one needs to ensure that
%if the last two unmatched nodes are $y$ and $\lambda \overline{\xi}$ in that order, for some non input-variable node $y$ then necessary
%      $y$ and $\lambda \overline{\xi}$ are consecutive nodes in the traversal.
%    This is because in a traversal, a non input-variable $y$ is always followed by a lambda node and whenever this lambda node is answered
%    there is only one way to extend the traversal : by using the copy cat rule to answer the $y$ node.

\begin{definition}
The \defname{reduction of a traversal} $t$ is define as the subsequence $ t
\filter r$ where $r$ denotes the first node in $t$ (which is necessarily $\tau(M)$'s root).
\end{definition}
The effect of this transformation is the elimination of the
``internal nodes'' of the computation. Since @-nodes and $\Sigma$-constants do not have pointers, the
reduction of traversal contains only nodes in $N_\lambda \union
N_{\sf var}$.

We define the set
$$\travset(M)^{\upharpoonright r} = \{ t  \upharpoonright r \ | \  t  \in \travset(M) \} \ . $$




\begin{lemma}
\label{lem:var_followedby_child} Suppose $M$ is in $\beta$-normal
form. Let $t \travset(M)$. If
$\Pstr{ t = u_1 \cdot (m){m} \cdot u_2 \cdot (n-m,30){n} }$
 where $m \in (N_{\sf var} \union N_{\Sigma}) \setminus (N^{\upharpoonright r}_{var} \union N^{\upharpoonright r}_{\Sigma})$
then $u_2 = \epsilon$.
\end{lemma}
\begin{proof}
By case analysis on the rule used to visit the node
$n$ in $t$. The only relevant rules are (Var), (Answer-var), (InputVar$^{val}$), (InputVar)
and the constant rules.
Since the term is in $\beta$-normal form, there is no @-node in $\tau(M)$ and therefore (Var) cannot be used.
Since $m$ is not hereditarily justified by the root, it is not an input-variable and therefore the rules
(InputVar$^{val}$) and (InputVar) cannot be used.
For the rule (Answer-var) the result follows from the well-bracketedness of traversals.
For constant rules, the result follows from the well-behaviour of constant rules (condition WB).
\end{proof}

\begin{lemma}[View of a traversal reduction]
\label{lem:redtrav_trav} Suppose that $M$ is a $\beta$-normal term and let $t$ be a traversal of $\tau(M)$ then
\begin{itemize}
\item[(i)] $ \pview{t \upharpoonright  r } = \pview{t} \upharpoonright r$\ ;
\item[(ii)] if $t^\omega \in N^{\filter r}$ {\it i.e.}~$t$'s last node is hereditarily justified by $r$, then
    $\oview{t \upharpoonright r } = \oview{t}$\ .
\end{itemize}
\end{lemma}
In the safe lambda calculus without interpreted constants this lemma
follows immediately from the fact that $\travset(M) =
\travset(M)^{\upharpoonright r }$. Here we prove the result in a
more general setting of a calculus extended with interpreted
constants whose corresponding traversal rules are
\emph{well-behaved}.


\begin{proof}
(i) By induction. It is trivially true for the empty
traversal and for the traversal $t = r$. Step case: consider a traversal $t$ and
suppose that the property (i) is verified for all traversal shorter
than $t$:
\begin{itemize}[-]
\item If $t = t' \cdot n$ with $n \in N_{\sf var} \union N_{\Sigma}$ then:
    \begin{align*}
    \pview{t} \upharpoonright  r
&= \pview{t' \cdot n} \upharpoonright  r & (\mbox{definition of } t)\\
        &= (\pview{t'} \cdot n) \upharpoonright  r  & (\mbox{P-view computation}) \\
        &= \pview{t'} \upharpoonright  r  \cdot (n \upharpoonright  r)            & (\mbox{def. of filtering $\upharpoonright$}) \\
        &= \pview{t' \upharpoonright  r } \cdot (n \upharpoonright  r)           & (\mbox{induction hypothesis}) \\
        &= \pview{t' \upharpoonright  r \cdot (n \upharpoonright  r) } & (\mbox{P-view computation, $n \in N_{\sf var} \union N_{\Sigma}$}) \\
        &= \pview{(t' \cdot n ) \upharpoonright  r  }           & (\mbox{def. of filtering $\upharpoonright$}) \\
        &= \pview{t \upharpoonright  r  }
 & (\mbox{definition of } t).
    \end{align*}


\item If $\Pstr{ t =  t' \cdot (m){m} \cdot  u \cdot (lmd-m,30){n}}$ with $n\in N_\lambda \setminus N^{\upharpoonright r}_\lambda$ then we have $u = \epsilon$ by lemma
    \ref{lem:var_followedby_child} and:
        \begin{align*}
        \pview{t} \upharpoonright  r
        &= \pview{\Pstr{t' \cdot (m){m} \cdot (n-m,60){n}}} \upharpoonright  r
                                                        & (u=\epsilon)\\
        &= (\Pstr{\pview{t'} \cdot (m){m} \cdot (lmd-m,60){n}} ) \upharpoonright  r
                                                        & (\mbox{P-view computation}) \\
        &= \pview{t'} \upharpoonright  r                & (m, n \not\in N^{\upharpoonright r}) \\
        &= \pview{t' \upharpoonright  r }               & \mbox{(induction hypothesis)} \\
        &= \pview{ (\Pstr{t' \cdot (m){m} \cdot (lmd-m,40){n}}) \upharpoonright r }
                                                        & (m, n \not\in N^{\upharpoonright r}) \\
        &= \pview{ t \upharpoonright r }             & \mbox{(def. of $t$ \& $u = \epsilon$).}
        \end{align*}

\item If $\Pstr{ t =  t' \cdot (m){m} \cdot u \cdot (lmd-m,30){n} }$ with $n\in N^{\upharpoonright r}_\lambda$ then:
        \begin{align*}
        \pview{t} \upharpoonright  r
        &= \pview{\Pstr{t' \cdot (m){m} \cdot u \cdot (n-m,40){n}}} \upharpoonright  r
                                                              & (\mbox{definition of } t)\\
        &= (\Pstr{\pview{t'} \cdot (m){m} \cdot  (lmd-m,60){n}}) \upharpoonright  r
                                                              & (\mbox{P-view computation}) \\
        &= \Pstr{ \pview{t'} \upharpoonright  r \cdot (m){m} \cdot  (lmd-m,60){n} }
                                                              & (m, n \in N^{\upharpoonright r}) \\
        &= \Pstr{ \pview{t'\upharpoonright r}  \cdot (m){m} \cdot  (lmd-m,60){n} }
                                                              & \mbox{(induction hypothesis)} \\
        &= \pview{ \Pstr{t' \upharpoonright r \cdot (m){m} \cdot {(u \upharpoonright r)} \cdot (lmd-m,35){n}}}
                                                           & (\mbox{P-view computation}) \\
        &= \pview{ (\Pstr{t' \cdot (m){m} \cdot u \cdot (lmd-m,35){n}}) \upharpoonright r }
                                                           & (m, n \in N^{\upharpoonright r}) \\
        &= \pview{ t \upharpoonright r }                & \mbox{(def. of $t$).}
        \end{align*}
\end{itemize}
(ii) By a straightforward induction similar to (i).
\end{proof}

\begin{remark}
\label{rem:inputvar}
Using the previous lemma we observe that in the definition of the rule \rulenamet{InputVar} we have
$n \in N_\lambda^{\filter r}$. Indeed,
$\oview{t_1 \cdot x } = \oview{ (t_1 \cdot x) \filter r}$ therefore $n$
is hereditarily enabled by $r$.
\end{remark}

\begin{lemma}[Traversal of $\beta$-normal terms]
\label{lem:betaeta_trav}
Let $M$ be a $\beta$-normal term, $r$ be the root of the tree $\tau(M)$ and
$t$ be a traversal of $\tau(M)$.
For any node $n$ occurring in $t$:
\begin{eqnarray*}
r \mbox{ does not hereditarily justify } n  \  \iff \   n \mbox{ is
hereditarily justified by some node in } N_\Sigma.
\end{eqnarray*}
\end{lemma}
\begin{proof}
 In a computation tree, the only nodes that do not have justification pointer are:
the root $r$, @-nodes and $\Sigma$-constant nodes. But since $M$ is
in $\beta$-normal form, there is no @-node in the computation tree.
Hence nodes are either hereditarily justified by $r$ or hereditarily
justified by a node in $N_\Sigma$. Moreover $r$ is not in $N_\Sigma$
therefore the ``or'' is exclusive : a node cannot be hereditarily
justified at the same time by $r$ and by some node in $N_\Sigma$.
\end{proof}


\section{Game semantics correspondence}
\label{sec:gamesemcorresp}

 We are working in the general setting of an applied
simply-typed $\lambda$-calculus with a given set of higher-order
constants $\Sigma$. The operational semantics of these constants is
given by certain reduction rules. We assume that a fully abstract
model of the calculus is provided by means of a category of
well-bracketed games. For instance, if $\Sigma$ consists
of the \pcf\ constants then we consider the traditional
category of games and innocent well-bracketed strategies
\cite{hylandong_pcf,abramsky94full}.


In the literature, a strategy is commonly defined as a set of plays closed by
even-length prefixing. However, for our purpose here, it is more convenient to represent strategies using \emph{prefix-closed} set of plays. This saves us from considerations on the parity of traversal length when
showing the correspondence between traversals and game semantics.
 For the rest of the section we fix a simply-typed term $\Gamma \vdash M :T$. We write $\sem{\Gamma \vdash M : T}$ for its strategy denotation (in the standard cartesian closed category of games and innocent strategies \cite{abramsky94full, hylandong_pcf}). We use the notation $\prefset(S)$ to denote the prefix-closure of the set $S$.

\subsection{Relating computation trees and games}
Let us first study an example:
\subsubsection{Example}
Consider the following term $M \equiv \lambda f z . (\lambda g x . f (f x)) (\lambda y. y) z$ of type $(o \typear o) \typear o \typear o$.
Its $\eta$-long normal form is $\lambda f z . (\lambda g x . f (f x)) (\lambda y. y) (\lambda .z)$.
The computation tree is:

$$
\tree{\lambda f z}
{ \tree{@}
    {
        \tree{\lambda g x}
            { \tree{f}{   \tree{\lambda}{ \tree{f}{  \tree{\lambda}{\TR{x}}} }  }
            }
        \tree{\lambda y}{\TR{y}}
        \tree{\lambda}{\TR{z}}
    }
}
$$

The arena for the type $(o \typear o) \typear o \typear o$ is:
$$\tree{q^1}
{
    \tree{q^3}
        {  \tree{q^4}
                {\TR{a^4_1} \TR{\ldots}}
            \TR{a^3_1} \TR{\ldots} }
    \tree{q^2}
    { \TR{a^2_1} \TR{a^2_2}\TR{\ldots} }
    \TR{a_1} \TR{a_2}\TR{\ldots}
}
$$

\newlength{\yNull}
\def\bow{\quad\psarc{->}(0,\yNull){1.5ex}{90}{270}}

The figure below represents the computation tree (left) and the
arena (right). The dashed line defines a partial function $\psi$
from the set of nodes in the computation tree to the set of moves.
For simplicity, we now omit answers moves when representing arenas.
$$
\tree{ \Rnode{root} {\lambda f z}^{[1]} }
     {  \tree{@^{[2]}}
        {   \tree{\lambda g x ^{[3]}}
                { \tree{\Rnode{f}{f^{[6]}}}{  \tree{\Rnode{lmd}\lambda^{[7]}}{ \tree{\Rnode{f2}{f^{[8]}}} {\tree{\Rnode{lmd2}\lambda^{[9]}}{\TR{x^{[10]}}}}}  }
                }
            \tree{\lambda y ^{[4]}}{\TR{y}}
            \tree{\lambda ^{[5]}}{\TR{\Rnode{z}z}}
        }
    }
\hspace{3cm}
  \tree[levelsep=12ex]{ \Rnode{q1}q^1 }
    {   \pstree[levelsep=4ex]{\TR{\Rnode{q3}q^3}}{\TR{\Rnode{q4}q^4}}
        \TR{\Rnode{q2}q^2}
        \TR{\Rnode{q5}q^5}
    }
\psset{nodesep=1pt,arrows=->,arcangle=-20,arrowsize=2pt 1,linestyle=dashed,linewidth=0.3pt}
\ncline{->}{root}{q1} \aput*{:U}{\varphi}
\ncarc{->}{z}{q2}
\ncline{->}{f}{q3}
\ncline{->}{lmd}{q4}
\ncline{->}{f2}{q3}
\ncline{->}{lmd2}{q4}
$$

Consider the justified sequence of moves $s \in \sem{M}$:
 $$s = \Pstr[0.6cm][5pt]{(q1){q}^1\ (q3-q1,60){q}^3\ (q4-q3,60){q}^4\ (q3b-q1){q}^3\ (q4b-q3b,60){q}^4\ (q2-q1,30){q}^2 }
\in \sem{M}$$

There is a corresponding justified sequence of nodes in the computation tree:
$$r = \Pstr[0.8cm]{
        (q1){\lambda f z} \cdot
        (q3-q1,60){f}^{[6]} \cdot
        (q4-q3,60){\lambda^{[7]}} \cdot
        (q3b-q1,60){f}^{[8]} \cdot
        (q4b-q3b,50){\lambda^{[9]}} \cdot
        (q2-q1,50){z} }$$
such that $s_i = \psi(r_i)$ for all $i < |s|$.

The sequence $r$ is in fact the reduction of the following
traversal:
$$t = \Pstr[1.1cm]{ (q1){\lambda f z} \cdot
            (n2){@^{[2]}} \cdot (n3-n2,60){\lambda g x^{[3]}} \cdot
            (q3-q1,60){f}^{[6]} \cdot (q4-q3,60){\lambda^{[7]}} \cdot
            (q3b-q1,40){f}^{[8]} \cdot (q4b-q3b,70){\lambda^{[9]}} \cdot
            (n8-n3,35){x^{[10]}} \cdot
            (n9-n2,30){\lambda^{[5]}} \cdot
            (q2-q1,35){z} }
$$

By representing side-by-side the computation tree and the type arena of a term in $\eta$-normal form we have observed
that some nodes of the computation tree can be mapped to question moves of the arena.
In the next section, we show how to define this mapping in a systematic manner.

\subsubsection{Formal definition}

We now establish formally the relationship between games and computation trees. Suppose $\Gamma \vdash M : T$
is in $\eta$-long normal form. We suppose that computation tree $\tau(M)$
is given by a pair $(V,E)$ where $V$ is the set of vertices of
and $E \subseteq V \times V$ is the parent-child relation. We have $V = N \union VL$ where $N$
and $VL$ are the set of nodes and value-leaves respectively.

\emph{Notations:}
We write $V_\$$ for $N_\$ \union (E(N_\$) \inter VL)$ where $\$$ ranges over $\{@, {\sf var}, \Sigma, {\sf fv} \}$.
Let $\mathcal{D}$ be the set of values of the base type $o$. If $n$ is a node in $N$ then the value-leaves attached to the node $n$ are written $v_n$ where $v$ ranges in $\mathcal{D}$.
Similarly, if $q$ is a question in $\sem{A}$ then the answer moves enabled by $q$ are written $v_q$ where $v$ ranges in $\mathcal{D}$.

\begin{definition}[Mapping from nodes to moves]\hfill
\label{def:phi_psi mapping}

    \begin{itemize}[-]
    \item Let $n$ be a node in $N_\lambda \union N_{\sf var}$ and $q$ be a question move of some game $A$
such that $n$ and $q$ are of type $(A_1,\ldots,A_p,o)$ for some $p\geq 0$. The function $\psi^{n,q}_A$ from $V^{\upharpoonright n}$ to $\sem{A}$ is defined as:
        \begin{eqnarray*}
        \psi^{n,q}_A &=& \{ n \mapsto q \} \union  \{ v_n \mapsto v_q \ | \ v \in \mathcal{D} \}\\
         &&\union \left\{
                        \begin{array}{ll}
                          \emptyset, & \hbox{if $p=0$\ ;} \\
                          \Union_{m \in N | n \vdash_i m} \psi^{m, q^i}_A, & \hbox{if $p\geq1$ and $n\in N_{\lambda}$\ ;} \\
                          \Union_{i=1..p} \psi^{n.i, q^i}_A, & \hbox{if $p\geq1$ and $n\in N_{\sf var}$\ .}
                        \end{array}
                      \right.
        \end{eqnarray*}
        where $\{ q^1, \ldots, q^p \} \union \{ v_q \ | \ v \in \mathcal{D} \}$ is the set of moves enabled by $q$ in $A$ (each $q^i$ being of type $A_i$).

    \item We use the abbreviation $\psi_n$
    for $\psi^{n,m}_{T(n)} : V^{\upharpoonright n} \rightarrow \sem{T(n)}$
    where $m$ denotes $\sem{T(n)}$'s initial move.

    \item Similarly we write $\psi_M$ (or just $\psi$ if this does not cause any ambiguity)
    for $\psi^{r,m}_{\Gamma\rightarrow T}$ where $m$ denote $\sem{\Gamma\rightarrow T}$'s initial move.\footnote{Arenas involved in the game semantics of simply-typed $\lambda$-calculus are all trees: they have a single initial move.}
    \end{itemize}
\end{definition}

It can easily be checked that the domain of definition of $\psi_n$ is indeed the set of nodes that are hereditarily enabled by $n$.

Let us detail a little the definition of $\psi_n$:
\begin{itemize}
\item If $p=0$ then $n$ is a dummy $\lambda$-node or a ground type variable: $\psi_n$ maps $n$ to the initial move $q$.

\item  If $p\geq 1$ and $n \in N_{\lambda}$ with $n$ labelled $\lambda \overline{\xi} = \lambda \xi_1 \ldots \xi_p$ then the sub-computation tree rooted at $n$ and the arena $\sem{T(n)}$ have the following forms (value-leaves and answer moves are not represented for simplicity):
    $$ \tree{ \Rnode{r}\lambda \overline{\xi}  ^{[n]}}
        {
            \tree[levelsep=6ex]{\alpha}
            {   \TR{\ldots} \TR{\ldots} \TR{\ldots}
            }
        }
    \hspace{3cm}
    \tree{ \Rnode{q0}m_n }
        {
            \tree[linestyle=dotted]{q^1}{\TR{} \TR{} }
            \tree[linestyle=dotted]{q^2}{\TR{} \TR{} }
            \TR{\ldots}
            \tree[linestyle=dotted]{q^p}{\TR{} \TR{} }
        }
    \psset{nodesep=1pt,arrows=->,arcangle=-20,arrowsize=2pt 1,linestyle=dashed,linewidth=0.3pt}
    \ncline{->}{r}{q0}
    \ncarc{->}{q2}{z}
    \ncline{->}{q3}{f}
    \ncline{->}{q4}{lmd}
    \ncline{->}{q3}{f2}
    \ncline{->}{q4}{lmd2}
    $$

    For each abstracted variable $\xi_i$ there exists a corresponding question move $q^i$ of the same order in the arena. $\psi_n$ maps each free occurrence of $\xi_i$ in the computation tree to the move $q^i$.

\item If $p\geq 1$ and $n\in N_{\sf var}$ then $n$ is labelled with a variable $x:(A_1,\ldots,A_p,o)$
with children nodes $\lambda \overline{\eta}_1$, \ldots, $\lambda \overline{\eta}_p$. The computation tree $\tau(M)$ rooted at $n$ and the arena $\sem{T(n)}$ have the following forms:
    $$\tree{\Rnode{r}{x^{[n]}}}
        {   \tree{\TR{\lambda \overline{\eta}_1}}{\vdots} \TR{\ldots}
        \tree{\TR{\lambda \overline{\eta}_p }}{\vdots}
        }
    \hspace{3cm}
    \tree{ \Rnode{q0}m_n }
        {
\tree[linestyle=dotted]{\Rnode{q1}{q^1}}{\TR{} \TR{} }
            \tree[linestyle=dotted]{\Rnode{q2}{q^2}}{\TR{} \TR{} }
            \TR{\ldots}
            \tree[linestyle=dotted]{\Rnode{qp}{q^p}}{\TR{} \TR{} }
        }
    \psset{nodesep=1pt,arrows=->,arcangle=-20,arrowsize=2pt 1,linestyle=dashed,linewidth=0.3pt}
    \ncline{->}{r}{q0}
    \ncarc{->}{q2}{z}
    \ncline{->}{q3}{f}
    \ncline{->}{q4}{lmd}
    \ncline{->}{q3}{f2}
    \ncline{->}{q4}{lmd2}
    $$

    and $\psi_n$ maps each node $\lambda \overline{\eta}_i$ to the question move $q^i$.
\end{itemize}

\begin{example}
Take $M = \lambda x . (\lambda g . g x) (\lambda y . y)$ with $x,y:o$
and $g:(o,o)$. The diagram below represents the computation tree
(middle), the arenas $\sem{(o,o), o}$ (left), $\sem{o , o}$ (right),
$\sem{o\rightarrow o}$ (rightmost), $\psi_{\lambda x}$,
$\psi_{\lambda g}$ and $\psi_{\lambda y}$ (dashed-lines).
$$\psset{levelsep=3.5ex}
\pstree{\TR[name=root]{\lambda x}}
{
    \pstree{\TR[name=App]{@}}
    {
            \pstree{\TR[name=lg]{\lambda g}}
                { \pstree{\TR[name=lgg]{g}}{
                        \pstree{\TR[name=lgg1]{\lambda}}
                        { \TR[name=lgg1x]{x}  } } }
            \pstree{\TR[name=ly]{\lambda y}}
                    {\TR[name=lyy]{y}}
    }
}
\rput(4.5cm,-1cm){
  \pstree{\TR[name=A1lx]{q_{\lambda x}}}
        { \TR[name=A1x]{q_x} }
}
\rput(-6cm,-1.5cm){
    \pstree{\TR[name=A2lg]{q_{\lambda g}}}
    {
        \pstree{\TR[name=A2g]{q_g}}
        {  \TR[name=A2g1]{q_{g_1}}   }
    }}
\rput(2.5cm,-1.5cm){
    \pstree{\TR[name=A3ly]{q_{\lambda y}}}
        { \TR[name=A3y]{q_y}
        }
}
\psset{nodesep=1pt,arrows=->,arcangle=-20,arrowsize=2pt 1,linestyle=dashed,linewidth=0.3pt}
\ncline{->}{root}{A1lx} \mput*{\psi_{\lambda x}}
\ncarc{->}{lgg1x}{A1x}
\ncline{->}{lg}{A2lg} \mput*{\psi_{\lambda g}}
\ncline{->}{lgg}{A2g}
\ncline{->}{lgg1}{A2g1}
\ncline{->}{ly}{A3ly} \mput*{\psi_{\lambda y}}
\ncline{->}{lyy}{A3y}
$$
\end{example}

\begin{property} \
\label{proper:psi_properties}
\begin{enumerate}[(i)]
\item $\psi$ maps $\lambda$-nodes to O-questions, variable nodes to
P-questions, value-leaves of $\lambda$-nodes to P-answers and
value-leaves of variable nodes to O-answers;
\item $\psi$ maps a node of a given order to a move of the same order;
\item Let $s \in \travset(M)^{\filter r}$. The P-view (resp. O-view) of $\psi(s)$ and $s$ are computed
identically {\it i.e.}~the set of occurrence positions that must be removed
from each sequences in order to obtain their respective P-view (resp. O-view) is the same for both sequence.
\end{enumerate}
\end{property}
\begin{proof}
(i) and (ii) are direct consequences of the definition.

(iii) Because of (i) and since $t$ and $\psi(t)$ have the
same pointers, the computations of the P-view (resp. O-view) of the
sequence of moves and the P-view (resp. O-view) of the sequence of
nodes follow the same steps.
\end{proof}
The fact that we have defined the order of the root node differently from the order of other $\lambda$-nodes
(Def. \ref{def:nodeorder}) should now make more sense to the reader: this definition permits us to state property (ii).
\smallskip

By extension, we can define the function $\psi_M$ on $\travset(M)^{\filter r}$, the set of justified
sequences of nodes that are hereditarily justified by (the only occurrence of) the root $r$:
\begin{definition}[Mapping sequences of nodes to sequences of moves]
We define the function $\psi_M : \travset(M)^{\filter r} \rightarrow \sem{\Gamma \rightarrow T}$ as follows.
If $s = s_0 s_1 \ldots \in \travset(M)^{\filter r}$ then:
$$\psi_M(s) = \psi_M(s_0)\ \psi_M(s_1)\  \psi_M(s_2) \ldots$$
where $\psi_M(s)$ is equipped with $s$'s pointers.

Thus the pointer-free version of this function is a monoid homomorphism.
\end{definition}


\subsection{Interaction games}
\label{sec:interaction_semantics}

In game semantics, strategy composition is achieved by performing a
CSP-like ``composition + hiding''. It is possible to define an
alternative semantics where the internal moves are not hidden when
performing composition. This semantics is named \emph{revealed
semantics} in \cite{willgreenlandthesis} and \emph{interaction}
semantics in \cite{DBLP:conf/sas/DimovskiGL05}.

In addition to the moves of the standard semantics, the interaction
semantics contains certain internal moves of the computation.
Consequently, the interaction semantics depends on the syntactical
structure of the term and therefore cannot lead to a full
abstraction result. However this semantics will prove to be useful
to identify a correspondence between the game semantics of a term
and the traversals of its computation tree.

Our interaction semantics will be calculated from the $\eta$-normal
form of a term. However we do not want to keep all the internal
moves: we only keep the internal moves that are produced when
composing two subterms of the computation tree joint by an @-node.
This means that when computing the strategy denoting $y N_1 \ldots
N_p$ where $y$ is a variable, we preserve the internal moves of
$N_1$, \ldots, $N_p$ while omitting the internal moves produced by
the copy-cat projection strategy denoting $y$.


\begin{definition} \hfill
\begin{itemize}
\item We call \defname{interaction type tree} or just \defname{interaction type},
a tree whose leaves are labelled with linear simple types and
nodes are labelled with symbol in $\{ ;, \langle \_\ ,\_
\rangle, \otimes, \dagger, \Lambda \}$.


Nodes labelled $;$, $\langle \_\ ,\_ \rangle$ or $\otimes$ are
binary nodes and nodes labelled $\dagger$ or $\Lambda$ are unary
nodes. If $T_1$ and $T_2$ are interaction types we write
$\langle T_1, T_2 \rangle$ to denote the interaction type
obtained by attaching $T_1$ and $T_2$ to a $\langle \_\ ,\_
\rangle$-node. Similarly we use the notations $T_1 \otimes T_2$,
$T_1 ; T_2$, $\Lambda(T_1)$ and $T_1^\dagger$.

\item To every node or leaf we can associate a linear type. We write
    $type(T)$ to denote the type associated to the root node. We
    sometime write the type in exponent {\it e.g.}
    $T^{A\rightarrow B}$ if $type(T) =A\rightarrow B$. This type
    is determined by the structure of the tree as follows:
    \begin{itemize}
    \item If $T$ is a leaf then $type(T)$ is define as the type that labels the leaf;

    \item $type\ (T^{!A \multimap B})^\dagger = !A \multimap !B$;

    \item $type\ \Lambda(T_1^{A \otimes B \multimap C}) = A \multimap (B \multimap C)$

    \item $type\ \langle T_1^{C \multimap A} , T_2^{C \multimap B} \rangle =
    C \multimap A \times B$;

    \item $type\ T_1^{A \multimap B} \otimes T_2^{C \multimap D} = (A \otimes C) \multimap (B \otimes D)$;

    \item $type\ T_1^{A \multimap B};T_2^{B \multimap C} = A \multimap C$.
    \end{itemize}

\end{itemize}

For the interaction type tree to be well-defined, it is required
that types of children nodes are consistent with the meaning of the
parent node; for instance the two children nodes of a ;-node must be
of type $A\multimap B$ and $B\multimap C$.

\end{definition}


Let $T$ be an interaction type tree. Each leaf or node of type $A$
in $T$ can be mapped to the (standard) game $\sem{A}$. By taking the
image of $T$ across this mapping we obtain a tree whose leaves and
nodes are labelled by games. This tree, written $\intersem{T}$, is
called an \defname{interaction game}.

A \defname{revealed strategy} $\Sigma$ on the interaction game $\intersem{T}$ is a compositions of several standard strategies in which certain internal moves are not hidden. Formally:
\begin{definition}[Revealed strategy]
A revealed strategy $\Sigma$ on an interaction game $\intersem{T}$,
written $\Sigma: \intersem{T}$, is an annotated interaction type
tree $T$ where
\begin{itemize}
\item each leaf $\sem{A}$ of $T$ is annotated with a (standard) strategy $\sigma$ on the game
$\sem{A}$;
\item each $;$-node is annotated with a set of indices $U \subseteq \nat$.
\end{itemize}
\end{definition}

The intuition behind this definition is that each $;$-node with children of type $A\multimap B$ and $B\multimap C$ is annotated with a set of indices $U$ indicating which components of $B$ should be uncovered when performing composition.
More precisely, if $B = B_0 \times \ldots \times B_l$ then the revealed strategy built by connecting two revealed strategies $\Sigma_1 : \intersem{A\multimap B}$ and $\Sigma_2 : \intersem{B\multimap C}$
using a $;$-node annotated with $U$ represents the
set of uncovered plays obtained
by performing the usual composition while ignoring and copying the internal moves already in $\Sigma_1$ and $\Sigma_2$ and preserving any internal
move produced by the composition in some component $B_k$ for $k \in U$.

\begin{example}
The diagrams below represent an interaction type tree $T$ (left),
the corresponding interaction game $\intersem{T}$ (middle) and a
revealed strategy $\Sigma$ (right):
$$
\pstree[levelsep=6ex]{\TR{;}}
        {
            \pstree[levelsep=6ex]{\TR{;}}
            { \TR{A\multimap B}
              \TR{B\multimap C}
            }
            \TR{C\multimap D}
        }
\hspace{1cm}
\pstree[levelsep=6ex]{\TR{;}}
        {
            \pstree[levelsep=6ex]{\TR{;}}
            { \TR{\sem{A\multimap B}}
              \TR{\sem{B\multimap C}}
            }
            \TR{\sem{C\multimap D}}
        }
\hspace{1cm}
\pstree[levelsep=6ex]{\TR{;^{\{0\}}}}
        {
            \pstree[levelsep=6ex]{\TR{;^{\{0\}}}}
            { \TR{A\multimap B^{\sigma_1}}
              \TR{B\multimap C^{\sigma_2}}
            }
            \TR{C\multimap D^{\sigma_3}}
        }
$$
\end{example}
A revealed strategy can also be written as an expression, for
instance the strategy represented above is given by the expression
$\Sigma = (\sigma_1 ;^{\{0\}} \sigma_2) ;^{\{0\}} \sigma_3$. We will
use the abbreviation $\Sigma_1 \fatsemi^U \Sigma_2$ for
$\Sigma_1^\dagger ; ^U \Sigma_2$.


\subsubsection{Uncovered play}

The analogous of a play in the interaction semantics is called an
\emph{uncovered play}, it is a play containing internal moves. The
moves are implicitly tagged so that it is possible to retrieve in
which component of the arena of which node/leaf-game the move
belongs to. A given move may belong to several games from different
nodes/leaves of the interaction game.

\begin{definition}
The \defname{set of possible moves} $M_T$ of an interaction game
$\intersem{T}$ is defined as $\mathcal{M}_T/\hspace{-0.5em}\sim_T$,
the quotient of the set $\mathcal{M}_T$ by the equivalence relation
$\sim_T \subseteq \mathcal{M}_T \times \mathcal{M}_T$ defined as follows:
For a single leaf tree $T$ labelled by a type $A$ we define
$\mathcal{M}_T = M_A$ and $\sim_T = id_{M_A}$. For other cases:
    \begin{align*}
        \mathcal{M}_{T^\dagger} &= \mathcal{M}_{T} + M_{type(T^\dagger)}
    &
        \mathcal{M}_{\Lambda(T)} &= \mathcal{M}_{T} + M_{type(\Lambda(T))}
    \\
        \sim_{T^\dagger} &= \left( \sim_{T}
        \union \left(type\ T^\dagger \leftrightarrow  type\ T\right)
        \right)^\star
    &
        \sim_{\Lambda(T)} &= \left( \sim_{T}
        \union \left(type\ \Lambda(T) \leftrightarrow type\ T\right)
        \right)^\star
    \end{align*}
    \begin{align*}
        \mathcal{M}_{\langle T_1^{C^1 \multimap A^1}, T_2^{C^2 \multimap B^2}\rangle}
        &= \mathcal{M}_{T_1} + \mathcal{M}_{T_2} + M_{C \multimap (A \otimes B)}
    \\
         \sim_{\langle T_1^{C^1 \multimap A^1}, T_2^{C^2 \multimap B^2}\rangle} &= \left( \sim_{T_1}
        \union \sim_{T_2} \union (C^1 \leftrightarrow C) \union (C^2 \leftrightarrow C)
        \union (A^1 \leftrightarrow A) \union (B^2 \leftrightarrow B)
        \right)^\star
    \\
    \\
        \mathcal{M}_{T_1^{A^1 \multimap B^1}\otimes T_2^{C^2 \multimap D^2}} &= \mathcal{M}_{T_1} +  \mathcal{M}_{T_2} + M_{A \otimes C \multimap B \otimes D }
        \\
         \sim_{T_1^{A^1 \multimap B^1}\otimes T_2^{C^2 \multimap D^2}} &= \left( \sim_{T_1}
        \union \sim_{T_2} \union (A^1 \leftrightarrow A)
        \union (B^1 \leftrightarrow B) \union (C^2 \leftrightarrow C)\union (D^2 \leftrightarrow D)
        \right)^\star
    \\
    \\
        \mathcal{M}_{T_1^{A \multimap B};T_2^{B \multimap C}} &=
            \mathcal{M}_{T_1} + \mathcal{M}_{T_2} + M_{A\multimap C}
        \\
         \sim_{T_1^{A^1 \multimap B^1};T_2^{B^2 \multimap C^2}} &= \left( \sim_{T_1}
        \union \sim_{T_2} \union (A^1 \leftrightarrow A)
        \union (B^1 \leftrightarrow B^2) \union (C \leftrightarrow C^2)
        \right)^\star
    \end{align*}
    where $A\leftrightarrow B$ denotes the implicit bijection between
    two isomorphic arenas $\sem{A}$ and $\sem{B}$; $R^\star$
    denotes the smallest superset of the relation $R$ complete
    by transitivity, reflexivity and symmetry.
\end{definition}

We call \defname{internal move} of the game $\intersem{T}$, any move
from $M_T$ which is not $\sim$-equivalent to any move in
$M_{type(T)}$.


A \defname{justified interaction sequence} of moves on the
interaction game $\intersem{T}$ is a sequence of moves from $M_T$
together with pointers. In contrast to the standard notion of
justified sequence, to each move in the sequence can be attached
several pointers. More precisely, if the equivalence class $m$ is
$\{m_1, \ldots, m_l \}$ then $m$ has one pointer for each
non-initial move $m_i$ in the equivalence class.

\begin{definition}[Filtering] We define several filtering operations
over justified interaction sequences. Let $s$ be a justified
sequence of moves on the interaction game $\intersem{T}$.
\begin{itemize}
\item  Let $T'$ be a subtree of $T$. We define the
filtering operator $s\upharpoonright T'$ to be the subsequence
of $s$ consisting of moves $\sim$-equivalent to some move in
$M_{T'}$. This operation causes some move to ``lose'' some of
their attached pointers: a given move $m$ with equivalence class
$\{m_1, \ldots, m_l \}$ may have up to $l$ pointers, but in
$s\upharpoonright T'$, only pointers associated to a $m_i$
belonging to $\mathcal{M}_{T'}$ are preserved.

Note that since $M_T$ is a set of equivalence classes with
respect to $\sim$, the filtering operator $\_ \filter T'$
implicitly performs the ``retagging'' of the moves to the
appropriate components of each game of the interaction game
$\intersem{T'}$.

\item  For any sub-game $A$ of the standard game $\sem{type(T')}$ we
define the filtering operator $s\upharpoonright A$ to be the
subsequence of $s$ consisting of moves from $A$ where at most
one pointer is kept for each move in the sequence: the one
corresponding to the class citizen from $A$.

\item For any initial move $m$ of the game $\sem{type(T)}$ occurring in $s$, $s
\hjfilter m$ is the subsequence of $s$ consisting of moves
that are \emph{hereditarily justified} by that particular occurrence of $m$ in $s \filter type(T)$.
% NOTE: it is important to precise ``in $s \filter type(T)$'' because $s$'justification
% pointers differs depending on the sub-interaction game considered.

%\item For any initial move $m$ of the game $\sem{type(T)}$, $s
%\hefilter m$ is the subsequence of $s$ consisting of moves
%that are \emph{hereditarily enabled} by $m$ in the game $\sem{type(T)}$.
\end{itemize}
By extension, we also define these operations on sets of justified
interaction sequences.
\end{definition}

Allowing moves to have multiple pointers complicates slightly the
presentation here, but this capability is necessary to model
strategy composition. Indeed, in game semantics after composing
strategies, the pointers from some moves may change! (See definition
of $\filter A,C$ in \cite{abramsky:game-semantics-tutorial}.)
However, for all the other operations on strategies that we will
used, the pointers will just be preserved. Formally we define this
property as follows: Let $s$ be an  interaction sequence on a game
$\intersem{T}$, $T'$ a direct subtree $T$ ({\it i.e.}~a subtree of
$T$ whose root is a child of $T$'s root), $A$ be a sub-game of
$\sem{type(T)}$ and $A'$ be a sub-game of $\sem{type(T')}$, then we
define the predicate $A'\stackrel{s}\hookrightarrow A$ as:
\begin{align*}
 A'\stackrel{s}\hookrightarrow A \mbox{ holds iff } &
 \Pstr{s_1\ (n){n'}\ s_2\ (m-n){m'}\ s_3 } = s\filter A'  \\
 & \implies \exists! m,n \in A | m \sim m' \zand n \sim n' \zand \Pstr{s_1\
(n){n}\  s_2\ (m-n){m}\ s_3} = s\filter A
\end{align*}

and we say that $s$'s justification is preserved from $A'$ to $A$
with respect to $\sim$.



\begin{definition}[Legal uncovered positions] We recall
that in the standard game semantics, the set of legal positions
$L_A$ of a game $A$ is the set of justified sequences of moves from
$M_A$ respecting visibility and alternation. We define the set of
\defname{legal uncovered position} $L_T$ of an interaction game $\intersem{T}$ as
follows:
    \begin{itemize}
    \item If $T$ is a leaf annotated by a type $A$ then $L_T =
    L_A$;
    \item If $T$ is a unary node with child node $T'$ then:
    $$L_T = \{ s \in JustSeq(T) \ | \ s \filter type(T) \in L_{type(T)} \zand  s \filter T' \in L_{T'} \} \ ;$$
    \item If $T$ is a binary node with children nodes $T_1$ and $T_2$ then:
    $$L_T = \{ s \in JustSeq(T) \ | \ s \filter type(T) \in L_{type(T)} \zand  s \filter T_1 \in L_{T_1}
    \zand  s \filter T_2 \in L_{T_2} \} \ .$$
    \end{itemize}
    where $JustSeq(T)$ denotes the set of justified interaction sequences on
    $\intersem{T}$.
\end{definition}

Revealed strategies can alternatively be represented as by means
of sets of uncovered positions:
\begin{definition}[Revealed strategies as set of uncovered positions]
\label{dfn:revealedstrat}
The set of uncovered positions of a revealed strategy is defined inductively on the
structure of the annotated interaction type tree underlying the
interaction strategy:
\begin{itemize}[-]
\item Leaf labelled with type $A$ and annotated by the strategy $\sigma$: The set of positions of the revealed strategy is precisely the set of positions of the standard strategy $\sigma$.

\item Tensor product, pairing, promotion, currying:
\begin{eqnarray*}
(\Sigma_1 : \intersem{T_1}) \otimes (\Sigma_2 : \intersem{T_2}) : \intersem{T} &=\{ s \in L_T \ | \  &s \filter T_1 \in \Sigma_1 \zand\ s \filter T_2 \in \Sigma_2 \\
&& \zand\ type(T_1)\stackrel{s}\hookrightarrow type(T) \\
&& \zand\ type(T_2)\stackrel{s}\hookrightarrow type(T) \}
\\ \\
\langle \Sigma_1 : \intersem{T_1}, \Sigma_2 : \intersem{T_2} \rangle : \intersem{T} &= \{ s \in L_T \ | &
   ( (s \filter T_1 \in \Sigma_1 \zand\ s \filter T_2 = \epsilon) \\
&&  \   \zor ( s \filter T_1 = \epsilon \zand s \filter T_2 \in \Sigma_2)) \\
&& \zand\ type(T_1)\stackrel{s}\hookrightarrow type(T) \\
&& \zand\ type(T_2)\stackrel{s}\hookrightarrow type(T) \}
\\ \\
(\Sigma' : \intersem{T'})^\dagger : \intersem{T} &= \{ s \in L_T \ | \ &
\mbox {for all occurrence $m$ in $s$ of an initial  }\\
&& \mbox{ $\sem{type(T)}$-move, $(s \filter m) \filter T' \in \Sigma'$} \\
&& \zand\ type(T')\stackrel{s}\hookrightarrow type(T) \}
\\ \\
\Lambda(\Sigma' : \intersem{T'}) : \intersem{T} &= \{ s \in L_T \ | & s \filter T' \in \Sigma' \ \zand\ type(T')\stackrel{s}\hookrightarrow type(T) \}
\end{eqnarray*}

\item Uncovered composition $(\Sigma_1 : \intersem{T_1})\ ;^U\ (\Sigma_2
:\intersem{T_2})$ defined on the game $\intersem{T}$ where
$type(T) = A \multimap C$, $type(T_1) = A^1 \multimap B_0 \times
\ldots \times B_l$ and $type(T_2) = B_0 \times \ldots \times B_l
\multimap C^2$. We first define
\begin{eqnarray*}
\Sigma_1 \| \Sigma_2 &= \{ u \in L_T  \ | \ & u \upharpoonright T_1 \in \Sigma_1 \mbox{ and } u \upharpoonright T_2 \in \Sigma_2 \\
&& \zand\ C^2\stackrel{u}\hookrightarrow C\ \zand\ (A^1)^-\stackrel{u}\hookrightarrow A^-  \\
&& \zand\ \parbox[t]{8cm}{for any initial $m$ in $A^1$, if $m$ is justified in $u \filter type(T_1)$ by $b\in B_j$,
itself justified by $c \in C^2$ in $u \filter type(T_2)$ then $m$ justified by $c$ in $u \filter type(T)$ \} }
\end{eqnarray*}
where $A^-$ denotes the set of non-initial moves of the game $A$. We can now define composition as:
$$ \Sigma_1 ;^U \Sigma_2 = \{ cover(u,(0..l)\setminus U) \ | \ u \in \Sigma_1 \| \Sigma_2 \}$$
where $cover(u,C) = u \filter \left( M_T \setminus \Union_{j\in
C} B_j \right)$ {\it i.e.}~the subsequence of $u$ obtained by
removing moves in $\Union_{j\in C} B_j$. Hence
$\Sigma_1;^{\{0..l\}} \Sigma_2 = \Sigma_1 \| \Sigma_2$.

In other words $\Sigma_1 ;^U \Sigma_1$ is the set of uncovered
plays obtained by performing the usual composition while
ignoring and copying the internal moves from arenas in
$\intersem{T_1}$ or $\intersem{T_2}$ and preserving any internal
move produced by the composition in some component $B_k$ for $k
\in U$.
\end{itemize}
\end{definition}

\begin{remark} \hfill
\label{rem:interstrat}
\begin{enumerate}[i.]
\item We observe that for all strategy operator
except composition, pointers associated to moves are preserved.
For strategy composition, additional pointers are
``created'' only for initial $A$-moves.
\item It is straightforward to generalize the pairing operator $\langle \Sigma_1, \Sigma_2 \rangle$ to more than two parameters: an interaction strategy $\langle \Sigma_1, \ldots, \Sigma_p \rangle$ for $p\geq2$
is defined on an interaction game whose root node has $p$ children.
\end{enumerate}
\end{remark}

We write $\mathcal{I}$ for the set of all revealed strategies. Note
that $\mathcal{I}$ is not a category since composition is not
associative and there is no identity interaction strategy.


\begin{lemma}[Complete interaction sequence]
\label{lem:inter_complete}
Let $u$ be an interaction sequence of some interaction strategy $\Sigma : \intersem{T}$
and suppose that the standard strategy denoting the leaves of $\Sigma$ are all well-bracketed.

Then for any node/leaf game $A$ of $T$ and interaction sequence $u\in \Sigma$ we have:
\begin{itemize}[i.]
\item $u \filter A$ is well-bracketed;

\item If $u \filter type(T)$ is complete (all question moves answered) then
    $u \filter A$ is complete.
\end{itemize}
\end{lemma}
\begin{proof}
By induction on the structure of the interaction game $\intersem{T}$. The base case is
trivial. We only treat composition, the other cases being trivial: Let $ u \in \Sigma_1 ; ^U \Sigma_2$ for some $U \subseteq \nat$ with
$\Sigma_1 : \intersem{T_1^{A\multimap B}}$ and $\Sigma_2 : \intersem{T_2^{B\multimap C}}$.

i. During composition, pointers attached to answer moves are preserved with respect to $\sim$
thus non-well-bracketing of $u\filter A\multimap C$ implies
either non-well-bracketing of $u\filter A\multimap B$ or $u\filter B\multimap C$.

For ii., suppose $u \filter type(T) = \Pstr{(q)q\ u'\ (a-q)a }$.
By well-bracketing (i.) and since $q$ and $a$ belong to $C$ we must have
$u \filter B\multimap C = \Pstr{(q)q \ldots (a-q)a}$ thus $u \filter B\multimap C$ is complete.
Suppose that $u \filter A\multimap B$ is not complete, then its first move is unanswered,
but since this is a $B$-move, it must also occur unanswered in $u \filter B\multimap C$ which is a contradiction
since we have just prove that $u \filter B\multimap C$ is complete. Thus $u \filter A\multimap B$  is also complete.

The induction hypothesis permits to conclude.
\end{proof}
Consequently if $u\filter type(T)$ is complete then $u$ is maximal {\em i.e.~no move (and in particular no internal move) can be played after $u$}.

\subsubsection{Modeling the $\lambda$-calculus in $\mathcal{I}$}

We would like to use revealed strategies from $\mathcal{I}$ to model terms of
the simply-typed lambda calculus.
Depending on the internal moves that we wish to hide, we obtain different possible interaction strategies for a given term.
The following definition fixes a unique strategy denotation which is computed from the $\eta$-normal form of the term.

\begin{definition}[Revealed denotation of a term]
\label{dfn:interactionstrategy_ofterms}
Let $\pi_i$ denote the $i^{th}$ projection copycat strategy $\pi_i : \sem{X_1 \times \ldots \times X_l} \rightarrow \sem{X_i}$.

The \defname{revealed game denotation} or \emph{revealed strategy} of
$M$ written $\intersem{\Gamma \vdash M : A}$ is defined as
$\sem{\Gamma \vdash M : A}$ if $M$ is in $\beta$-normal form, otherwise
it is defined by structural induction on the \emph{$\eta$-long normal form of $M$}:
\begin{eqnarray*}
\intersem{\Gamma \vdash \lambda \overline{\xi} . M  : A} &=& \Lambda^{|\overline{\xi}|}(\intersem{\Gamma, \overline{\xi} \vdash M : o })  \\
\intersem{\Gamma  \vdash x_i N_1 \ldots N_p :o} &=& \langle \pi_i, \intersem{\Gamma \vdash N_1 : A_1}, \ldots, \intersem{\Gamma \vdash N_p : A_p}  \rangle \fatsemi ^{\{1..p\}} ev^p \\
\intersem{\Gamma \vdash f N_1 \ldots N_p : o} &=& \langle \intersem{\Gamma \vdash N_1 : A_1}, \ldots, \intersem{\Gamma \vdash N_p : A_p} \rangle^\dagger\  \|\ \sem{f} \\
\intersem{\Gamma \vdash N_0 \ldots N_p : o} &=& \langle \intersem{\Gamma \vdash N_0 : A_0}, \ldots, \intersem{\Gamma \vdash N_p : A_p}  \rangle^\dagger\ \|\ ev^p
\end{eqnarray*}
where $\Gamma = x_1 : X_1 \ldots x_l : X_l$, $f : A_0$ is a $\Sigma$-constants, $p\geq 1$, $A_0 =
(A_1,\ldots,A_p,o)$, $ev^p$ denotes the evaluation strategy with
$p$ parameters and $X_i = A_0$ in the second equation.
\end{definition}

Figure \ref{fig:interaction_strategy_denotations} contains tree representations of the interaction games of the revealed strategy $\intersem{\Gamma \vdash M : A}$ for the application cases. These tree tell us all the information that we need about the strategy involved in $\intersem{M}$. For instance the revealed strategy $\Sigma$ is defined on the interaction arena $\intersem{T^{00}}$ whose root is $!A^0 \multimap B^0$; the strategy $ev$ is defined on the interaction arena $\intersem{T^1}$ with a single arena-node $!B^1 \multimap C^1$; thus plays of $ev$ do not contain uncovered moves.


    \begin{figure}[htbp]
        $$
        \tree[levelsep=6ex,thistreesep=3cm]{\TR{\intersem{N_0 N_1 \ldots N_p :o}:T [!A\multimap C]}}
                {   \tree[levelsep=6ex]{\TR{\Sigma^\dagger:T^0[!A^0\multimap !B_0^0\otimes \ldots \otimes !B_p^0 ]}}
                        {
                            \tree[levelsep=6ex,thistreesep=3cm]{\TR{\Sigma:T^{00}[!A^{00}\multimap B_0^{00}\times \ldots \times B_p^{00}]}}
                            {
                                \tree[levelsep=6ex]{\TR{\intersem{N_0}:T^{000}[!A^{000}\multimap B_0]}}{\Tfan[fansize=10ex]}
                                \TR{\ldots}
                                \tree[levelsep=6ex]{\TR{\intersem{N_p}:T^{00p}[!A^{00p}\multimap B_p]}}{\Tfan[fansize=10ex]}
                            }
                        }
                    \TR{ ev:T^1[!B_0^1 \otimes \ldots \otimes !B_p^1 \multimap C] }
                }
       $$
       \begin{center}
       \emph{Tree-representation of the revealed strategy $\intersem{\Gamma \vdash N_0 N_1 \ldots N_p :o}$.}
       \end{center}

        $$
        \tree[levelsep=6ex,thistreesep=3cm]{\TR{\intersem{x_i N_1 \ldots N_p :o}:T [!A\multimap C]}}
                {   \tree[levelsep=6ex]{\TR{\Sigma^\dagger:T^0[!A^0\multimap !B_0^0\otimes \ldots \otimes !B_p^0 ]}}
                        {
                            \tree[levelsep=6ex,thistreesep=3cm]{\TR{\Sigma:T^{00}[!A^{00}\multimap B_0^{00}\times \ldots \times B_p^{00}]}}
                            {
                                \TR{\pi_i:T^{000}[!A^{000}\multimap B_0]}
                                \tree[levelsep=6ex]{\TR{\intersem{N_1}:T^{001}[!A^{001}\multimap B_1]}}{\Tfan[fansize=10ex]}
                                \TR{\ldots}
                                \tree[levelsep=6ex]{\TR{\intersem{N_p}:T^{00p}[!A^{00p}\multimap B_p]}}{\Tfan[fansize=10ex]}
                            }
                        }
                    \TR{ ev:T^1[!B_0^1 \otimes \ldots \otimes !B_p^1 \multimap C] }
                }
        $$
       \begin{center}\emph{Tree-representation of the revealed strategy $\intersem{\overline{x}:\overline{X}\vdash x_i N_1 \ldots N_p :o}$}
       \end{center}
    \bigskip
    {\small
     Node labels are of the form $\Pi : T' [A]$ where $\Pi$ is a strategy, $T'$ is the corresponding interaction game and $A$ is the standard game lying at the root of the interaction game $T$. The games $A$, $B$ and $C$ are defined as follows:
    \begin{eqnarray*}
        A &=& \Gamma = X_1 \times \ldots \times X_n\\
        B &=& \underbrace{((B_1' \times \ldots \times B_p') \rightarrow o')}_{B_0} \times B_1 \times \ldots \times B_p\\
        C &=& o \ .
    \end{eqnarray*}
    Games are annotated with string  $s \in \{ 0..p \}^*$ in the exponent to indicate the path from the root to the corresponding node in the tree (each number in $s$ indicates which direction to take at the corresponding branch point).
   }
        \smallskip
       \caption{Tree-representation of the revealed strategy in the application case.}
      \label{fig:interaction_strategy_denotations}
    \end{figure}


\begin{remark}
When computing an interaction strategy of the form
$\intersem{y_i N_1 \ldots N_p}$ for some variable $y_i$, the
internal moves of $N_1$, \ldots, $N_p$ are preserved however the
internal moves produced by the copy-cat projection strategy denoting
$y_i$ are omitted.
\end{remark}

\begin{example}
Take the term $\lambda x . (\lambda f . f x) (\lambda y . y)$.
%Its computation tree is:
%$$
%\tree{\lambda x} {
%    \pstree[levelsep=4ex]{\TR{@}}
%    {       \pstree[levelsep=4ex]{\TR{\lambda f}}
%                { \tree{f}{  \tree{\lambda}{ \TR{x}  } } }
%            \pstree[levelsep=4ex]{\TR{\lambda y}}
%                    {\TR{y}}
%    } }
%$$
Its revealed strategy is $$\Lambda ( \langle \sem{ x:X \vdash \lambda f . f
x : (o\rightarrow o) \rightarrow o} , \sem{ x:X \vdash \lambda y . y
: o \rightarrow o} \rangle \| ev_2 ) \ .$$
\end{example}


\subsubsection{From interaction semantics to standard semantics and vice-versa}

In the standard semantics, given two strategies $\sigma : A
\rightarrow B$, $\tau : B \rightarrow C$ and a sequence $s \in
\sigma \fatsemi \tau$, it is possible to (uniquely) recover the
internal moves. The uncovered sequence is written ${\bf u}(s,
\sigma, \tau)$. The algorithm to obtain this unique uncovering is
given in part II of \cite{hylandong_pcf}. Therefore given a term
$M$, we can completely uncover the internal moves of a sequence
$s\in\sem{M}$ by performing the uncovering operation recursively at
every @-node of the computation tree.

Conversely, the standard semantics can be recovered from the
interaction semantics by filtering the moves, keeping only those
played in the root arena:
\begin{eqnarray}
 \sem{\Gamma \vdash M : T} = \intersem{\Gamma \vdash M : T} \upharpoonright \sem{\Gamma \rightarrow T} \label{eqn:int_std_gamsem}
\end{eqnarray}

\subsection{The correspondence theorem for the simply-typed $\lambda$-calculus without interpreted constants}
In this section, we establish a connection between the interaction
semantics of a simply-typed term without constants ($\Sigma =
\emptyset$) and the traversals of its computation tree: we show that
the set $\travset(M)$ of traversals of the computation tree is
isomorphic to the set of uncovered plays of the strategy denotation
(this is the counterpart of the ``Path-Traversal Correspondence'' of
\cite{OngLics2006}), and that the set of traversal reductions is
isomorphic to the strategy denotation.

\subsubsection{@-free traversals}

When defining computation trees, it was necessary to introduce
application nodes (labelled @) in order to connect the operator and
the operand of an application. The presence of @-nodes has also
another advantage: it ensures that the lambda-nodes are all at even
level in the computation tree, and thus a traversal respects a certain form of
alternation.

Application nodes are however redundant in the sense that they do
not play any role in the computation of the term. In fact it is
necessary to filter them out if we want to establish the
correspondence with the interaction game semantics.

\begin{definition}[@-free traversal]
\label{dfn:appnode_filter}
Let $t$ be a traversal of $\tau(M)$.
We write $t-@$ for the sequence of nodes-with-pointers obtained by
\begin{itemize}
\item removing from $t$ all @-nodes and value-leaves of some @-node;
\item replacing any link pointing to an @-node by a link pointing to the immediate predecessor of @ in $t$.
\end{itemize}

Suppose $u = t-@$ is a sequence of nodes obtained by applying the
previously defined transformation on the traversal $t$, then $t$ can
be partially recovered from $u$ by reinserting the @-nodes as
follows. For each @-node @ in the computation tree with parent node
denoted by $p$, we perform the following operations:
\begin{enumerate}
\item replace every occurrence of the pattern $p \cdot n$, where $n$ is a $\lambda$-nodes,
by $p \cdot @ \cdot n$;
\item replace any link in $u$ starting from a $\lambda$-node and pointing to $p$ by a link pointing to the inserted @-node;
\item if there is an occurrence in $u$ of a value-leaf $v_p$ pointing to $p$ then insert a value-leaf $v_@$
immediately before $v_p$ and make it point to the node immediately
following $p$ (which is also the $@$-node that we inserted in 1).
\end{enumerate}
We write $u+@$ for this second transformation.
\end{definition}
These transformations are well-defined because in a traversal, an @-node
always occurs in-between two nodes $n_1$ and $n_2$ such that  $n_1$ is the parent node of @
and $n_2$ is the first child node of @ in the computation tree:
$$      \pstree[levelsep=4ex]{\TR{n_1}\treelabel{0} }
        {
            \pstree[levelsep=3ex]{\TR{@}}
            {
                \tree{n_2}{\vdots}
                \TR[edge=\dedge]{}
                \TR[edge=\dedge]{}
            }
        }
$$
\begin{remark}
Justified sequences of nodes of the form $t-@$ for some traversal $t$ are not, strictly speaking, proper justified sequences of nodes since they do not respect alternation (two $\lambda$-nodes may become adjacent after removing a @-node)
and since any $\lambda$-node justified by @ becomes justified by @'s parent which is also a $\lambda$-node. However we will treat them just as justified sequence.
\end{remark}

\begin{lemma} \label{lem:minus_at_plus_at}
$$\forall t \in \travset(M), \quad (t-@)+@ = \left\{
            \begin{array}{ll}
              t, & \hbox{if $t^\omega \neq @$\ ;} \\
              \ip\ t, & \hbox{if $t^\omega = @$\ .}
            \end{array}
          \right.
$$
\end{lemma}
\proof
The result follows immediately from the definition of the operation -@ and +@.
\qed
\smallskip

We introduce the following notation:
$$
\travset(M)^{-@} = \{ t - @ \ | \  t \in \travset(M) \}
$$

\begin{remark}
If $M$ is $\beta$-normal then $\tau(M)$ does not contain any
@-node therefore all nodes are hereditarily justified by $r$ and we
have $\travset(M)^{-@} = \travset(M) = \travset(M)^{\upharpoonright
r }$.
\end{remark}

\paragraph{Mapping @-free traversals to interaction plays}
\hfill

\notetoself{
\begin{definition}[Mapping from nodes to moves]\hfill
    \label{def:theta mapping}
    Let $T$ be the interaction game of the interaction strategy $\intersem{M}$ and
    $M_T$ be the set of equivalence class of moves from $\mathcal{M}$.


    For $n \in N_{\sf prime}$, let $\Gamma(n) \vdash \kappa(n) : T(n)$ denote the subterm of $\elnf{M}$ rooted at $n$.
    We define the disjoint union of games:
    $$\mathcal{G}_M = \sem{\Gamma\rightarrow T} \quad \uplus \quad  \biguplus_{n \in N_{\sf prime} } \sem{T(n)}.$$
    $$\mathcal{G}_M = \sem{\Gamma\rightarrow T} \quad \uplus \quad  \biguplus_{n \in N_{\sf spawn} } \sem{T(n)}.$$

    We define the function $\varphi_M: V_\lambda \union V_{\sf var} \rightarrow M_T$
    as:
    \begin{equation*}
        \varphi_M = \psi_{M} \quad \union \Union_{n \in N_{\sf prime}} \psi_{n}
    \end{equation*}
    where $q_0$ denotes $\sem{\Gamma\rightarrow T}$'s initial move.

    We omit the subscript in $\varphi_M$ if it does not cause any ambiguity.
\end{definition}


$\varphi_M$ is indeed totally defined on $V_\lambda \union V_{\sf var} = V\setminus (V_@ \union V_\Sigma)$ (since a node is either hereditarily justified by the root, by a @-node or by a $\Sigma$-node).

\begin{remark}
\label{rem:phi_preserves_her_enabling}
$\varphi_M$ \defname{preserves hereditary enabling}: a node $n$ is hereditarily
 enabled by some node $n' \in N \inter E \relimg{N_@ \union N_\Sigma}$ in $\tau(M)$ if and only if
 $\varphi(n)$ and $\varphi(n')$  are both played in the same game $A \in \mathcal{G}$ and
the move $\varphi_M(n)$ is hereditarily enabled by $\varphi_M(n')$ in $A$.
\end{remark}

%If $t$ is a justified sequence of nodes in $V_\lambda \union V_{\sf var}$ then $?(\varphi(t)) =
%\varphi(?(t))$.
%where $?(\varphi(t))$ denotes the subsequence of $\varphi(t)$ consisting of the unanswered questions
%and $?(t)$ denotes the subsequence of $t$ consisting of the unmatched nodes (see the
%definition in section \ref{sec:adding_value_leaves}).

}

As we observed in a previous remark, sequences from $\travset(M)^{-@}$ are not, strictly speaking, proper justified sequences. Consequently the filtering operators introduced up to now are undefined on $\travset(M)^{-@}$. We now introduce a new filtering operation on $\travset(M)^{-@}$:
\begin{definition}
Let $\Delta \vdash \kappa(n) : A$ be some subterm of $\elnf{M}$ for some $n\in N_\lambda$.
We define the \defname{subterm filtering} operator on sequences of the form $t-@$ for some traversal $t$ of $M$ as follows:
$$ (t - @) \subtermfilter \kappa(n) = (t-@)\hefilter n = t\hefilter n \ . $$
\end{definition}
Note that this is well-defined because $t-@ = t'-@$ implies $t\hefilter n = t'\hefilter n$ (since @-nodes have no justifier).
In particular we have:
$$ (t - @) \subtermfilter M = t \hefilter r  = t \hjfilter r \ .$$
(Here hereditary justification and hereditarily enabling coincide because the root node can appear at most once in a traversal.)

\begin{lemma}[Filtering lemma]
\label{lem:varphi_filter}
Let $t$ be a traversal of $M$, $\Delta \vdash N : A$ be some subterm of $\elnf{M}$ and $m$ be an occurrence of an initial $A$-move in $\varphi(t-@)$ then:
$$(i) \quad \varphi_M((t-@)\subtermfilter N) = \varphi_M(t-@) \filter \sem{\Delta\rightarrow A} \ .$$
$$(ii) \quad \varphi_M((t-@)\subtermfilter N) \hjfilter m = \varphi_M(t\hjfilter n) \ .$$
where $n$ denotes the occurrence of $\tau(N)$'s root in $t$ whose image
by $\varphi_M$ is the occurrence $m$.

Consequently:
$$(iii) \quad  \varphi_M(\travset^{-@}(M)) \filter \sem{\Gamma \rightarrow T} = \psi_M(\travset^{\filter r}(M))\ .$$
\end{lemma}
\proof Let $t$ be a traversal of $M$:
$$\begin{array}{lrclr}
\mbox{i.}& \varphi( (t-@) \subtermfilter N ) &=& \varphi_M((t-@) \hefilter n ) & \mbox{(Def. subterm filtering)}\\
          &&=& \varphi_M(t-@) \filter \sem{\Delta \rightarrow A}  & \parbox[t]{5.5cm}{(By remark \ref{rem:phi_preserves_her_enabling}, $\varphi_M$ preserves hereditary enabling,  and  moves in $\sem{\Delta \rightarrow T}$ are all hereditarily enabled by the initial move $m = \varphi_M(n)$).} \\
\\
\mbox{ii.}& \varphi_M((t-@)\subtermfilter N) \hjfilter m
  &=& \varphi_M(t\hefilter n) \hjfilter m & \mbox{(Def. subterm filtering)}\\
  &&=& ( \varphi_M(t) \hefilter \varphi_M(n) ) \hjfilter m & \parbox[t]{5.5cm}{($\varphi_M$ maps the set of nodes hered. \emph{enabled} by $n$ to the set of moves hered. \emph{enabled} by $\varphi_M(n)$)} \\
  &&=& \varphi_M(t) \hefilter  m \hjfilter m & \mbox{($m = \varphi_M(n)$)} \\
  &&=& \varphi_M(t) \hjfilter m \ . \\
  \\
\mbox{iii.} & \varphi(t-@) \filter \sem{\Gamma \rightarrow T}
             &=& \varphi( (t-@) \subtermfilter M ) & \mbox{(by i.)} \\
           &&=& \varphi( (t-@) \hefilter r ) & \mbox{(Def. subterm filtering)} \\
           &&=& \varphi( t \hefilter r ) & \mbox{(@-node are not justified).} \qed
\end{array}$$

The function $\varphi$ regarded as a function from the set of vertices $V_\lambda \union V_{\sf var}$ of the computation tree to moves in arenas is not injective.
For instance the two occurrences of $x$ in the computation tree of the term $\lambda f x. f x x$ are mapped to the same question. However
the function $\varphi$ defined on the set of traversals to interaction plays of game semantics is injective:
\begin{lemma}[$\psi$ and $\varphi$ are injective]
\label{lem:varphiinjective}
For any two traversals $t_1$ and $t_2$:
\begin{itemize}
\item[(i)] If $\varphi (t_1 - @ ) = \varphi (t_2 - @ )$ then $t_1-@ =t_2 -@$\ ;
\item[(ii)] if $\psi (t_1 \upharpoonright r ) = \psi (t_2 \upharpoonright r )$ then $t_1\upharpoonright r = t_2\upharpoonright r$\ .
\end{itemize}
\end{lemma}
\begin{proof}
For any node $n$ of a traversal $t$ let us write $ptr(n)$ to denote the distance between $n$ and its justifier node in $t$. If $n$ has not link then we set $ptr(n)=0$. We also use the same notation for sequences of moves.

\begin{lemma}[Preleminary lemma]
\label{lem:varphiinjective:prelem}
\begin{equation}
\left(
  \begin{array}{ll}
    t \cdot n_1, t \cdot n_2 \in \travset \\
    \zand\ n_1 \neq n_2
  \end{array}
\right)
 \mbox{ implies } n_1,n_2 \in N^{\upharpoonright r}_{\lambda} \zand ( \varphi(n_1) \neq \varphi(n_2) \zor ptr(n_1) \neq ptr(n_2) ) \ . \end{equation}
\end{lemma}
\begin{proof}
Let $t \cdot n_1, t \cdot n_2 \in \travset$.
First we remark that the traversal rules have a weak form of determinism which ensures that $n_1$ and $n_2$ belong to the same category of node i.e.\ they must be both in $N_{\sf var}$, $N_@$ or $N_\lambda$.

Suppose that $n_1, n_2 \in N_@$ then $t \cdot n_1$ and $t \cdot n_2$ were formed using the (App) rule. Since this rule is deterministic we must have $n_1=n_2$ which violates the second hypothesis.


Suppose that $n_1,n_2\in N_{\sf var}$. The traversals $t \cdot n_1$ and $t \cdot n_2$ must have been formed using either rule (Lam) or (App). But these two rules are deterministic and their domains of definition are disjoint. Hence again the second hypothesis is violated.

Suppose that $n_1,n_2\in N_\lambda$ then
the traversals $t \cdot n_1$ and $t \cdot n_2$ must have been formed using either rule (Root), (App), (Var) or (InputVar). Since all these rules have disjoint domains of definition, the same rule must have been use to form $t \cdot n_1$ and $t \cdot n_2$. Supposed that one of the rules (Root), (App) and (Var) has been used then since they are all deterministic we have $n_1=n_2$ which violates the second hypothesis. Consequently, the rule (InputVar) must have been used and therefore $n_1,n_2 \in N_\lambda^{\upharpoonright r}$. By definition of (InputVar), in order to have $n_1\neq n_2$ and $\varphi(n_1) = \varphi(n_2)$, the parent node of the last node in $t$ must occurs at more than one position in $\oview{t}$ and $n_1,n_2$ correspond to the child node of two different occurrences of that parent node in $\oview{t}$. But then the links associated to $n_1$ and $n_2$ will point to their respective occurrence of that parent node in $\oview{t}$ hence $ptr(n_1) \neq ptr(n_2)$.
\end{proof}

\noindent {\it (continuation of the proof of Lemma \ref{lem:varphiinjective})}

(i) The result is trivial is either $t_1$ or $t_2$ is empty.
Suppose that $t_1-@\neq t_2-@$ then necessarily $t_1 \neq t_2$, thus there are some sequences $t'$, $u_1$, $u_2$ and some nodes $n_1,n_2$ such that
 $t_1 = t' \cdot n_1 \cdot u_1$, $t_2 = t' \cdot n_2 \cdot u_2$ with either $n_1\neq n_2$ or $ptr(n_1) \neq ptr(n_2)$.

If $n_1 = n_2$ then $ptr(n_1) \neq ptr(n_2)$ therefore $n_1,n_2 \not\in N_@$ (otherwise $ptr(n_1) = 0 = ptr(n_2)$). Since $ptr(\varphi(n_1)) = ptr(n_1)$ and  $ptr(\varphi(n_2)) = ptr(n_2)$ we must have $\varphi(t' \cdot n_1) \neq \varphi(t' \cdot n_2)$. Since $n_1,n_2 \not\in N_@$ we also have $\varphi((t' \cdot n_1)-@) \neq \varphi((t' \cdot n_2)-@)$. Hence $\varphi(t_1-@) \neq \varphi(t_2-@)$.

If $n_1 \neq n_2$ then by Lemma \ref{lem:varphiinjective:prelem} we have $n_1,n_2 \not\in N_@$ and $\varphi(n_1) \neq \varphi(n_2)$ or $ptr(n_1) \neq ptr(n_2)$ which again implies $\varphi(t_1-@) \neq \varphi(t_2-@)$.


(ii) Suppose that $t \upharpoonright r \neq t' \upharpoonright r$ then necessarily $t \neq t'$ which in turn implies that for some sequences $t_1'$, $t_2'$, $u_1$, $u_2$ and some nodes $n_1 \neq n_2$
we have $t_1 = t' \cdot n_1 \cdot u_1$, $t_2 = t' \cdot n_2 \cdot u_2$ and either $n_1\neq n_2$ or $ptr(n_1) \neq ptr(n_2)$.

If $n_1 = n_2$ then $ptr(n_1) \neq ptr(n_2)$. An   analysis of the traversal rules shows that the rule (InputVar) is the only rule which can visit the same node with two different pointers. Hence $n_1,n_2 \in N_\lambda^{\upharpoonright r}$.
Therefore $\psi( (t'\cdot n_1) \upharpoonright r ) = \psi( (t'\upharpoonright r) \cdot n_1 )  \neq \psi( (t'\upharpoonright r) \cdot n_2 )$. Hence    $\psi( t_1\upharpoonright r ) \neq \psi( t_2\upharpoonright r )$.

If $n_1 \neq n_2$ then we can use Lemma \ref{lem:varphiinjective:prelem}
to obtain $\psi( t_1\upharpoonright r ) \neq \psi( t_2\upharpoonright r )$.
\end{proof}

\begin{corollary} \
\label{cor:varphi_bij}
\begin{itemize}
\item[(i)] $\varphi$ defines a bijection from $\travset(M)^{-@}$
to $\varphi(\travset(M)^{-@})$\ ;
\item[(ii)] $\psi$ defines a bijection from $\travset(M)^{\upharpoonright r}$ to
$\psi(\travset(M)^{\upharpoonright r})$\ .
\end{itemize}
\end{corollary}

\subsubsection{The correspondence theorem}
We now state and prove the correspondence theorem for the
simply-typed $\lambda$-calculus without interpreted constants
($\Sigma = \emptyset$). The result extends immediately to the
simply-typed $\lambda$-calculus with \emph{uninterpreted} constants
since we can regard constants as being free variables.

\begin{lemma}[Local Traversal Extension]
\label{lem:local_traversal_progression}
Let $M'$ be a subterm of $M$, $t \in \travset(M)$,
$t' \in \travset(M')$ such that $t' \neq \epsilon$ and $t\subseqof t'$. If the
traversal $t' \cdot n$ of $\tau(M')$ can be formed using a rule different from \rulenamet{InputVar}
and $\rulename{InputVar^{val}}$ then either $t' \cdot n \subseqof t $ or
$ t' \cdot n \in \travset(M)$.
where $n$'s link in $t \cdot n$ points to the same node occurrence as in $t' \cdot n$.
\end{lemma}
\proof
By Case analysis on the traversal rule used to form $t'\cdot n$.
\qed

This lemma says that extending a traversal locally also extends the traversal globally: the traversal $t$ of $M$ can be extended by extending a ``sub-traversal'' $t'$ of some sub-term $M'$.
This is not obvious since $t'$ is a subsequence of $t$ which means that
the nodes in $t'$ are also present in $t$ with the same pointers but with some other nodes interleaved in between. However these interleaved nodes are inserted in a preservative way which allows us to apply the rule used to extend $t'$ on $t$.

The following theorem establishes a correspondence between the
game-denotation of a term and the set of traversals of its
computation tree:
\begin{theorem}[The Correspondence Theorem]
\label{thm:correspondence}
 For any simply-typed term $\Gamma \vdash M :T$,
the function $\varphi_M$ defines a bijection from $\travset(M)^{\upharpoonright
r}$ to $\sem{\Gamma \vdash M : T}$ and a bijection from
$\travset(M)^{-@}$ to $\intersem{\Gamma \vdash M : T}$:
\begin{eqnarray*}
 \varphi_M  &:& \travset(\Gamma \vdash M : T)^{-@} \stackrel{\cong}{\longrightarrow} \intersem{\Gamma \vdash M :T} \\
 \psi_M  &:& \travset(\Gamma \vdash M : T)^{\upharpoonright r} \stackrel{\cong}{\longrightarrow} \sem{\Gamma \vdash M :T} \ .
\end{eqnarray*}

\end{theorem}

%\begin{proposition}
%\label{prop:rel_gamesem_trav} Let $\Gamma \vdash M : T$ be a
%simply-typed $\lambda$-term and $r$ be the root of $\tau(M)$. Then:
%\begin{itemize}
%\item[(i)]  $\varphi_M(\travset(M)^{-@}) = \intersem{\Gamma \vdash M : T}$ \ ;
%\item[(ii)] $\varphi_M(\travset(M)^{\upharpoonright r}) = \sem{\Gamma \vdash M : T}$ \ .
%\end{itemize}
%\end{proposition}

\begin{remark}
\label{rem:corresp_proofreduction}
    By corollary \ref{cor:varphi_bij}, we just need to show that
    $\varphi_M$ defines \emph{surjections}, that is to
    say:
    \begin{eqnarray*}
    \varphi_M(\travset(M)^{-@}) &=& \intersem{\Gamma \vdash M : T} \\
    \psi_M(\travset(M)^{\upharpoonright r}) &=& \sem{\Gamma \vdash M :
    T}
    \end{eqnarray*}
    The first equation implies the second one, indeed:
    \begin{align*}
    \sem{\Gamma \vdash M : T} &= \intersem{\Gamma \vdash M : T} \upharpoonright \sem{\Gamma \rightarrow T} & \mbox{(eq. \ref{eqn:int_std_gamsem})} \\
            &= \varphi_M(\travset^{-@}(M)) \upharpoonright \sem{\Gamma \rightarrow T} & \mbox{(by (i))}\\
            &= \psi_M(\travset^{\upharpoonright r}(M)) & \mbox{(lemma \ref{lem:varphi_filter})}
    \end{align*}
    therefore we just need to prove the first equation.
\end{remark}

    Let us give a brief overview of the proof before giving it in full details.
    It proceeds by induction on the structure of the computation tree.
    The only non-trivial case is the application: the computation tree
    $\tau(M)$ has the following form:
        $$ \tree[levelsep=4ex]{\lambda \overline{\xi}}
            { \tree[levelsep=4ex]{@}
                {   \TR{\tau(N_0)} \TR{\ldots} \TR{\tau(N_p)}}}
        $$

    A traversal of $\tau(M)$ proceeds as follows: it starts at the root $\lambda \overline{\xi}$ of the tree $\tau(M)$ (rule
    (Root)), it then passes the node @ (rule (Lam)).
    After this initialization part, it proceeds by traversing the term $N_0$ (rule (App)).
    At some point, while traversing $N_0$, some variable $y_i$ bound by the root of $N_0$ is visited. The traversal
    of $N_0$ is interrupted and jumps (rule (Var)) to the root of $\tau(N_i)$. The process then goes on with $\tau(N_i)$.
    When traversing $N_i$, if the traversal encounters a variable bound by the root of $\tau(N_i)$ then the traversal of $N_i$
    is interrupted and
    the traversal of $N_0$ resumes.  This schema is repeated until the traversal of $\tau(N_0)$ is completed\footnote{Since we are considering
    simply-typed terms, the traversal does indeed terminate. However this will not be true anymore in the \pcf\ case.}.

    The traversal of $M$ is therefore made of an initialization part followed by an interleaving of a traversal of $N_0$ and
    several traversals of $N_i$ for $i=1..p$. This schema is reminiscent of the way the evaluation copycat map $ev$ works in game semantics.

    The key idea is that every time the traversal pauses the traversal of a subterm and switches to another one,
    the jump is permitted by one of the four ``copycat'' rules (Var), (Answer-@-$\lambda$), (Answer-$\lambda$-var) or (Answer-var).
    We show by (a second) induction that these copycat rules define precisely what the copycat strategy $ev$ performs on sets of plays.

%    In the game semantics, the evaluation map (a copy-cat strategy) copies this opening move to an initial move $m_0$ in the game
%    $B_0$ and the game continues in $B_0$. We reflect this in the traversal : we make $t$ follow
%    the ``script'' given by the traversal $t^0_{m_0}$.
%    The rule (App) allow us to initiate this simulation  by visiting the  first move in $t^0_{m_0}$: the root of $\tau(N_0)$.
%
%    This simulation continues until it reaches a node $\alpha_0$ which is hereditarily justified by the root
%    $\tau(N_0)$: $\alpha_0$ is present in the reduction of traversal of $t^0_{m_0}$ therefore $\varphi_{N_0}(\alpha_0)$ is an un-hidden move played in $A_0$.
%
%    In the game semantics this corresponds to a move played in a component $A_k$ for some $k\in 1..p$ of
%    of the game $B_0$ in which case the evaluation map copies the move to an initial move $m_1$ in the corresponding component $B_k$.
%
%    To reflect this the traversal now opens up a new thread and simulates the traversal $t^k_{m_1}$.  Again, this simulation stops when we reach a node
%    $\alpha_1$ in $t^k_{m_1}$ which is hereditarily justified by the root of $\tau(N_k)$: $\alpha_1$ must be present in the reduction of traversal
%    of $t^k_{m_1}$ therefore $\varphi_{N_k}(\alpha_1)$ is an un-hidden move played in $A_k$.
%    In the game semantics, this move $\alpha$ is copied back to the component $B_k$ of the game $B_0$.
%
%    The traversal now resumes the simulation of $t^0_{m_0}$. And the process goes continuously.
\smallskip

\begin{proof}
Let $\Gamma \vdash M : T$ be a simply-typed term where $\Gamma =
x_1:X_1,\ldots x_n:X_n$. We assume that $M$ is already in
$\eta$-long normal form. By remark \ref{rem:corresp_proofreduction} we just need to
show that $\varphi_M(\travset(M)^{-@}) = \intersem{\Gamma \vdash M : T}$.
We proceed by induction on the structure of $M$:
\begin{enumerate}[$\bullet$]
    \item (abstraction) $M \equiv \lambda \overline{\xi}. N : \overline{Y} \rightarrow B$ where $\overline{\xi} = \xi_1:Y_1,\ldots \xi_n:Y_n$. On the first hand we have:
\begin{eqnarray*}
\intersem{\Gamma \vdash \lambda \overline{\xi}. N:T} &=& \Lambda^n( \intersem{\overline{\xi}, \Gamma \vdash N: B } ) \\
        &\simeq& \intersem{\overline{\xi}, \Gamma \vdash N: B } \ .
\end{eqnarray*}
On the other hand, the computation tree $\tau(N)$ is isomorphic to
$\tau(\lambda \xi_1\ldots \xi_n . N)$ (up to a renaming of the root
of the computation tree) and $\travset(N)$ is isomorphic to
$\travset(\lambda \xi_1\ldots \xi_n . N)$.
Hence we can conclude using the induction hypothesis.

  \item (variable) $M \equiv x_i$. Since $M$ is in $\eta$-long normal form, $x$ must be of ground
      type. The computation tree $\tau(M)$ and the arena $\intersem{\Gamma \rightarrow o}$ are represented below
      (value leaves and answer moves are not represented):
        $$ \tree[levelsep=6ex]{ \lambda }{\TR{x_i}} \hspace{2cm}
        \tree{ q_0 }
        {   \tree[linestyle=dotted]{q^1}{\TR{} \TR{} }
            \tree[linestyle=dotted]{q^2}{\TR{} \TR{} }
            \TR{\ldots}
            \tree[linestyle=dotted]{q^n}{\TR{} \TR{} }
        }
        $$

        Let $\pi_i$ denote the $i$th projection of the interaction game
        semantics. We have:
        \begin{align*}
        \intersem{M} &= \pi_i = \prefset(\{ \Pstr{(q0){q_0} \cdot (qi){q^i} \cdot (vqi-qi){v_{q^i}} \cdot (vq0-q0){v_{q_0}} } \ | \ v\in \mathcal{D} \})\ .
        \end{align*}

        It is easy to see that traversals of $M$ are precisely
        the prefixes of $ \Pstr{ (lmd)\lambda \cdot (xi){x_i}
        \cdot (vxi-xi){v_{x_i}} \cdot (vlmd-lmd){v_{\lambda}}}$.
        $M$ is in $\beta$-normal therefore $\travset(M)^{-@} =
        \travset(M)$ and since $\varphi_M(\lambda) =
        q_0$ and $\varphi_M(x_i) = q^i$, we have:
        $$ \varphi_M(\travset^{-@}(M)) = \varphi_M(\travset(M)) = \varphi_M(\prefset( \lambda \cdot x_i \cdot v_{x_i} \cdot v_{\lambda}))
         = \intersem{M} \ .
        $$


    \item (application) $M = N_0 N_1 \ldots N_p :o$ where $N_0$ is not a variable.
    We have the typing judgments $\Gamma \vdash N_0 N_1 \ldots
    N_p : o$ and $\Gamma \vdash N_i : B_i$ for $i\in 0..p$ where
    $B_0 = (B_1,\ldots,B_p,o)$ and $p\geq 1$.

    The tree $\tau(M)$ has the following form:
    $$ \tree[levelsep=6ex]{\lambda^{[r]}}
        { \tree[levelsep=6ex]{@}
            {
            \tree[levelsep=3mm,edge=\noedge]{\TR{{\lambda y_1 \ldots y_p}^{[r_0]}}}{\Tr[ref=t]{\pstribox{\tau(N_0)}}}
            \tree[levelsep=3mm,edge=\noedge]{\TR{[r_1]}}{\Tr[ref=t]{\pstribox{\tau(N_1)}}}
             \TR{\ldots}
            \tree[levelsep=3mm,edge=\noedge]{\TR{[r_p]}}{\Tr[ref=t]{\pstribox{\tau(N_p)}}}
        }}
    $$
    where $r_j$ denote the root of $\tau(N_j)$ for $j\in \{0..p\}$.

    We have:
    $$
    \intersem{\Gamma \vdash M : o}
            =  \underbrace{\langle \intersem{\Gamma \vdash N_0 : B_0}, \ldots \intersem{\Gamma \vdash N_p : B_p} \rangle}_{\Sigma} \,^\dagger\ \| \ ev
    $$

    We define the games $A$, $B$ and $C$ are defined as follows:
    \begin{eqnarray*}
        A &=& \Gamma = X_1 \times \ldots \times X_n\\
        B &=& \underbrace{((B_1' \times \ldots \times B_p') \rightarrow o')}_{B_0} \times B_1 \times \ldots \times B_p\\
        C &=& o \ .
    \end{eqnarray*}

    Figure \ref{fig:interaction_strategy_denotations} shows
    a tree-representation of $\intersem{M}$ which fixes the names of the different games involved in the interaction strategy.

%    Since $\varphi_M = \psi_M \union \varphi_{N_0} \union
%    \varphi_{N_1}$ the induction hypothesis gives us:
%    \begin{align}
%    \varphi_{M} (\travset^{-@}(N_0)) &= \intersem{\Gamma \vdash N_0 : B_0} \label{eqn:ih_1} \\
%    \varphi_{M}(\travset^{-@}(N_1)) &= \intersem{\Gamma \vdash N_1 : B_1} \label{eqn:ih_2}
%    \end{align}
\begin{enumerate}
\item[$\subseteq$]
    We first prove that $\intersem{\Gamma \vdash M : T}
    \subseteq \varphi_{M}( \travset^{-@}(M) )$. Suppose $u \in
    \intersem{\Gamma \vdash M : T}$. We give a constructive
    proof that there is a traversal $t$ of $M$ such
    that $\varphi_M(t-@) = u$ by induction on the length of $u$.
    Let $q_o$ and $q_0'$ be the initial question of $C$
    and $B_0$ respectively.

    \emph{Base cases}:
    \begin{compactitem}[-]
    \item If $u=\epsilon$ then we take the empty traversal $t=\epsilon$ formed
with \rulenamet{Empty}. Clearly $\varphi(t) = u$.
    \item If $|u|=1$ then $u=q_0$ is the initial move in $C$. The traversal $t=\lambda$ formed with the rule \rulenamet{Root} verifies $\varphi(t) = u$.
    \item If $|u|=2$ then necessarily $u = q_0 \cdot q_0'$. The rules \rulenamet{Root}, \rulenamet{App}
and \rulenamet{Lam} permit us to build the traversal $t = \lambda^{[r]} \cdot @ \cdot \lambda \overline{y}^{[r_0]}$ which clearly verifies $\varphi_M(t-@) = u$.
    \end{compactitem}

    \emph{Step cases}: Suppose that $u = w \cdot m \in \intersem{\Gamma \vdash M : T}$
    for some move $m \in M_T$ where
    $w = \varphi_M(t-@)$ for some traversal $t$ of $\tau(M)$
    and $|w|>1$.

    By unraveling the definition of $u \in \intersem{\Gamma \vdash M : T}$ we have:
    \begin{eqnarray*}
      &&      \left\{
            \begin{array}{ll}
                u \in L_T\\
                u \upharpoonright T^0  \in \Sigma^\dagger \\
                u \upharpoonright T^1  \in  ev
            \end{array}
            \right. \\
    & \mbox{or equivalently} & \left\{
    \begin{array}{ll}
        u \in L_T \\
        \hbox{for any initial $m$ in $!B_0^0 \otimes \ldots \otimes !B_p^0$ there is $j \in \{0..p\}$ such that } \\
        \left\{\begin{array}{ll}
            u \filter m \filter T^{00j} \in \intersem{N_j} \label{eq:def_z} \\
            u \filter m \filter T^{00k} = \epsilon \quad \mbox{ for every } k\in \{1..p\}\setminus\{j\} \label{eq:b}
        \end{array}
        \right. \\
        u \upharpoonright T^1  \in  ev
    \end{array}
    \right.
    \end{eqnarray*}

We recall that $m \in M_T$ is an equivalence class of moves from $\mathcal{M}_T$. For any game $A$ appearing in the interaction game $T$ we will write ``$m \in A$'' to mean that some citizen of the class $m$ belongs to the set of moves $M_A$. Similarly, for any sub-interaction game $T'$ of $T$, we write ``$m \in T'$'' to mean that some citizen of the class $m$ belongs to the set of moves $\mathcal{M}_A$.

We do a case analysis on $m$: we either have $m\in C$ or $m\in T^0$:
    \begin{enumerate}[-]
    \item Suppose $m \in C$. $m$ is played by the strategy $ev$ whose plays do not contain any internal move. Hence $m$ is either $q_0$ or $v_{q_0}$ for some
    $v\in\mathcal{D}$. But since $q_0$ can occur only once in
    $u$ and $|u|>1$, $m$ must be $v_{q_0}$ for some
    $v\in \mathcal{D}$.  Moreover $m$ is a P-move played by the
    copy-cat strategy $ev$ in $B,C$ therefore it is the copy
    of the some move $v_{q_0'}$ answering the question $q_0'$ in the sub-game $o'$.

    In fact this move $v_{q_0'}$ is precisely $w$'s last move. Indeed
    suppose that $w = \ldots v_{q_0'} \cdot w'$. The play
    $w_{\prefixof v_{q_0'}}\filter A,B$ is complete since its
    first move $q_0'$ is answered by $v_{q_0'}$. Therefore by
    Lemma \ref{lem:inter_complete}(ii), $w_{\prefixof
    v_{q_0'}}\filter T^0$ is maximal. Thus moves in $w'$ must
    be played in $T^1$ by $ev$, but since $ev$ does not play internal
    moves, $w'$ is necessarily empty.

    Consequently, by the induction hypothesis, the last move in $t$ is $\varphi(v_{q_0'}) = v_{\lambda y_1}$.
    The rules \rulenamet{Answer-@-$\lambda$} and \rulenamet{Answer-$\lambda$-@} permits us to extend
    the traversal $t$ into $t \cdot v_@ \cdot v_{\lambda \overline{\xi}}$ where $v_@$ and $v_{\lambda
    \overline{\xi}}$ point to the second and first node of $t$ respectively. Clearly we have $\varphi_M((t\cdot v_@ \cdot v_{\lambda \overline{\xi}})-@) = u$.

    \item Suppose $m\in T^0$. Then $m$ is hereditarily justified by some initial move $b$ in $B_j$ for some $j\in \{0..p\}$.

        Since $u \filter b \filter T^{00j} \in \intersem{N_j}$, the outermost induction hypothesis gives us:
        \begin{equation}
        u \filter b \filter T^{00j} = \varphi_{N_j}(t_j-@)  \label{eqn:corresp_outmost_ih}
        \end{equation}
          for some traversal $t_j \in \travset(N_j)$. W.l.o.g we can assume that $t_j^\omega \neq @$.
        By Corollary \ref{cor:varphi_bij}, $\varphi_{N_j}$ is a bijection from $\travset(N_j)^{-@}$ to
        $\varphi_{N_j}( \travset(N_j)^{-@})$ therefore $t_j-@ = \varphi^{-1}_{N_j}( (u \filter b) \filter T^{00j} )$
        and we have:
        \begin{align}
         t_j - @ &= \varphi^{-1}_{N_j}(u \filter b \filter T^{00j}) &  \nonumber \\
                        &\in \varphi^{-1}_{M}(u \filter b \filter T^{00j}) & \parbox[t]{7cm}{($\varphi_M = \psi_M \union \Union_{k\in \{0..p\}} \varphi_{N_k}$ by definition.)} \label{eqn:proof_corres_1}
        \end{align}

        Note that $\varphi^{-1}_{M}(\ip(u \filter b \filter T^{00j}))$ is not a traversal but \emph{a set of} traversals since $\varphi_{M}^{-1}$ is not necessarily bijective on $\intersem{N_j}$. Thus we have to use set-membership in the equation instead of traversal equality.

        Since $\varphi^{-1}_{M}$ is monotonous and $u \filter b \filter T^{00j} \subseqof u \subseqof w$, all the traversals in $\varphi^{-1}_{M}(u \filter b \filter T^{00j})$ are subsequences of $\varphi^{-1}_{M}( w )$ thus:
        \begin{align*}
        t_j - @ &\subseqof \varphi^{-1}_{M}( w ) & \mbox{(by Eq. \ref{eqn:proof_corres_1})}\\
                &= t -@ & \mbox{($\varphi_{M}$ is bijective on $\travset(M)$ by Cor. \ref{cor:varphi_bij}).}
        \end{align*}

      Thus by Lemma \ref{lem:minus_at_plus_at}(ii) we have
        $\ip( t_j) \subseqof t$.

    Furthermore:
     \begin{align*}
    t_j\subseqof t
        &\implies \varphi_M (t_j) \subseqof \varphi_M (t) \\
        &\iff (w \cdot m) \filter b \filter T^{00j} \subseqof  w \\
        &\iff (w \filter b \filter T^{00j}) \cdot m \subseqof  w  & \mbox{ ($m$ is h.j. by $b$ and belongs to $T^{00j}$)} \\
        &\implies \left( \left(w \filter b \filter T^{00j} \cdot m \right) \filter b \filter T^{00j} \right) \cdot m \subseqof w & \mbox{ (by iterating the previous equation)} \\
        &\iff (w \filter b \filter T^{00j}) \cdot m  \cdot m \subseqof  w \ .
    \end{align*}
    The last equation is false since a given move cannot occur twice consecutively in a legal interaction play! Hence $t_j\not\subseqof t$.

    \begin{enumerate}[(a)]
    \item  Suppose $t_j$'s last move is \emph{not} visited by the rule \rulenamet{InputVar} nor
        $\rulename{InputVar^{val}}$. Since $\ip( t_j) \subseqof t$ and $t_j$ is a traversal of the subterm $N_j$ of $M$, by Lemma \ref{lem:local_traversal_progression}
        we have either $t_j\subseqof t$ or $t \cdot t_j^\omega$ is a traversal of $M$
        where $t_j^\omega$'s pointer is the same as in $t_j$. Hence, since $t_j\not\subseqof t$, $t \cdot t_j^\omega$ is a traversal of $M$.

        Furthermore we have
        \begin{align*}
            \varphi_M (t_j^\omega) &= (\varphi_M (t_j-@))^\omega & \mbox{($t_j^\omega \neq @$ by assumption)}\\
                                   &= ((w \cdot m) \filter b\filter T^{00j})^\omega & \mbox{(by Eq. \ref{eqn:corresp_outmost_ih})}\\
                                   &= ((w \filter b\filter T^{00j}) \cdot m))^\omega & \mbox{($m$ is h.j. by $b$ and belongs to $T^{00j}$)}\\
                                   &= m
        \end{align*}
        and therefore
        \begin{align*}
          \varphi_{M}((t \cdot t_j^\omega)-@)  &=  \varphi_{M}(t -@)  \cdot \varphi_{M}(t_j^\omega-@)\\
                &=   w \cdot \varphi_{M}(t_j^\omega-@) & \mbox{(by the innermost induction hypothesis)}\\
                &=   w \cdot m & \mbox{(by the previous equation).}
        \end{align*}
        Hence the traversal $t \cdot t_j^\omega$ meets the requirement.

    \item Suppose $t_j$'s last move is visited with the rule \rulenamet{InputVar}.

    Then $t_j$ is of the form
    $$\Pstr[18pt]{ t_j = t' (z)z \cdot t'' \cdot (n-z){t_j^\omega}}$$
for some $z \in N_\lambda^{\filter r_j}$ (see remark \ref{rem:inputvar})
and some input-variable $x \in N^{\filter r_j}_{var}$ occurs in $z\cdot t'$ such that $x$ is the pending node in $\ip t_j = t' \cdot z \cdot t''$ ({\it i.e.}~ with $?(t' \cdot z \cdot t'')^\omega = x$).

Suppose that $z\in N^{\filter r}$ then $z$ is a free variable of $M$ (and $N_j$).
Since the O-view of $t_j$ coincides with the O-view of $t$

    \item Suppose $t_j$'s last move is visited with the rule $\rulename{InputVar^{val}}$.
    This case is similar to the previous one but the rule $\rulename{InputVar^{val}}$ is used instead
    of $\rulename{InputVar}$.
    \end{enumerate}



\notetoself{PIECE OF OLD PROOF
%   \item Suppose that $m,m^1 \in T^{000}$.
%    The strategy $ev$ is responsible for switching of thread
%    in $B_0$ therefore, in the interaction semantics, there
%    must be a copycat move in-between two moves belonging to
%    two different threads. Since $m$ and $m^1$ are
%    consecutive moves in the sequence $u$, they must belong
%    to the same thread i.e. there are hereditarily justified
%    by the same initial $m_0$ in $B_0$.

Suppose that $m \in T^{000}$ and $m^1 \in T^{001}$.

    $t$ is obtained from $t-@$ by applying the
    transformation $+@$. We apply the same transformation to
    $u$ in order to make $O$-questions and $P$-questions in
    $u$ match with $\lambda$-nodes and variable nodes in
    $t'$ respectively. We write this sequence $u+@$. The
    $+@$ operation inserts nodes in the sequence but not at
    the end, therefore $m^1$, the last move in $u'$, is also
    the last move in $u'+@$. Let us note $n^1$ for the last
    move in $t'$.


        $n^1$ is a variable node then $m^1$ is a P-move and $m$ is an O-move
            and therefore $m$ is the copy of $m^1$ duplicated in $B_1$ by the evaluation strategy.
            Consequently, $m^1$ points to some $m^2$ and $m$ points to the node preceding $m^2$ denoted by $m^3$.
            The diagram below shows an example of such sequence:
                $$
                \begin{array}{ccccccccccc}
                  & A & \longrightarrow & ( (B_1' &\rightarrow & o') & \times & B_1 ) & \longrightarrow & o' \\
                  &&& &&&&&& \rnode{q0}{q_0 (\lambda \overline{\xi})} & O\\
                  &&& &&&&&  \\
                O &&& && \rnode{q1}{q_0' (\lambda \overline{y})} &&&&& P \\
                P &&& \rnode{m3}{m^3 (y_1)} &&&&&&& O \\
                O &&& &&&& \rnode{m2}{m^2 (\lambda \overline{z}^{[r_1]})} &&& P \\
                P &&& &&&& \rnode{m1}{m^1 (z)} &&& O \\
                O &&& \rnode{m}{m} &&&&&&& P \\
                \end{array}
                \ncline[nodesep=3pt]{->}{q1}{q0} \mput*{@}
                \nccurve[nodesep=3pt,ncurv=2,angleA=180,angleB=180]{->}{m1}{m2}
                \ncarc[nodesep=3pt,ncurv=1,angleA=90,angleB=180]{->}{m3}{q1}
                \ncarc[nodesep=3pt,ncurv=1,angleA=90,angleB=180]{->}{m}{m3}
                \ncline[nodesep=3pt]{->}{m2}{q0}
                $$

        $t'$  and $u+@$ have the following forms:
        \begin{eqnarray*}
                t'&=& \Pstr{ \ldots \cdot n^3 \cdot (n2){n^2} \cdot \ldots \cdot (n1-n2,30){n^1} } \\ \\
                u+@ &=& \Pstr{ \ldots \cdot (m3){m^3} \cdot (m2){m^2} \cdot \ldots \cdot (m1-m2,30){m^1} \cdot (m-m3,30){m} }
        \end{eqnarray*}

        Since $n^1$ is a variable node, $n^2$ must be a $\lambda$-node.
        $n^3$ is either a variable node or an @-node. In fact $n^3$ is necessarily a variable node. Indeed,
        $n^3$ is mapped to $m^3$ by $\varphi_{N_0}$ and $m^3$ belongs to $B_i'$ (i.e. it is not
        an internal move of $T^0$). The function $\varphi_{N_0}$ is defined in such a way that
        only nodes which are hereditarily justified by $r_0$ are mapped to nodes in $B_j'$.
        Consequently, since @-node don't have justifier, $n^3$ cannot be an @-node.

        Hence $n^1$ is a variable node, $n^2$ is a $\lambda$-node and $n^3$ is a variable node.


        We  can therefore apply the (Var) rule to $t'$ and we obtain a traversal of the following form:

        \begin{eqnarray*}
            t&=& \Pstr{ \ldots \cdot (n3){n^3} \cdot (n2){n^2} \cdot \ldots \cdot (n1-n2,30){n^1} \cdot (n-n3,30){n} }
        \end{eqnarray*}

        We have $\varphi(t'-@) = u'$ by the induction hypothesis and $\varphi(n) = m$ by definition of $\varphi$.
        Therefore since $m$ and $n$ point to the same position we have $\varphi(t-@) = u$.
}

    \end{enumerate}

\item[$\supseteq$]
  For the converse, $\varphi_{M}( \travset^{-@}(M) ) \subseteq \intersem{M}$, it is an easy induction
  on the traversal rules. We omit the details here.


\end{enumerate}


    \item (application') $M = x_i N_1 \ldots N_p :o$ with $X_i = B_0 = (B_1' \times \ldots \times B_p') \rightarrow o'$. The tree $\tau(M)$ has the following form:
    $$ \tree[levelsep=6ex]{\lambda^{[r]}}
        { \tree[levelsep=6ex]{x_i}
            {
            \tree[levelsep=3mm,edge=\noedge]{\TR{[r_1]}}{\Tr[ref=t]{\pstribox{\tau(N_1)}}}
             \TR{\ldots}
            \tree[levelsep=3mm,edge=\noedge]{\TR{[r_p]}}{\Tr[ref=t]{\pstribox{\tau(N_p)}}}
        }}
    $$
    The interaction strategy
    $\intersem{\Gamma \vdash M : o}
            =  \underbrace{\langle \pi_i, \intersem{\Gamma \vdash N_1 : B_1}, \ldots \intersem{\Gamma \vdash N_p : B_p} \rangle}_{\Sigma} \,^\dagger\ ;^{\{1..p\}} \ ev$
    is represented on Figure \ref{fig:interaction_strategy_denotations}.

    The proof is identical to the previous case except that in the $\subseteq$ part of the proof:
    \begin{itemize}
        \item In the base case of the induction where $|u|=2$,
        the rule \rulenamet{InputVar} is used instead of \rulenamet{App} to visit the node $x$ instead of $@$;
        \item in the step case of the induction, for the subcase $m\in C$, the rules \rulenamet{Answer-var-$\lambda$} and \rulenamet{Answer-$\lambda$-var} are used instead of \rulenamet{Answer-@-$\lambda$} and \rulenamet{Answer-$\lambda$-@} respectively;
        \item in the step case $m\in T^0$, when $m$ is hereditarily justified by a move $b \in B_j$ for
         $j\in \{1 .. p\}$ the proof remains unchanged. The case where $m$ is hereditarily justified by a move $b \in B_0$ is treated as follows: $m$ is played by the projection strategy $\pi$ denoting $x$.
         Since $m$ is played in $B_0' = B_1' \times \ldots \times B_p'$ it must be also hereditarily justified by some initial move $b'$ of $B_k'$ for some $k \in \{1.. p\}$ or by an initial move in $o'$. But moves of $B_k'$ are $\sim$-equivalent to the corresponding move in $B_k$ and similarly $o'$ is $\sim$-equivalent to $o$, therefore we fall back to the previous case of the induction where $m$ is hereditarily justified by some initial move $b\in B_k$ for $k\in \{1..p\}$ or some initial move in $o$!
    \end{itemize}


\end{enumerate}


\end{proof}


\begin{corollary} \hfill
\begin{enumerate}[i.]
\item Let $\tau(M')$ be a subtree of $\tau(M)$ for some subterm $M'$ of $M$, and  $N'$ denote the set
of nodes of $\tau(M')$. Then
$$t \in \travset(M) \implies t\filter N' \in \travset(M') \ .$$

\item If $M$ is in $\beta$-normal form then for any traversal $t$,
$\varphi_M(t)$ is a maximal play if and only if $t$ is a maximal
traversal.
\end{enumerate}
\end{corollary}
\begin{proof}
\begin{enumerate}[i.]
\item
 \todo
\item If $M$ is in $\beta$-normal form then
$\travset(M)^{\upharpoonright r} = \travset(M)$ therefore
$\varphi$ defines a bijection on $\travset(M)$. Let $t$ be a
traversal such that $\varphi(t)$ is a maximal play. Let $t'$ be
a traversal such that $t \sqsubseteq t'$. By monotonicity of
$\varphi$ we have $\varphi(t) \sqsubseteq \varphi(t')$ which
implies $\varphi(t) = \varphi(t')$ by maximality of $\varphi(t)$
which in turn implies $t'=t$ by injectivity of $\varphi$. The
other direction is proved identically using injectivity and
monotonicity of $\varphi^-1$.
\end{enumerate}
\end{proof}
\smallskip The following diagram recapitulates the main results of
this section:
$$
\xymatrix @C=6pc{
                                           & \travset(M)^{-@} \ar@/_/[dl]_{+@}  \ar[r]^{\varphi_M}_\cong & \intersem{M} \ar@/_/[dd]_{\_ \upharpoonright \sem{\Gamma\rightarrow T}} \\
\travset(M) \ar@/_/[ur]_{-@}^{} \ar[dr]^{\_ \upharpoonright r}  \\
                                           & \travset(M)^{\upharpoonright r} \ar[r]^{\varphi_M}_\cong & \sem{M} \ar@/_/[uu]^{\cong}_{\mbox{full uncovering}}
}
$$


\begin{example}
Take $M = \lambda f z . (\lambda g x . f x) (\lambda y. y) (f z) :
((o,o),o, o)$.  The figure below represents the computation tree
(left tree), the arena $\sem{((o,o),o, o)}$ (right tree) and
$\psi_M$ (dashed line). (Only question moves are shown for clarity.)
The justified sequence of nodes $t$ defined hereunder is an example
of traversal:

\begin{tabular}{lp{6.3cm}}
$\tree[levelsep=2.5ex,treesep=0.3cm]{ \Rnode{root}{\lambda f z} }
     {  \tree{@}
        {   \tree{\lambda g x}{
                  \tree{\Rnode{f}{f^{[1]}}}{
                            \tree{\Rnode{lmd}{\lambda^{[2]}}}
                            {\TR{x}}
                  }
                }
            \tree{ \lambda y }{\TR{y}}
            \tree{\lambda ^{[3]}}{
                \tree{\Rnode{f2}{f^{[4]}}} {
                \tree{\Rnode{lmd2}{\lambda^{[5]}}}{\TR{\Rnode{z}{z}}}
                }
            }
        }
     }
\hspace{1cm}
  \tree[levelsep=8ex,treesep=0.3cm]{ \Rnode{q0}q^0 }
    {   \pstree[levelsep=4ex]{\TR{\Rnode{q1}{q^1}}}{\TR{\Rnode{q2}{q^2}}}
        \TR{\Rnode{q3}q^3}
        \TR{\Rnode{q4}q^4}
    }
\psset{nodesep=1pt,arrows=->,arrowsize=2pt 1,linestyle=dashed,linewidth=0.3pt}
\ncline{->}{root}{q0} \mput*{\psi_M}
\ncarc[arcangle=-25]{->}{z}{q3}
\ncarc[arcangle=10]{->}{f}{q1}
\ncarc[arcangle=10]{->}{lmd}{q2}
\ncline{->}{f2}{q1}
\ncline{->}{lmd2}{q2}$
\hspace{2cm}
&
\begin{asparablank}
  \item  \Pstr[0.8cm]{
t = (n){\lambda f z} \
(n2){@} \
(n3-n2,60){\lambda g x} \
(n4-n,45){f^{[1]}} \
(n5-n4,45){\lambda^{[2]}} \
(n6-n3,45){x} \
(n7-n2,35){\lambda^{[3]}} \
(n8-n,35){f^{[4]}} \
(n9-n8,45){\lambda^{[5]}} \
(n10-n,35){z}
}

\item \Pstr[0.9cm]{
t\upharpoonright r = (n){\lambda f z} \ (n4-n,50){f}^{[1]} \
(n5-n4,60){\lambda}^{[2]} \ (n8-n,45){f}^{[4]} \
(n9-n8,60){\lambda}^{[5]} \ (n10-n,40){z}}
\item
\Pstr[0.8cm]{ {\psi_M(t\upharpoonright r) =\ } (n){q^0}\
(n4-n,60){q^1}\ (n5-n4,60){q^2}\ (n8-n,45){q^1}\ (n9-n8,60){q^2}\
(n10-n,38){q^3} \in \sem{M}\ .}
\end{asparablank}
\end{tabular}
\end{example}


    \section{Extension to PCF and IA terms}
         \input{corresp_pcf_ia.texi}


    \section{Applications}


\chapter{Game-Semantic Analysis via a Syntactic Argument}
    \label{chap:syntactic_gamesem}
    
\section{A game-semantic account of safety}
\label{sec:gamesemaccount} Our aim is to characterize safety by game
semantics. We shall assume that the reader is familiar with the
basics of game semantics; For an introduction, we recommend
\cite{abramsky:game-semantics-tutorial}. Recall that a
\emph{justified sequence} over an arena is an alternating sequence
of O-moves and P-moves such that every move $m$, except the opening
move, has a pointer to some earlier occurrence of the move $m_0$
such that $m_0$ enables $m$ in the arena. A \emph{play} is just a
justified sequence that satisfies Visibility and Well-Bracketing. A
basic result in game semantics is that $\lambda$-terms are denoted
by \emph{innocent strategies}, which are strategies that depend only
on the \emph{P-view} of a play. The main result
(Theorem~\ref{thm:safeincrejust}) of this section is that if a
$\lambda$-term is safe, then its game semantics (is an innocent
strategy that) is, what we call, \emph{P-incrementally justified}. In such a
strategy, pointers emanating from the P-moves of a play are uniquely
reconstructible from the underlying sequence of moves and pointers
from the O-moves therein: specifically a P-question always points to
the last pending O-question (in the P-view) of a greater order.

The proof of Theorem~\ref{thm:safeincrejust} depends on a
Correspondence Theorem (see the Appendix) that relates the strategy
denotation of a $\lambda$-term $M$ to the set of \emph{traversals}
over a souped-up abstract syntax tree of the $\eta$-long form of $M$.
In the language of game semantics, traversals are just (concrete
representations of) the \emph{uncovering} (in the sense of Hyland
and Ong \cite{hylandong_pcf}) of plays in the strategy denotation.

The useful transference technique between plays and traversals was
originally introduced by one of us \cite{OngLics2006} for studying
the decidability of monadic second-order theories of infinite structures generated by
higher-order grammars (in which the $\Sigma$-constants or terminal symbols are at most
order 1, and \emph{uninterpreted}).
% In this setting, free variables are interpreted
% as constructors and therefore they do not have the ``full power'' of
% true free variables and are limited to order $1$ at most. Also,
% although the grammar can perform higher-order computations, the
% structure being studied is itself of ground type.
In the Appendix, we present an extension of this framework to the
general case of the simply-typed lambda calculus with free variables
of any order. A new traversal rule is introduced to handle nodes
labelled with free variables. Also new nodes are added to the
computation tree to account for the answer moves of the game
semantics, thus enabling the framework to model languages with
interpreted constants such as \pcf~(by adding traversal rules to
handle constant nodes).

\subsection*{Incrementally-bound computation tree}
 In \cite{OngLics2006} the computation tree of a grammar is
defined as the unravelling of a finite graph representing the \emph{long
transform} of a grammar. Similarly we define the computation tree of
a $\lambda$-term as an abstract syntax tree of its $\eta$-long
normal form.  We write $l\langle t_1, \ldots, t_n \rangle$ with $n
\geq 0$ to denote the ordered tree with a root labelled $l$ with $n$
child-subtrees $t_1$, \ldots, $t_n$. In the following we consider arbitrary
simply-typed terms.

\begin{definition}\rm
\label{dfn:comptree}
  The \defname{computation tree} $\tau(M)$ of a simply-typed term
  $\Gamma \stentail M:T$ with variable names in a countable set
  $\mathcal{V}$ is a tree with labels in $$ \{ @ \} \union \mathcal{V}
  \union \{ \lambda x_1 \ldots x_n \ | \ x_1 ,\ldots, x_n \in
  \mathcal{V}, n\in\nat \}$$ defined from its $\eta$-long form as follows. Suppose $\overline{x} = x_1 \ldots x_n$ for $n\geq 0$ then
\begin{eqnarray*}
  \mbox{for $m\geq 0$, $z \in \mathcal{V}$: } \tau(\lambda \overline{x} . z s_1 \ldots s_m : o) &=& \lambda \overline{x} \langle z \langle\tau(s_1),\ldots,\tau(s_m)\rangle\rangle \\
  \mbox{for $m \geq 1$: } \tau(\lambda \overline{x} . (\lambda y.t) s_1 \ldots s_m :o) &=& \lambda \overline{x} \langle @ \langle \tau(\lambda y.t),\tau(s_1),\ldots,\tau(s_m) \rangle \rangle \ .
\end{eqnarray*}
\end{definition}

\begin{example}
\label{examp:comptree}
  Take $\stentail \lambda f^{o \typear o} .
(\lambda u^{o \typear o} . u) f : (o \typear o) \typear
o \typear o$.
\bigskip

\noindent
\begin{tabular}{cc}
Its $\eta$-long normal form is: & Its computation tree is:\\[8pt]
\begin{minipage}{0.45\textwidth}
\centering
$\begin{array}{ll}
 &\stentail  \lambda f^{o \typear o} z^o . \\
&\qquad(\lambda u^{o \typear o} v^o . u (\lambda.v)) \\
&\qquad(\lambda y^o. f y) \\
&\qquad(\lambda.z) \\
&: (o \typear o) \typear o \typear o
\end{array}$
\end{minipage}
&
\begin{minipage}{0.45\textwidth}
\centering
\psset{levelsep=5ex,linewidth=0.5pt,nodesep=1pt,arcangle=-20,arrowsize=2pt 1}
${\pstree{\TR{\lambda f z}}{\pstree{\TR{@}}{\pstree{\TR{\lambda u v}}{\pstree{\TR{u}}{\pstree{\TR{\lambda }}{\TR{v}}}}\pstree{\TR{\lambda y}}{\pstree{\TR{f}}{\pstree{\TR{\lambda }}{\TR{y}}}} \pstree{\TR{\lambda }}{\TR{z}}}}
}$
\end{minipage}
\end{tabular}
\end{example}

\begin{example}
  Take $\stentail \lambda u^o v^{((o \typear o) \typear o)} . (\lambda x^o . v (\lambda z^o . x)) u : o \typear ((o \typear o) \typear o) \typear o$.
  \bigskip

\noindent
\begin{tabular}{cc}
Its $\eta$-long normal form is: & Its computation tree is:\\[8pt]
\begin{minipage}{0.45\textwidth}
\centering
$\begin{array}{ll}
 &\stentail  \lambda u^o v^{((o \typear o) \typear o)} . \\
&\qquad(\lambda x^o . v (\lambda z^o . x)) u \\
&: o \typear ((o \typear o) \typear o) \typear o
\end{array}$
\end{minipage}
&
\begin{minipage}{0.45\textwidth}
\centering
$\pstree{\TR{\lambda u v}}{\pstree{\TR{@}}{\pstree{\TR{\lambda x}}{\pstree{\TR{v}}{\pstree{\TR{\lambda z}}{\TR{x}}}}\pstree{\TR{\lambda }}{\TR{u}}}}
$
\end{minipage}
\end{tabular}
\end{example}

Even-level nodes are $\lambda$-nodes (the root is on level 0). A
single $\lambda$-node can represent several consecutive variable
abstractions or it can just be a \emph{dummy lambda} if the
corresponding subterm is of ground type.  Odd-level nodes are
variable or application nodes.

The \defname{order} of a node $n$, written $\ord{n}$, is defined as
follows: @-nodes have order $0$. The order of a variable-node is the
type-order of the variable labelling it. The order of the root node
is the type-order of $(A_1,\ldots,A_p, T)$ where $A_1,\ldots, A_p$
are the types of the variables in the context $\Gamma$. Finally, the
order of a lambda node different from the root is the type-order of
the term represented by the sub-tree rooted at that node.

We say that a variable node $n$ labelled $x$ is \defname{bound} by a
node $m$, and $m$ is called the \defname{binder} of $n$, if $m$ is
the closest node in the path from $n$ to the root such that $m$ is
labelled $\lambda \overline{\xi}$ with $x\in \overline{\xi}$.


We introduce a class of computation trees in which the binder node
is uniquely determined by the nodes' orders:
\begin{definition}\rm
  A computation tree is \defname{incrementally-bound} if for all
  variable node $x$, either $x$ is \emph{bound} by the first
  $\lambda$-node in the path to the root with order $> \ord{x}$, or $x$
  is a \emph{free variable} and all the $\lambda$-nodes in the path to
  the root except the root have order $\leq \ord{x}$.
\end{definition}

\begin{proposition}[Safety and incremental-binding] \hfill
\label{prop:safe_imp_incrbound}
\begin{enumerate}[(i)]
\item If $M$ is safe then $\tau(M)$ is incrementally-bound.
\item Conversely, if $M$ is a \emph{closed} simply-typed term and $\tau(M)$
is incrementally-bound then $M$ is safe.
\end{enumerate}
\end{proposition}
\proof
  (i) Suppose that $M$ is safe. By Lemma
  \ref{prop:safe_iff_elnfsafe} the $\eta$-long form of $M$ is safe
  therefore $\tau(M)$ is the tree representation of a safe term.

In the safe lambda calculus, the variables in the context with the
lowest order must be all abstracted at once when using the
abstraction rule. Since the computation tree merges consecutive
abstractions into a single node, any variable $x$ occurring free in
the subtree rooted at a node $\lambda \overline{\xi}$ different from
the root must have order greater or equal to $\ord{\lambda
  \overline{\xi}}$. Conversely, if a lambda node $\lambda
\overline{\xi}$ binds a variable node $x$ then $\ord{\lambda
  \overline{\xi}} = 1+\max_{z\in\overline{\xi}} \ord{z} > \ord{x}$.

Let $x$ be a bound variable node. Its binder occurs in the path from
$x$ to the root, therefore, according to the previous observation,
$x$ must be bound by the first $\lambda$-node occurring in this path
with order $>\ord{x}$. Let $x$ be a free variable node then $x$ is
not bound by any of the $\lambda$-nodes occurring in the path to the
root. Once again, by the previous observation, all these
$\lambda$-nodes except the root have order smaller than $\ord{x}$.
Hence $\tau$ is incrementally-bound.

(ii) Let $M$ be a closed term such that $\tau(M)$ is
incrementally-bound.  W.l.o.g. we can assume that $M$ is in $\eta$-long
form.  We prove that $M$ is safe by induction on its structure. The
base case $M = \lambda \overline{\xi} . x$ for some variable $x$ is
trivial.  \emph{Step case:} If $M = \lambda \overline{\xi} . N_1
\ldots N_p$.  Let $i$ range over $1..p$. We have $N_i \equiv \lambda
\overline{\eta_i} . N'_i$ for some non-abstraction term $N'_i$. By
the induction hypothesis, $\lambda \overline{\xi} . N_i = \lambda
\overline{\xi} \overline{\eta_i} . N'_i$ is a safe closed term, and
consequently $N'_i$ is necessarily safe. Let $z$ be a free variable
of $N'_i$ not bound by $\lambda \overline{\eta_i}$ in $N_i$. Since
$\tau(M)$ is incrementally-bound we have $\ord{z} \geq \ord{\lambda
  \overline{\eta_1}} = \ord{N_i}$, thus we can abstract the variables $\overline{\eta_1}$ using \rulenamet{abs} which shows that $N_i$ is safe.  Finally
we conclude $\sentail M = \lambda \overline{\xi} . N_1 \ldots N_p :
T$ using the rules \rulenamet{app} and \rulenamet{abs}.  \qed



The assumption that $M$ is closed is necessary. For instance for
$x,y:o$, the computation trees $\tau(\lambda x y .x)$ and
$\tau(\lambda y . x)$ are both incrementally-bound but $\lambda x y
.x$ is safe and $\lambda y . x$ is not.

\subsection*{P-incrementally justified strategy}

We now consider the game-semantic model of the simply-typed lambda
calculus. The strategy denotation of a term-in-context $\Gamma
\stentail M : T$ is written $\sem{\Gamma
\stentail M : T}$. We define the \defname{order} of a move $m$,
written $\ord{m}$, to be the length of the path from $m$ to its
furthest leaf in the arena minus 1. (There are several ways to
define the order of a move; the definition chosen here is sound in
the current setting where each question move in the arena enables at
least one answer move.)
%{\it i.e.}~height of the subarena rooted at $q$ minus 2.

\begin{definition}\rm
  A strategy $\sigma$ is said to be \defname{P-incrementally
    justified} if for any play $s \, q \in \sigma$ where $q$ is a
  P-question, $q$ points to the last unanswered O-question in $\pview{s}$ with
  order strictly greater than $\ord{q}$.
\end{definition}
Note that although the pointer is determined by the P-view, the
choice of the move itself can be based on the whole history of the
play. Thus P-incremental justification does not imply innocence.

The definition suggests an algorithm that, given a play of a
P-incrementally justified denotation, uniquely recovers the pointers
from the underlying sequence of moves and from the pointers
associated to the O-moves therein. Hence:
\begin{lemma}
\label{lem:incrjustified_pointers_uniqu_recover} In P-incrementally
justified strategies, pointers emanating from P-moves are
superfluous.
\end{lemma}

\begin{example}
Copycat strategies, such as the identity strategy $id_A$ on game $A$
or the evaluation map $ev_{A,B}$ of type $(A \Rightarrow B) \times A
\typear B$, are all P-incrementally justified.\footnote{In such
strategies, a P-move $m$ is justified as follows: either $m$ points
to the preceding move in the P-view or the preceding move is of
smaller order and $m$ is justified by the second last O-move in the
P-view.}
\end{example}
%%%% the following example is wrong : ev is P-ij.
%
%\begin{example}
%Take the evaluation map $ev : (o^1 \Rightarrow o^2) \times o^3 \rightarrow o^4$ and the play $s = q^4 q^2 q^1 q^3 \in \sem{ev}$. We have $\ord{q^2} = 1 > \ord{q^1} = \ord{q^3} = 0$. Now $q^3$ points to $q^4$ but $q^2$ is the last unanswered O-question in $\pview{s}= s$ with order $>\ord{q^3}$, hence $\sem{ev}$ is not P-incrementally justified.
%\end{example}



The Correspondence Theorem~\ref{thm:correspondence}
% and Lemma \ref{lem:betanf_wellbehavedconst_trav_pview_red}
gives us the following equivalence:
\begin{proposition} % [Incremental-binding vs P-incremental justification]
\label{prop:Nher_incrbound_and_incrjustified} Let $\Gamma \stentail
M : T$ be a $\beta$-normal term. The computation tree $\tau(M)$ is
incrementally-bound if and only if $\sem{\Gamma \stentail M : T}$ is
P-incrementally justified.
\end{proposition}


\parpic[r]{
\pssetcomptree
\raisebox{-12pt}
{$\tree{\lambda^3}{\tree{f^2}{ \tree{\lambda y^1}{\TR{x^0} }}}$}
}
%\noindent \emph{Example:}
\begin{example}
Consider the $\beta$-normal term $\Gamma\stentail f (\lambda y .x) :
o$ where $y:o$ and $\Gamma =f:((o,o),o),~x:o$. The figure on the
right represents its computation tree with the node orders given as
superscripts.  The node $x$ is not incrementally-bound therefore $\tau(f
(\lambda y .x))$ is not incrementally-bound and by Proposition
\ref{prop:Nher_incrbound_and_incrjustified}, $\sem{\Gamma \stentail
f (\lambda y .x) : o}$ is not incrementally-justified (although
$\sem{\Gamma \stentail f : ((o,o),o)}$ and $\sem{\Gamma \stentail
\lambda
  y. x : (o,o)}$ are).
\end{example}
\smallskip

Propositions \ref{prop:safe_imp_incrbound} and
\ref{prop:Nher_incrbound_and_incrjustified} allow us to show the
following:
\begin{theorem}[Safety and P-incremental justification]
\label{thm:safeincrejust} \hfill
\begin{enumerate}[(i)]
\item If $\Gamma \sentail M : T$ then $\sem{\Gamma \sentail M : T}$ is P-incrementally justified.
\item If $\stentail M : T$ is a closed simply-typed term and $\sem{\stentail M : T}$ is P-incrementally justified then the $\beta$-normal form of $M$ is safe.
\end{enumerate}
\end{theorem}
\proof (i) Let $M$ be a safe simply-typed term. By Lemma
\ref{lem:safered_preserves_safety}, its $\beta$-normal form $M'$ is
also safe. By Proposition \ref{prop:safe_imp_incrbound}(i),
$\tau(M')$ is incrementally-bound and by Proposition
\ref{prop:Nher_incrbound_and_incrjustified}, $\sem{M'}$ is an
incrementally-justified. Finally the soundness of the game model
gives $\sem{M} = \sem{M'}$.  (ii) is a consequence of Lemma
\ref{lem:safered_preserves_safety}, Proposition
\ref{prop:Nher_incrbound_and_incrjustified} and
\ref{prop:safe_imp_incrbound}(ii) and soundness of the game model.
\qed



Putting Theorem \ref{thm:safeincrejust}(i) and Lemma
\ref{lem:incrjustified_pointers_uniqu_recover} together gives:
\begin{proposition}
  \label{prop:safe_ptr_recoverable} In the game semantics of safe
  $\lambda$-terms, pointers emanating from P-moves are unnecessary
  {\it i.e.}~they are uniquely recoverable from the underlying sequences of
  moves and from O-moves' pointers.
\end{proposition}

 \begin{example} If justification pointers are omitted then the denotations of the
   two Kierstead terms from Example~\ref{ex:kierstead} are not distinguishable.
   In the safe lambda calculus this ambiguity disappears
   since $M_1$ is safe whereas $M_2$ is not.
 \end{example}

In fact, as the last example highlights, pointers are superfluous at
order $3$ for safe terms whether from P-moves or O-moves. This is
because for question moves in the first two levels of an arena
(initial moves being at level $0$), the associated pointers are
uniquely recoverable thanks to the visibility condition. At the
third level, the question moves are all P-moves therefore their
associated pointers are uniquely recoverable by P-incremental
justification. This is not true anymore at order $4$: Take the safe
term $\psi:(((o^4,o^3),o^2),o^1) \sentail \psi (\lambda \varphi .
\varphi a) : o^0$ for some constant $a:o$, where $\varphi:(o,o)$.
Its strategy denotation contains plays whose underlying sequence of
moves is $q_0 \, q_1 \, q_2 \, q_3 \, q_2 \, q_3 \, q_4$. Since
$q_4$ is an O-move, it is not constrained by P-incremental
justification and thus it can point to any of the two occurrences of
$q_3$.\footnote{More generally, a P-incrementally justified strategy
can contain plays that are not ``O-incrementally justified'' since
it must take into account any possible strategy incarnating its
context, including those that are not P-incrementally justified. For
instance in the given example, there is one version of the play that
is not O-incrementally justified (the one where $q_4$ points to the
first occurrence of $q_3$). This play is involved in the strategy
composition $\sem{ \stentail M_2 : (((o,o),o),o)} ; \sem{
\psi:(((o,o),o),o) \stentail \psi (\lambda \varphi . \varphi a):o}$
where $M_2$ denotes the unsafe Kierstead term.}


\subsection*{Towards a fully abstract game model}\hfill

The standard game models which have been shown to be fully abstract
for PCF \cite{abramsky94full,hylandong_pcf} are of course also fully
abstract for the restricted language safe PCF. One may ask, however,
whether there exists a fully abstract model with respect to safe
context only.

Such model may be obtain by considering P-incrementally justified strategies
- which have been shown to compose in \cite{Blumphd}. Its is reasonable to think that
 O-moves also needs to be constrained by the symmetrical O-incremental justification, which corresponds to the requirement that contexts are safe. This line of work is still in progress.


\subsection*{Safe PCF and safe Idealised Algol}

\pcf\ is the simply-typed lambda calculus augmented with basic
arithmetic operators, if-then-else branching and a family of
recursion combinator $Y_A : ((A,A),A)$ for any type $A$.  We define
\emph{safe} \pcf\ to be \pcf\ where the application and abstraction
rules are constrained in the same way as the safe lambda calculus.
This language inherits the good properties of the safe lambda
calculus: No variable capture occurs when performing substitution
and safety is preserved by the reduction rules of the small-step
semantics of \pcf.

\subsubsection{Correspondence}

The computation tree of a \pcf\ term is defined as the least
upper-bound of the chain of computation trees of its \emph{syntactic
approximants} \cite{abramsky:game-semantics-tutorial}.  It is
obtained by infinitely expanding the $Y$ combinator, for instance
$\tau(Y (\lambda f x. f x))$ is the tree representation of the
$\eta$-long form of the infinite term $(\lambda f x. f x)
 ((\lambda f x. f x) ((\lambda f x. f x) ( \ldots$

It is straightforward to define the traversal rules modeling the
arithmetic constants of \pcf. Just as in the safe lambda calculus we
had to remove @-nodes in order to reveal the game-semantic
correspondence, in safe \pcf\ it is necessary to filter out the
constant nodes from the traversals. The Correspondence Theorem for
\pcf\ says that the interaction game semantics is isomorphic to the
set of traversals disposed of these superfluous nodes. This can
easily be shown for term approximants. It is then lifted to full
\pcf\ using the continuity of the function $\travset(\_)^{\filter
\theroot}$ from the set of computation trees (ordered by the
approximation ordering) to the set of sets of justified sequences of
nodes (ordered by subset inclusion). Finally computation trees of
safe \pcf\ terms are incrementally-bound thus we have
%Computation trees of safe \pcf\ terms are incrementally-bound.
%Moreover since \pcf\ constant are of order $1$ at most, the constant
%traversal rules are all \emph{well-behaved} (Lemma
%\ref{lem:sigma_order1_are_wellbehaved}) hence Lemma
%\ref{lem:betanf_wellbehavedconst_trav_pview_red} (from the Appendix)
%still holds and the game-semantic analysis of safety remains valid
%for \pcf. Hence we have:
\begin{theorem}
\label{thm:safepcfpincr} Safe PCF terms have P-incrementally
justified denotations. \qed
\end{theorem}


Similarly, we can define safe \ialgol\ to be safe \pcf\ augmented
with the imperative features of Idealized Algol (\ialgol\ for short)
\cite{Reynolds81}.  Adapting the game-semantic correspondence and
safety characterization to \ialgol\ seems feasible although the
presence of the base type \iavar, whose game arena $\iacom^{\nat}
\times \iaexp$ has infinitely many initial moves, causes a mismatch
between the simple tree representation of the term and its game
arena. It may be possible to overcome this problem by replacing the
notion of computation tree by a ``computation directed acyclic
graph''.

The possibility of representing plays \emph{without some or all of
  their pointers} under the safety assumption suggests potential
applications in algorithmic game semantics. Ghica and McCusker
\cite{ghicamccusker00} were the first to observe that pointers are
unnecessary for representing plays in the game semantics of the
second-order finitary fragment of Idealized Algol ($\ialgol_2$ for
short). Consequently observational equivalence for this fragment can
be reduced to the problem of equivalence of regular expressions.  At
order $3$, although pointers are necessary, deciding observational
equivalence of $\ialgol_3$ is EXPTIME-complete
\cite{DBLP:journals/apal/Ong04,DBLP:conf/fossacs/MurawskiW05}.
Restricting the problem to the safe fragment of $\ialgol_3$ may lead
to a lower complexity.

% (note that it is unlikely to obtain the complexity PSPACE because the
% set of complete plays of the safe term $\lambda f^{(o,o),o} . f
% (\lambda x^o . x)$ is not regular \cite{DBLP:journals/apal/Ong04}).

% Murawski showed the undecidability of program equivalence in
% $\ialgol_i$ for $i\geq4$ by encoding Turing machine computations
% into a finitary $IA_4$ term \cite{murawski03program}. The term
% constructed being not safe, the proof cannot be transposed to the
% safe fragments. Hence the question remains of whether observational
% equivalence is decidable for the \emph{safe} fragments of these
% language.

%In \cite{Ong02}, one of us showed that observational equivalence for
% finitary second-order \ialgol\ with recursion ($\ialgol_2 + Y_1$) is
% undecidable. The proof consists in reducing the Queue-Halting
% problem to the observational equivalence of two $\ialgol_2 + Y_1$
% terms. The same reduction is still valid in the safe fragment of
% $\ialgol_2 + Y_1$.  Consequently, observational equivalence of safe
% $\ialgol_2 + Y_1$ is also undecidable.

    \notetoself{This chapter is taken from my transfer report. I need to
rework it to integrate it correctly within the present thesis.}


Safety has been defined as a syntactical constraint. Since Game
Semantics is by essence syntax-independent, it seems difficult at
first sight to give a game-semantic characterization of a syntactic restriction such as the Safety Condition.
In fact, the Correspondence Theorem makes such analysis possible since it allows us to regard the plays of a strategy
as sequences of nodes of some AST of the term.


The main theorem of this chapter (theorem
\ref{thm:safe_ptr_recoverable}) states that pointers in a play of
the strategy denotation of a safe term can be uniquely recovered
from O-questions' pointers and from the underlying sequence of
moves. The proof is in several steps. We start by introducing the
notion of \emph{P-incrementally-justified strategies} and prove that
for plays of such strategies, pointers emanating from P-moves can be
reconstructed uniquely from the underlying sequences of moves and
from O-moves' pointers. We then introduce the notion of
\emph{incrementally-bound computation trees} and prove that
incremental-binding coincides with P-incremental-justification
(proposition \ref{prop:Nher_incrbound_iff_incrjustified}).


Finally, we show that safe simply-typed terms in $\beta$-normal form
have incrementally\--bound computation trees, consequently their
game denotation is P-incrementally-justified.


The first section of this chapter is concerned only with the safe $\lambda$-calculus without interpreted constants. In the next
section we extend the result by taking into account the interpreted
constants of \pcf\ and \ialgol. We define the language safe \ialgol\
(resp. safe \pcf) to be the fragment of \ialgol\ (resp. \pcf) where
the application and abstraction rules are constrained the same way
as in the safe $\lambda$-calculus. We show that safe \pcf\ terms are
denoted by P-incrementally-justified strategies and we give the key
elements for a possible extension of the result to Safe Idealized
Algol.

\section{Preliminaries}

In this section, we assume that we work in a general setting of a
language extending the simply-typed lambda calculus with new
constants and respecting the following prerequisites:
\begin{itemize}
\item A fully-abstract game-semantic model of the language is
defined;
\item A notion of safety is defined for the language such that the
restriction of the language to the safe pure simply-typed
fragment coincides with the definition of the Safe Lambda
Calculus and such that for any typable term $\Gamma \vdash M :
T$ we have $\forall z \in \Gamma . \ord{z} \geq \ord{T}$ ;
\item The small-step reduction semantics of the language preserves safety;
%\item Substitution preserves safety.
\item New traversal rules are defined to take into account the constants of the language.
\item Constant traversal rules are well-behaved (see Def.\
\ref{def:wellbehaved_traversal});
\item Constant traversal rules correctly model the behaviour of the constants in such a way
that the game-semantic correspondence (Theorem
\ref{thm:correspondence}) still holds.
\end{itemize}

The simply-typed lambda calculus is of course such a language, but
we will show that \pcf\ also lends itself into this setting.

For the rest of this section we fix a term $\Gamma \vdash M : T$
from this generic language. We will explicitly specify when a result
holds only in the pure (\ie no constants) simply-typed calculus
fragment of the language.

\subsection{Incremental binding}

In a computation tree, a binder node always occurs in the path from
the bound node to the root. We now introduce a class of computation
trees in which binder nodes can be uniquely recovered from the order
of the nodes. We call path any sequence of nodes such that for any
two consecutive nodes $a \cdot b$ in the sequence, $a$ is the parent
of $b$. We write $[n_1,n_2]$ to denote the path going from node
$n_1$ to node $n_2$ equipped with the justification pointers induced
by the enabling relation $\vdash$ (each node of the tree has a
unique enabler in the path to the root thus for each occurrence in
$[n_1,n_2]$ there is at most one occurrence of its enabler in
$[n_1,n_2]$). We write $]n_1,n_2]$ for the sub-sequence of
$[n_1,n_2]$ obtained by removing $n_1$ as welle as all the
associated pointers.

We recall that $\theroot$ denotes the root of the computation tree
$\tau(M)$ and $N^{\theroot\vdash}$ denotes the subset of $N$
consisting of nodes that are hereditarily enabled by $\theroot$.



\begin{definition}[Incrementally-bound computation tree]
Let $A$ be a subset of nodes of the computation tree. A variable
node $x$ of a computation tree is said to be
\defname{$A$-incrementally-bound} if its enabler is the first
$\lambda$-node from $A$ in the path to the root that has order
strictly greater than $\ord{x}$. Formally:
\begin{align*}
x \mbox{ is $A$-incrementally-bound} \  \iff \  \left\{
                                                  \begin{array}{ll}
                                                    x \hbox{ is enabled by } b \in [\theroot,x]\inter A \ ; \\
                                                    \ord{b} > \ord{x} \;\\
                                                    \forall \lambda\mbox{-node } n' \in ]n,x]\inter A  . \ord{n'} \leq \ord{x} \ .
                                                  \end{array}
                                                \right.
\end{align*}

This definition can be split into two cases:
\begin{enumerate}
\item $x$ is \emph{bound} by the first $\lambda$-node from $A$ occurring in the path to the root that has
order strictly greater than $\ord{x}$.
\item or $x$ is a \emph{free variable} and all the $\lambda$-nodes from from $A$ occurring in the path to the root except the root have order
 smaller or equal to $\ord{x}$.
\end{enumerate}

A computation tree is said to be \defname{$A$-incrementally-bound},
also abbreviated $A$-i.b., if all the variable nodes from $A$ are
$A$-incrementally-bound.

We say that a node (resp.\ a tree) is
\defname{incrementally-bound} if it is
\defname{$N$-incrementally-bound} where $N$ is the entire set of nodes of the computation tree.
\end{definition}

Clearly for any two sets of nodes $A$ and $B$ verifying $A\subseteq
B$ we have that $B$-incremental-binding implies
$A$-incremental-binding.


\smallskip

Let $\closure{M}$ denote the function that converts $M$ into the
closed term obtained from $M$ by abstracting all its free variables
(in order of appearance in the term). From the previous definition,
if $\tau(M)$ is $A$-i.b.\ then so is $\tau(\closure{M})$.

\smallskip

A node of the computation tree is said to be \defname{reachable} if
there is some traversal of the computation tree that visits it.


\begin{lemma}[Safe terms have incrementally-bound computation trees]
\label{lem:incrbound_iff_etanf_safe} Suppose that  $\Gamma \vdash M
:T$ is a simply-typed term.
\begin{itemize}
\item[(i)] If $M$ is a safe term then $\tau(M)$ is incrementally-bound ;
\item[(ii)] conversely, if $M$ is \emph{closed} and $\tau(M)$ is i.b.\ then the $\eta$-long normal form of $M$ is safe.
\end{itemize}
\end{lemma}
\begin{proof}
(i) Suppose that $M$ is safe. The safety property is preserved after
taking the $\eta$-long normal form, therefore $\tau(M)$ is the tree
representation of a safe term.

In the safe $\lambda$-calculus, the variables in the context with
the the lowest order must be all abstracted at once when using the
abstraction rule. Since the computation tree merges consecutive
abstractions into a single node, any variable $x$ occurring free in
the subtree rooted at a $\lambda$-node $\lambda \overline{\xi}$
different from the root must have order greater or equal to
$\ord{\lambda \overline{\xi}}$. Reciprocally, if a lambda node
$\lambda \overline{\xi}$ binds a variable node $x$ then
$\ord{\lambda \overline{\xi}} = 1+\max_{z\in\overline{\xi}} \ord{z}
> \ord{x}$.

Let $x$ be a bound variable node. Its binder occurs in the path from
$x$ to the root, therefore, according to the previous observation,
$x$ must be bound by the first $\lambda$-node occurring in $[r,x]$
with order strictly greater than $\ord{x}$. Let $x$ be a free
variable node then $x$ is not bound by any of the $\lambda$-nodes
occurring in $[\theroot,x]$. Once again, by the previous
observation, all these $\lambda$-nodes except $\theroot$ have order
smaller than $\ord{x}$. Hence $\tau$ is incrementally-bound.

(ii) Let $M$ be a closed term such that $\tau(M)$ is
incrementally-bound. We assume that $M$ is already in $\eta$-normal
form. We prove that $M$ is safe by induction on its structure. The
base case $M = \lambda \overline{\xi} . \alpha$ for some variable or
constant $\alpha$ is trivial. \emph{Step case:} If $M = \lambda
\overline{\xi} . N_1 \ldots N_p$. Let $i$ range over $1..p$. $N_i$
can be written $\lambda \overline{\eta_i} . N'_i$ where $N'_i$ is
not an abstraction. By the induction hypothesis, $\lambda
\overline{\xi} . N_i = \lambda \overline{\xi} \overline{\eta_i} .
N'_i$ is safe. Hence $\vdash \lambda \overline{\xi}
\overline{\eta_i} . N'_i$ is a valid judgment of safe
$\lambda$-calculus. But this judgment can only be derived using the
\rulenamet{Abs} rule on the term $N'_i$. Hence $N'_i$ is necessarily
safe. Let $z$ be a variable occurring free in $N'_i$. Since $M$ is
closed, $z$ is either bound by $\lambda \overline{\eta_1}$ or
$\lambda \overline{\xi}$. In the latter case, since $\tau(M)$ is
i.b., $\ord{z}$ is smaller than $\ord{\lambda
\overline{\eta_1}}=\ord{N_i}$ thus in both case we are allowed to
abstract the variables $\overline{\eta_1}$ using the rule
\rulenamet{Abs}. This shows that $N_i$ is safe.

Each of the $N_i$s is safe and $N_1 \ldots N_p$ is of type $o$
therefore by the rule \rulenamet{App} rule we have $\overline{\xi}
\vdash N_1 \ldots N_p$. Finally, \rulenamet{Abs} gives us the
judgement $\vdash M = \lambda \overline{\xi} . N_1 \ldots N_p$.
\end{proof}

Note that the hypothesis that $M$ is closed in (ii) is necessary.
For instance, the two terms $\lambda x y .x$ and $\lambda y . x$,
where $x,y:o$, have (isomorphic) incrementally-bound computation
trees. However $\lambda x y .x$ is safe whereas $\lambda y . x$ is
not.

\begin{corollary}
\label{cor:betared_preserve_incrbound} Suppose $M$ is a closed term
in $\eta$-long normal form. If $\tau(M)$ is incrementally-bound and
$M \betared N$ then $\tau(N)$ is incrementally-bound.
\end{corollary}
\proof Suppose that $\tau(M)$ is i.b. Then by Lemma
\ref{lem:incrbound_iff_etanf_safe}(ii), $M$ is safe and since safety
is preserved by $\beta$-reduction, so is $N$. Thus by Lemma
\ref{lem:incrbound_iff_etanf_safe}(i), $\tau(N)$ is
incrementally-bound. \qed
\smallskip

Note that this corollary  cannot be generalized to
$A$-incremental-binding for any set of node $A$. Take for instance
the eta-normal term $M = \lambda u^{o} v^{((o,o),o)} . (\lambda x^o
. v (\lambda z^o . x)) u$ which beta-reduces to $N = \lambda u v . v
(\lambda z . u)$. The computation trees are:
$$\pssetcomptree
\tau(M) = \pstree{\TR{\underline{\lambda u v}}}{\pstree{\TR{@}}{\pstree{\TR{\lambda
x}}{\pstree{\TR{\underline{v}}}{\pstree{\TR{\underline{\lambda
z}}}{\TR{x}}}}\pstree{\TR{\lambda }}{\TR{\underline{u}}}}} \hspace{2cm}
\tau(N) = \pstree{\TR{\underline{\lambda u v}}}{\pstree{\TR{\underline{v}}}{\pstree{\TR{\underline{\lambda z}}}{\TR{\underline{u}}}}}
$$
Take $A$ to be the set of nodes that are hereditarily justified by
the root (the nodes underlined in the above figure). Then $\tau(M)$
is $A$-incrementally-bound but $\tau(N)$ is not.


\subsection{P-incremental-justified strategies}
\begin{definition}[P-incremental-justification]
A strategy $\sigma$ on a game $A$ is
\emph{P-incrementally\-justified} if and only if for any sequence of
moves $s q \in P_A$ we have:
\begin{eqnarray*}
s q \in \sigma \wedge q \mbox{ is a P-question } &\implies&
\parbox[t]{9cm}{$q$  points to the last O-move in $\pview{s}$
with order strictly greater than $\ord{q}$.}
\end{eqnarray*}
\end{definition}




\begin{lemma}
\label{lem:incrjustified_pointers_uniqu_recover} Pointers emanating
from P-moves are superfluous for P-incrementally-justified
strategies.
\end{lemma}
\begin{proof}
Suppose $\sigma$ is a P-incrementally-justified strategy. We prove
that pointers attached to P-moves in a play $s\in \sigma$ are
uniquely recoverable by induction on the length of $s$. \noindent
\emph{Base case}: if $|s| \leq 1$ then there is no pointer to
recover. \noindent \emph{Step case}: suppose $s m \in \sigma$. If
$m$ is an answer move then by the well-bracketing condition $m$
points to the last unanswered question in $s$. If $m$ is a
P-question then by  P-incremental-justification of $\sigma$, $m$
points to the last O-move in $\pview{s}$ with order strictly greater
than $\ord{q}$. Since we have access to O-moves' pointers, we can
compute the P-view $\pview{s}$. Hence $m$'s pointer is uniquely
recoverable.
\end{proof}

%\begin{example}
%The denotation of the evaluation map $ev$ is
%P-incrementally-justified since it is the uncurrying of the identity
% map on the game A=>B.
%\end{example}



\begin{proposition}[Incremental-binding and P-incremental-justification]
\hfill

 \label{prop:Nher_incrbound_iff_incrjustified}

\begin{enumerate}[(i)]
\item Suppose $M$ is $\beta$-normal. Then if all the \emph{reachable} input-variable nodes of the computation tree
$\tau(\Gamma \vdash M : T)$ are
$N^{\theroot\vdash}$-incrementally-bound then $\sem{\Gamma
\vdash M : T}$ is P-incrementally-justified.

\item If $\sem{\Gamma \vdash M : T}$ is
P-incrementally-justified then all the \emph{reachable}
input-variable nodes of the computation tree $\tau(\Gamma \vdash
M : T)$ are $N^{\theroot\vdash}$-incrementally-bound.
\end{enumerate}
\end{proposition}

\begin{proof}
\noindent (i) Suppose that $\tau(M)$ is
$N^{\theroot\vdash}$-incrementally-bound, then so is
$\tau(\etalnf{\closure{M}})$. Thus by Corollary
\ref{cor:betared_preserve_incrbound} $\etalnf{\closure{M}}$ is safe
and since safety is preserved by $\beta$-reduction, so is its
beta-normal form. Thus by Lemma
\ref{lem:incrbound_iff_etanf_safe}(i),
$\tau(\betanf{\etalnf{\closure{M}}})$ is incrementally-bound. Hence
we can assume without loss of generality that $M$ is a closed term
in beta-normal form and prove that $\sem{M}$ is
P-incrementally-justified (This will imply that
$\sem{\betanf{\etalnf{\closure{M}}}}$ is P-i.j.\ since the two game
denotations are isomorphic).

Take a play $s \in \sem{\Gamma \vdash M : T}$ ending with a question
P-move $q$. By the Correspondence Theorem \ref{thm:correspondence},
there is a traversal $t$ of $\tau(M)$ starting with an occurrence
$r$ of the root $\theroot$ such that $\psi_M (t\filter r) = s$. We
assume $t$ to be the shortest such traversal, thus the last
occurrence of $t$ - let us name it $n$ - is hereditarily justified
by $r$ and is by definition an occurrence of a reachable node.
Moreover since $\psi_M$ maps $n$ to $q$, $n$ is necessarily an
occurrence of a variable node $x$. There are two cases:
\begin{itemize}
\item Suppose $x$ is bound variable. Let $m$ denote its justifier
in $t$ (which is an occurrence of $x$'s binder in $\tau(M)$). By
assumption $\tau(M)$ is $N^{\theroot\vdash}$-incrementally-bound
therefore since $n$ belongs to $N^{\theroot\vdash}$, $m$ must be
the last $\lambda$-node in $[\theroot,n]\ \inter
N^{\theroot\vdash}$ of order strictly greater than $\ord{n}$.

By the Path--P-view correspondence (Prop.\
\ref{prop:pviewtrav_is_path}) we have $[\theroot,n]\ \inter
N^{\theroot\vdash} = \pview{t} \filter r$. This is in turn is
equal to $\pview{?(t \filter r)}$ (by Lemma
\ref{lem:betanf_wellbehavedconst_trav_pview_red}, since $M$ is
in $\beta$-normal form).


By property \ref{proper:psi_properties} (iv), the P-view of
$?(s)$ and the P-view of $?(t \filter r)$ are computed similarly
and have the same pointers, therefore node $n$ and move $q$ both
point to the same position in the justified sequence
$\pview{?(t\filter r)}$ and $\pview{?(s)}$ respectively.
Moreover since $\psi_M$ maps nodes of a given order to moves of
the same order (property \ref{proper:psi_properties}) this means
that $q$ points to the last O-move in $\pview{?(s)}$ with order
$>\ord{q}$.

Finally Lemma \ref{lem:views_and_questionmarkfilter} gives us
$?(\pview{s}) = \pview{?(s)}$, and since $s$'s last move is a
question, $\pview{s}$ contains only question moves and therefore
$\pview{?(s)} = \pview{s}$. Thus $q$ points to the last O-move
in $\pview{s}$ with order is strictly greater than $\ord{q}$.


\item  Second case: $n$ is a free input-variable $x$.
Thus $n$ is justified by $r$, the first occurrence in $t$. By
definition of $\psi$, $x = \psi(n)$ must be a move enabled by
the initial move $q_0 = \psi(\theroot)$ in the arena
$\sem{\Gamma \rightarrow A}$, therefore we have $\ord{q_0} >
\ord{x}$. Furthermore since  $x$ is
$N^{\theroot\vdash}$-incrementally-bound all the $\lambda$-nodes
in $]\theroot,n]$ have order smaller than $\ord{n}$, thus by the
Correspondence Theorem, all the O-moves in $\pview{s}$ have
order smaller than $\ord{x}$.
\end{itemize}

\noindent (ii) Suppose $\sem{M}$ is P-incrementally-justified. Let
$x$ be a reachable input-variable node of $\tau(M)$: there exists a
traversal of the form $t \cdot x$ in $\travset(M)$ such that $x$ is
hereditarily justified by the first occurrence $r$ of $\tau(M)$'s
root in $t$.

The correspondence theorem tells us that $\varphi((t \cdot x)
\filter r) = \varphi((t \filter r) \cdot x)$ belongs to $\sem{M}$.
Since $\sem{M}$ is P-incrementally-justified, $\varphi(x)$ points to
the last O-move in $\pview{\varphi(t \filter r)}$ with order
strictly greater than $\ord{\varphi(x)}$. Consequently $x$ points to
the last $\lambda$-node in $\pview{t \filter r}$ with order strictly
greater than $\ord{x}$.

But by Lemma \ref{lem:pviewproj_wrt_theroot}, $\pview{t \filter r}$
contains $\pview{t} \filter r$ as a subsequence. Thus since by
P-visibility $m$ occurs in this subsequence, we have that $m$ is
also the last $\lambda$-node in $\pview{t} \filter r$ with order
strictly greater than $\ord{x}$. By the path-P-view correspondence
(Prop.\ \ref{prop:pviewtrav_is_path}) this can in turn be restated
as: $m$ is the last $\lambda$-node in $[\theroot,x[\  \inter\
N^{\theroot \vdash}$ with order strictly greater than $\ord{x}$.
Hence $\tau(M)$ is $N^{\vdash \theroot}$-incrementally-bound.
\end{proof}

\section{Safe $\lambda$-Calculus}

We now consider the special case of the Safe $\lambda$-Calculus
without interpreted constants. We show that pointers in the game
denotation of safe terms can be uniquely recovered. The example of
section \ref{subsec:pointer_necessary} gives a good intuition: in
order to distinguish the terms $M_1 = \lambda f . f (\lambda x . f
(\lambda y .y ))$ and $M_2 = \lambda f . f (\lambda x . f (\lambda y
.x ))$ it is necessary to keep pointers in the strategy plays. In
the Safe $\lambda$-Calculus, however, the ambiguity disappears since
$M_1$ is safe whereas $M_2$ is not (in the subterm $f (\lambda y .
x)$, the free variable $x$ has the same order as $y$ but it is not
abstracted together with $y$).



\begin{corollary}[of Proposition \ref{prop:Nher_incrbound_and_incrjustified}]
\label{cor:Nher_incrbound_iff_incrjustified}
  Suppose $\Gamma \vdash M : T$ is a pure (\ie with no interpreted constants) simply-typed term
  in $\beta$-normal form. Then $\sem{M}$ is P-incrementally-justified if and only if $\tau(M)$ is incrementally-bound.
\end{corollary}
\proof We first observe that all the variable nodes are
input-variable nodes. Indeed, let $x$ be a variable node of
$\tau(M)$. Since $M$ is $\beta$-normal, by lemma
\ref{lem:betanorm_enabling}, $x$ is either hereditarily enabled by
the root or by a constant in $N_\Sigma$. But the pure simply-typed
$\lambda$-calculus does not have constants thus $N_\Sigma =
\emptyset$ and $x$ is hereditarily enabled by the root, \ie it is an
input-variable node. Consequently, incremental-binding coincides
with $N^{\vdash \theroot}$-incremental-binding.

Furthermore, since all the input-variables are reachable, every node
of the computation tree can be reached by the traversal consisting
of the path from the root to that node, the \rulenamet{InputVar}
permitting us to visit the children of the input-variable nodes
occurring in the path.\qed
\smallskip

\parpic[r]{
    \pssetcomptree
     \tree[levelsep=4ex]{$\lambda x^3$}{\tree{$f^2$}{ \tree{$\lambda y^1$}{ \TR{$x^0$} }}}
} \noindent \emph{Examples:} Consider the $\beta$-normal term
$\lambda x . f (\lambda y .x)$ where $x,y:o$ and $f:(o,o),o$. The
figure on the right represents the computation tree with the order
of each node in the exponent part. Since node $x$ of order $0$ is
not bound by the order 1 node $\lambda y$, $\tau(M)$ is not
incrementally-bound and by proposition
\ref{prop:Nher_incrbound_and_incrjustified} $\sem{\lambda x . f
(\lambda y .x)}$ is not P-incrementally-justified. Similarly we can
check that $\sem{f (\lambda y .x)}$ is not P-incrementally-justified
whereas $\sem{\lambda y. x}$ is. Also, for any higher-order variable
$x:A$ the computation tree $\tau(x)$ is incrementally-bound
therefore the projection strategies $\pi_i$ are
P-incrementally-justified. From these examples we observe that
application does not preserve P-incremental-justification ($\sem{f}$
and $\sem{\lambda y. x}$ are P-incrementally-justified whereas
$\sem{f (\lambda y .x)}$ is not).
\smallskip

These examples suggest that P-incremental-justification is not a
compositional property. In Chapter \ref{chap:pincrjust} we will
identify a sufficient condition guaranteeing that the composition of
two P-incrementally-justified strategies gives a
P-incrementally-justified strategy. \smallskip


Putting Corollary \ref{cor:Nher_incrbound_iff_incrjustified} and
Lemma \ref{lem:incrbound_iff_etanf_safe} together gives us a
game-semantic characterization of safe terms:
\begin{corollary}[P-incrementally-justified strategies characterize safe closed $\eta\beta$-normal terms]
Let $\Gamma \vdash M : T$ be a simply-typed term (without
interpreted constants). Then:
$$ \sem{\Gamma \vdash M : T} \mbox{ is P-incrementally-justified if and only if $\etabetalnf{M}$ is safe,} $$
where $\etabetalnf{M}$ denotes the $\eta$-long normal form of the
$\beta$-normal form of $M$.
\end{corollary}



\begin{theorem}[P's pointers are superfluous for safe terms]
\label{thm:safe_ptr_recoverable} Pointers emanating from P-moves in the game semantics of
safe terms are uniquely recoverable.
\end{theorem}
\begin{proof}
Let $M$ be a safe simply-typed term. Then the $\beta$-normal form of
$M$ is also safe, thus by lemma \ref{lem:incrbound_iff_etanf_safe}
(i), $\tau(\betanf{M})$ is incrementally-bound and by proposition
\ref{prop:Nher_incrbound_and_incrjustified}, $\sem{\Gamma \vdash
\betanf{M} :T}$ is a P-incrementally-justified strategy. By lemma
\ref{lem:incrjustified_pointers_uniqu_recover}, P's pointers in
$\sem{\Gamma \vdash \betanf{M} :T}$ are uniquely recoverable.
Finally, the soundness of the game model gives $\sem{\Gamma \vdash
M:T} = \sem{\Gamma \vdash \betanf{M} : T}$.
\end{proof}


\section{Safe PCF and Safe Idealized Algol}

Safe Idealized Algol, or safe \ialgol\ for short, is Idealized Algol
where the application and abstraction rules are restricted the same
way as in the safe $\lambda$-calculus (see rules of section
\ref{sec:safe_nonhomog}).

The properties of the safe $\lambda$-calculus can be transposed
straightforwardly to safe \ialgol. In particular, it can be shown
that safety is preserved by $\beta$-reduction and that no variable
capture occurs when performing substitution on a safe term.

A natural question to ask is whether we can extend the result about
game semantics of safe $\lambda$-terms to safe \ialgol-terms. In
this section we lay out the key elements permitting to prove that
the pointers in the game semantics of safe IA terms can be recovered
uniquely.

Such result has potential applications in algorithmic game semantics.
For instance, by following the framework of \cite{ghicamccusker00},
it may be possible to give a characterisation of the game semantics
of some higher-order fragments of safe \ialgol\ using extended
regular expressions. Subsequently, this would lead to the
decidability of program equivalence for the considered fragment.


\subsection{Formation rules of Safe \ialgol}
We call safe \ialgol\ term any term that is typable within the
following system of formation rules:
$$ \rulename{var} \   \rulef{}{x : A\vdash x : A}
%\qquad  \rulename{const} \   \rulef{}{\vdash f : A} \quad f \in \Sigma
\qquad  \rulename{wk} \   \rulef{\Gamma \vdash M : A}{\Delta \vdash
M : A} \quad  \Gamma \subset \Delta$$

$$ \rulename{app} \  \rulef{\Gamma \vdash M : (A,\ldots,A_l,B)
                                        \qquad \Gamma \vdash N_1 : A_1
                                        \quad \ldots \quad \Gamma \vdash N_l : A_l  }
                                   {\Gamma  \vdash M N_1 \ldots N_l : B}
                                    \quad
\mbox{\fbox{$\forall y \in \Gamma : \ord{y} \geq \ord{B}$}}$$

$$ \rulename{abs} \   \rulef{\Gamma \union \overline{x} : \overline{A} \vdash M : B}
                                   {\Gamma  \vdash \lambda \overline{x} : \overline{A} . M : (\overline{A},B)} \quad
\mbox{\fbox{$\forall y \in \Gamma : \ord{y} \geq \ord{\overline{A},B}$}}$$

$$ \rulename{num} \rulef{}{\Gamma \vdash n :\texttt{exp}}
\qquad \rulename{succ} \rulef{\Gamma \vdash M:\texttt{exp} }{\Gamma
\vdash \texttt{succ}\ M:\texttt{exp}} \qquad \rulename{pred}
\rulef{\Gamma \vdash M:\texttt{exp} }{\Gamma \vdash \texttt{pred}\
M:\texttt{exp}}$$

$$
\rulename{cond} \rulef{\Gamma \vdash M : \texttt{exp} \qquad \Gamma
\vdash N_1 : \texttt{exp} \qquad \Gamma \vdash N_2 : \texttt{exp}
}{\Gamma \vdash \texttt{cond}\ M\ N_1\ N_2} \qquad  \rulename{rec}
\rulef{\Gamma \vdash M : A\rightarrow A }{ \Gamma \vdash Y_A M :
A}$$

$$ \rulename{seq} \rulef{\Gamma \vdash M : \texttt{com} \quad \Gamma \vdash N :A}
    {\Gamma \vdash \texttt{seq}_A \ M\ N\ : A} \quad A \in \{ \texttt{com}, \texttt{exp}\}$$

$$ \rulename{assign} \rulef{\Gamma \vdash M : \texttt{var} \quad \Gamma \vdash N : \texttt{exp}}
    {\Gamma \vdash \texttt{assign}\ M\ N\ : \texttt{com}}
\qquad
 \rulename{deref} \rulef{\Gamma \vdash M : \texttt{var}}
    {\Gamma \vdash \texttt{deref}\ M\ : \texttt{exp}}$$

$$ \rulename{new} \rulef{\Gamma, x : \texttt{var} \vdash M : A}
    {\Gamma \vdash \texttt{new } x \texttt{ in } M} \quad A \in \{ \texttt{com}, \texttt{exp}\}$$

$$ \rulename{mkvar} \rulef{\Gamma \vdash M_1 : \texttt{exp} \rightarrow \texttt{com} \quad \Gamma \vdash M_2 : \texttt{exp}}
    {\Gamma \vdash \texttt{mkvar } M_1\ M_2\ : \texttt{var}}$$

\subsection{Small-step semantics of Safe \ialgol}
In the first chapter we defined the operational semantics of
\ialgol\ using a big step semantics. The operational semantics of
\ialgol\ can be defined equivalently using a small-step semantics.
The reduction rules of the small-step semantics are of the form $s,e
\rightarrow s',e'$ where $s$ and $s'$ denotes the stores and $e$ and
$e'$ denotes \ialgol\ expressions.

Let us give the rules that tell how to reduce redexes:
\begin{itemize}
\item the reduction of safe-redex (relation $\beta_s$ from definition \ref{dfn:safereduction});
\item reduction rules for \pcf\ constants:
\begin{eqnarray*}
\pcfsucc\ n &\rightarrow& n+1 \\
\pcfpred\ n+1 &\rightarrow& n \\
\pcfpred\ 0 &\rightarrow& 0 \\
\pcfcond\ 0\ N_1 N_2 &\rightarrow& N_1 \\
\pcfcond\ n+1\ N_1 N_2 &\rightarrow& N_2 \\
Y\ M &\rightarrow& M (Y M)
\end{eqnarray*}
\item reduction rules for \ialgol\ constants:
\begin{eqnarray*}
\iaseq\ \iaskip\  M &\rightarrow& M \\
s, \ianewin{x}\ M &\rightarrow& (s|x\mapsto 0), M \\
s, \iaassign\ x\ n &\rightarrow& (s|x\mapsto n), \iaskip \\
s, \iaderef\ x &\rightarrow& s, s(x) \\
\iaassign\ (\iamkvar M N)\ n &\rightarrow& M n \\
\iaderef\ (\iamkvar M N) &\rightarrow& N
\end{eqnarray*}
\end{itemize}

Redex can also be reduced when they occur as subexpressions within a
larger expression. We make use of evaluation contexts to indicate
when such reduction can happen. Evaluation contexts are given by the
following grammar:
\begin{eqnarray*}
E[-] &::=& - |\ E N\ |\ \pcfsucc\ E\ |\ \pcfpred\ E\ |\ \pcfcond\ E\ N_1\ N_2\ |\ \\
&&    \iaseq\ E\ N\ |\ \iaderef\ E\ |\ \iaassign\ E\ n\ |\ \iaassign\ M\ E \ |\ \\
&&    \iamkvar\ M\ E\ |\ \iamkvar\ E\ M\ |\ \ianewin{x}\ E  .
\end{eqnarray*}

The small-step semantics is completed with following rule:
$$ \rulef{M \rightarrow N}{E[M] \rightarrow E[N]} $$

\begin{lemma}[Reduction preserves safety]
\label{lem:ia_safety_preserved} Let $M$ be a safe IA term. If
$M \rightarrow N$ then $N$ is also a safe term.
\end{lemma}
This can be proved easily by induction on the structure of M.


\subsection{Safe \pcf\ fragment}
In this section, we show how to extend the results obtained for the
safe $\lambda$-calculus to the \pcf\ fragment of safe \ialgol.

The $Y$ combinator needs a special treatment. In order to deal with
it, we follow the idea of \cite{abramsky:game-semantics-tutorial}:
we consider the sublanguage $\pcf_1$ of \pcf\ in which the only
allowed use of the $Y$ combinator is in terms of the form $Y(
\lambda x:A .x )$ for some type $A$. We will write $\Omega_A$ to
denote the non-terminating term $Y(\lambda x:A .x)$ for a given type
$A$.

We introduce the \emph{syntactic approximants} to $Y_A M$:
\begin{eqnarray*}
Y^0_A M &=& \Gamma \vdash \Omega_A : A\\
Y^{n+1}_A M &=& M( Y^n M )
\end{eqnarray*}
For any \pcf\ term $M$ and natural number $n$, we define $M_n$ to be
the $\pcf_1$ term obtained from $M$ by replacing each subterm of the
form $Y N$ with $Y^n N_n$. We have $\sem{M} = \Union_{n\in\omega}
\sem{M_n}$ (\cite{abramsky:game-semantics-tutorial}, lemma 16).


\subsubsection{Computation tree}

We would like to define a unique computation tree for terms that use
the $Y$ combinator.

Let us first define the computation tree for $\pcf_1$ terms. We
introduce a special $\Sigma$-constant $\bot$ representing the
non-terminating computation of ground type $\Omega_o$. Given any
type $A = (A_1, \ldots, A_n, o)$, the computation tree
$\tau(\Omega_A)$ is defined to be the tree representation of
$\lambda x_1:A_1 \ldots x_n:A_n . \bot$. The computation tree of a
$\pcf_1$ term is then computed inductively in the standard way.

We now introduce a partial order on the set of computation trees.

A \emph{tree} $t$ is a labelling function $t:T\rightarrow L$ where
$T$, called the domain of $t$ and written $dom(t)$, is a non-empty
prefix-closed subset of some free monoid $X^*$ and $L$ denotes the
set of possible labels. Intuitively, $T$ represents the structure of
the tree (the set of all paths) and $t$ is the labelling function
mapping paths to labels. Trees can be ordered using the
\emph{approximation ordering} defined in \cite{KNU02}, section 1: we
write $t' \sqsubseteq t$ if the tree $t'$ is obtained from $t$ by
replacing some of its subtrees by $\bot$. Formally:
$$t' \sqsubseteq t \quad \iff dom(t') \subseteq dom(t) \wedge \forall  w \in dom(t'). (t'(w) = t(w) \vee t'(w) = \bot).$$
The set of all trees together with the approximation ordering is a
complete partial order.

We now consider a strict subset of the set of all trees: the set of
computation trees. A computation tree is a tree which represents the
$\eta$-normal form of some (potentially infinite) \pcf\ term. In
other words a tree is a computation tree if it can be written
$\tau(M)$ for some infinite \pcf\ term $M$. The set $L$ of labels is
constituted of the $\Sigma$-constants, @, the special constant
$\bot$, variables and abstractions of any sequence of variables. We
will write $(CT, \sqsubseteq)$ to denote the set of computation
trees ordered by the approximation ordering $\sqsubseteq$ defined
above. $(CT, \sqsubseteq)$ is also a complete partial order.

It is easy to check that the sequence of computation trees
$(\tau(M_n))_{n\in\omega}$ is a chain. We can therefore define the
computation tree of a \pcf\ term $M$ to be the least upper-bound of
the chain of computation trees of its approximants:
$$\tau(M) = \Union_{n\in\omega}(\tau(M_n))_{n\in\omega}.$$

In other words, we construct the computation tree by expanding
infinitely any subterm of the form $Y M$. For instance consider the
term $M = Y (\lambda f x. f x)$ where $f:(o,o)$ and $x:o$. Its
computation tree $\tau(M)$, represented below, is a tree
representation of the $\eta$-normal form of the infinite term
$(\lambda f x. f x) ((\lambda f x. f x) ((\lambda f x. f x)  (
\ldots$.
$$\tau(M) = \pssetcomptree\tree{\lambda y}{
                \tree{@}{
                        \tree{\lambda f x} { \tree{f}{\tree{\lambda}{\TR{x}} }}
                        \TR{\tau(M)}
                        \tree{\lambda}{\TR{y}}
                }
            }
$$

The remaining operators of \ialgol\ are treated as standard
constants and the corresponding computation tree is constructed from
the $\eta$-normal form of the term in the standard way. For instance
the diagram below shows the computation tree for $\pcfcond\ b\ x\ y$
(left) and $\lambda x . 5$ (right):
$$
\pssetcomptree\tree{\lambda b x y}
     {  \tree{\pcfcond}
        {   \tree{\lambda} {\TR{b}}
            \tree{\lambda} {\TR{x}}
            \tree{\lambda} {\TR{y}}
        }
    }
\hspace{2cm} \tree{\lambda x}{  \TR{5} }
$$
The node labelled $5$ has, like any other node, children
value-leaves which are not represented on the diagram above for
simplicity.

\subsubsection{Traversal}

New traversal rules accompany the additional constants of \ialgol.
There is one additional rule for natural number constants:
\begin{itemize}
\item (Nat) If $t \cdot n$ is a traversal where $n$ denotes a node labelled with some numeral constant $i\in \nat$ then
            $\Pstr{t \cdot (n){n} \cdot (in-n){i_n}}$
            is also a traversal where $i_n$ denotes the value-leaf of $m$ corresponding to the value $i\in \nat$.
\end{itemize}

\noindent The traversals rules for \pcfpred\ and \pcfsucc\ are
defined similarly. For instance, the rules for \pcfsucc\ are:
\begin{itemize}
\item (Succ) If $t \cdot \pcfsucc$ is a traversal and $\lambda$ denotes the only child node of \pcfsucc\ then
$\Pstr{t \cdot (succ){\pcfsucc} \cdot (l-succ,35:1){\lambda}}$ is also a traversal.

\item (Succ') If
$\Pstr{ t_1 \cdot (succ){\pcfsucc} \cdot (l-succ,35:1){\lambda} \cdot t_2
\cdot (lv-l){i_{\lambda}}} $ is a traversal for some
$i \in \nat$ then $\Pstr{t_1 \cdot (succ){\pcfsucc} \cdot
(l-succ,35:1){\lambda} \cdot t_2 \cdot (lv-l){i_{\lambda}} \cdot
(succv-succ,25){(i+1)_{\pcfsucc}}}$ is also a traversal.
\end{itemize}

\noindent In the computation tree, nodes labelled with \pcfcond\
have three children nodes numbered from $1$ to $3$ corresponding to
the three parameters of the operator \pcfcond. The traversal rules
are:
\begin{itemize}
\item (Cond-If) If $t_1 \cdot \pcfcond$ is a traversal and $\lambda$ denotes the first child of \pcfcond\ then
$\Pstr{ t_1 \cdot (cond){{\pcfcond}} \cdot (l-cond,30:1){\lambda}}$
 is also a traversal.

\item (Cond-ThenElse) If
$\Pstr{t_1 \cdot (cond){\pcfcond} \cdot (l-cond,35:1){\lambda} \cdot t_2
\cdot (lv-l){i_{\lambda}}} $
then $\Pstr{t_1 \cdot
(cond){\pcfcond} \cdot (l-cond,35:1){\lambda} \cdot t_2 \cdot
(lv-l){i_{\lambda}} \cdot (condthenelse-cond,35:{2+[i>0]}){\lambda} }
$
is also a traversal.



\item (Cond') If
$\Pstr{t_1 \cdot (cond){\pcfcond} \cdot t_2 \cdot (l-cond,35:k){\lambda}
\cdot t_3 \cdot (lv-l){i_{\lambda}}}$
 for $k=2$ or $k=3$ then  $\Pstr{ t_1 \cdot
(cond){\pcfcond} \cdot t_2 \cdot (l-cond,35:k){\lambda} \cdot t_3
\cdot (lv-l){i_{\lambda}} \cdot (condv-cond,25){i_{\pcfcond}}}$
 is also a traversal.
\end{itemize}
It is easy to verify that these traversal rules are all
well-behaved. This completes the definition of traversal for the
\pcf\ subset of \ialgol.

\subsubsection{Interaction semantics}
We recall that the interaction semantics defined in section
\ref{sec:interaction_semantics} takes into account the constants
of the language. For any higher-order constant $f : (A_1,\ldots,A_p,B) \in \Sigma$, definition \ref{dfn:canonical_revealed_semantics} gives the  revealed strategy of a term of the form $\lambda \overline{\xi}. f N_1 \ldots
N_p$ as follows:
$$ \revsem{\lambda \overline{\xi}. f N_1 \ldots N_p} = \langle \revsem{N_1}, \ldots, \revsem{N_p} \rangle \fatsemi^{0..p-1} \sem{f}.$$
where $\sem{f}$ is the standard strategy denotation of the constant $f$.


\subsubsection{Removing $\Sigma$-nodes from the traversals}


\notetoself{Need to rework the following lemma}

\begin{lemma}[Projection lemma]
\label{lem:SIGMACONST:varphi_projection} Let $\Gamma \vdash M :T$ be
a term and $r$ be the root of $\tau(M)$. For any traversal $t$ of
the computation tree we have $ \varphi(\travset(M)^*) \filter
\sem{\Gamma \rightarrow T} = \varphi(\travset(M)^{\filter r}) $.
 Consequently,
$$\varphi(t^*) \filter \sem{\Gamma \rightarrow T} = \varphi(t\filter r).$$
\end{lemma}
\begin{proof}
    From the definition of $\varphi$, the nodes of the computation tree that $\varphi$ maps
    to moves in the arena $\sem{\Gamma \rightarrow T}$ are exactly the nodes that are hereditarily justified by $r$.
    The result follows from the fact that @-nodes, constant nodes and value-leaves of constant nodes
    are not hereditarily justified by the root.
\end{proof}


The following lemma is the counterpart of lemma
\ref{lem:varphiinjective} and it is proved identically.
\begin{lemma}[$\varphi$ is injective]
\label{lem:SIGMACONST:varphiinjective} $\varphi$ regarded as a
function defined on the set of sequences of nodes is injective in
the sense that for any two traversals $t_1$ and $t_2$:
\begin{itemize}
\item[(i)] if $\varphi (t_1^* ) = \varphi (t_2^* )$ then $t_1^* =t_2^*$;
\item[(ii)] if $\varphi (t_1 \filter r ) = \varphi (t_2 \filter r )$ then $t_1\filter r = t_2\filter r$.
\end{itemize}
\end{lemma}

\begin{corollary} \
\label{cor:SIGMACONST:varphi_bij}
\begin{itemize}
\item[(i)] $\varphi$ defines a bijection from $\travset(M)^*$
to $\varphi(\travset(M)^*)$;
\item[(ii)] $\varphi$ defines a bijection from $\travset(M)^{\filter r}$ to
$\varphi(\travset(M)^{\filter r})$.
\end{itemize}
\end{corollary}


\subsubsection{Correspondence theorem}
We would like to prove the counterpart of proposition
\ref{prop:rel_gamesem_trav} in the context of the simply-typed
$\lambda$-calculus \emph{with interpreted PCF constants}. The game
model of the language \pcf\ is given by the category $\mathcal{C}_b$
of well-bracketed strategies. Hence the well-bracketing assumption
stated at the beginning of section \ref{sec:gamesemcorresp} is
satisfied.

We first prove that $\travset(\_)^{\filter r}$ is continuous.
\begin{lemma}
\label{lem:travred_continuous} Let $(S,\subseteq)$ denote the set of
sets of justified sequences of nodes ordered by subset inclusion.
The function $\travset(\_)^{\filter r} : (CT,\sqsubseteq)
\rightarrow (S,\subseteq)$ is continuous.
\end{lemma}
\begin{proof} \
    \begin{description}
    \item[Monotonicity:] Let $T$ and $T'$ be two computation trees such that $T \sqsubseteq T'$
    and let $t$ be some traversal of $T$.
    Traversals ending with a node labelled $\bot$ are maximal therefore $\bot$ can only occur
    at the last position in a traversal. Let us prove the following two properties:
        \begin{itemize}
            \item[(i)]  If $t = t \cdot n$ with $n\neq \bot$ then $t$ is a traversal of $T'$;
            \item[(ii)] if $t= t_1 \cdot \bot$ then $t_1\in \travset(T')$.
        \end{itemize}

        (i) By induction on the length of $t$. It is trivial for the empty traversal.
            Suppose that $t = t_1 \cdot n$ is a traversal with $n \neq \bot$.
            By the induction hypothesis, $t_1$ is a traversal of $T'$.

            We observe that for all traversal rules, the traversal produced is of the form $t_1 \cdot n$ where
            $n$ is defined to be a child node or value-leaf of some node $m$ occurring in $t_1$.
            Moreover, the choice of the node $n$ only depends on the traversal $t_1$
            (for the constant rules, this is guaranteed by assumption (WB)).

            Since $T \sqsubseteq T'$, any node $m$ occurring in $t_1$ belongs
            to $T'$ and the children nodes and leaves of $m$ in $T$ also belong to the tree $T'$.
            Hence $n$ is also present in $T'$ and the rule used to produce the traversal $t$ of $T$
            can be used to produce the traversal $t$ of $T'$.

        (ii) $\bot$ can only occur at the last position in a traversal
        therefore $t_1$ does not end with $\bot$ and by (i) we have $t_1\in \travset(T')$.
\vspace{6pt}

        Hence we have:
        \begin{align*}
        \travset(T)^{\filter r} &= \{ t \filter r \ | \ t \in \travset(T)     \} \\
        & = \{ (t\cdot n) \filter r \ | \ t\cdot n \in \travset(T) \wedge n \neq \bot \}
            \union \{ (t \cdot \bot ) \filter r \ | \ t \cdot \bot \in \travset(T)  \} \\
\mbox{(by (i) and (ii))} \quad        & \subseteq  \{ (t\cdot n)
\filter r \ | \ t\cdot n \in \travset(T') \wedge n \neq \bot
\}
            \union \{ t \filter r \ | \ t \in \travset(T')  \} \\
        & = \travset(T')^{\filter r}
        \end{align*}

        \item[Continuity:] Let $t \in \travset \left( \Union_{n\in\omega} T_n \right)$.
        We write $t_i$ for the finite prefix of $t$ of length $i$.
        The set of traversals is prefix-closed therefore $t_i \in \travset \left( \Union_{n\in\omega} T_n \right)$ for any $i$.
        Since $t_i$ has finite length we have $t_i \in \travset(T_{j_i})$ for some $j_i \in \omega$.
        Therefore we have:
        \begin{align*}
          t \filter r &= (\bigvee_{i\in\omega} t_i ) \filter r   & (\mbox{the sequence $(t_i)_{i\in\omega}$ converges to $t$}) \\
          &= \Union_{i\in\omega} ( t_i \filter r )   & (\_ \filter r \mbox{ is continuous, lemma \ref{lem:projection_continuous}}) \\
          &\in \Union_{i\in\omega} \travset(T_{j_i})^{\filter r}   & (t_i \in \travset(T_{j_i})) \\
          &\subseteq \Union_{i\in\omega} \travset(T_i)^{\filter r}   & (\mbox{since } \{ j_i \sthat i \in \omega \} \subseteq \omega)
        \end{align*}

        Hence $\travset(\Union_{n\in\omega} T_n )^{\filter r} \subseteq \Union_{n\in\omega} \travset(T_n)^{\filter r}.$

    \end{description}
\end{proof}

\begin{proposition}
Let $\Gamma \vdash M : T$ be a PCF term and $r$ be the root of
$\tau(M)$. Then:
\begin{align*}
(i)  \quad\varphi_M(\travset(M)^*) = \revsem{M},  \\
(ii) \quad \varphi_M(\travset(M)^{\filter r}) = \sem{M}.
\end{align*}
\end{proposition}
\begin{proof}
We first prove the result for $\pcf_1$: (i) The proof is an
induction identical to the proof of proposition
\ref{prop:rel_gamesem_trav}. However we need to complete the case
analysis with the $\Sigma$-constant cases:
\begin{itemize}
\item The cases \pcfsucc, \pcfpred, \pcfcond\ and numeral constants are straightforward.

\item Suppose $M = \Omega_o$ then $\travset(\Omega_o) = \prefset ( \{ \lambda \cdot \bot \} )$ therefore
$\travset(\Omega_o)^{\filter r} = \prefset( \{ \lambda \} )$
and $\sem{\Omega_o} = \prefset( \{ q \})$ with $\varphi(\lambda) =
q$. Hence $\sem{\Omega_o} = \varphi
(\travset(\Omega_o)^{\filter r})$.
\end{itemize}
(ii) is a direct consequence of (i) and the Projection Lemma (Lemma
\ref{lem:SIGMACONST:varphi_projection}). \vspace{10pt}

\noindent We now extend the result to \pcf. Let $M$ be a \pcf\ term,
we have:
\begin{align*}
\sem{M} &= \Union_{n\in\omega} \sem{M_n} & (\mbox{\cite{abramsky:game-semantics-tutorial}, lemma 16})\\
&= \Union_{n\in\omega} \travset(\tau(M_n))^{\filter r} & (M_n \mbox{ is a $\pcf_1$ term}) \\
&= \travset(\Union_{n\in\omega} \tau(M_n) )^{\filter r} & (\mbox{by continuity of $\travset(\_)^{\filter r}$, lemma \ref{lem:travred_continuous}}) \\
&= \travset(\tau(M))^{\filter r} & (\mbox{by definition of } \tau(M)) \\
&= \travset(M)^{\filter r} & (\mbox{abbreviation}).
\end{align*}
\end{proof}

Hence by corollary \ref{cor:SIGMACONST:varphi_bij}, $\varphi$
defines a bijection from $\travset(M)^{\filter r}$ to
$\sem{M}$:
$$\varphi : \travset(M)^{\filter r} \stackrel{\cong}{\longrightarrow} \sem{M}.$$

\subsubsection{Example: \pcfsucc}

Consider the term $M = \pcfsucc\ 5$ whose computation tree is
represented below. The value-leaves are also represented on the
diagram, they are the vertices attached to their parent node with a
dashed line.
$$
\psmatrix[colsep=3ex,rowsep=2ex]
\lambda^0 \\
\pcfsucc & 0 & 1 & \ldots \\
\lambda^1 & 0 & 1 & \ldots \\
5 & 0 & 1 & \ldots \\
  & 0 & 1 & \ldots
\endpsmatrix
\ncline{1,1}{2,1} \ncline{2,1}{3,1} \ncline{3,1}{4,1}
\valueedge{1,1}{2,2} \valueedge{1,1}{2,3} \valueedge{1,1}{2,4}
\valueedge{2,1}{3,2} \valueedge{2,1}{3,3} \valueedge{2,1}{3,4}
\valueedge{3,1}{4,2} \valueedge{3,1}{4,3} \valueedge{3,1}{4,4}
\valueedge{4,1}{5,2} \valueedge{4,1}{5,3} \valueedge{4,1}{5,4}
$$

The following sequence of nodes is a traversal of $\tau(M)$:
$$ \Pstr[20pt]{ t = (l0){\lambda^0} \cdot (succ){\pcfsucc} \cdot (l1){\lambda^1} \cdot (c5){5} \cdot (v55-c5){5_5} \cdot (5l1-l1){5_{\lambda^1}} \cdot (6succ-succ){6_\pcfsucc} \cdot (6l0-l0,35){6_{\lambda^0}}}.
$$

The subsequences $t^*$ and $t \filter r$ are given by:
$$
\Pstr[17pt]{ t^* = (l0){\lambda^0} \cdot (l1-l0){\lambda^1} \cdot
(5l1-l1){5_{\lambda^1}} \cdot (6l0-l0){6_{\lambda^0}}.
\qquad  \mbox{ and } \qquad t
\filter r = (l0){\lambda^0} \cdot
(6l0-l0){6_{\lambda^0}}. }
$$
We have $\varphi(t^*) = q_0 \cdot q_5 \cdot 5_{q_5} \cdot 5_{q_0}$
and $\varphi(t\filter r) = q_0 \cdot 5_{q_0}$ where $q_0$
and $q_5$ denote the roots of two flat arenas over $\nat$. These two
sequences of moves correspond to some play of the interaction
semantics and the standard semantics respectively. The interaction
play is represented below:
$$\begin{array}{ccccc}
  \textbf{1} & \stackrel{5}{\multimap} & !\nat & \stackrel{\pcfsucc}{\multimap} & \nat \\
&&&&  \rnode{q0}{q_0} \\
&&  \rnode{q5}{q_5} \\
&&  \rnode{a5}{5_{q_5}} \\
&&&&  \rnode{a6}{6_{q_0}}
\end{array}
\nccurve[nodesep=2pt,ncurv=0.9,angleA=180,angleB=180]{->}{a5}{q5}
\nccurve[nodesep=2pt,ncurv=0.9,angleA=180,angleB=210]{->}{a6}{q0}
\ncarc[nodesep=2pt,ncurv=0.9,angleA=180,angleB=180]{->}{q5}{q0}
$$

\subsubsection{Another example : \pcfcond}

Consider the term $M = \lambda x y . \pcfcond\ 1\ x\ y$. Its
computation tree is represented below (without the value-leaves):
    $$ \pssetcomptree\tree{\lambda x y}
       {
          \tree{\pcfcond}
          {
            \tree{\lambda^1}{ \TR{1} }
            \tree{\lambda^2}{ \TR{x} }
            \tree{\lambda^3}{ \TR{y} }
          }
      }
    $$
For any value $v \in\mathcal{D}$ the following sequence of nodes is
a traversal of $\tau(M)$:
$$\Pstr[27pt]{ t = (lxy){\lambda x y} \cdot (cond){\pcfcond} \cdot (l1-cond){\lambda^1} \cdot (1){1} \cdot (v11-1){1_1}
    \cdot (l3){\lambda^3} \cdot (y-vxy){y} \cdot (vy-y){v_y}  \cdot (vl3-l3){v_{\lambda^3}} \cdot (vcond-cond,30){v_{\pcfcond}}
    \cdot (vlxy-lxy,30){v_{\lambda x y}}.
}
$$
The subsequences $t^*$ and $t \filter r$ are given by:
$$
\Pstr[17pt]{ t^* =  t = (lxy){\lambda x y} \cdot
        (l1-lxy){\lambda^1} \cdot
        (l3-lxy){\lambda^3} \cdot
        (y-vxy){y} \cdot
        (vy-y){v_y}  \cdot
        (vl3-l3){v_{\lambda^3}} \cdot
        (vlxy-lxy,35){v_{\lambda x y}}
\qquad  \mbox{ and } \qquad t \filter r =
(lxy){\lambda x y} \cdot (y-vxy){y} \cdot (vy-y){v_y}
\cdot (vlxy-lxy){v_{\lambda x y}}.
}
$$
The sequence of moves $\varphi(t^*)$ corresponds to some play of the
interaction semantics and the sequence $\varphi(t\filter r)$
is a play of the standard semantics obtained by hiding the internal
moves of $\varphi(t^*)$. The interaction play $\varphi(t^*)$ is
represented below:
$$\begin{array}{ccccccccccc}
!\nat & \otimes & !\nat & \stackrel{ \langle \sem{1}, \pi_1,
\pi_2\rangle }{\multimap} & !\nat & \otimes & !\nat & \otimes &
!\nat
& \stackrel{ \pcfcond}{\multimap} & \nat \\
&&&&&&&&&&  \rnode{q0}{q_0^{(\lambda x y)}} \\
&&&&  \rnode{qa}{q_a^{(\lambda^1)}} \\
&&&&  \rnode{1}{1} \\
&&&&&&  \rnode{qb}{q_b^{(\lambda^2)}} \\
&&  \rnode{qy}{q_y^{(y)}} \\
&&  \rnode{vqy}{v_{q_y}} \\
&&&&&&  \rnode{vqb}{v_{q_b}} \\
&&&&&&&&&& \rnode{vq0}{v_{q_0}}
\end{array}
\ncarc[nodesep=2pt,ncurv=0.9,angleA=180,angleB=180]{->}{vq0}{q0}
\ncarc[nodesep=2pt,ncurv=0.9,angleA=180,angleB=180]{->}{vqb}{qb}
\nccurve[nodesep=2pt,ncurv=0.9,angleA=180,angleB=180]{->}{vqy}{qy}
\ncarc[nodesep=2pt,ncurv=0.9,angleA=180,angleB=180]{->}{qy}{qb}
\ncarc[nodesep=2pt,ncurv=0.9,angleA=90,angleB=180]{->}{qb}{q0}
\nccurve[nodesepB=2pt,nodesepA=6pt,ncurv=0.9,angleA=180,angleB=180]{->}{1}{qa}
\ncarc[nodesep=2pt,ncurv=0.9,angleA=90,angleB=180]{->}{qa}{q0}
$$


\subsubsection{Game characterisation of safe terms}

A difficulty arises because of the presence of the Y combinator :
computation trees of \pcf\ terms are potentially infinite. Despite
this particularity, lemma \ref{lem:incrbound_iff_etanf_safe} still
holds in the \pcf\ setting:
\begin{lemma} \label{lem:pcf_safe_imp_incrbound} If $M$ is a safe
PCF term then $\tau(M)$ is incrementally-bound.
\end{lemma}
\begin{proof}
Let $i$ denote the number of occurrences of the Y combinator in $M$.
We first prove by induction on $i$ that $M_k$ is safe for any $k\in
\omega$. \emph{Base case:} $i=0$ then $M_k = M$. \emph{Step case:}
$i>0$. Let $Y_A N$ be a subterm of $M$. Since $M$ is safe, $N$ is
also safe. The number of occurrences of the Y combinator in $N$ is
smaller than $i$ therefore by the induction hypothesis $N_k$ is
safe. Consequently the term $Y_A^k N_k = \underbrace{N_k ( \ldots (
N_k}_{k \mbox{ times}} \Omega ) \ldots )$ is also safe and by
compositionality so is $M_k$.

Clearly, lemma \ref{lem:incrbound_iff_etanf_safe}(i) is remains
valid for infinite $\pcf_1$ terms (the subterms of the form $\Omega$
are just represented by the constant $\bot$ in the computation
tree), thus since $M_k$ is a safe $\pcf_1$ term, $\tau(M_k)$ is
incrementally-bound. Now let $z$ be a variable node in $\tau(M) =
\Union_{k\in\omega} \tau(M_k)$. There exists $k\in \omega$ such that
$z$ belongs to $\tau(M_k) \sqsubseteq \tau(M)$. If we write $r_k$ to
denote the root of the tree $\tau(M_k)$ then the path $[r_k,z]$ in
$\tau(M_k)$ is equal to the path $[r,z]$ in $\tau(M)$. Hence, since
the node $z$ is incrementally-bound in $\tau(M_k)$, it is also
incrementally-bound in $\tau(M)$.
\end{proof}


\begin{theorem}
Safe PCF terms are denoted by P-incrementally-justified strategies.
\end{theorem}
\begin{proof}
Let $M^{\infty}$ be the $\beta$-normal form of $M$ (i.e. the possibly infinite term obtained by reducing all the redexes in $M$). By lemma \ref{lem:ia_safety_preserved}, safety is preserved by small-step reduction therefore, by lemma \ref{lem:pcf_safe_imp_incrbound}, if $M$ is a \pcf\ term then $\tau(M^{\infty})$ is also
incrementally-bound.

Since \pcf\ constant rules are well-behaved (by Lemma
\ref{lem:sigma_order1_are_wellbehaved}), the result from Lemma
\ref{lem:betanf_wellbehavedconst_trav_pview_red} is also true for
Safe \pcf. Thus proposition
\ref{prop:Nher_incrbound_and_incrjustified}(i) remains valid for the
infinite computation trees of \pcf: infinite terms in $\beta$-nf
with an incrementally-bound computation tree are denoted by
P-incrementally-justified strategies. Consequently,
$\sem{M^{\infty}}$ is P-incrementally-justified. By soundness of the
game denotation, $\sem{M^{\infty}} = \sem{M}$, thus $\sem{M}$ is
P-incrementally-justified.
\end{proof}

Consequently, P-pointers are superfluous in the game denotation of safe \pcf\ terms {\it i.e.} pointers emanating from P-moves are uniquely recoverable.

\subsection{Safe \ialgol}

We are now in a position to consider the full safe Idealized Algol
language. The general idea is the same as for safe \pcf, however
there are some difficulties caused by the presence of the two new
base types \iavar\ and \iacom. We just give indications on how to
adapt our framework to the particular case of safe \ialgol\ without
giving the complete proofs. However we believe that enough
indications are given to convince the reader that the argument used
in the \pcf\ case can be easily adapted to \ialgol.

\subsubsection{Computation DAG}
In \pcf, arenas have a single initial move, therefore they can be
regarded as trees. In \ialgol, on the other hand, the base type
\iavar\ is represented by the infinite product of games
$\iacom^{\nat} \times \iaexp$ which has an infinite number of
initial moves. In order to preserve the relationship established
between arenas and computation trees, we need to accommodate the
definition of computation tree to reflect this property. The
consequence is that in \ialgol, ``computation trees'' become
``computation directed acyclic graphs (DAG)'': a computation DAG may
have (possibly infinitely) many roots and two nodes of a given level
can share children at the next level.


We use the notations $\mathcal{D}_{\iaexp} = \nat$ and
$\mathcal{D}_{\iacom} = \{ \iadone \}$ to denote the set of value
leaves of type \iaexp\ and \iacom\ respectively. There are two types
of value-leaves in the computation DAG: the value-leaf \iadone\ of
type \iacom\ and the value-leaves labelled in $\mathcal{D}_{\iaexp}$
of type \iaexp.

Let $n$ be a node. If $\kappa(n)$ is of type $(A_1,\ldots A_n,B)$,
we call $B$ the \emph{return type of $n$}. The set of value-leaves
of a node $n$ is given by $\mathcal{D}_{\iaexp}$ if the return type
of $n$ is \iaexp, by $\mathcal{D}_{\iacom}$ if its return type is
\iacom, and by $\mathcal{D}_{\iaexp} \union \{ \iadone \}$ if its
return type is \iavar.


Table \ref{tab:ia_computationdag} shows the computation DAG for each
construct of \ialgol. The value-leaves are represented in the DAGs
using the following abbreviations:
$$ \pssetcomptree\tree{n}{ \TRV{\mathcal{D}_\iaexp} }  \quad \mbox{ for }\quad
 \tree{n}{ \TRV{0} \TRV{1} \TRV{2} \TRV{\ldots} }
 \qquad \mbox{ and } \qquad
 \tree{n}{ \TRV{\mathcal{D}_\iadone} }  \quad \mbox{ for }\quad
 \tree{n}{ \TRV{\iadone }}.
$$

A term of type \iavar\ has a computation DAG with an infinite number
of root $\lambda$-nodes. Suppose that $M$ is a term of type \iavar,
then the computation DAG for $\lambda \overline{\xi} . M$ is
obtained by relabelling the root $\lambda$-nodes $\lambda^r$,
$\lambda^{w_0}$, $\lambda^{w_1}$, $\lambda^{w_2}$, \ldots into
$\lambda^r \overline{\xi}$, $\lambda^{w_0} \overline{\xi}$,
$\lambda^{w_1} \overline{\xi}$, $\lambda^{w_2} \overline{\xi}$,
\ldots. For a term $M$  of type \iaexp\ or \iacom, the computation
DAG for $\lambda \overline{\xi} . M$ is computed in the same way as
in the safe $\lambda$-calculus.

\begin{table}
\begin{center}
\begin{tabular}{cc}
$M$ & $\tau(M)$ \\ \hline \hline \\
x $: A \in \{ \iacom, \iaexp \}$ &
    $\psmatrix[colsep=3ex,rowsep=2ex] \lambda \\ x & \mathcal{D}_A \\  & \mathcal{D}_A \endpsmatrix
    \ncline{1,1}{2,1} \valueedge{1,1}{2,2} \valueedge{2,1}{3,2} $
\\ \\
x : \iavar &
    $\psmatrix[colsep=3ex,rowsep=3ex]
    \lambda^r & \lambda^{w_0} & \lambda^{w_1}  & \lambda^{w_2} & \lambda^{w_{\ldots}} \\
    \mathcal{D}_\iaexp &  & x & & \iadone \\
    &  &  & \mathcal{D}_\iaexp & \iadone
    \endpsmatrix
    \ncline{1,1}{2,3} \ncline{1,2}{2,3} \ncline{1,3}{2,3} \ncline{1,4}{2,3} \ncline{1,5}{2,3}
    \valueedge{2,3}{3,4} \valueedge{2,3}{3,5}
    \valueedge{1,1}{2,1}
    \valueedge{1,5}{2,5} \valueedge{1,4}{2,5} \valueedge{1,3}{2,5} \valueedge{1,2}{2,5}
    $
\\ \\
\iaskip : \iacom &
    $\psmatrix[colsep=3ex,rowsep=3ex] \lambda \\ \iaskip & \iadone \\  & \iadone \endpsmatrix
    \ncline{1,1}{2,1} \valueedge{1,1}{2,2} \valueedge{2,1}{3,2} $
\\ \\
$\iaassign\ L\ N :\iacom$ &
    $\psmatrix[colsep=3ex,rowsep=3ex] & \lambda \\ & \iaassign & \iadone \\ \tau(N:\iaexp)  & \tau(L:\iavar) & \iadone \endpsmatrix
    \ncline{1,2}{2,2} \ncline{2,2}{3,2} \ncline{2,2}{3,1}
    \valueedge{1,2}{2,3} \valueedge{2,2}{3,3} $
\\ \\
$\iaderef\ L :\iaexp$ &
    $\psmatrix[colsep=3ex,rowsep=3ex] \lambda \\ \iaderef & \iadone \\ \tau(L:\iavar) & \iadone \endpsmatrix
    \ncline{1,1}{2,1} \ncline{2,1}{3,1} \valueedge{1,1}{2,2} \valueedge{2,1}{3,2} $
\\ \\
$\iaseq_{\iaexp}\ N_1\ N_2 :\iacom$ &
    $\psmatrix[colsep=3ex,rowsep=3ex] & \lambda \\ & \iaseq_{\iaexp} & \mathcal{D}_\iaexp \\ \tau(N_1:\iacom)  & \tau(N_2:\iaexp) & \iadone \endpsmatrix
    \ncline{1,2}{2,2} \ncline{2,2}{3,2} \ncline{2,2}{3,1}
    \valueedge{1,2}{2,3} \valueedge{2,2}{3,3} $
\\ \\
$\iamkvar\ N_w\ N_r :\iavar$ &
    $\psmatrix[colsep=3ex,rowsep=3ex]
    \lambda^r & \lambda^{w_0} & \lambda^{w_1}  & \lambda^{w_2} & \lambda^{w_{\ldots}} \\
    \mathcal{D}_\iaexp &  & \iamkvar & & \iadone \\
    & \tau(N_r) & \tau(N_w) & \mathcal{D}_\iaexp & \iadone
    \endpsmatrix
    \ncline{1,1}{2,3} \ncline{1,2}{2,3} \ncline{1,3}{2,3} \ncline{1,4}{2,3} \ncline{1,5}{2,3}
    \ncline{2,3}{3,2} \ncline{2,3}{3,3}
    \valueedge{2,3}{3,4} \valueedge{2,3}{3,5}
    \valueedge{1,1}{2,1}
    \valueedge{1,5}{2,5} \valueedge{1,4}{2,5} \valueedge{1,3}{2,5} \valueedge{1,2}{2,5}
    $
\\ \\
$\ianewin{x}\ N : A \in \{ \iacom, \iaexp \} $ &
   $\psmatrix[colsep=3ex,rowsep=3ex] \lambda \\ \ianewin{x} & \mathcal{D}_A \\ \tau(N:A) & \mathcal{D}_A \endpsmatrix
    \ncline{1,1}{2,1} \ncline{2,1}{3,1} \valueedge{1,1}{2,2} \valueedge{2,1}{3,2} $
\end{tabular}
\end{center}
  \caption{Computation DAGs for the constructs of \ialgol.}
  \label{tab:ia_computationdag}
\end{table}


\subsubsection{Traversals}
Let $p$ be a node and suppose that its $i$th child $n$ has the
return type \iavar. Then $n$ is in fact constituted of several
$\lambda$-nodes : $\lambda^r \overline{\xi}$, $\lambda^{w_0}
\overline{\xi}$, \ldots. From $p$'s point of view, these nodes are
referenced as follows: $i.r$ refers to $\lambda^r \overline{\xi}$
and  $i.w_k$ refers to $\lambda^{w_k} \overline{\xi}$ for $k \in
\omega$.

\begin{itemize}
\item \emph{The application rule}

There are two rules (app$_{\iaexp}$) and (app$_{\iacom}$)
corresponding to traversals ending with an @-node of return type
\iaexp\ and \iacom\ respectively. These rules are identical to the
rule \iaexp\ of section \ref{subsec:traversal}.

The application rule for $@$-nodes with return type \iavar\ is:
$$(\mbox{app}_{\iavar})
\rulef{ \Pstr{t \cdot (lHyp){\lambda^k \overline{\xi}} \cdot
(appHyp-lHyp,35:0){@} \in \travset }
 }{\Pstr[18pt] {t \cdot (l){\lambda^k
\overline{\xi}} \cdot (app-l,35:0){@} \cdot (l2-app,35:0.k){\lambda^k
\overline{\eta}} \in \travset }}
 \ k \in \{ r, w_0, w_1, \ldots \}
$$


\item \emph{Input-variable rules}

There are two rules (InputVar$^{\iaexp}$) and (InputVar$^{\iacom}$)
which are the counterparts of rule (InputVar$^0$) of section
\ref{subsec:traversal} and are defined identically.

Let $x$ be an input-variable of type \iavar:
$$ (\mbox{InputVar}^{\iavar})
\rulef{t \cdot \lambda^r \overline{\xi} \cdot x \in \travset}
    {t \cdot \lambda^r \overline{\xi} \cdot \rnode{x}{x} \cdot v_x \in \travset }
\hspace{2cm} (\mbox{InputVar}^{' \iavar}) \rulef{t \cdot
\lambda^{w_i} \overline{\xi} \cdot x \in \travset}
    {t \cdot \lambda^{w_i} \overline{\xi} \cdot \rnode{x}{x} \cdot \iadone_x \in \travset }
$$

\item \emph{IA constants rules}

The rules for \ianew\ are purely structural, they are defined the
same way as the rules (app$_{\iaexp}$), (app$_{\iacom}$) and
(app$_{\iadone}$).

The rules for \iaderef\ are:
$$(\mbox{deref}) \rulef{t \cdot \iaderef \in \travset}{\Pstr[15pt]{t \cdot (d){\iaderef} \cdot (n-d,35:1.r){n} \in \travset }}
 \hspace{1.6cm} (\mbox{deref'})
\rulef{t \cdot \iaderef \cdot n \cdot t_2 \cdot v_n \in \travset} {t
\cdot \iaderef \cdot n \cdot t_2 \cdot v_n \cdot v_{\iaderef}\in
\travset }
$$


The rules for \iaassign\ are:
$$(\mbox{assign}) \rulef{t \cdot \iaassign \in \travset}{\Pstr[15pt]{t \cdot (ass){\iaassign} \cdot (n-ass,35:1){n} \in \travset} }
\hspace{1.6cm}
(\mbox{assign'})
\rulef{t \cdot \iaassign \cdot n \cdot t_2 \cdot v_n \in
\travset} {\Pstr[18pt]{t \cdot (ass){\iaassign} \cdot (n){n} \cdot
t_2 \cdot v_n \cdot (m-ass,15:2.w_n){m} \in \travset } }
$$
$$(\mbox{assign''})  \rulef{\Pstr{t \cdot (assHyp){\iaassign} \cdot t_2 \cdot (mHyp-assHyp,35:2.w_k){m} \cdot t_3 \cdot \iadone_m \in \travset}}
{t \cdot \iaassign \cdot t_2 \cdot m \cdot t_3 \cdot \iadone_m \cdot
\iadone_{\iaassign} \in \travset }
$$

The rules for $\iaseq_{\iaexp}$ are:
$$(\mbox{seq}) \rulef{t \cdot \iaseq \in \travset}{\Pstr[13pt]{t \cdot (seq){\iaseq} \cdot (n-seq,35:1){n} \in \travset } }
\hspace{1.6cm} (\mbox{seq'})
\rulef{t \cdot \iaseq \cdot n \cdot t_2 \cdot v_n \in
\travset} {\Pstr[18pt]{ t \cdot (seq){\iaseq} \cdot (n){n} \cdot t_2
\cdot v_n \cdot (m-seq,25:2){m} \in \travset }}
$$
$$(\mbox{seq''})  \rulef{\Pstr{t \cdot (seqHyp){\iaseq} \cdot t_2 \cdot (mHyp-seqHyp,35:2){m} \cdot t_3 \cdot v_m \in \travset}}
{t \cdot \iaseq \cdot t_2 \cdot m \cdot t_3 \cdot v_m \cdot
v_{\iaseq} \in \travset }$$




The rules for \iamkvar\ are:
$$(\mbox{mkvar}_r) \rulef{t \cdot \lambda^r \overline{\xi} \cdot \iamkvar \in \travset}{\Pstr[14pt]{t \cdot \lambda^r \overline{\xi} \cdot (d){\iamkvar} \cdot (n-d,35:1){n} \in \travset} }
\hspace{1cm} (\mbox{mkvar}_r')
\rulef{t \cdot \iamkvar \cdot n \cdot t_2 \cdot v_n \in \travset} {t
\cdot \iamkvar \cdot n \cdot t_2 \cdot v_n \cdot v_{\iamkvar}\in
\travset } $$
$$(\mbox{mkvar}_w) \rulef{t \cdot \lambda^{w_k} \overline{\xi} \cdot \iamkvar \in \travset}{\Pstr[15pt]{t \cdot \lambda^{w_k} \overline{\xi} \cdot (mk){\iamkvar} \cdot (n-mk,35:2){n} \in \travset} }$$
$$ (\mbox{mkvar}_w'')  \rulef{t \cdot \lambda^{w_k} \overline{\xi} \cdot \iamkvar \cdot n \cdot t_2 \cdot \iadone_n \in \travset}
{t \cdot \lambda^{w_k} \overline{\xi} \cdot \iamkvar \cdot n \cdot
t_2 \cdot \iadone_n \cdot \iadone_{\iamkvar} \in \travset }
$$
These four rules are not sufficient to model the constant \iamkvar.
Indeed, consider the term $\iaassign\ (\iamkvar\ (\lambda x . M) N)
7$. The rule (\mbox{mkvar}$_w''$) permits to traverse the node
\iamkvar\ and to go on by traversing the computation tree of
$\lambda x . M$. The problem is that when traversing $\tau(M)$, if
we reach a variable $x$, we are not able to relate $x$ to the value
$7$ that is assigned to the variable.

To overcome this problem, we need to define traversal rules for
variable in such a way that a variable node bound by the second
child of a $\iamkvar$-node is treated differently from other
variables.

\item \emph{Variable rules}
Let $x$ be a non input-variable node. It either corresponds to a $\lambda$-abstracted variable or
a block-allocated variable declared by the $\ianewin{x}$ construct.

\begin{itemize}
\item Suppose that $x$ is $\lambda$-abstracted and let $\lambda \overline{x}$ be its binder.
In \ialgol, the only constant nodes of order greater than 1 is
\iamkvar, therefore there are two cases: $\lambda \overline{x}$ is
either the child of a node in $N_@ \union N_{\sf var}$ or it is the
second child of a \iamkvar-node.

To handle the first case, we define a rule similar to the (Var) rule
of section \ref{subsec:traversal} with some modification to take
into account variables $x$ of type \iavar (in which case $x$ has
multiple parent $\lambda$-nodes). We do not give the details here
but it is easy to see how to redefine this rule.

To handle the case where $\lambda \overline{x}$ is the child of a
\iamkvar-node, we define the following rule:
$$ (\mbox{Var}_{\iamkvar})  \rulef{t \cdot \lambda^{w_k} \overline{\xi} \cdot \iamkvar \cdot \lambda \overline{x} \cdot t_2 \cdot x \in \travset}
{t \cdot \lambda^{w_k} \overline{\xi} \cdot \iamkvar \cdot \lambda
\overline{x} \cdot t_2 \cdot x \cdot k_{x} \in \travset }
$$

\item Suppose that $x$ is block-allocated with $\ianewin{x}$.

We call \emph{overwrite of $x$ relatively to an occurrence of a} ``\ianewin{x}''\emph{-node}, any sequence of nodes of the form
$\Pstr[17pt]{(decl){\ianewin{x}}\cdot \ldots \cdot \lambda^{w_k}\overline{\xi} \cdot (x-decl,25){x}}$ for some $k\in \mathcal{D}_{\iaexp}$ and node $\lambda^{w_k}\overline{\xi}$ parent
of $x$.
$$(\mbox{Var}_w)
    \rulef{
        t \cdot \lambda^{w_k} \overline{\xi} \cdot x \in \travset
    }
    {   t \cdot \lambda^{w_k} \overline{\xi} \cdot x \cdot \iadone_x \in
        \travset
    },
$$

$$(\mbox{Var}_r)
    \rulef{
        \Pstr[17pt]{t_1 \cdot (decl){\ianewin{x}} \cdot t_2 \cdot \lambda^r \overline{\xi} \cdot (x-decl,25){x} \in \travset}
    }
    {   t_1 \cdot \ianewin{x} \cdot t_2 \cdot \lambda^r \overline{\xi}
        \cdot x \cdot 0_x \in \travset
    }
    \mbox{ if $t_2$ contains no overwrite of $x$},
$$

$$(\mbox{Var'}_r)
    \rulef{
        \Pstr[15pt]{
            t_1 \cdot (decl){\ianewin{x}} \cdot t_2 \cdot \lambda^r \overline{\xi} \cdot (x-decl,25){x} \in \travset
        }
    }
    {
        t_1 \cdot \ianewin{x} \cdot t_2 \cdot \lambda^r \overline{\xi} \cdot x \cdot k_x \in \travset
    }
    \mbox{ if $\lambda^{w_k} \cdot x$ is the last overwrite of $x$ in } t_2. $$
\end{itemize}
\end{itemize}
 
\subsubsection{Game semantics correspondence}
The properties that we proved for computation trees and traversals
of the safe $\lambda$-calculus with constants can easily be lifted
to computation DAGs of \ialgol. In particular:
\begin{itemize}
\item constant traversal rules are well-behaved (for order-$0$ and order-$1$ constants, this is a consequence
of Lemma \ref{lem:sigma_order1_are_wellbehaved}; for $\iamkvar$
however it needs to be proved separately);
\item P-view of traversals are paths in the computation DAG;
\item the P-view of the reduction of a traversal is the reduction of the P-view,
and the O-view of a traversal is the O-view of its reduction
(Lemma \ref{lem:pview_trav_projection} and
\ref{lem:oview_trav_projection});
\item there is a mapping from vertices of the computation DAG to moves in the interaction game semantics;
\item there is a correspondence between traversals of the computation tree and plays in interaction game semantics;
\item consequently, there is a correspondence between the standard game semantics and
the set of justified sequences of nodes $\travset(M)^{\filter r}$.
\end{itemize}

\subsubsection{Game-semantic characterisation of safe terms}
Clearly, the computation DAG of a safe term is incrementally-bound.
By using the correspondence between traversals and plays, it is easy
to prove that incrementally-bound computation trees are denoted by
P-incrementally-justified strategies. Consequently, by lemma
\ref{lem:incrjustified_pointers_uniqu_recover}, P's pointers are superfluous in the
game semantics of safe \ialgol\ terms.

Since the game denotation of an \ialgol\ term is fully determined by
the set of complete plays, this pointer economy suggests that the
game denotation of a safe \ialgol\ can be represented in a compact
way. This raises the question of the decidability of observational
equivalence for safe \ialgol.



%%%%%%%%%%%%%%%%%%%%%%%%%%%%
%%%%%%%%%%%%%%%%%%%%%%%%%%%%
\notetoself{the following section needs to be integrate into the previous chapter.}


\section{Game-semantic of Safe PCF}
In this section will give a game-semantic characterization of Safe
PCF based on syntactical arguments.

\begin{definition}
We say that a PCF term is \defname{semi-safe} if it is of the form
$N_0 N_1 \ldots N_k$ for $k\geq 1$ where each of the $N_i$ is a Safe
PCF term or if it can be written $\lambda \overline{x} . N$ for some
safe PCF term $N$.
\end{definition}
Semi-safe terms are either safe or ``almost safe'' in the sense that
they can be turned into an equivalent (i.e.~with isomorphic game
semantics) safe term  by performing $\eta$-expansions. Indeed, let
$M$ be an semi-safe term that is unsafe. If $M$ is of the first form
$N_0 N_1 \ldots N_k : (A_1,\ldots,A_n)$ with $k\geq 1$ then let
$\varphi_i:A_i$ for $i\in\{1..n\}$ be fresh variables, using the
(app) and (abs) rules we can build the safe term $\lambda \varphi_1
\ldots \varphi_n . N_0 N_1 \ldots N_k \varphi_1 \ldots \varphi_n$.
If $M$ is of the second form $\lambda \overline{x} . N$ then using
the abstraction rule we can build the equivalent safe term $\lambda
\overline{y} \overline{x}. N$  where $\overline{y} = fv(\lambda
\overline{x}. N)$.

The $\beta$-normal form of a \pcf\ term is the possibly infinite
term obtained by reducing all the redexes in $M$.

\subsubsection{Safe terms vs P-i.j.\ strategies}

In the context of the simply typed lambda calculus, the
correspondence between safety and P-incremental justification was
first shown in \cite[Theorem 3(ii)]{blumong:safelambdacalculus}
using a syntactic argument:
\begin{theorem}[\cite{blumong:safelambdacalculus},Theorem 3(ii)]
\label{thm:safeincrejust}
 In the simply typed lambda calculus:
\begin{enumerate}[(i)]
\item If $M$ is safe then $\sem{M}$ is P-incrementally justified.
\item If $M$ is a closed term and $\sem{M}$ is
  P-incrementally justified then the $\eta$-long form of the
  $\beta$-normal form of $M$ is safe.
\end{enumerate}
\end{theorem}
In fact the following more precise result holds (the proof of the
previous theorem can be easily adapted to this one):
\begin{theorem}[Semi-safety and P-incremental justification]
\label{thm:semisafeincrejust} Let $\Gamma \vdash M : A$ be a simply typed term. Then:
\begin{enumerate}[(i)]
\item If $\Gamma \vdash M : A$ is semi-safe then $\sem{\Gamma \vdash M : A}$ is P-incrementally justified.
\item If $\sem{\Gamma \vdash M : A}$ is
  P-incrementally justified then $\etalnf{\betanf{M}}$ is
semi-safe if $M$ is open and safe if $M$ is closed.
\end{enumerate}
\end{theorem}



In the context of \pcf\ however, only the first part of the theorem
holds (see \cite{blumtransfer} for the proof). However (ii) does not
hold. Indeed, take the closed \pcf\ term $M = \lambda f x y. f
(\lambda z. \pcfcond (\pcfsucc\ x) y z )$ where $x,y,z:o$ and
$f:((o,o),o)$. $M$ is in normal form (conditional cannot be reduced
since the value of $x$ is undetermined). The $\eta$-long form of the
$\beta$-normal form of $M$ is therefore $M$ itself which is unsafe.
But clearly we have $\sem{M} = \sem{\lambda f x y. f (\lambda z.
z)}$, and since $\lambda f x y. f (\lambda z. z)$ is safe, by (i),
$\sem{M}$ is P-incrementally justified.

Such counter-example arises because the conditional operator of
\pcf\ permits us to construct terms in normal form that contain
``dead code'' {\it i.e.}~some subterm that will never be evaluated
for any value of M's parameters. In the example above, the dead code
consists of the subterm $y$. In general, if the dead code part of
the computation tree contains a variable that is not incrementally
bound then the resulting term will be unsafe even if the rest of the
tree is incrementally bound. In the example above, it was possible
to turn $M$ into the equivalent safe term $\lambda f x y. f (\lambda
z. z)$ by eliminating the dead code from $M$. In fact we can
generalise this method to any \pcf\ term with a P-incrementally
justified denotation.
\smallskip

Dead code elimination can be difficult to achieve in practice but it
is easy to define it formally: We say that a subterm $N$ occurring
in a context $C[-]$ in $M : (A_1, \ldots, A_n,o)$ is part of the
\defname{dead code} of $M$ if for any term $T_0$ of the form $M M_1
\ldots M_n$, any reduction sequence starting from $T_0$ does not
involve a reduction of the subterm $N$ {\it i.e.}~for any reduction
sequence $T_0 \redar T_1 \redar \ldots \redar T_k$, there is no
$j\in \{0.. k-1\}$ such that $T_j = C[N]$ and $T_{j+1} = C[N']$ for
some term $N'$.


Let $M$  be a \pcf\ term in $\eta$-nf. An occurrence of a variable
$x$ in $M$ is said to be a \defname{dead occurrence} if it occurs in
the dead code of $M$. In other words, it is a dead occurrence of $x$
if the corresponding node in the computation tree does not appear in
any traversal of $\travset(M)$. Equivalently, thanks to the
Correspondence Theorem, an occurrence of $x:B$ is dead if and only
if the initial move of the arena $\sem{B}$ does not appear in any
play of $\sem{M}$.


We define $M^*$ as the term obtained from $M$ after substituting all
subterms of the form  $x N_1 \dots N_k$ for some dead variable
occurrence $x:(B_1,\ldots, B_k, o)$ by the constant $0$. This
process is called \defname{dead variable elimination}. Note that if
$M$ is in $\eta\beta$-nf then so is $M^*$. We also write $\tau(M)^*$
to denote the equivalent transformation on the computation tree.
Since the computation tree is constructed from the $\eta$-nf of $M$,
we will use this notation even when $M$ is not in $\eta$-nf.



\begin{proposition}[Incremental-binding and P-incremental justification coincide] \
\label{prop:Nher_incrbound_and_incrjustified_pcf} Let $\Gamma \vdash
M : A$ be a PCF term in $\beta$-normal form.
\begin{enumerate}[(i)]
\item  If $\tau(\Gamma \vdash M : A)$ is incrementally-bound then $\sem{\Gamma \vdash M : A}$ is P-incrementally justified,
\item  if $\sem{\Gamma \vdash M : A}$ is P-incrementally justified
then $\tau(\Gamma \vdash M : A)^*$ is incrementally-bound.
\end{enumerate}
\end{proposition}
\begin{proof}
(i) The proof is exactly the same as in the simply typed lambda calculus case,
see \cite[Proposition 4.1.5(i)]{blumtransfer}.

\noindent (ii)
Take $\Gamma \vdash M : A$ a \pcf\ term in $\beta$-normal form denoted by $\sem{\Gamma \vdash M : A}$ P-incrementally justified. Let $r$ denote the root of $\tau(M)^*$.
Let $n$ be a node of $\tau(M)^*$ labelled by the variable $x$.
$\tau(M)^*$ is free from dead code therefore $n$ is not a dead occurrence of $x$ and there exists a traversal of $\tau(M)^*$ of the form $t \cdot x$.

\pcf\ constants are of order $1$ at most therefore they cannot
hereditarily justify a variable node, thus $x$ is necessarily
hereditarily justified by the only occurrence $r$ of the root of the
computation tree.

By considering $t\cdot x$ as a traversal of $\tau(M)$,  the
correspondence theorem gives $\varphi((t \cdot x) \filter r) =
\varphi((t \filter r) \cdot x) \in \sem{M}$. Since $\sem{M}$ is
P-incrementally justified, $\varphi(x)$ must point to the last
O-move in $\pview{\varphi(t \filter r)}$ with order strictly greater
than $\ord{\varphi(x)}$. Consequently $x$ points to the last node in
$\pview{t \filter r} \filter N^{\lambda}$ with order strictly
greater than $\ord{x}$. We have:
\begin{align*}
\pview{t \filter r} &= \pview{t} \filter N^{r \vdash} & (\mbox{by Lemma \ref{lem:betanf_wellbehavedconst_trav_pview_red}}) \\
& = [r,x[ \ \filter N^{r \vdash} & (\mbox{by Prop.\ \ref{prop:pviewtrav_is_path}})
\end{align*}
\notetoself{review use of Lemma
\ref{lem:betanf_wellbehavedconst_trav_pview_red}}

Since $M$ is in $\beta$-nf, the set of nodes not hereditarily
enabled by $r$ is exactly the set of nodes hereditarily enabled by
$N_{\Sigma}$ thus $[r,x[ \ \filter N^{r \vdash} = [r,x[\ \setminus\
N^{\filter \Sigma}$. Moreover \pcf\ constants are of order $1$ at
most therefore $N^{\filter \Sigma} = N_{\Sigma} \union N^c_{\Sigma}$
where $N^c_{\Sigma}$ is the set of children nodes of $N_{\Sigma}$.
Thus $\pview{t \filter r} \filter N^{\lambda} = ([r,x[\ \setminus\
N_{\Sigma} \setminus N^c_{\Sigma} ) \filter N^{\lambda} = ([r,x[\
\setminus\  N^c_{\Sigma} )  \filter N^{\lambda}$, and since
$N^c_{\Sigma}$ is constituted of order $0$ lambda-nodes only, $x$
must point to the last node in $[r,x[ \filter N^{\lambda}$ with
order strictly greater than $\ord{x}$.

Hence if $x$ is a bound variable node then it is bound by the
last $\lambda$-node in $[r,x[$ with order strictly greater than
$\ord{x}$ and if $x$ is a free variable then it points to $r$ and
therefore all the $\lambda$-node in $]r,x[$ have order smaller than
$\ord{x}$. Thus $\tau(M)^*$ is incrementally-bound.
\end{proof}

The counterpart of Lemma 4.1.6 from
\cite{blumtransfer} can be stated as follows in the context of PCF:
\begin{lemma}[Semi-safety and incrementally-binding]
\label{lem:incrbound_iff_etanf_safe_pcf} Let $\Gamma \vdash M : A$
be a PCF term.
\begin{itemize}
\item[(i)] If $\Gamma \vdash M : A$ is a semi-safe term then $\tau(\Gamma \vdash M : A)$ is incrementally-bound ;
\item[(ii)] conversely, if $\tau(\Gamma \vdash M : A)$ is incrementally-bound then the $\eta$-normal form of $\Gamma \vdash M : A$ is semi-safe if $M$ is open and safe if $M$ is closed.
\end{itemize}
\end{lemma}
The proof can be obtained by adapting the proof
of Lemma 4.1.6 from \cite{blumtransfer}.

\begin{theorem}[Semi-safety and P-incremental justification]
\label{thm:semisafeincrejust_pcf} Let $\Gamma \vdash M : A$ be a PCF term. Then:
\begin{enumerate}[(i)]
\item If $\Gamma \vdash M : A$ is semi-safe then $\sem{\Gamma \vdash M : A}$ is P-incrementally justified.
\item If $\sem{\Gamma \vdash M : A}$ is
  P-incrementally justified then $\etalnf{\betanf{M}}^*$ is
  semi-safe  if $M$ is open, and safe if $M$ is closed.
\end{enumerate}
\end{theorem}

\begin{proof}
\noindent(i)
A proof of this is given in the proof of Theorem 4.2.10 in \cite{blumtransfer}.

\noindent(ii) Suppose $M$ is a \pcf\ term with a P-incrementally
justified strategy denotation. By Proposition
\ref{prop:Nher_incrbound_and_incrjustified_pcf}(ii),
$\tau(\betanf{M})^* = \tau(\etalnf{\betanf{M}}^*)$ is
incrementally-bound. If $M$ is closed then so is
$\etalnf{\betanf{M}}^*$ therefore by Lemma
\ref{lem:incrbound_iff_etanf_safe_pcf},
$\etalnf{\etalnf{\betanf{M}}^*} = \etalnf{\betanf{M}}^*$ is safe. If
$M$ is open then so is $\etalnf{\betanf{M}}^*$ and by Lemma
\ref{lem:incrbound_iff_etanf_safe_pcf},
$\etalnf{\etalnf{\betanf{M}}^*} = \etalnf{\betanf{M}}^*$ is
semi-safe.
\end{proof}


We write \pcf' to denote the language obtained by extending \pcf\
with the $\pcfcase_k$ construct (see \cite{Abr02}).
The $\pcfcase_k$ construct is the obvious generalisation of the
conditional operator \pcfcond\ to $k$ branches instead of $2$. All the results obtained so far concerning Safe \pcf\ (including those
cited from \cite{blumtransfer}) can clearly be transposed to \pcf'.

\subsubsection{Definability result}

The previous theorem leads to the following definability result for safe \pcf':
\begin{proposition}[Definability for safe \pcf' terms]
\label{prop:safetydefinability} Let $\overline{A}=(A_1,\ldots, A_i)$
and $B =(B_1, \ldots, B_l,o)$ be two PCF types for some $i,l\geq 0$
and $\sigma$ be a well-bracketed innocent P-i.j.\ strategy with
finite view function defined on the game $!A_1 \otimes \ldots
\otimes !A_i \lingamear (!B_1 \lingamear \ldots \lingamear !B_l
\lingamear o) $. There exists a \emph{semi-safe} PCF' term
$\overline{x} : \overline{A} \vdash M : B$ in $\eta$-long normal
form such that:
$$ \sem{\overline{x} : \overline{A} \vdash M_\sigma : B} = \sigma $$
and a safe closed PCF' term $\vdash_s M'_\sigma : (\overline{A},B)$ in $\eta$-long normal form such that:
$$ \sem{\vdash M'_\sigma : (\overline{A},B)} \cong \sigma \ .$$
\end{proposition}
\begin{proof}
By the standard definability result for PCF', there is a term
$\overline{x} : \overline{A} \vdash N : B$ such that
$\sem{\overline{x} :\overline{A} \vdash N : B} = \sigma$. Take
$M_\sigma$ to be $\etalnf{\betanf{N}}^* $. We have
$\sem{\overline{x} : \overline{A} \vdash M_\sigma : B} =
\sem{\overline{x} :\overline{A} \vdash N : B} = \sigma$ and by
Theorem  \ref{thm:semisafeincrejust_pcf}(ii), $M_\sigma$ is
semi-safe. For the second part we just need to take $M'_\sigma =
\lambda \overline{x}. M_\sigma$.
\end{proof}



\subsubsection{Application of the definability result: a syntactic
argument showing compositionality of P-i.j.\ strategies}


We have already shown in Sec. \ref{sec:closedpij} that under certain
conditions, P-i.j.\ strategies compose. Here we will obtain a
slightly weaker version of this result using a much simpler argument
which exploits the definability result from the previous section.


 Let $\overline{A} = (A_1, \ldots, A_i)$, $B = (B_1, \ldots,
B_l,o)$ and $C=(C_1,\ldots,C_k,o)$ be three PCF types for some
$i\geq 1,l,k\geq 0$. Let $f:\ !A_1 \otimes \ldots \otimes !A_i
\lingamear B$ and $g:\ !B\lingamear C$ be two innocent
well-bracketed and P-incrementally justified strategies with finite
view function. We would like to find under which conditions the
composition $f\fatcompos g$ is also P-incrementally justified.

By the definability result, there are two closed safe terms (in $\eta$-nf) $\vdash M_f :(\overline{A},B)$  and $\vdash M_g :B \typear C$ such that $\sem{M_f} = f$
and $\sem{M_f} = g$.
We define the term $M_{f\fatcompos g} = \lambda \overline{x} . M_g (M_f \overline{x})$ for some fresh variables $\overline{x} : \overline{A}$. Clearly we have $\sem{M_{f\fatcompos g}} = \sem{M_f} \fatcompos \sem{M_g} = f\fatcompos g$.

\paragraph{Sufficient conditions}

By Theorem \ref{thm:semisafeincrejust_pcf}, we know that
$f\fatcompos g$ is P-incrementally justified just when
$\etalnf{\betanf{M_{f\fatcompos g}}}^*$ is safe. We will now exploit
this fact to extract a sufficient condition on the types $A$ and $B$
for the composition of $f$ and $g$ to be P-incrementally justified.

The term $M_f$ and $M_g$, being in $\eta$-nf, are of the following forms:
\begin{eqnarray*}
\vdash M_f &=& \lambda x_1^{A_1} \ldots x_i^{A_i} \varphi_1^{B_1} \ldots \varphi_l^{B_l} . N_f^o\\
\vdash  M_g &=& \lambda y^{ (B_1, \ldots, B_l,o)} \phi_1^{C_1} \ldots \phi_k^{C_k} . N_g^o
\end{eqnarray*}
for some distinct variables $x_1, \ldots, x_i$, $y$, $\varphi_1, \dots \varphi_l$, $\phi_1, \dots \phi_k$  and $\eta$-normal terms $N_f$ and $N_g$:
\begin{eqnarray*}
x_1:A, \ldots, x_i:A_i, \varphi_1:B_1, \dots, \varphi_l:B_l &\vdash& N_f :o \\
y: (B_1, \ldots, B_l,o), \phi_1:C_1, \dots, \phi_l:C_l &\vdash& N_g :o
\end{eqnarray*}



The fact that $M_f$ and $M_g$ are safe does not imply that $M_{f\fatcompos g}$ is: take $M_f = \lambda x^o z^o.x$ and $M_g = \lambda y^{(o,o)} . y a$ for some constant $a\in \Sigma$, then $\lambda x:A . M_g (M_f x) = \lambda x . (\lambda y . y a) ( \underline{(\lambda x z.x) x} )$ is unsafe because of the underlined subterm. However we have:
\begin{align*}
f\fatcompos g &= \sem{\lambda \overline{x} . M_g (M_f  \overline{x})} \\
 &= \sem{\lambda \overline{x} . (\lambda \phi_1\ldots \phi_k . N_g) [(M_f \overline{x}) / y]} \\
&= \sem{\lambda \overline{x} \phi_1 \dots \phi_k. N_g [(M_f  \overline{x}) / y]}
& \mbox{(the $x_j$'s and $\phi_j$'s are disjoint)}.
\end{align*}

We now concentrate on the term  $\lambda \overline{x} \phi_1 \dots
\phi_k. N_g [(M_f  \overline{x}) / y]$ and try to find a sufficient
condition guaranteeing its safety.

\subparagraph{A sufficient condition}
\begin{lemma}
Suppose that $\Gamma,y:B \vdash M$ is a safe term in $\eta$-nf and $\Gamma \vdash R : B$ is an almost safe application. Let $N$ denote the set of nodes of the computation tree $\tau(M)$. We have:
\begin{align*}
\Gamma \vdash M[R/y] :A \mbox{ safe }
\iff&  \forall x \in fv(R) . \\
    & \forall n_y \in N_{\sf fv} \mbox{ labelled $y$}.
      \forall m \in N_{\lambda} \inter ]r,n_y] : \ord{m} \leq \ord{x}
\end{align*}
\end{lemma}
\begin{proof}
Since $M$ is in $\eta$-nf, all the application to the variable $y$ are total (i.e.~of the form $y P_1 \ldots P_l :o$). Hence after substituting the safe term $N$ for $y$ in $M$, the only possible cause of unsafety is when
some variable free in $N$ becomes not safely bound in $\tau(M)$.
\end{proof}

Applying this lemma with $R= M_f \overline{x}$ gives us a sufficient
condition -- the right-hand side of the equivalence -- for $\lambda
x \phi_1 \dots \phi_k. N_g [(M_f \overline{x}) / y]$ to be safe, and
hence for $f\fatcompos g$ to be P-incrementally justified. Of course
it is not a necessary condition since $N_g[(M_f \overline{x}) /y]$
can be unsafe while its eta-beta normal form is safe.

\subparagraph{A simpler sufficient condition}
\begin{lemma}
If $y:B, \Sigma \vdash N : T$ and $\vdash M : (\overline{A}, B)$
are safe terms with $\ord{A_i} \geq \ord{B}$ for all $i\in 1..n$
then $\overline{x}:\overline{A}, \Sigma \vdash N[(M \overline{x})/y] :T$ is also safe.
\end{lemma}
\begin{proof}
Since $\ord{x_i} = \ord{A_i} \geq \ord{B} = \ord{M \overline{x}}$, we can use the application
rule of the safe lambda calculus to form the safe term $\overline{x}:\overline{A} \vdash M \overline{x}$.
Using the substitution lemma we have that $N[(M \overline{x})/y]$ is safe.
\end{proof}

Hence we obtain the following sufficient condition for $f\fatcompos
g$ to be P-incrementally justified:
$$\ord{A_i}\geq\ord{B} \mbox{ for all } 1 \leq i \leq n$$


Indeed the lemma gives that $\vdash \lambda \overline{x} \phi_1
\dots \phi_k. N_g [(M_f \overline{x}) / y]$ is safe and therefore
its denotation $\sem{\vdash \lambda \overline{x} \phi_1 \dots
\phi_k. N_g [(M_f \overline{x}) / y]} = f\fatcompos g$ is
P-incrementally justified.

Note that this condition is not necessary: Take $A=o$, $B=(o,o)$,
$C=(o,o)$ and consider the two safe terms $M_f = \lambda x^A u^o.u$
and $M_g = \lambda y^B . y a$ for  some constant $a:o$. Then we have
$M_{f\fatcompos g} = \lambda x . a$ which is safe hence $f\fatcompos
g$ is P-incrementally justified although $\ord{A} < \ord{B}$.

\begin{remark}
This result corroborates what we already know about compositionality
of P-i.j.\ strategies (see Sec. \ref{sec:closedpij}). Indeed, the
condition given hereinbefore implies that the strategy $f$ is
\emph{closed} P-i.j.\ (the $A_i$s are prime because we are working
with PCF types) and therefore by Prop.\ \ref{prop:closedpijcompose},
$f \fatcompos g$ must also be P-i.j.
\end{remark}




\paragraph{Counter-example: two P-i.j.\ strategies whose composition is not
P-i.j.}

We now give counter-example to show that P-i.j.\ strategies do not
compose in general.

\subparagraph{First attempt}

Take the types $A=o$, $B=(o,o)$, $C=o$, the variables
$x,u,v:o$, $y:B$ and $\varphi:((o,o),o)$ and $\Sigma$-constant $a:o$.
Consider the two safe terms $\vdash_s  M_f = \lambda xv.x : A\typear B$ and $\vdash_s M_g = \lambda y . \varphi (\lambda u . y a) : B\typear C$.
The $\eta\beta$-nf of $M_{f\fatcompos g}$ is $\vdash \lambda x . \varphi (\underline{\lambda u . x})$ which is unsafe because of the underlined term. It is then tempting to use
Theorem \ref{thm:safeincrejust}(ii) to conclude that
$\sem{M_{f\fatcompos g}}$ is not P-incrementally justified. However this theorem cannot be used here because $M_g$ contains an order $2$ constants ($\varphi$) therefore
$M_{f\fatcompos g}$ is not a valid simply typed $\lambda$-term (nor a \pcf-term).

\subparagraph{Second attempt} The previous example can be easily
changed into a working counter-example: we just need to elevate
$\varphi$ from the status of constant to variable.

Take $A=o$, $B=(o,o)$, $C=(((o,o),o),o)$, the variables
$x,u,v:o$, $y:B$ and $\varphi:((o,o),o)$ and the $\Sigma$-constant $a:o$. Consider the two safe terms $\vdash_s  M_f = \lambda xv.x : A\typear B$ and  $\vdash_s M_g = \lambda y \varphi. \varphi (\lambda u . y a) : B\typear C$.
The $\eta\beta$-nf of $M_{f\fatcompos g}$ is $\vdash \lambda x \varphi. \varphi (\underline{\lambda u . x})$ which is unsafe because of the underlined term, thus by Theorem \ref{thm:safeincrejust}(ii), $\sem{M_{f\fatcompos g}}=\sem{M_f} \fatcompos
\sem{M_g}$ is not P-incrementally justified. The following diagram illustrates a play that is not P-i.j.:
\begingroup
\def\sigcol#1{{\color{gray} #1}}
\def\mucol#1{{\color{red} #1}}
$$\begin{array}{ccccccccc}
A &  & \multicolumn{2}{c}{B} && \multicolumn{4}{c}{C}\\
\cline{1-1} \cline{3-4} \cline{6-9}
o & \stackrel{\sigcol{\sem{M_f}}}\longrightarrow & o, & o & \stackrel{\mucol{\sem{M_g}}}\longrightarrow & ((o, &o),& o),& o \\ \\
&&&&&&&&\rnode{n0}{\lambda x \varphi \omove  \mucol {\lambda y \varphi}}\\
&&&&&&&\rnode{n1}{\varphi  \pmove \mucol \varphi}\\
&&&&&&\rnode{n2}{\lambda u \omove  \mucol {\lambda u}} \\
&&&  \rnode{n3}{\omove \sigcol {\lambda x v} \pmove \mucol y} \\
\rnode{n4}{x \pmove \sigcol x}
\end{array}
\ncarc[arcangleA=20,arcangleB=20,linecolor=black]{->}{n4}{n0}
\ncarc[arcangleA=30,arcangleB=20,linecolor=red]{->}{n2}{n1}
\ncarc[arcangleA=30,arcangleB=20,linecolor=red]{->}{n1}{n0}
\ncarc[arcangleA=20,arcangleB=20,linecolor=red]{->}{n3}{n0}
\ncarc[arcangleA=20,arcangleB=20,linecolor=gray]{->}{n4}{n3}
$$
\endgroup

\subparagraph{Another counter-example with $\ord{B} = \ord{C}$.}

Let $A=o$, $B=C=(((o,o),o),o)$ and let $x:A$, $y:B$, $u:o$, $v,\varphi:((o,o),o)$
and $g:(o,o)$ be variables and  $a:o$ be a $\Sigma$-constant. Take the two safe terms $\vdash  M_f = \lambda x v.x$ and $\vdash M_g = \lambda y \varphi. \varphi (\lambda u . y (\lambda g. a))$.
The $\eta\beta$-nf of $M_{f\fatcompos g}$ is $\vdash \lambda x \varphi. \varphi (\underline{\lambda u . x})$ which is unsafe because of the underlined term, so
$f\fatcompos g$ is not P-incrementally justified.



\chapter{Game-Semantic Models of Safe Languages}
    \label{chap:model}
    \psset{linecolor=darkGreen,linewidth=0.5pt}

\section{Preliminaries}

We consider an arena $A$ and make the following two assumptions on it:
\begin{itemize}
\item (A1) For $A \neq \bot$ (the arena with a single initial question), each question move in the arena enables at least one answer move.
\item (A2) Answer moves do not enable any other move.
\end{itemize}

An arena is said to be \defname{prime} if it has a single initial move; a type is prime if its arena denotation is prime.

\subsection{Node-order}

\subsubsection{Definition}

We define the \defname{order of a move} $m$ in the arena $A$, written $\ord_A{m}$ (or just $\ord{m}$ where there is no ambiguity), as the length of the path from $m$ to its furthest leaf in $A$ minus 1
({\it i.e.}~the height of the subarena rooted at $m$ minus 2.). Because of assumptions (A1) and (A2),
for any move $m$ of $A \neq \bot$, $m$ is a question move if and only if $\ord{m} \geq 0$, and $m$ is an answer move if and only if $\ord{m} = -1$.

The \defname{order of an arena} $A$ is defined to be the maximal order of its initial moves. The order of a (simple, PCF or IA) type is defined as the order of the arena denoting it - or equivalently as 0 for ground type, $\ord{A\rightarrow B} = \max(1+\ord{A},\ord{B})$ and $\ord(A\times B) = \max(\ord A, \ord B)$. The order of a term is the order of its type.


\subsubsection{Node-order after composition}

Consider the arena $X\lingamear Y$ and let $m$ be a move of
$X\lingamear Y$. We write $\ord_{X\lingamear Y}{m}$ to denote the
order of $m$ in the arena ${X\lingamear Y}$. If $m$ belongs to $X$
(resp.~$Y$) then we write $\ord_X{m}$ (resp.~$\ord_Y{m}$) to denote
the order of the move $m$ in the arena $X$ (resp.~$Y$).

\begin{lemma}
\label{lem:compositionorder} Let $A$, $B$ and $C$ be three arenas.
We have:
$$\begin{array}{lll}
\forall m \in A:
    &  \ord_{A\lingamear B}{m} = \ord_{A\lingamear C}{m} \ ,\\
\forall m \in B:
    & \ord_{A\lingamear B}{m} \geq \ord_{B\lingamear C}{m}  & \mbox{for $m$ initial,}\\
    & \ord_{A\lingamear B}{m} = \ord_{B\lingamear C}{m} & \mbox{for $m$ non initial,} \\
\forall m \in C:
    & \ord_{A\lingamear C}{m} \geq \ord_{B\lingamear C}{m} \iff
\ord{A} \geq \ord{B}\ & \mbox{for $m$ initial,}\\
    & \ord_{A\lingamear C}{m} = \ord_{B\lingamear C}{m}   & \mbox{for $m$ non initial.}
\end{array}
$$
\end{lemma}

\subsection{Well-bracketing}

We call \defname{pending question} of a sequence of moves $s \in L_A$ the last unanswered question in $s$.

\begin{definition}\rm
A strategy $\sigma$ is said to be \defname{P-well-bracketed} if for any play $s \, a \in \sigma$ where $a$ is a  P-answer, $a$ points to the pending question in $s$.
\end{definition}



P-well-bracketing can be restated differently as the following proposition shows:
\begin{proposition}
\label{prop:char_wellbrack}
\rm We make assumption (A1) and (A2).
Let $\sigma$ be a strategy on an arena $A\neq \bot$.
The following statements are equivalent:
\begin{enumerate}
\item[(i)] $\sigma$ is P-well-bracketed,
\item[(ii)] for $s \, a \in \sigma$ with $a$ a P-answer, $a$ points to the pending question in $\pview{s}$,
\item[(iii)] for $s \, a \in \sigma$ with $a$ a P-answer, $a$ points to the last O-question in $\pview{s}$,
\item[(iv)] for $s \, a \in \sigma$ with $a$ a P-answer, $a$ points to the last O-move in $\pview{s}$ with order $>\ord{a}$.
\end{enumerate}
\end{proposition}
\begin{proof}
$(i)\iff(ii)$: \cite[Lemma 2.1]{McC96b} states that if P is to move then the pending question in $s$ is the same as that of $\pview{s}$.

$(ii)\iff(iii)$: Assumption (A2) implies that the pending question in $\pview{s}$ is also the last O-question occurring in $\pview{s}$.

$(iii)\iff(iv)$: Because of assumption (A1) and (A2),
for any move $m$, we have $m$ is a question move
if and only if $\ord{m} \geq 0$ if and only if $\ord{m} > \ord{a} = -1$.
\end{proof}




\begin{lemma}
\label{lem:justfied_by_unanswered}
Under assumption (A2), if $s$ be a justified sequence of moves satisfying alternation and visibility then any O-move (resp. P-move) in $s$ points to an \emph{unanswered} P question (resp. O-question).
\end{lemma}
\begin{proof}
Suppose that an O-move $c$ points to a P-move $d$ that has already been answered by the O-move $a$. The sequence $s$ as the following form:
$$ s= \ldots \Pstr{(d){d}  \ldots  (a-d,20){a}  \ldots  (c-d,20){c}}$$

By O-visibility, $d$ must belong to $\oview{s_{<c}}$. But since $a$ is an answer, by assumption (A2), it cannot justify any P-move, therefore
$\oview{s_{<q}}$ must contain an OP-arc ``hoping'' over $a$. We name the nodes of this arc $d^1$ and $c^1$:
$$ s = \ldots \Pstr[0.7cm]{(d){d}  \ldots  (d1){d^1} \ldots (a-d,20){a} \ldots
 (c1-d1,20){c^1} \ldots (c-d,25){c}}$$

By P-visibility, $d^1$ must belong to $\pview{s_{<c^1}}$. Consequently, $a$ does not belong to $\pview{s_{<c^1}}$ (otherwise the PO-arc $\Pstr[0.5cm]{(d){d} \quad (a-d,45){a}}$ would cause the P-view to jump over $d^1$).
Therefore there must be a PO-arc $\Pstr[0.5cm]{(d2){d^2} \quad (c2-d2,45){c^2}}$ in $\pview{s_{<c^1}}$ hoping over $a$:
$$ s = \ldots \Pstr[0.7cm]{(d){d}  \ldots
(d1){d^1} \ldots (d2){c^2} \ldots
(a-d,20){a} \ldots
 (c2-d2,20){d^2} \ldots (c1-d1,20){c^1} \ldots (c-d,25){c}}$$

This process can be repeated infinitely often by using alternatively O-visibility and P-visibility. This gives a contradiction since the sequence of moves $s_{<c}$ has finite length.
Hence $d$ cannot point to a question that has already been answered. Since, by assumption (A2), a question is enabled by another question, $d$ is necessarily justified by an unanswered question.
\end{proof}


\begin{lemma}
\label{lem:oq_in_pview_unanswered}
Under assumption (A2), if $s$ is a P-well-bracketed justified sequence of moves of odd length satisfying alternation and visibility then  all O-questions occurring in $\pview{s}$ are unanswered in $s$.
\end{lemma}
\begin{proof}
We proof the first part by induction on $s$.
The base case ($s = q$ with $q$ initial O-move) is trivial.

Suppose $\Pstr[0.4cm]{ s = s' \cdot (n)n \cdot u \cdot (m-n,45){m} }$.
Let $r$ be an O-question in $\pview{s} = \pview{s'} \cdot n \cdot m$.
If $r$ is the last move $m$ then it is necessarily unanswered.
If $r \in \pview{s'}$ then by the induction hypothesis, $r$ is unanswered in $s'$.
Suppose that $r$ is answered in $s$. This implies that some answer move $a$ in $u$ points to $r$:
$$\pstr[0.7cm][5pt]{ s = \underbrace{\cdots\ \nd(r){r}^O \cdots }_{s'} \
\nd(n){n}^P \ \underbrace{\cdots\ \nd(a-r,35){a}^P \cdots }_{u} \
\nd(m-n,30){m}^O } \ .$$

Since $m$ points to $n$, by lemma \ref{lem:justfied_by_unanswered}, $n$ is still unanswered at $s_{\prefixof a}$. Therefore the pending
question at $s_{\prefixof a}$ cannot be $r$. But $a$ is justified by $r$, therefore the well-bracketing condition is violated. Hence $r$ is
unanswered in $s$.
\end{proof}








\subsection{Interaction sequences} Let us first recall the
definition of an interaction sequence. Let $A$,$B$ and $C$ be three
games. We say that $u$  is an
\defname{interaction sequence} of $A$,$B$ and $C$ whenever $u\filter
A,B$ is a valid position of the game $A\lingamear B$ (i.e.~$u\filter
A,B \in P_{A\lingamear B}$) and  $u\filter B,C$ is a valid position
of the game $B\lingamear C$. We write $Int(A,B,C)$ to denote the set
of all such interaction sequences.

Let $\sigma:A\lingamear B$ and $\mu:B\lingamear C$ be two
strategies. We write $\sigma \parallel \mu$ to denote the set of
interaction sequences that unfold according to the strategy $\sigma$
in the $A,B$-projection of the game and to $\mu$ in the
$B,C$-projection:
$$ \sigma \parallel \mu = \{ u \in Int(A,B,C) \ | \ u\filter A,B \in \sigma \wedge u \filter B,C \in \mu \} \ .$$
The composite of $\sigma$ and $\mu$ is then defined as $\sigma ; \mu
= \{ u \filter A,C \ | \ u \in \sigma \parallel \tau \}$.

The diagram below shows the structure of an interaction sequence
from $\sigma \parallel \mu$. There are four states represented by
the rectangular boxes. The content of the state shows who is to play
in each of the game $A\lingamear B$, $B\lingamear C$ and
$A\lingamear C$. For instance in state $OPP$, it is O's turn to play
in $A\lingamear B$ and P's turn to play in $B\lingamear C$ and
$A\lingamear C$. Arrows represent the moves. When specifying
interaction sequence, the following bullet symbols are used to
represent moves: $\pmove$ for P-moves, $\omove$ for O-moves,
$\pomove$ for a move playing the role of P in $A\lingamear B$ and O
in $B\lingamear C$ and $\opmove$ for the symmetric of $\pomove$. We
sometimes add a subscript to the symbols $\pmove$ and $\omove$ to
denote the component in which the moves is played ($A$ or $C$).


\tikzstyle{state}=[rectangle,draw=blue!50,fill=blue!20,thick,minimum
height = 4ex, text width=4cm] \tikzstyle{move}=[->,shorten
<=1pt,>=latex',line width=1pt] \tikzstyle{intmove}=[dashed]
\tikzstyle{extomove}=[color=\extomovecolor]
\tikzstyle{genomove}=[]%[dashed]
\tikzstyle{genpmove}=[color=\genpmovecolor]
\def\sep{1.5cm}
\begin{figure}[htbp]
\begin{center}
\begin{tikzpicture}[node distance=1.7cm]

% the four states
\path
 node(oooT)  [state] {}
 node(opp)   [state, below of=oooT] {}
 node(pop)   [state, below of=opp]  {}
 node(oooB)  [state, below of=pop] {}
 node(title) [anchor=south, at=(oooT.north), minimum height = 4ex, text width=4cm] { };

\path
% text in the title centered in 3 columns
  ([xshift=-\sep]title) node {$A\lingamear B$}
        (title) node {$B\lingamear C$}
        ([xshift=\sep]title) node {$A\lingamear C$}

% text in the states centered in 3 columns
  ([xshift=-\sep]oooT) node {O}
        (oooT) node {O}
        ([xshift=\sep]oooT) node {O}
  ([xshift=-\sep]opp) node {O}
        (opp) node {P}
        ([xshift=\sep]opp) node {P}
  ([xshift=-\sep]pop) node {P}
        (pop) node {O}
        ([xshift=\sep]pop) node {P}
  ([xshift=-\sep]oooB) node {O}
        (oooB) node {O}
        ([xshift=\sep]oooB) node {O}

% text in between two arrows giving the arena of the move
  (oooT) to node {\bf C} (opp)
  (opp) to node {\bf B} (pop)
  (pop) to node {\bf A} (oooB)

% arrows representing the moves
  (opp.20)    edge[move, genpmove]
        node[right] {$\mu$}
        node[left]{$\pmove$} (oooT.-20)
  (oooT.-160) edge[move, extomove, genomove]
        node[left] {$env_\mu$}
        node[right]{$\omove$} (opp.160)
  (pop.20)    edge[move, genomove,genpmove,intmove]
        node[right] {$\sigma$}
        node[left]{$\pomove$} (opp.-20)
  (opp.-160)  edge[move, genomove, genpmove,intmove]
        node[left] {$\mu$}
        node[right]{$\opmove$}  (pop.160)
  (oooB.20)   edge[move, extomove,genomove]
        node[right] {$env_\sigma$}
        node[left]{$\omove$} (pop.-20)
  (pop.-160)  edge[move, genpmove]
        node[left] {$\sigma$}
        node[right]{$\pmove$} (oooB.160);

%\draw[move, genpmove] (3.5cm,-1cm) -- +(1,0) node[right] {Generalised P-move \& External P-move };
%\draw[move, genomove,genpmove] (3.5cm,-2cm) -- +(1,0) node[right] {Generalised O-move \& Generalised P-move};
%\draw[move, genomove,extomove] (3.5cm,-3cm) -- +(1,0) node[right] {Generalised O-move \& External O-move};
\draw[move] (3.5cm,-1cm) -- +(1cm,0cm) node[right] {External move};
\draw[move,intmove] (3.5cm,-2cm) -- +(1cm,0cm) node[right] {Internal
move}; \draw (3.5cm,-3cm) node[anchor=west]
{\textcolor{\extomovecolor}{External O-moves: $\omove$}}; \draw
(3.5cm,-4cm) node[anchor=west]
{\textcolor{\genpmovecolor}Generalised P-move: $\opmove, \pomove,
\pmove$};
\end{tikzpicture}
\end{center}
\caption{Structure of an interaction sequence.} \label{fig:interseq}
\end{figure}

Note that in state OPP, the alternation condition (for each of the
three games involved) prevents the players from playing in A.
Indeed, the O-moves in component $A$ of $A\lingamear B$ are also
$O$-moves in component $A$ of $A\lingamear C$ however the state name
indicates that the next move in $A\lingamear B$ must be an O-move
and the next move in $A\lingamear C$ must be a P-move.

Similarly, in the top state OOO, the players cannot make move in B
since the O-moves in component B of the game $B\lingamear C$
correspond to P-moves in the component B of $A\lingamear B$. However
the state name indicates that the next move in $A\lingamear B$ and
the next move in $B\lingamear C$ must be played by O.


Let $u \in Int(A,B,C)$ and $m$ be a move of $u$. The
\defname{component} of $m$ is $A,B$ if after playing $m$ the game is
under the control of the strategy $\sigma$ and $B,C$ otherwise (if
$\mu$ has control). In other words, the moves $\omove, \pmove \in A$
and $\opmove \in B$ shown on the diagram of Figure
\ref{fig:interseq} have component $A,B$ and $\omove, \pmove \in C$
and $\pomove \in B$ have component $B,C$.


Also we call \defname{generalized O-move in component $A,B$} moves
that play the role of O in the game $A\lingamear B$, that is to say
moves represented by $\opmove$ and $\omove_A$. Similarly $\pomove$
and $\pmove_A$ moves are the \defname{generalized P-moves in
component $A,B$}, $\omove_C$ and $\pomove$ moves are the
\defname{generalized O-moves in component $B,C$} and  $\pmove_C$ and
$\opmove$ moves are the \defname{generalized P-moves in component
$B,C$}.

The P-view (also called \emph{core} in
\cite{McCusker-GamesandFullAbstrac}) of an interaction sequence $u
\in Int(A,B,C)$, written $\overline{u}$ or $\pview{u}$ is defined
as:
\begin{align*}
\pview{u\cdot \extomove{n}} &= \extomove{n} &
\mbox{ if \extomove{$m$} is an \extomove{external O-move} initial in C,}\\
\pview{\Pstr{u\cdot (m)m\cdot v \cdot (n-m,45){\extomove{n}} }} &= \extomove{n} &\mbox{ if \extomove{$m$} is an \extomove{external O-move} non initial in C,}\\
\pview{u \cdot \genpmove{m}} &= \pview{u}\cdot \genpmove{m}  & \mbox{ if \genpmove{$m$} is a \genpmove{generalised P-move}.}\\
\end{align*}

We can show the following property by an easy induction :
\begin{lemma}
\label{lem:pviewAC_eq_ACpview}
 Let $u$ be an interaction sequence in $Int(A,B,C)$ then
$$\pview{u} \filter A,C = \pview{u \filter A,C} \ .$$
\end{lemma}
\begin{proof}
  By induction on $u$. It is trivial for the empty sequence.
Let $b$ be a move in $B$. We have $\pview{u b} \filter A,C =
\pview{u} \filter A,C$. By the I.H.\ this is equal to $\pview{u
\filter A,C} = \pview{u b\filter A,C}$. Let $m$ be a P-move in $A$
or $C$ then $\pview{u m} \filter A,C = (\pview{u} \filter A,C) m$
and by the I.H.\ this is equal to $\pview{u \filter A,C} m =
\pview{(u \filter A,C) m} = \pview{u m \filter A,C}$. Let $c$ be an
initial move in $C$. We have $\pview{u c \filter A,C}  = \pview{(u
\filter A,C) c} = c =  c \filter A,C = \pview{u c} \filter A,C$. Let
$u = \Pstr{u_1 (m){m} u_2 (n-m){n}}$ with $n$ an O-move in
$A\rightarrow C$. Then necessarily $m\in A,C$ and $ \pview{u\filter
A,C} = \pview{\Pstr[0.5cm]{u_1\filter A,C \cdot (m){m} \cdot
u_2\filter A,C \cdot (n-m,30){n}}} =
 \pview{u_1 \filter A,C} \Pstr{(m){m} (n-m){n}}$. By the I.H.\ this is equal to
$(\pview{u_1}\filter A,C) \Pstr{(m){m} (n-m){n}} = (\pview{u_1}
\Pstr{(m){m} (n-m){n}} ) \filter A,C  = \pview{u_1 \Pstr{(m){m} u_2
(n-m){n}}} \filter A,C$
\end{proof}


\subsection{P-incremental justification}


\begin{definition}\rm
A play $s m$ of even length is said to be \defname{P-incrementally
justified}, or \emph{P-i.j.} for short, if $m$ points to the last
unanswered O-question in $\pview{s}$ with order strictly greater
than $\ord{m}$.

 A strategy $\sigma$ is said to be \defname{P-incrementally justified}, if all plays in $\sigma$ ending with a P-question are
P-incrementally justified.
\end{definition}
Let $\sigma$ be a strategy. We write $Pij(\sigma)$ to denote the set of plays of $\sigma$ that are P-i.j.
We can define equivalently P-i.j.\ strategies as those verifying the relation $\sigma = Pij(\sigma)$.
\begin{proposition}
\label{prop:char_pincr}
\rm We make assumption (A1) and (A2).
Let $\sigma$ be a \emph{P-well-bracketed} strategy on an arena $A\neq \bot$.
The following statements are equivalent:
\begin{enumerate}
\item[(i)] $\sigma$ is P-incrementally justified,
\item[(ii)] for $s \, q \in \sigma$ with $q$ a P-question, $q$ points to the last O-question in $\pview{s}$ with order $>\ord{q}$,
\item[(iii)] for $s \, q \in \sigma$ with $q$ a P-question, $q$ points to the last O-move in $\pview{s}$ with order $>\ord{q}$.
\end{enumerate}
\end{proposition}
\begin{proof}
$(i)\iff(ii)$: By lemma \ref{lem:oq_in_pview_unanswered}, O-question occurring in $\pview{s}$ are all unanswered.

$(ii)\iff(iii)$: Because of (A1) and (A2), $\ord{q} \geq 0$ thus an O-move with order $>\ord{q}$ is necessarily an O-question.
\end{proof}

Putting proposition \ref{prop:char_pincr} and
\ref{prop:char_wellbrack} together we obtain:
\begin{proposition}
Under assumption (A1) and (A2).
A strategy $\sigma$ on $A\neq \bot$
is \emph{P-well-bracketed} and
 \emph{P-incrementally justified} if and only if
for $s \, m \in \sigma$, $m$ points to the last O-move in $\pview{s}$ with order $>\ord{m}$.
\end{proposition}




\section{Closed P-i.j.\ strategies}
\label{sec:closedpij}

\subsection{Definition}

\begin{definition}
\label{def:closedpij} Let $s m$ be an even-length play on some game
$A \rightarrow B$. $s m$ is said to be
\defname{closed P-incrementally justified} (closed P-i.j.\ for short)
just if
\begin{itemize}
\item $s m$ is P-incrementally justified;
\item and if $m$ is an initial move in $A$ then its justifier $n$ (initial in
$B$) verifies $\ord_A m \geq \ord_B n$.
\end{itemize}

\noindent A strategy $\sigma$ is \defname{closed P-i.j.} just if all
plays in $\sigma$ ending with a P-questions are closed P-i.j.
\end{definition}
An example of closed P-i.j.\ strategy is the identity strategy $id_A$
for any game $A$.

\begin{lemma}
\label{lem:closedpij_singleBinitmove} Let $\sigma : A \lingamear B$
be a P-i.j.\ strategy.
\begin{enumerate}[i.]
\item If for each initial move $m$ of $A$ occurring in some play of $\sigma$ we have $\ord_A m \geq \ord{B}$, then $\sigma$ is closed P-i.j.
\item Suppose that $A=A_1\times \ldots \times A_n$ where each of the $A_i$ are prime arenas. If for each initial move $m_i$ of $A_i$, for $i \in \{1..n\}$, occurring in some play of $\sigma$ we have $\ord A_i \geq \ord{B}$, then $\sigma$ is closed P-i.j.
\end{enumerate}
\end{lemma}
\begin{proof}
(i) This is a direct consequence of the definition since $\ord B \geq \ord_B b$ for every move $b$ initial in $B$.

(ii) Take an initial move $m$ of $A$. It is necessary an initial move of $A_i$ for some $i$ hence $\ord_A m = \ord_{A_i} m$ which is equal to $\ord A_i$ since $A_i$ is prime. By hypothesis this is in turn greater than $\ord{B}$ hence we can conclude using (i).
\end{proof}



We observe that every P-i.j.\ strategy $\sigma$ on the game $I
\lingamear A$ is closed P-i.j.\ while $\sigma : A$ is not
necessarily closed P-i.j.\footnote{In particular, every P-i.j.\
strategy $\sigma$ on the game $!A_1 \otimes \ldots \otimes !A_n
\lingamear B$, is isomorphic, up to arena-tagging of the moves, to
the closed P-i.j.\ strategy $\Lambda^n(\sigma)$ on the game $I
\lingamear (A_1,\ldots,A_n,B)$, where $\Lambda$ denotes the usual
{\it currying} isomorphism.}; hence the distinction between $I
\lingamear A$ and $A$ matters. This is because the definition of
closed P-i.j.\ strategy specifically refers to the moves of  the
arena in the left-hand side of the function space arrow
$\lingamear$, therefore the property is not valid up to an
isomorphism that retags the moves such as {\it currying}.

Consequently, it is possible to have two isomorphic strategies $\sigma$ and
$\mu$ such that one is closed P-i.j.\ but not the other. In contrast, the ``ordinary'' P-incremental
justification condition is preserved across the  {\it curry} isomorphism. A consequence of this remark is that the category of closed P-i.j.\ strategies
that we will introduce later on, is not closed (neither monoidal closed nor cartesian closed) and
that it only admits a weak form of {\it curry} isomorphism.

\subsection{Compositionality - A semantic proof}

{\bf Notation} In plays representations, the symbol $\omove$ stands
for an O-move and $\pmove$ for a P-move. Suppose the game considered
is $L\lingamear R$ for some game $L$ and $R$ then whenever the
sub-arena in which the move is played is known, it is specified in
subscripts ($\omove_L$, $\pmove_L$, $\omove_R$ or $\pmove_R$). For
interaction sequences in $Int(A,B,C)$ we use the symbols $\omove_A$,
$\pmove_A$, $\omove_C$, $\pmove_C$, $\opmove$ and $\pomove$ as
defined in Figure \ref{fig:interseq}. We use the variable $X$ to
denote one of the component $A,B$ or $B,C$, the variable  $Y$ then
denotes the other component. We write $s \subseqof t$ to say that
$s$ is a subsequence (with pointers) of $t$, $s \prefixof t$ to say
that $s$ is a prefix (with pointers) of $t$ and  $s \suffixof t$ to
say that $s$ is a suffix of $t$.

We now prove several useful lemmas which will become useful when studying compositionality of P-i.j.\ strategies.

\begin{lemma}
\label{lem:interjump}
Let $X$ be a component (either  $A,B$ or  $B,C$).
Let $u$ be an interaction sequence of the form
$ u =
\Pstr[0.5cm][2pt]{ \ldots (b){\stk \beta \pmove}  \ldots
 {n}  \ldots  (a-b,30){\stk \alpha\omove}
\ldots m}$ where:
\begin{itemize}[-]
\item $\alpha,\beta$ are external moves in component $X$ (necessarily both played in $A$ or in $C$),
\item  $m$ is either played in $B$ or an external P-move in $X$,
\item  $\alpha$ is visible at $m$ in $X$ \emph{i.e.}~$\alpha\in \pview{u \filter X}$ (consequently $\beta$ is also visible).
\end{itemize}
Then $n \not\in \pview{u \filter A, C}$.
\end{lemma}
\begin{proof}
Since $\alpha$ is an O-move, $\alpha$ and $\beta$ are necessarily
played in the same arena ($A$ or $C$). Take $v=u$ if $m$ is a
generalized O-move in $X$ and $v=u_{<z}$ otherwise (if $m$ is a
generalized P-move in $X$). The third assumption implies
$\alpha,\beta\in \pview{v}$. The last move in $v$ is necessarily a
generalized O-move in component $X$ (see diagram of Figure
\ref{fig:interseq}) therefore by \cite[Lemma 3.3.1]{Harmer2005} we
have $\pview{v \filter X} = \pview{\overline{v} \filter X} \subseqof
\overline{v} \subseqof \overline{u}$. Thus $\alpha,\beta \in
\overline{u}$ and since $\alpha,\beta$ are played in $A,C$ we have
$\alpha,\beta  \in \overline{u} \filter A,C = \pview{u
\filter A,C}$ (Lemma \ref{lem:pviewAC_eq_ACpview}). Finally
since $n$ lies underneath the $\beta$-$\alpha$ PO-arc it cannot
appear in the P-view  $\pview{u \filter A,C}$.
\end{proof}

\begin{lemma}
\label{lem:in_pviewAC_imp_in_pviewX}
Let $u$ be an interaction sequence in $Int(A,B,C)$ and
$n$ be a move of $u$ such that $n\in\pview{u \filter A,C}$:
\begin{enumerate}[i.]
\item
if all the moves in $u_{\suffixof n}$
are played in $C$  then $n \in \pview{u \filter B,C}$;
\item
if all the moves in $u_{\suffixof n}$ are played in $A$ then $n \in \pview{u \filter A,B}$.
\end{enumerate}
\end{lemma}
\begin{proof}
\begin{enumerate}[(i)]
\item
We show the contrapositive. Suppose that $n \not\in\pview{u \filter B,C}$. This must be due to one of the following  two
reasons:
\begin{itemize}[-]
\item $\pview{u \filter B,C}$ contains an initial move $c_0 \in C$
occurring after $n$ in $u$.


By \cite[Lemma 3.3.1]{Harmer2005}
we have $\pview{u \filter B,C} = \pview{\overline{u} \filter B,C} \subseqof \pview{u}$, thus $c_0$ also occurs in $\pview{u}$.
Since $c_0$ belongs to $C$ we have
$c_0 \in \pview{u} \filter A,C=
\pview{u \filter A,C}$ (Lemma \ref{lem:pviewAC_eq_ACpview}).
Thus the P-view $\pview{u \filter A,C}$
starts with the initial move $c_0$ and
since $n$ occurs before $c_0$, $n$ does not occur in the P-view.

\item $n$ lies underneath a PO-arc $\beta$-$\alpha$ visible
at $ u \filter B,C$.
By assumption, since $\alpha$ occurs after $n$ in $u$, it must belong to $C$. We can therefore apply Lemma \ref{lem:interjump}
with $X\assignar B,C$ which gives
$n \not\in\pview{u \filter A,C}$.
\end{itemize}

\item Suppose that $n \not\in\pview{u \filter A,B}$ then either:
\begin{itemize}[-]
\item $\pview{u \filter A,B}$ contains an initial move $b_0 \in B$
occurring after $n$ in $u$. But this is impossible since by assumption all the moves occurring after $n$ in $u$ belong to $A$.

\item or $n$ lies underneath a PO-arc $\beta$-$\alpha$ in $A,B$.
By assumption, since $\alpha$ occurs after $n$ it must belong to $A$. We can then conclude using
Lemma \ref{lem:interjump} with $X\assignar A,B$.
\end{itemize}
\end{enumerate}
\end{proof}

Note that we cannot completely relax the assumption
which says that moves in $u_{\suffixof n}$ are all in the same component.
For instance take $u = \Pstr[0.5cm]{(co){\omove_C}\thinspace
(b0-co){\opmove} \thinspace
(n){\stk{\pmove_A}{n}} \thinspace
(b1-co){\opmove}}$ then we have $n\in\pview{u\filter A,C}$ but $n\notin\pview{u\filter A,B}$.


%%%%%%%%%%%
% This commented Lemma could be useful be we did not make use of it eventually.
%
% \begin{lemma}
%\label{lem:oviewsegmentinB}
%For any legal sequence $s = \ldots x \cdot r \cdot y$ of a game $A\lingamear B$ if $x, y \in A$ and $x$ is O-visible from $y$ then any move in $r$ occurring in $\oview{s}$ belongs to $A$.
%\end{lemma}
%\begin{proof}
%We proceed by induction on the length of the segment $r$.
%Base case $r=\epsilon$ is trivial. Suppose $r = r' \cdot m$.
%If $y$ is an O-move then by the Switching Condition
%$m$ is necessarily in $A$. Clearly $x$ is O-visible from $m$ thus  by the I.H.\ any move from $r$ occurring in the O-view is in $A$.
%
%If $y$ is a P-move then it cannot point to an initial move in $B$. Indeed, suppose that it points to an initial O-move $b_0 \in B$ then
%we have $\oview{s} = b_0 \cdot y$ which contradicts the fact that $x\in \oview{s}$.
%Thus $y$ points to a move in $A$ and again we can conclude using the induction hypothesis.
%\end{proof}


\begin{lemma}[P-visibility decomposition (from $C$)]
\label{lem:middlepomove}
Let $u = \ldots n' \cdot r \cdot m \in Int(A,B,C)$ where
$n'$ is a $\omove_A$-move verifying $n' \in \pview{u\filter A,C}$ and $m$ is in $\{ \pmove_C, \opmove, \pomove \}$. Then there is a $\pomove$-move $\gamma$ in $r \cdot m$ such that $\gamma \in \pview{u\filter B,C}$ , $n' \in \pview{u_{\leq \gamma} \filter A,B}$ and $\gamma$ is justified by a move occurring before $n'$.
\end{lemma}
\begin{proof}
By induction on $|r|$.
If $r=\epsilon$ then necessarily $u = \ldots \stk{\omove_A}{n'} \thinspace\stk \pomove m$ where $m$ points before $n'$ ($n'$ being played in $A$ cannot justify $m$ played in $B$) so we just need to take $\gamma = m$.
If $|r|=1$ then either
$u = \ldots \stk{\omove_A}{n'} \pomove\thinspace\stk {\pmove_C} m$
or $u = \ldots \stk{\omove_A}{n'} \pomove\thinspace\stk \opmove m$.
In both cases we can take $\gamma$ to be the $\pomove$-move between $n'$ and $m$.
Suppose $|r|>1$. Let $m^-$ denote the move preceding $m$ in $u$.
We proceed by case analysis:
\begin{enumerate}[i.]
\item Suppose $m = \pmove_C$ and $m^- = \omove_C$.
Let $q$ be the external P-move that justifies $m^-$.
Since $n' \in \pview{u\filter A,C}$, $q$ must occur after $n'$ in $u$:
$$
\begin{array}{ccccl}
A & \stackrel\sigma{\longrightarrow} & B & \stackrel\mu{\longrightarrow} & C \\
&\vdots&&\vdots\\
n' \omove\\
&\vdots&&\vdots  \\
&& & &  \rnode{q}{\pmove}q  \\
&\vdots&&\vdots  \\
&& & &  \rnode{mp}{\omove}m^-  \\
&& & &  \rnode{m}{\pmove}m  \\
\end{array}
\ncarc[arcangleA=60,arcangleB=60]{->}{mp}{q}
 $$
Thus we can use the induction hypothesis (with $u\assignar u_{\prefixof q}$): there is a $\pomove$-move $\gamma$
in $u_{]n',q]}$ pointing before $n'$ such that $\gamma \in \pview{u_{\prefixof q} \filter B,C}$, $n' \in \pview{u_{\prefixof \gamma} \filter A,B}$.
Moreover $\pview{u_{\prefixof q} \filter B,C} \prefixof \pview{u_{\prefixof m} \filter B,C}$ (since $q$ is visible from $m$ in $B,C$) thus we have $\gamma \in \pview{u_{\prefixof m} \filter B,C}$ as required.

\item Suppose $m = \pmove_C$ and $m^- = \pomove \in B$.
Again we can conclude using
the induction hypothesis with $u \assignar u_{\prefixof m^-}$.

\item Suppose $m = \pomove \in B$.

Suppose that all the moves in $r$ are in $A$.
Then $r$ is of the form $(\pmove_A \omove_A)^*$ (where $(\cdot)^*$ denotes the Kleenee star operator).
We just need to take $\gamma = m$.
Indeed, moves in $u_{\suffixof m}$ are all in $A$
and by assumption $n'\in\pview{u\filter A,C}$  therefore
Lemma \ref{lem:in_pviewAC_imp_in_pviewX}(ii) gives
$n'\in\pview{u\filter A,B}$.
Also, since $m$ is a $\pomove$-move,
its justifier is a $\opmove$-move but $r$ contains only $\omove$ and $\pmove$ moves hence $m$'s justifier must occur before $n'$.

Suppose that $r$ contains at least one move in $B$. Let $b$ be the last such move, then $u$ is of the form $\ldots n' \cdot \ldots \cdot \stk\opmove  b \cdot (\pmove_A \omove_A)^* \cdot\thinspace\stk\pomove m $. We then have
$u\filter B,C = \ldots n' \cdot \ldots \cdot
\thinspace\stk\opmove b \thinspace\cdot \stk\pomove m $ thus $b \in \pview{u\filter B,C}$. We can then conclude by applying the induction hypothesis with $u \assignar u_{\prefixof b}$.

\item Suppose $m = \pomove \in B$.
If $m^- = \opmove \in B$ then the I.H.\ with $u \assignar u_{\prefixof m^-}$ permits us to conclude.
If $m^- = \omove \in C$ then we conlude by applying  the I.H.\ on $u \assignar u_{\prefixof q}$ where $q$ is the external P-move in $C$ justifying
$m^-$.
\end{enumerate}
\end{proof}

We now show the lemma symmetric to the previous one:
\begin{lemma}[P-visibility decomposition (from $A$)]
\label{lem:middleopmove}
Let $u = \ldots n' \cdot r \cdot m \in Int(A,B,C)$ where
$n'$ is an O-move \emph{non initial} in $C$ verifying $n' \in \pview{u\filter A,C}$ and $m$ is in $\{\pmove_A, \opmove, \pomove\}$. Then there is a $\opmove$-move $\gamma$ in $r \cdot m$ such that $\gamma \in \pview{u\filter A,B}$ , $n' \in \pview{u_{\leq \gamma} \filter B,C}$ and $\gamma$ is justified by a move occurring before $n'$.
\end{lemma}
\begin{proof}
The proof is almost symmetrical to the previous one (Lemma \ref{lem:middlepomove}). We proceed by induction on $|r|$.
If $r=\epsilon$ then necessarily $u = \ldots \stk {\omove_C} {n'} \thinspace\stk \opmove m$ where $m$ points before $n'$ (it cannot point to $n'$
since $n'$ is not initial in $C$). Thus we just need to take $\gamma = m$.

If $|r|=1$ then either
$u = \ldots \stk {\omove_C} {n'} \thinspace\opmove\thinspace\thinspace\stk{\pmove_A} m$
or $u = \ldots \stk {\omove_C} {n'} \thinspace\opmove\thinspace\thinspace\stk \pomove m$.
In both cases we can take $\gamma$ to be the $\opmove$-move between $n'$ and $m$.
Suppose $|r|>1$. Let $m^-$ denote the move preceding $m$ in $u$.
We do a case analysis:
\begin{enumerate}[i.]
\item Suppose $m = \pmove_A$ and $m^- = \omove_A$.
Let $q$ be the external P-move that justifies $m^-$.
Since $n' \in \pview{u\filter A,C}$, $q$ must occur after $n'$ in $u$:
$$
\begin{array}{rcccl}
A & \stackrel\sigma{\longrightarrow} & B & \stackrel\mu{\longrightarrow} & C \\
&\vdots&&\vdots\\
&&&& \omove\ n'\\
&\vdots&&\vdots  \\
q\rnode{q}{\pmove}  \\
&\vdots&&\vdots  \\
m^- \rnode{mp}{\omove}  \\
m \rnode{m}{\pmove}  \\
\end{array}
\ncarc[arcangleA=-45,arcangleB=-45]{->}{mp}{q}
 $$
Thus we can use the induction hypothesis (with $u\assignar u_{\prefixof q}$): there is a $\opmove$-move $\gamma$
in $u_{]n',q]}$ pointing before $n'$ such that $\gamma \in \pview{u_{\prefixof q} \filter A,B}$, $n' \in \pview{u_{\prefixof \gamma} \filter B,C}$.
Moreover $\pview{u_{\prefixof q} \filter A,B} \prefixof \pview{u_{\prefixof m} \filter A,B}$ (since $q$ is visible from $m$ in $A,B$) thus we have $\gamma \in \pview{u_{\prefixof m} \filter A,B}$ as required.

\item Suppose $m = \pmove_A$ and $m^- = \pomove$ then again we can conclude using the I.H.\ with $u \assignar u_{\prefixof m^-}$.

\item Suppose $m = \opmove$.
\begin{itemize}[-]
\item Suppose that $r$ does not contain any move in $B$  then $r$ is of the form $(\pmove_C \omove_C)^*$.

We just need to take $\gamma = m$.
Indeed:
\begin{enumerate}
\item By lemmma \ref{lem:in_pviewAC_imp_in_pviewX}(i)
we have $n'\in \pview{u\filter B,C}$.

\item  $m$ is justified by a move occurring before $n'$.
Indeed, if $m$ is justified by a $\pomove$-move then since $n' \cdot r$ contains only $\omove$ and $\pmove$ moves, $m$'s justifier must occur before $n'$.
If $m$'s justifier is an initial $\omove_C$-move $c_i$, then
by P-visibility we have $c_i \in \pview{u\filter B,C}$
but since the P-view computation ``stops'' when reaching an initial moves, in order to guarantee that $n'$ also belongs to the P-view (as shown in (a)) it must
occurs after $c_i$.
\end{enumerate}


\item Suppose that $r$ contains some move in $B$. Let $b$ be the last such move. Then $u$ is of the form $u = \ldots n' \cdot \ldots \cdot \stk\opmove  b \cdot (\pmove_A \omove_A)^* \cdot\ \stk\pomove m $.
So we have
$u\filter B,C = \ldots n' \cdot \ldots \cdot \stk\opmove  b \cdot \stk\pomove m $ hence $b \in \pview{u\filter B,C}$. We can now
conclude by applying the I.H.\ with $u \assignar u_{\prefixof b}$.
\end{itemize}

\item Suppose $m = \pomove \in B$.
If $m^- = \pomove \in B$ then the I.H.\ with $u \assignar u_{\prefixof m^-}$ permits us to conclude.
If $m^- = \omove \in A$ then we conclude by applying the I.H.\ on $u \assignar u_{\prefixof q}$ where $q$ is the external P-move in $A$ justifying $m^-$.
\end{enumerate}
\end{proof}

We now use the two preceding Lemmas to show
the following useful result:
\begin{lemma}[Increasing order lemma]
\label{lem:increasing_order}
Let $u = \ldots n' \cdot r \cdot m \in Int(A,B,C)$ where
\begin{enumerate}
\item
$n'$ is an external O-move in compoment $X$
($n'=\omove_A$ and $X=A,B$, or $n'=\omove_C$ and $X=B,C$)  non initial in $C$,
\item $n' \in \pview{u\filter A,C}$,
\item $m$ is either played in $B$
($\opmove$ or $\pomove$) or is an external
 P-move in $Y$
($\pmove_C$ if $n'=\omove_A$ and
$\pmove_A$ if $n'=\omove_C$),
\item $m$'s justifier occurs before $n'$,
\item $u\filter X$ is P-i.j.,
\item $u_{\prefixof b}\filter Y$ is P-i.j.\ for all non-initial B-move $b$ occurring in $u$.
\end{enumerate}
Then:
$$ \ord_{Y} m \geq \ord_{A\lingamear C} n' \ .$$
\end{lemma}
\begin{proof}
If $n' =\omove_C$ (resp.~if $n'=\omove_A$)
then by Lemma \ref{lem:middleopmove}
(resp.~Lemma \ref{lem:middlepomove})
there is an occurrence in $r \cdot m$ of a non-initial B-move $\gamma$ of type $\opmove$
(resp.~$\pomove$) such that $\gamma \in \pview{u\filter Y}$ , $n' \in \pview{u_{\leq \gamma} \filter X}$ and $\gamma$ is justified by a move occurring before $n'$. By the $6^{th}$ hypothesis, $u_{\prefixof \gamma}\filter Y$ is P-i.j.

There are six possible cases depending on
the type of the moves $n'$ and $m$:
$(n',m) \in \{ \omove_A \} \times \{\pmove_C,\opmove,\pomove \}
\union \{ \omove_C \} \times \{\pmove_A,\opmove,\pomove \} $).
The following diagram illustrates the cases $(n',m)
 = (\omove_A,\pmove_C)$ (left)
and  $(n',m)
 = (\omove_C,\pmove_A)$  (right):
$$
\begin{array}{ccccc}
A & \longrightarrow & B &
 \longrightarrow & C \\
&\vdots&&\vdots\\
&&&& \rnode{n}{\omove} \\
&\vdots&\rnode{gj}{\opmove}&\vdots\\
n' \omove \\
&\vdots&&\vdots  \\
&&\rnode{g}{\gamma} \pomove \\
&\vdots&&\vdots  \\
&&&&\rnode{m}{m} \pmove \\
\end{array}
\ncarc[arcangleA=30,arcangleB=30]{->}{m}{n}
\ncarc[arcangleA=30,arcangleB=30]{->}{g}{gj}
\hspace{2cm} \begin{array}{ccccc}
A & \longrightarrow & B & \longrightarrow & C \\
&\vdots&&\vdots\\
& \rnode{n}{\omove} \\
&\vdots& &\rnode{gj}\vdots\\
&&&&n' \omove \\
&\vdots&&\vdots  \\
&&\rnode{g}{\gamma} \opmove \\
&\vdots&&\vdots  \\
\rnode{m}{m} \pmove \\
\end{array}
\ncarc[arcangleA=30,arcangleB=30]{->}{m}{n}
\ncarc[arcangleA=30,arcangleB=30]{->}{g}{gj}
 $$

We have:
\begin{equation}
\ord_Y \gamma \geq \ord_X \gamma \label{eqn:gammaorderXY}
\end{equation}
Indeed, if $n' =\omove_C$ then $X=B,C$ and $Y=A,B$ and by Lemma
\ref{lem:compositionorder} we have $\ord_{A\lingamear B} \gamma \geq
\ord_{B\lingamear C} \gamma$. If $n=\omove_A$ then $\gamma$ is a
$\pomove$-move therefore it is not initial in $B$ and Lemma
\ref{lem:compositionorder} gives $\ord_{A\lingamear B} \gamma =
\ord_{B\lingamear C} \gamma$.

Hence:
\begin{align*}
\ord_{A\lingamear C} n'
& = \ord_{X} n' & \mbox{(n' non initial in $C$ \& Lemma \ref{lem:compositionorder})} \\
& \leq \ord_{X} \gamma & \mbox{($u_{\prefixof \gamma}\filter Y$ is P-i.j. \& $\gamma$'s justifier occurs before $n'$)} \\
& \leq \ord_{Y} \gamma & \mbox{(By Eq.\ \ref{eqn:gammaorderXY})} \\
& \leq \ord_{Y} m & \mbox{($u\filter X$ is P-i.j. \&
4$^{th}$ assumption: $m$'s justifier occurs before $\gamma$)}.
\end{align*}
\end{proof}


\begin{lemma}
\label{lem:visibleatprefixofu}
Let $u\in Int(A,B,C)$ such that
$u = \ldots \gamma \ldots \delta \ldots m$
where $m$ is a generalized P-move in $X$,
$\gamma \in \pview{u\filter A,C}$  and $\delta \in \pview{u\filter X}$. Then $\gamma \in \pview{u_{\prefixof \delta} \filter A,C}$.
\end{lemma}
\begin{proof}
First we remark than $\delta$ must occur in $\pview{u}$.
Indeed, $\delta \in \pview{u\filter X} = \pview{u_{< m} \filter X} \cdot m$ therefore $\delta \in \pview{u_{< m} \filter X}$ and since the move preceding $m$ in $u$ is necessarily a generalized O-move in $X$, we can use Lemma 3.3.1 from \cite{Harmer2005}:
\begin{align*}
\delta \in \pview{u_{< m} \filter X}
&= \pview{\pview{u_{<m}}\filter X} & \mbox{(Lemma 3.3.1 from \cite{Harmer2005})}\\
&\subseqof \pview{u_{<m}} \\
&\subseqof \pview{u} \ .
\end{align*}

Clearly, $\pview{u_{\prefixof \delta} \filter A,C}$ is a prefix of $\pview{u \filter A,C}$, indeed:
\begin{align*}
\pview{u_{\prefixof \delta} \filter A,C}
& = \pview{u_{\prefixof \delta}}\filter A,C
  & \mbox{(Lemma \ref{lem:pviewAC_eq_ACpview})}  \\
& \prefixof \pview{u}\filter A,C
  & \mbox{($\delta \in \pview{u}$)} \\
& = \pview{u\filter A,C}
  & \mbox{(Lemma \ref{lem:pviewAC_eq_ACpview})} \ .
\end{align*}

Finally since $\gamma \in \pview{u\filter A,C}$ and $\gamma$ occurs before $\delta$ in $u$, we necessarily have $\gamma \in \pview{u_{\prefixof \delta}\filter A,C}$.
\end{proof}

\begin{lemma}
\label{lem:compos_auxiliary_lemma}
Let $X$ be a component and $u \in Int(A,B,C)$ such that
the projection of $u$ on the component $X$ has the form:
$$ u \filter X =
\Pstr[0.5cm][2pt]{ \ldots (n){n}  \ldots
 {\stk {n'}{\omove}}  \ldots  (m-n,30){\stk m {\pmove}}
}$$
and
\begin{enumerate}
  \item $m$ and $n'$ are external move in $X$ ({\it i.e.}~in $A$ if $X =A,B$ and in $C$ if $X=B,C$);
  \item $u\filter X$ is P-i.j.;
  \item $u_{\prefixof b}\filter Y$ is P-i.j.\ for all non-initial B-move $b$ occurring in $u$.
\end{enumerate}
Then either $\ord_{A\lingamear C} n' \leq \ord_{A\lingamear C} m$ or
$n' \not \in \pview{u\filter A,C}$.
\end{lemma}
\begin{proof}

Suppose that $n'$ occurs in the P-view $\pview{u\filter X}$. Then we have
\begin{equation}
\ord_{A\lingamear C} n'  = \ord_{B\lingamear C} n' \ . \label{eqn:ordnp}
\end{equation}
Indeed, if $X$ is the component $B,C$ then necessarily $n'$ is not initial in $C$ (otherwise it would be the first move in $\pview{u \filter B,C}$, which is not the case since by visibility $n$ must occur before $n'$ in the P-view) and
if $X=A,B$ then $n'$ is in $A$. Thus in both cases, Lemma \ref{lem:compositionorder} gives us the claimed equality.

Hence we have
\begin{align*}
\ord_{A\lingamear C} n'
& = \ord_{X} n' & \mbox{(Eq.\
\ref{eqn:ordnp})} \\
& \leq \ord_{X} m & \mbox{($u\filter X$ is P-i.j.)} \\
& = \ord_{A\lingamear C} m & \mbox{(Lemma \ref{lem:compositionorder} \& $m$ is not initial in $C$)} \ .
\end{align*}

Suppose that $n'$ does not occur in the P-view $\pview{u \filter X}$, then $n'$ lies underneath a PO arc occurring in $\pview{u \filter X}$. Let us denote this arc by $\beta$-$\alpha$ where $\beta$ and $\alpha$ denote the arc's nodes. We have:
$$ u \filter X = \ldots
\Pstr[0.5cm]{
 (n){n} \ldots (b){\stk\beta \pmove} \ldots \stk{n'} {\omove}
\ldots (a-b){\stk\alpha \omove}  \ldots (m-n){\stk m {\pmove} }
} $$
with $\ord_X \alpha \leq \ord_X m$ (by P-i.j.\ of $u \filter X$).

\begin{enumerate}[i.]
\item Suppose $\alpha$ is an external move then so is $\beta$. Indeed, if $X=B,C$ and $\alpha = \omove_C$ then $\alpha$ can only point to another move in $C$ and
if $X=A,B$ and $\alpha = \omove_A$ then since $\alpha$ is an O-move in $A,B$, it is not initial in $A$ and therefore its justifier must also be in $A$.

Then instancing Lemma \ref{lem:interjump} with
$n \assignar n'$ gives us $n' \not\in\pview{u \filter A,C}$.

\item Suppose $\alpha$ is a $B$-move then necessarily so is $\beta$. Indeed, if $X=A,B$ then $\alpha \in B$
can only point to a move in $B$, and if $X=B,C$ then
since $\alpha$ is an O-move in the game $B,C$ it is not initial in $B$ and therefore its justfier must also be in $B$.

Now suppose that $n' \in \pview{u\filter A,C}$,
then by Lemma \ref{lem:visibleatprefixofu}
(with $\delta,\gamma \assignar \alpha,n'$)
we have $n' \in \pview{u_{\prefixof \alpha}\filter A,C}$,
and $u_{\prefixof \alpha}\filter Y$ is P-i.j.\ by hypothesis 3. This permits us to apply Lemma \ref{lem:increasing_order} on $u_{\prefixof \alpha}$:
\begin{align*}
\ord_{A\lingamear C} n'
& \leq \ord_{Y} \alpha & \mbox{(Lemma \ref{lem:increasing_order} with $u\assignar u_{\prefixof \alpha}$)} \\
& = \ord_{X} \alpha & \mbox{(Lemma \ref{lem:compositionorder} \& $\alpha$ non initial in $B$)} \\
& \leq \ord_{X} m & \mbox{($u \filter X$ is P-i.j.)} \\
& = \ord_{A\lingamear C} m & \mbox{(Lemma \ref{lem:compositionorder} \& $m$ is not initial in $C$)} \ .
\end{align*}
\end{enumerate}
\end{proof}


\begin{proposition}
\label{prop:closedpijcompose} Let $\sigma : A \lingamear B$ and $\mu
: B \lingamear C$ be two well-bracketed (P-visible) strategies then
\begin{enumerate}[(I)]
\item $\sigma$ closed P-i.j.\ $\wedge$ $\mu$ P-i.j.
$\implies$ $\sigma ; \mu$  P-i.j.;
\item $\sigma, \mu$ closed P-i.j.
$\implies$ $\sigma ; \mu$ closed P-i.j.
\end{enumerate}
\end{proposition}

\begin{proof}
Well-bracketing is preserved by strategy composition (see \cite[Proposition 2.5]{abramsky94full}) thus
$\sigma ; \mu$ is well-bracketed so we can use the definition of P-i.j.\ from Proposition \ref{prop:char_wellbrack}.

\noindent (I) Let us prove that $\sigma ; \mu$ is P-i.j..
Let $u$ be a play of the interaction $\sigma\ \|\ \mu$ between $\sigma$ and $\mu$
ending with an external P-move $m$
justified by $n$ in $\pview{u \filter A , C}$.
Let $n'$ be an external O-move occurring betweeen $n$ and $m$:
$$ u \filter A,C =
\Pstr[0.5cm][2pt]{ \ldots (n){\stk {n} \omove}  \ldots
 {\stk {n'} \omove}  \ldots  (m-n,30){\stk m \pmove}
}
$$
To show that $u \filter A,C$ is P-incrementally justified, we just
need to prove that either $n'\not\in \pview{u \filter A,C}$ or
$\ord_{A\lingamear C} n' \leq \ord_{A\lingamear C} m$. Note that if
$n'\in \pview{u \filter A,C}$ then necessarily $n'$ is not initial
in $C$ because $n$ occurs before $n'$ in $\pview{u \filter A,C}$.

Let $E$ denote one of the two external arenas ($A$ or $C$), $X$ be
the corresponding component ({\it i.e.}~$X=A,B$ if $E=A$ and $X=B,C$
if $E=C$) and $Y$ denote the other component.
    \begin{enumerate}[1)]
    \item Suppose $m$ and $n$ are two external moves in $E$.

        \begin{enumerate}[{1}.a)]
        \item Suppose $n' \in E$.

        This case corresponds to the situation handled by
        Lemma \ref{lem:compos_auxiliary_lemma}: we have
        either $\ord_{A\lingamear C} n' \leq
        \ord_{A\lingamear C} m$ or $n' \not\in \pview{u
        \filter A,C}$.

        \item Suppose $n' \not\in E$.

        Suppose that $n' \in \pview{u\filter A,C}$, then by
        Lemma \ref{lem:increasing_order} with $X\assignar Y$
        we have $ \ord_{A\lingamear C} n'  \leq \ord_X m$
        and since $m$ is not initial in $C$, Lemma
        \ref{lem:compositionorder} gives $\ord_X m =
        \ord_{A\lingamear C} m$, thus $\ord_{A\lingamear C}
        n' \leq \ord_{A\lingamear C} m$.
        \end{enumerate}

        \item \label{case:mA} Suppose $m \in A$ and $n \in C$.

        Then $m$ is an initial move in $A$
        pointing to a $\opmove$-move
        $b_0$ initial in $B$ which in turn points to the $\omove_C$-move $n$ initial in $C$.

        This situation cannot be handled similarly as the
        previous case. Indeed the pointer associated to the move
        $m$ in the game $A,C$ is not the same as the one
        attached to the corresponding move in the game $A,B$
        (see in \cite{Abr02} for the definition of the
        projection operation over the overall component A,C),
        hence we cannot use Lemma \ref{lem:increasing_order}
        since the condition requiring that $m$ points before
        $n'$ is not necessarily met. A more detailed analysis is
        therefore required.

        Let us assume that $n'\in \pview{u\filter A,C}$ and
        prove that we necessarily have $\ord_{A\lingamear C} n'
        \leq \ord_{A\lingamear C} m$. We do a case analysis:
        \begin{itemize}[-]
        \item Suppose $n'$ occurs before $b_0$.
        Note that we cannot apply Lemma \ref{lem:increasing_order} on $u$
        since $m$ does not point before $b_0$.
        Up to now we have only used the fact that $\sigma$ and $\mu$ are P-i.j. The assumption that $\sigma$ is  \emph{closed} P-i.j.\ now becomes crucial.

        Since $n' \in \pview{u\filter A,C}$ and
        $b_0 \in \pview{u\filter B,C}$, applying Lemma \ref{lem:visibleatprefixofu}
        with $X\assignar B,C$ and $\delta,\gamma \assignar b_0,n'$ gives
        $n' \in \pview{u_{\prefixof b_0}\filter A,C}$. This allows us to apply Lemma \ref{lem:increasing_order} on $u_{\prefixof b_0}$:
            \begin{align*}
            \ord_{A\lingamear C} m
            = \ord_A m
            & \geq \ord_B b_0 & \mbox{($u \filter A,B$ is closed P-i.j., $m$ is initial in $A$)} \\
            & = \ord_{B\lingamear C} b_0  \\
            & \geq \ord_{A\lingamear C} n' & \mbox{(Lemma \ref{lem:increasing_order} on $u_{\prefixof b_0}$ with $X\assignar A,B$)} \ .
            \end{align*}

        \item Suppose $n'$ occurs after $b_0$ (and necessarily before $m$).

            \begin{enumerate}[a.]
            \item Suppose $n'\in C$. Since $m$'s justifier occurs before $n'$ in $u$, we can use Lemma \ref{lem:increasing_order} which gives $\ord_{A\lingamear C} n' \leq \ord_{A\lingamear B} m
                = \ord_{A\lingamear C} m$.

            \item Suppose $n'\in A$.
By Lemma \ref{lem:compos_auxiliary_lemma} with $X
\assignar A,B$, since $n' \in \pview{u \filter A,C}$, we
have $\ord_{A\lingamear C} n' \leq \ord_{A\lingamear C}
m$.
\smallskip

        Note that we could not use Lemma
        \ref{lem:increasing_order} on $u$ directly since
        both $m$ and $n'$ are played in $A$. Also, in
        the ideal case where $n'$ is hereditarily
        enabled by the initial move $m$, we can
        immediately conclude $\ord_{A\lingamear C} n'
        \leq \ord_{A\lingamear C} m$; however this
        argument does not work in general: there may be
        more than one initial move in $A$ in which case
        $n'$ can be hereditarily enabled by an initial
        $A$-move distinct from $m$.
            \end{enumerate}
        \end{itemize}

    \end{enumerate}

\noindent (II) We now show that $\sigma;\mu$ is closed P-i.j.\
provided that both $\sigma$ and $\mu$ are. Take a play $s m \in
\sigma ; \mu$ such that $m$ is initial in $A$ and let $n$ be the
initial move of $C$ justifying $m$. Let $u \in \sigma \ \|\ \mu$ be
the uncovering of $s m$ ($s m = u \filter A,C$) and $b_0$ be the
initial $B$-move justifying $m$ in $u$.
 We have:
\begin{align*}
\ord_A m & \geq \ord_B b_0 & \mbox{($u \filter A,B \in \sigma$ is closed P-i.j.)} \\
 & \geq \ord_C n & \mbox{($u_{\prefixof b_0} \filter B,C \in \mu $ is closed P-i.j.)}.
\end{align*}
\end{proof}

{\it Remark:} The second part of the proposition only gives a
\emph{sufficient} condition for $\sigma ; \mu$ to be closed P-i.j.
In fact it is possible to have that $\sigma ; \mu$ is closed P-i.j.\
although $\mu$ is not.


\subsection{Tensor product}

 Given two strategies $\sigma :\ A
\lingamear B$  and $\tau :\ C\lingamear D$, their tensor product
$\sigma \otimes \tau :\ A\otimes B \lingamear C\otimes D$ is defined
as
$$\sigma \otimes \tau = \{ s \in L_{A\otimes C \lingamear B\otimes
D} \ | \ s \filter A,B \in \sigma \wedge s \filter C,D \in \tau \} $$
 where $A\otimes B$ denotes the tensor product of the games $A$ and $B$ (see \cite{abramsky:game-semantics-tutorial}).
\begin{proposition}
Let $\sigma :\ A \lingamear B$  and $\tau :\ C\lingamear D$.
\begin{enumerate}
\item If $\sigma$ and $\tau$ are P-i.j.\
then so is $\sigma \otimes \tau$;
\item If $\sigma$ and $\tau$ are closed P-i.j.\ then so is $\sigma \otimes \tau$.
\end{enumerate}
\end{proposition}

\begin{proof}
By establishing the state diagram of the game $A\otimes C \lingamear
B\otimes D$ one can show easily that only player O can switch
between the subgames $A\lingamear B$ and $C\lingamear D$.
Consequently, in the P-view of a play of the game $A\otimes C
\lingamear B\otimes D$, all the moves are played in the same subgame
({\it i.e.~} all in $A\lingamear B$ or all in $C\lingamear D$).
Hence if the last move of a play $m$ is played in $A\lingamear B$
then $\pview{s\filter A,B} = \pview{s} \filter A,B = \pview{s}$ (and
reciprocally if $m$ is played in $C\lingamear D$). The first part of
the proposition then follows immediately. The second part is also
straightforward.
\end{proof}


\subsection{Pairing} Given two strategies $\sigma :\ C \lingamear A$
and $\tau :\ C\lingamear B$, the pairing $\langle \sigma , \tau
\rangle :\ C \lingamear A\& B$ is defined as
\begin{align*}
\langle \sigma , \tau \rangle
    &= \{ s \in L_{C \lingamear A\& B} \ | \ s \filter C,A \in \sigma \wedge s \filter B = \epsilon \} \\
    & \union \{ s \in L_{C \lingamear A\& B} \ | \ s \filter C,B \in \tau \wedge s \filter A = \epsilon \}
\ .
\end{align*}
 where $A\& B$ denotes the product of the games $A$ and $B$ (see \cite{abramsky:game-semantics-tutorial}).

\begin{proposition}
\label{prop:pij_paring} Let $\sigma :\ C \lingamear A$  and $\tau :\
C\lingamear B$.
\begin{enumerate}
\item If $\sigma$ and $\tau$ are P-i.j.\ then so is $\langle \sigma , \tau \rangle$;
\item If $\sigma$ and $\tau$ are closed P-i.j.\ then so is $\langle \sigma , \tau \rangle$.
\end{enumerate}
\end{proposition}
The proof is immediate.


\subsection{Promotion} \emph{Notation:} Let $s$ be a play. We call
\defname{thread} a maximal subsequence of $s$ constituted of moves
that are hereditarily justified by the same occurrence of an initial
move. Let $m$ be a move occurring in $s$. We call thread of $m$ the
only thread in $s$ containing $m$.


We recall some definitions. Let $A$ and $B$ be two well-opened
games. Given a strategy  $\sigma :\ !A \lingamear B$, its promotion
$\sigma^\dag :\ !A\lingamear !B$ is defined as
$$ \sigma^\dag = \{ s \in L_{!A\lingamear !B}\ |\ \mbox{for all inital $m$ in $B$, } s\filter m \in \sigma \}$$
and for $\mu :\ !B\lingamear C$ the composite strategy $\sigma
\fatcompos \mu$ is defined as:
$$ \sigma \fatcompos \mu = \sigma^\dag ; \mu \ .$$

Since $B$ is well-opened, plays of $\sigma$ are constituted of a
single thread initiated by some initial $B$-move. Plays of
$\sigma^\dag$ however, are interleaves of potentially infinitely many single-threaded
plays of $\sigma$. One can show easily, using the visibility condition, that the thread of a $P$-move
is always the same as the thread of the preceding $O$-move. Consequently, the P-view of a play is equal to the P-view of the current thread:
if the current thread of a play $s$ is opened by an initial move $b \in B$ then
$\pview{s} = \pview{s \filter b} = \pview{s} \filter b$.


The state of the game is given by an infinite sequence of symbols in $\{O, P\}$, each element of the
sequence indicating who is to play in the corresponding thread.
The diagram on Figure \ref{fig:promotion_state_diagram} illustrates
how the state changes as a play of $\sigma^\dag$ unfolds.
The initial state of the game is $O^\omega$ - an infinite
sequence of O's -- which indicates that O is to play in all the
threads. When O plays an initial move in $B$, it ``opens'' a new
thread so the state of the game becomes $O^k P O^\omega$ where $k$
is the index of the thread being opened. By alternation, $P$ now has to play. His move must be played in a thread
already opened by $O$ and in which $P$ is to play; only one thread is in such state: the $k$th one. Hence after P's move
we are back to state $O^\omega$.

\tikzstyle{state}=[rectangle,draw=blue!50,fill=blue!20,thick,minimum
height = 4ex, text width=1.2cm, text centered]
\tikzstyle{state_nobg}=[thick,minimum
height = 4ex, text width=1.2cm, text centered]
\tikzstyle{omove}=[->,shorten <=1pt,>=latex',line width=0.5pt,bend left=10]
\tikzstyle{pmove}=[->,shorten <=1pt,>=latex',line width=0.5pt,bend left=10, draw=blue!50]
\begin{figure}[htbp]
\begin{center}
\begin{tikzpicture}[node distance=2cm]
% the states
\path
 node(init)  [state, text width=4cm] {$O^\omega$}
 (init)+(-2.8cm,-3cm)
 node(p)     [state, anchor=east,] {$PO^\omega$}
 node(p1)    [state, right of=p]  {$OPO^\omega$}
 node(p2)    [state_nobg, right of=p1] {\ldots}
 node(p3)    [state, right of=p2] {$O^kPO^\omega$}
 node(p4)    [state_nobg, right of=p3] {\ldots} ;
\path
% arrows representing the moves
  ([xshift=-1.4cm]init.south)  edge[omove] node[right]{O} ([xshift=0.2cm]p.north)
  (p.north)    edge[pmove] node[left]{P} ([xshift=-1.5cm]init.south)
  ([xshift=-0cm]init.south)   edge[omove] node[right]{O} ([xshift=0.2cm]p1.north)
  (p1.north)   edge[pmove] node[left]{P} ([xshift=-0.2cm]init.south)
  ([xshift=1cm]init.south)   edge[omove] node[right]{O} ([xshift=0cm]p3.north)
  ([xshift=-0.2cm]p3.north)   edge[pmove] node[left]{P} ([xshift=0.8cm]init.south);
\end{tikzpicture}
\end{center}
\caption{State diagram for plays of $\sigma^\dag$.}
\label{fig:promotion_state_diagram}
\end{figure}



\begin{proposition}
\label{prop:fatcompos_pij} If $A$ and $B$ are two well-opened games
and $\sigma :\ !A \lingamear B$ is a well-bracketed P-i.j.\ strategy
then $\sigma^\dag$ is also well-bracketed and P-i.j. Furthermore if
$\sigma$ is closed P-i.j.\ then so is $\sigma^\dagger$.
\end{proposition}
\begin{proof}
$\sigma^\dag$ is well-bracketed by \cite[Proposition
2.10.]{abramsky94full}. For P-incremental justification, the result is a direct consequence of the
fact that the P-view of a play in $\sigma^\dag$ is equal to the P-view of the current thread.
For closed P-incremental justification, the result is immediate.
\end{proof}

From propositions \ref{prop:closedpijcompose} and
\ref{prop:fatcompos_pij} we obtain:
\begin{corollary}
Let $A$ and $B$ be two well-opened games. Let $\sigma :\ !A
\lingamear B$ and $\mu :\ !B\lingamear C$ be two well-bracketed
strategies then:
\begin{enumerate}
\item If $\sigma$ is closed P-i.j.\ and $\mu$ is P-i.j.\ then $\sigma \fatcompos \mu :\ !A \lingamear
C$ is also P-i.j.;
\item If $\sigma$ and $\mu$ are closed P-i.j.\ then so is $\sigma \fatcompos \mu :\ !A \lingamear C$.
\end{enumerate}
\end{corollary}

\subsection{The category $\mathcal{G}_{Pij}$ of closed P-i.j.\ strategies}

We define the category $\mathcal{G}_{Pij}$ as follows:
\begin{itemize}
\item the objects are games (as defined in \cite{abramsky:game-semantics-tutorial}),
\item the morphisms from $A$ to $B$ are the closed P-incrementally-justified strategies
on the game $A\rightarrow B$,
\item morphisms are composed using the standard game-semantic strategy composition.
\end{itemize}
This indeed defines a category. Indeed we have shown in the previous
section that closed P-i.j.\ compose, strategy composition is
associative (\cite{abramsky94full,hylandong_pcf}) and finally the
identity strategy $id_A$ for any game $A$ is closed P-i.j.

We mentioned before that such category cannot be cartesian closed. Indeed, remember that
a P-i.j.\ strategy from $A$ to $B$ is said to be \emph{closed} P-i.j.\ provided that some condition
on the arena $A\rightarrow B$ holds. This condition refers precisely to the structure of the arenas $A$ and $B$ and consequently relies on the fact the arena considered is exactly $A\rightarrow B$ (and not any other isomorphic arena).



\section{Modeling the Safe Lambda Calculus in $\mathcal{G}_{Pij}^{inn}$}

Consider the category $\mathcal{G}_{Pij}$ defined in section
\ref{sec:closedpij}. In this section we show how the safe lambda
calculus can be modeled in the sub-category
$\mathcal{G}_{Pij}^{inn}$ of innocent P-ij strategies.

\subsection{The language}
We recall the definition of the safe simply-typed lambda calculus
\cite{blumong:safelambdacalculus}.  We use sequents of the form
$\Gamma \vdash_s M : A$ to represent terms-in-context where $\Gamma$
is the context and $A$ is the type of $M$. For simplicity we write
$(A_1, \cdots, A_n, B)$ to mean $A_1 \typear \cdots \typear A_n
\typear B$, where $B$ is not necessarily ground.

\begin{definition}\rm
The \defname{safe lambda calculus}, or Safe $\Lambda^{\rightarrow}$ for short, is a sub-system of the
  simply-typed lambda calculus defined by induction over the
  following rules:
$$ \rulename{var} \ \rulef{}{x : A\vdash_s x : A} \quad
\rulename{wk} \ \rulef{\Gamma \vdash_s s : A}{\Delta \vdash_s s : A} \quad
\Gamma \subset \Delta$$
$$ \rulename{app} \ \rulef{\Gamma \vdash_s s : (A_1,\ldots,A_n,B) \
  \Gamma \vdash_s t_1 : A_1 \; \ldots \; \Gamma \vdash_s t_n : A_n
} {\Gamma \vdash_s s t_1 \ldots t_n : B} \ \min_{y:Y \in \Gamma} \ord Y \geq \ord B$$
$$ \rulename{abs} \ \rulef{\Gamma, x_1 : A_1, \ldots, x_n : A_n
  \vdash_s s : B} {\Gamma \vdash_s \lambda x_1 \ldots x_n . s :
  (A_1, \ldots ,A_n,B)} \ \min_{y:Y \in \Gamma} \ord Y \geq \ord (A_1, \ldots ,A_n,B)$$
%  where $\ord{\Gamma}$ denotes the set $\{ \ord{y} : y
%\in \Gamma \}$ and ``$c \sqsubseteq S$'' means that $c$ is a
%lower-bound of the set $S$.
\end{definition}


\subsection{Game-semantic denotation}

In \cite{blumong:safelambdacalculus} we showed that in the game
semantic model safe lambda terms are denoted by P-i.j.\ strategies.
The argument was syntactic: it is based on the analysis of a special
kind of abstract syntax tree of a term called computation tree
\cite{OngLics2006}. Here we give another proof based on a semantic
argument that uses the results of section \ref{sec:closedpij}.


\begin{proposition}
\label{prop:safe_closepij_sem}
  Safe simply-typed terms are denoted by closed P-i.j.\ strategies.
\end{proposition}
\begin{proof}
  By induction on the formation rules.
  \begin{enumerate}
    \item (var) $\sem{x:A \vdash_s x:A } = id_A$. Clearly the
    identity strategy is closed P-i.j.

    \item (wk) Take $\Gamma \subset \Delta $ and suppose $\sem{\Gamma \vdash_s
    s : A}$ is closed P-i.j. Up to an appropriate retagging of
    the moves the two strategies $\sem{\Delta \vdash_s s : A}$
    and $\sem{\Gamma \vdash_s s : A}$ are isomorphic. Hence
    $\sem{\Delta \vdash_s s : A}$ is P-i.j. It is also closed
    P-i.j.\ since none of the new initial moves introduced by
    $\Delta$ occurs in any play of the strategy.

    \item (app) Suppose that $\sem{\Gamma \vdash_s s :
    (A_1,\ldots,A_n,B)}$ and $\sem{\Gamma \vdash_s t_i : A_i}$
    for $i \in \{1..n\}$ are closed P-i.j.\ and $\ord{B}
    \sqsubseteq \ord{\Gamma}$. We have $\sem{ \Gamma \vdash_s s
    t_1 \ldots t_n : B} = \langle \sem{\Gamma \vdash s},
    \sem{\Gamma \vdash t_1}, \ldots, \sem{\Gamma \vdash t_n}
    \rangle \fatsemi ev_n$ where $ev_n$ is the $n$-parameter
    evaluation strategy. By Proposition \ref{prop:pij_paring},
    $\langle \sem{\Gamma \vdash s}, \sem{\Gamma \vdash t_1} ,
    \ldots, \sem{\Gamma \vdash t_n} \rangle$ is closed P-i.j.
    The evaluation map $ev_n$ is P-i.j.\ (but not necessarily
    closed P-i.j.) therefore by Proposition
    \ref{prop:closedpijcompose} I. $\sem{ \Gamma \vdash_s s t_1
    \ldots t_n : B}$ is P-i.j. The arena of the game
    $\sem{\Gamma}$ is of type $\sem{Y_1} \times \ldots \times
    \sem{Y_n}$ where the $Y_i$s are the types of the variables
    in the context $\Gamma$. The $\sem{Y_i}$s are all prime
    since we work with pure simple types without product.
    Moreover the side-condition of the rule gives $\ord Y_i \geq
    \ord B$ for all $i \in \{1..n\}$. Hence by Lemma
    \ref{lem:closedpij_singleBinitmove}(ii), $\sem{ \Gamma
    \vdash_s s t_1 \ldots t_n : B}$ is closed P-i.j.

    \item (abs) Suppose that $\sem{\Gamma, x_1 : A_1, \ldots, x_n : A_n \vdash_s
    s : B}$ is closed P-i.j. Then the isomorphic strategy $\sigma = \sem{\Gamma \vdash_s \lambda
    x_1 \ldots x_n . s : (A_1,\ldots,A_n,B)}$ is also P-i.j.
    Again, using the side-condition, Lemma \ref{lem:closedpij_singleBinitmove}(ii)
    implies that $\sigma$ is closed P-i.j.
  \end{enumerate}
\end{proof}

\subsection{The safe lambda calculus with product ($\Lambda^{\rightarrow}_\times$)}
We will now show how product types and pairing can be added to the safe lambda calculus.
This can be done trivially as follows. Types are now given by the following grammar:
\begin{align*}
T ::=& \ B,  \quad \mbox{for some base type $B$} \\
   &| \ T \rightarrow T \\
   &| \ T \times T
\end{align*}
and the typing system is extended with the following three rules:
$$ \rulename{\times} \ \rulef{\Gamma \vdash_s s : A \qquad \Gamma \vdash_s t : B}
{\Gamma \vdash_s \langle s, t \rangle : A \times B}
\qquad \rulename{\pi_1} \ \rulef{\Gamma \vdash_s s : A \times B}
{\Gamma \vdash_s \pi_1 s : A} \qquad
 \rulename{\pi_2} \ \rulef{\Gamma \vdash_s s : A \times B}
{\Gamma \vdash_s \pi_2 s : B}$$

One can easily check that most of the good properties of the safe
lambda calculus remains in this extended calculus: the free
variables of a term have order greater than the order of the term
itself; the no-variable-renaming Lemma holds and terms are denoted
by P-i.j.\ strategies. However in general, terms are not denoted by
\emph{closed} P-i.j.\ strategies. Indeed, in the safe lambda calculus
without product - as defined in the previous section - all the
arenas involved are prime {\it i.e.}~they have a single initial
move. In the present case however, since types can be constructed
using the cartesian product, the corresponding arenas can have many
initial moves. Consequently Lemma
\ref{lem:closedpij_singleBinitmove}(ii) cannot be used anymore! Here
is a counter-example: Take the term $x : (o^1\rightarrow o^2)\times
o^3 \vdash \lambda y^o . \pi_2 x : o^4 \rightarrow o^5$ denoted by
some P-i.j.\ strategy $\sigma$ containing the play $q^5 q^3$. We have
$\ord_{(o^1\rightarrow o^2)\times o^3} q^3 = 0 < 1 = \ord_{o^4
\rightarrow o^5} q^5$ therefore $\sigma$ is not closed P-i.j.


There are two different approaches to overcome this problem. The
first one consists in restricting the types of the variables
appearing in the context of a term. More precisely we require that
whenever a variable has product type $A \times B$ we have $\ord A =
\ord B$. One can easily check that this rules out the problem
underlined in the previous counter-example and that it guarantees
that terms are then indeed denoted by closed P-i.j.\ strategies.

The other approach is less restrictive but requires us to modify
slightly the side-conditions of the application and abstraction
rules: instead of requiring that all variables in the context have
order greater than the order of the term, we require that the order
of \emph{any prime sub-type of any variable} in the context has
order greater that the order of the term. The set $Pr(A)$ of prime
sub-types of a type $A$ being defined as follows:
\begin{align*}
Pr(B) &= \{ B \} \qquad \mbox{ for some base type } B \\
Pr(A\rightarrow B) &= \{ A\rightarrow B \} \\
Pr(A\times B) &= Pr(A) \union Pr(B)
\end{align*}

This gives rise the following calculus:
\begin{definition}\rm
The \defname{safe lambda calculus with product}, or Safe
$\Lambda^{\rightarrow}_\times$ for short, is given by induction over
the following rules:
$$ \rulename{var} \ \rulef{}{x : A\vdash_s x : A} \quad
\rulename{wk} \ \rulef{\Gamma \vdash_s s : A}{\Delta \vdash_s s : A} \quad
\Gamma \subset \Delta$$
$$ \rulename{\times} \ \rulef{\Gamma \vdash_s s : A \qquad \Gamma \vdash_s t : B}
{\Gamma \vdash_s \langle s, t \rangle : A \times B}
\qquad \rulename{\pi_1} \ \rulef{\Gamma \vdash_s s : A \times B}
{\Gamma \vdash_s \pi_1 s : A} \qquad
 \rulename{\pi_2} \ \rulef{\Gamma \vdash_s s : A \times B}
{\Gamma \vdash_s \pi_2 s : B}$$
$$ \rulename{app} \ \rulef{\Gamma \vdash_s s : (A_1,\ldots,A_n,B) \
  \Gamma \vdash_s t_1 : A_1 \; \ldots \; \Gamma \vdash_s t_n : A_n
} {\Gamma \vdash_s s t_1 \ldots t_n : B} \ C(\Gamma ; B)$$
$$ \rulename{abs} \ \rulef{\Gamma, x_1 : A_1, \ldots, x_n : A_n
  \vdash_s s : B} {\Gamma \vdash_s \lambda x_1 \ldots x_n . s :
  (A_1, \ldots ,A_n,B)} \ C(\Gamma ; (A_1, \ldots ,A_n,B) )$$

where the side-condition $C(\Gamma ; B)$ expresses that $\forall y:Y
\in \Gamma. \forall Y' \in Pr(Y) . \ord Y' \geq \ord B$.
\end{definition}

One can show by induction on the rules that the game denotations of
terms of this calculus are closed P-i.j.\ (the argument is similar to
the one used in the proof of Proposition
\ref{prop:safe_closepij_sem}). However this syntax does not
completely capture all the closed P-i.j.\ strategies. Take for
instance the simply-typed term $x:(o\rightarrow o)\times o \vdash \lambda z^o. \
(\pi_1 x) : o \rightarrow (o \rightarrow o)$; Its denotation is
closed P-i.j.\ but it is not typable in
$\Lambda^{\rightarrow}_\times$.


\section{Modeling Safe PCF in $\mathcal{G}_{Pij}^{inn}$}

\subsection{Safe PCF}

\notetoself{Insert here the definition of Safe PCF from transfer
thesis or refer to a definition given in a previous chapter.}

\subsection{Game-semantic denotation (semantic argument)}

\begin{proposition}
\label{prop:safepcf_closedpij} Safe PCF terms are denoted by closed
P-incrementally justified strategies.
\end{proposition}
\begin{proof}
We first prove the result for $\pcf_1$ - the fragment of \pcf\
containing terms of the form $\Omega_A = Y (\lambda x : A.x)$ but
where no other use of Y is allowed (see
\cite{abramsky:game-semantics-tutorial}). The proof is by structural
induction over the structure of the term.
\begin{itemize}
\item The strategy $\sem{\Omega_A} = \bot$ is
clearly closed P-i.j.

\item The functional rules are treated the same way as in the
corresponding proof for the safe lambda calculus.

\item For the arithmetic rules, we observe that the strategies
$succ$, $pred$ and $cond$ are all closed P-i.j. The fact that
pairing and strategy composition preserve closed P-incremental
justification permits us to conclude.
\end{itemize}

We now lift the result to full PCF using the technique of
\emph{syntactic approximant} (see
\cite{abramsky:game-semantics-tutorial}). By \cite[lemma
16]{abramsky:game-semantics-tutorial} we have
$$ \sem{M} = \Union_{n\in\omega} \sem{M_n}$$
where $M_n$ is the $\pcf_1$ term obtained from $M$ by replacing each
subterm of the form $Y N$ with $Y^n N_n$, and $Y^n F$ denotes the
$n$th approximant of $Y F$. Since the $M_n$s are $\pcf_1$ terms, by
the previous result each $\sem{M_n}$ is closed P-i.j.\ and since
closed P-incremental justification is clearly a continuous property,
$\sem{M}$ is also closed P-i.j.
\end{proof}


\subsection{Full abstraction}

\subsubsection{O-incremental justification}
 
\defname{O-incremental justification} is the counterpart of P-incremental justification ({\it i.e.}~the role of O and P is exchanged in the definition).

O-incremental justification relates to P-incremental justification very much like O-visibility relates to P-visibility
(see \cite[Sec.~3.6]{Harmer2005}).

Let $\sigma : A$ and $\mu : A \rightarrow o$ be two strategies and $q$ be the initial move of the game $A \rightarrow o$. Then P-views of plays in $A$ correspond to O-views
in the game $A \rightarrow o$. Indeed, for $s\in L_A$ we have $q s \in L_{A \rightarrow o}$ and
due to alternation, $q \pview{s}^A = \oview{q s }_{A \rightarrow o}$.

Consequently, if $\sigma$ is P-i.j.\ then the play involved in the interaction between $\sigma$ and $\mu$
are all O-i.j.\ from $\mu$'s perspective. Indeed, let $u \in \sigma \| \mu$ with $|u|\geq1$. Then $u=q v$
and $u\filter A = v \filter A$ is P-i.j. By the previous remark, this implies that $q (v\filter A) = (q v)\filter (A \rightarrow o) = u \filter (A \rightarrow o)$ is O-i.j.
\smallskip

Now if we regard $\sigma$ as the denotation of some closed term $\vdash M:A$ and $\mu$ as the
denotation of some context $x:A \vdash C[x]:o$ then what the previous remark says is that
non O-i.j.\ plays are useless for the purpose of studying observational equivalence!
This suggests that it is not necessary to include non O-i.j.\ plays in the game denotation of safe terms. However before removing completely those plays from the game model, we have to ensure that this does not prevent us from constructing a category:
\begin{lemma}
\label{lem:oij_decomp}
Let $\sigma : A\rightarrow B$ and $\tau : A\rightarrow B$ be closed P-i.j.\ strategies and suppose
that $u\in \sigma \| \tau$ such that for all external O-moves $o$ of $u$, we have that $u_{\prefixof o} \filter A,C$ satisfies
O-incremental justification. Then, for any generalized O-move $m$ of $u$ in component $X$, we have that
$u_{\prefixof m} \filter X$ satisfies O-incremental justification
\end{lemma}

This lemma states that O-i.j plays cannot be obtained from the interaction of plays that are not O-i.j. In other words, if we write $\mathcal{O}(\sigma)$ for the set of O-i.j.\ plays of $\sigma$, then the Lemma can be restated equivalently as:
\begin{eqnarray}
     \forall \sigma, \tau\ \mbox{closed P-i.j.}: \mathcal{O}(\sigma) ; \mathcal{O}(\tau) \supseteq \mathcal{O}(\sigma ; \tau)
     \label{eqn:oijdecomp_1}
\end{eqnarray}
which in turn is equivalent to
\begin{eqnarray}
    \forall \sigma, \tau\ \mbox{closed P-i.j.}: \mathcal{O}( \mathcal{O}(\sigma) ; \mathcal{O}(\tau) ) = \mathcal{O}(\sigma ; \tau)
    \label{eqn:oijdecomp_2}
\end{eqnarray}
Indeed, Eq.~\ref{eqn:oijdecomp_1} implies the right-to-left inclusion and the other inclusion
is given by the fact that $\mathcal{O}(\sigma) ; \mathcal{O}(\tau) \subseteq \sigma;\tau$.


In some sense, Lemma \ref{lem:oij_decomp} is the dual of the proposition stating that closed P-i.j.\
strategies compose, since  the latter can be reexpressed more succinctly with the relation:
\begin{eqnarray}
     \forall \sigma, \tau .\, \mathcal{P}(\sigma) ; \mathcal{P}(\tau) \subseteq \mathcal{P}(\sigma ; \tau)
     \label{eqn:pijcomp_1}
\end{eqnarray}
where $\mathcal{P}(\sigma)$ is define as be the largest even-length-prefix-closed subset of $\sigma$ consisting of closed P-i.j.\ plays.

\subsubsection{A category of incremental strategies}

\notetoself{
-Incremental strategies means O-i.j.\ and closed P-i.j.

}

\subsubsection{Full abstraction}

The fully-abstract game-model of PCF is also fully-abstract for the
safe fragment of PCF when observational equivalence is defined with
respect to unrestricted ({\it i.e.}~possibly unsafe) PCF contexts.
However one may ask what is a fully abstract model of Safe PCF with
respect to \emph{safe} contexts.




\notetoself{
By the definability results for Safe PCF, it should be possible to
prove that the category $\mathcal{C}^{inn}_{OP-incr}$ of
OP-incrementally justified and innocent strategies is fully abstract
for Safe PCF. We can use the same proof as in the PCF case: we have
a compact test strategy $\alpha:A\rightarrow N$ and by definability,
there must be some context $x:A \vdash C[x] : N$ such that $\sem{x:A
\vdash C[x] : N} = \alpha$. The definability result for Safe PCF
gives us that $\lambda x . C[x] : A \rightarrow N$ is safe which in
turns implies that $x:A \vdash C[x] : N$ is safe since $\ord{N} =
0$.
}

\subsubsection{Algorithmic game semantics}
We recall that Strongly Safe IA $\subseteq$ Safe IA $\subseteq$ IA.
Up to order $3$, it is conservative, with respect to observational equivalence, to add unsafe context to safe ones.
At order $4$, it is not conservative anymore.

\paragraph{Observational equivalence}
\begin{table}
\begin{tabular}{|c|c||c|c|c|c|c|}
    \cline{3-7}
  \multicolumn{2}{c|}{}  & \multicolumn{5}{c|}{Finitary fragments} \\ \hline
  \multirow{2}{*}{$L$} & \multirow{2}{*}{$C[\_]$} &   order 2          &  order 2       & order 3     & order 3 & \multirow{2}{*}{order 4}  \\
                       &                          &    + while         &   + $Y_1$      & + while     & +$Y_0$  &          \\ \hline \hline

  \multirow{4}{*}{IA}  & \multirow{2}{*}{IA}      & \multirow{4}{2cm}{PSPACE$^{(1)}$ \\ {\small $\preccurlyeq$ DFA}} & \multirow{4}{*}{U$^{(2)}$} & \multirow{4}{2.8cm}{EXP-complete$^{(3)}$ \\ {\small $\preccurlyeq$ VPA} }  & \multirow{4}{2cm}{D$^{(4)}$ \\ {\small $\preccurlyeq_{exp}$ DPDA\\ $\succcurlyeq$ DPDA} } & \multirow{2}{*}{U$^{(5)}$}\\
                       &                          &                    &                    &  & & \\
\cline{2-2}\cline{7-7} & \multirow{2}{*}{Safe IA} &                    &                    &  & & \multirow{2}{*}{?} \\
                       &                          &                    &                    &  & & \\ \hline

  \multirow{4}{*}{Safe IA} & \multirow{2}{*}{IA}      & \multirow{4}{2cm}{PSPACE \\ {\small $\preccurlyeq$ DFA}} & \multirow{4}{*}{U} & \multirow{4}{2.3cm}{EXP-complete \\ {\small $\preccurlyeq$ VPA}} & \multirow{4}{2cm}{D \\ {\small $\preccurlyeq_{exp}$ DPDA\\ $\succcurlyeq$ DPDA} } & \multirow{2}{*}{U} \\
                           &                          &                    &                & & & \\
\cline{2-2}\cline{7-7}     & \multirow{2}{*}{Safe IA} &                    &                & & & \multirow{2}{*}{?} \\
                           &                          &                    &                & & & \\ \hline

  \multirow{4}{*}{St. Safe IA} & \multirow{2}{*}{IA}           & \multirow{4}{*}{D} & \multirow{4}{*}{?} & \multirow{4}{*}{D} & \multirow{4}{*}{D} & \multirow{2}{*}{?} \\
                               &                               &                    &                    &                    &                    & \\
\cline{2-2} \cline{7-7}        &  \multirow{2}{*}{St. Safe IA} &                    &                    &                    &                    & \multirow{2}{*}{?} \\
                               &                               &                    &                    &                    &                    & \\ \hline
\end{tabular}
\caption{Decidability (and complexity) of observational equivalence for some finitary fragments of IA}

U stands for Undecidable and D stands for decidable with unknown complexity, $\preccurlyeq P$ means ``reducible to problem $P$''
and $\succcurlyeq P$ means ``at least as hard as problem $P$''.
\begin{asparaenum}
\item[1.] See \cite{ghicamccusker00}.
\item[2.] Showed by Ong in \cite{OngLics2006}.
\item[3.] See \cite{DBLP:conf/fossacs/MurawskiW05}.
\item[4.] See \cite{DBLP:conf/icalp/MurawskiOW05}.
\item[5.] By encoding of $\Sigma$-machine (turing complete) into IA$_4$, see \cite{murawski03program}.

\end{asparaenum}
\end{table}

\paragraph{Observational approximation}

Observational approximation has been shown to be undecidable at order $1$ already, for the fragment $IA_1 + Y_0$ (\cite{DBLP:conf/fossacs/MurawskiW05}).


\notetoself{
- Characterization of the set of complete plays for Safe IA.
(Easy adaptation of the corresponding result for IA. In the present case however, the proof relies
on the fact that plays of the strategy are O-i.j. (in order for $\alpha$ to be P-i.j.)
}


\subsection{What is a model of Safe PCF/Safe IA?}

\notetoself{
- Define the notion of incremental category.

- Show that any incremental category is a model of Safe PCF and that any model of Safe PCF
is an incremental category.

- Show that the category of games and OP-i.j. strategies is an incremental category.

}



\section{Modeling Safe IA in $\mathcal{G}_{Pij}$}
\notetoself{
- I need to merge this section with the other note on Safe IA.

- $\iavar =  \iacom^{\omega}\times \iaexp$

- Any strategy on the game $I \lingamear\ !\iavar$ is P-i.j.\ (and
thus closed P-i.j.) since there is no P-question in the arena
\iavar. Hence the strategy $cell$ is P-i.j. }

\subsubsection{Game-semantic denotation}

In this section, our aim is to extend the game-semantic characterization of safety to Safe IA.

We first observe that the result extends extends trivially from Safe PCF to Strongly Safe IA:
\begin{proposition}
  Strongly Safe IA terms are denoted by closed P-i.j. strategies.
\end{proposition}
\begin{proof}
The proof is an adaptation of the proof for Safe PCF. We first show that the result holds for the
fragment of Strongly Safe IA in which the only allowed uses of $Y$ are in terms of the form $\Omega$.
This is done by induction over the structure of the term:
The functional rules and the arithmetic rules are treated
the same way as in the proof for Safe PCF. For the imperative rules, we
observe that the strategies $assign$, $deref$, $mkvar$, $seq$ and
$cell$ are all closed P-i.j. The fact that pairing, tensor product
and strategy composition all preserve closed P-incremental
justification permits us to conclude.

The result is then lifted to the whole of Strongly Safe IA using syntactic approximants as in the PCF case.
\end{proof}

Now we would like to extend this result to full Safe IA:
\begin{proposition}
\label{prop:safeia_closedpij} Safe IA terms are denoted by closed
P-incrementally justified strategies.
\end{proposition}

This happens to be less trivial than for the previous restricted fragment of Safe IA. We first introduce some definitions:

\begin{definition}[P-i.j. modulo $\mathfrak{M}$]
\label{def:pij_modulo} Let $\sigma$ be a strategy on some game $A$
and $\mathfrak{M}$ be a set of moves. We say that $\sigma$ is P-i.j.
modulo $\mathfrak{M}$ iff for all $s m \in \sigma$ with $m \not\in
\mathfrak{M}$, the play $s m$ is P-i.j.

Similarly we say that $\sigma$ is \emph{closed} P-i.j. modulo
$\mathfrak{M}$ iff for all $s m \in \sigma$ with $m \not\in
\mathfrak{M}$ the play $s m$ is \emph{closed} P-i.j.

Hence a strategy is P-i.j. if and only if it is P-i.j. modulo
$\emptyset$.
\end{definition}

Given a term $\Gamma | \Gamma^{\ianew} \safeentail M : A$, we write
$\sem{\Gamma | \Gamma^{\ianew} \safeentail M : A}$ to denote the
game denotation of the corresponding IA term {\it i.e.}
$\sem{\Gamma, \Gamma^{\ianew} \vdash M : A}$. Instead of showing
Proposition \ref{prop:safeia_closedpij} we will prove the following
more general result:
\begin{proposition}
\label{prop:safeia_closedpijmodulo} Let $\Gamma | \Gamma^{\ianew}
\safeentail M : A $ be a Safe IA term. Its denotation $\sem{\Gamma |
\Gamma^{\ianew} \safeentail M : A}$ is closed P-i.j. modulo
$\mathfrak{M}_{\Gamma^{\ianew}}$ where
$\mathfrak{M}_{\Gamma^{\ianew}}$ is the set of initial moves in
$\Gamma^{\ianew}$.
\end{proposition}

\begin{remark}
Since the context $\Gamma^{\ianew}$ contains variable of type
\iavar\ only, $\mathfrak{M}_{\Gamma^{\ianew}}$ contains only moves
of the form `$read$' or `$write_i$' for some $i\in \nat$.
\end{remark}

\begin{lemma}
\label{lem:leftcompos_preserv_pijmodulo}
 Let $\sigma : A \rightarrow
B$ and $\mu : B \rightarrow C$.
  Let $\mathfrak{M}$ be any set of moves initial in $A$.
  If $\sigma$ is closed  P-i.j. modulo $\mathfrak{M}$ and $\mu$ is
  P-i.j. (resp. closed P-i.j.) then $\sigma \fatsemi \mu$ is P-i.j. (resp. closed P-i.j.) modulo $\mathfrak{M}$.
\end{lemma}
\notetoself{
\begin{proof}
Let us analyze the proof of compositionality for closed P-i.j.
strategies.

\end{proof}
}


\begin{lemma}
\label{lem:cellcomposition_preserve_pijmodulo} Let $\tau : I
\rightarrow C_2$, $\sigma : C_1 \otimes C_2 \rightarrow B$  and
$\mathfrak{M}$ be any set of moves initial in $C_1 \otimes
C_2$.
  If $\tau$ is P-i.j. and
  $\sigma$ is P-i.j. (resp. closed P-i.j.) modulo $\mathfrak{M}$
  then $(id_{C_1} \otimes \tau) \fatsemi \sigma$ is P-i.j. (resp. closed P-i.j.) modulo $\mathfrak{M} \inter C_1$.
\end{lemma}
\notetoself{
\begin{proof}
\end{proof}
}


\notetoself{
\begin{proof}[Proof of Prop.\ \ref{prop:safeia_closedpijmodulo}]
We first prove the result for the fragment of Safe IA where the only allowed uses of the $Y$ combinator is in terms of the form $\Omega$. By induction and and case analysis on the structure of Safe IA terms:
\begin{itemize}
  \item[$\rulename{var}$, $\rulename{var^\ianew}$
  , $\rulename{wk}$, $\rulename{wk^\ianew}$]

  These cases are treated the same way as in the corresponding
  proof for the Safe Lambda Calculus.

  \item[$\rulename{app}$]

  \item[$\rulename{abs}$]


  \item[$\rulename{const}$]
  \item[$\rulename{succ}$, $\rulename{pred}$,$\rulename{cond}$]

  \item[$\rulename{seq}$]
  \item[$\rulename{assign}$]
  \item[$\rulename{deref}$]
  \item[$\rulename{new}$] $\Gamma | \Gamma^\ianew \safeentail \ianewin{x}\ M : B$.

Let $cell : I \rightarrow !\iavar$ denotes the ``storage cell''
strategy (see \cite{abramsky:game-semantics-tutorial}).  Let
$\sigma = \sem{\Gamma | \Gamma^\ianew, x : \iavar \safeentail M
: B}$.  We have $\sem{\Gamma | \Gamma^\ianew \safeentail
\ianewin{x}\  M : B} = (id_{\Gamma,\Gamma^\iavar} \otimes cell)
\fatcompos \sigma$. By induction hypothesis $\sigma$ is closed
P-i.j. modulo $\mathfrak{M}_{\Gamma^{\ianew} \otimes !\iavar}$
and one can easily check that $cell$ is P-incrementally
justified. Instancing Lemma
\ref{lem:cellcomposition_preserve_pijmodulo} with $\tau
\leftarrow cell$, $C_1 \leftarrow \Gamma \otimes  \Gamma^\ianew$
and $C_2\leftarrow !\iavar$ gives us the desired result.
  \item[$\rulename{mkvar}$]

\end{itemize}
The result is then lifted to the whole of Safe IA using the technique of syntactic approximants and using the fact that the ``closed P-i.j.'' property is continuous.
\end{proof}
}


\subsection{Algorithmic game semantics}

There is an important theorem (\cite{AM97a}) in game semantics
which states that two IA terms are equivalent if and only if the set
of complete plays of their game denotations are equal. This result was used in \cite{ghicamccusker00} to show that observational
equivalence for the $IA_2$ fragment of IA is decidable -- the set of
complete plays being representable by regular expressions. In
\cite{Ong02} it was shown that it is still decidable
 for $IA_3+Y_0$. Indeed, for this fragment, the set of complete plays becomes context-free
therefore the problem reduces to the DPDA equivalence problem which
is itself decidable (with an unknown complexity).

Imposing the safety condition should lead to some improvement in
complexity. The complexity of  Safe $IA_3$ (resp. Safe $IA'_3$) for
instance, must be lower than the complexity of the DPDA equivalence
problem. Moreover the fact that Safe $IA_3$ (resp. Safe $IA'_3$)
contains terms whose denotation is context free -- e.g. $\lambda f .
f (\lambda x .x )$ -- strongly suggests that its complexity is
strictly higher than the complexity of regular language equivalence.

Murawski \cite{Murawski2003} has shown that observational
equivalence for $\ialgol_4$ is undecidable. The proofs proceeds by
showing that the computations of $\Gamma$-machine -- some variation
of queue machines that are Turing complete -- are representable
using IA terms.

Does this result extend to the safe fragments? For Safe IA, it does,
simply because the $IA_4$ term exhibited in \cite{Murawski2003} to
represent computations of the $\Gamma$-machine is also a Safe IA
term. The same argument does not carry over to Very Safe IA since
the term is not typable in this language. We do not know whether
observational equivalence is decidable for Strongly Safe $IA_4$.



\subsection{Expressivity of Safe IA/Strongly Safe IA}

Murawski representability : Safe IA representable languages are
exactly the context free languages. For Strongly Safe IA however, we
believe that the representable languages are a proper subclass of
the context free languages.













\section{Remarks}
\subsection{Homogeneity constraint}

Type homogeneity is not preserved after composition. Indeed the
types  $o \typear (o \typear o)$ and $(o \typear o) \typear \left((o
\typear o) \typear o \right)$ are homogeneous but $o \typear
\left((o \typear o) \typear o\right)$ is not.

If $A\typear B$ and $B \typear C$ are homogeneous types then  a
sufficient condition for $A\typear C$ to be homogeneous is
``$\ord{A} \geq \ord{B}$''.




\chapter{Conclusion}
    \label{chap:conclusion}
    \chapter{Further possible developments}

In the previous chapter, we have given an account of the game
semantics of Safe $\lambda$-Calculus. However the nature of this
calculus is still not well known. We propose the following possible
roadmap for further research:
\begin{enumerate}
\item prove or disprove that observational equivalence is decidable for Safe \ialgol;
\item find a categorical interpretation of the Safe $\lambda$-Calculus;
\item study the proof theory obtained by the Curry-Howard isomorphism and determine whether it has nice properties that can be helpful in theorem proving;
\item In \cite{DBLP:conf/tlca/LeivantM93}, the $\lambda$-calculus is used to
give several characterisations of the complexity class P. We would
like to investigate whether, by following similar techniques, we can
obtain a characterisation of a different complexity class using the
Safe $\lambda$-Calculus.
\end{enumerate}


In a more general direction of research, we would like to study the
class of languages for which pointers are uniquely recoverable. We
name this class PUR for ``Pointer Uniquely Recoverable''.

We proved that Safe $\lambda$-Calculus is a PUR-language. Another
example is the Serially Re-entrant Idealized Algol (SRIA) proposed
by Abramsky  in \cite{abramsky:mchecking_ia}. This language allows
multiple occurrences or uses of arguments, as long as they do not
overlap in time. In the game semantics denotation of a SRIA term
there is at most one pending occurrence of a question at any time.
Each move has therefore a unique justifier and consequently
justification pointers may be ignored. Safe \ialgol\ is not a
sublanguage of SRIA. One reason for this is that none of the two
Kierstead terms $\lambda f . f (\lambda x . f (\lambda y .y ))$ and
$\lambda f . f (\lambda x . f (\lambda y .x ))$ are Serially
Re-entrant whereas the first one is safe. Conversely, SRIA is not a
sublanguage of Safe \ialgol\ since the term $\lambda f g. f (\lambda
x . g (\lambda y .x ))$ where $f,g:((o,o),o)$ belongs to SRIA but
not to Safe \ialgol. SRIA and Safe \ialgol\ are therefore two
different examples of languages with pointer-less game semantics.

Finitary $\ialgol_2$ is also an example of PUR-language for which
observational equivalence is decidable. As we indicated in the first
chapter, decidability of observational equivalence is a very
appealing property which has immediate applications in the domain of
program verification. Intuitively, PUR-languages seem to be good
candidates of languages for which observational equivalence is
decidable. It would be interesting to discover classes of PUR
languages having this appealing property.

Another possible way to generate PUR-languages might be to constrain
the types of an existing language. In \cite{DBLP:conf/tlca/Joly01},
a notion of ``complexity'' is defined for $\lambda$-terms. It is
proved that a type $T$ can be generated from a finite set of
combinators if and only if there is a constant bounding the
complexity of every closed normal $\lambda$-term of type $T$;
consequently, the only inhabited finitely generated types are the
type of rank $\leq 2$ and the types $(A_1, A_2, \ldots, A_n, o)$
such that for all $i = 1..n$: $A_i = o$ , $A_i = o \rightarrow o$ or
$A_i = o^k \rightarrow o \rightarrow o$.

We know that imposing the first of these two type restrictions to
Finitary \ialgol\ leads to a PUR language. Is is also the case when
imposing the second type restriction?



\bibliographystyle{plain}
\bibliography{../bib/dphil-all}

\printindex

    %adds the bibliography to the table of contents
    \addcontentsline{toc}{chapter}
         {\protect\numberline{Bibliography\hspace{-96pt}}}


\end{document}

\makeindex

\hypersetup{%
  pdftitle = {The safe lambda calculus},
  pdfkeywords = {safety restriction, lambda calculus, game semantics},
  pdfauthor = {William Blum}
}

\author{William Blum}
\title{The safe lambda calculus}
\college{Linacre College}
\degree{Doctor of Philosophy}
\degreedate{?}
\renewcommand{\crest}{\beltcrest}

%\institution{Oxford University Computing Laboratory}
\date{Draft of \today}

%set the number of sectioning levels that get number and appear in the contents
\setcounter{secnumdepth}{3}
\setcounter{tocdepth}{3}

\begin{document}
\maketitle

%\setcounter{chapter}{0}
%\chapapp{Chapter}
\begin{abstract}
We consider a syntactic restriction for higher-order grammars called \emph{safety}  that  constrains occurrences of variables in the production rules according to their type-theoretic order. We transpose and generalize this restriction to the setting of the simply-typed lambda calculus, giving us what we call the \emph{safe lambda calculus}. We study this language under different angles. First we give an account of its game semantic model. For that purpose, we introduce a new concrete presentation of game semantics based on the theory of \emph{traversals}: We show that the \emph{revealed game denotation} of a term can be computed by traversing some souped-up version of the abstract syntax tree of the term using adequately defined traversal rules. This result was presented at the Galop workshop at ETAPS 2008. This allows us to give a game-semantic analysis of safety via syntactic reasoning: We show that  safe lambda-terms are denoted by what we call \emph{P-incrementally justified strategies}. This result was presented at TLCA 2007.

We study the expressivity of the calculus and show a result in the
same vein as Schwichtenberg's 1976 characterization of the
simply-typed lambda calculus, we show that the numeric functions
representable in the safe lambda calculus are exactly the
multivariate polynomials; thus conditional is not definable. We
also give a characterization of representable word functions.
We then study the complexity of deciding beta-eta equality of two safe simply-typed terms and show that this problem is PSPACE-hard.

Finally we consider extension of the safety restriction to functional languages with recursion and references such as Idealized Algol.

\end{abstract}

\begin{romanpages}
\tableofcontents
\listoffigures
\listoftables
\end{romanpages}

\listoftodos
\bigskip

%\chapter*{Acknowledgment}
%\input{acknowledgment.texi}

\chapter{Introduction}
\input{chap_introduction.texi}



\chapter{Background}
    \section{Lambda Calculus}
    \input{sec_lambdacalculus.texi}

    \section{Higher-Order Grammars and the Safety Restriction}
    \newcommand\lcalculrec{\Lambda^{\rightarrow}_\Sigma+Y}

\begin{proposition}
Higher-order recursion schemes are equivalent to the simply-typed lambda calculus extended with recursion and $\Sigma$-constants.
\end{proposition}
This is shown straightforwardly by showing that every higher-order recursion scheme can be converted into an equivalent lambda-term and conversely.

Let $\lcalculrec$ denotes the simply-typed lambda calculus extended with the typed-constants $\Sigma$ and the recursion combinator $Y$.


\begin{itemize}
\item First direction: Take a recursion scheme $\mathcal{R} = \langle \Sigma, \mathcal{N}, \mathcal{R}, S \rangle$.
We can construct an equivalent lambda term over the constant $\Sigma$ by induction on the rewriting rules as follows: we define a function $\Pi : \mathcal{A}(\Sigma,\mathcal{N}) \funto $ 

\end{itemize}

    \input{chap_languages.texi}

    \section{Game Semantics}
    \chapter{Game semantics}

The aim of this chapter is to introduce game semantics. It starts
with a history of game semantics and a presentation of the full
abstraction problem for PCF which has been solved using game
semantics. It then goes on by introducing the basic notions of game
semantics and by giving a categorical interpretation of games.
Finally we show how games are used to define a syntax-independent
model of programming languages like PCF and Idealized Algol (IA).

This chapter is largely based on the tutorial by Samson Abramsky tutorial on Game Semantics \cite{AM98a}.
Most of the proof will be omitted and we refer the reader to
\cite{hylandong_pcf, abramsky94full} for a deeper description
of game semantics with complete proofs.

\section{History}

\subsection{Game semantics}

In the 1950s, Paul Lorenzen invented Game semantics as a tool to
study semantics of intuitionistic logic \citep{lor61}.

Four decade later, Abramsky proved the full completeness of
Multiplicative Linear Logic (MLL) using game semantics
\citep{abramsky92games}. Shortly after, game semantics has been used
as tool to study models of programming languages. In game semantics,
the meaning of a program is given by a strategy in a two-player
game. One player, the Opponent, represents the environment while the
other, the Proponent, represents the system.


\subsection{Model of programming languages}

Before the 1980s, there were many approaches to define models for
programming languages. Among the successful ones, there were the
axiomatic, operational and denotational semantics:
\begin{itemize}
\item Operational semantics gives a meaning to a program by describing the
behaviour of a machine executing the program. It is defined formally
by giving a state transition system.
\item Axiomatic semantics defined the behaviour of the program
with axioms and is used to prove program correctness by static
analysis of the code of the program.
\item The denotational semantics approach consists in mapping a program to a mathematical structure
having good properties such as compositionality. This mapping is
achieved by structural induction on the syntax of the program.
\end{itemize}

In the 1990s, three different independent research groups: Samson
Abramsky, Radhakrishnan Jagadeesan and Pasquale Malacaria
\citep{abramsky94full}, Martin Hyland and Luke Ong
\citep{hylandong_pcf} and Nickau \citep{Nickau:lfcs94} have
introduced game semantics, a new kind of semantics, in order to
solve a long standing problem in the semanticists community :
finding a fully abstract model for PCF.

\subsection{The problem of full abstraction for PCF}

PCF is a simple programming language introduced in a classical paper
by Plotkin ``LCF considered as a programming language''
(\cite{DBLP:journals/tcs/Plotkin77}). PCF is based on LCF, the Logic
of Computable Functions devised by Dana Scott in \cite{scott_lcf}.
It is a simply typed lambda calculus extended with arithmetic
operators, conditional and recursion.

The problem of the Full Abstraction for PCF goes back to the 1970s.
In \citep{scott93}, Scott gave a model for PCF based on domain
theory. This model gives a sound interpretation of observational
equivalence: if two terms have the same domain theoretic
interpretation then they are observationally equivalent. However the
converse is not true: there exist two PCF terms which are
observationally equivalent but have different domain theoretic
denotation. We say that the model is not fully abstract.

The key reason why the domain theoretic model of PCF is not fully
abstract is that the parallel-or operator defined by the following
truth table
\begin{center}
\begin{tabular}{l|lll}
p-or  & $\bot$ & tt & ff \\ \hline
$\bot$ & $\bot$ & tt & $\bot$\\
tt & tt & tt & tt\\
ff & $\bot$ & tt & ff\\
\end{tabular}
\end{center}
is not definable as a PCF term! It is possible to create two
different PCF terms that always behave the same except when they are
apply to a term computing p-or. Since p-or is not definable in PCF,
these two terms will have the same denotation. This implies that the
model is not fully abstract.

One can patch PCF by adding the operator $p-or$, the resulting
language ``PCF+p-or'' now becomes fully-abstracted by Scott domain
theoretic model \citep{DBLP:journals/tcs/Plotkin77}. However the
language we are now dealing with is strictly more powerful than PCF,
it allows parallel execution of commands whereas PCF only permits
sequential execution.

Another approach consists in getting rid of the undefinable elements
(like p-or) by strengthening the conditions on the function used in
the model (a condition stronger than strictness and continuity) but
unfortunately this approach did not succeed.

The only successful approaches to obtain a fully abstract model for
PCF were the ones taken by Ambramsky, Jagadeesan and Malacaria
\citep{abramsky94full}, Hyland and Ong \citep{hylandong_pcf} and
Nickau \citep{Nickau:lfcs94}, all based on game semantics.

This result has then been adapted to other varieties of programming
paradigm including languages with stores (Idealized Algol),
call-by-value \citep{honda99gametheoretic, abramsky98callbyvalue}
and call-by-name, general referencees
\citep{DBLP:conf/lics/AbramskyHM98}, polymorphism
\citep{DBLP:journals/apal/AbramskyJ05}, control features
(continuation and exception), non determinism, concurrency. In all
these cases, the game semantics model led to a syntax-independent
fully abstract model of the corresponding language.

\section{Games}
\label{sec:catgames}

We now introduce formally the notion of game that will be used in
the following section to give a model of the programming languages
PCF and Idealized Algol. The definitions are taken from
\cite{abramsky:game-semantics, hylandong_pcf, abramsky94full}.


\subsection{Arenas and Games}

The games we are interested in are two-players games. The players are named O for Opponent and P for Proponent.

The game played by O and P is constraint by something called
\emph{arena}. The arena defines the possible moves of the game. By
analogy with real board games, the arena represents the board
together with the rules that tell how players can make their moves
on the board. In fact the analogy with board game stops here. Our
games can be thought as dialog games: one person O interviews
another person P, P tries to answer the initial O-question by
possibly asking O some precisions about its initial question.
Moreover, the notion of winner and winning strategy will not be
relevant in our setting.


More formally, the arena can be seen as a forest of trees whose nodes are possible questions and leaves are possible answers.
The arena is partitioned into two kinds of moves: the moves that can be played by P and the ones that can be played by O.
A move is either a question to the other player or an answer to a question previously asked by the other player.

Each move of the game must be justified by another move that has already been played by the other player. This justification relation
is induced by the edges of the forest arena. Moreover, an answer must always be justified by the question that it answers and a question
is always justified by another question.

\begin{dfn}[Arena]
An arena is a structure $\langle M, \lambda, \vdash \rangle$ where:
\begin{itemize}
\item $M$ is the set of possible moves;
\item $(M,\vdash)$ is a forest of trees;

\item $\lambda : M \rightarrow \{ O, P\} \times \{Q, A\}$ is a labeling functions indicating whether a given move
    is a question or an answer and whether it can be played by O or by P.

    $\lambda = [\lambda^{OP},\lambda^{QA}]$ where $\lambda^{OP} : M \rightarrow  \{ O, P\}$
    and $\lambda^{QA} : M \rightarrow  \{ Q, A\}$.

    \begin{itemize}
    \item If $\lambda^{OP} (m) = O$, we call $m$ and O-move otherwise $m$ is a P-move.
    $\lambda^{QA} (m) = Q$ indicates that $m$ is a question otherwise $m$ is an answer.

    \item For any leaf $l$ of the tree $(M,\vdash)$, $\lambda^{QA} (l) = A$ and for any node
    $n \in (M,\vdash)$, $\lambda^{QA} (n) = Q$.
    \end{itemize}

\item The forest of tree $(M,\vdash)$ respect the following condition:
    \begin{itemize}
    \item[(e1)] The roots are O-moves: for any root $r$ of $(M,\vdash)$, $\lambda^{OP} (r) = O$.
    \item[(e2)] Answers are enabled by questions: $m \vdash n  \zand \lambda^{QA}(n) = A \imp \lambda^{QA}(m) = Q$.
    % Or more succinctly, if we write $\dashv$ the relation $\vdash^-1$: $\lambda^{QA} \left( \dashv( (\lambda^{QA})^{-1}(\{A\}) ) \right) = \{ O \}$
    \item[(e3)] A player move must be justified by a move played by the other player:
         $m\vdash n \imp \lambda^{OP}(m) \neq \lambda^{OP}(n)$.
    \end{itemize}
\end{itemize}
\end{dfn}

For commodity we write the set $\{O,P\} \times \{Q,A\}$ as $\{OQ,OA,PQ,PA\}$.
$\overline{\lambda}$ denotes the labeling function $\lambda$ with the question and answer swapped. For instance:
$$\overline{\lambda(m)} = OQ \iff \lambda(m) = PQ$$

The roots of the forest of tree $(M,\vdash)$ are the \emph{initial moves}.

For example, the simplest possible arena is written $\mathbf{1}$ and
denotes the arena which set of moves $M$ is empty.

\begin{exmp}[The flat arena]
\label{exmp:flatarena}

 Let $A$ be any countable set then the flat arena over $A$
is defined to be the arena $\langle M, \lambda, \vdash \rangle$ such
that $M$ has one move $q$ with $\lambda(q) = OQ$ and for each
element in $A$, there is a corresponding move $a_i$ in $M$ with
$\lambda(a_i) = PA$ for some $i \in \nat$. The enabling relation
$\vdash$ is defined to be $\{ q \vdash a_i \ | i \in \nat \}$.

This arena is represented by the following tree:
\begin{center}
  \pstree[levelsep=6ex]
    { \TR{$q$} }
    {    \TR{$a_1$} \TR{$a_2$} \TR{\ldots} }
\end{center}
The vertices represent the moves and the edges represent the
enabling relation.

The flat arena over $\nat$ and $\mathbb{B}$ is written
$\mathbf{int}$ and  $\mathbf{bool}$ respectively.

\end{exmp}

Once the arena has been defined, the bases of the game are set and the players have something to play with.
We now need to describe the state of the game, for that purpose
we introduced \emph{justified sequences of moves}. Sequence of moves are used to record the history of all the moves that have been
played.

\begin{dfn}[Justified sequence of moves]
A justified sequence is a sequence of moves $s$ together with an associated sequence of pointers. Any
move $m$ in the sequence that is not initial has as pointer that points to a previous move $n$ that justifies it (i.e. $n \vdash m$).
\end{dfn}

The pointers of a justified sequences are represented with arrows.
This is an example of justified sequence of moves:
$$\rnode{q4}{q}^4
\rnode{q3}{q}^3 \rnode{q2}{q}^2 \rnode{q3b}{q}^3 \rnode{q2b}{q}^2
\rnode{q1}{q}^1 \bkptrc{q3}{q4} \bkptrc{q2}{q3}
\bkptrc[ncurv=0.6]{q3b}{q4} \bkptrc{q2b}{q3b}$$

The first move of a justified sequence must be an O-move since
initial moves are all O-moves.

Notation: we write $s t$ or sometimes $s \cdot t$ do denote the
sequences obtain by concatenating $s$ and $t$. The empty sequence is
written $\epsilon$.

 A justified sequence has two particular subsequences which
will be of particular interest later on when we introduce
strategies. These subsequences are called the P-view and the O-view
of the sequence. The idea is that a view describes the local context
of the game. Here is the formal definition:

\begin{dfn}[View]
Given a justified sequence of moves $s$. We define the proponent view (P-view) noted $\pview{s}$ by induction:
\begin{align*}
\pview{\epsilon} &= \epsilon \\
\pview{s \cdot m} &= \pview{s} \cdot \ m && \mbox{ if $m$ is a P-move} \\
\pview{s \cdot m} &= m && \mbox{ if $m$ is initial (O-move) } \\
\pview{ s \cdot \rnode{m}{m} \cdot t \cdot \rnode{n}{n} \bkptra{50}{n}{m} } &=
 \pview{s} \cdot \rnode{mm}{m} \cdot \rnode{nn}{n} \bkptra{70}{nn}{mm} && \mbox{ if $n$ is a non initial O-move }
\end{align*}
The O-view $\oview{s}$ is defined similarly:
\begin{align*}
\oview{\epsilon} &= \epsilon \\
\oview{s \cdot m} &= \oview{s} \cdot \ m && \mbox{ if $m$ is a O-move} \\
\oview{ s \cdot \rnode{m}{m} \cdot t \cdot \rnode{n}{n} \bkptra{50}{n}{m} } &=
 \pview{s} \cdot \rnode{mm}{m} \cdot \rnode{nn}{n} \bkptra{70}{nn}{mm} && \mbox{ if $n$ is a P-move }
\end{align*}
\end{dfn}

In fact not all justified sequences will be of interest for the
games that we will use. We call \emph{legal position} any justified
sequence verifying two additional conditions: alternation and
visibility. Alternation says that players O and P plays
alternatively. Visibility expresses that each non-initial move is
justified by a move situated in the local context at that point.
Intuitively, the visibility condition gives some coherence to the
justification pointers of the sequence.

\begin{dfn}[Legal position]
A legal position is a justified sequence of move $s$ respecting the following constraint:
\begin{itemize}
\item Alternation: For any subsequence $m \cdot n$ of $s$, $\lambda^{OP}(m) \neq \lambda^{OP}(n)$.
\item Visibility: For any subsequence $t m$ of $s$ where $m$ is not initial, if $m$ is a P-move then $m$ points to a move in $\pview{s}$
and if $m$ is a O-move then $m$ points to a move in $\oview{s}$.
\end{itemize}

The set of legal position of an arena $A$ is noted $L_A$.
\end{dfn}

We say that a move $n$ is hereditarily justified by a move $m$ if there is a sequence of move
$m_1, \ldots, m_q$ such that:
$$ m \vdash m_1 \vdash m_2 \vdash \ldots m_q \vdash n$$
If a move has no justification pointer, we says that it is an
\emph{initial move} (in that case it must be a root of the forest
arena).

Suppose that $n$ is an occurrence of a move in the sequence $s$ then
$s \upharpoonright n$ denotes the subsequence of $s$ containing all the moves hereditarily justified by $n$.
Similarly, $s \upharpoonright I$ denotes the
subsequence of $s$ containing all the moves hereditarily justified by the moves in $I$.

\begin{dfn}[Game]
A game is a structure $\langle M, \lambda, \vdash, P \rangle$ such that
\begin{itemize}
\item $ \langle M, \lambda, \vdash \rangle$ is an arena.
\item $P$ is called the set of valid positions, it is:
    \begin{itemize}
    \item a non-empty prefix closed subset of the set of legal position
    \item closed by initial hereditary filtering: if $s$ is a valid position then for any set $I$ of occurrences of initial moves
    in $s$, $s\upharpoonright I$ is also a valid position.
    \end{itemize}
\end{itemize}
\end{dfn}

\begin{exmp}  Consider the flat arena  $\mathbf{int}$.
The set of valid position $P = \{ \epsilon, q \} \union \{ q \cdot
a_i \ | i \in \nat \}$ defines a game on the arena $\mathbf{int}$.
\end{exmp}

\subsection{Constructions on games}
\label{sec:gameconstruction}

We now define game constructors that will be useful later on.

Consider the two functions $f : A \rightarrow C$ and $g : B
\rightarrow C$, we write $[f,g]$ to denote the pairing of $f$ and
$g$ defined on the direct sum $A + B$. Given a game $A$ with a set
of moves $M_A$, we use the filtering operator $s \upharpoonright A$
do denote the subsequence of $s$ consisting of all moves in $M_A$.
Although this notation conflicts with the hereditarily filtering
operator, it should not cause any confusion.

\subsubsection{Tensor product}
Given two games $A$ and $B$ we define the tensor product constructor
$A \otimes B$ as follows:
\begin{eqnarray*}
  M_{A \otimes B} &=& M_A + M_B \\
  \lambda_{A\otimes B} &=& [\lambda_A,\lambda_B] \\
  \vdash_{A\otimes B} & = & \vdash_{A}\ \union\ \vdash_{B} \\
  P_{A\otimes B} & = & \{ s \in L_{A\otimes B} | s \upharpoonright A \in P_A \wedge s \ \upharpoonright B \in P_B  \}.
\end{eqnarray*}

In particular,  $n$ is initial in $A\otimes B$ if and only if $n$ is
initial in A or B. And $m \vdash_{A\otimes B} n$  holds if and only if $m
\vdash_{A} n$ or $m \vdash_{B} n$ holds.

\subsubsection{Function space}
The game $A \otimes B$ is defined as follows:
\begin{eqnarray*}
  M_{A \multimap B} &=& M_A + M_B \\
  \lambda_{A\multimap B} &=& [\overline{\lambda_A},\lambda_B] \\
  \vdash_{A\multimap B} & = & \vdash_{A}\ \union\ \vdash_{B}\ \union\  \{ (m,n) \ |\ m \mbox{ initial in } B \wedge n \mbox{ initial in } A \} \\
  P_{A\otimes B} & = & \{ s \in L_{A\otimes B} | s \upharpoonright A \in P_A \wedge s \ \upharpoonright B \in P_B  \}.
\end{eqnarray*}

\subsubsection{Cartesian product}
The game $A \& B$ is defined as follows:
\begin{eqnarray*}
  M_{A \& B} &=& M_A + M_B \\
  \lambda_{A\& B} &=& [\lambda_A,\lambda_B] \\
  \vdash_{A\& B} & = & \vdash_{A}\ \union\ \vdash_{B} \\
  P_{A\& B} & = & \{ s \in L_{A\otimes B} | s \upharpoonright A \in P_A \wedge s \ \upharpoonright B = \epsilon  \} \\
        &&   \union \{ s \in L_{A\otimes B} | s \upharpoonright A \in P_B \wedge s \ \upharpoonright A = \epsilon  \}.
\end{eqnarray*}

A play of the game $A \& B$ is either a play of $A$ or a play of $B$ whether a play
of the game $A \otimes B$ may be an interleaving of plays on $A$ and plays on $B$.

\subsection{Representation of plays}

Plays of the game are usually represented in a table diagram. The
columns of the table correspond to the different components of the
arena and each row corresponds to one move in the play. The first
row always represents an O-move, this is because O is the only
player who can open a game (since roots of the arena are O-moves).

As an example the play
$$\rnode{q1}{q}\
 \rnode{q2}{q}
 \ \rnode{a2}{8}
\  \rnode{a1}{12}
  \bkptrc{a1}{q1}
\bkptrc{a2}{q2} $$
on the
game $\textbf{int} \multimap \textbf{int} $ can be represented by
the following diagram:

\begin{center}
\begin{tabular}{cccc}
\textbf{int} & $\imp$ & \textbf{int} & \\
&& q & O\\
q  &&& P\\
8  &&& O\\
&& 12 & P
\end{tabular}
\end{center}

When it is necessary, the justification pointers of the play can also
be shown on the diagram.


\subsection{Strategy}

\subsubsection{Definition}

During a game, the player who has to play may have several choices
for his next move. The move that he makes is chosen according to a
given strategy.

A strategy is a rule telling the player which move to make when the
game is in a given position. More abstractly, a strategy is a
partial function mapping legal position where Proponent has to move
to P-moves.

\begin{dfn}[Strategy]
A strategy for player P on a given game $\langle M, \lambda, \vdash, P \rangle$ is a
non-empty set of even-length positions from $P$ such that:
\begin{enumerate}
\item (\emph{no unreachable position}) $sab \in \sigma \imp s \in \sigma$
\item (\emph{determinacy}) $sab, sac \in \sigma \quad \imp \quad  b = c$  and $b$ has the same justifier as
$c$.
\end{enumerate}
\end{dfn}

The idea is that the presence of the even-length sequence $s a b$ in
$\sigma$ tells the player P that whenever the game is in position
$s$ and player O plays the move $a$ then it must respond by playing
the move $b$.

The first condition ensures that the strategy $\sigma$ only
considers positions that the strategy itself could have led to in a
previous move. The second condition in the definition requires that
this choice of move is deterministic (i.e. there is a function $f$
from the set of odd length position to the set of moves $M$ such
that $f(s a) = b$).


For any game $A$, the smallest possible strategy is the strategy
that never respond given by $\{ \epsilon \}$. It is called the
\emph{empty strategy} and denoted $\bot$.

\subsubsection{Copy-cat strategy}

For any arena $A$ there is a strategy on the game $A \multimap A$
called the \emph{copy-cat strategy}. We write $A_1$ and $A_2$ to
denote the first and second copy of the arena $A$ in the game $A
\multimap A$. If $A$ is the arena $A_1$ then $A^\perp$ denotes the
arena $A_2$ and reciprocally.

Let $A$ be one of the arena $A_1$ or $A_2$. The copy-cat strategy
operates as follows: whenever P has to respond to an O-move played
in $A$, it replicates the move played by O in the arena $A^{\perp}$
after that $O$ has to respond in $A^{\perp}$ and $P$ replicates this
response in $(A^\perp)^\perp = A$ and so on and so forth.


More formally, the copy-cat strategy is defined by:
$$ \textsf{id}_A = \{ s \in P^{\textsf{even}}_{A \multimap A} \ | \ \forall t \sqsubseteq^{\textsf{even}} s\ .\ t \upharpoonright A_1 = t \upharpoonright A_2 \}$$
where $P^{\textsf{even}}_A$ denotes the valid position of even
length in the game $A$ and $t \sqsubseteq^{\textsf{even}} s$ denotes
that $t$ is an even length prefix of $s$.

The copy-cat strategy is also called \emph{identity strategy} since
it is the identity for strategy composition as we will see in the
next paragraph.

\begin{exmp} The copy-cat strategy on $\textbf{int}$ is:
$$\begin{array}{ccc}
\textbf{int} & \imp & \textbf{int} \\
&& q\\
q \\
n \\
&& n
\end{array}
$$
Note that we introduced this type of diagram to represent plays of
games but, as we can see here, the same diagrams can be used to
represent strategies when the play represented is general enough.

The copy-cat strategy on $\textbf{int} \typar \textbf{int}$ is given
by the following diagram:
$$\begin{array}{ccccccc}
(\textbf{int} & \imp & \textbf{int}) & \imp & (\textbf{int} & \imp & \textbf{int}) \\
&&&& && q\\
&& q\\
q \\
&&&& q \\
&&&& m \\
m\\
&& n \\
&&&& && n
\end{array}$$
\end{exmp}

\subsubsection{Composition}

It is well-known that any model of the simply typed lambda-calculus
is a cartesian closed category \citep{CroleRL:catt}. Games are used
to give a fully-abstract model of PCF, an extended simply typed
lambda calculus, therefore the game model should fit into a
cartesian closed category. This category will have games as objects
and strategies as morphisms. In a category, morphisms should be able
to compose together, therefore there should be an appropriate notion
of strategy composition.

Composition of strategies is an essential feature of game semantics.
As we will see in the following section, in the game model of PCF,
strategies represent programs. Therefore, strategy composition will
prove to be very useful : obtaining the model of a composed program
boils down to composing the strategies of the composing programs.

The way composition is defined for strategies is similar to
``parallel composition plus hiding'' in the trace semantics of CSP
\citep{hoare_csp}. Consider two strategies $\sigma : A \multimap B$
and $\tau : B \multimap C$ that we wish to compose.

For any sequence of moves $u$ on three arenas $A$, $B$, $C$, we call
projection of $s$ on the game $A \multimap B$ and we note $u
\upharpoonright A,B$ the subsequence of $s$ obtained by removing
from $u$ the moves in $C$ and pointers to moves in $C$. The
projection on $B \multimap C$ is defined similarly.

The definition of the projection on $A \multimap B$ differs
slightly: $u \upharpoonright A,C$ is the subsequence of $u$
consisting of the moves from $A$ and $C$ with some additional
pointers: we add a pointer from $a \in A$ to $c\in C$ whenever $a$
points to some move $b \in B$ itself pointing to $c$. All the
pointers to moves in $B$ are removed.


First we remark that for a given legal position $s$ in the game $A
\multimap C$, there is what is called an \emph{uncovering} of $s$.
The uncovering of $s$ is the maximal justified sequence of moves $u$
from the games $A$, $B$ and $C$ such that:
\begin{itemize}
\item The sequence $s$, considered as a pointer-less sequence, is a subsequence of
$u$;
\item the projection of $u$ on the game $A \multimap B$ lies in the
strategy $\sigma$;
\item the projection of $u$ on the game $B \multimap C$
lies in the strategy $\tau$;
\item and the projection of $u$ on the game $A \multimap C$ is a subsequence of $s$ (here the term ``subsequence'' refers to the sequence of nodes together with the auxiliary sequence of pointers).
\end{itemize}
This uncovering, noted $uncover(s, \sigma, \tau)$, is
defined uniquely for given strategies $\sigma$, $\tau$ and legal
position $s$ (this is proved in part II of \cite{hylandong_pcf}).

We define $\sigma \| \tau $ to be the set of uncovering of legal
positions in $A \multimap C$:
$$ \sigma \| \tau = \{ uncover(s, \sigma, \tau) \ | \ s \mbox{ is a legal position in } A \multimap C \}$$

The composition of $\sigma$, $\tau$ is defined to be the set of
projections of uncovering of legal positions in $A \multimap C$:

\begin{dfn}[Strategy composition]
Consider $\sigma : A \multimap B$ and  $\tau : B \multimap C$ two
strategies. We define $\sigma ; \tau$ to be:
$$ \sigma ; \tau = \{ u \upharpoonright A,C \ | \ u \in \sigma \|
\tau \}$$
\end{dfn}

It can be verified that composition is well-defined and associative
\citep{hylandong_pcf} and that the copy-cat strategy $\textsf{id}_A$ is the identity for composition.

\subsubsection{Constraint on strategies}

Different classes of strategies will be considered depending on the
features of the language that we want to model. Here is a list of
common restrictions that we will consider:
\begin{itemize}
\item \emph{Well-bracketing:} In a well-bracketed strategies the players always answer the last unanswered question (called the pending question) first.
If we represent Opponent's question as ``['', Proponent's answer as
``]'', Proponent's question as ``('' and Opponent's answers as ``)''
then requiring that the last pending question is answered first is
the same as requiring that the string representing the play is a
prefix of a well-bracketed sequence.

\item \emph{History-free strategies:} A strategy is history-free if the Proponent's move at any position of the game where he has to play
is determined by the last move of the Opponent. In other words, the
history prior to the last move is ignored by the Proponent when
deciding how to respond.

\item \emph{History-sensitive strategies:} The Proponent follows a history-sensitive strategy if he needs to have access to the full
history of the moves in order to decide which move to make.

\item \emph{Innocence:} a strategy is innocent if it determines Proponent's moves based on a restricted view of the history of the play, mainly the P-view
at that point. Such strategies can be specified by a partial
function mapping P-views to P-moves. However not every partial
function from P-views to P-moves gives rise to an innocent strategy
(a sufficient condition is given in \cite{hylandong_pcf}).
\end{itemize}

The formal definition of innocence follows:
\begin{dfn}[Innocence]
Given positions $sab, ta \in L_A$ where $sab$ has even length and
$\pview{sa} = \pview{ta}$, there is a unique extension of $ta$ by
the move $b$ together with a justification pointer such that
$\pview{sab} = \pview{sa}$. We write this extension
$\textsf{match}(sab,ta)$.

The strategy $\sigma:A$ is \emph{innocent} if and only if:
$$ \left(
     \begin{array}{c}
       \pview{sa} = \pview{ta} \\
       sab \in \sigma \\
       t\in \sigma \wedge ta \in P_A \\
     \end{array}
   \right)
\quad \imp\quad  \textsf{match}(sab,ta) \in \sigma$$

\end{dfn}


\subsection{Categorical interpretation of games}

In this section we recall some results about the categorical representation of Games.
These results with complete details and proofs can be found in \cite{McC96b,hylandong_pcf,abramsky94full}.
We refer the reader to \cite{CroleRL:catt} for more information about category theory.

We consider the category $\mathcal{G}$ whose objects are games and morphisms are
strategies. A morphism from $A$ to $B$ is a strategy on the game $A \multimap B$.

Three other sub-categories of $\mathcal{G}$ are considered: each of them correspond to some restriction on strategies:
$\mathcal{G}_i$ is the sub-category
of $\mathcal{G}$ whose morphisms are the innocent strategies,
$\mathcal{G}_b$ has only the well-bracketed strategies and $\mathcal{G}_{ib}$ has the innocent and well-bracketed strategies.

\begin{prop}
$\mathcal{G}$, $\mathcal{G}_i$, $\mathcal{G}_b$ and $\mathcal{G}_{ib}$ are categories.
\end{prop}

Proving this requires to prove that composition of strategies is well-defined, associative, has a unit (the copy-cat strategy), preserves innocence and
well-bracketedness. See \cite{hylandong_pcf,abramsky94full} for a proof.


\subsubsection{Monoidal structure}

We have already defined the tensor product on games in section \ref{sec:gameconstruction}.
We now define the corresponding transformation on morphisms:
given two strategies $\sigma : A \multimap B$ and $\tau : C \multimap D$ the strategy
$\sigma \otimes \tau : (A \otimes C) \multimap (B\otimes D)$ is defined by:
$$ \sigma \otimes \tau = \{ s \in L_{A \otimes C \multimap B\otimes D} \ s \upharpoonright A,B \in \sigma
\wedge s \upharpoonright C,D \in \tau \}$$

It can be shown that the tensor product is associative, commutative and has
$I = \langle \emptyset, \emptyset,\emptyset, \{ \epsilon \} \rangle $ as identity.
Hence the game categories $\mathcal{G}$ is a symmetric monoidal categories. Moreover
$\mathcal{G}_i$ and  $\mathcal{G}_b$ are sub-symmetric monoidal categories of $\mathcal{G}$,
and $\mathcal{G}_{ib}$ is a sub-symmetric monoidal category of $\mathcal{G}_i$, $\mathcal{G}_b$ and
$\mathcal{G}$.

\subsubsection{Closed structure}

For any game $A$, $B$ and $C$,
to any strategy $\sigma : A\otimes B \multimap C$, there is a corresponding strategy
$\tau : A\otimes B \multimap C$ obtained by relabeling the moves in $\sigma$. This transformation
is in fact an isomorphism: the hom-set $\mathcal{G}(A\otimes B, C)$ is isomorphic to the hom-set
$\mathcal{G}(A,B\multimap C)$. Hence $\mathcal{G}$ is an autonomous (i.e. symmetric monoidal closed) category.

$\mathcal{G}_i$ and  $\mathcal{G}_b$ are sub-autonomous categories of $\mathcal{G}$,
and $\mathcal{G}_{ib}$ is a sub-autonomous category of $\mathcal{G}_i$, $\mathcal{G}_b$ and
$\mathcal{G}$.

\subsubsection{Cartesian product}
The cartesian product defined in section \ref{sec:gameconstruction} is indeed a cartesian product in the category
$\mathcal{G}$, $\mathcal{G}_i$, $\mathcal{G}_b$ and $\mathcal{G}_{ib}$.

The projections $\pi_1:A \& B \rightarrow A$ and $\pi_1:A \& B \rightarrow B$ are given by the obvious copy-cat strategies.
Given two category morphisms $\sigma :C \rightarrow A$ and $\tau : C \rightarrow B$ the pairing function
$\langle \sigma, \tau \rangle : C \rightarrow A \& B$ is given by:
\begin{eqnarray*}
\langle \sigma, \tau \rangle &=& \{ s \in L_{C\multimap A\&B} \ | \ s \upharpoonright C,A \in \sigma \wedge s \upharpoonright B = \epsilon  \} \\
&\union& \{ s \in L_{C\multimap A\&B} \ | \ s \upharpoonright C,A \in \sigma \wedge s \upharpoonright B = \epsilon  \}
\end{eqnarray*}

\subsubsection{Cartesian closed structure}
Having defined the cartesian product is not enough to turn $\mathcal{G}$ into a cartesian closed category :
we also need to define a terminal object $I$ and the exponential construct $A \imp B$ for any two games $A$ and $B$.
In fact, this cannot be done in the current categories $\mathcal{G}$ and we have to move on to another category
of games noted $\mathcal{C}$ whose objects and morphisms are certain sub-classes of games and strategies.

Before introducing the category $\mathcal{C}$ we need some new definitions:


For any game $A$ we define the exponential game noted $!A$.
The game $!A$ corresponds to a repeated version of the game $A$. Plays of $!A$ are interleaving of plays of
$A$. It is defined as follows:
\begin{eqnarray*}
  M_{!A} &=& M_A \\
  \lambda_{!A} &=& \lambda_A \\
  \vdash_{!A} & = & \vdash_{A} \\
  P_{!A} & = & \{ s \in L_{!A} | \mbox{ for each initial move $m$, } s \upharpoonright m \in P_A \}
\end{eqnarray*}
The following equalities hold:
\begin{eqnarray*}
  !(A \& B) &=& !A \otimes !B\\
  I &=& !I
\end{eqnarray*}

\begin{dfn}[Well-opened games]
A game $A$ is well-opened if for any position $s \in P_A$ the only initial move is the first
one.
\end{dfn}

Well-opened games have single thread of dialog. Then can be turned into games with multiple-thread of dialog
using the promotion operator:

\begin{dfn}[Promotion]
Consider a well-opened game $B$.
Given a strategy on ${!A} \multimap B$, we define it promotion $\sigma^\dagger : {!A} \multimap {!B}$ to be the
strategy which plays several copies of $\sigma$. It is formally defined by:
$$ \sigma^\dagger = \{ s \in L_{{!A} \multimap !B} \ | \ \mbox{ for all initial $m$, } s \upharpoonright m \in \sigma  \}.$$
\end{dfn}

It can be shown that promotion is well-defined (it is indeed a strategy) and that it preserves innocence and
well-bracketedness.


We now introduce the category of well-opened games:
\begin{dfn}[Category of well-opened games]
The category $\mathcal{C}$ of well-opened games is defined as follow:
\begin{enumerate}
\item The objects are the well-opened games,
\item a morphism $\sigma : A \rightarrow B$ is a strategy for the game $!A \multimap B$,
\item the identity map for $A$ is the copy-cat strategy on $!A \multimap A$ (which is well-defined for well-opened games).
It is called dereliction, noted
$\textsf{der}_A$ and defined formally by:
$$ \textsf{der}_A = \{ s \in P^{\textsf{even}}_{{!A} \multimap A} \ | \ \forall t \sqsubseteq^{\textsf{even}} s \ . \ t \upharpoonright {!A} = t \upharpoonright A \},$$
\item composition of morphisms $\sigma : {!A} \multimap B$ and $\tau : {!B} \multimap C$ is defined to be
the strategy $\sigma^\dagger;\tau$ on the game ${!A} \multimap C$.
\end{enumerate}
\end{dfn}
$\mathcal{C}$ is a well-defined category and the three sub-categories
$\mathcal{C}_i$, $\mathcal{C}_b$, $\mathcal{C}_{ib}$ corresponding to sub-category
with innocent strategies, well-bracketed strategies and innocent and well-bracketed strategies respectively.


The category $\mathcal{C}$ has a terminal object $I$, for any two games $A$ and $B$ a product $A \& B$ and
an exponential $A \imp B$. Moreover the hom-sets $\mathcal{C}(A \& B,C)$ and
$\mathcal{C}(A,!B \multimap C)$ are isomorphic. Indeed:
\begin{eqnarray*}
\mathcal{C}(A\& B,C) &=& \mathcal{G}(!(A\& B),C) \\
&=& \mathcal{G}({!A}\otimes {!B}),C) \\
&\cong& \mathcal{G}({!A}, {!B} \multimap C) \qquad  \mbox{($\mathcal{G}$ is a closed monoidal category)}\\
&=& \mathcal{C}(A, {!B} \multimap C)
\end{eqnarray*}
Hence $\mathcal{C}$ is a cartesian closed category. Moreover $\mathcal{C}_i$ and $\mathcal{C}_b$
are sub-cartesian closed caterogies of $\mathcal{C}$ and $\mathcal{C}_{ib}$ is as sub-cartesian closed category
of each of $\mathcal{C}$, $\mathcal{C}_i$ and $\mathcal{C}_b$.



\subsubsection{Order enrichment}

Strategies can be ordered using the inclusion ordering.
The set of strategies on a given game $A$ is a pointed directed complete partial order under this ordering: the
least upper bounds is the union of two strategies and the least element is the empty strategy $\{ \epsilon \}$.

The category  $\mathcal{C}$ and  $\mathcal{G}$ are cpo-enriched.





directe It is possible to define an order on strategies


\subsection{Arena of order at most 2}
In this section, we consider a restricted class of arena and prove a
property on the games played on these arenas.

The height of the arena is the length of the longest sequence of moves
$m_1 \ldots m_h$ in $M$ such that $m_1 \vdash m_2 \vdash \ldots \vdash m_h$.

The order of an arena $\langle M, \lambda, \vdash \rangle$ is defined to be
$h-2$ where $h$ is the height of the forest of trees $(M, \vdash)$.


\begin{lem}[Pointers are superfluous up to order 2]
Let $A$ be the arena of order at most 2. Let $s$ be a justified sequence of moves in the arena $A$ satisfying
 alternation, visibility and well-bracketing then
the pointers of the sequence $s$ can be reconstructed uniquely.
\end{lem}



\begin{proof}
In the graphic representation of the arena, we display the sub-arena by decreasing order of sub-arena order.
It is safe to do so since in the definition of the forest of tree of an arena, the children nodes
are not ordered.

Let $A$ be an arena of order 2. We assume that $A$ has only one root. The arena $A$ has therefore the following shape:
\begin{center}
\
  \pstree[levelsep=6ex]
    { \TR{$q$} }
    {
\SubTree{$T_1$} \SubTree[linestyle=none]{$\ldots$} \SubTree{$T_n$}
    \TR{$a_1$} \TR{$a_2$} \TR{\ldots} }
\end{center}

where each triangle $T_i$ represents an arena of order 0 or 1.

We will see that the following proof can easily be adapted to take into account the general case of forest arenas (multiple roots).

We write $I_k$, for $k=0$ or $1$, the set of indices $i$ such that the arena $T_i$ has order $k$:
$$I_k = \{ i \in 1.. n\ |\ \order{T_i} = k \}$$

Here is a graphic representation of the arenas $T_i$ for $i \in I_0$ and $T_j$ for $j \in I_1$:
\begin{center}
\
  \pstree[levelsep=6ex]
    {\TR{$q^i$}}
    { \TR{$a_1^i$} \TR{$a_2^i$} \TR{\ldots} }
\hspace{2cm}
  \pstree[levelsep=6ex]
    { \TR{$p^j$} }
    {
      \pstree[levelsep=6ex]
        { \TR{$q^j$} }
        { \TR{$a_1^j$} \TR{$a_2^j$} \TR{\ldots} }
      \TR{$b_1^j$} \TR{$b_2^j$} \TR{\ldots}
    }
\end{center}



For any justified sequence of moves $u$, we write $?(u)$ for the
subsequence of $u$ consisting of the questions in the sequence $u$
that are still pending at the end of the sequence.

Let $L$ be the following language $L = \{\ p^i q^i\ | \ i \in I_1
\}$. We consider the following cases:

\begin{center}
\begin{tabular}{c|c|l|l}
Case & $\lambda_{OP}(m)$ & $?(u) \in$ & condition \\ \hline
0 & O & $\{ \epsilon \}$ \\
A & P & $q$ \\
B & O & $q \cdot L^* \cdot p^i$     & $i \in I_1$ \\
C & P & $q \cdot L^* \cdot p^i q^i$ & $i \in I_1$ \\
D & O & $q \cdot L^* \cdot q^i$      & $i \in I_0$ \\
\end{tabular}
\end{center}

We use the notation $\hat{s}$ to denote a legal and well-bracketed
\emph{justified} sequence of moves and $s$ to denote the same
sequence of moves with pointers removed.

Note that the well-bracketing condition already tells us how to
uniquely recover the pointers for P answer moves: a P-answers points
to the last pending question having the same tag. However for O
answers, we will see that the visibility condition already ensures
the unique recoverability of the pointer and that the
well-bracketing condition is not needed.


We prove by induction on the sequence of moves $u$ that $?(u)$
corresponds to either case 0, A, B, C or D and that the pointers in
$u$ can be recovered uniquely.

\textbf{Base cases:}

If $u$ is the empty sequence $\epsilon$ then there is no pointer to
recover and it corresponds to case 0.

If $u$ is a singleton then it must be the initial question $q$ and
there is not pointer to recover. This corresponds to case A.

\textbf{Step case:}

Consider a legal well-bracketed justified sequence $\hat{s}$ where
$s = u \cdot m$ and $m \in M_A$. The induction hypothesis tells us
that the pointers of $u$ can be recovered (and therefore the P-view
or O-view at that point can be computed) and that $u$ corresponds to
one of the cases 0,A,B,C or D.

We proceed by case analysis on $u$:

\begin{description}

\item[case 0] This case cannot happen because $?(u) = \epsilon$ ($u$ is a complete play) implies that there cannot be any further move $m$.

Indeed the visibility condition implies that $m$ must point to a
P-question in the O-view at that point. But since $u$ is a complete
play, the O-view is $\oview{\hat{u}} = q a$ which does not contain
any P-question. Hence the move $m$ cannot be justified and is not
valid.


\item[case A] $?(u) = q$ and the last move $m$ is played by P.
    There are several cases:
    \begin{itemize}
    \item $m$ is an answer $a_k$ (to the initial question
    $q$) for some $k$, then $m$ points to $q$:

    $\hat{s} = \justseq{ q & \ldots & m \pointto{ll}}$

    and $?(s) = \epsilon$ therefore $s$ correspond to the case 0 (complete play).

    \item $m = q^i$ where $q^i$ is an order 0 question ($i \in I_0$).
    Then $q^i$ points to the initial question $q$ and $s$ falls into category D.

    \item $m = p^i$, a first order question, then $p^i$ points to $q$,

    $?(s)= q p^i$ and it is O's turn after $s$ therefore $s$ falls into category B.

    \end{itemize}


\item[case B] $?(u) \in q \cdot L^* \cdot p^i$ where $i \in I_1$ and O plays the move $m$.

We now analyse the different possible O-moves:
\begin{itemize}
\item Suppose that O gives the (tagged) answer $b^j$ for some $j \in I_1$ then
the visibility condition constraints it to point to a question in
the O-view at that point.

We remark that the last move in $\hat{u}$ must be $p^i$. Indeed,
suppose that there is a move $x \in M_A$ such that $\hat{u} =
\justseq{q & \ldots & p^i\ x \pointto{ll}}$ then by visibility, the
O-move $x$ should points to a move in the O-view a that point. The
O-view is $q p^i$, therefore $x$ can only points to $p^i$. But then,
$p^i$ is not a pending question in $s$ which is a contradiction.


Therefore $\oview{\hat{u}} = \oview{ \justseq{ q & \ldots & p^i
\pointto{ll}} } = q p^i$.

Hence $b^j$ can only point to $p^i$ (and therefore $i=j$).

We then have $?(s) = ?(u \cdot b^i) \in  q \cdot L^*$ which is
covered by case A and C.

\item The only other possible O-move is $q^i$ which, again by the visibility condition, points necessarily
to the previous move $p^i$. We then have $?(s) = ?(u \cdot q^i) \in
q \cdot L^* \cdot p^i q^i$. This falls into category C.

\end{itemize}

\item[case C] $?(u) \in q \cdot L^* \cdot p^i q^i$ where $i \in I_1$ and the move $m$ is played by $P$.

Suppose $m$ is an answer, then the well-bracketing condition imposes
to answer to $q^i$ first. The move $m$ is therefore an integer $a^i$
pointing to $q^i$. We then have $?(s) = ?(u \cdot a^i) \in  q \cdot
L^* \cdot p^i$. This correspond to case B.


Suppose $m$ is a question then there are two cases:
\begin{itemize}
\item $m = q^j$ with $j \in I_0$, the pointer goes to the initial question $q$ and $s$ falls into category D.
\item $m = p^j$ with $j \in I_1$, the pointer goes to the initial question $q$ and $s$ falls into category B.
\end{itemize}

\item[case D] $?(u) \in q \cdot L^* \cdot q^i$ where $i \in I_0$ and the move $m$ is played by $O$.

    The same argument as in case B holds. However there is now another possible move:
    the answer $m = a^i_k$ for some $k$.  This moves can only points to
    $q^i$ (this is the only pending question tagged by $i \in I_0$).

    Then $?(\hat{s}) = ?(\hat{u}\cdot a^i_k) = ?(\justseq{ q & \ldots & q^i \pointto{ll} & \ldots & a^i_k \pointto{ll}}) \in q \cdot L^* $ therefore $s$ falls either into category A or C.

\end{description}

This completes the induction.

How to generalize the proof to arenas that have multiple roots
(forest arenas)? In fact there is no ambiguity since all the moves
are implicitly tagged according to the arena that they belong to.
Therefore in the induction, it suffices to ignore the moves that
belong to another tree (as if they were part of a different game
played in parallel).


\end{proof}


\subsection{Pointer-less strategies}
\label{subsec:ptrless_strat}

Up to order 2, the semantics of PCF terms is entirely defined by
pointer-less strategies. In other words, the pointers can be
uniquely reconstructed from any non justified sequence of moves
satisfying the visibility and well-bracketing condition.

At level 3 however, pointers cannot be omitted in general. Here is an example
taken from \cite{abramsky:game-semantics} illustrating this. Consider the
following two terms of type $((\nat \typar \nat) \typar \nat) \typar
\nat$:

$$M_1 = \lambda f . f (\lambda x . f (\lambda y .y ))$$
$$M_2 = \lambda f . f (\lambda x . f (\lambda y .x ))$$

We assign tags to the types in order to identify in which arena the
questions are asked: $((\nat^1 \typar \nat^2) \typar \nat^3) \typar
\nat^4$. Consider now the following pointer-less sequence of moves
$s = q^4 q^3 q^2 q^3 q^2 q^1$. It is possible to retrieve the
pointers of the first five moves but there is an ambiguity for the
last move: does it point to the first or second occurrence of $q^2$
in the sequence $s$?

Note that the visibility condition does not eliminate the ambiguity,
since the two occurrences of $q^2$ both appear in the P-view at that
point (after recovering the pointers of $s$ up to the second last
move we get:
$$s = \rnode{q4}{q}^4
\rnode{q3}{q}^3
\rnode{q2}{q}^2
\rnode{q3b}{q}^3
\rnode{q2b}{q}^2
\rnode{q1}{q}^1
\bkptrc{q3}{q4}
\bkptrc{q2}{q3}
\bkptrc[ncurv=0.6]{q3b}{q4}
\bkptrc{q2b}{q3b}$$

 therefore the P-view of $s$ is $s$ itself.)

In fact these two different possibilities correspond to two
different strategies. Suppose that the link goes to the first
occurrence of $q^2$ then it means that the proponent is requesting
the value of the variable $x$ bound in the subterm $\lambda x . f (
\lambda y. ... )$. If P needs to know the value of $x$, this is
because P is in fact following the strategy of the subterm $\lambda
y . x$. And the entire play is part of the strategy $\sem{M_2}$.

Similarly, if the link points to the second occurrence of $q^2$ then
the play belongs to the strategy $\sem{M_1}$.

\section{Game model for PCF}
\subsection{Syntax of the PCF language}
PCF is a simply-type $\lambda$-calculus with the following
additions: integer constants  (of ground type), first-order
arithmetic operators, if-then-else branching, and the recursion
combinator $Y_A : (A\rightarrow A)\rightarrow A$ for any type $A$.

The types of PCF are given by the following grammar:
$$ T ::= \texttt{exp}\ |\ T \rightarrow T$$

The following grammar gives the structure of terms:
\begin{eqnarray*}
 M ::= x\ |\ \lambda x :A . M \ |\ M M \ |\ \\
\ |\ n \ |\ \texttt{succ } M \ |\  \texttt{pred } M \\
\ |\ \texttt{cond } M M M \ |\ \texttt{Y}_A\ M
\end{eqnarray*}

where $x$ ranges over a set of countably many variables and $n$
ranges over the set of natural numbers.

Terms are generated according to the formation rules given in table
\ref{tab:pcf_formrules} where the judgement is of the form $ \Gamma  \vdash M : A$.

\begin{table}[htbp]
$$ (var) \rulef{}{x_1:A_1, x_2:A_2, \ldots x_n : A_n  \vdash x_i : A_i}\ i \in 1..n$$
$$ (app) \rulef{\Gamma \vdash M : A\rightarrow B \qquad \Gamma \vdash N:A}{\Gamma \vdash M\ N : B}
\qquad (abs) \rulef{\Gamma, x:A \vdash M : B}{\Gamma \vdash \lambda x :A . M : A\rightarrow B}$$

$$ (const) \rulef{}{\Gamma \vdash n :\texttt{exp}}
\qquad (succ) \rulef{\Gamma \vdash M:\texttt{exp} }{\Gamma \vdash \texttt{succ}\ M:\texttt{exp}}
\qquad (pred) \rulef{\Gamma \vdash M:\texttt{exp} }{\Gamma \vdash \texttt{pred}\ M:\texttt{exp}}$$

$$
(cond) \rulef{\Gamma \vdash M : exp \qquad \Gamma \vdash N_1 : exp \qquad \Gamma \vdash N_2 : exp }{\Gamma \vdash \texttt{cond}\ M\ N_1\ N_2}
\qquad  (rec) \rulef{\Gamma \vdash M : A\rightarrow A }{ \Gamma \vdash Y_A M : A}$$

\caption{Formation rules for PCF terms}
\label{tab:pcf_formrules}
\end{table}

\subsection{Operational semantics of PCF}

We give the big-step operational semantics of PCF. The notation $M \eval V$ means
that the closed term $M$ evaluates to the canonical form $V$. The canonical forms are given by the following
grammar:
$$V ::= n\ |\ \lambda x. M$$
In other word, a canonical form is either a number or a function.

The operational semantics is given for closed terms therefore the context $\Gamma$ is not present in
the evaluation rules.

The full operational semantics is given in table \ref{tab:bigstep_pcf}.

\begin{table}[htbp]
$$\rulef{}{V \eval V} \quad \mbox{ provided that $V$ is in canonical form.} $$

$$ \rulef{M \eval \lambda x. M' \quad M'\subst{x}{N}}{M N \eval V}$$

$$\rulef{M \eval n}{\texttt{succ}\ M \eval n+1}
\qquad \rulef{M \eval n+1}{\texttt{pred}\ M \eval n}
\qquad \rulef{M \eval 0}{\texttt{pred}\ M \eval 0}$$

$$\rulef{M \eval 0 \quad N_1 \eval V}{\texttt{cond}\ M N_1 N_2  \eval V}
\qquad
 \rulef{M \eval n+1 \quad N_2 \eval V}{\texttt{cond}\ M N_1 N_2  \eval V}$$

$$\rulef{M (\mathrm{Y} M) \eval V }{\texttt{Y} M \eval V}$$
\label{tab:bigstep_pcf}
\caption{Big-step operational semantics of PCF}
\end{table}



\section{Idealized Algol (IA)}
\label{sec:ia}

\subsection{The syntax of IA}
IA is an extension of PCF introduced by J.C. Reynold in
\cite{Reynolds81}. It adds imperative features such as local variables and sequential composition.

The description of the language that we give here follows the one of \cite{abramsky:game-semantics}.

On top of \texttt{exp}, PCF has the following two new types:
 \texttt{com} for commands and \texttt{var} for variables.

There is a constant \texttt{skip} of type \texttt{com} which corresponds to the command that do
nothing. Commands can be composed using the sequential composition operator \texttt{seq}.
Local variable are declared using the \texttt{new} operator, variable content is written
using \texttt{assign} and retrieved using \texttt{deref}.

The new formations rules are given in table \ref{tab:ia_formrules}.

\begin{table}[htbp]
$$ \rulef{\Gamma \vdash M : \texttt{com} \quad \Gamma \vdash N :A}
    {\Gamma \vdash \texttt{seq}_A \ M\ N\ : A} \quad A \in \{ \texttt{com}, \texttt{exp}\}$$

$$ \rulef{\Gamma \vdash M : \texttt{var} \quad \Gamma \vdash N : \texttt{exp}}
    {\Gamma \vdash \texttt{assign}\ M\ N\ : \texttt{com}}
\qquad
 \rulef{\Gamma \vdash M : \texttt{var}}
    {\Gamma \vdash \texttt{deref}\ M\ : \texttt{exp}}$$

$$ \rulef{\Gamma, x : \texttt{var} \vdash M : A}
    {\Gamma \vdash \texttt{new } x \texttt{ in } M} \quad A \in \{ \texttt{com}, \texttt{exp}\}$$

$$ \rulef{\Gamma \vdash M_1 : \texttt{exp} \rightarrow \texttt{com} \quad \Gamma \vdash M_2 : \texttt{exp}}
    {\Gamma \vdash \texttt{mkvar } M_1\ M_2\ : \texttt{var}}$$

\caption{Formation rules for IA terms}
\label{tab:ia_formrules}
\end{table}

If $\vdash M : A$ (i.e. $M$ can be formed with an empty context), we say that $M$ is a close term.

\subsection{Operational semantics}

In IA the semantics is given in a slightly different form from PCF.
In PCF, the evaluation rules were given for closed terms only. Suppose that we
proceed the same way for IA and consider the evaluation rule for the $\texttt{new}$ construct:
the conclusion is $\texttt{new } x:=0 \texttt{ in } M$ and the premise
is an evaluation for a certain term constructed from $M$, more precisely the term $M$
where \emph{some} occurrences of $x$ are replaced by the value $0$.
Because of the presence of the \texttt{assign} operator, we cannot simply replace all
the occurrences of $x$ in $M$ (the required substitution is  more complicated
than the substitution used for beta-reduction).


Therefore, instead of giving the semantics for closed term we consider terms
whose free variables are all of type \texttt{var}. These free variables are ``closed'' by mean of
stores. A store is a function mapping free variables of type \texttt{var} to natural numbers.
Suppose $\Gamma$ is a context containing only variable of type \texttt{var}, then we say that
$\Gamma$ is a \texttt{var}-context. A store whose domain $\Gamma$ is called a $\Gamma$-store.

The notation $s\ |\ x \mapsto n$ refers to the store that maps $x$ to $n$
and otherwise maps variables according to the store $s$.


The canonical forms for IA are given by the grammar:
$$ V ::= n\ |\ \lambda x. M\ |\ x\ |\  \texttt{mkvar} M N$$

where $n \in \nat$ and $x:var$.


A program is now defined by a term together with a $\Gamma$-store such that $\Gamma \vdash M : A$.
The evaluation semantics is expressed by the judgment form
$$s,M \eval s', V$$
where $s$ and $s'$ are $\Gamma$-stores,
$\Gamma \vdash M : A$ and $\Gamma \vdash V : A$ where $V$ is in canonical form.

The operational semantics for IA is given by the rule of PCF (table \ref{tab:bigstep_pcf})
together with the rules of table \ref{tab:bigstep_ia} where the following abbreviation is used:
$$ \rulef{M_1 \eval V_1 \quad M_2 \eval V_2}{M \eval V} \qquad \mbox{for} \qquad
  \rulef{s,M_1 \eval s',V_1 \quad s', M_2 \eval s'',V_2 }{s,M \eval s'',V}
$$


\begin{table}[htbp]
$$\mbox{\textbf{Sequencing }}
    \rulef{M \eval \iaskip \quad N \eval V}{\texttt{seq } M\ N \eval V}
$$

$$\mbox{\textbf{Variables }}
    \rulef{s,N \eval s',n \quad s',M \eval s'',x}{s, \assign\ M\ N \eval (s''\ |\ x \mapsto n),\iaskip}
\qquad
    \rulef{s,M \eval s',x }{s, \deref\ M \eval s',s'(x)}$$

$$\mbox{\texttt{\textbf{mkvar}}}
    \rulef{N \eval n \quad M \eval \texttt{mkvar } M_1\ M_2 \quad M_1\ n \eval \iaskip}
    {\assign\ M\ N \eval \iaskip}
\qquad
    \rulef{N \eval \texttt{mkvar } M_1\ M_2 \quad M_2\ \eval n}
    {\deref\ M \eval n}
$$

$$\mbox{\textbf{Block}}
    \rulef{(s\ |\ x \mapsto 0),M \eval (s'\ |\ x \mapsto n),V }
    {s, \texttt{new } x \texttt{ in } M \eval s',V}
$$

\label{tab:bigstep_ia}
\caption{Big-step operational semantics of IA}
\end{table}

\subsection{Game semantics}

As we have seen in section \ref{sec:catgames}, games and strategies
form a cartesian closed category, therefore games can model the simply-typed $\lambda$-calculus. Let us first
explain how this is achieved before extending the model to PCF and IA.

\subsubsection{Simply typed $\lambda$-calculus}

In the cartesian closed category $\mathcal{C}$, the objects are the arenas and the morphisms are the strategies.

In the games that we describe here, the Opponent represents the environment while
the Proponent plays according to a strategy imposed by the program itself.


Given a simple type $A$, we will model it as an arena $\sem{A}$.
A context $\Gamma = x_1 :A_1, \ldots x_n:A_n$ will be mapped to the arena
$\sem{\Gamma} = \sem{A_1} \times \ldots \times \sem{A_n}$ and a term $\Gamma \vdash M : A$
will be modeled by a strategy on the arena $\sem{\Gamma} \rightarrow \sem{A}$.
Since $\mathcal{C}$ is cartesian closed, there is is a terminal object $\textbf{1}$ (the empty arena) that
models the empty context ($\sem{\Gamma} = \textbf{1}$).


The base type \texttt{exp} is interpreted by the following flat arena of natural numbers noted $\nat$:
$$  \pstree[levelsep=6ex]
    {\TR[name=R]{q}}
    { \TR{1} \TR{2} \TR{\ldots}
    }
$$
In this arena, there is only one question: the initial O-question, P can then answer it by playing a natural number $i \in \nat$.
There are only two kinds strategy on this arena:
\begin{itemize}
\item the empty strategy where P never answer the initial question. This corresponds to a non terminating computation;
\item the strategies where P answers by playing a number $n$. This models the constants of the language.
\end{itemize}

Given the interpretation of base types, we define the interpretation of $A\rightarrow B$ by induction:
$$\sem{A \rightarrow B} = \sem{A} \Rightarrow \sem{B}$$

where the operator $\Rightarrow$ denotes the arena construction $!A
\multimap B$ which exists because $\mathcal{C}$ is cartesian closed.

Graphically if we represent the arena $A$ and $B$ by two triangles, the arena for $A \rightarrow B$ would be represented by:
\begin{center}
\psset{xunit=.5pt,yunit=.5pt,runit=.5pt}
\begin{pspicture}(150,80)
\rput[tr](150,80){ \pnode(27,40){a} \pstribox{A} }
\rput[bl](0,0){ \pnode(27,40){b} \pstribox{B} }
\ncline{->}{a}{b}
\end{pspicture}
\end{center}


Variables are interpreted by projection:
$$\sem{x_1 : A_1, \ldots, x_n:A_n \vdash x_i : A_i} = \pi_i : \sem{A_i} \times \ldots \times \sem{A_i} \times \ldots \times \sem{A_n} \rightarrow  \sem{A_i}$$

The abstraction $\Gamma \vdash \lambda x :A.M : A \rightarrow B$ is modeled by a strategy on the arena
$\sem{\Gamma} \rightarrow (\sem{A}\Rightarrow\sem{B})$. This strategy is obtain by using the currying operator of the
cartesian closed category:
$$\sem{\Gamma \vdash \lambda x :A.M : A \rightarrow B} = \Lambda( \sem{\Gamma, x :A \vdash M : B})$$

The application $\Gamma \vdash M N$ is modeled using the evaluation map $ev_{A,B} : (A\Rightarrow B)\times A \rightarrow B$:

$$\sem{\Gamma \vdash M N} = \langle \sem{\Gamma \vdash M, \Gamma \vdash N} \rangle; ev_{A,B}$$


\subsubsection{PCF}

We now show how to model the PCF constructs in the game semantics setting.
In the following, the sub-arena of a game are tagged in order to distinguish identical arenas that are present in different components of the game.
Moves are also tagged in the exponent in order to identify the sub-arena in which moves are played. We will omit the pointers in the play
when they are not essential for the understanding of the model (moreover we will see later on that under certain assumptions
up to order 2, pointers can be recovered uniquely).

The successor arithmetic operator is modeled by the following strategy on the arena $\nat^1 \Rightarrow \nat^0$:
$$\sem{\texttt{succ}} = \{q^0 \cdot q^1 \cdot n^1 \cdot (n+1)^0\ |\ n \in \nat \}$$

The predecessor arithmetic operator is denoted by the strategy
$$\sem{\texttt{pred}} = \{q^0 \cdot q^1 \cdot n^1 \cdot (n-1)^0\ |\ n >0 \} \union \{ q^0 \cdot q^1 \cdot 0^1 \cdot 0^0 \} $$

Then given a term $\Gamma \vdash \texttt{succ} M : \texttt{exp}$ we define:
$$\sem{\Gamma \vdash \texttt{succ } M : \texttt{exp}} = \sem{\Gamma \vdash M} ; \sem{\texttt{succ}} $$
$$\sem{\Gamma \vdash \texttt{pred } M : \texttt{exp}} = \sem{\Gamma \vdash M} ; \sem{\texttt{pred}} $$


The conditional operator is denoted by the following strategy on the arena $\nat^3 \times \nat^2 \times \nat ^1 \Rightarrow \nat^0$:
$$\sem{\texttt{cond}} =
    \{ q^0 \cdot q^3 \cdot 0 \cdot q^2 \cdot n^2 \cdot n^0 \ | \ n \in \nat \}
    \union
    \{ q^0 \cdot q^3 \cdot m \cdot q^2 \cdot n^2 \cdot n^0 \ | \ m >0, n \in \nat \}
    $$


Given a term $\Gamma \vdash \texttt{cond} M\ N_1\ N_2$ we define:
$$\sem{\Gamma \vdash \texttt{cond} M\ N_1\ N_2} =
\langle \sem{\Gamma \vdash M}, \sem{\Gamma \vdash N_1}, \sem{\Gamma \vdash N_2} \rangle ; \sem{\texttt{cond}}$$


The interpretation of the \texttt{Y} combinator is a bit more complicated.

Consider the term $\Gamma \vdash M : A \rightarrow A$, its semantics $f$ is a strategy on $\sem{\Gamma} \times \sem{A} \rightarrow \sem{A}$.
We define the chain $g_n$ of strategies on the arena $\sem{\Gamma} \rightarrow \sem{A}$ as follows:
\begin{eqnarray*}
g_0 &=& \perp \\
g_{n+1} &=&  F(g_n) = \langle id_{\sem{\Gamma}}, g_n\rangle ; f
\end{eqnarray*}

where $\perp$ denotes the empty strategy $\{ \epsilon \}$.

It is easy to see that indeed the $g_n$ forms a chain.
We define $\sem{\texttt{Y } M}$ to be the least upper bound of the chain $g_n$
(i.e. the  least fixed point of $F$). Its existence is guaranteed by the fact that
the category of games is cpo-enriched.

\subsubsection{IA}

It is easy to check that all the strategies given until now are well-bracketed and innocent.
From now on, we will only require well-bracketing and we will introduce strategies that are
not innocent. This is a necessity if we want to give a model of memory cells that correspond
to variables. The intuition behind this fact is that a cell needs to remember what was the last value written in it
in order to be able to return it when it is read, and this can only be done by looking at the whole history of moves,
not only those present in the P-view.





\subsection{Full-abstraction}
In this section we recall the standard full abstraction result proved in  \cite{abramsky94full}
and \cite{hylandong_pcf}.

A context noted $C[-]$ is a term containing a hole denoted by $-$. If $C[-]$ is a context then $C[A]$ denotes the term obtained
after replacing the hole by the term $A$.

\begin{dfn}[Observational preorder]
Let $\vdash M : A$ and $\vdash N : A$ be two closed terms. We define the relation $\sqsubseteq$ as follows:


$M \sqsubseteq N$ if and only if for all context $C[-]$ such that $C[M]$ and $C[M]$ are well-formed terms if
$C[M] \eval$ then $C[N] \eval$.
\end{dfn}


\begin{lem}[Soundness for PCF terms] Let $M$ be a PCF term.
If $M \eval V$ then $\sem{M} = \sem{V}$.
\end{lem}

\begin{lem}[Soundness for IA terms] Let $\Gamma \vdash M : A$ be an IA term and a $\Gamma$ store $s$.
If $s,M \eval s',V$ then the plays of $\sem{s,M} : I \multimap A \otimes !\Gamma$ which begin
with a move of $A$ are identical to those of $\sem{s',V}$.
\end{lem}


\begin{lem}[Computational adequacy for PCF terms]
All PCF terms are computable. (i.e. $\sem{M} \neq \perp$ implies $M \eval$)
\end{lem}

\begin{lem}[Computational adequacy for IA terms]
All IA terms are computable. (i.e. $\sem{M} \neq \perp$ implies $M \eval$)
\end{lem}


The following result follows from soundness and computational adequacy of the model.
\begin{prop}[Inequational soundness]
\label{prop:ineqsoundness}
Let $M$ and $N$ be two closed terms then
$$\sem{M} \subseteq \sem{N} \implies  M \sqsubseteq N $$
\end{prop}

\begin{prop}[Definability]
\label{prop:definability}
Let $\sigma$ be a compact well-bracketed on a game $A$ denoting a IA type. Then there is
an IA-term $M$ such that $\sem{M} = \sigma$.
\end{prop}

The final standard result of game semantics can then be proved using proposition \ref{prop:ineqsoundness} and \ref{prop:definability}:
\begin{thm}[Full abstraction]
Let $M$ and $N$ be two closed IA-terms.
$$\sem{M} \precsim_b \sem{N} \ \iff \ M \sqsubseteq N$$
\end{thm}

where $\precsim_b$ denotes the intrinsic preorder of the category $\mathcal{C}_b$.


\subsection{Call-by-Value first-order Idealized Algol}

Game semantics for call-by-value programming Language.


\chapter{The Safe Lambda Calculus}
\label{chap:safelambda}
    \section{Introduction}

\subsection*{Background}

The \emph{safety condition} was introduced by Knapik, Niwi{\'n}ski and
Urzyczyn at FoSSaCS 2002 \cite{KNU02} in a seminal study of the
algorithmics of infinite trees generated by higher-order grammars. The
idea, however, goes back some twenty years to Damm \cite{Dam82} who
introduced an essentially equivalent\footnote{See de Miranda's
 thesis \cite{demirandathesis} for a proof.} syntactic
restriction (for generators of word languages) in the form of
\emph{derived types}.
% Level-$n$ tree grammars as defined by Damm correspond exactly to a
% subset of safe level-$n$ grammars -- namely the safe complete grammars
% -- and every safe grammar corresponds to a safe complete one.
A higher-order grammar (that is assumed to be \emph{homogeneously
  typed}) is said to be \emph{safe} if it obeys certain syntactic
conditions that constrain the occurrences of variables in the
production (or rewrite) rules according to their type-theoretic
order. Though the formal definition of safety is somewhat intricate,
the condition itself is manifestly important. As we survey in the
following, higher-order \emph{safe} grammars capture fundamental
structures in computation, offer clear algorithmic advantages, and
lend themselves to a number of compelling characterizations:

\begin{itemize}
\item \emph{Word languages}. Damm and Goerdt \cite{DG86} have shown
  that the word languages generated by order-$n$ \emph{safe} grammars
  form an infinite hierarchy as $n$ varies over the natural numbers.
  The hierarchy gives an attractive classification of the
  semi-decidable languages: Levels 0, 1 and 2 of the hierarchy are
  respectively the regular, context-free, and indexed languages (in
  the sense of Aho \cite{Aho68}), although little is known about
  higher orders.

  Remarkably, for generating word languages, order-$n$ \emph{safe}
  grammars are equivalent to order-$n$ pushdown automata \cite{DG86},
  which are in turn equivalent to order-$n$ indexed grammars
  \cite{Mas74,Mas76}.

\item \emph{Trees}. Knapik \emph{et al.} have shown that the Monadic
  Second Order (MSO) theories of trees generated by \emph{safe}
  (deterministic) grammars of every finite order are
  decidable\footnote{It has recently been shown
    \cite{OngLics2006} that trees generated by \emph{unsafe}
    deterministic grammars (of every finite order) also have decidable
    MSO theories. More precisely, the MSO theory of trees generated by order-$n$
recursion schemes is $n$-EXPTIME complete.}.

  They have also generalized the equi-expressivity result due to Damm
  and Goerdt \cite{DG86} to an equivalence result with respect to
  generating trees: A ranked tree is generated by an order-$n$ \emph{safe}
  grammar if and only if it is generated by an order-$n$ pushdown
  automaton.

\item \emph{Graphs}. Caucal \cite{Cau02} has shown that the MSO
  theories of graphs generated\footnote{These are precisely the
    configuration graphs of higher-order pushdown systems.} by
  \emph{safe} grammars of every finite order are decidable. In a recent preprint \cite{hague-sto07}, however,
  Hague \emph{et al.} have
  shown that the MSO theories of graphs generated by order-$n$
  \emph{unsafe} grammars are undecidable, but deciding their modal
  mu-calculus theories is $n$-EXPTIME complete.
\end{itemize}

\subsection*{Overview}

In this paper, we aim to understand the safety condition in the
setting of the lambda calculus. Our first task is to transpose it to
the lambda calculus and pin it down as an appropriate sub-system of
the simply-typed theory. A first version of the \emph{safe lambda
  calculus} has appeared in an unpublished technical report
\cite{safety-mirlong2004}. Here we propose a more general and cleaner
version where terms are no longer required to be homogeneously typed
(see Section~\ref{sec:safe} for a definition). The formation rules of
the calculus are designed to maintain a simple invariant: Variables
that occur free in a safe $\lambda$-term have orders no smaller than
that of the term itself.  We can now explain the sense in which the
safe lambda calculus is safe by establishing its salient property: No
variable capture can ever occur when substituting a safe term into
another. In other words, in the safe lambda calculus, it is
\emph{safe} to use capture-\emph{permitting} substitution when
performing $\beta$-reduction.


There is no need for new names when computing $\beta$-reductions of
safe $\lambda$-terms, because one can safely ``reuse'' variable names
in the input term. Safe lambda calculus is thus cheaper to compute in
this na\"ive sense. Intuitively one would expect the safety constraint
to lower the expressivity of the simply-typed lambda calculus. Our
next contribution is to give a precise measure of the expressivity
deficit of the safe lambda calculus. An old result of Schwichtenberg
\cite{citeulike:622637} says that the numeric functions representable
in the simply-typed lambda calculus are exactly the multivariate
polynomials \emph{extended with the conditional function}.  In the
same vein, we show that the numeric functions representable in the
safe lambda calculus are exactly the multivariate polynomials.

Our last contribution is to give a game-semantic account of the safe
lambda calculus.
% Not much is known about the safe $\lambda$-calculus, and many problems
% remain to be studied concerning its computational power, the
% complexity classes that it characterizes, its interpretation under the
% Curry-Howard isomorphism and its game-semantic characterization. This
% paper is a contribution to the last problem.
%
% The difficulty in giving a game-semantic account of safety lies in the
% fact that it is a syntactic restriction whereas game semantics is
% syntax-independent. The solution consists in finding a particular
% syntactic representation of terms on which the plays of the game
% denotation can be represented.  To achieve this, we use ideas recently
% introduced by the second author \cite{OngLics2006}: a term is
% canonically represented by a certain abstract syntax tree of its
% $\eta$-long normal form referred as the \emph{computation tree}. This
% abstract syntax tree is specially designed to establish a
% correspondence with the game arena of the term. A computation is
% described by a justified sequence of nodes of the computation tree
% respecting some formation rules and called a
% \emph{traversal}. Traversals permit us to model $\beta$-reductions
% without altering the structure of the computation tree via
% substitution. A notable property is that \emph{P-views} (in the
% game-semantic sense) of traversals corresponds to paths in the
% computation tree.  We show that traversals are just representations of
% the uncovering of plays of the game-semantic denotation. We then
% define a \emph{reduction} operation which eliminates traversal nodes
% that are ``internal'' to the computation, this implements the
% counterpart of the hiding operation of game semantics. Thus, we obtain
% an isomorphism between the strategy denotation of a term and the set
% of reductions of traversals of its computation tree.
Using a correspondence result relating the game semantics of a
$\lambda$-term $M$ to a set of \emph{traversals} \cite{OngLics2006}
over a certain abstract syntax tree of the $\eta$-long form of $M$
(called \emph{computation tree}), we show that safe terms are denoted
by \emph{P-incrementally justified strategies}. In such a strategy,
pointers emanating from the P-moves of a play are uniquely
reconstructible from the underlying sequence of moves and the pointers
associated to the O-moves therein: Specifically, a P-question always
points to the last pending O-question (in the P-view) of a greater
order. Consequently pointers in the game semantics of safe
$\lambda$-terms are only necessary from order 4 onwards. Finally we
prove that a $\eta$-long $\beta$-normal $\lambda$-term is \emph{safe}
if and only if its strategy denotation is (innocent and)
\emph{P-incrementally justified}.



% \subsection*{Related work}

% \noindent\emph{The safety condition for higher-order grammars}

% \smallskip

% \noindent We have mentioned the result of Knapik \emph{et al.}~\cite{KNU02} that
% infinite trees generated by \emph{safe} higher-order grammars have
% decidable MSO theories.  A natural question to ask is whether the
% \emph{safety condition} is really necessary.  This has then been
% partially answered by Aehlig \emph{et al.}
% \cite{DBLP:conf/tlca/AehligMO05} where it was shown that safety is not
% a requirement at level $2$ to guarantee MSO decidability. Also, for
% the restricted case of word languages, the same authors have shown
% \cite{DBLP:conf/fossacs/AehligMO05} that level $2$ safe higher-order
% grammars are as powerful as (non-deterministic) unsafe ones.  De
% Miranda's thesis \cite{demirandathesis} proposes a unified framework
% for the study of higher-order grammars and gives a detailed analysis
% of the safety constraint at level 2.

% More recently, one of us obtained a more general result and showed
% that the MSO theory of infinite trees generated by higher-order
% grammars of any level, \emph{whether safe or not}, is decidable
% \cite{OngLics2006}.  Using an argument based on innocent
% game-semantics, he establishes a correspondence between the tree
% generated by a higher-order grammar called \emph{value tree} and a
% certain regular tree called \emph{computation tree}. Paths in the
% value tree correspond to traversals in the computation tree.
% Decidability is then obtain by reducing the problem to the acceptance
% of the (annotated) computation tree by a certain alternating parity
% tree automaton.  The approach that we follow in
% Sec. \ref{sec:correspondence} uses many ingredients introduced in this
% paper.


% The equivalence of \emph{safe} higher-order grammars and higher-order
% deterministic push-down automata for the purpose of generating
% infinite trees \cite{KNU02} has its counterpart in the general (not
% necessarily safe) case: the forthcoming paper \cite{hague-sto07}
% establishes the equivalence of order-$n$ higher-order grammars and
% order-$n$ \emph{collapsible pushdown automata}. Those automata form a
% new kind of pushdown systems in which every stack symbol has a link to
% a stack situated somewhere below it and with an additional stack
% operation whose effect is to ``collapse'' a stack $s$ to the state
% indicated by the link from the top stack symbol.

% \medskip

% \noindent\emph{Computation trees and traversals}

% \smallskip

% \noindent In \cite{DBLP:conf/lics/AspertiDLR94}, a notion of graph
% based on Lamping's graphs \cite{lamping} is introduced to represent
% $\lambda$-terms. The authors unify different notions of paths
% (regular, legal, consistent and persistent paths) that have appeared
% in the literature as ways to implement graph-based reduction of
% $\lambda$-expressions. We can regard a traversal as an alternative
% notion of path adapted to the graph representation of
% $\lambda$-expressions given by computation trees.

% The traversals of a computation tree provide a way to perform
% \emph{local computation} of $\beta$-reductions as opposed to a global
% approach where the $\beta$-reduction is implemented by performing
% substitutions. A notion of local computation of $\beta$-reduction has
% been investigated by Danos and Regnier
% \cite{DanosRegnier-Localandasynchronou} through the use of special
% graphs called ``virtual nets'' that embed the lambda-calculus.


\section{The safe lambda calculus}
\label{sec:safe}
\subsection*{Higher-order safe grammars}
We first present the safety restriction as it was originally defined
\cite{KNU02}. We consider simple types generated by the grammar $A \,
::= \, o \; | \; A \typear A$. By convention, $\rightarrow$ associates
to the right. Thus every type can be written as $A_1 \typear \cdots
\typear A_n \typear o$, which we shall abbreviate to $(A_1, \cdots,
A_n, o)$ (in case $n = 0$, we identify $(o)$ with $o$). The
\emph{order} of a type is given by $\ord{o} = 0$ and $\ord{A \typear
  B} = \max(\ord{A}+1, \ord{B})$. We assume an infinite set of typed
variables. The order of a typed term or symbol is defined to be the
order of its type.

A (higher-order) \defname{grammar} is a tuple $\langle
\Sigma, \mathcal{N}, \mathcal{R}, S \rangle$, where $\Sigma$ is a
ranked alphabet (in the sense that each symbol $f \in \Sigma$ has an
arity $\mathit{ar}(f) \geq 0$) of \emph{terminals}\footnote{Each $f \in
  \Sigma$ of arity $r \geq 0$ is assumed to have type $(\underbrace{o,
    \cdots, o}_r, o)$.}; $\mathcal{N}$ is a finite set of typed
\emph{non-terminals}; $S$ is a distinguished ground-type symbol of
$\mathcal{N}$, called the start symbol; $\mathcal{R}$ is a finite set
of production (or rewrite) rules, one for each non-terminal $F : (A_1,
\ldots, A_n, o) \in \mathcal{N}$, of the form $ F z_1 \ldots z_m
\rightarrow e$ where each $z_i$ (called \emph{parameter}) is a
variable of type $A_i$ and $e$ is an applicative term of type $o$
generated from the typed symbols in $\Sigma \union \mathcal{N} \union \{z_1,
\ldots, z_m \}$. We say that the grammar is \emph{order-$n$} just in
case the order of the highest-order non-terminal is $n$.

The \defname{tree generated by a recursion scheme} $G$ is a possibly
infinite applicative term, but viewed as a $\Sigma$-labelled tree;
it is \emph{constructed from the terminals in $\Sigma$}, and is obtained by
unfolding the rewrite rules of $G$ \emph{ad infinitum}, replacing
formal by actual parameters each time, starting from the start symbol
$S$. See e.g.~\cite{KNU02} for a formal definition.

\pssetcomptree
\parpic[r]{
$\tree[levelsep=3ex,nodesep=1pt,treesep=1cm,linewidth=0.5pt]{g}
{  \TR{a}
    \tree{g}{\TR{a} \tree{h}{\tree{h}{\vdots}}}
}$
}
\begin{example}\rm\label{eg:running}
  Let $G$ be the following order-2 recursion scheme:
\[\begin{array}{rll}
  S & \rightarrow & H \, a\\
  H \, z^o & \rightarrow & F \, (g \,
  z)\\
  F \, \phi^{(o, o)} & \rightarrow & \phi \, (\phi \, (F \, h))\\
\end{array}\]
where the arities of the terminals $g, h, a$ are $2, 1, 0$ respectively.
The tree generated by $G$ is defined by the infinite term $g \, a \, (g \, a \, (h \, (h \, (h \,
\cdots))))$.%  The only infinite \emph{path} in the
% tree is the node-sequence $\epsilon \cdot 2 \cdot 22 \cdot 221 \cdot
% 2211 \cdots$.

%(with the corresponding \textbfit{trace} $g \, g \, h \, h \, h \,
%\cdots \; \in \; \Sigma^\omega$).
\end{example}

A type $(A_1, \cdots, A_n, o)$ is said to be \defname{homogeneous} if
$\ord{A_1} \geq \ord{A_2}\geq \cdots \geq \ord{A_n}$, and each $A_1$,
\ldots, $A_n$ is homogeneous \cite{KNU02}.  We reproduce the following
definition from \cite{KNU02}.

\begin{definition}[Safe grammar]\rm
  (All types are assumed to be homogeneous.) A term of order $k > 0$
  is \emph{unsafe} if it contains an occurrence of a parameter of
  order strictly less than $k$, otherwise the term is \emph{safe}. An
  occurrence of an unsafe term $t$ as a subexpression of a term $t'$
  is \emph{safe} if it is in the context $\cdots (ts) \cdots$,
  otherwise the occurrence is \emph{unsafe}. A grammar is
  \defname{safe} if no unsafe term has an unsafe occurrence at a
  right-hand side of any production.
%   A rewrite rule $F z_1 \ldots z_m \rightarrow e$ is said to be
%   \defname{unsafe} if the righthand term $e$ has a subterm $t$ such
%   that
% \begin{enumerate}[(i)]
% \item $t$ occurs in an {\em operand} ({\it i.e.}~second) position of some
%   occurrence of the implicit application operator {\it i.e.}~$e$ has the
%   form $\cdots (s \, t) \cdots $ for some $s$
% \item $t$ contains an occurrence of a parameter $z_i$ (say) whose
%   order is less than that of $t$.
% \end{enumerate}
% A homogeneous grammar is said to be \defname{safe} if none of its
% rewrite rules is unsafe.
\end{definition}

\begin{example}\begin{inparaenum}[(i)] \item Take $\; H : ((o, o), o), \; f : (o, o, o)$; the
    following rewrite rules are unsafe (in each case we underline the
    unsafe subterm that occurs unsafely):
\[\begin{array}{rll}
G^{(o, o)} \, x & \quad \rightarrow \quad & H \, \underline{(f \, {x})} \\
F^{((o, o), o, o, o)} \, z \, x \, y & \quad \rightarrow \quad & f \, (F \, \underline{(F \, z
\, {y})} \, y \, (z \, x) ) \, x
\end{array}\]
\item The order-2 grammar defined in Example~\ref{eg:running} is
  unsafe.
\end{inparaenum}
% The
% reader is referred to the literature
% \cite{KNU02,demirandathesis,safety-mirlong2004}
% for details about the safety restriction for higher-order grammars.
\end{example}

\subsection*{Safety adapted to the lambda calculus}
We assume a set $\Xi$ of higher-order constants.
We use sequents of the form $\Gamma \vdash_\Xi M : A$ to represent
terms-in-context where $\Gamma$ is the context and $A$ is the type of
$M$. For simplicity
we write $(A_1, \cdots, A_n, B)$ to mean $A_1 \typear \cdots \typear
A_n \typear B$, where $B$ is not necessarily ground.

\begin{definition}\rm
\begin{inparaenum}[(i)]
\item The \defname{safe lambda calculus} is a sub-system of the
  simply-typed lambda calculus defined by induction over the
  following rules:
$$ \rulename{var} \ \rulef{}{x : A\vdash_\Xi x : A} \quad
\rulename{const} \ \rulef{}{\vdash_\Xi f : A} \quad f \in \Xi \quad
\rulename{wk} \ \rulef{\Gamma \vdash_\Xi s : A}{\Delta \vdash_\Xi s : A} \quad
\Gamma \subset \Delta$$
$$ \rulename{app} \ \rulef{\Gamma \vdash_\Xi s : (A_1,\ldots,A_n,B) \
  \Gamma \vdash_\Xi t_1 : A_1 \; \ldots \; \Gamma \vdash_\Xi t_n : A_n
} {\Gamma \vdash_\Xi s t_1 \ldots t_n : B} \ \ord{B} \sqsubseteq
\ord{\Gamma}$$
$$ \rulename{abs} \ \rulef{\Gamma, x_1 : A_1, \ldots, x_n : A_n
  \vdash_\Xi s : B} {\Gamma \vdash_\Xi \lambda x_1 \ldots x_n . s :
  (A_1, \ldots ,A_n,B)} \ \ord{A_1, \ldots ,A_n,B} \sqsubseteq
\ord{\Gamma}$$ where $\ord{\Gamma}$ denotes the set $\{ \ord{y} : y
\in \Gamma \}$ and ``$c \sqsubseteq S$'' means that $c$ is a
lower-bound of the set $S$. For convenience, we shall omit the
subscript from $\vdash_\Xi$ whenever the generator-set $\Xi$ is clear from
the context.

\noindent \item The sub-system that is defined by the same rules in
(i), such that all types that occur in them are homogeneous, is called
the \defname{homogeneous safe lambda calculus}.
\end{inparaenum}
\end{definition}

The safe lambda calculus deviates from the standard definition of the simply-typed lambda calculus in a number of ways. First the rules $\rulename{app}$ and $\rulename{abs}$
respectively can perform multiple applications and abstract several
variables at once. (Of course this feature alone does not alter
expressivity.) Crucially, the side-conditions in the application rule
and abstraction rules require that variables in the typing context
have order no smaller than that of the term being formed.  We do not
impose any constraint on types. In particular, type-homogeneity as
used originally to define safe grammars \cite{KNU02} is not required
here. Another difference is that we allow $\Xi$-constants to have
arbitrary higher-order types.  % Thus our formulation
% of the safe lambda calculus is more general than the one proposed in
% the technical report \cite{safety-mirlong2004}. (It is possible to
% reconcile the two definitions by adding the further constraint that
% each type occurring in our rules is homogeneous and by restricting
% constants to at most order 1.)

\begin{example}[Kierstead terms]
\label{ex:kierstead}
Consider the terms $M_1 = \lambda f . f (\lambda x . f (\lambda y . y
))$ and $M_2 = \lambda f . f (\lambda x . f (\lambda y .x ))$ where
$x,y:o$ and $f:((o,o),o)$. The term $M_2$ is not safe because in the
subterm $f (\lambda y . x)$, the free variable $x$ has order $0$ which
is smaller than $\ord{\lambda y . x} = 1$.  On the other hand, $M_1$
is safe.
%On the other hand, $M_1$ is safe as the following proof tree shows:
%$$
% \rulef{
%     \rulef{
%        \rulef{}{f \vdash f} {\sf(var)}
%        \
%        \rulef{
%             \rulef{
%                \rulef{
%                    \rulef{}{f \vdash f} {\sf(var)}
%                }
%                {f , x \vdash f } {\sf(wk)}
%                \
%                \rulef{
%                    \rulef{
%                        \rulef{}{y \vdash y} {\sf(var)}
%                    }
%                    {y \vdash \lambda y . y } \rulenamet{abs}
%                }
%                {f , x \vdash \lambda y .y } {\sf(wk)}
%             }
%             {f , x \vdash f (\lambda y .y )} {\sf(app)}
%        }
%        { f  \vdash \lambda x . f (\lambda y .y )} \rulenamet{abs}
%     }
%     {
%        f  \vdash f (\lambda x . f (\lambda y .y ))} {\sf(app)}
%     }
% { \vdash M_1 = \lambda f . f (\lambda x . f (\lambda y .y )) } \rulenamet{abs}
%$$
\end{example}

It is easy to see that valid typing judgements of the safe lambda
calculus satisfy the following simple invariant:
\begin{lemma}
\label{lem:ordfreevar}
If $\Gamma \vdash M : A$ then every variable in $\Gamma$ occurring
free in $M$ has order at least $ord(M)$.
\end{lemma}


When restricted to the homogeneously-typed
sub-system, the safe lambda calculus captures the original notion
of safety due to Knapik \emph{et al.} in the context of higher-order
grammars:

\begin{proposition} Let $G = \langle \Sigma, \mathcal{N}, \mathcal{R},
  S \rangle$ be a grammar and let $e$ be an applicative term generated
  from the symbols in $\mathcal{N} \cup \Sigma \cup \makeset{z_1^{A_1},
    \cdots, z_m^{A_m}}$.  A rule $F z_1 \ldots z_m \rightarrow e$ in
  $\mathcal{R}$ is safe if and only if $ z_1 : A_1, \cdots, z_m : A_m
  \vdash_{\Sigma \cup \mathcal{N}} e : o$ is a valid typing judgement
  of the \emph{homogeneous} safe lambda calculus.
\end{proposition}

\emph{In what sense is the safe lambda calculus safe?} A basic idea
in the lambda calculus is that when performing $\beta$-reduction, one
must use capture-\emph{avoiding} substitution, which is standardly
implemented by renaming bound variables afresh upon each substitution.
In the safe lambda calculus, however, variable capture can never
happen (as the following lemma shows). Substitution can therefore be
implemented simply by capture-\emph{permitting} replacement, without
any need for variable renaming. In the following, we write
$M\captsubst{N}{x}$ to denote the capture-\emph{permitting}
substitution\footnote{This substitution is done by
textually replacing all free occurrences of $x$ in $M$ by $N$ without performing variable renaming.  In particular for the abstraction
  case we have
$(\lambda y_1\ldots y_n . M)\captsubst{N}{x} = \lambda y_1\ldots y_n . M\captsubst{N}{x}$ when $x\not\in
  \{ y_1\ldots y_n \}$.}
%\footnote{This substitution is implemented by textually
%  replacing all free occurrences of $x$ in $M$ by $N$ without
%  performing variable renaming.  In particular for the abstraction
%  case $(\lambda \overline{y} . P)\captsubst{N}{x}$ is defined as
%  $\lambda \overline{y} . P\captsubst{N}{x}$ if $x\not\in
%  \overline{y}$ and $\lambda \overline{y} . P$ elsewhere.}
of $N$ for $x$ in $M$.

\begin{lemma}[No variable capture]\label{lem:nvc}
\label{lem:homog_nocapture} There is
no variable capture when performing capture-permitting
substitution of $N$ for $x$ in $M$
provided that $\Gamma, x:B \vdash M : A$ and $\Gamma \vdash  N : B$ are valid judgments of the safe lambda calculus.
\end{lemma}

\proof
  We proceed by structural induction. The variable, constant and
  application cases are trivial. For the abstraction case, suppose $M = \lambda \overline{y}. R$ where $\overline{y} = y_1
  \ldots y_p$. If $x \in \overline{y}$ then $M \captsubst{N}{x} = M$ and there is no variable capture.

 If $x \not\in \overline{y}$ then we have $M \captsubst{N}{x} = \lambda \overline{y} . R \captsubst{N}{x}$.  By the induction hypothesis there is no variable capture in $R \captsubst{N}{x}$.  Thus variable capture can only happen if the following two conditions are met: $x$ occurs freely in $R$, and some variable $y_i$ for $1 \leq i \leq p$ occurs freely in $N$. By Lemma \ref{lem:ordfreevar}, the latter condition  implies $\ord{y_i} \geq \ord{N} = \ord{x}$.  Since $x \not \in \overline{y}$, the former condition implies that $x$ occurs freely in the safe term $\lambda \overline{y}. R$
  therefore Lemma \ref{lem:ordfreevar} gives $ \ord{x} \geq
  \ord{\lambda \overline{y} . R} \geq 1+ \ord{y_i} > \ord{y_i}$ which  gives a contradiction.
\qed


\begin{remark}
  A version of the No-variable-capture Lemma also holds in safe
  grammars, as is implicit in (for example Lemma 3.2 of) the original
  paper \cite{KNU02}.
\end{remark}

\begin{example}
  In order to contract the $\beta$-redex in the term
\[f:(o,o,o),x:o
  \vdash (\lambda \varphi^{(o,o)} x^o . \varphi \, x) (\underline{f \,
    x}) : (o,o)\] one should rename the bound variable $x$ with a fresh name to
  prevent the capture of the free occurrence of $x$ in the underlined term during substitution. Consequently, by the previous lemma,
  the term is not safe. Indeed, it cannot be because $\ord{x} = 0 < 1
  = \ord{f x}$.
\end{example}

Note that it is not the case that $\lambda$-terms
that satisfy the No-variable-capture Lemma are necessarily safe. For instance the $\beta$-redex in $\lambda y^o
z^o. (\lambda x^o .y) z$ can be contracted using capture-permitting
substitution, even though the term is not safe.

\subsection*{Reductions and transformations preserving safety}

From now on we will use the standard notation $M\subst{N}{x}$ to
denote the substitution of $N$ for $x$ in $M$.  It is understood that,
provided that $M$ and $N$ are safe, this substitution is
capture-permitting.


\begin{lemma}[Substitution preserves safety]
\label{lem:subst_preserve_safety}
If $\Gamma, x :B \vdash M : A$ and $\Gamma \vdash N : B$ then $\Gamma \vdash M[N/x] : A$.
\end{lemma}
This is proved by an easy induction on the structure of the safe term $M$.


It is desirable to have an appropriate notion of reduction for our
calculus. However the standard $\beta$-reduction rule is not
adequate. Indeed, safety is not preserved by $\beta$-reduction as the
following example shows. Suppose that $w,x,y,z : o$ and $f : (o,o,o)
\in \Sigma$ then the safe term $(\lambda x y . f x y) z w$
$\beta$-reduces to $(\underline{\lambda y . f z y}) w$ which is unsafe
since the underlined order-1 subterm contains a free occurrence of the
ground-type $z$. However if we perform one more reduction we obtain
the safe term $f z w$. This suggests an alternative notion of
reduction that performs simultaneous reduction of ``consecutive''
$\beta$-redexes. In order to define this reduction we first introduce
the appropriate notion of redex.

In the simply-typed lambda calculus a redex is a term of the form
$(\lambda x . M) N$. In the safe lambda calculus, a redex is a
succession of several standard redexes:

\begin{definition}\rm
Let $l\geq 1$ and $n\geq 1$. We use the abbreviations $\overline{x}$ and $\overline{x}:\overline{A}$  for $x_1 \ldots x_n$ and $x_1:A_1, \ldots, x_n : A_n$ respectively.

A \defname{safe redex} is a safe term of the form $(\lambda
\overline{x} . M) N_1 \ldots N_l$ such that the variables
$\overline{x}$ are abstracted altogether by one instance of the
\rulenamet{abs} rule (possibly followed by \rulenamet{wk}) and the
term $(\lambda \overline{x}.M)$ is applied to $N_1$, \ldots, $N_l$
by one instance of the \rulenamet{app} rule. Thus $M$, the and the
$N_i$'s are also safe.
\end{definition}
For instance, in the case $n<l$, a safe redex has a derivation tree of the following  form:
$$   \rulef{
            \rulef{\rulef{\rulef{\ldots}{\Gamma', \overline{x}:\overline{A} \vdash M : (A_{n+1}, \ldots, A_l, B)}}{\Gamma' \vdash \lambda \overline{x} . M : (A_1, \ldots, A_l, B)} \rulename{abs}}{\Gamma \vdash \lambda \overline{x} . M : (A_1, \ldots, A_l, B)}\rulename{wk}
            \quad
            \rulef{\ldots}{\Gamma \vdash N_1 :A_1}  \ \ldots \  \rulef{\ldots}{\Gamma \vdash N_l :A_l}
    }
    {
       \Gamma \vdash (\lambda \overline{x} . M) N_1 \ldots N_l : B
    } \rulename{app}
$$


We are now in a position to define a notion of reduction for safe terms.
\begin{definition}\rm
\label{dfn:safereduction} We use the
abbreviations $\overline{x} = x_1 \ldots x_n$,
$\overline{N} = N_1 \ldots N_l$.
The relation $\beta_s$ is defined on the set of safe redexes as:
\begin{eqnarray*}
  \beta_s &=&
  \{  \ (\lambda \overline{x} . M) N_1 \ldots N_l \mapsto \lambda x_{l+1} \ldots x_n. M\subst{\overline{N}}{x_1 \ldots x_l} \mbox{, for $n> l$}
  \} \\
  &\cup&
  \{ \ (\lambda \overline{x}  . M) N_1 \ldots N_l \mapsto M\subst{N_1 \ldots N_n}{\overline{x}} N_{n+1} \ldots N_l
  \mbox{, for $n\leq l$} \} \ .
\end{eqnarray*}
where $M\subst{R_1 \ldots R_k}{z_1 \ldots z_k}$ denotes the simultaneous substitution in $M$ of $R_1$,\ldots,$R_k$ for $z_1, \ldots, z_k$.  The
\defname{safe $\beta$-reduction}, written $\betasred$, is the
compatible closure of the relation $\beta_s$ with respect to the
formation rules of the safe lambda calculus.
\end{definition}

\noindent \emph{Remark:} The $\beta_s$-reduction is a multi-step
$\beta$-reduction \ie it is a subset of the transitive closure of $\betared$.


\begin{lemma}[$\beta_s$-reduction preserves safety]
\label{lem:safered_preserve_safety}
If $\Gamma \vdash s :A$ and $s \betasred t$ then $\Gamma \vdash t :A$.
\end{lemma}

\proof
  It suffices to show that the relation $\beta_s$ preserves safety.
Suppose that $s\ \beta_s\ t$ where $s$ is the
safe-redex $(\lambda x_1 \ldots x_n . M) N_1
  \ldots N_l $ with $x_1 : B_1, \ldots, x_n: B_n$
and $M$ of type $C$.  W.l.o.g we can assume that the last rule used
to form the term $s$ is \rulenamet{app} (and not the weakening rule
\rulenamet{wk}, thus  we have $\Gamma = fv(s)$.

Suppose $n>l$ then $A = (B_{l+1}, \ldots, B_n, C)$. By Lemma \ref{lem:subst_preserve_safety} we can form the safe term %\begin{equation}
$\Gamma, x_{l+1}:B_{l+1}, \ldots x_n :B_{n}\vdash M\subst{\overline{N}}{x_1 \ldots x_l} : C$. %\label{jud:substsafe}\ .
%\end{equation}
By Lemma \ref{lem:ordfreevar}, since $s$ is safe, all the variables
in $\Gamma$ have order $\geq \ord{A}$. This ensures that the
side-condition of the \rulenamet{abs} rule is verified when
abstracting the variables $x_{l+1} \ldots x_n$, which gives us the
judgement $\Gamma \vdash t :A$.

Suppose $n \leq l$. The substitution lemma gives
$\Gamma \vdash M\subst{N_1 \ldots N_n}{\overline{x}} : C$ and using \rulenamet{app} we form $\Gamma \vdash t :A$.
  \qed


In general, safety is not preserved by $\eta$-expansion; for instance we have
% $f:o,o \vdash f$ but $f:o,o \not \vdash \lambda x^o . f x$.
%This remark remains true for closed terms, for instance
$\vdash \lambda y^o z^o . y : (o,o,o)$ but
$\not \vdash \lambda x^o . (\lambda y^o z^o . y) x : (o,o,o)$.
However safety is preserved by $\eta$-reduction:

\begin{lemma}[$\eta$-reduction preserves safety]
  $\Gamma \vdash \lambda \varphi . s \varphi :A $ with $\varphi$ not
  occurring free in $s$ implies $\Gamma \vdash s :A$.
\end{lemma}
\proof
  Suppose $\Gamma \vdash \lambda \varphi . s \varphi :A$. If $s$ is an  abstraction then by construction of the safe term $\lambda \varphi . s \varphi$, $s$ is necessarily safe.  If $s = N_0 \ldots N_p$ with
  $p\geq 1$ then again, since $\lambda \varphi . N_0 \ldots N_p
  \varphi$ is safe, each of the $N_i$ is safe for $0 \leq i \leq p$
  and for any $z\in fv(\lambda \varphi . s \varphi)$, $\ord{z} \geq
  \ord{\lambda \varphi . s \varphi} = \ord{s}$. Since  $\varphi$ does not occur free in $s$ we have $fv(s) = fv(\lambda \varphi . s \varphi)$, thus we can use the application rule to form $fv(s) \vdash N_0 \ldots N_p : A$. The weakening rules permits us to conclude $\Gamma \vdash s :A$. \qed



The $\eta$-long normal form (or simply $\eta$-long form) of a term
% (also called \emph{long reduced form}, \emph{$\eta$-normal form} and
% \emph{extensional form} in the literature
% \cite{DBLP:journals/tcs/JensenP76,DBLP:journals/tcs/Huet75,huet76})
is obtained by hereditarily $\eta$-expanding every subterm occurring
at an operand position. Formally the \defname{$\eta$-long form}
$\elnf{t}$ of a term $t: (A_1,\ldots,A_n,o)$ with $n \geq 0$ is
defined by cases according to the syntactic shape of $t$:
\begin{eqnarray*}
  \elnf{\lambda x . s } &=& \lambda x . \elnf{s} \\
  \elnf{x s_1 \ldots s_m } &=& \lambda \overline{\varphi} . x \elnf{s_1}\ldots \elnf{s_m} \elnf{\varphi_1} \ldots \elnf{\varphi_n} \\
  \elnf{(\lambda x . s) s_1 \ldots s_p } &=& \lambda \overline{\varphi} . (\lambda x . \elnf{s}) \elnf{s_1} \ldots \elnf{s_p} \elnf{\varphi_1} \ldots \elnf{\varphi_n}
\end{eqnarray*}
where $m \geq 0$, $p\geq 1$, $x$ is a  variable or constant, $\overline{\varphi} = \varphi_1 \ldots \varphi_n$ and each $\varphi_i : A_i$ is a fresh variable.

%\begin{remark}
%  Converting a term to its $\eta$-long normal form does not introduce
%  new redex therefore the $\eta$-long normal form of $\beta$-normal
%  term is a $\beta$-normal term.
%\end{remark}

\begin{lemma}[$\eta$-long normalization preserves safety]
\label{lem:elnf_preserves_safety}
If $\Gamma \vdash s :A$ then $\Gamma \vdash \elnf{s} :A$.
\end{lemma}
\proof

 First we observe that for any variable or constant $x:A$ we have $x:A \vdash \elnf{x} :A$. We show this by induction on $\ord{x}$.
It is verified for any ground type variable $x$
since $x = \elnf{x}$.
Step case: $x:A$ with $A=(A_1, \ldots, A_n,o)$ and $n>0$. Let $\varphi_i:A_i$ be fresh variables for $1\leq i\leq n$.
Since $\ord{A_i} < \ord{x}$ the induction hypothesis gives $\varphi_i :A_i \vdash \elnf{\varphi_i} : A_i$. Using \rulenamet{wk} we obtain $x:A, \overline{\varphi} : \overline{A}
  \vdash \elnf{\varphi_i} :A_i$.  The application rule gives $x :A, \overline{\varphi} : \overline{A} \vdash x \elnf{\varphi_1} \ldots \elnf{\varphi_n}
  : o$ and the abstraction rule gives $ x :A \vdash \lambda
  \overline{\varphi} . x \elnf{\varphi_1} \ldots \elnf{\varphi_n} =
  \elnf{x} :A$.


We now prove the lemma by induction on $s$.
The base case is covered by the previous observation.
\emph{Step case:}
\begin{compactitem}
\item $s = x s_1 \ldots s_m$ with $x: (B_1, \ldots, B_m, A)$, $A = (A_1, \ldots, A_n, o)$ for some $m\geq 0$, $n>0$ and $s_i : B_i$ for $1 \leq i \leq
  m$.  Let $\varphi_i: A_i$ be fresh variables for $1\leq i \leq
  n$. By the previous observation we have $\varphi_i :A_i \vdash \elnf{\varphi_i} :A_i$, the weakening rule then gives us $\Gamma , \overline{\varphi} : \overline{A}
  \vdash \elnf{\varphi_i} : A_i$.  Since the judgement
  $\Gamma \vdash x s_1 \ldots s_m : A$ is formed using the \rulenamet{app} rule, each $s_j$ must be safe for $1\leq j \leq m$, thus by the induction hypothesis we have $\Gamma \vdash \elnf{s_j} : B_j$ and by weakening we get $\Gamma, \overline{\varphi} :\overline{A} \vdash \elnf{s_j} : B_j$.  The \rulenamet{app}
  rule then gives $\Gamma, \overline{\varphi} :\overline{A} \vdash x \elnf{s_1} \ldots \elnf{s_m} \elnf{\varphi_1} \ldots \elnf{\varphi_n} : o$. Finally
  the \rulenamet{abs} rule gives $\Gamma \vdash \lambda \overline{\varphi} . x
  \elnf{s_1} \ldots \elnf{s_m} \elnf{\varphi_1} \ldots
  \elnf{\varphi_n} = \elnf{s} : A$, the side-condition of \rulenamet{abs} being verified since $\ord{\elnf{s}} = \ord{s}$.


\item $s = t s_0 \ldots s_m$ where $t$ is an abstraction.
For some fresh variables $\varphi_1$, \ldots, $\varphi_n$
we have $\elnf{s} = \lambda \overline{\varphi}. \elnf{t} \elnf{s_0} \ldots \elnf{s_m} \elnf{\varphi_1}
  \ldots \elnf{\varphi_n}$. Again, using the induction hypothesis we can easily derive $\Gamma \vdash
 \lambda \overline{\varphi}. \elnf{t} \elnf{s_0} \ldots \elnf{s_m} \elnf{\varphi_1} \ldots \elnf{\varphi_n} : A$.

\item $s = \lambda \overline{\eta} . t $ where
$\overline{\eta} : \overline{B}$ and $t:C$ is not an abstraction. The induction hypothesis gives $\Gamma,
  \overline{\eta} : \overline{B} \vdash \elnf{t} : C$ and using
\rulenamet{abs} we get $\Gamma \vdash \lambda \overline{\eta} . \elnf{t} = \elnf{s} : A$.  \qed
\end{compactitem}


Note that the converse does not hold in general, for instance $\lambda
x^o . f^{(o,o,o)} x^o$ is unsafe although $\elnf{\lambda x . f x} =
\lambda x^o y^o . f x y$ is safe.


%\notetoself{Check and prove the following lemma}
% For terms with homogeneous types however, the converse does hold:
%\begin{lemma}
%If $\Gamma \vdash \elnf{s} : T$ is homogeneously safe (i.e. it is a
%safe judgement of the safe $\lambda$-calculus and each sequent
%occurring at the nodes of the proof tree is homogeneously typed)
%then $\Gamma \vdash s :T$ is homogeneously safe.
%\end{lemma}

    \input{safe_homogeneous.texi}
    \newcommand\bigo{\mathcal{O}} % big O notation
\newcommand\booltype{\mathbf{B}}

\section{Complexity of the Safe Lambda Calculus}
Here we study the problem of deciding beta-eta equivalence of two safe lambda terms.

\subsection{Statman's result}

A famous result by Statman  states that deciding the $\beta\eta$-equality of two first-order typable lambda terms is not elementary recursive \cite{Statman:1979:TLE}.
The idea of the proof is to encode the Henkin quantifier elimination of Type Theory into the simply-typed lambda calculus. The encoding relies on the fact that the function $\sf sg$ (conditional) can be encoded in the lambda-calculus. Hence the argument does not carry on   in the Safe Lambda Calculus since the conditional operator is not definable (\cite{blumong:safelambdacalculus}).

Mairson gave a simpler proof of Statman's theorem in \cite{mairson1992spt} which also proceeds by encoding the Henkin quantifier elimination procedure into the lambda-calculus but is much easier to understand as it makes use of list iteration to perform quantifier elimination.

It turns out that both encodings rely on the use of unsafe terms in order to implement the quantifier elimination procedure.

%Here we adapt Mairson's proof to produce a safe encoding of the quantifier elimination procedure, thus showing:
%\begin{theorem}
%The Safe Lambda Calculus is not elementary recursive.
%\end{theorem}

We recall the definition of the theory. Let $\mathcal{D}_0 = \{\mathbf{true},\mathbf{false}\}$ and $\mathcal{D}_{k+1} =powerset(\mathcal{D}_k)$.
For any $k\geq0$, we write $x^k$, $y^k$ and $z^k$ to denote variables ranging over $\mathcal{D}_k$. Prime formulas are $x^0$, $\mathbf{true}\in y^1$, $\mathbf{false}\in y^1$, and  $x^k \in y^{k+1}$. Formulae are built up from prime formulas using the logical connectives $\zand$,$\zor$,$\rightarrow$,$\neg$ and the quantifiers
$\forall$ and $\exists$. Meyer showed that deciding the truth of such formulae requires nonelementary time \cite{Meyer1974}.
\smallskip

In Mairson's encoding, all formula variables of a given order $k$ are encoded by terms of the same type $\Delta_k$. Using this encoding,
unsafety manifests itself in two different ways.
\begin{enumerate}[1.]
  \item
        First in the encoding of set membership. The prime formula $x^k \in y^{k+1}$ is encoded as \begin{equation} x^k : \Delta_k, y^{k+1}:\Delta_{k+1} \vdash y^{k+1} (\lambda y^k : \Delta_k . OR (eq_k~\underline{x^k}~y^k)~F : \Delta_k \typear \Delta_{k+1} \typear \Delta_0 \label{eqn:setmembership}\end{equation}
for some terms $OR$, $F$, $eq_k$.
This term is unsafe because of the underline occurrence of $x^k$ which is not abstracted together with $y^k$.

\item Secondly, quantifier elimination is performed by using a list iterator $\mathbf{D}_{k+1}$ which acts like the $fold\_left$ function from functional programming languages over the list of all elements of $\mathcal{D}_k$.
Thus for instance the formula $\forall x^0 . \exists y^0 . x^0 \zor y^0$
is encoded as $$\vdash \mathbf{D}_1 (\lambda x^0:\Delta_0. AND (\mathbf{D}_1 (\lambda y^0:\Delta_0. OR (\underline{x^0} \zor y^0)) F)) T$$ which is unsafe because of the underlined occurrence.

More generally, supposing that we find a way to encode set membership with a safe term, then the encoding of the formula will be safe if and only if for any variable $x$ in the formula, its binder is precisely the first quantifier $\exists z$ or $\forall z$ in the path to the root of the formula AST verifying $\ord{z} \geq \ord{x}$. For instance the formula $\forall x^k . \exists y^{k+1} . x^k \in y^{k+1}$ would be encoded by an unsafe term whereas the encoding of $\forall y^{k+1} . \exists x^k . x^k \in y^{k+1}$ would be safe.
\end{enumerate}

Surprisingly, the unsafety of the quantifier elimination procedure can be
easily overcome. The idea is as follows. We introduce multiple domains of representation for a given formula. An element of $\mathcal{D}_k$ is thereby represented by countably many terms of type $\Delta_k^n$ where $n\in\nat$ indicates the level of the representation. The type $\Delta_k^n$ is defined in such a way that its order strictly increases as $n$ grows. Moreover there exists a term that can reduce the level of representation of a given term. In the formula representation, each variable can now be encoded with a different level of representation. Since there are infinitely many levels, it is always possible to find an assignment of levels to variables such that the resulting encoding term is safe.

For set-membership, however, there is no obvious way to obtain a safe encoding. The set-member function from Eq.\ \ref{eqn:setmembership} can be turned into a safe term provided that we have access to a function permitting us to increase the representation level of term, but to our knowledge, such transformation cannot be expressed in the simply-typed lambda-calculus.



\subsubsection{Encoding basic boolean operations}

We assume a ground type $o$.
%For any type $\mu$ we define the type $\booltype_\mu \equiv (\mu\typear\mu)\typear \mu$.
%We abbreviate $\booltype_0$ into $\booltype$.
%We introduce the following hierarchy of types: $\sigma_0 \equiv o$, $\sigma_{n+1} \equiv \booltype_{\sigma_n}$ for $n\geq1$.
%Note that the order of $\sigma_n$ strictly increases as $n$ increases.
We introduce a parameterized type for encoding booleans defined by $\booltype_{-1} \equiv o$ and $\booltype_{n+1} \equiv \booltype_n\typear\booltype_n\typear\booltype_n$ for $n\geq0$.
We have $\ord{\booltype_n} = n+1$ for $n\geq-1$.


The representation of the truth values $\mathbf{true}$ and $\mathbf{false}$ will be parameterized by $n \in \nat$ as follows
\begin{align*}
  T^n &\equiv \lambda x^{\booltype_{n-1}} y^{\booltype_{n-1}} .x : \booltype_{n}\\
  F^n &\equiv \lambda x^{\booltype_{n-1}} y^{\booltype_{n-1}} .y : \booltype_{n}
\end{align*}
Clearly these terms as safe. Moreover the following relations hold for all $n$:
\begin{align*}
  T^{n+1}~T^n~F^n &\betared^*  T^n \\
  F^{n+1}~T^n~F^n &\betared^*  F^n
\end{align*}
Hence it is possible to lower the representation level of a term encoding a boolean value by applying the two terms $T^n$ and $F^n$ to it.
For $i\in\nat$, we define the function $\_ \downarrow_i$ that lowers the level-representation of a term, turning a term of type $\booltype_n$ for some $n\in\nat$ to a term of type $\booltype_{\min(i,l)}$:
$$ (M :\booltype_n)\downarrow_i = \left\{
  \begin{array}{ll}
    M~T^{n-1}~F^{n-1}~\ldots~T^{i+1}~F^{i+1}:\booltype_i, & \hbox{if $n>i$;} \\
M:\booltype_n, & \hbox{otherwise.}
  \end{array}
  \right.
$$


Boolean functions are encoded by the following level-parameterized terms:
\begin{align*}
AND^n &\equiv \lambda p : \booltype_n \lambda q : \booltype_n \lambda x:\booltype_{n-1} \lambda y:\booltype_{n-1} . p~(q~x~y)~y : \booltype_n\typear\booltype_n\typear\booltype_n \\
OR^n &\equiv \lambda p : \booltype_n \lambda q : \booltype_n \lambda x:\booltype_{n-1} \lambda y:\booltype_{n-1} . p~x~(q~x~y) : \booltype_n\typear\booltype_{n}\typear\booltype_n \\
NOT^n &\equiv \lambda p : \booltype_n \lambda q : \booltype_n \lambda x:\booltype_{n-1} \lambda y:\booltype_{n-1} . p~y~x : \booltype_n\typear\booltype_n\typear\booltype_n \\
IF^n &\equiv \lambda p : \booltype_n \lambda q : \booltype_n \lambda x:\booltype_{n-1} \lambda y:\booltype_{n-1} . OR^n (NOT^n p)~q : \booltype_n\typear\booltype_n\typear\booltype_n
\end{align*}
which are all safe terms.

\subsubsection{Coding elements of the type hierarchy}
For any $n\in\nat$ we define the hierarchy of type $\Delta_k^n$ as follows:
$\Delta_0^n \equiv \booltype_n$ and $\Delta_{k+1}^n \equiv {\Delta_k^n}^*$ where for any type $\alpha$, $\alpha^* = (\alpha \typear \tau \typear \tau)\typear \tau \typear \tau$.

An occurrence of a formula variable $x^k$ will be encoded as a term variable $x^k:\Delta_{k}^n$ for some level of representation $n\in\nat$.

Following Mairson's  proof, we encode the set $\mathcal{D}_0$ as the list $\mathbf{D}_0$ containing $\mathbf{true}$ and $\mathbf{false}$, and we parameterized this representation by $n\in \nat$:
$$\mathbf{D}_0^n \equiv \lambda c:\booltype_n \typear \tau \typear \tau . \lambda e : \tau . c~T^n~(c~F^n~e) : \Delta_1^n$$
and for $k\geq 0$, the higher-order set $\mathcal{D}_{k+1}$ is represented by the parameterized term:
$$\mathbf{D}_{k+1}^n \equiv powerset~\mathbf{D}_k^n : \Delta_{k+2}^n$$
where the term $powerset$ taken from \cite{mairson1992spt} is reproduced here:
\begin{align*}
  powerset &\equiv \lambda A^* :(\alpha \typear \alpha^{**} \typear \alpha^{**}) \typear \alpha^{**} \typear \alpha^{**}.\\
&\qquad  A^*~double~(\lambda c:\alpha^* \typear \tau\typear \tau.\lambda b:\tau . c ( \lambda c':\alpha\typear \tau\typear \tau. \lambda b':\tau.b') b)\\
powerset &: ((\alpha \typear \alpha^{**} \typear \alpha^{**}) \typear \alpha^{**} \typear \alpha^{**})\typear \alpha^{**}
\end{align*}
with
\begin{align*}
  double &\equiv \lambda x :\alpha.\lambda l : (\alpha^* \typear \tau\typear \tau)\typear \tau\typear \tau. \\
  & \qquad \lambda c:\alpha^*\typear \tau\typear \tau.\lambda b:\tau. \\
  & \qquad \qquad l(\lambda e:\alpha^*.c (\lambda c':\alpha\typear \tau\typear \tau.\lambda b':\tau.c'~x~(e~c'~b')))(l~c~b)\\
double &: \alpha \typear \alpha^{**} \typear \alpha^{**}
\end{align*}

It can be checked that these two terms are safe.

\subsubsection{Quantifier elimination}
Following \cite{mairson1992spt}, quantifier elimination interprets $\forall x^k.\Phi(x^k)$ as the iterated conjunction $\mathbf{D}_k(\lambda x^k:\Delta^k.AND(\hat\Phi~x^k))~T$ where $\hat\Phi$ is the interpretation of $\Phi$; similarly $\exists x^k.\Phi(x^k)$  is interpreted by the iterated disjunction $\mathbf{D}_k(\lambda x^k:\Delta^k.AND(\hat\Phi~x^k))~T$.

Let $x^{k_p}_p \ldots x^{k_1}_1$ for $p\geq1$ be the list of variables appearing in the formula. W.l.o.g.\ we can assume that they are given in the order of appearance of their binder in the formula \ie $x^{k_p}_p$ is bound by the leftmost binder. We assign representation levels to variables as follows. The right-most variable is assigned level $1$ \ie $x^{k_1}_1 : \Delta^1_{k_1}$; suppose that $x^{k_i}_i :\Delta^l_{k_l}$ for $1\leq i< p$ then the representation level of variable $x^{k_{i+1}}_{i+1}$ is defined as
the smallest $l'\in\nat$ such that $\ord{\Delta^{l'}_{k_i}} > \ord{\Delta^{l}_{k_{i-1}}}$.

This way, since variables that are bound first have higher order, the variables
 that are bound in the nested list-iterations (corresponding to the nested quantifiers in the formula) are necessarily safely bound.


\subsubsection{Coding set theory in the $\mathcal{D}_k$}
To complete the interpretation of prime formulas, we would need to show how to encode set membership. Unfortunately, this seems to be impossible in the safe lambda calculus. It would turn to be possible if we had at hand a function $\_ \uparrow^k$, counterpart of $\_ \downarrow_k$, that increases the representation level of a term to level $k$. Here is how we would proceed if such function were representable in the safe lambda calculus.

Firstly, the formulae ``$\mathbf{true} \in y^1$'' and ``$\mathbf{false} \in y^1$'' can be encoded by the safe terms $y^1 (\lambda x^0 . OR^0~x^0) F^0$ and $y^1 (\lambda x^0. OR^0(NOT^0~x^0)) F^0$ respectively.
For the general case ``$x^k\in y^{k+1}$''
we proceed as in \cite{mairson1992spt} by introducing lambda-terms encoding set equality, set membership and subset tests, and we further parameterize these encoding by $n\in\nat$.

Equality of booleans is encoded by:
$$ eq_0^n \equiv \lambda x^0 : \booltype_n .\lambda y^0 : \booltype_n. OR^n (AND^n~x^0~y^0) (AND^n (NOT^n~x^0)(NOT^n~y^0)) \ .$$

We now use variable of type $\Delta_{k+1}^n$ as iterators over list of elements of type $\Delta_k^n$ and we instantiate the type variable $\tau$ as $\booltype_n$ in order to iterate a level-$n$ Boolean function. We define the set membership function as follows. Note that
the level of representation differs from input to output: \begin{align*}
  member_{k+1}^{n+1} &\equiv \lambda x^k : \Delta_k^{n+1}.\lambda y^{k+1}:\Delta_{k+1}^{n+1}. \\
& \qquad (y^{k+1}\downarrow_n) (\lambda y^k : \Delta_k^n . OR^n (eq_k^{n+1}~x^k~(y^k\uparrow^{n+1})))~F^n \\
  & : \Delta_k^{n+1} \typear \Delta_{k+1}^{n+1} \typear \booltype_n
\\
  subset_{k+1}^{n+1} &\equiv \lambda x^{k+1} : \Delta_{k+1}^{n+1}.\lambda y^{k+1}:\Delta_{k+1}^{n+1}. \\
  & \qquad (x^{k+1}\downarrow_n) (\lambda x^k : \Delta_k^n . AND^n (member_{k+1}^{n+1}~x^k~y^{k+1}))~T^n \\
  & : \Delta_{k+1}^{n+1} \typear \Delta_{k+1}^{n+1} \typear\booltype_n
\\
  eq_{k+1}^{n+1} &\equiv \lambda x^{k+1} : \Delta_{k+1}^{n+1}.\lambda y^{k+1}:\Delta_{k+1}^{n+1}. \\
   & \qquad
   (\lambda op:\Delta_{k+1}^n\typear\Delta_{k+1}^n\typear\booltype_n. AND^n (op~x^{k+1}~y^{k+1})(op~y^{k+1}~x^{k+1}))~subset_{k+1}^{n+1} \\
  & : \Delta_{k+1}^{n+1} \typear \Delta_{k+1}^{n+1} \typear \booltype_n
\end{align*}
The terms $eq_{k+1}^n$ and $subset_{k+1}^n$ are safe, and so is $member_{k+1}^n$ thanks to the fact that $y^k$ has a lower representation level than $x^k$.

The formula $x^k\in y^{k+1}$ is then encoded by the term
$$x^k:\Delta_k^n, y^{k+1}:\Delta_{k+1}^{n'}\vdash \left(member_{k+1}^{\min(n,n')} (x^k\downarrow_{\min(n,n')})~(y^{k+1}\downarrow_{\min(n,n')})\right)\downarrow_0$$


\subsection{NP-hardness}
To show NP-hardness it suffices to observe that the encoding of SAT in the simply-typed lambda calculus from the paper\cite{asperti-np} relies only on safe terms.

\subsection{PSPACE-hardness}

We encode QBF into the calculus.
We assume that the quantified propositional formula is given in prenex form:
$$\$_{n-1} x_{n-1} \ldots \$_0 x_0 . \psi(x_0, \ldots, x_{n-1})$$
where $\$_i \in \{\exists,\forall\}$ for $0\leq i\leq n-1$.

The encoding is as follows:
\begin{align*}
\sem{1} &= T^0  : \booltype \\
\sem{0} &= F^0 : \booltype \\
\sem{x_i} &= x_i\downarrow_0 = x_i~T^{i-1}~F^{i-1}\ldots T^1~F^1: \booltype \qquad \hbox{where $x_i:\booltype_i$}\\
\sem{\psi_1\zand \psi_2} &= AND^0~\sem{\psi_1}~\sem{\psi_2}
:\booltype  \\
\sem{\psi_1\zor \psi_2} &= OR^0~\sem{\psi_1}~\sem{\psi_2}
:\booltype  \\
\sem{\neg \psi} &= NOT^0~\sem{\psi}
:\booltype  \\
\sem{\forall x_i.\psi(\ldots, x_i, \ldots)} & = \mathbf{D}_0^i(\lambda x^{\booltype_i} AND^0~\sem{\psi(\ldots, x_i, \ldots)})~T^0 :\booltype\\
\sem{\exists x_i.\psi(\ldots, x_i, \ldots)} & = \mathbf{D}_0^i(\lambda x^{\booltype_i}.OR^0~\sem{\psi(\ldots, x_i, \ldots)})~F^0 :\booltype
\end{align*}
The size of $\sem{\psi}$ is in $\bigo(|\psi|^2)$.

It is easy to check that this encoding is safe.
\begin{example}
  The formula $\forall x \exists y \exists z (x\zor y\zor z)\zand(\neg x\zor \neg y\zor \neg z)$ is represented by the safe term:
\begin{align*}
\vdash &\mathbf{D}_0^2(\lambda x^{\booltype_2}. AND^0\\
&\quad\quad (\mathbf{D}_0^1(\lambda x^{\booltype_1}.OR^0\\
&\quad\quad\quad (\mathbf{D}_0^0(\lambda x^{\booltype_0}.OR^0\\
&\quad\quad\quad\quad (AND^0 (OR^0(OR^0~(x~T^1 F^1 T^0 F^0)~(y~T^0 F^0))z) \\
&\quad\quad\quad\quad\quad (OR^0(OR^0(NEG^0 (x~T^1 F^1 T^0 F^0))(NEG^0 (y~T^0 F^0)))(NEG^0~z))) \\
&\quad\quad\quad )F^0)\\
&\quad\quad)F^0)\\
&\quad) T^0
\end{align*}
\end{example}
This gives us:
\begin{theorem}
  Deciding $\beta\eta$-equality of two terms of the Safe Lambda Calculus is PSPACE-hard.
\end{theorem}

% NP \subseteq PSPACE \subseteq EXP


    \section{Expressivity}
\subsection{Numeric functions representable in the safe lambda
calculus}

Natural numbers can be encoded into the simply-typed lambda calculus
using the Church Numerals: each $n\in\nat$ is encoded into the term
$\encode{n} = \lambda s z. s^n z$ of type $I = ((o,o),o,o)$ where
$o$ is a ground type. In 1976 Schwichtenberg \cite{citeulike:622637}
showed the following:


\begin{theorem}[Schwichtenberg 1976]
The numeric functions representable by simply-typed $\lambda$-terms
of type $I\rightarrow \ldots \rightarrow I$ using the Church Numeral
encoding are exactly the multivariate polynomials \emph{extended
with the conditional function}.
\end{theorem}

If we restrict ourselves to safe terms, the representable functions
are exactly the multivariate polynomials:
\begin{theorem}
\label{thm:polychar} The functions representable by safe
$\lambda$-expressions of type $I\rightarrow \ldots \rightarrow I$
are exactly the multivariate polynomials.
\end{theorem}

\begin{corollary}
The conditional operator $C:I\rightarrow I\rightarrow I \rightarrow
I$ verifying  $C t y z \rightarrow_\beta y$  if $t \rightarrow_\beta
\encode{0}$ and $C t y z \rightarrow_\beta z$ if $t
\rightarrow_\beta \encode{n+1}$ is not definable in the safe
simply-typed lambda calculus.
\end{corollary}
\proof
  Natural numbers are encoded using Church Numerals: $\encode{n} =
  \lambda s z. s^n z$.  Addition: For $n,m \in \nat$, $\encode{n+m} =
  \lambda \alpha^{(o,o)} x^o . (\encode{n} \alpha) (\encode{m} \alpha
  x)$. Multiplication: $\encode{n . m} = \lambda \alpha^{(o,o)}
  . \encode{n} (\encode{m} \alpha)$.  All these terms are safe and
  clearly any multivariate polynomial $P(n_1, \ldots, n_k)$ can be
  computed by composing the addition and multiplication terms as
  appropriate.

For the converse, let $U$ be a safe $\lambda$-term of type
$I\rightarrow I\rightarrow I$.  The generalization to terms of type
$I^n \rightarrow I$ for $n>2$ is immediate (they correspond to
polynomials with $n$ variables). W.l.o.g we can assume that $U =
\lambda x y \alpha z. u$ where $u$ is a safe term of ground type in
$\beta$-normal form with $fv(u) \subseteq \{ x, y : I, z :o, \alpha
: o\rightarrow o \}$.

\emph{Notation:} Let $T$ be a set of terms of type $\tau \rightarrow
\tau$ and $T'$ be a set of terms of type $\tau$ then $T \cdot T'$
denotes the set of terms $\{ s s' : \tau \ | \ s \in T \wedge s' \in
T' \}$. We also define $T^k \cdot T'$ recursively as follows:  $T^0
\cdot T' = T'$ and for $k\geq 0$, $T^{k+1} \cdot T' = T \cdot (T^k
\cdot T')$ ({\it i.e.}~$T^k \cdot T'$ denotes $\{ s_1( \ldots (s_k
s'))  \ | \ s_1, \ldots, s_k \in T \wedge s' \in T' \}$). We define
$T^+\cdot T' = \Union_{k > 0} T^k \cdot T'$ and $T^*\cdot T' =
(T^+\cdot T') \union T'$. For two sets of terms $T$ and $T'$, we
write $T =_\beta T'$ to express that any term of $T$ is
$\beta$-convertible to some term $t'$ of $T'$ and reciprocally.

Let us write $\mathcal{N}^\tau$ for the set of $\beta$-normal terms
of type $\tau$ where $\tau$ ranges in $\{ o, o\rightarrow o, I \}$
and with free variables in $\{ x,y:I, z:o, \alpha:o\rightarrow o\}$.
We write $\mathcal{A}^\tau$ for the subset of $\mathcal{N}^\tau$
consisting of applications only ({\it i.e.}~not abstractions). Let
$B$ be the set of terms of type $(o,o)$ defined by $B = \{ \alpha \}
\union \{ \lambda a.b \ | \ b \in \{a,z\}, a \neq z \}$. It is easy
to see that the following equations hold:
\begin{eqnarray*}
\mathcal{A}^I &=& \{ x,y \} \\
\mathcal{N}^{(o,o)} &=& B \union \mathcal{A}^I \cdot
\mathcal{N}^{(o,o)} = (\mathcal{A}^I)^* \cdot B \\
\mathcal{A}^{(o,o)} &=& \{ \alpha \} \union (\mathcal{A}^I)^+ \cdot B \\
\mathcal{A}^o = \mathcal{N}^o &=& \{ z \} \union \mathcal{A}^{(o,o)} \cdot \mathcal{N}^o = (\mathcal{A}^{(o,o)})^* \cdot \{ z \}
\end{eqnarray*}
Hence $\mathcal{A}^o = \left( \{\alpha \} \union \{x,y\}^+ \cdot
\left( \{\alpha \} \union \{\lambda a.b \ | \ b \in \{a,z\}, a \neq
z \} \right) \right)^* \cdot \{ z \}$. Since $u$ is safe, it cannot
contain terms of the form $\lambda a . z$ with $a \neq z$ occurring
at an operand position, therefore since $u$ belongs to
$\mathcal{A}^o$ we have:
\begin{equation}
u \in \left( \{\alpha\} \union \{x,y\}^+ \cdot \{\alpha,
\underline{i} \} \right)^* \cdot \{ z \} \label{eqn:u}
\end{equation}
where $\underline{i}$ is the identity term of type $o\rightarrow o$.


We observe that $\encode{k} \underline{i} =_\beta \underline{i}$ for
all $k \in \nat$ and for $l\geq 1$, for all $k_1, \ldots k_l \in
\nat$, $\encode{k_1}\ldots \encode{k_l} \alpha =_\beta
\encode{k_1\times \ldots \times k_l} \alpha$. Hence for all $m,n \in
\nat$ we have:
\begin{equation}
\begin{array}{llr}
\{\encode{m},\encode{n}\}^+ \cdot \{\alpha, \underline{i} \} &=_\beta
\{ \underline{i} \} \union
\{ \encode{m^i n^j} \alpha \ |\ i+j \geq 1 \} \nonumber \\
&= \{ \encode{m^i n^j} \alpha \ |\ i,j \geq 0 \} & ( \mbox{since } \underline{i} = \encode{0} \alpha) \end{array}
\label{eqn:intermediate}
\end{equation}
therefore:
$$\begin{array}{llr}
u[\encode{m}, \encode{n}/x,y] &\in \left( \{ \alpha \} \union \{\encode{m},\encode{n}\}^+ \cdot \{\alpha, \underline{i} \} \right)^* \cdot \{ z \}  & \mbox{(by Eq.\ \ref{eqn:u})} \\
&=_\beta \left( \{\alpha \} \union \{ \encode{m^i n^j}
\alpha \ | \ i,j \geq 0 \} \right)^* \cdot \{ z \} & \mbox{(by Eq.\ \ref{eqn:intermediate})}  \\
&=_\beta \left\{ \encode{m^i n^j}
\alpha \ | \ i,j \geq 0 \right\}^* \cdot \{ z \} & \mbox{($\alpha z =_\beta \encode{1} \alpha z$)}.
\end{array}$$

Furthermore, for all $m,n,r,i,j\in \nat$ we have $\encode{m^i n^j}
\alpha (\alpha^r z) =_\beta \alpha^{r + m^i n^j} z$, hence
$u[\encode{m} \encode{n}/x,y] =_\beta \alpha^{p(m,n)} z$ where
$p(m,n) = \sum_{0\leq k \leq d} m^{i_k} n^{j_k}$ for some $i_k,j_k
\geq 0$, $k \in\{ 0,..,d \}$ and $d\geq 0$. Thus $U \encode{m}
\encode{n} =_\beta \encode{p(m,n)}$. \qed


For instance, the term $ C = \lambda F G H \alpha x . H (
\underline{\lambda y . G \alpha x} ) (F \alpha x)$ used by
Schwichtenberg \cite{citeulike:622637} to define the conditional
operator is unsafe since the underlined subterm is of order $1$,
occurs at an operand position and contains an occurrence of $x$ of
order $0$.

    \input{safe_misc.texi}
    \input{sec_safeia.texi}


\chapter{A concrete presentation of game semantics}
    \label{chap:concrete_gamesem}
    We make an explicit correspondence between the game denotation of a
term and its syntax. Our approach follows ideas recently introduced
in \cite{OngLics2006}, mainly the notion of computation tree of a
simply-typed $\lambda$-term and traversals over the computation
tree. A computation tree can be regarded as an abstract syntax tree
(AST) of the $\eta$-long normal form of a term. A traversal is a
justified sequence of nodes of the computation tree respecting some
formation rules. Traversals are used to describe computations. An
interesting property is that the \emph{P-view} of a traversal
(computed in the same way as P-view of plays in Game Semantics) is a
path in the computation tree.

The main result of this paper is called the
\emph{Correspondence Theorem} (theorem \ref{thm:correspondence}). It
states that traversals over the computation tree are just
representations of the uncovering of plays in the
strategy-denotation of the term. Hence there is an isomorphism
between the strategy denotation of a term and its revealed game
denotation ({\it i.e.}~its strategy denotation where internal moves are
not hidden after composition). This theorem permits us to explore
the effect that a given syntactic restriction (such as the safety restriction) has on the strategy
denoting a term.

To really make use of the Correspondence Theorem, it will be
necessary to restate it in the standard game-semantic framework in
which internal moves are hidden. For that purpose, we will define a
\emph{reduction} operation on traversals responsible of eliminating
the ``internal nodes'' of the computation. This leads to a
correspondence between the standard game denotation of a term and
the set of reductions of traversals over its computation tree.
Fortunately, the reduction operation preserves the good properties
of traversals. This is guaranteed by the facts that the P-view of
the reduction of a traversal is equal to the reduction of the P-view
of the traversal, and the O-view of a traversal is the same as the
O-view of its reduction (lemma \ref{lem:redtrav_trav}). \vspace{8pt}

\emph{Related works}: Traversals of a computation tree provide a way
to perform \emph{local computation} of $\beta$-reductions as opposed
to a global approach where the $\beta$-reduction is implemented by
performing substitutions. A notion of local computation of
$\beta$-reduction has been investigated in
\cite{DanosRegnier-Localandasynchronou} through the use of special
graphs called ``virtual nets'' that embed the lambda-calculus.

In \cite{DBLP:conf/lics/AspertiDLR94}, a notion of graph based on
Lamping's graphs \citep{lamping} is introduced to represent
$\lambda$-terms. The authors unify different notions of paths
(regular, legal, consistent and persistent paths) that have appeared
in the literature as ways to implement graph-based reduction of
lambda-expressions. We can regard a traversal as an alternative
notion of path adapted to the graph representation of
$\lambda$-expressions given by computation trees.



%Is there any unsafe term whose game semantics is a strategy where
%pointers can be recovered?
%
%The answer is yes: take the term $T_i = (\lambda x y . y) M_i S$
%where $i =1..2$ and $\Gamma \vdash_s S : A$. $T_1$ and $T_2$ both
%$\beta$-reduce to the safe term $S$, therefore
%$\sem{T_1}=\sem{T_2}=\sem{S}$. But $T_1$ is safe whereas $T_2$ is
%unsafe. Since it is possible to recover the pointer from the game
%semantics of $S$, it is as well possible to recover the pointer from
%the semantics of $T_2$ which is unsafe.

\section{Computation tree}
We work in the general setting of the simply-typed
$\lambda$-calculus extended with a fixed set $\Sigma$ of
higher-order uninterpreted constants \footnote{A constant $f$ is
  \emph{uninterpreted} if the small-step semantics of the language
  does not contain any rule of the form $f \dots \rightarrow e$. $f$
  can be regarded as a data constructor.}

For the rest of the section we fix a simply-typed term $\Gamma \vdash M :T$.

\subsection{$\eta$-long normal form}

The $\eta$-long normal form appeared in
\citep{DBLP:journals/tcs/JensenP76} and
\citep{DBLP:journals/tcs/Huet75} under the names \emph{long reduced
form} and \emph{$\eta$-normal form} respectively. It was then
investigated in \citep{huet76} under the name \emph{extensional
form}.

The $\eta$-expansion of $M: A\typear B$ is defined to be the term
$\lambda x . M x : A\typear B$ where $x:A$ is a fresh variable. A
term $M : (A_1,\ldots,A_n,o)$ can be expanded in several steps into
$\lambda \varphi_1 \ldots \varphi_l . M \varphi_1 \ldots \varphi_l$
where the $\varphi_i:A_i$ are fresh variables. The $\eta$-normal
form of a term is obtained by hereditarily $\eta$-expanding every
subterm occurring at an operand position.

\begin{definition}[$\eta$-long normal form]
A simply-typed term is either an abstraction or it can be written uniquely as
$s_0 s_1 \ldots s_m$ where $m\geq0$ and $s_0$ is a variable, a $\Sigma$-constant or an abstraction.
The $\eta$-long normal form of a term $t$, written $\elnf{t}$ or sometimes $\etanf{t}$,
is defined as follows:
\begin{align*}
\elnf{\lambda x . s } &= \lambda x . \elnf{s} \\
\elnf{\alpha s_1 \ldots s_m : (A_1,\ldots,A_n,o)} &= \lambda \overline{\varphi} . \alpha \elnf{s_1}\ldots \elnf{s_m} \elnf{\varphi_1} \ldots \elnf{\varphi_n}
& \mbox{with $m,n\geq0$}\\
\elnf{(\lambda x . s) s_1 \ldots s_p : (A_1,\ldots,A_n,o) } &= \lambda \overline{\varphi} . (\lambda x . \elnf{s}) \elnf{s_1} \ldots \elnf{s_p} \elnf{\varphi_1} \ldots \elnf{\varphi_n}
& \mbox{with $p\geq 1,n\geq 0$}
\end{align*}
where $x$ and each $\varphi_i : A_i$ are variables and $\alpha$ is
either a variable or a constant.
\end{definition}

For $n=0$, the first clause in the definition becomes:
$$\elnf{x s_1 \ldots s_m : o} = \lambda . x \elnf{s_1} \elnf{s_2} \ldots \elnf{s_m},$$
and we deliberately keep the \textsl{dummy} lambda in the right-hand
side of the equation because it will play an important role in the
correspondence with game semantics.



Note that our version of the $\eta$-long normal form is defined not only for $\beta$-normal terms but also for any simply-typed term.
Moreover it is defined in such a way that $\beta$-normality is preserved:
\begin{lemma}
The $\eta$-long normal form of a term in $\beta$-normal form is also in $\beta$-normal form.
\end{lemma}
\begin{proof}
By induction on the structure of the term and the order of its type.
\emph{Base case}:
If $M=x:0$ then $\elnf{x} = \lambda . x$ is also in $\beta$-nf.
\emph{Step case}:
The case $M = (\lambda x . s) s_1 \ldots s_m : (A_1,\ldots,A_n,o)$ with $m>0$ is not possible since $M$ is in
$\beta$-normal form.
Suppose $M = \lambda x . s$ then $s$ is in $\beta$-nf. By the induction hypothesis $\elnf{s}$ is also in $\beta$-nf and therefore
so is $\elnf{M} = \lambda x . \elnf{s}$.

Suppose $M= \alpha s_1 \ldots s_m : (A_1,\ldots,A_n,o)$. Let $i,j$
range over $1..n$ and $1..m$ respectively. The $s_j$ are in
$\beta$-nf and the $\varphi_i$ are variables of order smaller than
$M$, therefore by the induction hypothesis the $\elnf{\varphi_i}$ and
the $\elnf{s_j}$ are in $\beta$-nf. Hence $\elnf{M}$ is also in
$\beta$-nf.
\end{proof}

\begin{lemma}[$\eta$-long normalisation preserves safety]
If $\Gamma \vdash s$ then $\Gamma \vdash \elnf{s}$.
\end{lemma}
\begin{proof}

First we observe that for any variable or constant $x$ we have $x \vdash \elnf{x}$. The proof is by induction on $\ord{x}$. Base case: $x$ is of ground type and we have $x \vdash x = \elnf{x}$. Step case:
$x:(A_1, \ldots, A_n,o)$ with $n>0$. Let $\varphi_i:A_i$ be fresh variables for $1\leq i\leq n$. The (var) rules gives $\varphi_i  \vdash \varphi_i$ and since $\ord{A_i} < \ord{x}$ the induction hypothesis gives $\varphi_i \vdash \elnf{\varphi_i}$. Using (wk) we obtain $x, \overline{\varphi} \vdash \elnf{\varphi_i}$.
The application rule gives $x, \overline{\varphi} \vdash x \elnf{\varphi_1} \ldots \elnf{\varphi_n} : o$ and the abstraction rule gives $ x \vdash \lambda \overline{\varphi} . x \elnf{\varphi_1} \ldots \elnf{\varphi_n} = \elnf{x}$.


We now prove the lemma by induction on the structure of $s$.
The base case (where $s$ is some variable $x$) is covered by the previous observation.
\emph{Step case:}
\begin{itemize}
\item $s = x s_1 \ldots s_m$ with $x: (B_1, \ldots, B_m, A_1, \ldots, A_n, o)$ with $m\geq 0$, $n>0$ and $s_i : B_i$ for $1 \leq i \leq m$.

Let $\varphi_i: A_i$ be fresh variables for $1\leq i \leq n$. By the previous observation we have $\varphi_i \vdash \elnf{\varphi_i}$ which in turn gives $\Gamma , \overline{\varphi} \vdash \elnf{\varphi_i}$ using the weakening rule.

The judgement $\Gamma \vdash x s_1 \ldots s_m$ is formed using the (app) rule therefore each $s_j$ is safe for $1\leq j \leq m$. By the induction hypothesis we have $\Gamma \vdash \elnf{s_j}$ and by weakening we get $\Gamma, \overline{\varphi} \vdash \elnf{s_j}$.

The application rule gives $\Gamma, \overline{\varphi} \vdash
x \elnf{s_1} \ldots \elnf{s_m} \elnf{\varphi_1} \ldots \elnf{\varphi_n} : o$. Finally the (abs) rule gives $\Gamma \vdash \lambda \overline{\varphi} . x \elnf{s_1} \ldots \elnf{s_m}  \elnf{\varphi_1} \ldots \elnf{\varphi_n} = \elnf{s}$, the side-condition of (abs) being met since $\ord{\elnf{s}} = \ord{s}$.


\item $s = t s_0 \ldots s_m$ where $t$ is an abstraction. Again, using the induction hypothesis it is easy to show that $\Gamma \vdash \elnf{s} = \elnf{t} \elnf{s_0} \ldots \elnf{s_m} \elnf{\varphi_1} \ldots \elnf{\varphi_n}$ holds for some fresh variables $\varphi_1$, \ldots, $\varphi_n$.

\item $s = \lambda \overline{\eta} . t$ where $t$ is not an abstraction. By the induction hypothesis we have $\Gamma, \overline{\eta} \vdash \elnf{t}$ and by the abstraction rule we have $\Gamma \vdash \lambda \overline{\eta} . \elnf{t} = \elnf{s}$.
\end{itemize}
\end{proof}

Note that in general the converse does not hold, for instance $\lambda x^o . f^{o,(o,o),o} x^o$ is unsafe although $\elnf{\lambda x . f x} = \lambda x^o \varphi^{o,o} . f x \varphi$ is safe (and not homogeneous). For terms with homogeneous types however, the converse does hold:
\begin{lemma}
If $\Gamma \vdash \elnf{s}$ is homogeneously safe (i.e. it is a safe judgement of the safe $\lambda$-calculus and each sequent occurring at the nodes of the proof tree is homogeneously typed) then
$\Gamma \vdash s$ is homogeneously safe.
\end{lemma}


\subsection{Computation tree}
The computation tree of a term is a certain tree representation of its
$\eta$-long normal form. It is defined as follows:

\begin{definition}
\label{dfn:comptree} Let $M$ be a simply-typed term in $\eta$-normal
form. Then $M$ is either an abstraction or it can be written
uniquely as $s_0 s_1 \ldots s_m : o$ for some $m\geq0$ where $s_0$
is a variable, a constant or an abstraction and each of the $s_j$
for $j\in 1..m$ is in $\eta$-normal form. The
\defname{computation tree} $\tau(M)$ of $M$ is defined by induction
on the structure of the term:
\begin{enumerate}[-]
\item If $n\geq0$ and $s$ is not an abstraction then:
$$ \tau(\lambda x_1 \ldots x_n . s) =
      \pstree[levelsep=3ex]
        { \TR{\lambda x_1 \ldots x_n} }
        { \SubTree{\tau(s)^{-}} }
$$
where $\tau(s)^{-}$ denotes the tree obtained after deleting the root of $\tau(s)$.

\item If $m\geq0$ and $\alpha$ is a variable or constant then:
$$ \tau( \alpha s_1 \ldots s_m : o) =
    \tree{\lambda}
    {
        \pstree[levelsep=3ex]
            { \TR{\alpha} }
            { \SubTree{\tau(s_1)} \SubTree[linestyle=none]{\ldots} \SubTree{\tau(s_m)}
            }
    }
$$

\item If $n \geq 1$ then:
$$ \tau((\lambda x.s) s_1 \ldots s_n : o) =
    \tree{\lambda}
    {
        \pstree[levelsep=3ex]
            { \TR{@} }
            {
            \SubTree{\tau(\lambda x.s)}    \SubTree{\tau(s_1)} \SubTree[linestyle=none]{\ldots} \SubTree{\tau(s_n)}
            }
    }
$$
\end{enumerate}

If $M$ is not in $\eta$-normal form then $\tau(M)$ is defined as the
computation tree of its $\eta$-normal form ($\tau(M) =
\tau(\etanf{M})$).
\end{definition}

The nodes (and leaves) of the tree are of three kinds:
\begin{itemize}
\item $\lambda$-nodes labelled $\lambda \overline{x}$ (note that a $\lambda$-node represents several consecutive variable abstractions),
\item application nodes labelled @,
\item variable or constant nodes labelled $\alpha$ for some constant or variable $\alpha$.
\end{itemize}
A node is said to be \defname{prime} if it is the 0$^{th}$ child of an @-node.

\emph{Notations:} We write $r$ for the root of $\tau(M)$. We write $E$ to denote the parent-child relation
of the tree, $N$ for the set of nodes of $\tau(M)$,
$N_\Sigma$ for the set of $\Sigma$-labelled nodes, $N_@$ for the set
of @-labelled nodes, $N_{\sf var}$ for the set of variable nodes,
$N_{\sf fv}$ for the subset of $N_{\sf var}$ constituted of free-variable
nodes, $N_{\sf spawn}$ for the set $N \inter E \relimg{N_@ \union N_\Sigma}$ constituted of children of constant-nodes and @-nodes and $N_{\sf prime}$ for the set of prime nodes.


Let $\mathcal{T}$ denote the set of $\lambda$-terms.
Each subtree of the computation tree $\tau(M)$ represents a subterm of $\elnf{M}$.
We define the function $\kappa : N \rightarrow \mathcal{T}$ that maps a node $n \in N$ to the subterm of $\elnf{M}$
corresponding to the subtree of $\tau(M)$ rooted at $n$.
In particular $\kappa(r) = \elnf{M}$.

\begin{definition}[Type and order of a node]
\label{def:nodeorder}
Suppose $\Gamma \vdash M : T$.
The \defname{type} of a node $n$ of $\tau(M)$ written $type(n)$ is defined as follows:
\begin{eqnarray*}
type(r) &=& \Gamma \rightarrow T \\
type(\alpha:A) &=& A, \mbox{ where $\alpha$ is a variable or constant} \\
type(n) &=& \hbox{ type of the term $\kappa(n)$ for $n \in (N_\lambda \union N_@) \setminus \{r \}$\ .}
\end{eqnarray*}
The order of a node $n$ written $\ord{n}$ is defined to be the order of the type of $n$.
\end{definition}

In particular, $\ord{@} = 0$, $\ord{\lambda \overline{\xi}} = 1+
\max_{z\in \overline{\xi}} \ord{z}$ for $\lambda \overline{\xi}\neq
r$ and if $r=\lambda \overline{\xi}$ then $\ord{r} = 1 + \max_{z\in
\overline{\xi}\union \Gamma} \ord{z}$ with the convention $\max
\emptyset = -1$.

\begin{remark} \hfill
\begin{itemize}
\item In a computation tree, nodes at even level are $\lambda$-nodes and nodes at odd level are either application nodes,
variable or constant nodes;

\item for any ground type variable or constant $\alpha$,
$\tau(\alpha) = \tau(\lambda . \alpha) =  \pstree[levelsep=3ex]
    { \TR{\lambda } }
    { \TR{\alpha}
    }$;

\item for any higher-order variable or constant $\alpha : (A_1,\ldots,A_p,o)$, the computation tree $\tau(\alpha)$ has the following form:
$ \pstree[levelsep=3ex]{\TR{\lambda}}
        {\pstree[levelsep=3ex]
                { \TR{\alpha} }
                { \tree{\lambda \overline{\xi_1}}{\TR{\ldots}} \TR{\ldots} \tree{\lambda \overline{\xi_p}}{\TR{\ldots}}
                }
        }
$;

\item for any tree of the form
        $ \pstree[levelsep=4ex]
            { \TR{\lambda \overline{\varphi}} }
            { \pstree[levelsep=3ex]
                {\TR{n}}
                {\TR{\lambda \overline{\xi_1}} \TR{\ldots} \TR{\lambda \overline{\xi_p}}}
            }
        $,
    we have $\ord{\kappa(n)}=0$.

\end{itemize}
\end{remark}


\subsection{Pointers and justified sequence of nodes}

\begin{definition}[Binder]
Let $n$ be a variable node of the computation tree labelled $x$. We
say that a node $n$ is bound by the node $m$, and $m$ is called the
binder of $n$, if $m$ is the closest node in the path from $n$ to
the root of the tree such that $m$ is labelled $\lambda
\overline{\xi}$ with $x\in \overline{\xi}$.
\end{definition}

\begin{definition}[Enabling]
The \defname{enabling relation} $\vdash$ is defined on the set of
nodes of the computation tree as follows. We write $m \vdash n$ and
we say that $m$ enables $n$ if and only if
\begin{itemize}
\item $n$ is a bound variable node and $m$ is the binder of $n$. We will write $m \vdash_i n$ to precise that $n$
is the $i^{\sf th}$ variable bound by $m$;
\item or $n$ is a free variable node and $m$ is the root of the computation
tree;
\item or $n$ is a $\lambda$-node and $m$ is the parent node of $n$.
\end{itemize}
\end{definition}

We say that a node $n_0$ of a justified sequence is
\defname{hereditarily justified} by $n_p$ if there are nodes $n_1,
\ldots, n_{p-1}$ in the sequence such that $n_i$ points to $n_{i+1}$
for all $i\in 0..p-1$.

For any set of nodes $S$ we write $S^{\upharpoonright r}$ for $\{ n \in S \ | \ r  \vdash^* n \}$ -- the subset of $S$ constituted of
nodes hereditarily enabled by $r$.
We call \defname{input-variables nodes} the elements of $N_{\sf var}^{\upharpoonright r}$ i.e.\
variables that are hereditarily enabled by the root. $N_{\sf var}^{\upharpoonright r}$ is also the set of nodes that are hereditarily enabled by a free variable or by a variable bound by the root.
\smallskip

We use the following numbering conventions:
the first child of a @-node is numbered $0$;
the first child of a variable or constant node is numbered $1$;
and variables in $\overline{\xi}$ are numbered from $1$ onward ($\overline{\xi} = \xi_1 \ldots \xi_n$).
We write $n.i$ to denote the $i$th child of node $n$.

\begin{definition}[Justified sequence of nodes]
A \defname{justified sequence of nodes} is a sequence of nodes of
the computation tree $\tau(M)$ with pointers such that each variable
or $\lambda$-node $n$ different from the root has a pointer to a
node $m$ occurring before it the sequence and such that $m \vdash
n$.

If $n$ points to $m$ then we say that $m$ \emph{justifies} $n$. We
represent the pointer in the sequence as follows \Pstr[0.4cm]{
(m){m} \ldots (n-m,45:i) n }. where the label indicates that either
$n$ is labelled with the $i$th variable abstracted by the
$\lambda$-node $m$ or that $n$ is the $i^{\sf th}$ child of $m$.
\end{definition}

Note that justified sequences are also defined for open terms:
occurrences of nodes in $N_{\sf fv}$ must point to an occurrence of the
root of the computation tree. Thus a pointer in a justified sequence of nodes has
one of the following forms:
$$
\Pstr[18pt]{ (m){r} \cdot \ldots \cdot (n-m,40){z} }
\hspace{1.5cm}
\Pstr{ (m){\lambda \overline{\xi}} \cdot \ldots \cdot (n-m,40:i){\xi_i} }
\hspace{1.5cm}
\Pstr{ (m){@} \cdot \ldots \cdot (n-m,40:j){\lambda \overline{\eta}} }
\hspace{1.5cm}
\Pstr{ (m){\alpha } \cdot \ldots \cdot (n-m,40:k){\lambda \overline{\eta}} }
$$
for some occurrences $r$ of $\tau(M)$'s root, $z \in N_{\sf fv}$,
bound variables $\xi_1,
\ldots \xi_n$, $\alpha \in N_{\Sigma} \union
N_{\sf var}$, $i \in 1..n$, $j$ ranges from $0$ to the number of
children nodes of @ minus 1 and $k \in 1 ..arity(\alpha)$.
\bigskip

\emph{Notations}: We write $s = t$ to denote that the justified sequences $t$ and $s$
have same nodes \emph{and} pointers. Justified sequence of nodes can
be ordered using the prefix ordering: $t \sqsubseteq t'$ if and only
if $t=t'$ or the sequence of nodes $t$ is a finite prefix of $t'$
(and the pointers of $t$ are the same as the pointers of the
corresponding prefix of $t'$). Note that with this definition,
infinite justified sequences can also be compared. This ordering
gives rise to a complete partial order.
We say that a node $n_0$ of a justified sequence is \defname{hereditarily justified} by $n_p$ if there are nodes $n_1, n_2, \ldots n_{p-1}$ in the sequence such that for all $i\in 0..p-1$, $n_i$ points to $n_{i+1}$.
We write $t^\omega$ to denote the last occurrence of $t$ and $\ip(t)$ for the immediate prefix of $t$ obtained by removing $t$'s last node.

We define a filtering operation on sequences of nodes:
\begin{definition}[Hereditary filtering]
Let $s$ be a justified  sequence of nodes from $\tau(M)$
and $n$ be an occurrence in $t$ of some node $n \in N_{\sf spawn}$.

We write $s \upharpoonright n$ to denote the subsequence of $s$ constituted of nodes that are hereditarily justified by $n$, where the pointer's target of all occurrences of free variable nodes in $t$ are set to $n$ (instead of $t$'s first node).

Thus $s \upharpoonright n$ is a valid justified sequence of nodes of the tree $\tau(\kappa(n))$.
\end{definition}


\begin{lemma}
\label{lem:filtercontinous}
The filtering function $\_ \upharpoonright n$ defined on the cpo of justified sequences ordered by the prefix ordering
is continuous.
\end{lemma}
\begin{proof}
Clearly $\_ \upharpoonright n$ is monotonous.
Suppose that $(t_i)_{i\in\omega}$ is a chain of justified sequence of nodes. Let $u$ be a finite prefix of $(\bigvee t_i) \filter n$.
Then $u = s \filter n$ for some finite prefix $s$ of $\bigvee t_i$. Since $s$ is finite we must have $s \sqsubseteq t_j$ for some $j\in\omega$.
Therefore $u \sqsubseteq t_j \filter n \sqsubseteq \bigvee (t_j \filter  n)$.
This is valid for any finite prefix $u$ therefore $(\bigvee t_i) \filter  n \sqsubseteq \bigvee (t_j \filter n)$.
\end{proof}



The notion of \defname{P-view} $\pview{t}$ of a justified sequence
of nodes $t$ is defined the same way as the P-view of a justified
sequences of moves in Game Semantics:

\begin{definition}[P-view of justified sequence of nodes]
The P-view of a justified sequence of nodes $t$ of $\tau(M)$, written $\pview{t}$, is defined as follows:
\begin{eqnarray*}
 \pview{\epsilon} &=&  \epsilon \\
 \pview{s \cdot n }  &=&  \pview{s} \cdot n \qquad \mbox{for $n \notin N_\lambda$, }\\
 \pview{\Pstr{ s \cdot (m){m} \cdot \ldots \cdot (lmd-m,25){\lambda \overline{\xi}}}} &=&
        \Pstr{ \pview{s} \cdot (m2){m} \cdot (lmd2-m2,60){\lambda \overline{\xi}} } \\
 \pview{s \cdot r }  &=&  r
\end{eqnarray*}
where $r$ is the root of the tree $\tau(M)$.

The equalities in the definition determine pointers implicitly. For
instance in the second clause, if in the left-hand side, $n$ points
to some node in $s$  that is also present in $\pview{s}$ then in the
right-hand side, $n$ points to that occurrence of the node in
$\pview{s}$.
\end{definition}

The O-view of $s$, written $\oview{s}$, is defined dually.
\begin{definition}[O-view of justified sequence of nodes]
The O-view of a justified sequence of nodes $t$ of $\tau(M)$, written $\oview{t}$, is defined as follows:
\begin{eqnarray*}
 \oview{\epsilon} &=&  \epsilon \\
 \oview{s \cdot \lambda \overline{\xi} }  &=&  \oview{s} \cdot \lambda \overline{\xi} \\
 \oview{\Pstr{s \cdot (m){m} \cdot \ldots \cdot (x-m,30){x}}} &=&
    \Pstr{ \oview{s} \cdot (m2){m} \cdot (n2-m2,60){x} } \qquad \mbox{ for $x \in N_{\sf var}$ }\\
 \oview{s \cdot n }  &=&  n \qquad \mbox{ for $x \in N_@ \union N_\Sigma$ }
\end{eqnarray*}
\end{definition}

We borrow some game semantic terminology:
\begin{definition} A justified sequence of nodes $s$ satisfies:
\begin{itemize}[-]
\item \defname{Alternation} if for any two consecutive nodes in $s$, one is a $\lambda$-node
and the other is not;
\item \defname{P-visibility} if every variable node in $s$ points to a node occurring in the P-view a that point;
\item  \defname{O-visibility} if every lambda node in $s$ points to a node occurring in the O-view a that point.
\end{itemize}
\end{definition}

\begin{property}
\label{proper:pview_visibility}
The P-view (resp. O-view) of a justified sequence verifying P-visibility (resp. O-visibility)
is a well-formed justified sequence verifying P-visibility (resp. P-visibility).
\end{property}
This is proved by an easy induction.

\subsection{Adding value-leaves to the computation tree}
\label{sec:adding_value_leaves}

We now add another ingredient to the computation tree defined in
the previous section. Let $\mathcal{D}$
denote the set of values of base type $o$.  We add
\defname{value-leaves} to $\tau(M)$ as follows: Every node $n \in \tau(M)$ has one child leaf labelled $v_n$ for every possible value $v \in \mathcal{D}$.
We write $V$ for the set of nodes and leaves of
the computation tree.  For $\$$ ranging in $\{@, \lambda, var \}$,
we write $V_\$$ to denote the set $N_\$ \union \{ v_n \ | \ n \in
N_\$, v \in \mathcal{D} \}$.

%If $n$ is a $\lambda$-node then its value-leaves are numbered from $1$ onwards.
%If $n$ is a variable or constant node then its children nodes are numbered from $1$ to $arity(n)$ and
%its value-leaves are numbered from $arity(n)+1$ onwards.
%If $n$ is an application node then its value-leaves are numbered from $1$ onwards.

Everything that we have defined for computation tree can be lifted
to this new version of computation tree. The node order of a
value-leaf is defined to be $0$. The enabling relation $\vdash$ is
extended so that every leaf is enabled by its parent node. The
definition of justified sequence does not change.
When representing a link in a justified sequence going from a value-leaf $v_n$ to a node $n$,
we label the link with $v$:
$$
\Pstr{ (n){n} \cdot \ldots \cdot (vn-n,40:v){v_n} }
$$

For the definition
of P-view, O-view and visibility, value-leaves are treated as
$\lambda$-nodes if they are at odd level in the computation tree and
as variable nodes if they are at an even level.

From now the term ``computation tree'' refers to this extended
definition.
\vspace{10pt}

We say that a node $n$ in of a justified sequence of nodes is
\defname{matched} by the value-leaf $v_n$ if there is an occurrence of $v_n$ for some value $v$ in the
sequence that points to $n$, otherwise we say that $n$ is
\defname{unmatched}. The last unmatched node is called the
\defname{pending node}.  A justified sequence of nodes is
\defname{well-bracketed} if each value-leaf occurring in it is justified by the pending node at that point.
If $t$ is a traversal then we write
$?(t)$ to denote the subsequence of $t$ consisting only of unmatched
nodes.

\subsection{Traversal of the computation tree}
\label{subsec:traversal}
A \emph{traversal} is a justified sequence of nodes of the computation tree where each node indicates a step that is taken during the evaluation of the term.

\subsubsection{Traversals for simply-typed $\lambda$-terms}

We first consider the simply-typed $\lambda$-calculus without interpreted constants.
Everything remains valid in the presence of \emph{uninterpreted} constants as we can just
consider them as free variables.

We define the notion of traversal over the computation tree $\tau(M)$.
We will then we show how to extend the notion of traversal to more general settings with interpreted constants.

\begin{definition}[Traversals for simply-typed $\lambda$-terms] \rm
\label{def:traversal} The set $\travset(M)$ of \defname{traversals}
over $\tau(M)$ is defined by induction over the following rules:

\noindent \emph{Initialization rules}
\begin{description}
\item[\rulenamet{Empty}] $\epsilon \in \travset(M)$.
\item[\rulenamet{Root}] The single-node sequence $r$, where $r$ denotes the root of $\tau(M)$, is a traversal.
%$ r \in \travset(M)$.
\end{description}

\noindent \emph{Structural rules}
\begin{description}
\item[\rulenamet{Lam}] If $t \cdot \lambda \overline{\xi}$ is a traversal then so is
$t \cdot \lambda \overline{\xi} \cdot n$ where $n$ denotes $\lambda
\overline{\xi}$'s child.

Moreover if $n$ is a variable node then it
points to the only
occurrence of its enabler that is still present in $\pview{t
\cdot \lambda \overline{\xi}}$.
In particular, if $n$ is a free variable node then $n$ points to the first node of $t$ (the root). (Prop. \ref{prop:pviewtrav_is_path} will show that indeed $n$'s enabler occurs exactly once in the P-view since P-views correspond to paths in the tree.)

\item[\rulenamet{App}] If $t \cdot @$ is a traversal then so is \Pstr[0.4cm]{t \cdot (m) @  \cdot (n-m,40:0) n}.
%{\em i.e.}~the next visited node is the $0^{th}$ child node of
%@: the node corresponding to the operator of the application.
\end{description}

\noindent \emph{Input-variable rules}
\begin{description}
\item[\rulenamet{InputVar$^{val}$}] If $t_1 \cdot x \cdot t_2$ is a traversal
with $x \in N_{\sf var}^{\upharpoonright r}$ and $?(t_1 \cdot x
\cdot t_2)=?(t_1) \cdot x$ then so is \Pstr[0.4cm]{t_1 \cdot
(x){x} \cdot t_2 \cdot (xv-x,38:v){v_x} } for all $v \in
\mathcal{D}$.

\item[\rulenamet{InputVar}] If $t_1 \cdot x \cdot t_2$ is a traversal with
  $x \in N_{\sf var}^{\upharpoonright r}$ and $x$ is the pending node in $t$ ($?(t_1 \cdot x \cdot
  t_2)=?(t_1) \cdot x$) then so is $t_1 \cdot x \cdot t_2 \cdot
  n$ for any $\lambda$-node $n$ whose parent occurs in
  $\oview{t_1 \cdot x}$, $n$ pointing to some occurrence of its
  parent node in $\oview{t_1 \cdot x}$.
\end{description}

\noindent \emph{Copy-cat answer rules}
\begin{description}
\item[\rulenamet{Answer-@-$\lambda$}]
  If \Pstr{t \cdot (app){@} \cdot (lz-app,60:0){\lambda
\overline{z}}  \ldots  (lzv-lz,60:v){v}_{\lambda \overline{z}} }
is a traversal then so is \Pstr[0.6cm]{t \cdot (app){@} \cdot
(lz-app,60){\lambda \overline{z}} \ldots
(lzv-lz,60:v){v}_{\lambda \overline{z}} \cdot
(appv-app,45:v){v}_@}.

\item[\rulenamet{Answer-$\lambda$-@}] If \Pstr[0.4cm]{t \cdot \lambda \overline{\xi} \cdot (x){@}  \ldots   (xv-x,50:v){v}_@}
is a traversal then so is \Pstr[0.5cm]{t \cdot (lmd){\lambda
\overline{\xi}} \cdot (x){@}  \ldots  (xv-x,50:v){v}_@  \cdot
(lmdv-lmd,30:v){v}_{\lambda \overline{\xi}} }.

\item[\rulenamet{Answer-var-$\lambda$}] If \Pstr[0.4cm]{t \cdot y \cdot (lmd){\lambda \overline{\xi}}
\ldots (lmdv-lmd,50:v){v}_{\lambda \overline{\xi}} } is a
traversal for some variable $y\not\in N_{\sf var}^{\upharpoonright
r}$ then so is \Pstr[0.7cm]{t \cdot (y){y} \cdot (lmd){\lambda
\overline{\xi}} \ldots (lmdv-lmd,30:v){v}_{\lambda
\overline{\xi}}  \cdot (vy-y,50:v){v}_y }.

\item[\rulenamet{Answer-$\lambda$-var}] If \Pstr[0.4cm]{t \cdot \lambda \overline{\xi} \cdot (x){x}  \ldots   (xv-x,50:v){v}_x}
is a traversal then so is \Pstr[0.5cm]{t \cdot (lmd){\lambda
\overline{\xi}} \cdot (x){x}  \ldots  (xv-x,50:v){v}_x  \cdot
(lmdv-lmd,30:v){v}_{\lambda \overline{\xi}} }.
\end{description}

\begin{description}
\item[\rulenamet{Var}]
If \Pstr[0.5cm]{t' \cdot (n){n} \cdot (lx){\lambda \overline{x}}
    \ldots (x-lx,50:i){x_i} } is a traversal for some variable
    $x_i$ not in $N_{\sf var}^{\upharpoonright r}$ then
so is \Pstr[0.6cm]{ t' \cdot (n){n} \cdot
    (lx){\lambda \overline{x}}  \ldots (x-lx,30:i){x_i}  \cdot
    (letai-n,40:i){\lambda \overline{\eta_i}}
     }.
\end{description}
A traversal that cannot be extended by any rule is said to be \emph{maximal}.
\end{definition}


A traversal always starts by visiting the root. Then it mainly
follows the structure of the tree.

The \rulenamet{Var} rule is particular and needs further explanation.
This rule permits the traversal to jump across the computation tree. The idea is that after visiting a
non-input variable node $x$, a jump can be made to the node corresponding to
the subterm that would be substituted for $x$ if all the
$\beta$-redexes occurring in the term were to be reduced.


Let $\lambda \overline{x}$ be $x$'s binder and suppose $x$ is the $i$th variable in $\overline{x}$.
The binding node necessarily occurs previously in the traversal (this will be proved in Prop. \ref{prop:pviewtrav_is_path}). Since $x$ is not hereditarily justified by the root, $\lambda \overline{x}$ is not the root of the tree and therefore it is not the first node of the traversal.
We do a case analysis on the node preceding $\lambda \overline{x}$:
    \begin{itemize}[-]
    \item If it is an @-node then $\lambda \overline{x}$ is necessarily the first child node of that node
    and it has has exactly $|\overline{x}|$ siblings:
    $$\pstree[levelsep=7ex]{\TR{\stackrel{\vdots}{@}}}
    {   \pstree[linestyle=dotted,levelsep=4ex]{\TR{\lambda \overline{x}}\treelabel{0}}
            {\TR{x }}
        \tree{\lambda \overline{\eta_1}}{\vdots}\treelabel{1}
        \TR[edge=\dotedge]{}
        \tree{\lambda \overline{\eta_i}}{\vdots}\treelabel{i}
        \TR[edge=\dotedge]{}
        \tree{\lambda \overline{\eta_{|x|}}}{\vdots}\treelabel{|x|}
    }
    $$
    In that case, the next step of the traversal is a jump to $\lambda \overline{\eta_i}$ -- the $i$th child of
    @ -- which corresponds to the subterm that would be substituted for $x$ if the $\beta$-reduction was
    performed:
    $$\Pstr[19pt]{ t' \cdot
            (n){@} \cdot
            (lx){\lambda \overline{x}} \cdot \ldots \cdot
            (x-lx,40:i){x} \cdot
            (mi-n,40:i){\lambda \overline{\eta_i}} \cdot \ldots
            \in {\travset(M)}   }
    $$

    \item If it is a variable node $y$, then
    the node $\lambda \overline{x}$ was necessarily added to the traversal $t_{\leq y}$ using the \rulenamet{Var} rule (see proposition \ref{prop:pviewtrav_is_path}(i)).
    Therefore $y$ is substituted by the term $\kappa(\lambda \overline{x})$ during the evaluation of the term.

    Consequently, during reduction, the variable $x$ will be substituted by the subterm represented by
    the $i$th child node of $y$. Hence the following justified sequence is also a traversal:
    $$\Pstr[18pt]{ t' \cdot
            (y){y} \cdot
            (lx){\lambda \overline{x}} \cdot \ldots \cdot
            (x-lx,40:i){x} \cdot
            (mi-y,40:i){\lambda \overline{\eta_i}} \cdot \ldots
    }
    $$
    \end{itemize}

\begin{remark}
Our notions of computation tree and traversal differ slightly from \cite{OngLics2006}:
\begin{itemize}[-]
    \item In \cite{OngLics2006} computation trees can have uninterpreted first-order constants. But as we have already observed, uninterpreted constants can be just regarded as free variables thus we do not lose any expressivity here.

    \item In \cite{OngLics2006}, constants are restricted to order one at most since computation tree
    are used to model computation of tree structures. Here we don't need this restriction (as long as constants are uninterpreted - so we can regard them as free variables).


    \item In our setting, we have to deal with \emph{free} variables.
    To model free variables we need the traversal rules \rulenamet{InputVar$^{val}$}, \rulenamet{InputVar}
    as well as the copy-cat answer rules. Whereas in \cite{OngLics2006}, the rule called \rulenamet{Sig} suffices to model the first-order constants necessary to construct tree structures.

    \item In our setting, the introduction of value-leaves
    is necessary in order to model free variables as well as interpreted constants. (We will use them to model the constants of \pcf\ and \ialgol).
    \end{itemize}
\end{remark}

\begin{example}
Consider the following computation tree:
$$\tree{\lambda}
{
    \tree{@}
    {
        \pstree[levelsep=8ex,linestyle=dotted]{\TR{\lambda y}\treelabel{0} }
        {
            \pstree[levelsep=8ex]{\TR{y}}
            {
                \tree{\lambda \overline{\eta_1}}{\vdots} \treelabel{1}
                \TR[edge=\dotedge]{}
                \tree{\lambda \overline{\eta_i}}{\vdots}\treelabel{i}
                \TR[edge=\dotedge]{}
                \tree{\lambda \overline{\eta_n}}{\vdots}\treelabel{n}
            }
        }
        \pstree[levelsep=6ex,linestyle=dotted]{\TR{\lambda \overline{x}}\treelabel{1}}{ \tree{x_i}{\TR{} \TR{} } }
    }
}
$$
An example of traversal of this tree is:
\vspace{0.3cm}
$$ \Pstr{ \lambda \cdot
            (app){@}  \cdot
            (ly){\lambda y} \cdot \ldots \cdot
            (y-ly,40:1){y} \cdot
            (lx-app,50:1){\lambda \overline{x}} \cdot \ldots \cdot
            (x-lx,40:i){x_i} \cdot
            (leta-y,50:i){\lambda \overline{\eta_i} } \cdot \ldots
        }$$
\end{example}

\subsubsection{Traversals for interpreted constants}

\begin{definition}[Well-behaved traversal rule]
\label{def:wellbehaved_traversal} A traversal rule is
\defname{well-behaved} if it can be stated under the following form:
$$\rulef{t = t_1\cdot n \cdot t_2 \in \travset \quad ?(t) = ?(t_1) \cdot n \quad P(t)}
  { \stackrel{  \rule{0pt}{3pt} }{\Pstr[5pt]{ t' = t_1\cdot (n){n} \cdot t_2 \cdot (m-n,35){m} \in \travset}}
   }
    \ m\in S(t)
   $$
such that:
\begin{enumerate}[i.]
  \item $n$ is a variable or a constant node ($n \in N_{\Sigma}\union N_{\sf var}$);
  \item $P$ expresses some condition on $t$;
  \item for every traversal $t$, $S(t)$ is some subset of $E(n)$, the set of children $\lambda$-nodes and value-leaves of $n$.
  If $S(t)$ has more than one element then the rule is non-deterministic.
\end{enumerate}
\end{definition}
Note that if $t$ is well-bracketed then $t'$ is also well-bracketed
and if $?(t)$ satisfies alternation and visibility then so does
$?(t')$.


\begin{example} The rule (InputVar$^{val}$) is an example of non-deterministic well-behaved traversal rule for which $S(t)$ is exactly the set of all children value-leaves of $n$:
$S(t) = \{ v_n \ | \ v \in \mathcal{D} \} $.
However (InputVar) is not well-behaved since it can jump to any node in the O-view at that point and not necessarily to a children node of the last pending node.
\end{example}

In the presence of higher-order interpreted constants, additional rules must be specified to indicate how
the constant nodes should be traversed in the computation tree. These rules
are specific to the language that is being studied.
In the last section of this chapter we will define such traversals for the interpreted constants of
\pcf\ and \ialgol.

From now on, we consider a simply-typed $\lambda$-calculus language extended with
higher-order interpreted constants for which some constant traversal rules have been defined
and we take the following condition as a prerequisite:
\begin{center}
  \textbf{(Condition WB)} The constant traversal rules are well-behaved.
\end{center}


\subsubsection{Some properties of traversals}

\begin{proposition}[counterpart of proposition 6 from \cite{OngHoMchecking2006}]
\label{prop:pviewtrav_is_path}
Let $t$ be a traversal. Then:
\begin{itemize}
\item[(i)] $t$ is a well-defined and well-bracketed justified sequence;
\item[(ii)] $t$ is a well-defined justified sequence verifying alternation, P-visibility and O-visibility;
\item[(iii)] If $t^\omega \in N$ {\it i.e.}~$t$'s last node is not a value-leaf, then $\pview{t}$ is the path in the computation tree going from the root to the node $t^\omega$.
\end{itemize}
\end{proposition}

This is the counterpart of proposition 6 from
\cite{OngHoMchecking2006} which is proved by induction on the
traversal rules. This proof can be easily adapted to take into
account the constant rules (using the assumption that constants
rules are well-behaved) and the presence of value-leaves in the
traversal.
\begin{proof}
The proof of (i), (ii) and (iii) is done simultaneously by induction on the traversal rules. We consider the rules \rulenamet{Var} and \rulenamet{Lam} only.

Rule \rulenamet{Var}: we just give a partial proof of (i). See proposition 6 from \cite{OngHoMchecking2006} for the details of (i), (ii) and (iii). We have to show that in the second case of the \rulenamet{Var} rule, where $p$ is a variable node $y$, the node $\lambda \overline{x}$ has necessarily been added to the traversal $t_{\leq y}$ using the \rulenamet{Var} rule. This is immediate since if the rule \rulenamet{InputVar} was used to produce $t_{<y} \cdot y \cdot \lambda \overline{x}$ this would imply that $\lambda \overline{x}$ is hereditarily justified by the root which in turn implies that $x_i$ is an input-variable which contradicts \rulenamet{Var}'s hypothesis.

Rule \rulenamet{Lam}: we need to show that $n$'s enabler occurs only once in the P-view at that point. By the induction hypothesis we have (by (iii)) that $\pview{t \cdot \lambda \overline{\xi}}$ is a path in the computation tree from the root to $\lambda \overline{\xi}$. $n$'s enabler occurs only once in this path: it is precisely it's binding node. Therefore the traversal $t \cdot \lambda \overline{\xi} \cdot n$ is well-defined and $t \cdot \lambda \overline{\xi} \cdot n$ satisfies P-visibility. Thus (i) and (ii) are verified. Furthermore $n$ is a child of $\lambda \overline{\xi}$ therefore (iii) also holds.
\end{proof}

%In particular to prove that the copy-cat rules are well-defined, one needs to ensure that
%if the last two unmatched nodes are $y$ and $\lambda \overline{\xi}$ in that order, for some non input-variable node $y$ then necessary
%      $y$ and $\lambda \overline{\xi}$ are consecutive nodes in the traversal.
%    This is because in a traversal, a non input-variable $y$ is always followed by a lambda node and whenever this lambda node is answered
%    there is only one way to extend the traversal : by using the copy cat rule to answer the $y$ node.

\begin{definition}
The \defname{reduction of a traversal} $t$ is define as the subsequence $ t
\filter r$ where $r$ denotes the first node in $t$ (which is necessarily $\tau(M)$'s root).
\end{definition}
The effect of this transformation is the elimination of the
``internal nodes'' of the computation. Since @-nodes and $\Sigma$-constants do not have pointers, the
reduction of traversal contains only nodes in $N_\lambda \union
N_{\sf var}$.

We define the set
$$\travset(M)^{\upharpoonright r} = \{ t  \upharpoonright r \ | \  t  \in \travset(M) \} \ . $$




\begin{lemma}
\label{lem:var_followedby_child} Suppose $M$ is in $\beta$-normal
form. Let $t \travset(M)$. If
$\Pstr{ t = u_1 \cdot (m){m} \cdot u_2 \cdot (n-m,30){n} }$
 where $m \in (N_{\sf var} \union N_{\Sigma}) \setminus (N^{\upharpoonright r}_{var} \union N^{\upharpoonright r}_{\Sigma})$
then $u_2 = \epsilon$.
\end{lemma}
\begin{proof}
By case analysis on the rule used to visit the node
$n$ in $t$. The only relevant rules are (Var), (Answer-var), (InputVar$^{val}$), (InputVar)
and the constant rules.
Since the term is in $\beta$-normal form, there is no @-node in $\tau(M)$ and therefore (Var) cannot be used.
Since $m$ is not hereditarily justified by the root, it is not an input-variable and therefore the rules
(InputVar$^{val}$) and (InputVar) cannot be used.
For the rule (Answer-var) the result follows from the well-bracketedness of traversals.
For constant rules, the result follows from the well-behaviour of constant rules (condition WB).
\end{proof}

\begin{lemma}[View of a traversal reduction]
\label{lem:redtrav_trav} Suppose that $M$ is a $\beta$-normal term and let $t$ be a traversal of $\tau(M)$ then
\begin{itemize}
\item[(i)] $ \pview{t \upharpoonright  r } = \pview{t} \upharpoonright r$\ ;
\item[(ii)] if $t^\omega \in N^{\filter r}$ {\it i.e.}~$t$'s last node is hereditarily justified by $r$, then
    $\oview{t \upharpoonright r } = \oview{t}$\ .
\end{itemize}
\end{lemma}
In the safe lambda calculus without interpreted constants this lemma
follows immediately from the fact that $\travset(M) =
\travset(M)^{\upharpoonright r }$. Here we prove the result in a
more general setting of a calculus extended with interpreted
constants whose corresponding traversal rules are
\emph{well-behaved}.


\begin{proof}
(i) By induction. It is trivially true for the empty
traversal and for the traversal $t = r$. Step case: consider a traversal $t$ and
suppose that the property (i) is verified for all traversal shorter
than $t$:
\begin{itemize}[-]
\item If $t = t' \cdot n$ with $n \in N_{\sf var} \union N_{\Sigma}$ then:
    \begin{align*}
    \pview{t} \upharpoonright  r
&= \pview{t' \cdot n} \upharpoonright  r & (\mbox{definition of } t)\\
        &= (\pview{t'} \cdot n) \upharpoonright  r  & (\mbox{P-view computation}) \\
        &= \pview{t'} \upharpoonright  r  \cdot (n \upharpoonright  r)            & (\mbox{def. of filtering $\upharpoonright$}) \\
        &= \pview{t' \upharpoonright  r } \cdot (n \upharpoonright  r)           & (\mbox{induction hypothesis}) \\
        &= \pview{t' \upharpoonright  r \cdot (n \upharpoonright  r) } & (\mbox{P-view computation, $n \in N_{\sf var} \union N_{\Sigma}$}) \\
        &= \pview{(t' \cdot n ) \upharpoonright  r  }           & (\mbox{def. of filtering $\upharpoonright$}) \\
        &= \pview{t \upharpoonright  r  }
 & (\mbox{definition of } t).
    \end{align*}


\item If $\Pstr{ t =  t' \cdot (m){m} \cdot  u \cdot (lmd-m,30){n}}$ with $n\in N_\lambda \setminus N^{\upharpoonright r}_\lambda$ then we have $u = \epsilon$ by lemma
    \ref{lem:var_followedby_child} and:
        \begin{align*}
        \pview{t} \upharpoonright  r
        &= \pview{\Pstr{t' \cdot (m){m} \cdot (n-m,60){n}}} \upharpoonright  r
                                                        & (u=\epsilon)\\
        &= (\Pstr{\pview{t'} \cdot (m){m} \cdot (lmd-m,60){n}} ) \upharpoonright  r
                                                        & (\mbox{P-view computation}) \\
        &= \pview{t'} \upharpoonright  r                & (m, n \not\in N^{\upharpoonright r}) \\
        &= \pview{t' \upharpoonright  r }               & \mbox{(induction hypothesis)} \\
        &= \pview{ (\Pstr{t' \cdot (m){m} \cdot (lmd-m,40){n}}) \upharpoonright r }
                                                        & (m, n \not\in N^{\upharpoonright r}) \\
        &= \pview{ t \upharpoonright r }             & \mbox{(def. of $t$ \& $u = \epsilon$).}
        \end{align*}

\item If $\Pstr{ t =  t' \cdot (m){m} \cdot u \cdot (lmd-m,30){n} }$ with $n\in N^{\upharpoonright r}_\lambda$ then:
        \begin{align*}
        \pview{t} \upharpoonright  r
        &= \pview{\Pstr{t' \cdot (m){m} \cdot u \cdot (n-m,40){n}}} \upharpoonright  r
                                                              & (\mbox{definition of } t)\\
        &= (\Pstr{\pview{t'} \cdot (m){m} \cdot  (lmd-m,60){n}}) \upharpoonright  r
                                                              & (\mbox{P-view computation}) \\
        &= \Pstr{ \pview{t'} \upharpoonright  r \cdot (m){m} \cdot  (lmd-m,60){n} }
                                                              & (m, n \in N^{\upharpoonright r}) \\
        &= \Pstr{ \pview{t'\upharpoonright r}  \cdot (m){m} \cdot  (lmd-m,60){n} }
                                                              & \mbox{(induction hypothesis)} \\
        &= \pview{ \Pstr{t' \upharpoonright r \cdot (m){m} \cdot {(u \upharpoonright r)} \cdot (lmd-m,35){n}}}
                                                           & (\mbox{P-view computation}) \\
        &= \pview{ (\Pstr{t' \cdot (m){m} \cdot u \cdot (lmd-m,35){n}}) \upharpoonright r }
                                                           & (m, n \in N^{\upharpoonright r}) \\
        &= \pview{ t \upharpoonright r }                & \mbox{(def. of $t$).}
        \end{align*}
\end{itemize}
(ii) By a straightforward induction similar to (i).
\end{proof}

\begin{remark}
\label{rem:inputvar}
Using the previous lemma we observe that in the definition of the rule \rulenamet{InputVar} we have
$n \in N_\lambda^{\filter r}$. Indeed,
$\oview{t_1 \cdot x } = \oview{ (t_1 \cdot x) \filter r}$ therefore $n$
is hereditarily enabled by $r$.
\end{remark}

\begin{lemma}[Traversal of $\beta$-normal terms]
\label{lem:betaeta_trav}
Let $M$ be a $\beta$-normal term, $r$ be the root of the tree $\tau(M)$ and
$t$ be a traversal of $\tau(M)$.
For any node $n$ occurring in $t$:
\begin{eqnarray*}
r \mbox{ does not hereditarily justify } n  \  \iff \   n \mbox{ is
hereditarily justified by some node in } N_\Sigma.
\end{eqnarray*}
\end{lemma}
\begin{proof}
 In a computation tree, the only nodes that do not have justification pointer are:
the root $r$, @-nodes and $\Sigma$-constant nodes. But since $M$ is
in $\beta$-normal form, there is no @-node in the computation tree.
Hence nodes are either hereditarily justified by $r$ or hereditarily
justified by a node in $N_\Sigma$. Moreover $r$ is not in $N_\Sigma$
therefore the ``or'' is exclusive : a node cannot be hereditarily
justified at the same time by $r$ and by some node in $N_\Sigma$.
\end{proof}


\section{Game semantics correspondence}
\label{sec:gamesemcorresp}

 We are working in the general setting of an applied
simply-typed $\lambda$-calculus with a given set of higher-order
constants $\Sigma$. The operational semantics of these constants is
given by certain reduction rules. We assume that a fully abstract
model of the calculus is provided by means of a category of
well-bracketed games. For instance, if $\Sigma$ consists
of the \pcf\ constants then we consider the traditional
category of games and innocent well-bracketed strategies
\cite{hylandong_pcf,abramsky94full}.


In the literature, a strategy is commonly defined as a set of plays closed by
even-length prefixing. However, for our purpose here, it is more convenient to represent strategies using \emph{prefix-closed} set of plays. This saves us from considerations on the parity of traversal length when
showing the correspondence between traversals and game semantics.
 For the rest of the section we fix a simply-typed term $\Gamma \vdash M :T$. We write $\sem{\Gamma \vdash M : T}$ for its strategy denotation (in the standard cartesian closed category of games and innocent strategies \cite{abramsky94full, hylandong_pcf}). We use the notation $\prefset(S)$ to denote the prefix-closure of the set $S$.

\subsection{Relating computation trees and games}
Let us first study an example:
\subsubsection{Example}
Consider the following term $M \equiv \lambda f z . (\lambda g x . f (f x)) (\lambda y. y) z$ of type $(o \typear o) \typear o \typear o$.
Its $\eta$-long normal form is $\lambda f z . (\lambda g x . f (f x)) (\lambda y. y) (\lambda .z)$.
The computation tree is:

$$
\tree{\lambda f z}
{ \tree{@}
    {
        \tree{\lambda g x}
            { \tree{f}{   \tree{\lambda}{ \tree{f}{  \tree{\lambda}{\TR{x}}} }  }
            }
        \tree{\lambda y}{\TR{y}}
        \tree{\lambda}{\TR{z}}
    }
}
$$

The arena for the type $(o \typear o) \typear o \typear o$ is:
$$\tree{q^1}
{
    \tree{q^3}
        {  \tree{q^4}
                {\TR{a^4_1} \TR{\ldots}}
            \TR{a^3_1} \TR{\ldots} }
    \tree{q^2}
    { \TR{a^2_1} \TR{a^2_2}\TR{\ldots} }
    \TR{a_1} \TR{a_2}\TR{\ldots}
}
$$

\newlength{\yNull}
\def\bow{\quad\psarc{->}(0,\yNull){1.5ex}{90}{270}}

The figure below represents the computation tree (left) and the
arena (right). The dashed line defines a partial function $\psi$
from the set of nodes in the computation tree to the set of moves.
For simplicity, we now omit answers moves when representing arenas.
$$
\tree{ \Rnode{root} {\lambda f z}^{[1]} }
     {  \tree{@^{[2]}}
        {   \tree{\lambda g x ^{[3]}}
                { \tree{\Rnode{f}{f^{[6]}}}{  \tree{\Rnode{lmd}\lambda^{[7]}}{ \tree{\Rnode{f2}{f^{[8]}}} {\tree{\Rnode{lmd2}\lambda^{[9]}}{\TR{x^{[10]}}}}}  }
                }
            \tree{\lambda y ^{[4]}}{\TR{y}}
            \tree{\lambda ^{[5]}}{\TR{\Rnode{z}z}}
        }
    }
\hspace{3cm}
  \tree[levelsep=12ex]{ \Rnode{q1}q^1 }
    {   \pstree[levelsep=4ex]{\TR{\Rnode{q3}q^3}}{\TR{\Rnode{q4}q^4}}
        \TR{\Rnode{q2}q^2}
        \TR{\Rnode{q5}q^5}
    }
\psset{nodesep=1pt,arrows=->,arcangle=-20,arrowsize=2pt 1,linestyle=dashed,linewidth=0.3pt}
\ncline{->}{root}{q1} \aput*{:U}{\varphi}
\ncarc{->}{z}{q2}
\ncline{->}{f}{q3}
\ncline{->}{lmd}{q4}
\ncline{->}{f2}{q3}
\ncline{->}{lmd2}{q4}
$$

Consider the justified sequence of moves $s \in \sem{M}$:
 $$s = \Pstr[0.6cm][5pt]{(q1){q}^1\ (q3-q1,60){q}^3\ (q4-q3,60){q}^4\ (q3b-q1){q}^3\ (q4b-q3b,60){q}^4\ (q2-q1,30){q}^2 }
\in \sem{M}$$

There is a corresponding justified sequence of nodes in the computation tree:
$$r = \Pstr[0.8cm]{
        (q1){\lambda f z} \cdot
        (q3-q1,60){f}^{[6]} \cdot
        (q4-q3,60){\lambda^{[7]}} \cdot
        (q3b-q1,60){f}^{[8]} \cdot
        (q4b-q3b,50){\lambda^{[9]}} \cdot
        (q2-q1,50){z} }$$
such that $s_i = \psi(r_i)$ for all $i < |s|$.

The sequence $r$ is in fact the reduction of the following
traversal:
$$t = \Pstr[1.1cm]{ (q1){\lambda f z} \cdot
            (n2){@^{[2]}} \cdot (n3-n2,60){\lambda g x^{[3]}} \cdot
            (q3-q1,60){f}^{[6]} \cdot (q4-q3,60){\lambda^{[7]}} \cdot
            (q3b-q1,40){f}^{[8]} \cdot (q4b-q3b,70){\lambda^{[9]}} \cdot
            (n8-n3,35){x^{[10]}} \cdot
            (n9-n2,30){\lambda^{[5]}} \cdot
            (q2-q1,35){z} }
$$

By representing side-by-side the computation tree and the type arena of a term in $\eta$-normal form we have observed
that some nodes of the computation tree can be mapped to question moves of the arena.
In the next section, we show how to define this mapping in a systematic manner.

\subsubsection{Formal definition}

We now establish formally the relationship between games and computation trees. Suppose $\Gamma \vdash M : T$
is in $\eta$-long normal form. We suppose that computation tree $\tau(M)$
is given by a pair $(V,E)$ where $V$ is the set of vertices of
and $E \subseteq V \times V$ is the parent-child relation. We have $V = N \union VL$ where $N$
and $VL$ are the set of nodes and value-leaves respectively.

\emph{Notations:}
We write $V_\$$ for $N_\$ \union (E(N_\$) \inter VL)$ where $\$$ ranges over $\{@, {\sf var}, \Sigma, {\sf fv} \}$.
Let $\mathcal{D}$ be the set of values of the base type $o$. If $n$ is a node in $N$ then the value-leaves attached to the node $n$ are written $v_n$ where $v$ ranges in $\mathcal{D}$.
Similarly, if $q$ is a question in $\sem{A}$ then the answer moves enabled by $q$ are written $v_q$ where $v$ ranges in $\mathcal{D}$.

\begin{definition}[Mapping from nodes to moves]\hfill
\label{def:phi_psi mapping}

    \begin{itemize}[-]
    \item Let $n$ be a node in $N_\lambda \union N_{\sf var}$ and $q$ be a question move of some game $A$
such that $n$ and $q$ are of type $(A_1,\ldots,A_p,o)$ for some $p\geq 0$. The function $\psi^{n,q}_A$ from $V^{\upharpoonright n}$ to $\sem{A}$ is defined as:
        \begin{eqnarray*}
        \psi^{n,q}_A &=& \{ n \mapsto q \} \union  \{ v_n \mapsto v_q \ | \ v \in \mathcal{D} \}\\
         &&\union \left\{
                        \begin{array}{ll}
                          \emptyset, & \hbox{if $p=0$\ ;} \\
                          \Union_{m \in N | n \vdash_i m} \psi^{m, q^i}_A, & \hbox{if $p\geq1$ and $n\in N_{\lambda}$\ ;} \\
                          \Union_{i=1..p} \psi^{n.i, q^i}_A, & \hbox{if $p\geq1$ and $n\in N_{\sf var}$\ .}
                        \end{array}
                      \right.
        \end{eqnarray*}
        where $\{ q^1, \ldots, q^p \} \union \{ v_q \ | \ v \in \mathcal{D} \}$ is the set of moves enabled by $q$ in $A$ (each $q^i$ being of type $A_i$).

    \item We use the abbreviation $\psi_n$
    for $\psi^{n,m}_{T(n)} : V^{\upharpoonright n} \rightarrow \sem{T(n)}$
    where $m$ denotes $\sem{T(n)}$'s initial move.

    \item Similarly we write $\psi_M$ (or just $\psi$ if this does not cause any ambiguity)
    for $\psi^{r,m}_{\Gamma\rightarrow T}$ where $m$ denote $\sem{\Gamma\rightarrow T}$'s initial move.\footnote{Arenas involved in the game semantics of simply-typed $\lambda$-calculus are all trees: they have a single initial move.}
    \end{itemize}
\end{definition}

It can easily be checked that the domain of definition of $\psi_n$ is indeed the set of nodes that are hereditarily enabled by $n$.

Let us detail a little the definition of $\psi_n$:
\begin{itemize}
\item If $p=0$ then $n$ is a dummy $\lambda$-node or a ground type variable: $\psi_n$ maps $n$ to the initial move $q$.

\item  If $p\geq 1$ and $n \in N_{\lambda}$ with $n$ labelled $\lambda \overline{\xi} = \lambda \xi_1 \ldots \xi_p$ then the sub-computation tree rooted at $n$ and the arena $\sem{T(n)}$ have the following forms (value-leaves and answer moves are not represented for simplicity):
    $$ \tree{ \Rnode{r}\lambda \overline{\xi}  ^{[n]}}
        {
            \tree[levelsep=6ex]{\alpha}
            {   \TR{\ldots} \TR{\ldots} \TR{\ldots}
            }
        }
    \hspace{3cm}
    \tree{ \Rnode{q0}m_n }
        {
            \tree[linestyle=dotted]{q^1}{\TR{} \TR{} }
            \tree[linestyle=dotted]{q^2}{\TR{} \TR{} }
            \TR{\ldots}
            \tree[linestyle=dotted]{q^p}{\TR{} \TR{} }
        }
    \psset{nodesep=1pt,arrows=->,arcangle=-20,arrowsize=2pt 1,linestyle=dashed,linewidth=0.3pt}
    \ncline{->}{r}{q0}
    \ncarc{->}{q2}{z}
    \ncline{->}{q3}{f}
    \ncline{->}{q4}{lmd}
    \ncline{->}{q3}{f2}
    \ncline{->}{q4}{lmd2}
    $$

    For each abstracted variable $\xi_i$ there exists a corresponding question move $q^i$ of the same order in the arena. $\psi_n$ maps each free occurrence of $\xi_i$ in the computation tree to the move $q^i$.

\item If $p\geq 1$ and $n\in N_{\sf var}$ then $n$ is labelled with a variable $x:(A_1,\ldots,A_p,o)$
with children nodes $\lambda \overline{\eta}_1$, \ldots, $\lambda \overline{\eta}_p$. The computation tree $\tau(M)$ rooted at $n$ and the arena $\sem{T(n)}$ have the following forms:
    $$\tree{\Rnode{r}{x^{[n]}}}
        {   \tree{\TR{\lambda \overline{\eta}_1}}{\vdots} \TR{\ldots}
        \tree{\TR{\lambda \overline{\eta}_p }}{\vdots}
        }
    \hspace{3cm}
    \tree{ \Rnode{q0}m_n }
        {
\tree[linestyle=dotted]{\Rnode{q1}{q^1}}{\TR{} \TR{} }
            \tree[linestyle=dotted]{\Rnode{q2}{q^2}}{\TR{} \TR{} }
            \TR{\ldots}
            \tree[linestyle=dotted]{\Rnode{qp}{q^p}}{\TR{} \TR{} }
        }
    \psset{nodesep=1pt,arrows=->,arcangle=-20,arrowsize=2pt 1,linestyle=dashed,linewidth=0.3pt}
    \ncline{->}{r}{q0}
    \ncarc{->}{q2}{z}
    \ncline{->}{q3}{f}
    \ncline{->}{q4}{lmd}
    \ncline{->}{q3}{f2}
    \ncline{->}{q4}{lmd2}
    $$

    and $\psi_n$ maps each node $\lambda \overline{\eta}_i$ to the question move $q^i$.
\end{itemize}

\begin{example}
Take $M = \lambda x . (\lambda g . g x) (\lambda y . y)$ with $x,y:o$
and $g:(o,o)$. The diagram below represents the computation tree
(middle), the arenas $\sem{(o,o), o}$ (left), $\sem{o , o}$ (right),
$\sem{o\rightarrow o}$ (rightmost), $\psi_{\lambda x}$,
$\psi_{\lambda g}$ and $\psi_{\lambda y}$ (dashed-lines).
$$\psset{levelsep=3.5ex}
\pstree{\TR[name=root]{\lambda x}}
{
    \pstree{\TR[name=App]{@}}
    {
            \pstree{\TR[name=lg]{\lambda g}}
                { \pstree{\TR[name=lgg]{g}}{
                        \pstree{\TR[name=lgg1]{\lambda}}
                        { \TR[name=lgg1x]{x}  } } }
            \pstree{\TR[name=ly]{\lambda y}}
                    {\TR[name=lyy]{y}}
    }
}
\rput(4.5cm,-1cm){
  \pstree{\TR[name=A1lx]{q_{\lambda x}}}
        { \TR[name=A1x]{q_x} }
}
\rput(-6cm,-1.5cm){
    \pstree{\TR[name=A2lg]{q_{\lambda g}}}
    {
        \pstree{\TR[name=A2g]{q_g}}
        {  \TR[name=A2g1]{q_{g_1}}   }
    }}
\rput(2.5cm,-1.5cm){
    \pstree{\TR[name=A3ly]{q_{\lambda y}}}
        { \TR[name=A3y]{q_y}
        }
}
\psset{nodesep=1pt,arrows=->,arcangle=-20,arrowsize=2pt 1,linestyle=dashed,linewidth=0.3pt}
\ncline{->}{root}{A1lx} \mput*{\psi_{\lambda x}}
\ncarc{->}{lgg1x}{A1x}
\ncline{->}{lg}{A2lg} \mput*{\psi_{\lambda g}}
\ncline{->}{lgg}{A2g}
\ncline{->}{lgg1}{A2g1}
\ncline{->}{ly}{A3ly} \mput*{\psi_{\lambda y}}
\ncline{->}{lyy}{A3y}
$$
\end{example}

\begin{property} \
\label{proper:psi_properties}
\begin{enumerate}[(i)]
\item $\psi$ maps $\lambda$-nodes to O-questions, variable nodes to
P-questions, value-leaves of $\lambda$-nodes to P-answers and
value-leaves of variable nodes to O-answers;
\item $\psi$ maps a node of a given order to a move of the same order;
\item Let $s \in \travset(M)^{\filter r}$. The P-view (resp. O-view) of $\psi(s)$ and $s$ are computed
identically {\it i.e.}~the set of occurrence positions that must be removed
from each sequences in order to obtain their respective P-view (resp. O-view) is the same for both sequence.
\end{enumerate}
\end{property}
\begin{proof}
(i) and (ii) are direct consequences of the definition.

(iii) Because of (i) and since $t$ and $\psi(t)$ have the
same pointers, the computations of the P-view (resp. O-view) of the
sequence of moves and the P-view (resp. O-view) of the sequence of
nodes follow the same steps.
\end{proof}
The fact that we have defined the order of the root node differently from the order of other $\lambda$-nodes
(Def. \ref{def:nodeorder}) should now make more sense to the reader: this definition permits us to state property (ii).
\smallskip

By extension, we can define the function $\psi_M$ on $\travset(M)^{\filter r}$, the set of justified
sequences of nodes that are hereditarily justified by (the only occurrence of) the root $r$:
\begin{definition}[Mapping sequences of nodes to sequences of moves]
We define the function $\psi_M : \travset(M)^{\filter r} \rightarrow \sem{\Gamma \rightarrow T}$ as follows.
If $s = s_0 s_1 \ldots \in \travset(M)^{\filter r}$ then:
$$\psi_M(s) = \psi_M(s_0)\ \psi_M(s_1)\  \psi_M(s_2) \ldots$$
where $\psi_M(s)$ is equipped with $s$'s pointers.

Thus the pointer-free version of this function is a monoid homomorphism.
\end{definition}


\subsection{Interaction games}
\label{sec:interaction_semantics}

In game semantics, strategy composition is achieved by performing a
CSP-like ``composition + hiding''. It is possible to define an
alternative semantics where the internal moves are not hidden when
performing composition. This semantics is named \emph{revealed
semantics} in \cite{willgreenlandthesis} and \emph{interaction}
semantics in \cite{DBLP:conf/sas/DimovskiGL05}.

In addition to the moves of the standard semantics, the interaction
semantics contains certain internal moves of the computation.
Consequently, the interaction semantics depends on the syntactical
structure of the term and therefore cannot lead to a full
abstraction result. However this semantics will prove to be useful
to identify a correspondence between the game semantics of a term
and the traversals of its computation tree.

Our interaction semantics will be calculated from the $\eta$-normal
form of a term. However we do not want to keep all the internal
moves: we only keep the internal moves that are produced when
composing two subterms of the computation tree joint by an @-node.
This means that when computing the strategy denoting $y N_1 \ldots
N_p$ where $y$ is a variable, we preserve the internal moves of
$N_1$, \ldots, $N_p$ while omitting the internal moves produced by
the copy-cat projection strategy denoting $y$.


\begin{definition} \hfill
\begin{itemize}
\item We call \defname{interaction type tree} or just \defname{interaction type},
a tree whose leaves are labelled with linear simple types and
nodes are labelled with symbol in $\{ ;, \langle \_\ ,\_
\rangle, \otimes, \dagger, \Lambda \}$.


Nodes labelled $;$, $\langle \_\ ,\_ \rangle$ or $\otimes$ are
binary nodes and nodes labelled $\dagger$ or $\Lambda$ are unary
nodes. If $T_1$ and $T_2$ are interaction types we write
$\langle T_1, T_2 \rangle$ to denote the interaction type
obtained by attaching $T_1$ and $T_2$ to a $\langle \_\ ,\_
\rangle$-node. Similarly we use the notations $T_1 \otimes T_2$,
$T_1 ; T_2$, $\Lambda(T_1)$ and $T_1^\dagger$.

\item To every node or leaf we can associate a linear type. We write
    $type(T)$ to denote the type associated to the root node. We
    sometime write the type in exponent {\it e.g.}
    $T^{A\rightarrow B}$ if $type(T) =A\rightarrow B$. This type
    is determined by the structure of the tree as follows:
    \begin{itemize}
    \item If $T$ is a leaf then $type(T)$ is define as the type that labels the leaf;

    \item $type\ (T^{!A \multimap B})^\dagger = !A \multimap !B$;

    \item $type\ \Lambda(T_1^{A \otimes B \multimap C}) = A \multimap (B \multimap C)$

    \item $type\ \langle T_1^{C \multimap A} , T_2^{C \multimap B} \rangle =
    C \multimap A \times B$;

    \item $type\ T_1^{A \multimap B} \otimes T_2^{C \multimap D} = (A \otimes C) \multimap (B \otimes D)$;

    \item $type\ T_1^{A \multimap B};T_2^{B \multimap C} = A \multimap C$.
    \end{itemize}

\end{itemize}

For the interaction type tree to be well-defined, it is required
that types of children nodes are consistent with the meaning of the
parent node; for instance the two children nodes of a ;-node must be
of type $A\multimap B$ and $B\multimap C$.

\end{definition}


Let $T$ be an interaction type tree. Each leaf or node of type $A$
in $T$ can be mapped to the (standard) game $\sem{A}$. By taking the
image of $T$ across this mapping we obtain a tree whose leaves and
nodes are labelled by games. This tree, written $\intersem{T}$, is
called an \defname{interaction game}.

A \defname{revealed strategy} $\Sigma$ on the interaction game $\intersem{T}$ is a compositions of several standard strategies in which certain internal moves are not hidden. Formally:
\begin{definition}[Revealed strategy]
A revealed strategy $\Sigma$ on an interaction game $\intersem{T}$,
written $\Sigma: \intersem{T}$, is an annotated interaction type
tree $T$ where
\begin{itemize}
\item each leaf $\sem{A}$ of $T$ is annotated with a (standard) strategy $\sigma$ on the game
$\sem{A}$;
\item each $;$-node is annotated with a set of indices $U \subseteq \nat$.
\end{itemize}
\end{definition}

The intuition behind this definition is that each $;$-node with children of type $A\multimap B$ and $B\multimap C$ is annotated with a set of indices $U$ indicating which components of $B$ should be uncovered when performing composition.
More precisely, if $B = B_0 \times \ldots \times B_l$ then the revealed strategy built by connecting two revealed strategies $\Sigma_1 : \intersem{A\multimap B}$ and $\Sigma_2 : \intersem{B\multimap C}$
using a $;$-node annotated with $U$ represents the
set of uncovered plays obtained
by performing the usual composition while ignoring and copying the internal moves already in $\Sigma_1$ and $\Sigma_2$ and preserving any internal
move produced by the composition in some component $B_k$ for $k \in U$.

\begin{example}
The diagrams below represent an interaction type tree $T$ (left),
the corresponding interaction game $\intersem{T}$ (middle) and a
revealed strategy $\Sigma$ (right):
$$
\pstree[levelsep=6ex]{\TR{;}}
        {
            \pstree[levelsep=6ex]{\TR{;}}
            { \TR{A\multimap B}
              \TR{B\multimap C}
            }
            \TR{C\multimap D}
        }
\hspace{1cm}
\pstree[levelsep=6ex]{\TR{;}}
        {
            \pstree[levelsep=6ex]{\TR{;}}
            { \TR{\sem{A\multimap B}}
              \TR{\sem{B\multimap C}}
            }
            \TR{\sem{C\multimap D}}
        }
\hspace{1cm}
\pstree[levelsep=6ex]{\TR{;^{\{0\}}}}
        {
            \pstree[levelsep=6ex]{\TR{;^{\{0\}}}}
            { \TR{A\multimap B^{\sigma_1}}
              \TR{B\multimap C^{\sigma_2}}
            }
            \TR{C\multimap D^{\sigma_3}}
        }
$$
\end{example}
A revealed strategy can also be written as an expression, for
instance the strategy represented above is given by the expression
$\Sigma = (\sigma_1 ;^{\{0\}} \sigma_2) ;^{\{0\}} \sigma_3$. We will
use the abbreviation $\Sigma_1 \fatsemi^U \Sigma_2$ for
$\Sigma_1^\dagger ; ^U \Sigma_2$.


\subsubsection{Uncovered play}

The analogous of a play in the interaction semantics is called an
\emph{uncovered play}, it is a play containing internal moves. The
moves are implicitly tagged so that it is possible to retrieve in
which component of the arena of which node/leaf-game the move
belongs to. A given move may belong to several games from different
nodes/leaves of the interaction game.

\begin{definition}
The \defname{set of possible moves} $M_T$ of an interaction game
$\intersem{T}$ is defined as $\mathcal{M}_T/\hspace{-0.5em}\sim_T$,
the quotient of the set $\mathcal{M}_T$ by the equivalence relation
$\sim_T \subseteq \mathcal{M}_T \times \mathcal{M}_T$ defined as follows:
For a single leaf tree $T$ labelled by a type $A$ we define
$\mathcal{M}_T = M_A$ and $\sim_T = id_{M_A}$. For other cases:
    \begin{align*}
        \mathcal{M}_{T^\dagger} &= \mathcal{M}_{T} + M_{type(T^\dagger)}
    &
        \mathcal{M}_{\Lambda(T)} &= \mathcal{M}_{T} + M_{type(\Lambda(T))}
    \\
        \sim_{T^\dagger} &= \left( \sim_{T}
        \union \left(type\ T^\dagger \leftrightarrow  type\ T\right)
        \right)^\star
    &
        \sim_{\Lambda(T)} &= \left( \sim_{T}
        \union \left(type\ \Lambda(T) \leftrightarrow type\ T\right)
        \right)^\star
    \end{align*}
    \begin{align*}
        \mathcal{M}_{\langle T_1^{C^1 \multimap A^1}, T_2^{C^2 \multimap B^2}\rangle}
        &= \mathcal{M}_{T_1} + \mathcal{M}_{T_2} + M_{C \multimap (A \otimes B)}
    \\
         \sim_{\langle T_1^{C^1 \multimap A^1}, T_2^{C^2 \multimap B^2}\rangle} &= \left( \sim_{T_1}
        \union \sim_{T_2} \union (C^1 \leftrightarrow C) \union (C^2 \leftrightarrow C)
        \union (A^1 \leftrightarrow A) \union (B^2 \leftrightarrow B)
        \right)^\star
    \\
    \\
        \mathcal{M}_{T_1^{A^1 \multimap B^1}\otimes T_2^{C^2 \multimap D^2}} &= \mathcal{M}_{T_1} +  \mathcal{M}_{T_2} + M_{A \otimes C \multimap B \otimes D }
        \\
         \sim_{T_1^{A^1 \multimap B^1}\otimes T_2^{C^2 \multimap D^2}} &= \left( \sim_{T_1}
        \union \sim_{T_2} \union (A^1 \leftrightarrow A)
        \union (B^1 \leftrightarrow B) \union (C^2 \leftrightarrow C)\union (D^2 \leftrightarrow D)
        \right)^\star
    \\
    \\
        \mathcal{M}_{T_1^{A \multimap B};T_2^{B \multimap C}} &=
            \mathcal{M}_{T_1} + \mathcal{M}_{T_2} + M_{A\multimap C}
        \\
         \sim_{T_1^{A^1 \multimap B^1};T_2^{B^2 \multimap C^2}} &= \left( \sim_{T_1}
        \union \sim_{T_2} \union (A^1 \leftrightarrow A)
        \union (B^1 \leftrightarrow B^2) \union (C \leftrightarrow C^2)
        \right)^\star
    \end{align*}
    where $A\leftrightarrow B$ denotes the implicit bijection between
    two isomorphic arenas $\sem{A}$ and $\sem{B}$; $R^\star$
    denotes the smallest superset of the relation $R$ complete
    by transitivity, reflexivity and symmetry.
\end{definition}

We call \defname{internal move} of the game $\intersem{T}$, any move
from $M_T$ which is not $\sim$-equivalent to any move in
$M_{type(T)}$.


A \defname{justified interaction sequence} of moves on the
interaction game $\intersem{T}$ is a sequence of moves from $M_T$
together with pointers. In contrast to the standard notion of
justified sequence, to each move in the sequence can be attached
several pointers. More precisely, if the equivalence class $m$ is
$\{m_1, \ldots, m_l \}$ then $m$ has one pointer for each
non-initial move $m_i$ in the equivalence class.

\begin{definition}[Filtering] We define several filtering operations
over justified interaction sequences. Let $s$ be a justified
sequence of moves on the interaction game $\intersem{T}$.
\begin{itemize}
\item  Let $T'$ be a subtree of $T$. We define the
filtering operator $s\upharpoonright T'$ to be the subsequence
of $s$ consisting of moves $\sim$-equivalent to some move in
$M_{T'}$. This operation causes some move to ``lose'' some of
their attached pointers: a given move $m$ with equivalence class
$\{m_1, \ldots, m_l \}$ may have up to $l$ pointers, but in
$s\upharpoonright T'$, only pointers associated to a $m_i$
belonging to $\mathcal{M}_{T'}$ are preserved.

Note that since $M_T$ is a set of equivalence classes with
respect to $\sim$, the filtering operator $\_ \filter T'$
implicitly performs the ``retagging'' of the moves to the
appropriate components of each game of the interaction game
$\intersem{T'}$.

\item  For any sub-game $A$ of the standard game $\sem{type(T')}$ we
define the filtering operator $s\upharpoonright A$ to be the
subsequence of $s$ consisting of moves from $A$ where at most
one pointer is kept for each move in the sequence: the one
corresponding to the class citizen from $A$.

\item For any initial move $m$ of the game $\sem{type(T)}$ occurring in $s$, $s
\hjfilter m$ is the subsequence of $s$ consisting of moves
that are \emph{hereditarily justified} by that particular occurrence of $m$ in $s \filter type(T)$.
% NOTE: it is important to precise ``in $s \filter type(T)$'' because $s$'justification
% pointers differs depending on the sub-interaction game considered.

%\item For any initial move $m$ of the game $\sem{type(T)}$, $s
%\hefilter m$ is the subsequence of $s$ consisting of moves
%that are \emph{hereditarily enabled} by $m$ in the game $\sem{type(T)}$.
\end{itemize}
By extension, we also define these operations on sets of justified
interaction sequences.
\end{definition}

Allowing moves to have multiple pointers complicates slightly the
presentation here, but this capability is necessary to model
strategy composition. Indeed, in game semantics after composing
strategies, the pointers from some moves may change! (See definition
of $\filter A,C$ in \cite{abramsky:game-semantics-tutorial}.)
However, for all the other operations on strategies that we will
used, the pointers will just be preserved. Formally we define this
property as follows: Let $s$ be an  interaction sequence on a game
$\intersem{T}$, $T'$ a direct subtree $T$ ({\it i.e.}~a subtree of
$T$ whose root is a child of $T$'s root), $A$ be a sub-game of
$\sem{type(T)}$ and $A'$ be a sub-game of $\sem{type(T')}$, then we
define the predicate $A'\stackrel{s}\hookrightarrow A$ as:
\begin{align*}
 A'\stackrel{s}\hookrightarrow A \mbox{ holds iff } &
 \Pstr{s_1\ (n){n'}\ s_2\ (m-n){m'}\ s_3 } = s\filter A'  \\
 & \implies \exists! m,n \in A | m \sim m' \zand n \sim n' \zand \Pstr{s_1\
(n){n}\  s_2\ (m-n){m}\ s_3} = s\filter A
\end{align*}

and we say that $s$'s justification is preserved from $A'$ to $A$
with respect to $\sim$.



\begin{definition}[Legal uncovered positions] We recall
that in the standard game semantics, the set of legal positions
$L_A$ of a game $A$ is the set of justified sequences of moves from
$M_A$ respecting visibility and alternation. We define the set of
\defname{legal uncovered position} $L_T$ of an interaction game $\intersem{T}$ as
follows:
    \begin{itemize}
    \item If $T$ is a leaf annotated by a type $A$ then $L_T =
    L_A$;
    \item If $T$ is a unary node with child node $T'$ then:
    $$L_T = \{ s \in JustSeq(T) \ | \ s \filter type(T) \in L_{type(T)} \zand  s \filter T' \in L_{T'} \} \ ;$$
    \item If $T$ is a binary node with children nodes $T_1$ and $T_2$ then:
    $$L_T = \{ s \in JustSeq(T) \ | \ s \filter type(T) \in L_{type(T)} \zand  s \filter T_1 \in L_{T_1}
    \zand  s \filter T_2 \in L_{T_2} \} \ .$$
    \end{itemize}
    where $JustSeq(T)$ denotes the set of justified interaction sequences on
    $\intersem{T}$.
\end{definition}

Revealed strategies can alternatively be represented as by means
of sets of uncovered positions:
\begin{definition}[Revealed strategies as set of uncovered positions]
\label{dfn:revealedstrat}
The set of uncovered positions of a revealed strategy is defined inductively on the
structure of the annotated interaction type tree underlying the
interaction strategy:
\begin{itemize}[-]
\item Leaf labelled with type $A$ and annotated by the strategy $\sigma$: The set of positions of the revealed strategy is precisely the set of positions of the standard strategy $\sigma$.

\item Tensor product, pairing, promotion, currying:
\begin{eqnarray*}
(\Sigma_1 : \intersem{T_1}) \otimes (\Sigma_2 : \intersem{T_2}) : \intersem{T} &=\{ s \in L_T \ | \  &s \filter T_1 \in \Sigma_1 \zand\ s \filter T_2 \in \Sigma_2 \\
&& \zand\ type(T_1)\stackrel{s}\hookrightarrow type(T) \\
&& \zand\ type(T_2)\stackrel{s}\hookrightarrow type(T) \}
\\ \\
\langle \Sigma_1 : \intersem{T_1}, \Sigma_2 : \intersem{T_2} \rangle : \intersem{T} &= \{ s \in L_T \ | &
   ( (s \filter T_1 \in \Sigma_1 \zand\ s \filter T_2 = \epsilon) \\
&&  \   \zor ( s \filter T_1 = \epsilon \zand s \filter T_2 \in \Sigma_2)) \\
&& \zand\ type(T_1)\stackrel{s}\hookrightarrow type(T) \\
&& \zand\ type(T_2)\stackrel{s}\hookrightarrow type(T) \}
\\ \\
(\Sigma' : \intersem{T'})^\dagger : \intersem{T} &= \{ s \in L_T \ | \ &
\mbox {for all occurrence $m$ in $s$ of an initial  }\\
&& \mbox{ $\sem{type(T)}$-move, $(s \filter m) \filter T' \in \Sigma'$} \\
&& \zand\ type(T')\stackrel{s}\hookrightarrow type(T) \}
\\ \\
\Lambda(\Sigma' : \intersem{T'}) : \intersem{T} &= \{ s \in L_T \ | & s \filter T' \in \Sigma' \ \zand\ type(T')\stackrel{s}\hookrightarrow type(T) \}
\end{eqnarray*}

\item Uncovered composition $(\Sigma_1 : \intersem{T_1})\ ;^U\ (\Sigma_2
:\intersem{T_2})$ defined on the game $\intersem{T}$ where
$type(T) = A \multimap C$, $type(T_1) = A^1 \multimap B_0 \times
\ldots \times B_l$ and $type(T_2) = B_0 \times \ldots \times B_l
\multimap C^2$. We first define
\begin{eqnarray*}
\Sigma_1 \| \Sigma_2 &= \{ u \in L_T  \ | \ & u \upharpoonright T_1 \in \Sigma_1 \mbox{ and } u \upharpoonright T_2 \in \Sigma_2 \\
&& \zand\ C^2\stackrel{u}\hookrightarrow C\ \zand\ (A^1)^-\stackrel{u}\hookrightarrow A^-  \\
&& \zand\ \parbox[t]{8cm}{for any initial $m$ in $A^1$, if $m$ is justified in $u \filter type(T_1)$ by $b\in B_j$,
itself justified by $c \in C^2$ in $u \filter type(T_2)$ then $m$ justified by $c$ in $u \filter type(T)$ \} }
\end{eqnarray*}
where $A^-$ denotes the set of non-initial moves of the game $A$. We can now define composition as:
$$ \Sigma_1 ;^U \Sigma_2 = \{ cover(u,(0..l)\setminus U) \ | \ u \in \Sigma_1 \| \Sigma_2 \}$$
where $cover(u,C) = u \filter \left( M_T \setminus \Union_{j\in
C} B_j \right)$ {\it i.e.}~the subsequence of $u$ obtained by
removing moves in $\Union_{j\in C} B_j$. Hence
$\Sigma_1;^{\{0..l\}} \Sigma_2 = \Sigma_1 \| \Sigma_2$.

In other words $\Sigma_1 ;^U \Sigma_1$ is the set of uncovered
plays obtained by performing the usual composition while
ignoring and copying the internal moves from arenas in
$\intersem{T_1}$ or $\intersem{T_2}$ and preserving any internal
move produced by the composition in some component $B_k$ for $k
\in U$.
\end{itemize}
\end{definition}

\begin{remark} \hfill
\label{rem:interstrat}
\begin{enumerate}[i.]
\item We observe that for all strategy operator
except composition, pointers associated to moves are preserved.
For strategy composition, additional pointers are
``created'' only for initial $A$-moves.
\item It is straightforward to generalize the pairing operator $\langle \Sigma_1, \Sigma_2 \rangle$ to more than two parameters: an interaction strategy $\langle \Sigma_1, \ldots, \Sigma_p \rangle$ for $p\geq2$
is defined on an interaction game whose root node has $p$ children.
\end{enumerate}
\end{remark}

We write $\mathcal{I}$ for the set of all revealed strategies. Note
that $\mathcal{I}$ is not a category since composition is not
associative and there is no identity interaction strategy.


\begin{lemma}[Complete interaction sequence]
\label{lem:inter_complete}
Let $u$ be an interaction sequence of some interaction strategy $\Sigma : \intersem{T}$
and suppose that the standard strategy denoting the leaves of $\Sigma$ are all well-bracketed.

Then for any node/leaf game $A$ of $T$ and interaction sequence $u\in \Sigma$ we have:
\begin{itemize}[i.]
\item $u \filter A$ is well-bracketed;

\item If $u \filter type(T)$ is complete (all question moves answered) then
    $u \filter A$ is complete.
\end{itemize}
\end{lemma}
\begin{proof}
By induction on the structure of the interaction game $\intersem{T}$. The base case is
trivial. We only treat composition, the other cases being trivial: Let $ u \in \Sigma_1 ; ^U \Sigma_2$ for some $U \subseteq \nat$ with
$\Sigma_1 : \intersem{T_1^{A\multimap B}}$ and $\Sigma_2 : \intersem{T_2^{B\multimap C}}$.

i. During composition, pointers attached to answer moves are preserved with respect to $\sim$
thus non-well-bracketing of $u\filter A\multimap C$ implies
either non-well-bracketing of $u\filter A\multimap B$ or $u\filter B\multimap C$.

For ii., suppose $u \filter type(T) = \Pstr{(q)q\ u'\ (a-q)a }$.
By well-bracketing (i.) and since $q$ and $a$ belong to $C$ we must have
$u \filter B\multimap C = \Pstr{(q)q \ldots (a-q)a}$ thus $u \filter B\multimap C$ is complete.
Suppose that $u \filter A\multimap B$ is not complete, then its first move is unanswered,
but since this is a $B$-move, it must also occur unanswered in $u \filter B\multimap C$ which is a contradiction
since we have just prove that $u \filter B\multimap C$ is complete. Thus $u \filter A\multimap B$  is also complete.

The induction hypothesis permits to conclude.
\end{proof}
Consequently if $u\filter type(T)$ is complete then $u$ is maximal {\em i.e.~no move (and in particular no internal move) can be played after $u$}.

\subsubsection{Modeling the $\lambda$-calculus in $\mathcal{I}$}

We would like to use revealed strategies from $\mathcal{I}$ to model terms of
the simply-typed lambda calculus.
Depending on the internal moves that we wish to hide, we obtain different possible interaction strategies for a given term.
The following definition fixes a unique strategy denotation which is computed from the $\eta$-normal form of the term.

\begin{definition}[Revealed denotation of a term]
\label{dfn:interactionstrategy_ofterms}
Let $\pi_i$ denote the $i^{th}$ projection copycat strategy $\pi_i : \sem{X_1 \times \ldots \times X_l} \rightarrow \sem{X_i}$.

The \defname{revealed game denotation} or \emph{revealed strategy} of
$M$ written $\intersem{\Gamma \vdash M : A}$ is defined as
$\sem{\Gamma \vdash M : A}$ if $M$ is in $\beta$-normal form, otherwise
it is defined by structural induction on the \emph{$\eta$-long normal form of $M$}:
\begin{eqnarray*}
\intersem{\Gamma \vdash \lambda \overline{\xi} . M  : A} &=& \Lambda^{|\overline{\xi}|}(\intersem{\Gamma, \overline{\xi} \vdash M : o })  \\
\intersem{\Gamma  \vdash x_i N_1 \ldots N_p :o} &=& \langle \pi_i, \intersem{\Gamma \vdash N_1 : A_1}, \ldots, \intersem{\Gamma \vdash N_p : A_p}  \rangle \fatsemi ^{\{1..p\}} ev^p \\
\intersem{\Gamma \vdash f N_1 \ldots N_p : o} &=& \langle \intersem{\Gamma \vdash N_1 : A_1}, \ldots, \intersem{\Gamma \vdash N_p : A_p} \rangle^\dagger\  \|\ \sem{f} \\
\intersem{\Gamma \vdash N_0 \ldots N_p : o} &=& \langle \intersem{\Gamma \vdash N_0 : A_0}, \ldots, \intersem{\Gamma \vdash N_p : A_p}  \rangle^\dagger\ \|\ ev^p
\end{eqnarray*}
where $\Gamma = x_1 : X_1 \ldots x_l : X_l$, $f : A_0$ is a $\Sigma$-constants, $p\geq 1$, $A_0 =
(A_1,\ldots,A_p,o)$, $ev^p$ denotes the evaluation strategy with
$p$ parameters and $X_i = A_0$ in the second equation.
\end{definition}

Figure \ref{fig:interaction_strategy_denotations} contains tree representations of the interaction games of the revealed strategy $\intersem{\Gamma \vdash M : A}$ for the application cases. These tree tell us all the information that we need about the strategy involved in $\intersem{M}$. For instance the revealed strategy $\Sigma$ is defined on the interaction arena $\intersem{T^{00}}$ whose root is $!A^0 \multimap B^0$; the strategy $ev$ is defined on the interaction arena $\intersem{T^1}$ with a single arena-node $!B^1 \multimap C^1$; thus plays of $ev$ do not contain uncovered moves.


    \begin{figure}[htbp]
        $$
        \tree[levelsep=6ex,thistreesep=3cm]{\TR{\intersem{N_0 N_1 \ldots N_p :o}:T [!A\multimap C]}}
                {   \tree[levelsep=6ex]{\TR{\Sigma^\dagger:T^0[!A^0\multimap !B_0^0\otimes \ldots \otimes !B_p^0 ]}}
                        {
                            \tree[levelsep=6ex,thistreesep=3cm]{\TR{\Sigma:T^{00}[!A^{00}\multimap B_0^{00}\times \ldots \times B_p^{00}]}}
                            {
                                \tree[levelsep=6ex]{\TR{\intersem{N_0}:T^{000}[!A^{000}\multimap B_0]}}{\Tfan[fansize=10ex]}
                                \TR{\ldots}
                                \tree[levelsep=6ex]{\TR{\intersem{N_p}:T^{00p}[!A^{00p}\multimap B_p]}}{\Tfan[fansize=10ex]}
                            }
                        }
                    \TR{ ev:T^1[!B_0^1 \otimes \ldots \otimes !B_p^1 \multimap C] }
                }
       $$
       \begin{center}
       \emph{Tree-representation of the revealed strategy $\intersem{\Gamma \vdash N_0 N_1 \ldots N_p :o}$.}
       \end{center}

        $$
        \tree[levelsep=6ex,thistreesep=3cm]{\TR{\intersem{x_i N_1 \ldots N_p :o}:T [!A\multimap C]}}
                {   \tree[levelsep=6ex]{\TR{\Sigma^\dagger:T^0[!A^0\multimap !B_0^0\otimes \ldots \otimes !B_p^0 ]}}
                        {
                            \tree[levelsep=6ex,thistreesep=3cm]{\TR{\Sigma:T^{00}[!A^{00}\multimap B_0^{00}\times \ldots \times B_p^{00}]}}
                            {
                                \TR{\pi_i:T^{000}[!A^{000}\multimap B_0]}
                                \tree[levelsep=6ex]{\TR{\intersem{N_1}:T^{001}[!A^{001}\multimap B_1]}}{\Tfan[fansize=10ex]}
                                \TR{\ldots}
                                \tree[levelsep=6ex]{\TR{\intersem{N_p}:T^{00p}[!A^{00p}\multimap B_p]}}{\Tfan[fansize=10ex]}
                            }
                        }
                    \TR{ ev:T^1[!B_0^1 \otimes \ldots \otimes !B_p^1 \multimap C] }
                }
        $$
       \begin{center}\emph{Tree-representation of the revealed strategy $\intersem{\overline{x}:\overline{X}\vdash x_i N_1 \ldots N_p :o}$}
       \end{center}
    \bigskip
    {\small
     Node labels are of the form $\Pi : T' [A]$ where $\Pi$ is a strategy, $T'$ is the corresponding interaction game and $A$ is the standard game lying at the root of the interaction game $T$. The games $A$, $B$ and $C$ are defined as follows:
    \begin{eqnarray*}
        A &=& \Gamma = X_1 \times \ldots \times X_n\\
        B &=& \underbrace{((B_1' \times \ldots \times B_p') \rightarrow o')}_{B_0} \times B_1 \times \ldots \times B_p\\
        C &=& o \ .
    \end{eqnarray*}
    Games are annotated with string  $s \in \{ 0..p \}^*$ in the exponent to indicate the path from the root to the corresponding node in the tree (each number in $s$ indicates which direction to take at the corresponding branch point).
   }
        \smallskip
       \caption{Tree-representation of the revealed strategy in the application case.}
      \label{fig:interaction_strategy_denotations}
    \end{figure}


\begin{remark}
When computing an interaction strategy of the form
$\intersem{y_i N_1 \ldots N_p}$ for some variable $y_i$, the
internal moves of $N_1$, \ldots, $N_p$ are preserved however the
internal moves produced by the copy-cat projection strategy denoting
$y_i$ are omitted.
\end{remark}

\begin{example}
Take the term $\lambda x . (\lambda f . f x) (\lambda y . y)$.
%Its computation tree is:
%$$
%\tree{\lambda x} {
%    \pstree[levelsep=4ex]{\TR{@}}
%    {       \pstree[levelsep=4ex]{\TR{\lambda f}}
%                { \tree{f}{  \tree{\lambda}{ \TR{x}  } } }
%            \pstree[levelsep=4ex]{\TR{\lambda y}}
%                    {\TR{y}}
%    } }
%$$
Its revealed strategy is $$\Lambda ( \langle \sem{ x:X \vdash \lambda f . f
x : (o\rightarrow o) \rightarrow o} , \sem{ x:X \vdash \lambda y . y
: o \rightarrow o} \rangle \| ev_2 ) \ .$$
\end{example}


\subsubsection{From interaction semantics to standard semantics and vice-versa}

In the standard semantics, given two strategies $\sigma : A
\rightarrow B$, $\tau : B \rightarrow C$ and a sequence $s \in
\sigma \fatsemi \tau$, it is possible to (uniquely) recover the
internal moves. The uncovered sequence is written ${\bf u}(s,
\sigma, \tau)$. The algorithm to obtain this unique uncovering is
given in part II of \cite{hylandong_pcf}. Therefore given a term
$M$, we can completely uncover the internal moves of a sequence
$s\in\sem{M}$ by performing the uncovering operation recursively at
every @-node of the computation tree.

Conversely, the standard semantics can be recovered from the
interaction semantics by filtering the moves, keeping only those
played in the root arena:
\begin{eqnarray}
 \sem{\Gamma \vdash M : T} = \intersem{\Gamma \vdash M : T} \upharpoonright \sem{\Gamma \rightarrow T} \label{eqn:int_std_gamsem}
\end{eqnarray}

\subsection{The correspondence theorem for the simply-typed $\lambda$-calculus without interpreted constants}
In this section, we establish a connection between the interaction
semantics of a simply-typed term without constants ($\Sigma =
\emptyset$) and the traversals of its computation tree: we show that
the set $\travset(M)$ of traversals of the computation tree is
isomorphic to the set of uncovered plays of the strategy denotation
(this is the counterpart of the ``Path-Traversal Correspondence'' of
\cite{OngLics2006}), and that the set of traversal reductions is
isomorphic to the strategy denotation.

\subsubsection{@-free traversals}

When defining computation trees, it was necessary to introduce
application nodes (labelled @) in order to connect the operator and
the operand of an application. The presence of @-nodes has also
another advantage: it ensures that the lambda-nodes are all at even
level in the computation tree, and thus a traversal respects a certain form of
alternation.

Application nodes are however redundant in the sense that they do
not play any role in the computation of the term. In fact it is
necessary to filter them out if we want to establish the
correspondence with the interaction game semantics.

\begin{definition}[@-free traversal]
\label{dfn:appnode_filter}
Let $t$ be a traversal of $\tau(M)$.
We write $t-@$ for the sequence of nodes-with-pointers obtained by
\begin{itemize}
\item removing from $t$ all @-nodes and value-leaves of some @-node;
\item replacing any link pointing to an @-node by a link pointing to the immediate predecessor of @ in $t$.
\end{itemize}

Suppose $u = t-@$ is a sequence of nodes obtained by applying the
previously defined transformation on the traversal $t$, then $t$ can
be partially recovered from $u$ by reinserting the @-nodes as
follows. For each @-node @ in the computation tree with parent node
denoted by $p$, we perform the following operations:
\begin{enumerate}
\item replace every occurrence of the pattern $p \cdot n$, where $n$ is a $\lambda$-nodes,
by $p \cdot @ \cdot n$;
\item replace any link in $u$ starting from a $\lambda$-node and pointing to $p$ by a link pointing to the inserted @-node;
\item if there is an occurrence in $u$ of a value-leaf $v_p$ pointing to $p$ then insert a value-leaf $v_@$
immediately before $v_p$ and make it point to the node immediately
following $p$ (which is also the $@$-node that we inserted in 1).
\end{enumerate}
We write $u+@$ for this second transformation.
\end{definition}
These transformations are well-defined because in a traversal, an @-node
always occurs in-between two nodes $n_1$ and $n_2$ such that  $n_1$ is the parent node of @
and $n_2$ is the first child node of @ in the computation tree:
$$      \pstree[levelsep=4ex]{\TR{n_1}\treelabel{0} }
        {
            \pstree[levelsep=3ex]{\TR{@}}
            {
                \tree{n_2}{\vdots}
                \TR[edge=\dedge]{}
                \TR[edge=\dedge]{}
            }
        }
$$
\begin{remark}
Justified sequences of nodes of the form $t-@$ for some traversal $t$ are not, strictly speaking, proper justified sequences of nodes since they do not respect alternation (two $\lambda$-nodes may become adjacent after removing a @-node)
and since any $\lambda$-node justified by @ becomes justified by @'s parent which is also a $\lambda$-node. However we will treat them just as justified sequence.
\end{remark}

\begin{lemma} \label{lem:minus_at_plus_at}
$$\forall t \in \travset(M), \quad (t-@)+@ = \left\{
            \begin{array}{ll}
              t, & \hbox{if $t^\omega \neq @$\ ;} \\
              \ip\ t, & \hbox{if $t^\omega = @$\ .}
            \end{array}
          \right.
$$
\end{lemma}
\proof
The result follows immediately from the definition of the operation -@ and +@.
\qed
\smallskip

We introduce the following notation:
$$
\travset(M)^{-@} = \{ t - @ \ | \  t \in \travset(M) \}
$$

\begin{remark}
If $M$ is $\beta$-normal then $\tau(M)$ does not contain any
@-node therefore all nodes are hereditarily justified by $r$ and we
have $\travset(M)^{-@} = \travset(M) = \travset(M)^{\upharpoonright
r }$.
\end{remark}

\paragraph{Mapping @-free traversals to interaction plays}
\hfill

\notetoself{
\begin{definition}[Mapping from nodes to moves]\hfill
    \label{def:theta mapping}
    Let $T$ be the interaction game of the interaction strategy $\intersem{M}$ and
    $M_T$ be the set of equivalence class of moves from $\mathcal{M}$.


    For $n \in N_{\sf prime}$, let $\Gamma(n) \vdash \kappa(n) : T(n)$ denote the subterm of $\elnf{M}$ rooted at $n$.
    We define the disjoint union of games:
    $$\mathcal{G}_M = \sem{\Gamma\rightarrow T} \quad \uplus \quad  \biguplus_{n \in N_{\sf prime} } \sem{T(n)}.$$
    $$\mathcal{G}_M = \sem{\Gamma\rightarrow T} \quad \uplus \quad  \biguplus_{n \in N_{\sf spawn} } \sem{T(n)}.$$

    We define the function $\varphi_M: V_\lambda \union V_{\sf var} \rightarrow M_T$
    as:
    \begin{equation*}
        \varphi_M = \psi_{M} \quad \union \Union_{n \in N_{\sf prime}} \psi_{n}
    \end{equation*}
    where $q_0$ denotes $\sem{\Gamma\rightarrow T}$'s initial move.

    We omit the subscript in $\varphi_M$ if it does not cause any ambiguity.
\end{definition}


$\varphi_M$ is indeed totally defined on $V_\lambda \union V_{\sf var} = V\setminus (V_@ \union V_\Sigma)$ (since a node is either hereditarily justified by the root, by a @-node or by a $\Sigma$-node).

\begin{remark}
\label{rem:phi_preserves_her_enabling}
$\varphi_M$ \defname{preserves hereditary enabling}: a node $n$ is hereditarily
 enabled by some node $n' \in N \inter E \relimg{N_@ \union N_\Sigma}$ in $\tau(M)$ if and only if
 $\varphi(n)$ and $\varphi(n')$  are both played in the same game $A \in \mathcal{G}$ and
the move $\varphi_M(n)$ is hereditarily enabled by $\varphi_M(n')$ in $A$.
\end{remark}

%If $t$ is a justified sequence of nodes in $V_\lambda \union V_{\sf var}$ then $?(\varphi(t)) =
%\varphi(?(t))$.
%where $?(\varphi(t))$ denotes the subsequence of $\varphi(t)$ consisting of the unanswered questions
%and $?(t)$ denotes the subsequence of $t$ consisting of the unmatched nodes (see the
%definition in section \ref{sec:adding_value_leaves}).

}

As we observed in a previous remark, sequences from $\travset(M)^{-@}$ are not, strictly speaking, proper justified sequences. Consequently the filtering operators introduced up to now are undefined on $\travset(M)^{-@}$. We now introduce a new filtering operation on $\travset(M)^{-@}$:
\begin{definition}
Let $\Delta \vdash \kappa(n) : A$ be some subterm of $\elnf{M}$ for some $n\in N_\lambda$.
We define the \defname{subterm filtering} operator on sequences of the form $t-@$ for some traversal $t$ of $M$ as follows:
$$ (t - @) \subtermfilter \kappa(n) = (t-@)\hefilter n = t\hefilter n \ . $$
\end{definition}
Note that this is well-defined because $t-@ = t'-@$ implies $t\hefilter n = t'\hefilter n$ (since @-nodes have no justifier).
In particular we have:
$$ (t - @) \subtermfilter M = t \hefilter r  = t \hjfilter r \ .$$
(Here hereditary justification and hereditarily enabling coincide because the root node can appear at most once in a traversal.)

\begin{lemma}[Filtering lemma]
\label{lem:varphi_filter}
Let $t$ be a traversal of $M$, $\Delta \vdash N : A$ be some subterm of $\elnf{M}$ and $m$ be an occurrence of an initial $A$-move in $\varphi(t-@)$ then:
$$(i) \quad \varphi_M((t-@)\subtermfilter N) = \varphi_M(t-@) \filter \sem{\Delta\rightarrow A} \ .$$
$$(ii) \quad \varphi_M((t-@)\subtermfilter N) \hjfilter m = \varphi_M(t\hjfilter n) \ .$$
where $n$ denotes the occurrence of $\tau(N)$'s root in $t$ whose image
by $\varphi_M$ is the occurrence $m$.

Consequently:
$$(iii) \quad  \varphi_M(\travset^{-@}(M)) \filter \sem{\Gamma \rightarrow T} = \psi_M(\travset^{\filter r}(M))\ .$$
\end{lemma}
\proof Let $t$ be a traversal of $M$:
$$\begin{array}{lrclr}
\mbox{i.}& \varphi( (t-@) \subtermfilter N ) &=& \varphi_M((t-@) \hefilter n ) & \mbox{(Def. subterm filtering)}\\
          &&=& \varphi_M(t-@) \filter \sem{\Delta \rightarrow A}  & \parbox[t]{5.5cm}{(By remark \ref{rem:phi_preserves_her_enabling}, $\varphi_M$ preserves hereditary enabling,  and  moves in $\sem{\Delta \rightarrow T}$ are all hereditarily enabled by the initial move $m = \varphi_M(n)$).} \\
\\
\mbox{ii.}& \varphi_M((t-@)\subtermfilter N) \hjfilter m
  &=& \varphi_M(t\hefilter n) \hjfilter m & \mbox{(Def. subterm filtering)}\\
  &&=& ( \varphi_M(t) \hefilter \varphi_M(n) ) \hjfilter m & \parbox[t]{5.5cm}{($\varphi_M$ maps the set of nodes hered. \emph{enabled} by $n$ to the set of moves hered. \emph{enabled} by $\varphi_M(n)$)} \\
  &&=& \varphi_M(t) \hefilter  m \hjfilter m & \mbox{($m = \varphi_M(n)$)} \\
  &&=& \varphi_M(t) \hjfilter m \ . \\
  \\
\mbox{iii.} & \varphi(t-@) \filter \sem{\Gamma \rightarrow T}
             &=& \varphi( (t-@) \subtermfilter M ) & \mbox{(by i.)} \\
           &&=& \varphi( (t-@) \hefilter r ) & \mbox{(Def. subterm filtering)} \\
           &&=& \varphi( t \hefilter r ) & \mbox{(@-node are not justified).} \qed
\end{array}$$

The function $\varphi$ regarded as a function from the set of vertices $V_\lambda \union V_{\sf var}$ of the computation tree to moves in arenas is not injective.
For instance the two occurrences of $x$ in the computation tree of the term $\lambda f x. f x x$ are mapped to the same question. However
the function $\varphi$ defined on the set of traversals to interaction plays of game semantics is injective:
\begin{lemma}[$\psi$ and $\varphi$ are injective]
\label{lem:varphiinjective}
For any two traversals $t_1$ and $t_2$:
\begin{itemize}
\item[(i)] If $\varphi (t_1 - @ ) = \varphi (t_2 - @ )$ then $t_1-@ =t_2 -@$\ ;
\item[(ii)] if $\psi (t_1 \upharpoonright r ) = \psi (t_2 \upharpoonright r )$ then $t_1\upharpoonright r = t_2\upharpoonright r$\ .
\end{itemize}
\end{lemma}
\begin{proof}
For any node $n$ of a traversal $t$ let us write $ptr(n)$ to denote the distance between $n$ and its justifier node in $t$. If $n$ has not link then we set $ptr(n)=0$. We also use the same notation for sequences of moves.

\begin{lemma}[Preleminary lemma]
\label{lem:varphiinjective:prelem}
\begin{equation}
\left(
  \begin{array}{ll}
    t \cdot n_1, t \cdot n_2 \in \travset \\
    \zand\ n_1 \neq n_2
  \end{array}
\right)
 \mbox{ implies } n_1,n_2 \in N^{\upharpoonright r}_{\lambda} \zand ( \varphi(n_1) \neq \varphi(n_2) \zor ptr(n_1) \neq ptr(n_2) ) \ . \end{equation}
\end{lemma}
\begin{proof}
Let $t \cdot n_1, t \cdot n_2 \in \travset$.
First we remark that the traversal rules have a weak form of determinism which ensures that $n_1$ and $n_2$ belong to the same category of node i.e.\ they must be both in $N_{\sf var}$, $N_@$ or $N_\lambda$.

Suppose that $n_1, n_2 \in N_@$ then $t \cdot n_1$ and $t \cdot n_2$ were formed using the (App) rule. Since this rule is deterministic we must have $n_1=n_2$ which violates the second hypothesis.


Suppose that $n_1,n_2\in N_{\sf var}$. The traversals $t \cdot n_1$ and $t \cdot n_2$ must have been formed using either rule (Lam) or (App). But these two rules are deterministic and their domains of definition are disjoint. Hence again the second hypothesis is violated.

Suppose that $n_1,n_2\in N_\lambda$ then
the traversals $t \cdot n_1$ and $t \cdot n_2$ must have been formed using either rule (Root), (App), (Var) or (InputVar). Since all these rules have disjoint domains of definition, the same rule must have been use to form $t \cdot n_1$ and $t \cdot n_2$. Supposed that one of the rules (Root), (App) and (Var) has been used then since they are all deterministic we have $n_1=n_2$ which violates the second hypothesis. Consequently, the rule (InputVar) must have been used and therefore $n_1,n_2 \in N_\lambda^{\upharpoonright r}$. By definition of (InputVar), in order to have $n_1\neq n_2$ and $\varphi(n_1) = \varphi(n_2)$, the parent node of the last node in $t$ must occurs at more than one position in $\oview{t}$ and $n_1,n_2$ correspond to the child node of two different occurrences of that parent node in $\oview{t}$. But then the links associated to $n_1$ and $n_2$ will point to their respective occurrence of that parent node in $\oview{t}$ hence $ptr(n_1) \neq ptr(n_2)$.
\end{proof}

\noindent {\it (continuation of the proof of Lemma \ref{lem:varphiinjective})}

(i) The result is trivial is either $t_1$ or $t_2$ is empty.
Suppose that $t_1-@\neq t_2-@$ then necessarily $t_1 \neq t_2$, thus there are some sequences $t'$, $u_1$, $u_2$ and some nodes $n_1,n_2$ such that
 $t_1 = t' \cdot n_1 \cdot u_1$, $t_2 = t' \cdot n_2 \cdot u_2$ with either $n_1\neq n_2$ or $ptr(n_1) \neq ptr(n_2)$.

If $n_1 = n_2$ then $ptr(n_1) \neq ptr(n_2)$ therefore $n_1,n_2 \not\in N_@$ (otherwise $ptr(n_1) = 0 = ptr(n_2)$). Since $ptr(\varphi(n_1)) = ptr(n_1)$ and  $ptr(\varphi(n_2)) = ptr(n_2)$ we must have $\varphi(t' \cdot n_1) \neq \varphi(t' \cdot n_2)$. Since $n_1,n_2 \not\in N_@$ we also have $\varphi((t' \cdot n_1)-@) \neq \varphi((t' \cdot n_2)-@)$. Hence $\varphi(t_1-@) \neq \varphi(t_2-@)$.

If $n_1 \neq n_2$ then by Lemma \ref{lem:varphiinjective:prelem} we have $n_1,n_2 \not\in N_@$ and $\varphi(n_1) \neq \varphi(n_2)$ or $ptr(n_1) \neq ptr(n_2)$ which again implies $\varphi(t_1-@) \neq \varphi(t_2-@)$.


(ii) Suppose that $t \upharpoonright r \neq t' \upharpoonright r$ then necessarily $t \neq t'$ which in turn implies that for some sequences $t_1'$, $t_2'$, $u_1$, $u_2$ and some nodes $n_1 \neq n_2$
we have $t_1 = t' \cdot n_1 \cdot u_1$, $t_2 = t' \cdot n_2 \cdot u_2$ and either $n_1\neq n_2$ or $ptr(n_1) \neq ptr(n_2)$.

If $n_1 = n_2$ then $ptr(n_1) \neq ptr(n_2)$. An   analysis of the traversal rules shows that the rule (InputVar) is the only rule which can visit the same node with two different pointers. Hence $n_1,n_2 \in N_\lambda^{\upharpoonright r}$.
Therefore $\psi( (t'\cdot n_1) \upharpoonright r ) = \psi( (t'\upharpoonright r) \cdot n_1 )  \neq \psi( (t'\upharpoonright r) \cdot n_2 )$. Hence    $\psi( t_1\upharpoonright r ) \neq \psi( t_2\upharpoonright r )$.

If $n_1 \neq n_2$ then we can use Lemma \ref{lem:varphiinjective:prelem}
to obtain $\psi( t_1\upharpoonright r ) \neq \psi( t_2\upharpoonright r )$.
\end{proof}

\begin{corollary} \
\label{cor:varphi_bij}
\begin{itemize}
\item[(i)] $\varphi$ defines a bijection from $\travset(M)^{-@}$
to $\varphi(\travset(M)^{-@})$\ ;
\item[(ii)] $\psi$ defines a bijection from $\travset(M)^{\upharpoonright r}$ to
$\psi(\travset(M)^{\upharpoonright r})$\ .
\end{itemize}
\end{corollary}

\subsubsection{The correspondence theorem}
We now state and prove the correspondence theorem for the
simply-typed $\lambda$-calculus without interpreted constants
($\Sigma = \emptyset$). The result extends immediately to the
simply-typed $\lambda$-calculus with \emph{uninterpreted} constants
since we can regard constants as being free variables.

\begin{lemma}[Local Traversal Extension]
\label{lem:local_traversal_progression}
Let $M'$ be a subterm of $M$, $t \in \travset(M)$,
$t' \in \travset(M')$ such that $t' \neq \epsilon$ and $t\subseqof t'$. If the
traversal $t' \cdot n$ of $\tau(M')$ can be formed using a rule different from \rulenamet{InputVar}
and $\rulename{InputVar^{val}}$ then either $t' \cdot n \subseqof t $ or
$ t' \cdot n \in \travset(M)$.
where $n$'s link in $t \cdot n$ points to the same node occurrence as in $t' \cdot n$.
\end{lemma}
\proof
By Case analysis on the traversal rule used to form $t'\cdot n$.
\qed

This lemma says that extending a traversal locally also extends the traversal globally: the traversal $t$ of $M$ can be extended by extending a ``sub-traversal'' $t'$ of some sub-term $M'$.
This is not obvious since $t'$ is a subsequence of $t$ which means that
the nodes in $t'$ are also present in $t$ with the same pointers but with some other nodes interleaved in between. However these interleaved nodes are inserted in a preservative way which allows us to apply the rule used to extend $t'$ on $t$.

The following theorem establishes a correspondence between the
game-denotation of a term and the set of traversals of its
computation tree:
\begin{theorem}[The Correspondence Theorem]
\label{thm:correspondence}
 For any simply-typed term $\Gamma \vdash M :T$,
the function $\varphi_M$ defines a bijection from $\travset(M)^{\upharpoonright
r}$ to $\sem{\Gamma \vdash M : T}$ and a bijection from
$\travset(M)^{-@}$ to $\intersem{\Gamma \vdash M : T}$:
\begin{eqnarray*}
 \varphi_M  &:& \travset(\Gamma \vdash M : T)^{-@} \stackrel{\cong}{\longrightarrow} \intersem{\Gamma \vdash M :T} \\
 \psi_M  &:& \travset(\Gamma \vdash M : T)^{\upharpoonright r} \stackrel{\cong}{\longrightarrow} \sem{\Gamma \vdash M :T} \ .
\end{eqnarray*}

\end{theorem}

%\begin{proposition}
%\label{prop:rel_gamesem_trav} Let $\Gamma \vdash M : T$ be a
%simply-typed $\lambda$-term and $r$ be the root of $\tau(M)$. Then:
%\begin{itemize}
%\item[(i)]  $\varphi_M(\travset(M)^{-@}) = \intersem{\Gamma \vdash M : T}$ \ ;
%\item[(ii)] $\varphi_M(\travset(M)^{\upharpoonright r}) = \sem{\Gamma \vdash M : T}$ \ .
%\end{itemize}
%\end{proposition}

\begin{remark}
\label{rem:corresp_proofreduction}
    By corollary \ref{cor:varphi_bij}, we just need to show that
    $\varphi_M$ defines \emph{surjections}, that is to
    say:
    \begin{eqnarray*}
    \varphi_M(\travset(M)^{-@}) &=& \intersem{\Gamma \vdash M : T} \\
    \psi_M(\travset(M)^{\upharpoonright r}) &=& \sem{\Gamma \vdash M :
    T}
    \end{eqnarray*}
    The first equation implies the second one, indeed:
    \begin{align*}
    \sem{\Gamma \vdash M : T} &= \intersem{\Gamma \vdash M : T} \upharpoonright \sem{\Gamma \rightarrow T} & \mbox{(eq. \ref{eqn:int_std_gamsem})} \\
            &= \varphi_M(\travset^{-@}(M)) \upharpoonright \sem{\Gamma \rightarrow T} & \mbox{(by (i))}\\
            &= \psi_M(\travset^{\upharpoonright r}(M)) & \mbox{(lemma \ref{lem:varphi_filter})}
    \end{align*}
    therefore we just need to prove the first equation.
\end{remark}

    Let us give a brief overview of the proof before giving it in full details.
    It proceeds by induction on the structure of the computation tree.
    The only non-trivial case is the application: the computation tree
    $\tau(M)$ has the following form:
        $$ \tree[levelsep=4ex]{\lambda \overline{\xi}}
            { \tree[levelsep=4ex]{@}
                {   \TR{\tau(N_0)} \TR{\ldots} \TR{\tau(N_p)}}}
        $$

    A traversal of $\tau(M)$ proceeds as follows: it starts at the root $\lambda \overline{\xi}$ of the tree $\tau(M)$ (rule
    (Root)), it then passes the node @ (rule (Lam)).
    After this initialization part, it proceeds by traversing the term $N_0$ (rule (App)).
    At some point, while traversing $N_0$, some variable $y_i$ bound by the root of $N_0$ is visited. The traversal
    of $N_0$ is interrupted and jumps (rule (Var)) to the root of $\tau(N_i)$. The process then goes on with $\tau(N_i)$.
    When traversing $N_i$, if the traversal encounters a variable bound by the root of $\tau(N_i)$ then the traversal of $N_i$
    is interrupted and
    the traversal of $N_0$ resumes.  This schema is repeated until the traversal of $\tau(N_0)$ is completed\footnote{Since we are considering
    simply-typed terms, the traversal does indeed terminate. However this will not be true anymore in the \pcf\ case.}.

    The traversal of $M$ is therefore made of an initialization part followed by an interleaving of a traversal of $N_0$ and
    several traversals of $N_i$ for $i=1..p$. This schema is reminiscent of the way the evaluation copycat map $ev$ works in game semantics.

    The key idea is that every time the traversal pauses the traversal of a subterm and switches to another one,
    the jump is permitted by one of the four ``copycat'' rules (Var), (Answer-@-$\lambda$), (Answer-$\lambda$-var) or (Answer-var).
    We show by (a second) induction that these copycat rules define precisely what the copycat strategy $ev$ performs on sets of plays.

%    In the game semantics, the evaluation map (a copy-cat strategy) copies this opening move to an initial move $m_0$ in the game
%    $B_0$ and the game continues in $B_0$. We reflect this in the traversal : we make $t$ follow
%    the ``script'' given by the traversal $t^0_{m_0}$.
%    The rule (App) allow us to initiate this simulation  by visiting the  first move in $t^0_{m_0}$: the root of $\tau(N_0)$.
%
%    This simulation continues until it reaches a node $\alpha_0$ which is hereditarily justified by the root
%    $\tau(N_0)$: $\alpha_0$ is present in the reduction of traversal of $t^0_{m_0}$ therefore $\varphi_{N_0}(\alpha_0)$ is an un-hidden move played in $A_0$.
%
%    In the game semantics this corresponds to a move played in a component $A_k$ for some $k\in 1..p$ of
%    of the game $B_0$ in which case the evaluation map copies the move to an initial move $m_1$ in the corresponding component $B_k$.
%
%    To reflect this the traversal now opens up a new thread and simulates the traversal $t^k_{m_1}$.  Again, this simulation stops when we reach a node
%    $\alpha_1$ in $t^k_{m_1}$ which is hereditarily justified by the root of $\tau(N_k)$: $\alpha_1$ must be present in the reduction of traversal
%    of $t^k_{m_1}$ therefore $\varphi_{N_k}(\alpha_1)$ is an un-hidden move played in $A_k$.
%    In the game semantics, this move $\alpha$ is copied back to the component $B_k$ of the game $B_0$.
%
%    The traversal now resumes the simulation of $t^0_{m_0}$. And the process goes continuously.
\smallskip

\begin{proof}
Let $\Gamma \vdash M : T$ be a simply-typed term where $\Gamma =
x_1:X_1,\ldots x_n:X_n$. We assume that $M$ is already in
$\eta$-long normal form. By remark \ref{rem:corresp_proofreduction} we just need to
show that $\varphi_M(\travset(M)^{-@}) = \intersem{\Gamma \vdash M : T}$.
We proceed by induction on the structure of $M$:
\begin{enumerate}[$\bullet$]
    \item (abstraction) $M \equiv \lambda \overline{\xi}. N : \overline{Y} \rightarrow B$ where $\overline{\xi} = \xi_1:Y_1,\ldots \xi_n:Y_n$. On the first hand we have:
\begin{eqnarray*}
\intersem{\Gamma \vdash \lambda \overline{\xi}. N:T} &=& \Lambda^n( \intersem{\overline{\xi}, \Gamma \vdash N: B } ) \\
        &\simeq& \intersem{\overline{\xi}, \Gamma \vdash N: B } \ .
\end{eqnarray*}
On the other hand, the computation tree $\tau(N)$ is isomorphic to
$\tau(\lambda \xi_1\ldots \xi_n . N)$ (up to a renaming of the root
of the computation tree) and $\travset(N)$ is isomorphic to
$\travset(\lambda \xi_1\ldots \xi_n . N)$.
Hence we can conclude using the induction hypothesis.

  \item (variable) $M \equiv x_i$. Since $M$ is in $\eta$-long normal form, $x$ must be of ground
      type. The computation tree $\tau(M)$ and the arena $\intersem{\Gamma \rightarrow o}$ are represented below
      (value leaves and answer moves are not represented):
        $$ \tree[levelsep=6ex]{ \lambda }{\TR{x_i}} \hspace{2cm}
        \tree{ q_0 }
        {   \tree[linestyle=dotted]{q^1}{\TR{} \TR{} }
            \tree[linestyle=dotted]{q^2}{\TR{} \TR{} }
            \TR{\ldots}
            \tree[linestyle=dotted]{q^n}{\TR{} \TR{} }
        }
        $$

        Let $\pi_i$ denote the $i$th projection of the interaction game
        semantics. We have:
        \begin{align*}
        \intersem{M} &= \pi_i = \prefset(\{ \Pstr{(q0){q_0} \cdot (qi){q^i} \cdot (vqi-qi){v_{q^i}} \cdot (vq0-q0){v_{q_0}} } \ | \ v\in \mathcal{D} \})\ .
        \end{align*}

        It is easy to see that traversals of $M$ are precisely
        the prefixes of $ \Pstr{ (lmd)\lambda \cdot (xi){x_i}
        \cdot (vxi-xi){v_{x_i}} \cdot (vlmd-lmd){v_{\lambda}}}$.
        $M$ is in $\beta$-normal therefore $\travset(M)^{-@} =
        \travset(M)$ and since $\varphi_M(\lambda) =
        q_0$ and $\varphi_M(x_i) = q^i$, we have:
        $$ \varphi_M(\travset^{-@}(M)) = \varphi_M(\travset(M)) = \varphi_M(\prefset( \lambda \cdot x_i \cdot v_{x_i} \cdot v_{\lambda}))
         = \intersem{M} \ .
        $$


    \item (application) $M = N_0 N_1 \ldots N_p :o$ where $N_0$ is not a variable.
    We have the typing judgments $\Gamma \vdash N_0 N_1 \ldots
    N_p : o$ and $\Gamma \vdash N_i : B_i$ for $i\in 0..p$ where
    $B_0 = (B_1,\ldots,B_p,o)$ and $p\geq 1$.

    The tree $\tau(M)$ has the following form:
    $$ \tree[levelsep=6ex]{\lambda^{[r]}}
        { \tree[levelsep=6ex]{@}
            {
            \tree[levelsep=3mm,edge=\noedge]{\TR{{\lambda y_1 \ldots y_p}^{[r_0]}}}{\Tr[ref=t]{\pstribox{\tau(N_0)}}}
            \tree[levelsep=3mm,edge=\noedge]{\TR{[r_1]}}{\Tr[ref=t]{\pstribox{\tau(N_1)}}}
             \TR{\ldots}
            \tree[levelsep=3mm,edge=\noedge]{\TR{[r_p]}}{\Tr[ref=t]{\pstribox{\tau(N_p)}}}
        }}
    $$
    where $r_j$ denote the root of $\tau(N_j)$ for $j\in \{0..p\}$.

    We have:
    $$
    \intersem{\Gamma \vdash M : o}
            =  \underbrace{\langle \intersem{\Gamma \vdash N_0 : B_0}, \ldots \intersem{\Gamma \vdash N_p : B_p} \rangle}_{\Sigma} \,^\dagger\ \| \ ev
    $$

    We define the games $A$, $B$ and $C$ are defined as follows:
    \begin{eqnarray*}
        A &=& \Gamma = X_1 \times \ldots \times X_n\\
        B &=& \underbrace{((B_1' \times \ldots \times B_p') \rightarrow o')}_{B_0} \times B_1 \times \ldots \times B_p\\
        C &=& o \ .
    \end{eqnarray*}

    Figure \ref{fig:interaction_strategy_denotations} shows
    a tree-representation of $\intersem{M}$ which fixes the names of the different games involved in the interaction strategy.

%    Since $\varphi_M = \psi_M \union \varphi_{N_0} \union
%    \varphi_{N_1}$ the induction hypothesis gives us:
%    \begin{align}
%    \varphi_{M} (\travset^{-@}(N_0)) &= \intersem{\Gamma \vdash N_0 : B_0} \label{eqn:ih_1} \\
%    \varphi_{M}(\travset^{-@}(N_1)) &= \intersem{\Gamma \vdash N_1 : B_1} \label{eqn:ih_2}
%    \end{align}
\begin{enumerate}
\item[$\subseteq$]
    We first prove that $\intersem{\Gamma \vdash M : T}
    \subseteq \varphi_{M}( \travset^{-@}(M) )$. Suppose $u \in
    \intersem{\Gamma \vdash M : T}$. We give a constructive
    proof that there is a traversal $t$ of $M$ such
    that $\varphi_M(t-@) = u$ by induction on the length of $u$.
    Let $q_o$ and $q_0'$ be the initial question of $C$
    and $B_0$ respectively.

    \emph{Base cases}:
    \begin{compactitem}[-]
    \item If $u=\epsilon$ then we take the empty traversal $t=\epsilon$ formed
with \rulenamet{Empty}. Clearly $\varphi(t) = u$.
    \item If $|u|=1$ then $u=q_0$ is the initial move in $C$. The traversal $t=\lambda$ formed with the rule \rulenamet{Root} verifies $\varphi(t) = u$.
    \item If $|u|=2$ then necessarily $u = q_0 \cdot q_0'$. The rules \rulenamet{Root}, \rulenamet{App}
and \rulenamet{Lam} permit us to build the traversal $t = \lambda^{[r]} \cdot @ \cdot \lambda \overline{y}^{[r_0]}$ which clearly verifies $\varphi_M(t-@) = u$.
    \end{compactitem}

    \emph{Step cases}: Suppose that $u = w \cdot m \in \intersem{\Gamma \vdash M : T}$
    for some move $m \in M_T$ where
    $w = \varphi_M(t-@)$ for some traversal $t$ of $\tau(M)$
    and $|w|>1$.

    By unraveling the definition of $u \in \intersem{\Gamma \vdash M : T}$ we have:
    \begin{eqnarray*}
      &&      \left\{
            \begin{array}{ll}
                u \in L_T\\
                u \upharpoonright T^0  \in \Sigma^\dagger \\
                u \upharpoonright T^1  \in  ev
            \end{array}
            \right. \\
    & \mbox{or equivalently} & \left\{
    \begin{array}{ll}
        u \in L_T \\
        \hbox{for any initial $m$ in $!B_0^0 \otimes \ldots \otimes !B_p^0$ there is $j \in \{0..p\}$ such that } \\
        \left\{\begin{array}{ll}
            u \filter m \filter T^{00j} \in \intersem{N_j} \label{eq:def_z} \\
            u \filter m \filter T^{00k} = \epsilon \quad \mbox{ for every } k\in \{1..p\}\setminus\{j\} \label{eq:b}
        \end{array}
        \right. \\
        u \upharpoonright T^1  \in  ev
    \end{array}
    \right.
    \end{eqnarray*}

We recall that $m \in M_T$ is an equivalence class of moves from $\mathcal{M}_T$. For any game $A$ appearing in the interaction game $T$ we will write ``$m \in A$'' to mean that some citizen of the class $m$ belongs to the set of moves $M_A$. Similarly, for any sub-interaction game $T'$ of $T$, we write ``$m \in T'$'' to mean that some citizen of the class $m$ belongs to the set of moves $\mathcal{M}_A$.

We do a case analysis on $m$: we either have $m\in C$ or $m\in T^0$:
    \begin{enumerate}[-]
    \item Suppose $m \in C$. $m$ is played by the strategy $ev$ whose plays do not contain any internal move. Hence $m$ is either $q_0$ or $v_{q_0}$ for some
    $v\in\mathcal{D}$. But since $q_0$ can occur only once in
    $u$ and $|u|>1$, $m$ must be $v_{q_0}$ for some
    $v\in \mathcal{D}$.  Moreover $m$ is a P-move played by the
    copy-cat strategy $ev$ in $B,C$ therefore it is the copy
    of the some move $v_{q_0'}$ answering the question $q_0'$ in the sub-game $o'$.

    In fact this move $v_{q_0'}$ is precisely $w$'s last move. Indeed
    suppose that $w = \ldots v_{q_0'} \cdot w'$. The play
    $w_{\prefixof v_{q_0'}}\filter A,B$ is complete since its
    first move $q_0'$ is answered by $v_{q_0'}$. Therefore by
    Lemma \ref{lem:inter_complete}(ii), $w_{\prefixof
    v_{q_0'}}\filter T^0$ is maximal. Thus moves in $w'$ must
    be played in $T^1$ by $ev$, but since $ev$ does not play internal
    moves, $w'$ is necessarily empty.

    Consequently, by the induction hypothesis, the last move in $t$ is $\varphi(v_{q_0'}) = v_{\lambda y_1}$.
    The rules \rulenamet{Answer-@-$\lambda$} and \rulenamet{Answer-$\lambda$-@} permits us to extend
    the traversal $t$ into $t \cdot v_@ \cdot v_{\lambda \overline{\xi}}$ where $v_@$ and $v_{\lambda
    \overline{\xi}}$ point to the second and first node of $t$ respectively. Clearly we have $\varphi_M((t\cdot v_@ \cdot v_{\lambda \overline{\xi}})-@) = u$.

    \item Suppose $m\in T^0$. Then $m$ is hereditarily justified by some initial move $b$ in $B_j$ for some $j\in \{0..p\}$.

        Since $u \filter b \filter T^{00j} \in \intersem{N_j}$, the outermost induction hypothesis gives us:
        \begin{equation}
        u \filter b \filter T^{00j} = \varphi_{N_j}(t_j-@)  \label{eqn:corresp_outmost_ih}
        \end{equation}
          for some traversal $t_j \in \travset(N_j)$. W.l.o.g we can assume that $t_j^\omega \neq @$.
        By Corollary \ref{cor:varphi_bij}, $\varphi_{N_j}$ is a bijection from $\travset(N_j)^{-@}$ to
        $\varphi_{N_j}( \travset(N_j)^{-@})$ therefore $t_j-@ = \varphi^{-1}_{N_j}( (u \filter b) \filter T^{00j} )$
        and we have:
        \begin{align}
         t_j - @ &= \varphi^{-1}_{N_j}(u \filter b \filter T^{00j}) &  \nonumber \\
                        &\in \varphi^{-1}_{M}(u \filter b \filter T^{00j}) & \parbox[t]{7cm}{($\varphi_M = \psi_M \union \Union_{k\in \{0..p\}} \varphi_{N_k}$ by definition.)} \label{eqn:proof_corres_1}
        \end{align}

        Note that $\varphi^{-1}_{M}(\ip(u \filter b \filter T^{00j}))$ is not a traversal but \emph{a set of} traversals since $\varphi_{M}^{-1}$ is not necessarily bijective on $\intersem{N_j}$. Thus we have to use set-membership in the equation instead of traversal equality.

        Since $\varphi^{-1}_{M}$ is monotonous and $u \filter b \filter T^{00j} \subseqof u \subseqof w$, all the traversals in $\varphi^{-1}_{M}(u \filter b \filter T^{00j})$ are subsequences of $\varphi^{-1}_{M}( w )$ thus:
        \begin{align*}
        t_j - @ &\subseqof \varphi^{-1}_{M}( w ) & \mbox{(by Eq. \ref{eqn:proof_corres_1})}\\
                &= t -@ & \mbox{($\varphi_{M}$ is bijective on $\travset(M)$ by Cor. \ref{cor:varphi_bij}).}
        \end{align*}

      Thus by Lemma \ref{lem:minus_at_plus_at}(ii) we have
        $\ip( t_j) \subseqof t$.

    Furthermore:
     \begin{align*}
    t_j\subseqof t
        &\implies \varphi_M (t_j) \subseqof \varphi_M (t) \\
        &\iff (w \cdot m) \filter b \filter T^{00j} \subseqof  w \\
        &\iff (w \filter b \filter T^{00j}) \cdot m \subseqof  w  & \mbox{ ($m$ is h.j. by $b$ and belongs to $T^{00j}$)} \\
        &\implies \left( \left(w \filter b \filter T^{00j} \cdot m \right) \filter b \filter T^{00j} \right) \cdot m \subseqof w & \mbox{ (by iterating the previous equation)} \\
        &\iff (w \filter b \filter T^{00j}) \cdot m  \cdot m \subseqof  w \ .
    \end{align*}
    The last equation is false since a given move cannot occur twice consecutively in a legal interaction play! Hence $t_j\not\subseqof t$.

    \begin{enumerate}[(a)]
    \item  Suppose $t_j$'s last move is \emph{not} visited by the rule \rulenamet{InputVar} nor
        $\rulename{InputVar^{val}}$. Since $\ip( t_j) \subseqof t$ and $t_j$ is a traversal of the subterm $N_j$ of $M$, by Lemma \ref{lem:local_traversal_progression}
        we have either $t_j\subseqof t$ or $t \cdot t_j^\omega$ is a traversal of $M$
        where $t_j^\omega$'s pointer is the same as in $t_j$. Hence, since $t_j\not\subseqof t$, $t \cdot t_j^\omega$ is a traversal of $M$.

        Furthermore we have
        \begin{align*}
            \varphi_M (t_j^\omega) &= (\varphi_M (t_j-@))^\omega & \mbox{($t_j^\omega \neq @$ by assumption)}\\
                                   &= ((w \cdot m) \filter b\filter T^{00j})^\omega & \mbox{(by Eq. \ref{eqn:corresp_outmost_ih})}\\
                                   &= ((w \filter b\filter T^{00j}) \cdot m))^\omega & \mbox{($m$ is h.j. by $b$ and belongs to $T^{00j}$)}\\
                                   &= m
        \end{align*}
        and therefore
        \begin{align*}
          \varphi_{M}((t \cdot t_j^\omega)-@)  &=  \varphi_{M}(t -@)  \cdot \varphi_{M}(t_j^\omega-@)\\
                &=   w \cdot \varphi_{M}(t_j^\omega-@) & \mbox{(by the innermost induction hypothesis)}\\
                &=   w \cdot m & \mbox{(by the previous equation).}
        \end{align*}
        Hence the traversal $t \cdot t_j^\omega$ meets the requirement.

    \item Suppose $t_j$'s last move is visited with the rule \rulenamet{InputVar}.

    Then $t_j$ is of the form
    $$\Pstr[18pt]{ t_j = t' (z)z \cdot t'' \cdot (n-z){t_j^\omega}}$$
for some $z \in N_\lambda^{\filter r_j}$ (see remark \ref{rem:inputvar})
and some input-variable $x \in N^{\filter r_j}_{var}$ occurs in $z\cdot t'$ such that $x$ is the pending node in $\ip t_j = t' \cdot z \cdot t''$ ({\it i.e.}~ with $?(t' \cdot z \cdot t'')^\omega = x$).

Suppose that $z\in N^{\filter r}$ then $z$ is a free variable of $M$ (and $N_j$).
Since the O-view of $t_j$ coincides with the O-view of $t$

    \item Suppose $t_j$'s last move is visited with the rule $\rulename{InputVar^{val}}$.
    This case is similar to the previous one but the rule $\rulename{InputVar^{val}}$ is used instead
    of $\rulename{InputVar}$.
    \end{enumerate}



\notetoself{PIECE OF OLD PROOF
%   \item Suppose that $m,m^1 \in T^{000}$.
%    The strategy $ev$ is responsible for switching of thread
%    in $B_0$ therefore, in the interaction semantics, there
%    must be a copycat move in-between two moves belonging to
%    two different threads. Since $m$ and $m^1$ are
%    consecutive moves in the sequence $u$, they must belong
%    to the same thread i.e. there are hereditarily justified
%    by the same initial $m_0$ in $B_0$.

Suppose that $m \in T^{000}$ and $m^1 \in T^{001}$.

    $t$ is obtained from $t-@$ by applying the
    transformation $+@$. We apply the same transformation to
    $u$ in order to make $O$-questions and $P$-questions in
    $u$ match with $\lambda$-nodes and variable nodes in
    $t'$ respectively. We write this sequence $u+@$. The
    $+@$ operation inserts nodes in the sequence but not at
    the end, therefore $m^1$, the last move in $u'$, is also
    the last move in $u'+@$. Let us note $n^1$ for the last
    move in $t'$.


        $n^1$ is a variable node then $m^1$ is a P-move and $m$ is an O-move
            and therefore $m$ is the copy of $m^1$ duplicated in $B_1$ by the evaluation strategy.
            Consequently, $m^1$ points to some $m^2$ and $m$ points to the node preceding $m^2$ denoted by $m^3$.
            The diagram below shows an example of such sequence:
                $$
                \begin{array}{ccccccccccc}
                  & A & \longrightarrow & ( (B_1' &\rightarrow & o') & \times & B_1 ) & \longrightarrow & o' \\
                  &&& &&&&&& \rnode{q0}{q_0 (\lambda \overline{\xi})} & O\\
                  &&& &&&&&  \\
                O &&& && \rnode{q1}{q_0' (\lambda \overline{y})} &&&&& P \\
                P &&& \rnode{m3}{m^3 (y_1)} &&&&&&& O \\
                O &&& &&&& \rnode{m2}{m^2 (\lambda \overline{z}^{[r_1]})} &&& P \\
                P &&& &&&& \rnode{m1}{m^1 (z)} &&& O \\
                O &&& \rnode{m}{m} &&&&&&& P \\
                \end{array}
                \ncline[nodesep=3pt]{->}{q1}{q0} \mput*{@}
                \nccurve[nodesep=3pt,ncurv=2,angleA=180,angleB=180]{->}{m1}{m2}
                \ncarc[nodesep=3pt,ncurv=1,angleA=90,angleB=180]{->}{m3}{q1}
                \ncarc[nodesep=3pt,ncurv=1,angleA=90,angleB=180]{->}{m}{m3}
                \ncline[nodesep=3pt]{->}{m2}{q0}
                $$

        $t'$  and $u+@$ have the following forms:
        \begin{eqnarray*}
                t'&=& \Pstr{ \ldots \cdot n^3 \cdot (n2){n^2} \cdot \ldots \cdot (n1-n2,30){n^1} } \\ \\
                u+@ &=& \Pstr{ \ldots \cdot (m3){m^3} \cdot (m2){m^2} \cdot \ldots \cdot (m1-m2,30){m^1} \cdot (m-m3,30){m} }
        \end{eqnarray*}

        Since $n^1$ is a variable node, $n^2$ must be a $\lambda$-node.
        $n^3$ is either a variable node or an @-node. In fact $n^3$ is necessarily a variable node. Indeed,
        $n^3$ is mapped to $m^3$ by $\varphi_{N_0}$ and $m^3$ belongs to $B_i'$ (i.e. it is not
        an internal move of $T^0$). The function $\varphi_{N_0}$ is defined in such a way that
        only nodes which are hereditarily justified by $r_0$ are mapped to nodes in $B_j'$.
        Consequently, since @-node don't have justifier, $n^3$ cannot be an @-node.

        Hence $n^1$ is a variable node, $n^2$ is a $\lambda$-node and $n^3$ is a variable node.


        We  can therefore apply the (Var) rule to $t'$ and we obtain a traversal of the following form:

        \begin{eqnarray*}
            t&=& \Pstr{ \ldots \cdot (n3){n^3} \cdot (n2){n^2} \cdot \ldots \cdot (n1-n2,30){n^1} \cdot (n-n3,30){n} }
        \end{eqnarray*}

        We have $\varphi(t'-@) = u'$ by the induction hypothesis and $\varphi(n) = m$ by definition of $\varphi$.
        Therefore since $m$ and $n$ point to the same position we have $\varphi(t-@) = u$.
}

    \end{enumerate}

\item[$\supseteq$]
  For the converse, $\varphi_{M}( \travset^{-@}(M) ) \subseteq \intersem{M}$, it is an easy induction
  on the traversal rules. We omit the details here.


\end{enumerate}


    \item (application') $M = x_i N_1 \ldots N_p :o$ with $X_i = B_0 = (B_1' \times \ldots \times B_p') \rightarrow o'$. The tree $\tau(M)$ has the following form:
    $$ \tree[levelsep=6ex]{\lambda^{[r]}}
        { \tree[levelsep=6ex]{x_i}
            {
            \tree[levelsep=3mm,edge=\noedge]{\TR{[r_1]}}{\Tr[ref=t]{\pstribox{\tau(N_1)}}}
             \TR{\ldots}
            \tree[levelsep=3mm,edge=\noedge]{\TR{[r_p]}}{\Tr[ref=t]{\pstribox{\tau(N_p)}}}
        }}
    $$
    The interaction strategy
    $\intersem{\Gamma \vdash M : o}
            =  \underbrace{\langle \pi_i, \intersem{\Gamma \vdash N_1 : B_1}, \ldots \intersem{\Gamma \vdash N_p : B_p} \rangle}_{\Sigma} \,^\dagger\ ;^{\{1..p\}} \ ev$
    is represented on Figure \ref{fig:interaction_strategy_denotations}.

    The proof is identical to the previous case except that in the $\subseteq$ part of the proof:
    \begin{itemize}
        \item In the base case of the induction where $|u|=2$,
        the rule \rulenamet{InputVar} is used instead of \rulenamet{App} to visit the node $x$ instead of $@$;
        \item in the step case of the induction, for the subcase $m\in C$, the rules \rulenamet{Answer-var-$\lambda$} and \rulenamet{Answer-$\lambda$-var} are used instead of \rulenamet{Answer-@-$\lambda$} and \rulenamet{Answer-$\lambda$-@} respectively;
        \item in the step case $m\in T^0$, when $m$ is hereditarily justified by a move $b \in B_j$ for
         $j\in \{1 .. p\}$ the proof remains unchanged. The case where $m$ is hereditarily justified by a move $b \in B_0$ is treated as follows: $m$ is played by the projection strategy $\pi$ denoting $x$.
         Since $m$ is played in $B_0' = B_1' \times \ldots \times B_p'$ it must be also hereditarily justified by some initial move $b'$ of $B_k'$ for some $k \in \{1.. p\}$ or by an initial move in $o'$. But moves of $B_k'$ are $\sim$-equivalent to the corresponding move in $B_k$ and similarly $o'$ is $\sim$-equivalent to $o$, therefore we fall back to the previous case of the induction where $m$ is hereditarily justified by some initial move $b\in B_k$ for $k\in \{1..p\}$ or some initial move in $o$!
    \end{itemize}


\end{enumerate}


\end{proof}


\begin{corollary} \hfill
\begin{enumerate}[i.]
\item Let $\tau(M')$ be a subtree of $\tau(M)$ for some subterm $M'$ of $M$, and  $N'$ denote the set
of nodes of $\tau(M')$. Then
$$t \in \travset(M) \implies t\filter N' \in \travset(M') \ .$$

\item If $M$ is in $\beta$-normal form then for any traversal $t$,
$\varphi_M(t)$ is a maximal play if and only if $t$ is a maximal
traversal.
\end{enumerate}
\end{corollary}
\begin{proof}
\begin{enumerate}[i.]
\item
 \todo
\item If $M$ is in $\beta$-normal form then
$\travset(M)^{\upharpoonright r} = \travset(M)$ therefore
$\varphi$ defines a bijection on $\travset(M)$. Let $t$ be a
traversal such that $\varphi(t)$ is a maximal play. Let $t'$ be
a traversal such that $t \sqsubseteq t'$. By monotonicity of
$\varphi$ we have $\varphi(t) \sqsubseteq \varphi(t')$ which
implies $\varphi(t) = \varphi(t')$ by maximality of $\varphi(t)$
which in turn implies $t'=t$ by injectivity of $\varphi$. The
other direction is proved identically using injectivity and
monotonicity of $\varphi^-1$.
\end{enumerate}
\end{proof}
\smallskip The following diagram recapitulates the main results of
this section:
$$
\xymatrix @C=6pc{
                                           & \travset(M)^{-@} \ar@/_/[dl]_{+@}  \ar[r]^{\varphi_M}_\cong & \intersem{M} \ar@/_/[dd]_{\_ \upharpoonright \sem{\Gamma\rightarrow T}} \\
\travset(M) \ar@/_/[ur]_{-@}^{} \ar[dr]^{\_ \upharpoonright r}  \\
                                           & \travset(M)^{\upharpoonright r} \ar[r]^{\varphi_M}_\cong & \sem{M} \ar@/_/[uu]^{\cong}_{\mbox{full uncovering}}
}
$$


\begin{example}
Take $M = \lambda f z . (\lambda g x . f x) (\lambda y. y) (f z) :
((o,o),o, o)$.  The figure below represents the computation tree
(left tree), the arena $\sem{((o,o),o, o)}$ (right tree) and
$\psi_M$ (dashed line). (Only question moves are shown for clarity.)
The justified sequence of nodes $t$ defined hereunder is an example
of traversal:

\begin{tabular}{lp{6.3cm}}
$\tree[levelsep=2.5ex,treesep=0.3cm]{ \Rnode{root}{\lambda f z} }
     {  \tree{@}
        {   \tree{\lambda g x}{
                  \tree{\Rnode{f}{f^{[1]}}}{
                            \tree{\Rnode{lmd}{\lambda^{[2]}}}
                            {\TR{x}}
                  }
                }
            \tree{ \lambda y }{\TR{y}}
            \tree{\lambda ^{[3]}}{
                \tree{\Rnode{f2}{f^{[4]}}} {
                \tree{\Rnode{lmd2}{\lambda^{[5]}}}{\TR{\Rnode{z}{z}}}
                }
            }
        }
     }
\hspace{1cm}
  \tree[levelsep=8ex,treesep=0.3cm]{ \Rnode{q0}q^0 }
    {   \pstree[levelsep=4ex]{\TR{\Rnode{q1}{q^1}}}{\TR{\Rnode{q2}{q^2}}}
        \TR{\Rnode{q3}q^3}
        \TR{\Rnode{q4}q^4}
    }
\psset{nodesep=1pt,arrows=->,arrowsize=2pt 1,linestyle=dashed,linewidth=0.3pt}
\ncline{->}{root}{q0} \mput*{\psi_M}
\ncarc[arcangle=-25]{->}{z}{q3}
\ncarc[arcangle=10]{->}{f}{q1}
\ncarc[arcangle=10]{->}{lmd}{q2}
\ncline{->}{f2}{q1}
\ncline{->}{lmd2}{q2}$
\hspace{2cm}
&
\begin{asparablank}
  \item  \Pstr[0.8cm]{
t = (n){\lambda f z} \
(n2){@} \
(n3-n2,60){\lambda g x} \
(n4-n,45){f^{[1]}} \
(n5-n4,45){\lambda^{[2]}} \
(n6-n3,45){x} \
(n7-n2,35){\lambda^{[3]}} \
(n8-n,35){f^{[4]}} \
(n9-n8,45){\lambda^{[5]}} \
(n10-n,35){z}
}

\item \Pstr[0.9cm]{
t\upharpoonright r = (n){\lambda f z} \ (n4-n,50){f}^{[1]} \
(n5-n4,60){\lambda}^{[2]} \ (n8-n,45){f}^{[4]} \
(n9-n8,60){\lambda}^{[5]} \ (n10-n,40){z}}
\item
\Pstr[0.8cm]{ {\psi_M(t\upharpoonright r) =\ } (n){q^0}\
(n4-n,60){q^1}\ (n5-n4,60){q^2}\ (n8-n,45){q^1}\ (n9-n8,60){q^2}\
(n10-n,38){q^3} \in \sem{M}\ .}
\end{asparablank}
\end{tabular}
\end{example}


    \section{Extension to PCF and IA terms}
    \input{corresp_pcf_ia.texi}

    \section{Related works and conclusion}    \input{corresp_conclusion.texi}

\chapter{Game-Semantic Analysis of Safety via a Syntactic Argument}
    \label{chap:syntactic_gamesem}
    
\section{A game-semantic account of safety}
\label{sec:gamesemaccount} Our aim is to characterize safety by game
semantics. We shall assume that the reader is familiar with the
basics of game semantics; For an introduction, we recommend
\cite{abramsky:game-semantics-tutorial}. Recall that a
\emph{justified sequence} over an arena is an alternating sequence
of O-moves and P-moves such that every move $m$, except the opening
move, has a pointer to some earlier occurrence of the move $m_0$
such that $m_0$ enables $m$ in the arena. A \emph{play} is just a
justified sequence that satisfies Visibility and Well-Bracketing. A
basic result in game semantics is that $\lambda$-terms are denoted
by \emph{innocent strategies}, which are strategies that depend only
on the \emph{P-view} of a play. The main result
(Theorem~\ref{thm:safeincrejust}) of this section is that if a
$\lambda$-term is safe, then its game semantics (is an innocent
strategy that) is, what we call, \emph{P-incrementally justified}. In such a
strategy, pointers emanating from the P-moves of a play are uniquely
reconstructible from the underlying sequence of moves and pointers
from the O-moves therein: specifically a P-question always points to
the last pending O-question (in the P-view) of a greater order.

The proof of Theorem~\ref{thm:safeincrejust} depends on a
Correspondence Theorem (see the Appendix) that relates the strategy
denotation of a $\lambda$-term $M$ to the set of \emph{traversals}
over a souped-up abstract syntax tree of the $\eta$-long form of $M$.
In the language of game semantics, traversals are just (concrete
representations of) the \emph{uncovering} (in the sense of Hyland
and Ong \cite{hylandong_pcf}) of plays in the strategy denotation.

The useful transference technique between plays and traversals was
originally introduced by one of us \cite{OngLics2006} for studying
the decidability of monadic second-order theories of infinite structures generated by
higher-order grammars (in which the $\Sigma$-constants or terminal symbols are at most
order 1, and \emph{uninterpreted}).
% In this setting, free variables are interpreted
% as constructors and therefore they do not have the ``full power'' of
% true free variables and are limited to order $1$ at most. Also,
% although the grammar can perform higher-order computations, the
% structure being studied is itself of ground type.
In the Appendix, we present an extension of this framework to the
general case of the simply-typed lambda calculus with free variables
of any order. A new traversal rule is introduced to handle nodes
labelled with free variables. Also new nodes are added to the
computation tree to account for the answer moves of the game
semantics, thus enabling the framework to model languages with
interpreted constants such as \pcf~(by adding traversal rules to
handle constant nodes).

\subsection*{Incrementally-bound computation tree}
 In \cite{OngLics2006} the computation tree of a grammar is
defined as the unravelling of a finite graph representing the \emph{long
transform} of a grammar. Similarly we define the computation tree of
a $\lambda$-term as an abstract syntax tree of its $\eta$-long
normal form.  We write $l\langle t_1, \ldots, t_n \rangle$ with $n
\geq 0$ to denote the ordered tree with a root labelled $l$ with $n$
child-subtrees $t_1$, \ldots, $t_n$. In the following we consider arbitrary
simply-typed terms.

\begin{definition}\rm
\label{dfn:comptree}
  The \defname{computation tree} $\tau(M)$ of a simply-typed term
  $\Gamma \stentail M:T$ with variable names in a countable set
  $\mathcal{V}$ is a tree with labels in $$ \{ @ \} \union \mathcal{V}
  \union \{ \lambda x_1 \ldots x_n \ | \ x_1 ,\ldots, x_n \in
  \mathcal{V}, n\in\nat \}$$ defined from its $\eta$-long form as follows. Suppose $\overline{x} = x_1 \ldots x_n$ for $n\geq 0$ then
\begin{eqnarray*}
  \mbox{for $m\geq 0$, $z \in \mathcal{V}$: } \tau(\lambda \overline{x} . z s_1 \ldots s_m : o) &=& \lambda \overline{x} \langle z \langle\tau(s_1),\ldots,\tau(s_m)\rangle\rangle \\
  \mbox{for $m \geq 1$: } \tau(\lambda \overline{x} . (\lambda y.t) s_1 \ldots s_m :o) &=& \lambda \overline{x} \langle @ \langle \tau(\lambda y.t),\tau(s_1),\ldots,\tau(s_m) \rangle \rangle \ .
\end{eqnarray*}
\end{definition}

\begin{example}
\label{examp:comptree}
  Take $\stentail \lambda f^{o \typear o} .
(\lambda u^{o \typear o} . u) f : (o \typear o) \typear
o \typear o$.
\bigskip

\noindent
\begin{tabular}{cc}
Its $\eta$-long normal form is: & Its computation tree is:\\[8pt]
\begin{minipage}{0.45\textwidth}
\centering
$\begin{array}{ll}
 &\stentail  \lambda f^{o \typear o} z^o . \\
&\qquad(\lambda u^{o \typear o} v^o . u (\lambda.v)) \\
&\qquad(\lambda y^o. f y) \\
&\qquad(\lambda.z) \\
&: (o \typear o) \typear o \typear o
\end{array}$
\end{minipage}
&
\begin{minipage}{0.45\textwidth}
\centering
\psset{levelsep=5ex,linewidth=0.5pt,nodesep=1pt,arcangle=-20,arrowsize=2pt 1}
${\pstree{\TR{\lambda f z}}{\pstree{\TR{@}}{\pstree{\TR{\lambda u v}}{\pstree{\TR{u}}{\pstree{\TR{\lambda }}{\TR{v}}}}\pstree{\TR{\lambda y}}{\pstree{\TR{f}}{\pstree{\TR{\lambda }}{\TR{y}}}} \pstree{\TR{\lambda }}{\TR{z}}}}
}$
\end{minipage}
\end{tabular}
\end{example}

\begin{example}
  Take $\stentail \lambda u^o v^{((o \typear o) \typear o)} . (\lambda x^o . v (\lambda z^o . x)) u : o \typear ((o \typear o) \typear o) \typear o$.
  \bigskip

\noindent
\begin{tabular}{cc}
Its $\eta$-long normal form is: & Its computation tree is:\\[8pt]
\begin{minipage}{0.45\textwidth}
\centering
$\begin{array}{ll}
 &\stentail  \lambda u^o v^{((o \typear o) \typear o)} . \\
&\qquad(\lambda x^o . v (\lambda z^o . x)) u \\
&: o \typear ((o \typear o) \typear o) \typear o
\end{array}$
\end{minipage}
&
\begin{minipage}{0.45\textwidth}
\centering
$\pstree{\TR{\lambda u v}}{\pstree{\TR{@}}{\pstree{\TR{\lambda x}}{\pstree{\TR{v}}{\pstree{\TR{\lambda z}}{\TR{x}}}}\pstree{\TR{\lambda }}{\TR{u}}}}
$
\end{minipage}
\end{tabular}
\end{example}

Even-level nodes are $\lambda$-nodes (the root is on level 0). A
single $\lambda$-node can represent several consecutive variable
abstractions or it can just be a \emph{dummy lambda} if the
corresponding subterm is of ground type.  Odd-level nodes are
variable or application nodes.

The \defname{order} of a node $n$, written $\ord{n}$, is defined as
follows: @-nodes have order $0$. The order of a variable-node is the
type-order of the variable labelling it. The order of the root node
is the type-order of $(A_1,\ldots,A_p, T)$ where $A_1,\ldots, A_p$
are the types of the variables in the context $\Gamma$. Finally, the
order of a lambda node different from the root is the type-order of
the term represented by the sub-tree rooted at that node.

We say that a variable node $n$ labelled $x$ is \defname{bound} by a
node $m$, and $m$ is called the \defname{binder} of $n$, if $m$ is
the closest node in the path from $n$ to the root such that $m$ is
labelled $\lambda \overline{\xi}$ with $x\in \overline{\xi}$.


We introduce a class of computation trees in which the binder node
is uniquely determined by the nodes' orders:
\begin{definition}\rm
  A computation tree is \defname{incrementally-bound} if for all
  variable node $x$, either $x$ is \emph{bound} by the first
  $\lambda$-node in the path to the root with order $> \ord{x}$, or $x$
  is a \emph{free variable} and all the $\lambda$-nodes in the path to
  the root except the root have order $\leq \ord{x}$.
\end{definition}

\begin{proposition}[Safety and incremental-binding] \hfill
\label{prop:safe_imp_incrbound}
\begin{enumerate}[(i)]
\item If $M$ is safe then $\tau(M)$ is incrementally-bound.
\item Conversely, if $M$ is a \emph{closed} simply-typed term and $\tau(M)$
is incrementally-bound then $M$ is safe.
\end{enumerate}
\end{proposition}
\proof
  (i) Suppose that $M$ is safe. By Lemma
  \ref{prop:safe_iff_elnfsafe} the $\eta$-long form of $M$ is safe
  therefore $\tau(M)$ is the tree representation of a safe term.

In the safe lambda calculus, the variables in the context with the
lowest order must be all abstracted at once when using the
abstraction rule. Since the computation tree merges consecutive
abstractions into a single node, any variable $x$ occurring free in
the subtree rooted at a node $\lambda \overline{\xi}$ different from
the root must have order greater or equal to $\ord{\lambda
  \overline{\xi}}$. Conversely, if a lambda node $\lambda
\overline{\xi}$ binds a variable node $x$ then $\ord{\lambda
  \overline{\xi}} = 1+\max_{z\in\overline{\xi}} \ord{z} > \ord{x}$.

Let $x$ be a bound variable node. Its binder occurs in the path from
$x$ to the root, therefore, according to the previous observation,
$x$ must be bound by the first $\lambda$-node occurring in this path
with order $>\ord{x}$. Let $x$ be a free variable node then $x$ is
not bound by any of the $\lambda$-nodes occurring in the path to the
root. Once again, by the previous observation, all these
$\lambda$-nodes except the root have order smaller than $\ord{x}$.
Hence $\tau$ is incrementally-bound.

(ii) Let $M$ be a closed term such that $\tau(M)$ is
incrementally-bound.  W.l.o.g. we can assume that $M$ is in $\eta$-long
form.  We prove that $M$ is safe by induction on its structure. The
base case $M = \lambda \overline{\xi} . x$ for some variable $x$ is
trivial.  \emph{Step case:} If $M = \lambda \overline{\xi} . N_1
\ldots N_p$.  Let $i$ range over $1..p$. We have $N_i \equiv \lambda
\overline{\eta_i} . N'_i$ for some non-abstraction term $N'_i$. By
the induction hypothesis, $\lambda \overline{\xi} . N_i = \lambda
\overline{\xi} \overline{\eta_i} . N'_i$ is a safe closed term, and
consequently $N'_i$ is necessarily safe. Let $z$ be a free variable
of $N'_i$ not bound by $\lambda \overline{\eta_i}$ in $N_i$. Since
$\tau(M)$ is incrementally-bound we have $\ord{z} \geq \ord{\lambda
  \overline{\eta_1}} = \ord{N_i}$, thus we can abstract the variables $\overline{\eta_1}$ using \rulenamet{abs} which shows that $N_i$ is safe.  Finally
we conclude $\sentail M = \lambda \overline{\xi} . N_1 \ldots N_p :
T$ using the rules \rulenamet{app} and \rulenamet{abs}.  \qed



The assumption that $M$ is closed is necessary. For instance for
$x,y:o$, the computation trees $\tau(\lambda x y .x)$ and
$\tau(\lambda y . x)$ are both incrementally-bound but $\lambda x y
.x$ is safe and $\lambda y . x$ is not.

\subsection*{P-incrementally justified strategy}

We now consider the game-semantic model of the simply-typed lambda
calculus. The strategy denotation of a term-in-context $\Gamma
\stentail M : T$ is written $\sem{\Gamma
\stentail M : T}$. We define the \defname{order} of a move $m$,
written $\ord{m}$, to be the length of the path from $m$ to its
furthest leaf in the arena minus 1. (There are several ways to
define the order of a move; the definition chosen here is sound in
the current setting where each question move in the arena enables at
least one answer move.)
%{\it i.e.}~height of the subarena rooted at $q$ minus 2.

\begin{definition}\rm
  A strategy $\sigma$ is said to be \defname{P-incrementally
    justified} if for any play $s \, q \in \sigma$ where $q$ is a
  P-question, $q$ points to the last unanswered O-question in $\pview{s}$ with
  order strictly greater than $\ord{q}$.
\end{definition}
Note that although the pointer is determined by the P-view, the
choice of the move itself can be based on the whole history of the
play. Thus P-incremental justification does not imply innocence.

The definition suggests an algorithm that, given a play of a
P-incrementally justified denotation, uniquely recovers the pointers
from the underlying sequence of moves and from the pointers
associated to the O-moves therein. Hence:
\begin{lemma}
\label{lem:incrjustified_pointers_uniqu_recover} In P-incrementally
justified strategies, pointers emanating from P-moves are
superfluous.
\end{lemma}

\begin{example}
Copycat strategies, such as the identity strategy $id_A$ on game $A$
or the evaluation map $ev_{A,B}$ of type $(A \Rightarrow B) \times A
\typear B$, are all P-incrementally justified.\footnote{In such
strategies, a P-move $m$ is justified as follows: either $m$ points
to the preceding move in the P-view or the preceding move is of
smaller order and $m$ is justified by the second last O-move in the
P-view.}
\end{example}
%%%% the following example is wrong : ev is P-ij.
%
%\begin{example}
%Take the evaluation map $ev : (o^1 \Rightarrow o^2) \times o^3 \rightarrow o^4$ and the play $s = q^4 q^2 q^1 q^3 \in \sem{ev}$. We have $\ord{q^2} = 1 > \ord{q^1} = \ord{q^3} = 0$. Now $q^3$ points to $q^4$ but $q^2$ is the last unanswered O-question in $\pview{s}= s$ with order $>\ord{q^3}$, hence $\sem{ev}$ is not P-incrementally justified.
%\end{example}



The Correspondence Theorem~\ref{thm:correspondence}
% and Lemma \ref{lem:betanf_wellbehavedconst_trav_pview_red}
gives us the following equivalence:
\begin{proposition} % [Incremental-binding vs P-incremental justification]
\label{prop:Nher_incrbound_and_incrjustified} Let $\Gamma \stentail
M : T$ be a $\beta$-normal term. The computation tree $\tau(M)$ is
incrementally-bound if and only if $\sem{\Gamma \stentail M : T}$ is
P-incrementally justified.
\end{proposition}


\parpic[r]{
\pssetcomptree
\raisebox{-12pt}
{$\tree{\lambda^3}{\tree{f^2}{ \tree{\lambda y^1}{\TR{x^0} }}}$}
}
%\noindent \emph{Example:}
\begin{example}
Consider the $\beta$-normal term $\Gamma\stentail f (\lambda y .x) :
o$ where $y:o$ and $\Gamma =f:((o,o),o),~x:o$. The figure on the
right represents its computation tree with the node orders given as
superscripts.  The node $x$ is not incrementally-bound therefore $\tau(f
(\lambda y .x))$ is not incrementally-bound and by Proposition
\ref{prop:Nher_incrbound_and_incrjustified}, $\sem{\Gamma \stentail
f (\lambda y .x) : o}$ is not incrementally-justified (although
$\sem{\Gamma \stentail f : ((o,o),o)}$ and $\sem{\Gamma \stentail
\lambda
  y. x : (o,o)}$ are).
\end{example}
\smallskip

Propositions \ref{prop:safe_imp_incrbound} and
\ref{prop:Nher_incrbound_and_incrjustified} allow us to show the
following:
\begin{theorem}[Safety and P-incremental justification]
\label{thm:safeincrejust} \hfill
\begin{enumerate}[(i)]
\item If $\Gamma \sentail M : T$ then $\sem{\Gamma \sentail M : T}$ is P-incrementally justified.
\item If $\stentail M : T$ is a closed simply-typed term and $\sem{\stentail M : T}$ is P-incrementally justified then the $\beta$-normal form of $M$ is safe.
\end{enumerate}
\end{theorem}
\proof (i) Let $M$ be a safe simply-typed term. By Lemma
\ref{lem:safered_preserves_safety}, its $\beta$-normal form $M'$ is
also safe. By Proposition \ref{prop:safe_imp_incrbound}(i),
$\tau(M')$ is incrementally-bound and by Proposition
\ref{prop:Nher_incrbound_and_incrjustified}, $\sem{M'}$ is an
incrementally-justified. Finally the soundness of the game model
gives $\sem{M} = \sem{M'}$.  (ii) is a consequence of Lemma
\ref{lem:safered_preserves_safety}, Proposition
\ref{prop:Nher_incrbound_and_incrjustified} and
\ref{prop:safe_imp_incrbound}(ii) and soundness of the game model.
\qed



Putting Theorem \ref{thm:safeincrejust}(i) and Lemma
\ref{lem:incrjustified_pointers_uniqu_recover} together gives:
\begin{proposition}
  \label{prop:safe_ptr_recoverable} In the game semantics of safe
  $\lambda$-terms, pointers emanating from P-moves are unnecessary
  {\it i.e.}~they are uniquely recoverable from the underlying sequences of
  moves and from O-moves' pointers.
\end{proposition}

 \begin{example} If justification pointers are omitted then the denotations of the
   two Kierstead terms from Example~\ref{ex:kierstead} are not distinguishable.
   In the safe lambda calculus this ambiguity disappears
   since $M_1$ is safe whereas $M_2$ is not.
 \end{example}

In fact, as the last example highlights, pointers are superfluous at
order $3$ for safe terms whether from P-moves or O-moves. This is
because for question moves in the first two levels of an arena
(initial moves being at level $0$), the associated pointers are
uniquely recoverable thanks to the visibility condition. At the
third level, the question moves are all P-moves therefore their
associated pointers are uniquely recoverable by P-incremental
justification. This is not true anymore at order $4$: Take the safe
term $\psi:(((o^4,o^3),o^2),o^1) \sentail \psi (\lambda \varphi .
\varphi a) : o^0$ for some constant $a:o$, where $\varphi:(o,o)$.
Its strategy denotation contains plays whose underlying sequence of
moves is $q_0 \, q_1 \, q_2 \, q_3 \, q_2 \, q_3 \, q_4$. Since
$q_4$ is an O-move, it is not constrained by P-incremental
justification and thus it can point to any of the two occurrences of
$q_3$.\footnote{More generally, a P-incrementally justified strategy
can contain plays that are not ``O-incrementally justified'' since
it must take into account any possible strategy incarnating its
context, including those that are not P-incrementally justified. For
instance in the given example, there is one version of the play that
is not O-incrementally justified (the one where $q_4$ points to the
first occurrence of $q_3$). This play is involved in the strategy
composition $\sem{ \stentail M_2 : (((o,o),o),o)} ; \sem{
\psi:(((o,o),o),o) \stentail \psi (\lambda \varphi . \varphi a):o}$
where $M_2$ denotes the unsafe Kierstead term.}


\subsection*{Towards a fully abstract game model}\hfill

The standard game models which have been shown to be fully abstract
for PCF \cite{abramsky94full,hylandong_pcf} are of course also fully
abstract for the restricted language safe PCF. One may ask, however,
whether there exists a fully abstract model with respect to safe
context only.

Such model may be obtain by considering P-incrementally justified strategies
- which have been shown to compose in \cite{Blumphd}. Its is reasonable to think that
 O-moves also needs to be constrained by the symmetrical O-incremental justification, which corresponds to the requirement that contexts are safe. This line of work is still in progress.


\subsection*{Safe PCF and safe Idealised Algol}

\pcf\ is the simply-typed lambda calculus augmented with basic
arithmetic operators, if-then-else branching and a family of
recursion combinator $Y_A : ((A,A),A)$ for any type $A$.  We define
\emph{safe} \pcf\ to be \pcf\ where the application and abstraction
rules are constrained in the same way as the safe lambda calculus.
This language inherits the good properties of the safe lambda
calculus: No variable capture occurs when performing substitution
and safety is preserved by the reduction rules of the small-step
semantics of \pcf.

\subsubsection{Correspondence}

The computation tree of a \pcf\ term is defined as the least
upper-bound of the chain of computation trees of its \emph{syntactic
approximants} \cite{abramsky:game-semantics-tutorial}.  It is
obtained by infinitely expanding the $Y$ combinator, for instance
$\tau(Y (\lambda f x. f x))$ is the tree representation of the
$\eta$-long form of the infinite term $(\lambda f x. f x)
 ((\lambda f x. f x) ((\lambda f x. f x) ( \ldots$

It is straightforward to define the traversal rules modeling the
arithmetic constants of \pcf. Just as in the safe lambda calculus we
had to remove @-nodes in order to reveal the game-semantic
correspondence, in safe \pcf\ it is necessary to filter out the
constant nodes from the traversals. The Correspondence Theorem for
\pcf\ says that the interaction game semantics is isomorphic to the
set of traversals disposed of these superfluous nodes. This can
easily be shown for term approximants. It is then lifted to full
\pcf\ using the continuity of the function $\travset(\_)^{\filter
\theroot}$ from the set of computation trees (ordered by the
approximation ordering) to the set of sets of justified sequences of
nodes (ordered by subset inclusion). Finally computation trees of
safe \pcf\ terms are incrementally-bound thus we have
%Computation trees of safe \pcf\ terms are incrementally-bound.
%Moreover since \pcf\ constant are of order $1$ at most, the constant
%traversal rules are all \emph{well-behaved} (Lemma
%\ref{lem:sigma_order1_are_wellbehaved}) hence Lemma
%\ref{lem:betanf_wellbehavedconst_trav_pview_red} (from the Appendix)
%still holds and the game-semantic analysis of safety remains valid
%for \pcf. Hence we have:
\begin{theorem}
\label{thm:safepcfpincr} Safe PCF terms have P-incrementally
justified denotations. \qed
\end{theorem}


Similarly, we can define safe \ialgol\ to be safe \pcf\ augmented
with the imperative features of Idealized Algol (\ialgol\ for short)
\cite{Reynolds81}.  Adapting the game-semantic correspondence and
safety characterization to \ialgol\ seems feasible although the
presence of the base type \iavar, whose game arena $\iacom^{\nat}
\times \iaexp$ has infinitely many initial moves, causes a mismatch
between the simple tree representation of the term and its game
arena. It may be possible to overcome this problem by replacing the
notion of computation tree by a ``computation directed acyclic
graph''.

The possibility of representing plays \emph{without some or all of
  their pointers} under the safety assumption suggests potential
applications in algorithmic game semantics. Ghica and McCusker
\cite{ghicamccusker00} were the first to observe that pointers are
unnecessary for representing plays in the game semantics of the
second-order finitary fragment of Idealized Algol ($\ialgol_2$ for
short). Consequently observational equivalence for this fragment can
be reduced to the problem of equivalence of regular expressions.  At
order $3$, although pointers are necessary, deciding observational
equivalence of $\ialgol_3$ is EXPTIME-complete
\cite{DBLP:journals/apal/Ong04,DBLP:conf/fossacs/MurawskiW05}.
Restricting the problem to the safe fragment of $\ialgol_3$ may lead
to a lower complexity.

% (note that it is unlikely to obtain the complexity PSPACE because the
% set of complete plays of the safe term $\lambda f^{(o,o),o} . f
% (\lambda x^o . x)$ is not regular \cite{DBLP:journals/apal/Ong04}).

% Murawski showed the undecidability of program equivalence in
% $\ialgol_i$ for $i\geq4$ by encoding Turing machine computations
% into a finitary $IA_4$ term \cite{murawski03program}. The term
% constructed being not safe, the proof cannot be transposed to the
% safe fragments. Hence the question remains of whether observational
% equivalence is decidable for the \emph{safe} fragments of these
% language.

%In \cite{Ong02}, one of us showed that observational equivalence for
% finitary second-order \ialgol\ with recursion ($\ialgol_2 + Y_1$) is
% undecidable. The proof consists in reducing the Queue-Halting
% problem to the observational equivalence of two $\ialgol_2 + Y_1$
% terms. The same reduction is still valid in the safe fragment of
% $\ialgol_2 + Y_1$.  Consequently, observational equivalence of safe
% $\ialgol_2 + Y_1$ is also undecidable.

    \clearpage
    \notetoself{This chapter is taken from my transfer report. I need to
rework it to integrate it correctly within the present thesis.}


Safety has been defined as a syntactical constraint. Since Game
Semantics is by essence syntax-independent, it seems difficult at
first sight to give a game-semantic characterization of a syntactic restriction such as the Safety Condition.
In fact, the Correspondence Theorem makes such analysis possible since it allows us to regard the plays of a strategy
as sequences of nodes of some AST of the term.


The main theorem of this chapter (theorem
\ref{thm:safe_ptr_recoverable}) states that pointers in a play of
the strategy denotation of a safe term can be uniquely recovered
from O-questions' pointers and from the underlying sequence of
moves. The proof is in several steps. We start by introducing the
notion of \emph{P-incrementally-justified strategies} and prove that
for plays of such strategies, pointers emanating from P-moves can be
reconstructed uniquely from the underlying sequences of moves and
from O-moves' pointers. We then introduce the notion of
\emph{incrementally-bound computation trees} and prove that
incremental-binding coincides with P-incremental-justification
(proposition \ref{prop:Nher_incrbound_iff_incrjustified}).


Finally, we show that safe simply-typed terms in $\beta$-normal form
have incrementally\--bound computation trees, consequently their
game denotation is P-incrementally-justified.


The first section of this chapter is concerned only with the safe $\lambda$-calculus without interpreted constants. In the next
section we extend the result by taking into account the interpreted
constants of \pcf\ and \ialgol. We define the language safe \ialgol\
(resp. safe \pcf) to be the fragment of \ialgol\ (resp. \pcf) where
the application and abstraction rules are constrained the same way
as in the safe $\lambda$-calculus. We show that safe \pcf\ terms are
denoted by P-incrementally-justified strategies and we give the key
elements for a possible extension of the result to Safe Idealized
Algol.

\section{Preliminaries}

In this section, we assume that we work in a general setting of a
language extending the simply-typed lambda calculus with new
constants and respecting the following prerequisites:
\begin{itemize}
\item A fully-abstract game-semantic model of the language is
defined;
\item A notion of safety is defined for the language such that the
restriction of the language to the safe pure simply-typed
fragment coincides with the definition of the Safe Lambda
Calculus and such that for any typable term $\Gamma \vdash M :
T$ we have $\forall z \in \Gamma . \ord{z} \geq \ord{T}$ ;
\item The small-step reduction semantics of the language preserves safety;
%\item Substitution preserves safety.
\item New traversal rules are defined to take into account the constants of the language.
\item Constant traversal rules are well-behaved (see Def.\
\ref{def:wellbehaved_traversal});
\item Constant traversal rules correctly model the behaviour of the constants in such a way
that the game-semantic correspondence (Theorem
\ref{thm:correspondence}) still holds.
\end{itemize}

The simply-typed lambda calculus is of course such a language, but
we will show that \pcf\ also lends itself into this setting.

For the rest of this section we fix a term $\Gamma \vdash M : T$
from this generic language. We will explicitly specify when a result
holds only in the pure (\ie no constants) simply-typed calculus
fragment of the language.

\subsection{Incremental binding}

In a computation tree, a binder node always occurs in the path from
the bound node to the root. We now introduce a class of computation
trees in which binder nodes can be uniquely recovered from the order
of the nodes. We call path any sequence of nodes such that for any
two consecutive nodes $a \cdot b$ in the sequence, $a$ is the parent
of $b$. We write $[n_1,n_2]$ to denote the path going from node
$n_1$ to node $n_2$ equipped with the justification pointers induced
by the enabling relation $\vdash$ (each node of the tree has a
unique enabler in the path to the root thus for each occurrence in
$[n_1,n_2]$ there is at most one occurrence of its enabler in
$[n_1,n_2]$). We write $]n_1,n_2]$ for the sub-sequence of
$[n_1,n_2]$ obtained by removing $n_1$ as welle as all the
associated pointers.

We recall that $\theroot$ denotes the root of the computation tree
$\tau(M)$ and $N^{\theroot\vdash}$ denotes the subset of $N$
consisting of nodes that are hereditarily enabled by $\theroot$.



\begin{definition}[Incrementally-bound computation tree]
Let $A$ be a subset of nodes of the computation tree. A variable
node $x$ of a computation tree is said to be
\defname{$A$-incrementally-bound} if its enabler is the first
$\lambda$-node from $A$ in the path to the root that has order
strictly greater than $\ord{x}$. Formally:
\begin{align*}
x \mbox{ is $A$-incrementally-bound} \  \iff \  \left\{
                                                  \begin{array}{ll}
                                                    x \hbox{ is enabled by } b \in [\theroot,x]\inter A \ ; \\
                                                    \ord{b} > \ord{x} \;\\
                                                    \forall \lambda\mbox{-node } n' \in ]n,x]\inter A  . \ord{n'} \leq \ord{x} \ .
                                                  \end{array}
                                                \right.
\end{align*}

This definition can be split into two cases:
\begin{enumerate}
\item $x$ is \emph{bound} by the first $\lambda$-node from $A$ occurring in the path to the root that has
order strictly greater than $\ord{x}$.
\item or $x$ is a \emph{free variable} and all the $\lambda$-nodes from from $A$ occurring in the path to the root except the root have order
 smaller or equal to $\ord{x}$.
\end{enumerate}

A computation tree is said to be \defname{$A$-incrementally-bound},
also abbreviated $A$-i.b., if all the variable nodes from $A$ are
$A$-incrementally-bound.

We say that a node (resp.\ a tree) is
\defname{incrementally-bound} if it is
\defname{$N$-incrementally-bound} where $N$ is the entire set of nodes of the computation tree.
\end{definition}

Clearly for any two sets of nodes $A$ and $B$ verifying $A\subseteq
B$ we have that $B$-incremental-binding implies
$A$-incremental-binding.


\smallskip

Let $\closure{M}$ denote the function that converts $M$ into the
closed term obtained from $M$ by abstracting all its free variables
(in order of appearance in the term). From the previous definition,
if $\tau(M)$ is $A$-i.b.\ then so is $\tau(\closure{M})$.

\smallskip

A node of the computation tree is said to be \defname{reachable} if
there is some traversal of the computation tree that visits it.


\begin{lemma}[Safe terms have incrementally-bound computation trees]
\label{lem:incrbound_iff_etanf_safe} Suppose that  $\Gamma \vdash M
:T$ is a simply-typed term.
\begin{itemize}
\item[(i)] If $M$ is a safe term then $\tau(M)$ is incrementally-bound ;
\item[(ii)] conversely, if $M$ is \emph{closed} and $\tau(M)$ is i.b.\ then the $\eta$-long normal form of $M$ is safe.
\end{itemize}
\end{lemma}
\begin{proof}
(i) Suppose that $M$ is safe. The safety property is preserved after
taking the $\eta$-long normal form, therefore $\tau(M)$ is the tree
representation of a safe term.

In the safe $\lambda$-calculus, the variables in the context with
the the lowest order must be all abstracted at once when using the
abstraction rule. Since the computation tree merges consecutive
abstractions into a single node, any variable $x$ occurring free in
the subtree rooted at a $\lambda$-node $\lambda \overline{\xi}$
different from the root must have order greater or equal to
$\ord{\lambda \overline{\xi}}$. Reciprocally, if a lambda node
$\lambda \overline{\xi}$ binds a variable node $x$ then
$\ord{\lambda \overline{\xi}} = 1+\max_{z\in\overline{\xi}} \ord{z}
> \ord{x}$.

Let $x$ be a bound variable node. Its binder occurs in the path from
$x$ to the root, therefore, according to the previous observation,
$x$ must be bound by the first $\lambda$-node occurring in $[r,x]$
with order strictly greater than $\ord{x}$. Let $x$ be a free
variable node then $x$ is not bound by any of the $\lambda$-nodes
occurring in $[\theroot,x]$. Once again, by the previous
observation, all these $\lambda$-nodes except $\theroot$ have order
smaller than $\ord{x}$. Hence $\tau$ is incrementally-bound.

(ii) Let $M$ be a closed term such that $\tau(M)$ is
incrementally-bound. We assume that $M$ is already in $\eta$-normal
form. We prove that $M$ is safe by induction on its structure. The
base case $M = \lambda \overline{\xi} . \alpha$ for some variable or
constant $\alpha$ is trivial. \emph{Step case:} If $M = \lambda
\overline{\xi} . N_1 \ldots N_p$. Let $i$ range over $1..p$. $N_i$
can be written $\lambda \overline{\eta_i} . N'_i$ where $N'_i$ is
not an abstraction. By the induction hypothesis, $\lambda
\overline{\xi} . N_i = \lambda \overline{\xi} \overline{\eta_i} .
N'_i$ is safe. Hence $\vdash \lambda \overline{\xi}
\overline{\eta_i} . N'_i$ is a valid judgment of safe
$\lambda$-calculus. But this judgment can only be derived using the
\rulenamet{Abs} rule on the term $N'_i$. Hence $N'_i$ is necessarily
safe. Let $z$ be a variable occurring free in $N'_i$. Since $M$ is
closed, $z$ is either bound by $\lambda \overline{\eta_1}$ or
$\lambda \overline{\xi}$. In the latter case, since $\tau(M)$ is
i.b., $\ord{z}$ is smaller than $\ord{\lambda
\overline{\eta_1}}=\ord{N_i}$ thus in both case we are allowed to
abstract the variables $\overline{\eta_1}$ using the rule
\rulenamet{Abs}. This shows that $N_i$ is safe.

Each of the $N_i$s is safe and $N_1 \ldots N_p$ is of type $o$
therefore by the rule \rulenamet{App} rule we have $\overline{\xi}
\vdash N_1 \ldots N_p$. Finally, \rulenamet{Abs} gives us the
judgement $\vdash M = \lambda \overline{\xi} . N_1 \ldots N_p$.
\end{proof}

Note that the hypothesis that $M$ is closed in (ii) is necessary.
For instance, the two terms $\lambda x y .x$ and $\lambda y . x$,
where $x,y:o$, have (isomorphic) incrementally-bound computation
trees. However $\lambda x y .x$ is safe whereas $\lambda y . x$ is
not.

\begin{corollary}
\label{cor:betared_preserve_incrbound} Suppose $M$ is a closed term
in $\eta$-long normal form. If $\tau(M)$ is incrementally-bound and
$M \betared N$ then $\tau(N)$ is incrementally-bound.
\end{corollary}
\proof Suppose that $\tau(M)$ is i.b. Then by Lemma
\ref{lem:incrbound_iff_etanf_safe}(ii), $M$ is safe and since safety
is preserved by $\beta$-reduction, so is $N$. Thus by Lemma
\ref{lem:incrbound_iff_etanf_safe}(i), $\tau(N)$ is
incrementally-bound. \qed
\smallskip

Note that this corollary  cannot be generalized to
$A$-incremental-binding for any set of node $A$. Take for instance
the eta-normal term $M = \lambda u^{o} v^{((o,o),o)} . (\lambda x^o
. v (\lambda z^o . x)) u$ which beta-reduces to $N = \lambda u v . v
(\lambda z . u)$. The computation trees are:
$$\pssetcomptree
\tau(M) = \pstree{\TR{\underline{\lambda u v}}}{\pstree{\TR{@}}{\pstree{\TR{\lambda
x}}{\pstree{\TR{\underline{v}}}{\pstree{\TR{\underline{\lambda
z}}}{\TR{x}}}}\pstree{\TR{\lambda }}{\TR{\underline{u}}}}} \hspace{2cm}
\tau(N) = \pstree{\TR{\underline{\lambda u v}}}{\pstree{\TR{\underline{v}}}{\pstree{\TR{\underline{\lambda z}}}{\TR{\underline{u}}}}}
$$
Take $A$ to be the set of nodes that are hereditarily justified by
the root (the nodes underlined in the above figure). Then $\tau(M)$
is $A$-incrementally-bound but $\tau(N)$ is not.


\subsection{P-incremental-justified strategies}
\begin{definition}[P-incremental-justification]
A strategy $\sigma$ on a game $A$ is
\emph{P-incrementally\-justified} if and only if for any sequence of
moves $s q \in P_A$ we have:
\begin{eqnarray*}
s q \in \sigma \wedge q \mbox{ is a P-question } &\implies&
\parbox[t]{9cm}{$q$  points to the last O-move in $\pview{s}$
with order strictly greater than $\ord{q}$.}
\end{eqnarray*}
\end{definition}




\begin{lemma}
\label{lem:incrjustified_pointers_uniqu_recover} Pointers emanating
from P-moves are superfluous for P-incrementally-justified
strategies.
\end{lemma}
\begin{proof}
Suppose $\sigma$ is a P-incrementally-justified strategy. We prove
that pointers attached to P-moves in a play $s\in \sigma$ are
uniquely recoverable by induction on the length of $s$. \noindent
\emph{Base case}: if $|s| \leq 1$ then there is no pointer to
recover. \noindent \emph{Step case}: suppose $s m \in \sigma$. If
$m$ is an answer move then by the well-bracketing condition $m$
points to the last unanswered question in $s$. If $m$ is a
P-question then by  P-incremental-justification of $\sigma$, $m$
points to the last O-move in $\pview{s}$ with order strictly greater
than $\ord{q}$. Since we have access to O-moves' pointers, we can
compute the P-view $\pview{s}$. Hence $m$'s pointer is uniquely
recoverable.
\end{proof}

%\begin{example}
%The denotation of the evaluation map $ev$ is
%P-incrementally-justified since it is the uncurrying of the identity
% map on the game A=>B.
%\end{example}



\begin{proposition}[Incremental-binding and P-incremental-justification]
\hfill

 \label{prop:Nher_incrbound_iff_incrjustified}

\begin{enumerate}[(i)]
\item Suppose $M$ is $\beta$-normal. Then if all the \emph{reachable} input-variable nodes of the computation tree
$\tau(\Gamma \vdash M : T)$ are
$N^{\theroot\vdash}$-incrementally-bound then $\sem{\Gamma
\vdash M : T}$ is P-incrementally-justified.

\item If $\sem{\Gamma \vdash M : T}$ is
P-incrementally-justified then all the \emph{reachable}
input-variable nodes of the computation tree $\tau(\Gamma \vdash
M : T)$ are $N^{\theroot\vdash}$-incrementally-bound.
\end{enumerate}
\end{proposition}

\begin{proof}
\noindent (i) Suppose that $\tau(M)$ is
$N^{\theroot\vdash}$-incrementally-bound, then so is
$\tau(\etalnf{\closure{M}})$. Thus by Corollary
\ref{cor:betared_preserve_incrbound} $\etalnf{\closure{M}}$ is safe
and since safety is preserved by $\beta$-reduction, so is its
beta-normal form. Thus by Lemma
\ref{lem:incrbound_iff_etanf_safe}(i),
$\tau(\betanf{\etalnf{\closure{M}}})$ is incrementally-bound. Hence
we can assume without loss of generality that $M$ is a closed term
in beta-normal form and prove that $\sem{M}$ is
P-incrementally-justified (This will imply that
$\sem{\betanf{\etalnf{\closure{M}}}}$ is P-i.j.\ since the two game
denotations are isomorphic).

Take a play $s \in \sem{\Gamma \vdash M : T}$ ending with a question
P-move $q$. By the Correspondence Theorem \ref{thm:correspondence},
there is a traversal $t$ of $\tau(M)$ starting with an occurrence
$r$ of the root $\theroot$ such that $\psi_M (t\filter r) = s$. We
assume $t$ to be the shortest such traversal, thus the last
occurrence of $t$ - let us name it $n$ - is hereditarily justified
by $r$ and is by definition an occurrence of a reachable node.
Moreover since $\psi_M$ maps $n$ to $q$, $n$ is necessarily an
occurrence of a variable node $x$. There are two cases:
\begin{itemize}
\item Suppose $x$ is bound variable. Let $m$ denote its justifier
in $t$ (which is an occurrence of $x$'s binder in $\tau(M)$). By
assumption $\tau(M)$ is $N^{\theroot\vdash}$-incrementally-bound
therefore since $n$ belongs to $N^{\theroot\vdash}$, $m$ must be
the last $\lambda$-node in $[\theroot,n]\ \inter
N^{\theroot\vdash}$ of order strictly greater than $\ord{n}$.

By the Path--P-view correspondence (Prop.\
\ref{prop:pviewtrav_is_path}) we have $[\theroot,n]\ \inter
N^{\theroot\vdash} = \pview{t} \filter r$. This is in turn is
equal to $\pview{?(t \filter r)}$ (by Lemma
\ref{lem:betanf_wellbehavedconst_trav_pview_red}, since $M$ is
in $\beta$-normal form).


By property \ref{proper:psi_properties} (iv), the P-view of
$?(s)$ and the P-view of $?(t \filter r)$ are computed similarly
and have the same pointers, therefore node $n$ and move $q$ both
point to the same position in the justified sequence
$\pview{?(t\filter r)}$ and $\pview{?(s)}$ respectively.
Moreover since $\psi_M$ maps nodes of a given order to moves of
the same order (property \ref{proper:psi_properties}) this means
that $q$ points to the last O-move in $\pview{?(s)}$ with order
$>\ord{q}$.

Finally Lemma \ref{lem:views_and_questionmarkfilter} gives us
$?(\pview{s}) = \pview{?(s)}$, and since $s$'s last move is a
question, $\pview{s}$ contains only question moves and therefore
$\pview{?(s)} = \pview{s}$. Thus $q$ points to the last O-move
in $\pview{s}$ with order is strictly greater than $\ord{q}$.


\item  Second case: $n$ is a free input-variable $x$.
Thus $n$ is justified by $r$, the first occurrence in $t$. By
definition of $\psi$, $x = \psi(n)$ must be a move enabled by
the initial move $q_0 = \psi(\theroot)$ in the arena
$\sem{\Gamma \rightarrow A}$, therefore we have $\ord{q_0} >
\ord{x}$. Furthermore since  $x$ is
$N^{\theroot\vdash}$-incrementally-bound all the $\lambda$-nodes
in $]\theroot,n]$ have order smaller than $\ord{n}$, thus by the
Correspondence Theorem, all the O-moves in $\pview{s}$ have
order smaller than $\ord{x}$.
\end{itemize}

\noindent (ii) Suppose $\sem{M}$ is P-incrementally-justified. Let
$x$ be a reachable input-variable node of $\tau(M)$: there exists a
traversal of the form $t \cdot x$ in $\travset(M)$ such that $x$ is
hereditarily justified by the first occurrence $r$ of $\tau(M)$'s
root in $t$.

The correspondence theorem tells us that $\varphi((t \cdot x)
\filter r) = \varphi((t \filter r) \cdot x)$ belongs to $\sem{M}$.
Since $\sem{M}$ is P-incrementally-justified, $\varphi(x)$ points to
the last O-move in $\pview{\varphi(t \filter r)}$ with order
strictly greater than $\ord{\varphi(x)}$. Consequently $x$ points to
the last $\lambda$-node in $\pview{t \filter r}$ with order strictly
greater than $\ord{x}$.

But by Lemma \ref{lem:pviewproj_wrt_theroot}, $\pview{t \filter r}$
contains $\pview{t} \filter r$ as a subsequence. Thus since by
P-visibility $m$ occurs in this subsequence, we have that $m$ is
also the last $\lambda$-node in $\pview{t} \filter r$ with order
strictly greater than $\ord{x}$. By the path-P-view correspondence
(Prop.\ \ref{prop:pviewtrav_is_path}) this can in turn be restated
as: $m$ is the last $\lambda$-node in $[\theroot,x[\  \inter\
N^{\theroot \vdash}$ with order strictly greater than $\ord{x}$.
Hence $\tau(M)$ is $N^{\vdash \theroot}$-incrementally-bound.
\end{proof}

\section{Safe $\lambda$-Calculus}

We now consider the special case of the Safe $\lambda$-Calculus
without interpreted constants. We show that pointers in the game
denotation of safe terms can be uniquely recovered. The example of
section \ref{subsec:pointer_necessary} gives a good intuition: in
order to distinguish the terms $M_1 = \lambda f . f (\lambda x . f
(\lambda y .y ))$ and $M_2 = \lambda f . f (\lambda x . f (\lambda y
.x ))$ it is necessary to keep pointers in the strategy plays. In
the Safe $\lambda$-Calculus, however, the ambiguity disappears since
$M_1$ is safe whereas $M_2$ is not (in the subterm $f (\lambda y .
x)$, the free variable $x$ has the same order as $y$ but it is not
abstracted together with $y$).



\begin{corollary}[of Proposition \ref{prop:Nher_incrbound_and_incrjustified}]
\label{cor:Nher_incrbound_iff_incrjustified}
  Suppose $\Gamma \vdash M : T$ is a pure (\ie with no interpreted constants) simply-typed term
  in $\beta$-normal form. Then $\sem{M}$ is P-incrementally-justified if and only if $\tau(M)$ is incrementally-bound.
\end{corollary}
\proof We first observe that all the variable nodes are
input-variable nodes. Indeed, let $x$ be a variable node of
$\tau(M)$. Since $M$ is $\beta$-normal, by lemma
\ref{lem:betanorm_enabling}, $x$ is either hereditarily enabled by
the root or by a constant in $N_\Sigma$. But the pure simply-typed
$\lambda$-calculus does not have constants thus $N_\Sigma =
\emptyset$ and $x$ is hereditarily enabled by the root, \ie it is an
input-variable node. Consequently, incremental-binding coincides
with $N^{\vdash \theroot}$-incremental-binding.

Furthermore, since all the input-variables are reachable, every node
of the computation tree can be reached by the traversal consisting
of the path from the root to that node, the \rulenamet{InputVar}
permitting us to visit the children of the input-variable nodes
occurring in the path.\qed
\smallskip

\parpic[r]{
    \pssetcomptree
     \tree[levelsep=4ex]{$\lambda x^3$}{\tree{$f^2$}{ \tree{$\lambda y^1$}{ \TR{$x^0$} }}}
} \noindent \emph{Examples:} Consider the $\beta$-normal term
$\lambda x . f (\lambda y .x)$ where $x,y:o$ and $f:(o,o),o$. The
figure on the right represents the computation tree with the order
of each node in the exponent part. Since node $x$ of order $0$ is
not bound by the order 1 node $\lambda y$, $\tau(M)$ is not
incrementally-bound and by proposition
\ref{prop:Nher_incrbound_and_incrjustified} $\sem{\lambda x . f
(\lambda y .x)}$ is not P-incrementally-justified. Similarly we can
check that $\sem{f (\lambda y .x)}$ is not P-incrementally-justified
whereas $\sem{\lambda y. x}$ is. Also, for any higher-order variable
$x:A$ the computation tree $\tau(x)$ is incrementally-bound
therefore the projection strategies $\pi_i$ are
P-incrementally-justified. From these examples we observe that
application does not preserve P-incremental-justification ($\sem{f}$
and $\sem{\lambda y. x}$ are P-incrementally-justified whereas
$\sem{f (\lambda y .x)}$ is not).
\smallskip

These examples suggest that P-incremental-justification is not a
compositional property. In Chapter \ref{chap:pincrjust} we will
identify a sufficient condition guaranteeing that the composition of
two P-incrementally-justified strategies gives a
P-incrementally-justified strategy. \smallskip


Putting Corollary \ref{cor:Nher_incrbound_iff_incrjustified} and
Lemma \ref{lem:incrbound_iff_etanf_safe} together gives us a
game-semantic characterization of safe terms:
\begin{corollary}[P-incrementally-justified strategies characterize safe closed $\eta\beta$-normal terms]
Let $\Gamma \vdash M : T$ be a simply-typed term (without
interpreted constants). Then:
$$ \sem{\Gamma \vdash M : T} \mbox{ is P-incrementally-justified if and only if $\etabetalnf{M}$ is safe,} $$
where $\etabetalnf{M}$ denotes the $\eta$-long normal form of the
$\beta$-normal form of $M$.
\end{corollary}



\begin{theorem}[P's pointers are superfluous for safe terms]
\label{thm:safe_ptr_recoverable} Pointers emanating from P-moves in the game semantics of
safe terms are uniquely recoverable.
\end{theorem}
\begin{proof}
Let $M$ be a safe simply-typed term. Then the $\beta$-normal form of
$M$ is also safe, thus by lemma \ref{lem:incrbound_iff_etanf_safe}
(i), $\tau(\betanf{M})$ is incrementally-bound and by proposition
\ref{prop:Nher_incrbound_and_incrjustified}, $\sem{\Gamma \vdash
\betanf{M} :T}$ is a P-incrementally-justified strategy. By lemma
\ref{lem:incrjustified_pointers_uniqu_recover}, P's pointers in
$\sem{\Gamma \vdash \betanf{M} :T}$ are uniquely recoverable.
Finally, the soundness of the game model gives $\sem{\Gamma \vdash
M:T} = \sem{\Gamma \vdash \betanf{M} : T}$.
\end{proof}


\section{Safe PCF and Safe Idealized Algol}

Safe Idealized Algol, or safe \ialgol\ for short, is Idealized Algol
where the application and abstraction rules are restricted the same
way as in the safe $\lambda$-calculus (see rules of section
\ref{sec:safe_nonhomog}).

The properties of the safe $\lambda$-calculus can be transposed
straightforwardly to safe \ialgol. In particular, it can be shown
that safety is preserved by $\beta$-reduction and that no variable
capture occurs when performing substitution on a safe term.

A natural question to ask is whether we can extend the result about
game semantics of safe $\lambda$-terms to safe \ialgol-terms. In
this section we lay out the key elements permitting to prove that
the pointers in the game semantics of safe IA terms can be recovered
uniquely.

Such result has potential applications in algorithmic game semantics.
For instance, by following the framework of \cite{ghicamccusker00},
it may be possible to give a characterisation of the game semantics
of some higher-order fragments of safe \ialgol\ using extended
regular expressions. Subsequently, this would lead to the
decidability of program equivalence for the considered fragment.


\subsection{Formation rules of Safe \ialgol}
We call safe \ialgol\ term any term that is typable within the
following system of formation rules:
$$ \rulename{var} \   \rulef{}{x : A\vdash x : A}
%\qquad  \rulename{const} \   \rulef{}{\vdash f : A} \quad f \in \Sigma
\qquad  \rulename{wk} \   \rulef{\Gamma \vdash M : A}{\Delta \vdash
M : A} \quad  \Gamma \subset \Delta$$

$$ \rulename{app} \  \rulef{\Gamma \vdash M : (A,\ldots,A_l,B)
                                        \qquad \Gamma \vdash N_1 : A_1
                                        \quad \ldots \quad \Gamma \vdash N_l : A_l  }
                                   {\Gamma  \vdash M N_1 \ldots N_l : B}
                                    \quad
\mbox{\fbox{$\forall y \in \Gamma : \ord{y} \geq \ord{B}$}}$$

$$ \rulename{abs} \   \rulef{\Gamma \union \overline{x} : \overline{A} \vdash M : B}
                                   {\Gamma  \vdash \lambda \overline{x} : \overline{A} . M : (\overline{A},B)} \quad
\mbox{\fbox{$\forall y \in \Gamma : \ord{y} \geq \ord{\overline{A},B}$}}$$

$$ \rulename{num} \rulef{}{\Gamma \vdash n :\texttt{exp}}
\qquad \rulename{succ} \rulef{\Gamma \vdash M:\texttt{exp} }{\Gamma
\vdash \texttt{succ}\ M:\texttt{exp}} \qquad \rulename{pred}
\rulef{\Gamma \vdash M:\texttt{exp} }{\Gamma \vdash \texttt{pred}\
M:\texttt{exp}}$$

$$
\rulename{cond} \rulef{\Gamma \vdash M : \texttt{exp} \qquad \Gamma
\vdash N_1 : \texttt{exp} \qquad \Gamma \vdash N_2 : \texttt{exp}
}{\Gamma \vdash \texttt{cond}\ M\ N_1\ N_2} \qquad  \rulename{rec}
\rulef{\Gamma \vdash M : A\rightarrow A }{ \Gamma \vdash Y_A M :
A}$$

$$ \rulename{seq} \rulef{\Gamma \vdash M : \texttt{com} \quad \Gamma \vdash N :A}
    {\Gamma \vdash \texttt{seq}_A \ M\ N\ : A} \quad A \in \{ \texttt{com}, \texttt{exp}\}$$

$$ \rulename{assign} \rulef{\Gamma \vdash M : \texttt{var} \quad \Gamma \vdash N : \texttt{exp}}
    {\Gamma \vdash \texttt{assign}\ M\ N\ : \texttt{com}}
\qquad
 \rulename{deref} \rulef{\Gamma \vdash M : \texttt{var}}
    {\Gamma \vdash \texttt{deref}\ M\ : \texttt{exp}}$$

$$ \rulename{new} \rulef{\Gamma, x : \texttt{var} \vdash M : A}
    {\Gamma \vdash \texttt{new } x \texttt{ in } M} \quad A \in \{ \texttt{com}, \texttt{exp}\}$$

$$ \rulename{mkvar} \rulef{\Gamma \vdash M_1 : \texttt{exp} \rightarrow \texttt{com} \quad \Gamma \vdash M_2 : \texttt{exp}}
    {\Gamma \vdash \texttt{mkvar } M_1\ M_2\ : \texttt{var}}$$

\subsection{Small-step semantics of Safe \ialgol}
In the first chapter we defined the operational semantics of
\ialgol\ using a big step semantics. The operational semantics of
\ialgol\ can be defined equivalently using a small-step semantics.
The reduction rules of the small-step semantics are of the form $s,e
\rightarrow s',e'$ where $s$ and $s'$ denotes the stores and $e$ and
$e'$ denotes \ialgol\ expressions.

Let us give the rules that tell how to reduce redexes:
\begin{itemize}
\item the reduction of safe-redex (relation $\beta_s$ from definition \ref{dfn:safereduction});
\item reduction rules for \pcf\ constants:
\begin{eqnarray*}
\pcfsucc\ n &\rightarrow& n+1 \\
\pcfpred\ n+1 &\rightarrow& n \\
\pcfpred\ 0 &\rightarrow& 0 \\
\pcfcond\ 0\ N_1 N_2 &\rightarrow& N_1 \\
\pcfcond\ n+1\ N_1 N_2 &\rightarrow& N_2 \\
Y\ M &\rightarrow& M (Y M)
\end{eqnarray*}
\item reduction rules for \ialgol\ constants:
\begin{eqnarray*}
\iaseq\ \iaskip\  M &\rightarrow& M \\
s, \ianewin{x}\ M &\rightarrow& (s|x\mapsto 0), M \\
s, \iaassign\ x\ n &\rightarrow& (s|x\mapsto n), \iaskip \\
s, \iaderef\ x &\rightarrow& s, s(x) \\
\iaassign\ (\iamkvar M N)\ n &\rightarrow& M n \\
\iaderef\ (\iamkvar M N) &\rightarrow& N
\end{eqnarray*}
\end{itemize}

Redex can also be reduced when they occur as subexpressions within a
larger expression. We make use of evaluation contexts to indicate
when such reduction can happen. Evaluation contexts are given by the
following grammar:
\begin{eqnarray*}
E[-] &::=& - |\ E N\ |\ \pcfsucc\ E\ |\ \pcfpred\ E\ |\ \pcfcond\ E\ N_1\ N_2\ |\ \\
&&    \iaseq\ E\ N\ |\ \iaderef\ E\ |\ \iaassign\ E\ n\ |\ \iaassign\ M\ E \ |\ \\
&&    \iamkvar\ M\ E\ |\ \iamkvar\ E\ M\ |\ \ianewin{x}\ E  .
\end{eqnarray*}

The small-step semantics is completed with following rule:
$$ \rulef{M \rightarrow N}{E[M] \rightarrow E[N]} $$

\begin{lemma}[Reduction preserves safety]
\label{lem:ia_safety_preserved} Let $M$ be a safe IA term. If
$M \rightarrow N$ then $N$ is also a safe term.
\end{lemma}
This can be proved easily by induction on the structure of M.


\subsection{Safe \pcf\ fragment}
In this section, we show how to extend the results obtained for the
safe $\lambda$-calculus to the \pcf\ fragment of safe \ialgol.

The $Y$ combinator needs a special treatment. In order to deal with
it, we follow the idea of \cite{abramsky:game-semantics-tutorial}:
we consider the sublanguage $\pcf_1$ of \pcf\ in which the only
allowed use of the $Y$ combinator is in terms of the form $Y(
\lambda x:A .x )$ for some type $A$. We will write $\Omega_A$ to
denote the non-terminating term $Y(\lambda x:A .x)$ for a given type
$A$.

We introduce the \emph{syntactic approximants} to $Y_A M$:
\begin{eqnarray*}
Y^0_A M &=& \Gamma \vdash \Omega_A : A\\
Y^{n+1}_A M &=& M( Y^n M )
\end{eqnarray*}
For any \pcf\ term $M$ and natural number $n$, we define $M_n$ to be
the $\pcf_1$ term obtained from $M$ by replacing each subterm of the
form $Y N$ with $Y^n N_n$. We have $\sem{M} = \Union_{n\in\omega}
\sem{M_n}$ (\cite{abramsky:game-semantics-tutorial}, lemma 16).


\subsubsection{Computation tree}

We would like to define a unique computation tree for terms that use
the $Y$ combinator.

Let us first define the computation tree for $\pcf_1$ terms. We
introduce a special $\Sigma$-constant $\bot$ representing the
non-terminating computation of ground type $\Omega_o$. Given any
type $A = (A_1, \ldots, A_n, o)$, the computation tree
$\tau(\Omega_A)$ is defined to be the tree representation of
$\lambda x_1:A_1 \ldots x_n:A_n . \bot$. The computation tree of a
$\pcf_1$ term is then computed inductively in the standard way.

We now introduce a partial order on the set of computation trees.

A \emph{tree} $t$ is a labelling function $t:T\rightarrow L$ where
$T$, called the domain of $t$ and written $dom(t)$, is a non-empty
prefix-closed subset of some free monoid $X^*$ and $L$ denotes the
set of possible labels. Intuitively, $T$ represents the structure of
the tree (the set of all paths) and $t$ is the labelling function
mapping paths to labels. Trees can be ordered using the
\emph{approximation ordering} defined in \cite{KNU02}, section 1: we
write $t' \sqsubseteq t$ if the tree $t'$ is obtained from $t$ by
replacing some of its subtrees by $\bot$. Formally:
$$t' \sqsubseteq t \quad \iff dom(t') \subseteq dom(t) \wedge \forall  w \in dom(t'). (t'(w) = t(w) \vee t'(w) = \bot).$$
The set of all trees together with the approximation ordering is a
complete partial order.

We now consider a strict subset of the set of all trees: the set of
computation trees. A computation tree is a tree which represents the
$\eta$-normal form of some (potentially infinite) \pcf\ term. In
other words a tree is a computation tree if it can be written
$\tau(M)$ for some infinite \pcf\ term $M$. The set $L$ of labels is
constituted of the $\Sigma$-constants, @, the special constant
$\bot$, variables and abstractions of any sequence of variables. We
will write $(CT, \sqsubseteq)$ to denote the set of computation
trees ordered by the approximation ordering $\sqsubseteq$ defined
above. $(CT, \sqsubseteq)$ is also a complete partial order.

It is easy to check that the sequence of computation trees
$(\tau(M_n))_{n\in\omega}$ is a chain. We can therefore define the
computation tree of a \pcf\ term $M$ to be the least upper-bound of
the chain of computation trees of its approximants:
$$\tau(M) = \Union_{n\in\omega}(\tau(M_n))_{n\in\omega}.$$

In other words, we construct the computation tree by expanding
infinitely any subterm of the form $Y M$. For instance consider the
term $M = Y (\lambda f x. f x)$ where $f:(o,o)$ and $x:o$. Its
computation tree $\tau(M)$, represented below, is a tree
representation of the $\eta$-normal form of the infinite term
$(\lambda f x. f x) ((\lambda f x. f x) ((\lambda f x. f x)  (
\ldots$.
$$\tau(M) = \pssetcomptree\tree{\lambda y}{
                \tree{@}{
                        \tree{\lambda f x} { \tree{f}{\tree{\lambda}{\TR{x}} }}
                        \TR{\tau(M)}
                        \tree{\lambda}{\TR{y}}
                }
            }
$$

The remaining operators of \ialgol\ are treated as standard
constants and the corresponding computation tree is constructed from
the $\eta$-normal form of the term in the standard way. For instance
the diagram below shows the computation tree for $\pcfcond\ b\ x\ y$
(left) and $\lambda x . 5$ (right):
$$
\pssetcomptree\tree{\lambda b x y}
     {  \tree{\pcfcond}
        {   \tree{\lambda} {\TR{b}}
            \tree{\lambda} {\TR{x}}
            \tree{\lambda} {\TR{y}}
        }
    }
\hspace{2cm} \tree{\lambda x}{  \TR{5} }
$$
The node labelled $5$ has, like any other node, children
value-leaves which are not represented on the diagram above for
simplicity.

\subsubsection{Traversal}

New traversal rules accompany the additional constants of \ialgol.
There is one additional rule for natural number constants:
\begin{itemize}
\item (Nat) If $t \cdot n$ is a traversal where $n$ denotes a node labelled with some numeral constant $i\in \nat$ then
            $\Pstr{t \cdot (n){n} \cdot (in-n){i_n}}$
            is also a traversal where $i_n$ denotes the value-leaf of $m$ corresponding to the value $i\in \nat$.
\end{itemize}

\noindent The traversals rules for \pcfpred\ and \pcfsucc\ are
defined similarly. For instance, the rules for \pcfsucc\ are:
\begin{itemize}
\item (Succ) If $t \cdot \pcfsucc$ is a traversal and $\lambda$ denotes the only child node of \pcfsucc\ then
$\Pstr{t \cdot (succ){\pcfsucc} \cdot (l-succ,35:1){\lambda}}$ is also a traversal.

\item (Succ') If
$\Pstr{ t_1 \cdot (succ){\pcfsucc} \cdot (l-succ,35:1){\lambda} \cdot t_2
\cdot (lv-l){i_{\lambda}}} $ is a traversal for some
$i \in \nat$ then $\Pstr{t_1 \cdot (succ){\pcfsucc} \cdot
(l-succ,35:1){\lambda} \cdot t_2 \cdot (lv-l){i_{\lambda}} \cdot
(succv-succ,25){(i+1)_{\pcfsucc}}}$ is also a traversal.
\end{itemize}

\noindent In the computation tree, nodes labelled with \pcfcond\
have three children nodes numbered from $1$ to $3$ corresponding to
the three parameters of the operator \pcfcond. The traversal rules
are:
\begin{itemize}
\item (Cond-If) If $t_1 \cdot \pcfcond$ is a traversal and $\lambda$ denotes the first child of \pcfcond\ then
$\Pstr{ t_1 \cdot (cond){{\pcfcond}} \cdot (l-cond,30:1){\lambda}}$
 is also a traversal.

\item (Cond-ThenElse) If
$\Pstr{t_1 \cdot (cond){\pcfcond} \cdot (l-cond,35:1){\lambda} \cdot t_2
\cdot (lv-l){i_{\lambda}}} $
then $\Pstr{t_1 \cdot
(cond){\pcfcond} \cdot (l-cond,35:1){\lambda} \cdot t_2 \cdot
(lv-l){i_{\lambda}} \cdot (condthenelse-cond,35:{2+[i>0]}){\lambda} }
$
is also a traversal.



\item (Cond') If
$\Pstr{t_1 \cdot (cond){\pcfcond} \cdot t_2 \cdot (l-cond,35:k){\lambda}
\cdot t_3 \cdot (lv-l){i_{\lambda}}}$
 for $k=2$ or $k=3$ then  $\Pstr{ t_1 \cdot
(cond){\pcfcond} \cdot t_2 \cdot (l-cond,35:k){\lambda} \cdot t_3
\cdot (lv-l){i_{\lambda}} \cdot (condv-cond,25){i_{\pcfcond}}}$
 is also a traversal.
\end{itemize}
It is easy to verify that these traversal rules are all
well-behaved. This completes the definition of traversal for the
\pcf\ subset of \ialgol.

\subsubsection{Interaction semantics}
We recall that the interaction semantics defined in section
\ref{sec:interaction_semantics} takes into account the constants
of the language. For any higher-order constant $f : (A_1,\ldots,A_p,B) \in \Sigma$, definition \ref{dfn:canonical_revealed_semantics} gives the  revealed strategy of a term of the form $\lambda \overline{\xi}. f N_1 \ldots
N_p$ as follows:
$$ \revsem{\lambda \overline{\xi}. f N_1 \ldots N_p} = \langle \revsem{N_1}, \ldots, \revsem{N_p} \rangle \fatsemi^{0..p-1} \sem{f}.$$
where $\sem{f}$ is the standard strategy denotation of the constant $f$.


\subsubsection{Removing $\Sigma$-nodes from the traversals}


\notetoself{Need to rework the following lemma}

\begin{lemma}[Projection lemma]
\label{lem:SIGMACONST:varphi_projection} Let $\Gamma \vdash M :T$ be
a term and $r$ be the root of $\tau(M)$. For any traversal $t$ of
the computation tree we have $ \varphi(\travset(M)^*) \filter
\sem{\Gamma \rightarrow T} = \varphi(\travset(M)^{\filter r}) $.
 Consequently,
$$\varphi(t^*) \filter \sem{\Gamma \rightarrow T} = \varphi(t\filter r).$$
\end{lemma}
\begin{proof}
    From the definition of $\varphi$, the nodes of the computation tree that $\varphi$ maps
    to moves in the arena $\sem{\Gamma \rightarrow T}$ are exactly the nodes that are hereditarily justified by $r$.
    The result follows from the fact that @-nodes, constant nodes and value-leaves of constant nodes
    are not hereditarily justified by the root.
\end{proof}


The following lemma is the counterpart of lemma
\ref{lem:varphiinjective} and it is proved identically.
\begin{lemma}[$\varphi$ is injective]
\label{lem:SIGMACONST:varphiinjective} $\varphi$ regarded as a
function defined on the set of sequences of nodes is injective in
the sense that for any two traversals $t_1$ and $t_2$:
\begin{itemize}
\item[(i)] if $\varphi (t_1^* ) = \varphi (t_2^* )$ then $t_1^* =t_2^*$;
\item[(ii)] if $\varphi (t_1 \filter r ) = \varphi (t_2 \filter r )$ then $t_1\filter r = t_2\filter r$.
\end{itemize}
\end{lemma}

\begin{corollary} \
\label{cor:SIGMACONST:varphi_bij}
\begin{itemize}
\item[(i)] $\varphi$ defines a bijection from $\travset(M)^*$
to $\varphi(\travset(M)^*)$;
\item[(ii)] $\varphi$ defines a bijection from $\travset(M)^{\filter r}$ to
$\varphi(\travset(M)^{\filter r})$.
\end{itemize}
\end{corollary}


\subsubsection{Correspondence theorem}
We would like to prove the counterpart of proposition
\ref{prop:rel_gamesem_trav} in the context of the simply-typed
$\lambda$-calculus \emph{with interpreted PCF constants}. The game
model of the language \pcf\ is given by the category $\mathcal{C}_b$
of well-bracketed strategies. Hence the well-bracketing assumption
stated at the beginning of section \ref{sec:gamesemcorresp} is
satisfied.

We first prove that $\travset(\_)^{\filter r}$ is continuous.
\begin{lemma}
\label{lem:travred_continuous} Let $(S,\subseteq)$ denote the set of
sets of justified sequences of nodes ordered by subset inclusion.
The function $\travset(\_)^{\filter r} : (CT,\sqsubseteq)
\rightarrow (S,\subseteq)$ is continuous.
\end{lemma}
\begin{proof} \
    \begin{description}
    \item[Monotonicity:] Let $T$ and $T'$ be two computation trees such that $T \sqsubseteq T'$
    and let $t$ be some traversal of $T$.
    Traversals ending with a node labelled $\bot$ are maximal therefore $\bot$ can only occur
    at the last position in a traversal. Let us prove the following two properties:
        \begin{itemize}
            \item[(i)]  If $t = t \cdot n$ with $n\neq \bot$ then $t$ is a traversal of $T'$;
            \item[(ii)] if $t= t_1 \cdot \bot$ then $t_1\in \travset(T')$.
        \end{itemize}

        (i) By induction on the length of $t$. It is trivial for the empty traversal.
            Suppose that $t = t_1 \cdot n$ is a traversal with $n \neq \bot$.
            By the induction hypothesis, $t_1$ is a traversal of $T'$.

            We observe that for all traversal rules, the traversal produced is of the form $t_1 \cdot n$ where
            $n$ is defined to be a child node or value-leaf of some node $m$ occurring in $t_1$.
            Moreover, the choice of the node $n$ only depends on the traversal $t_1$
            (for the constant rules, this is guaranteed by assumption (WB)).

            Since $T \sqsubseteq T'$, any node $m$ occurring in $t_1$ belongs
            to $T'$ and the children nodes and leaves of $m$ in $T$ also belong to the tree $T'$.
            Hence $n$ is also present in $T'$ and the rule used to produce the traversal $t$ of $T$
            can be used to produce the traversal $t$ of $T'$.

        (ii) $\bot$ can only occur at the last position in a traversal
        therefore $t_1$ does not end with $\bot$ and by (i) we have $t_1\in \travset(T')$.
\vspace{6pt}

        Hence we have:
        \begin{align*}
        \travset(T)^{\filter r} &= \{ t \filter r \ | \ t \in \travset(T)     \} \\
        & = \{ (t\cdot n) \filter r \ | \ t\cdot n \in \travset(T) \wedge n \neq \bot \}
            \union \{ (t \cdot \bot ) \filter r \ | \ t \cdot \bot \in \travset(T)  \} \\
\mbox{(by (i) and (ii))} \quad        & \subseteq  \{ (t\cdot n)
\filter r \ | \ t\cdot n \in \travset(T') \wedge n \neq \bot
\}
            \union \{ t \filter r \ | \ t \in \travset(T')  \} \\
        & = \travset(T')^{\filter r}
        \end{align*}

        \item[Continuity:] Let $t \in \travset \left( \Union_{n\in\omega} T_n \right)$.
        We write $t_i$ for the finite prefix of $t$ of length $i$.
        The set of traversals is prefix-closed therefore $t_i \in \travset \left( \Union_{n\in\omega} T_n \right)$ for any $i$.
        Since $t_i$ has finite length we have $t_i \in \travset(T_{j_i})$ for some $j_i \in \omega$.
        Therefore we have:
        \begin{align*}
          t \filter r &= (\bigvee_{i\in\omega} t_i ) \filter r   & (\mbox{the sequence $(t_i)_{i\in\omega}$ converges to $t$}) \\
          &= \Union_{i\in\omega} ( t_i \filter r )   & (\_ \filter r \mbox{ is continuous, lemma \ref{lem:projection_continuous}}) \\
          &\in \Union_{i\in\omega} \travset(T_{j_i})^{\filter r}   & (t_i \in \travset(T_{j_i})) \\
          &\subseteq \Union_{i\in\omega} \travset(T_i)^{\filter r}   & (\mbox{since } \{ j_i \sthat i \in \omega \} \subseteq \omega)
        \end{align*}

        Hence $\travset(\Union_{n\in\omega} T_n )^{\filter r} \subseteq \Union_{n\in\omega} \travset(T_n)^{\filter r}.$

    \end{description}
\end{proof}

\begin{proposition}
Let $\Gamma \vdash M : T$ be a PCF term and $r$ be the root of
$\tau(M)$. Then:
\begin{align*}
(i)  \quad\varphi_M(\travset(M)^*) = \revsem{M},  \\
(ii) \quad \varphi_M(\travset(M)^{\filter r}) = \sem{M}.
\end{align*}
\end{proposition}
\begin{proof}
We first prove the result for $\pcf_1$: (i) The proof is an
induction identical to the proof of proposition
\ref{prop:rel_gamesem_trav}. However we need to complete the case
analysis with the $\Sigma$-constant cases:
\begin{itemize}
\item The cases \pcfsucc, \pcfpred, \pcfcond\ and numeral constants are straightforward.

\item Suppose $M = \Omega_o$ then $\travset(\Omega_o) = \prefset ( \{ \lambda \cdot \bot \} )$ therefore
$\travset(\Omega_o)^{\filter r} = \prefset( \{ \lambda \} )$
and $\sem{\Omega_o} = \prefset( \{ q \})$ with $\varphi(\lambda) =
q$. Hence $\sem{\Omega_o} = \varphi
(\travset(\Omega_o)^{\filter r})$.
\end{itemize}
(ii) is a direct consequence of (i) and the Projection Lemma (Lemma
\ref{lem:SIGMACONST:varphi_projection}). \vspace{10pt}

\noindent We now extend the result to \pcf. Let $M$ be a \pcf\ term,
we have:
\begin{align*}
\sem{M} &= \Union_{n\in\omega} \sem{M_n} & (\mbox{\cite{abramsky:game-semantics-tutorial}, lemma 16})\\
&= \Union_{n\in\omega} \travset(\tau(M_n))^{\filter r} & (M_n \mbox{ is a $\pcf_1$ term}) \\
&= \travset(\Union_{n\in\omega} \tau(M_n) )^{\filter r} & (\mbox{by continuity of $\travset(\_)^{\filter r}$, lemma \ref{lem:travred_continuous}}) \\
&= \travset(\tau(M))^{\filter r} & (\mbox{by definition of } \tau(M)) \\
&= \travset(M)^{\filter r} & (\mbox{abbreviation}).
\end{align*}
\end{proof}

Hence by corollary \ref{cor:SIGMACONST:varphi_bij}, $\varphi$
defines a bijection from $\travset(M)^{\filter r}$ to
$\sem{M}$:
$$\varphi : \travset(M)^{\filter r} \stackrel{\cong}{\longrightarrow} \sem{M}.$$

\subsubsection{Example: \pcfsucc}

Consider the term $M = \pcfsucc\ 5$ whose computation tree is
represented below. The value-leaves are also represented on the
diagram, they are the vertices attached to their parent node with a
dashed line.
$$
\psmatrix[colsep=3ex,rowsep=2ex]
\lambda^0 \\
\pcfsucc & 0 & 1 & \ldots \\
\lambda^1 & 0 & 1 & \ldots \\
5 & 0 & 1 & \ldots \\
  & 0 & 1 & \ldots
\endpsmatrix
\ncline{1,1}{2,1} \ncline{2,1}{3,1} \ncline{3,1}{4,1}
\valueedge{1,1}{2,2} \valueedge{1,1}{2,3} \valueedge{1,1}{2,4}
\valueedge{2,1}{3,2} \valueedge{2,1}{3,3} \valueedge{2,1}{3,4}
\valueedge{3,1}{4,2} \valueedge{3,1}{4,3} \valueedge{3,1}{4,4}
\valueedge{4,1}{5,2} \valueedge{4,1}{5,3} \valueedge{4,1}{5,4}
$$

The following sequence of nodes is a traversal of $\tau(M)$:
$$ \Pstr[20pt]{ t = (l0){\lambda^0} \cdot (succ){\pcfsucc} \cdot (l1){\lambda^1} \cdot (c5){5} \cdot (v55-c5){5_5} \cdot (5l1-l1){5_{\lambda^1}} \cdot (6succ-succ){6_\pcfsucc} \cdot (6l0-l0,35){6_{\lambda^0}}}.
$$

The subsequences $t^*$ and $t \filter r$ are given by:
$$
\Pstr[17pt]{ t^* = (l0){\lambda^0} \cdot (l1-l0){\lambda^1} \cdot
(5l1-l1){5_{\lambda^1}} \cdot (6l0-l0){6_{\lambda^0}}.
\qquad  \mbox{ and } \qquad t
\filter r = (l0){\lambda^0} \cdot
(6l0-l0){6_{\lambda^0}}. }
$$
We have $\varphi(t^*) = q_0 \cdot q_5 \cdot 5_{q_5} \cdot 5_{q_0}$
and $\varphi(t\filter r) = q_0 \cdot 5_{q_0}$ where $q_0$
and $q_5$ denote the roots of two flat arenas over $\nat$. These two
sequences of moves correspond to some play of the interaction
semantics and the standard semantics respectively. The interaction
play is represented below:
$$\begin{array}{ccccc}
  \textbf{1} & \stackrel{5}{\multimap} & !\nat & \stackrel{\pcfsucc}{\multimap} & \nat \\
&&&&  \rnode{q0}{q_0} \\
&&  \rnode{q5}{q_5} \\
&&  \rnode{a5}{5_{q_5}} \\
&&&&  \rnode{a6}{6_{q_0}}
\end{array}
\nccurve[nodesep=2pt,ncurv=0.9,angleA=180,angleB=180]{->}{a5}{q5}
\nccurve[nodesep=2pt,ncurv=0.9,angleA=180,angleB=210]{->}{a6}{q0}
\ncarc[nodesep=2pt,ncurv=0.9,angleA=180,angleB=180]{->}{q5}{q0}
$$

\subsubsection{Another example : \pcfcond}

Consider the term $M = \lambda x y . \pcfcond\ 1\ x\ y$. Its
computation tree is represented below (without the value-leaves):
    $$ \pssetcomptree\tree{\lambda x y}
       {
          \tree{\pcfcond}
          {
            \tree{\lambda^1}{ \TR{1} }
            \tree{\lambda^2}{ \TR{x} }
            \tree{\lambda^3}{ \TR{y} }
          }
      }
    $$
For any value $v \in\mathcal{D}$ the following sequence of nodes is
a traversal of $\tau(M)$:
$$\Pstr[27pt]{ t = (lxy){\lambda x y} \cdot (cond){\pcfcond} \cdot (l1-cond){\lambda^1} \cdot (1){1} \cdot (v11-1){1_1}
    \cdot (l3){\lambda^3} \cdot (y-vxy){y} \cdot (vy-y){v_y}  \cdot (vl3-l3){v_{\lambda^3}} \cdot (vcond-cond,30){v_{\pcfcond}}
    \cdot (vlxy-lxy,30){v_{\lambda x y}}.
}
$$
The subsequences $t^*$ and $t \filter r$ are given by:
$$
\Pstr[17pt]{ t^* =  t = (lxy){\lambda x y} \cdot
        (l1-lxy){\lambda^1} \cdot
        (l3-lxy){\lambda^3} \cdot
        (y-vxy){y} \cdot
        (vy-y){v_y}  \cdot
        (vl3-l3){v_{\lambda^3}} \cdot
        (vlxy-lxy,35){v_{\lambda x y}}
\qquad  \mbox{ and } \qquad t \filter r =
(lxy){\lambda x y} \cdot (y-vxy){y} \cdot (vy-y){v_y}
\cdot (vlxy-lxy){v_{\lambda x y}}.
}
$$
The sequence of moves $\varphi(t^*)$ corresponds to some play of the
interaction semantics and the sequence $\varphi(t\filter r)$
is a play of the standard semantics obtained by hiding the internal
moves of $\varphi(t^*)$. The interaction play $\varphi(t^*)$ is
represented below:
$$\begin{array}{ccccccccccc}
!\nat & \otimes & !\nat & \stackrel{ \langle \sem{1}, \pi_1,
\pi_2\rangle }{\multimap} & !\nat & \otimes & !\nat & \otimes &
!\nat
& \stackrel{ \pcfcond}{\multimap} & \nat \\
&&&&&&&&&&  \rnode{q0}{q_0^{(\lambda x y)}} \\
&&&&  \rnode{qa}{q_a^{(\lambda^1)}} \\
&&&&  \rnode{1}{1} \\
&&&&&&  \rnode{qb}{q_b^{(\lambda^2)}} \\
&&  \rnode{qy}{q_y^{(y)}} \\
&&  \rnode{vqy}{v_{q_y}} \\
&&&&&&  \rnode{vqb}{v_{q_b}} \\
&&&&&&&&&& \rnode{vq0}{v_{q_0}}
\end{array}
\ncarc[nodesep=2pt,ncurv=0.9,angleA=180,angleB=180]{->}{vq0}{q0}
\ncarc[nodesep=2pt,ncurv=0.9,angleA=180,angleB=180]{->}{vqb}{qb}
\nccurve[nodesep=2pt,ncurv=0.9,angleA=180,angleB=180]{->}{vqy}{qy}
\ncarc[nodesep=2pt,ncurv=0.9,angleA=180,angleB=180]{->}{qy}{qb}
\ncarc[nodesep=2pt,ncurv=0.9,angleA=90,angleB=180]{->}{qb}{q0}
\nccurve[nodesepB=2pt,nodesepA=6pt,ncurv=0.9,angleA=180,angleB=180]{->}{1}{qa}
\ncarc[nodesep=2pt,ncurv=0.9,angleA=90,angleB=180]{->}{qa}{q0}
$$


\subsubsection{Game characterisation of safe terms}

A difficulty arises because of the presence of the Y combinator :
computation trees of \pcf\ terms are potentially infinite. Despite
this particularity, lemma \ref{lem:incrbound_iff_etanf_safe} still
holds in the \pcf\ setting:
\begin{lemma} \label{lem:pcf_safe_imp_incrbound} If $M$ is a safe
PCF term then $\tau(M)$ is incrementally-bound.
\end{lemma}
\begin{proof}
Let $i$ denote the number of occurrences of the Y combinator in $M$.
We first prove by induction on $i$ that $M_k$ is safe for any $k\in
\omega$. \emph{Base case:} $i=0$ then $M_k = M$. \emph{Step case:}
$i>0$. Let $Y_A N$ be a subterm of $M$. Since $M$ is safe, $N$ is
also safe. The number of occurrences of the Y combinator in $N$ is
smaller than $i$ therefore by the induction hypothesis $N_k$ is
safe. Consequently the term $Y_A^k N_k = \underbrace{N_k ( \ldots (
N_k}_{k \mbox{ times}} \Omega ) \ldots )$ is also safe and by
compositionality so is $M_k$.

Clearly, lemma \ref{lem:incrbound_iff_etanf_safe}(i) is remains
valid for infinite $\pcf_1$ terms (the subterms of the form $\Omega$
are just represented by the constant $\bot$ in the computation
tree), thus since $M_k$ is a safe $\pcf_1$ term, $\tau(M_k)$ is
incrementally-bound. Now let $z$ be a variable node in $\tau(M) =
\Union_{k\in\omega} \tau(M_k)$. There exists $k\in \omega$ such that
$z$ belongs to $\tau(M_k) \sqsubseteq \tau(M)$. If we write $r_k$ to
denote the root of the tree $\tau(M_k)$ then the path $[r_k,z]$ in
$\tau(M_k)$ is equal to the path $[r,z]$ in $\tau(M)$. Hence, since
the node $z$ is incrementally-bound in $\tau(M_k)$, it is also
incrementally-bound in $\tau(M)$.
\end{proof}


\begin{theorem}
Safe PCF terms are denoted by P-incrementally-justified strategies.
\end{theorem}
\begin{proof}
Let $M^{\infty}$ be the $\beta$-normal form of $M$ (i.e. the possibly infinite term obtained by reducing all the redexes in $M$). By lemma \ref{lem:ia_safety_preserved}, safety is preserved by small-step reduction therefore, by lemma \ref{lem:pcf_safe_imp_incrbound}, if $M$ is a \pcf\ term then $\tau(M^{\infty})$ is also
incrementally-bound.

Since \pcf\ constant rules are well-behaved (by Lemma
\ref{lem:sigma_order1_are_wellbehaved}), the result from Lemma
\ref{lem:betanf_wellbehavedconst_trav_pview_red} is also true for
Safe \pcf. Thus proposition
\ref{prop:Nher_incrbound_and_incrjustified}(i) remains valid for the
infinite computation trees of \pcf: infinite terms in $\beta$-nf
with an incrementally-bound computation tree are denoted by
P-incrementally-justified strategies. Consequently,
$\sem{M^{\infty}}$ is P-incrementally-justified. By soundness of the
game denotation, $\sem{M^{\infty}} = \sem{M}$, thus $\sem{M}$ is
P-incrementally-justified.
\end{proof}

Consequently, P-pointers are superfluous in the game denotation of safe \pcf\ terms {\it i.e.} pointers emanating from P-moves are uniquely recoverable.

\subsection{Safe \ialgol}

We are now in a position to consider the full safe Idealized Algol
language. The general idea is the same as for safe \pcf, however
there are some difficulties caused by the presence of the two new
base types \iavar\ and \iacom. We just give indications on how to
adapt our framework to the particular case of safe \ialgol\ without
giving the complete proofs. However we believe that enough
indications are given to convince the reader that the argument used
in the \pcf\ case can be easily adapted to \ialgol.

\subsubsection{Computation DAG}
In \pcf, arenas have a single initial move, therefore they can be
regarded as trees. In \ialgol, on the other hand, the base type
\iavar\ is represented by the infinite product of games
$\iacom^{\nat} \times \iaexp$ which has an infinite number of
initial moves. In order to preserve the relationship established
between arenas and computation trees, we need to accommodate the
definition of computation tree to reflect this property. The
consequence is that in \ialgol, ``computation trees'' become
``computation directed acyclic graphs (DAG)'': a computation DAG may
have (possibly infinitely) many roots and two nodes of a given level
can share children at the next level.


We use the notations $\mathcal{D}_{\iaexp} = \nat$ and
$\mathcal{D}_{\iacom} = \{ \iadone \}$ to denote the set of value
leaves of type \iaexp\ and \iacom\ respectively. There are two types
of value-leaves in the computation DAG: the value-leaf \iadone\ of
type \iacom\ and the value-leaves labelled in $\mathcal{D}_{\iaexp}$
of type \iaexp.

Let $n$ be a node. If $\kappa(n)$ is of type $(A_1,\ldots A_n,B)$,
we call $B$ the \emph{return type of $n$}. The set of value-leaves
of a node $n$ is given by $\mathcal{D}_{\iaexp}$ if the return type
of $n$ is \iaexp, by $\mathcal{D}_{\iacom}$ if its return type is
\iacom, and by $\mathcal{D}_{\iaexp} \union \{ \iadone \}$ if its
return type is \iavar.


Table \ref{tab:ia_computationdag} shows the computation DAG for each
construct of \ialgol. The value-leaves are represented in the DAGs
using the following abbreviations:
$$ \pssetcomptree\tree{n}{ \TRV{\mathcal{D}_\iaexp} }  \quad \mbox{ for }\quad
 \tree{n}{ \TRV{0} \TRV{1} \TRV{2} \TRV{\ldots} }
 \qquad \mbox{ and } \qquad
 \tree{n}{ \TRV{\mathcal{D}_\iadone} }  \quad \mbox{ for }\quad
 \tree{n}{ \TRV{\iadone }}.
$$

A term of type \iavar\ has a computation DAG with an infinite number
of root $\lambda$-nodes. Suppose that $M$ is a term of type \iavar,
then the computation DAG for $\lambda \overline{\xi} . M$ is
obtained by relabelling the root $\lambda$-nodes $\lambda^r$,
$\lambda^{w_0}$, $\lambda^{w_1}$, $\lambda^{w_2}$, \ldots into
$\lambda^r \overline{\xi}$, $\lambda^{w_0} \overline{\xi}$,
$\lambda^{w_1} \overline{\xi}$, $\lambda^{w_2} \overline{\xi}$,
\ldots. For a term $M$  of type \iaexp\ or \iacom, the computation
DAG for $\lambda \overline{\xi} . M$ is computed in the same way as
in the safe $\lambda$-calculus.

\begin{table}
\begin{center}
\begin{tabular}{cc}
$M$ & $\tau(M)$ \\ \hline \hline \\
x $: A \in \{ \iacom, \iaexp \}$ &
    $\psmatrix[colsep=3ex,rowsep=2ex] \lambda \\ x & \mathcal{D}_A \\  & \mathcal{D}_A \endpsmatrix
    \ncline{1,1}{2,1} \valueedge{1,1}{2,2} \valueedge{2,1}{3,2} $
\\ \\
x : \iavar &
    $\psmatrix[colsep=3ex,rowsep=3ex]
    \lambda^r & \lambda^{w_0} & \lambda^{w_1}  & \lambda^{w_2} & \lambda^{w_{\ldots}} \\
    \mathcal{D}_\iaexp &  & x & & \iadone \\
    &  &  & \mathcal{D}_\iaexp & \iadone
    \endpsmatrix
    \ncline{1,1}{2,3} \ncline{1,2}{2,3} \ncline{1,3}{2,3} \ncline{1,4}{2,3} \ncline{1,5}{2,3}
    \valueedge{2,3}{3,4} \valueedge{2,3}{3,5}
    \valueedge{1,1}{2,1}
    \valueedge{1,5}{2,5} \valueedge{1,4}{2,5} \valueedge{1,3}{2,5} \valueedge{1,2}{2,5}
    $
\\ \\
\iaskip : \iacom &
    $\psmatrix[colsep=3ex,rowsep=3ex] \lambda \\ \iaskip & \iadone \\  & \iadone \endpsmatrix
    \ncline{1,1}{2,1} \valueedge{1,1}{2,2} \valueedge{2,1}{3,2} $
\\ \\
$\iaassign\ L\ N :\iacom$ &
    $\psmatrix[colsep=3ex,rowsep=3ex] & \lambda \\ & \iaassign & \iadone \\ \tau(N:\iaexp)  & \tau(L:\iavar) & \iadone \endpsmatrix
    \ncline{1,2}{2,2} \ncline{2,2}{3,2} \ncline{2,2}{3,1}
    \valueedge{1,2}{2,3} \valueedge{2,2}{3,3} $
\\ \\
$\iaderef\ L :\iaexp$ &
    $\psmatrix[colsep=3ex,rowsep=3ex] \lambda \\ \iaderef & \iadone \\ \tau(L:\iavar) & \iadone \endpsmatrix
    \ncline{1,1}{2,1} \ncline{2,1}{3,1} \valueedge{1,1}{2,2} \valueedge{2,1}{3,2} $
\\ \\
$\iaseq_{\iaexp}\ N_1\ N_2 :\iacom$ &
    $\psmatrix[colsep=3ex,rowsep=3ex] & \lambda \\ & \iaseq_{\iaexp} & \mathcal{D}_\iaexp \\ \tau(N_1:\iacom)  & \tau(N_2:\iaexp) & \iadone \endpsmatrix
    \ncline{1,2}{2,2} \ncline{2,2}{3,2} \ncline{2,2}{3,1}
    \valueedge{1,2}{2,3} \valueedge{2,2}{3,3} $
\\ \\
$\iamkvar\ N_w\ N_r :\iavar$ &
    $\psmatrix[colsep=3ex,rowsep=3ex]
    \lambda^r & \lambda^{w_0} & \lambda^{w_1}  & \lambda^{w_2} & \lambda^{w_{\ldots}} \\
    \mathcal{D}_\iaexp &  & \iamkvar & & \iadone \\
    & \tau(N_r) & \tau(N_w) & \mathcal{D}_\iaexp & \iadone
    \endpsmatrix
    \ncline{1,1}{2,3} \ncline{1,2}{2,3} \ncline{1,3}{2,3} \ncline{1,4}{2,3} \ncline{1,5}{2,3}
    \ncline{2,3}{3,2} \ncline{2,3}{3,3}
    \valueedge{2,3}{3,4} \valueedge{2,3}{3,5}
    \valueedge{1,1}{2,1}
    \valueedge{1,5}{2,5} \valueedge{1,4}{2,5} \valueedge{1,3}{2,5} \valueedge{1,2}{2,5}
    $
\\ \\
$\ianewin{x}\ N : A \in \{ \iacom, \iaexp \} $ &
   $\psmatrix[colsep=3ex,rowsep=3ex] \lambda \\ \ianewin{x} & \mathcal{D}_A \\ \tau(N:A) & \mathcal{D}_A \endpsmatrix
    \ncline{1,1}{2,1} \ncline{2,1}{3,1} \valueedge{1,1}{2,2} \valueedge{2,1}{3,2} $
\end{tabular}
\end{center}
  \caption{Computation DAGs for the constructs of \ialgol.}
  \label{tab:ia_computationdag}
\end{table}


\subsubsection{Traversals}
Let $p$ be a node and suppose that its $i$th child $n$ has the
return type \iavar. Then $n$ is in fact constituted of several
$\lambda$-nodes : $\lambda^r \overline{\xi}$, $\lambda^{w_0}
\overline{\xi}$, \ldots. From $p$'s point of view, these nodes are
referenced as follows: $i.r$ refers to $\lambda^r \overline{\xi}$
and  $i.w_k$ refers to $\lambda^{w_k} \overline{\xi}$ for $k \in
\omega$.

\begin{itemize}
\item \emph{The application rule}

There are two rules (app$_{\iaexp}$) and (app$_{\iacom}$)
corresponding to traversals ending with an @-node of return type
\iaexp\ and \iacom\ respectively. These rules are identical to the
rule \iaexp\ of section \ref{subsec:traversal}.

The application rule for $@$-nodes with return type \iavar\ is:
$$(\mbox{app}_{\iavar})
\rulef{ \Pstr{t \cdot (lHyp){\lambda^k \overline{\xi}} \cdot
(appHyp-lHyp,35:0){@} \in \travset }
 }{\Pstr[18pt] {t \cdot (l){\lambda^k
\overline{\xi}} \cdot (app-l,35:0){@} \cdot (l2-app,35:0.k){\lambda^k
\overline{\eta}} \in \travset }}
 \ k \in \{ r, w_0, w_1, \ldots \}
$$


\item \emph{Input-variable rules}

There are two rules (InputVar$^{\iaexp}$) and (InputVar$^{\iacom}$)
which are the counterparts of rule (InputVar$^0$) of section
\ref{subsec:traversal} and are defined identically.

Let $x$ be an input-variable of type \iavar:
$$ (\mbox{InputVar}^{\iavar})
\rulef{t \cdot \lambda^r \overline{\xi} \cdot x \in \travset}
    {t \cdot \lambda^r \overline{\xi} \cdot \rnode{x}{x} \cdot v_x \in \travset }
\hspace{2cm} (\mbox{InputVar}^{' \iavar}) \rulef{t \cdot
\lambda^{w_i} \overline{\xi} \cdot x \in \travset}
    {t \cdot \lambda^{w_i} \overline{\xi} \cdot \rnode{x}{x} \cdot \iadone_x \in \travset }
$$

\item \emph{IA constants rules}

The rules for \ianew\ are purely structural, they are defined the
same way as the rules (app$_{\iaexp}$), (app$_{\iacom}$) and
(app$_{\iadone}$).

The rules for \iaderef\ are:
$$(\mbox{deref}) \rulef{t \cdot \iaderef \in \travset}{\Pstr[15pt]{t \cdot (d){\iaderef} \cdot (n-d,35:1.r){n} \in \travset }}
 \hspace{1.6cm} (\mbox{deref'})
\rulef{t \cdot \iaderef \cdot n \cdot t_2 \cdot v_n \in \travset} {t
\cdot \iaderef \cdot n \cdot t_2 \cdot v_n \cdot v_{\iaderef}\in
\travset }
$$


The rules for \iaassign\ are:
$$(\mbox{assign}) \rulef{t \cdot \iaassign \in \travset}{\Pstr[15pt]{t \cdot (ass){\iaassign} \cdot (n-ass,35:1){n} \in \travset} }
\hspace{1.6cm}
(\mbox{assign'})
\rulef{t \cdot \iaassign \cdot n \cdot t_2 \cdot v_n \in
\travset} {\Pstr[18pt]{t \cdot (ass){\iaassign} \cdot (n){n} \cdot
t_2 \cdot v_n \cdot (m-ass,15:2.w_n){m} \in \travset } }
$$
$$(\mbox{assign''})  \rulef{\Pstr{t \cdot (assHyp){\iaassign} \cdot t_2 \cdot (mHyp-assHyp,35:2.w_k){m} \cdot t_3 \cdot \iadone_m \in \travset}}
{t \cdot \iaassign \cdot t_2 \cdot m \cdot t_3 \cdot \iadone_m \cdot
\iadone_{\iaassign} \in \travset }
$$

The rules for $\iaseq_{\iaexp}$ are:
$$(\mbox{seq}) \rulef{t \cdot \iaseq \in \travset}{\Pstr[13pt]{t \cdot (seq){\iaseq} \cdot (n-seq,35:1){n} \in \travset } }
\hspace{1.6cm} (\mbox{seq'})
\rulef{t \cdot \iaseq \cdot n \cdot t_2 \cdot v_n \in
\travset} {\Pstr[18pt]{ t \cdot (seq){\iaseq} \cdot (n){n} \cdot t_2
\cdot v_n \cdot (m-seq,25:2){m} \in \travset }}
$$
$$(\mbox{seq''})  \rulef{\Pstr{t \cdot (seqHyp){\iaseq} \cdot t_2 \cdot (mHyp-seqHyp,35:2){m} \cdot t_3 \cdot v_m \in \travset}}
{t \cdot \iaseq \cdot t_2 \cdot m \cdot t_3 \cdot v_m \cdot
v_{\iaseq} \in \travset }$$




The rules for \iamkvar\ are:
$$(\mbox{mkvar}_r) \rulef{t \cdot \lambda^r \overline{\xi} \cdot \iamkvar \in \travset}{\Pstr[14pt]{t \cdot \lambda^r \overline{\xi} \cdot (d){\iamkvar} \cdot (n-d,35:1){n} \in \travset} }
\hspace{1cm} (\mbox{mkvar}_r')
\rulef{t \cdot \iamkvar \cdot n \cdot t_2 \cdot v_n \in \travset} {t
\cdot \iamkvar \cdot n \cdot t_2 \cdot v_n \cdot v_{\iamkvar}\in
\travset } $$
$$(\mbox{mkvar}_w) \rulef{t \cdot \lambda^{w_k} \overline{\xi} \cdot \iamkvar \in \travset}{\Pstr[15pt]{t \cdot \lambda^{w_k} \overline{\xi} \cdot (mk){\iamkvar} \cdot (n-mk,35:2){n} \in \travset} }$$
$$ (\mbox{mkvar}_w'')  \rulef{t \cdot \lambda^{w_k} \overline{\xi} \cdot \iamkvar \cdot n \cdot t_2 \cdot \iadone_n \in \travset}
{t \cdot \lambda^{w_k} \overline{\xi} \cdot \iamkvar \cdot n \cdot
t_2 \cdot \iadone_n \cdot \iadone_{\iamkvar} \in \travset }
$$
These four rules are not sufficient to model the constant \iamkvar.
Indeed, consider the term $\iaassign\ (\iamkvar\ (\lambda x . M) N)
7$. The rule (\mbox{mkvar}$_w''$) permits to traverse the node
\iamkvar\ and to go on by traversing the computation tree of
$\lambda x . M$. The problem is that when traversing $\tau(M)$, if
we reach a variable $x$, we are not able to relate $x$ to the value
$7$ that is assigned to the variable.

To overcome this problem, we need to define traversal rules for
variable in such a way that a variable node bound by the second
child of a $\iamkvar$-node is treated differently from other
variables.

\item \emph{Variable rules}
Let $x$ be a non input-variable node. It either corresponds to a $\lambda$-abstracted variable or
a block-allocated variable declared by the $\ianewin{x}$ construct.

\begin{itemize}
\item Suppose that $x$ is $\lambda$-abstracted and let $\lambda \overline{x}$ be its binder.
In \ialgol, the only constant nodes of order greater than 1 is
\iamkvar, therefore there are two cases: $\lambda \overline{x}$ is
either the child of a node in $N_@ \union N_{\sf var}$ or it is the
second child of a \iamkvar-node.

To handle the first case, we define a rule similar to the (Var) rule
of section \ref{subsec:traversal} with some modification to take
into account variables $x$ of type \iavar (in which case $x$ has
multiple parent $\lambda$-nodes). We do not give the details here
but it is easy to see how to redefine this rule.

To handle the case where $\lambda \overline{x}$ is the child of a
\iamkvar-node, we define the following rule:
$$ (\mbox{Var}_{\iamkvar})  \rulef{t \cdot \lambda^{w_k} \overline{\xi} \cdot \iamkvar \cdot \lambda \overline{x} \cdot t_2 \cdot x \in \travset}
{t \cdot \lambda^{w_k} \overline{\xi} \cdot \iamkvar \cdot \lambda
\overline{x} \cdot t_2 \cdot x \cdot k_{x} \in \travset }
$$

\item Suppose that $x$ is block-allocated with $\ianewin{x}$.

We call \emph{overwrite of $x$ relatively to an occurrence of a} ``\ianewin{x}''\emph{-node}, any sequence of nodes of the form
$\Pstr[17pt]{(decl){\ianewin{x}}\cdot \ldots \cdot \lambda^{w_k}\overline{\xi} \cdot (x-decl,25){x}}$ for some $k\in \mathcal{D}_{\iaexp}$ and node $\lambda^{w_k}\overline{\xi}$ parent
of $x$.
$$(\mbox{Var}_w)
    \rulef{
        t \cdot \lambda^{w_k} \overline{\xi} \cdot x \in \travset
    }
    {   t \cdot \lambda^{w_k} \overline{\xi} \cdot x \cdot \iadone_x \in
        \travset
    },
$$

$$(\mbox{Var}_r)
    \rulef{
        \Pstr[17pt]{t_1 \cdot (decl){\ianewin{x}} \cdot t_2 \cdot \lambda^r \overline{\xi} \cdot (x-decl,25){x} \in \travset}
    }
    {   t_1 \cdot \ianewin{x} \cdot t_2 \cdot \lambda^r \overline{\xi}
        \cdot x \cdot 0_x \in \travset
    }
    \mbox{ if $t_2$ contains no overwrite of $x$},
$$

$$(\mbox{Var'}_r)
    \rulef{
        \Pstr[15pt]{
            t_1 \cdot (decl){\ianewin{x}} \cdot t_2 \cdot \lambda^r \overline{\xi} \cdot (x-decl,25){x} \in \travset
        }
    }
    {
        t_1 \cdot \ianewin{x} \cdot t_2 \cdot \lambda^r \overline{\xi} \cdot x \cdot k_x \in \travset
    }
    \mbox{ if $\lambda^{w_k} \cdot x$ is the last overwrite of $x$ in } t_2. $$
\end{itemize}
\end{itemize}
 
\subsubsection{Game semantics correspondence}
The properties that we proved for computation trees and traversals
of the safe $\lambda$-calculus with constants can easily be lifted
to computation DAGs of \ialgol. In particular:
\begin{itemize}
\item constant traversal rules are well-behaved (for order-$0$ and order-$1$ constants, this is a consequence
of Lemma \ref{lem:sigma_order1_are_wellbehaved}; for $\iamkvar$
however it needs to be proved separately);
\item P-view of traversals are paths in the computation DAG;
\item the P-view of the reduction of a traversal is the reduction of the P-view,
and the O-view of a traversal is the O-view of its reduction
(Lemma \ref{lem:pview_trav_projection} and
\ref{lem:oview_trav_projection});
\item there is a mapping from vertices of the computation DAG to moves in the interaction game semantics;
\item there is a correspondence between traversals of the computation tree and plays in interaction game semantics;
\item consequently, there is a correspondence between the standard game semantics and
the set of justified sequences of nodes $\travset(M)^{\filter r}$.
\end{itemize}

\subsubsection{Game-semantic characterisation of safe terms}
Clearly, the computation DAG of a safe term is incrementally-bound.
By using the correspondence between traversals and plays, it is easy
to prove that incrementally-bound computation trees are denoted by
P-incrementally-justified strategies. Consequently, by lemma
\ref{lem:incrjustified_pointers_uniqu_recover}, P's pointers are superfluous in the
game semantics of safe \ialgol\ terms.

Since the game denotation of an \ialgol\ term is fully determined by
the set of complete plays, this pointer economy suggests that the
game denotation of a safe \ialgol\ can be represented in a compact
way. This raises the question of the decidability of observational
equivalence for safe \ialgol.



%%%%%%%%%%%%%%%%%%%%%%%%%%%%
%%%%%%%%%%%%%%%%%%%%%%%%%%%%
\notetoself{the following section needs to be integrate into the previous chapter.}


\section{Game-semantic of Safe PCF}
In this section will give a game-semantic characterization of Safe
PCF based on syntactical arguments.

\begin{definition}
We say that a PCF term is \defname{semi-safe} if it is of the form
$N_0 N_1 \ldots N_k$ for $k\geq 1$ where each of the $N_i$ is a Safe
PCF term or if it can be written $\lambda \overline{x} . N$ for some
safe PCF term $N$.
\end{definition}
Semi-safe terms are either safe or ``almost safe'' in the sense that
they can be turned into an equivalent (i.e.~with isomorphic game
semantics) safe term  by performing $\eta$-expansions. Indeed, let
$M$ be an semi-safe term that is unsafe. If $M$ is of the first form
$N_0 N_1 \ldots N_k : (A_1,\ldots,A_n)$ with $k\geq 1$ then let
$\varphi_i:A_i$ for $i\in\{1..n\}$ be fresh variables, using the
(app) and (abs) rules we can build the safe term $\lambda \varphi_1
\ldots \varphi_n . N_0 N_1 \ldots N_k \varphi_1 \ldots \varphi_n$.
If $M$ is of the second form $\lambda \overline{x} . N$ then using
the abstraction rule we can build the equivalent safe term $\lambda
\overline{y} \overline{x}. N$  where $\overline{y} = fv(\lambda
\overline{x}. N)$.

The $\beta$-normal form of a \pcf\ term is the possibly infinite
term obtained by reducing all the redexes in $M$.

\subsubsection{Safe terms vs P-i.j.\ strategies}

In the context of the simply typed lambda calculus, the
correspondence between safety and P-incremental justification was
first shown in \cite[Theorem 3(ii)]{blumong:safelambdacalculus}
using a syntactic argument:
\begin{theorem}[\cite{blumong:safelambdacalculus},Theorem 3(ii)]
\label{thm:safeincrejust}
 In the simply typed lambda calculus:
\begin{enumerate}[(i)]
\item If $M$ is safe then $\sem{M}$ is P-incrementally justified.
\item If $M$ is a closed term and $\sem{M}$ is
  P-incrementally justified then the $\eta$-long form of the
  $\beta$-normal form of $M$ is safe.
\end{enumerate}
\end{theorem}
In fact the following more precise result holds (the proof of the
previous theorem can be easily adapted to this one):
\begin{theorem}[Semi-safety and P-incremental justification]
\label{thm:semisafeincrejust} Let $\Gamma \vdash M : A$ be a simply typed term. Then:
\begin{enumerate}[(i)]
\item If $\Gamma \vdash M : A$ is semi-safe then $\sem{\Gamma \vdash M : A}$ is P-incrementally justified.
\item If $\sem{\Gamma \vdash M : A}$ is
  P-incrementally justified then $\etalnf{\betanf{M}}$ is
semi-safe if $M$ is open and safe if $M$ is closed.
\end{enumerate}
\end{theorem}



In the context of \pcf\ however, only the first part of the theorem
holds (see \cite{blumtransfer} for the proof). However (ii) does not
hold. Indeed, take the closed \pcf\ term $M = \lambda f x y. f
(\lambda z. \pcfcond (\pcfsucc\ x) y z )$ where $x,y,z:o$ and
$f:((o,o),o)$. $M$ is in normal form (conditional cannot be reduced
since the value of $x$ is undetermined). The $\eta$-long form of the
$\beta$-normal form of $M$ is therefore $M$ itself which is unsafe.
But clearly we have $\sem{M} = \sem{\lambda f x y. f (\lambda z.
z)}$, and since $\lambda f x y. f (\lambda z. z)$ is safe, by (i),
$\sem{M}$ is P-incrementally justified.

Such counter-example arises because the conditional operator of
\pcf\ permits us to construct terms in normal form that contain
``dead code'' {\it i.e.}~some subterm that will never be evaluated
for any value of M's parameters. In the example above, the dead code
consists of the subterm $y$. In general, if the dead code part of
the computation tree contains a variable that is not incrementally
bound then the resulting term will be unsafe even if the rest of the
tree is incrementally bound. In the example above, it was possible
to turn $M$ into the equivalent safe term $\lambda f x y. f (\lambda
z. z)$ by eliminating the dead code from $M$. In fact we can
generalise this method to any \pcf\ term with a P-incrementally
justified denotation.
\smallskip

Dead code elimination can be difficult to achieve in practice but it
is easy to define it formally: We say that a subterm $N$ occurring
in a context $C[-]$ in $M : (A_1, \ldots, A_n,o)$ is part of the
\defname{dead code} of $M$ if for any term $T_0$ of the form $M M_1
\ldots M_n$, any reduction sequence starting from $T_0$ does not
involve a reduction of the subterm $N$ {\it i.e.}~for any reduction
sequence $T_0 \redar T_1 \redar \ldots \redar T_k$, there is no
$j\in \{0.. k-1\}$ such that $T_j = C[N]$ and $T_{j+1} = C[N']$ for
some term $N'$.


Let $M$  be a \pcf\ term in $\eta$-nf. An occurrence of a variable
$x$ in $M$ is said to be a \defname{dead occurrence} if it occurs in
the dead code of $M$. In other words, it is a dead occurrence of $x$
if the corresponding node in the computation tree does not appear in
any traversal of $\travset(M)$. Equivalently, thanks to the
Correspondence Theorem, an occurrence of $x:B$ is dead if and only
if the initial move of the arena $\sem{B}$ does not appear in any
play of $\sem{M}$.


We define $M^*$ as the term obtained from $M$ after substituting all
subterms of the form  $x N_1 \dots N_k$ for some dead variable
occurrence $x:(B_1,\ldots, B_k, o)$ by the constant $0$. This
process is called \defname{dead variable elimination}. Note that if
$M$ is in $\eta\beta$-nf then so is $M^*$. We also write $\tau(M)^*$
to denote the equivalent transformation on the computation tree.
Since the computation tree is constructed from the $\eta$-nf of $M$,
we will use this notation even when $M$ is not in $\eta$-nf.



\begin{proposition}[Incremental-binding and P-incremental justification coincide] \
\label{prop:Nher_incrbound_and_incrjustified_pcf} Let $\Gamma \vdash
M : A$ be a PCF term in $\beta$-normal form.
\begin{enumerate}[(i)]
\item  If $\tau(\Gamma \vdash M : A)$ is incrementally-bound then $\sem{\Gamma \vdash M : A}$ is P-incrementally justified,
\item  if $\sem{\Gamma \vdash M : A}$ is P-incrementally justified
then $\tau(\Gamma \vdash M : A)^*$ is incrementally-bound.
\end{enumerate}
\end{proposition}
\begin{proof}
(i) The proof is exactly the same as in the simply typed lambda calculus case,
see \cite[Proposition 4.1.5(i)]{blumtransfer}.

\noindent (ii)
Take $\Gamma \vdash M : A$ a \pcf\ term in $\beta$-normal form denoted by $\sem{\Gamma \vdash M : A}$ P-incrementally justified. Let $r$ denote the root of $\tau(M)^*$.
Let $n$ be a node of $\tau(M)^*$ labelled by the variable $x$.
$\tau(M)^*$ is free from dead code therefore $n$ is not a dead occurrence of $x$ and there exists a traversal of $\tau(M)^*$ of the form $t \cdot x$.

\pcf\ constants are of order $1$ at most therefore they cannot
hereditarily justify a variable node, thus $x$ is necessarily
hereditarily justified by the only occurrence $r$ of the root of the
computation tree.

By considering $t\cdot x$ as a traversal of $\tau(M)$,  the
correspondence theorem gives $\varphi((t \cdot x) \filter r) =
\varphi((t \filter r) \cdot x) \in \sem{M}$. Since $\sem{M}$ is
P-incrementally justified, $\varphi(x)$ must point to the last
O-move in $\pview{\varphi(t \filter r)}$ with order strictly greater
than $\ord{\varphi(x)}$. Consequently $x$ points to the last node in
$\pview{t \filter r} \filter N^{\lambda}$ with order strictly
greater than $\ord{x}$. We have:
\begin{align*}
\pview{t \filter r} &= \pview{t} \filter N^{r \vdash} & (\mbox{by Lemma \ref{lem:betanf_wellbehavedconst_trav_pview_red}}) \\
& = [r,x[ \ \filter N^{r \vdash} & (\mbox{by Prop.\ \ref{prop:pviewtrav_is_path}})
\end{align*}
\notetoself{review use of Lemma
\ref{lem:betanf_wellbehavedconst_trav_pview_red}}

Since $M$ is in $\beta$-nf, the set of nodes not hereditarily
enabled by $r$ is exactly the set of nodes hereditarily enabled by
$N_{\Sigma}$ thus $[r,x[ \ \filter N^{r \vdash} = [r,x[\ \setminus\
N^{\filter \Sigma}$. Moreover \pcf\ constants are of order $1$ at
most therefore $N^{\filter \Sigma} = N_{\Sigma} \union N^c_{\Sigma}$
where $N^c_{\Sigma}$ is the set of children nodes of $N_{\Sigma}$.
Thus $\pview{t \filter r} \filter N^{\lambda} = ([r,x[\ \setminus\
N_{\Sigma} \setminus N^c_{\Sigma} ) \filter N^{\lambda} = ([r,x[\
\setminus\  N^c_{\Sigma} )  \filter N^{\lambda}$, and since
$N^c_{\Sigma}$ is constituted of order $0$ lambda-nodes only, $x$
must point to the last node in $[r,x[ \filter N^{\lambda}$ with
order strictly greater than $\ord{x}$.

Hence if $x$ is a bound variable node then it is bound by the
last $\lambda$-node in $[r,x[$ with order strictly greater than
$\ord{x}$ and if $x$ is a free variable then it points to $r$ and
therefore all the $\lambda$-node in $]r,x[$ have order smaller than
$\ord{x}$. Thus $\tau(M)^*$ is incrementally-bound.
\end{proof}

The counterpart of Lemma 4.1.6 from
\cite{blumtransfer} can be stated as follows in the context of PCF:
\begin{lemma}[Semi-safety and incrementally-binding]
\label{lem:incrbound_iff_etanf_safe_pcf} Let $\Gamma \vdash M : A$
be a PCF term.
\begin{itemize}
\item[(i)] If $\Gamma \vdash M : A$ is a semi-safe term then $\tau(\Gamma \vdash M : A)$ is incrementally-bound ;
\item[(ii)] conversely, if $\tau(\Gamma \vdash M : A)$ is incrementally-bound then the $\eta$-normal form of $\Gamma \vdash M : A$ is semi-safe if $M$ is open and safe if $M$ is closed.
\end{itemize}
\end{lemma}
The proof can be obtained by adapting the proof
of Lemma 4.1.6 from \cite{blumtransfer}.

\begin{theorem}[Semi-safety and P-incremental justification]
\label{thm:semisafeincrejust_pcf} Let $\Gamma \vdash M : A$ be a PCF term. Then:
\begin{enumerate}[(i)]
\item If $\Gamma \vdash M : A$ is semi-safe then $\sem{\Gamma \vdash M : A}$ is P-incrementally justified.
\item If $\sem{\Gamma \vdash M : A}$ is
  P-incrementally justified then $\etalnf{\betanf{M}}^*$ is
  semi-safe  if $M$ is open, and safe if $M$ is closed.
\end{enumerate}
\end{theorem}

\begin{proof}
\noindent(i)
A proof of this is given in the proof of Theorem 4.2.10 in \cite{blumtransfer}.

\noindent(ii) Suppose $M$ is a \pcf\ term with a P-incrementally
justified strategy denotation. By Proposition
\ref{prop:Nher_incrbound_and_incrjustified_pcf}(ii),
$\tau(\betanf{M})^* = \tau(\etalnf{\betanf{M}}^*)$ is
incrementally-bound. If $M$ is closed then so is
$\etalnf{\betanf{M}}^*$ therefore by Lemma
\ref{lem:incrbound_iff_etanf_safe_pcf},
$\etalnf{\etalnf{\betanf{M}}^*} = \etalnf{\betanf{M}}^*$ is safe. If
$M$ is open then so is $\etalnf{\betanf{M}}^*$ and by Lemma
\ref{lem:incrbound_iff_etanf_safe_pcf},
$\etalnf{\etalnf{\betanf{M}}^*} = \etalnf{\betanf{M}}^*$ is
semi-safe.
\end{proof}


We write \pcf' to denote the language obtained by extending \pcf\
with the $\pcfcase_k$ construct (see \cite{Abr02}).
The $\pcfcase_k$ construct is the obvious generalisation of the
conditional operator \pcfcond\ to $k$ branches instead of $2$. All the results obtained so far concerning Safe \pcf\ (including those
cited from \cite{blumtransfer}) can clearly be transposed to \pcf'.

\subsubsection{Definability result}

The previous theorem leads to the following definability result for safe \pcf':
\begin{proposition}[Definability for safe \pcf' terms]
\label{prop:safetydefinability} Let $\overline{A}=(A_1,\ldots, A_i)$
and $B =(B_1, \ldots, B_l,o)$ be two PCF types for some $i,l\geq 0$
and $\sigma$ be a well-bracketed innocent P-i.j.\ strategy with
finite view function defined on the game $!A_1 \otimes \ldots
\otimes !A_i \lingamear (!B_1 \lingamear \ldots \lingamear !B_l
\lingamear o) $. There exists a \emph{semi-safe} PCF' term
$\overline{x} : \overline{A} \vdash M : B$ in $\eta$-long normal
form such that:
$$ \sem{\overline{x} : \overline{A} \vdash M_\sigma : B} = \sigma $$
and a safe closed PCF' term $\vdash_s M'_\sigma : (\overline{A},B)$ in $\eta$-long normal form such that:
$$ \sem{\vdash M'_\sigma : (\overline{A},B)} \cong \sigma \ .$$
\end{proposition}
\begin{proof}
By the standard definability result for PCF', there is a term
$\overline{x} : \overline{A} \vdash N : B$ such that
$\sem{\overline{x} :\overline{A} \vdash N : B} = \sigma$. Take
$M_\sigma$ to be $\etalnf{\betanf{N}}^* $. We have
$\sem{\overline{x} : \overline{A} \vdash M_\sigma : B} =
\sem{\overline{x} :\overline{A} \vdash N : B} = \sigma$ and by
Theorem  \ref{thm:semisafeincrejust_pcf}(ii), $M_\sigma$ is
semi-safe. For the second part we just need to take $M'_\sigma =
\lambda \overline{x}. M_\sigma$.
\end{proof}



\subsubsection{Application of the definability result: a syntactic
argument showing compositionality of P-i.j.\ strategies}


We have already shown in Sec. \ref{sec:closedpij} that under certain
conditions, P-i.j.\ strategies compose. Here we will obtain a
slightly weaker version of this result using a much simpler argument
which exploits the definability result from the previous section.


 Let $\overline{A} = (A_1, \ldots, A_i)$, $B = (B_1, \ldots,
B_l,o)$ and $C=(C_1,\ldots,C_k,o)$ be three PCF types for some
$i\geq 1,l,k\geq 0$. Let $f:\ !A_1 \otimes \ldots \otimes !A_i
\lingamear B$ and $g:\ !B\lingamear C$ be two innocent
well-bracketed and P-incrementally justified strategies with finite
view function. We would like to find under which conditions the
composition $f\fatcompos g$ is also P-incrementally justified.

By the definability result, there are two closed safe terms (in $\eta$-nf) $\vdash M_f :(\overline{A},B)$  and $\vdash M_g :B \typear C$ such that $\sem{M_f} = f$
and $\sem{M_f} = g$.
We define the term $M_{f\fatcompos g} = \lambda \overline{x} . M_g (M_f \overline{x})$ for some fresh variables $\overline{x} : \overline{A}$. Clearly we have $\sem{M_{f\fatcompos g}} = \sem{M_f} \fatcompos \sem{M_g} = f\fatcompos g$.

\paragraph{Sufficient conditions}

By Theorem \ref{thm:semisafeincrejust_pcf}, we know that
$f\fatcompos g$ is P-incrementally justified just when
$\etalnf{\betanf{M_{f\fatcompos g}}}^*$ is safe. We will now exploit
this fact to extract a sufficient condition on the types $A$ and $B$
for the composition of $f$ and $g$ to be P-incrementally justified.

The term $M_f$ and $M_g$, being in $\eta$-nf, are of the following forms:
\begin{eqnarray*}
\vdash M_f &=& \lambda x_1^{A_1} \ldots x_i^{A_i} \varphi_1^{B_1} \ldots \varphi_l^{B_l} . N_f^o\\
\vdash  M_g &=& \lambda y^{ (B_1, \ldots, B_l,o)} \phi_1^{C_1} \ldots \phi_k^{C_k} . N_g^o
\end{eqnarray*}
for some distinct variables $x_1, \ldots, x_i$, $y$, $\varphi_1, \dots \varphi_l$, $\phi_1, \dots \phi_k$  and $\eta$-normal terms $N_f$ and $N_g$:
\begin{eqnarray*}
x_1:A, \ldots, x_i:A_i, \varphi_1:B_1, \dots, \varphi_l:B_l &\vdash& N_f :o \\
y: (B_1, \ldots, B_l,o), \phi_1:C_1, \dots, \phi_l:C_l &\vdash& N_g :o
\end{eqnarray*}



The fact that $M_f$ and $M_g$ are safe does not imply that $M_{f\fatcompos g}$ is: take $M_f = \lambda x^o z^o.x$ and $M_g = \lambda y^{(o,o)} . y a$ for some constant $a\in \Sigma$, then $\lambda x:A . M_g (M_f x) = \lambda x . (\lambda y . y a) ( \underline{(\lambda x z.x) x} )$ is unsafe because of the underlined subterm. However we have:
\begin{align*}
f\fatcompos g &= \sem{\lambda \overline{x} . M_g (M_f  \overline{x})} \\
 &= \sem{\lambda \overline{x} . (\lambda \phi_1\ldots \phi_k . N_g) [(M_f \overline{x}) / y]} \\
&= \sem{\lambda \overline{x} \phi_1 \dots \phi_k. N_g [(M_f  \overline{x}) / y]}
& \mbox{(the $x_j$'s and $\phi_j$'s are disjoint)}.
\end{align*}

We now concentrate on the term  $\lambda \overline{x} \phi_1 \dots
\phi_k. N_g [(M_f  \overline{x}) / y]$ and try to find a sufficient
condition guaranteeing its safety.

\subparagraph{A sufficient condition}
\begin{lemma}
Suppose that $\Gamma,y:B \vdash M$ is a safe term in $\eta$-nf and $\Gamma \vdash R : B$ is an almost safe application. Let $N$ denote the set of nodes of the computation tree $\tau(M)$. We have:
\begin{align*}
\Gamma \vdash M[R/y] :A \mbox{ safe }
\iff&  \forall x \in fv(R) . \\
    & \forall n_y \in N_{\sf fv} \mbox{ labelled $y$}.
      \forall m \in N_{\lambda} \inter ]r,n_y] : \ord{m} \leq \ord{x}
\end{align*}
\end{lemma}
\begin{proof}
Since $M$ is in $\eta$-nf, all the application to the variable $y$ are total (i.e.~of the form $y P_1 \ldots P_l :o$). Hence after substituting the safe term $N$ for $y$ in $M$, the only possible cause of unsafety is when
some variable free in $N$ becomes not safely bound in $\tau(M)$.
\end{proof}

Applying this lemma with $R= M_f \overline{x}$ gives us a sufficient
condition -- the right-hand side of the equivalence -- for $\lambda
x \phi_1 \dots \phi_k. N_g [(M_f \overline{x}) / y]$ to be safe, and
hence for $f\fatcompos g$ to be P-incrementally justified. Of course
it is not a necessary condition since $N_g[(M_f \overline{x}) /y]$
can be unsafe while its eta-beta normal form is safe.

\subparagraph{A simpler sufficient condition}
\begin{lemma}
If $y:B, \Sigma \vdash N : T$ and $\vdash M : (\overline{A}, B)$
are safe terms with $\ord{A_i} \geq \ord{B}$ for all $i\in 1..n$
then $\overline{x}:\overline{A}, \Sigma \vdash N[(M \overline{x})/y] :T$ is also safe.
\end{lemma}
\begin{proof}
Since $\ord{x_i} = \ord{A_i} \geq \ord{B} = \ord{M \overline{x}}$, we can use the application
rule of the safe lambda calculus to form the safe term $\overline{x}:\overline{A} \vdash M \overline{x}$.
Using the substitution lemma we have that $N[(M \overline{x})/y]$ is safe.
\end{proof}

Hence we obtain the following sufficient condition for $f\fatcompos
g$ to be P-incrementally justified:
$$\ord{A_i}\geq\ord{B} \mbox{ for all } 1 \leq i \leq n$$


Indeed the lemma gives that $\vdash \lambda \overline{x} \phi_1
\dots \phi_k. N_g [(M_f \overline{x}) / y]$ is safe and therefore
its denotation $\sem{\vdash \lambda \overline{x} \phi_1 \dots
\phi_k. N_g [(M_f \overline{x}) / y]} = f\fatcompos g$ is
P-incrementally justified.

Note that this condition is not necessary: Take $A=o$, $B=(o,o)$,
$C=(o,o)$ and consider the two safe terms $M_f = \lambda x^A u^o.u$
and $M_g = \lambda y^B . y a$ for  some constant $a:o$. Then we have
$M_{f\fatcompos g} = \lambda x . a$ which is safe hence $f\fatcompos
g$ is P-incrementally justified although $\ord{A} < \ord{B}$.

\begin{remark}
This result corroborates what we already know about compositionality
of P-i.j.\ strategies (see Sec. \ref{sec:closedpij}). Indeed, the
condition given hereinbefore implies that the strategy $f$ is
\emph{closed} P-i.j.\ (the $A_i$s are prime because we are working
with PCF types) and therefore by Prop.\ \ref{prop:closedpijcompose},
$f \fatcompos g$ must also be P-i.j.
\end{remark}




\paragraph{Counter-example: two P-i.j.\ strategies whose composition is not
P-i.j.}

We now give counter-example to show that P-i.j.\ strategies do not
compose in general.

\subparagraph{First attempt}

Take the types $A=o$, $B=(o,o)$, $C=o$, the variables
$x,u,v:o$, $y:B$ and $\varphi:((o,o),o)$ and $\Sigma$-constant $a:o$.
Consider the two safe terms $\vdash_s  M_f = \lambda xv.x : A\typear B$ and $\vdash_s M_g = \lambda y . \varphi (\lambda u . y a) : B\typear C$.
The $\eta\beta$-nf of $M_{f\fatcompos g}$ is $\vdash \lambda x . \varphi (\underline{\lambda u . x})$ which is unsafe because of the underlined term. It is then tempting to use
Theorem \ref{thm:safeincrejust}(ii) to conclude that
$\sem{M_{f\fatcompos g}}$ is not P-incrementally justified. However this theorem cannot be used here because $M_g$ contains an order $2$ constants ($\varphi$) therefore
$M_{f\fatcompos g}$ is not a valid simply typed $\lambda$-term (nor a \pcf-term).

\subparagraph{Second attempt} The previous example can be easily
changed into a working counter-example: we just need to elevate
$\varphi$ from the status of constant to variable.

Take $A=o$, $B=(o,o)$, $C=(((o,o),o),o)$, the variables
$x,u,v:o$, $y:B$ and $\varphi:((o,o),o)$ and the $\Sigma$-constant $a:o$. Consider the two safe terms $\vdash_s  M_f = \lambda xv.x : A\typear B$ and  $\vdash_s M_g = \lambda y \varphi. \varphi (\lambda u . y a) : B\typear C$.
The $\eta\beta$-nf of $M_{f\fatcompos g}$ is $\vdash \lambda x \varphi. \varphi (\underline{\lambda u . x})$ which is unsafe because of the underlined term, thus by Theorem \ref{thm:safeincrejust}(ii), $\sem{M_{f\fatcompos g}}=\sem{M_f} \fatcompos
\sem{M_g}$ is not P-incrementally justified. The following diagram illustrates a play that is not P-i.j.:
\begingroup
\def\sigcol#1{{\color{gray} #1}}
\def\mucol#1{{\color{red} #1}}
$$\begin{array}{ccccccccc}
A &  & \multicolumn{2}{c}{B} && \multicolumn{4}{c}{C}\\
\cline{1-1} \cline{3-4} \cline{6-9}
o & \stackrel{\sigcol{\sem{M_f}}}\longrightarrow & o, & o & \stackrel{\mucol{\sem{M_g}}}\longrightarrow & ((o, &o),& o),& o \\ \\
&&&&&&&&\rnode{n0}{\lambda x \varphi \omove  \mucol {\lambda y \varphi}}\\
&&&&&&&\rnode{n1}{\varphi  \pmove \mucol \varphi}\\
&&&&&&\rnode{n2}{\lambda u \omove  \mucol {\lambda u}} \\
&&&  \rnode{n3}{\omove \sigcol {\lambda x v} \pmove \mucol y} \\
\rnode{n4}{x \pmove \sigcol x}
\end{array}
\ncarc[arcangleA=20,arcangleB=20,linecolor=black]{->}{n4}{n0}
\ncarc[arcangleA=30,arcangleB=20,linecolor=red]{->}{n2}{n1}
\ncarc[arcangleA=30,arcangleB=20,linecolor=red]{->}{n1}{n0}
\ncarc[arcangleA=20,arcangleB=20,linecolor=red]{->}{n3}{n0}
\ncarc[arcangleA=20,arcangleB=20,linecolor=gray]{->}{n4}{n3}
$$
\endgroup

\subparagraph{Another counter-example with $\ord{B} = \ord{C}$.}

Let $A=o$, $B=C=(((o,o),o),o)$ and let $x:A$, $y:B$, $u:o$, $v,\varphi:((o,o),o)$
and $g:(o,o)$ be variables and  $a:o$ be a $\Sigma$-constant. Take the two safe terms $\vdash  M_f = \lambda x v.x$ and $\vdash M_g = \lambda y \varphi. \varphi (\lambda u . y (\lambda g. a))$.
The $\eta\beta$-nf of $M_{f\fatcompos g}$ is $\vdash \lambda x \varphi. \varphi (\underline{\lambda u . x})$ which is unsafe because of the underlined term, so
$f\fatcompos g$ is not P-incrementally justified.



\chapter{Game-Semantic Models of Safe Languages}
    \label{chap:model}
    \psset{linecolor=darkGreen,linewidth=0.5pt}

\section{Preliminaries}

We consider an arena $A$ and make the following two assumptions on it:
\begin{itemize}
\item (A1) For $A \neq \bot$ (the arena with a single initial question), each question move in the arena enables at least one answer move.
\item (A2) Answer moves do not enable any other move.
\end{itemize}

An arena is said to be \defname{prime} if it has a single initial move; a type is prime if its arena denotation is prime.

\subsection{Node-order}

\subsubsection{Definition}

We define the \defname{order of a move} $m$ in the arena $A$, written $\ord_A{m}$ (or just $\ord{m}$ where there is no ambiguity), as the length of the path from $m$ to its furthest leaf in $A$ minus 1
({\it i.e.}~the height of the subarena rooted at $m$ minus 2.). Because of assumptions (A1) and (A2),
for any move $m$ of $A \neq \bot$, $m$ is a question move if and only if $\ord{m} \geq 0$, and $m$ is an answer move if and only if $\ord{m} = -1$.

The \defname{order of an arena} $A$ is defined to be the maximal order of its initial moves. The order of a (simple, PCF or IA) type is defined as the order of the arena denoting it - or equivalently as 0 for ground type, $\ord{A\rightarrow B} = \max(1+\ord{A},\ord{B})$ and $\ord(A\times B) = \max(\ord A, \ord B)$. The order of a term is the order of its type.


\subsubsection{Node-order after composition}

Consider the arena $X\lingamear Y$ and let $m$ be a move of
$X\lingamear Y$. We write $\ord_{X\lingamear Y}{m}$ to denote the
order of $m$ in the arena ${X\lingamear Y}$. If $m$ belongs to $X$
(resp.~$Y$) then we write $\ord_X{m}$ (resp.~$\ord_Y{m}$) to denote
the order of the move $m$ in the arena $X$ (resp.~$Y$).

\begin{lemma}
\label{lem:compositionorder} Let $A$, $B$ and $C$ be three arenas.
We have:
$$\begin{array}{lll}
\forall m \in A:
    &  \ord_{A\lingamear B}{m} = \ord_{A\lingamear C}{m} \ ,\\
\forall m \in B:
    & \ord_{A\lingamear B}{m} \geq \ord_{B\lingamear C}{m}  & \mbox{for $m$ initial,}\\
    & \ord_{A\lingamear B}{m} = \ord_{B\lingamear C}{m} & \mbox{for $m$ non initial,} \\
\forall m \in C:
    & \ord_{A\lingamear C}{m} \geq \ord_{B\lingamear C}{m} \iff
\ord{A} \geq \ord{B}\ & \mbox{for $m$ initial,}\\
    & \ord_{A\lingamear C}{m} = \ord_{B\lingamear C}{m}   & \mbox{for $m$ non initial.}
\end{array}
$$
\end{lemma}

\subsection{Well-bracketing}

We call \defname{pending question} of a sequence of moves $s \in L_A$ the last unanswered question in $s$.

\begin{definition}\rm
A strategy $\sigma$ is said to be \defname{P-well-bracketed} if for any play $s \, a \in \sigma$ where $a$ is a  P-answer, $a$ points to the pending question in $s$.
\end{definition}



P-well-bracketing can be restated differently as the following proposition shows:
\begin{proposition}
\label{prop:char_wellbrack}
\rm We make assumption (A1) and (A2).
Let $\sigma$ be a strategy on an arena $A\neq \bot$.
The following statements are equivalent:
\begin{enumerate}
\item[(i)] $\sigma$ is P-well-bracketed,
\item[(ii)] for $s \, a \in \sigma$ with $a$ a P-answer, $a$ points to the pending question in $\pview{s}$,
\item[(iii)] for $s \, a \in \sigma$ with $a$ a P-answer, $a$ points to the last O-question in $\pview{s}$,
\item[(iv)] for $s \, a \in \sigma$ with $a$ a P-answer, $a$ points to the last O-move in $\pview{s}$ with order $>\ord{a}$.
\end{enumerate}
\end{proposition}
\begin{proof}
$(i)\iff(ii)$: \cite[Lemma 2.1]{McC96b} states that if P is to move then the pending question in $s$ is the same as that of $\pview{s}$.

$(ii)\iff(iii)$: Assumption (A2) implies that the pending question in $\pview{s}$ is also the last O-question occurring in $\pview{s}$.

$(iii)\iff(iv)$: Because of assumption (A1) and (A2),
for any move $m$, we have $m$ is a question move
if and only if $\ord{m} \geq 0$ if and only if $\ord{m} > \ord{a} = -1$.
\end{proof}




\begin{lemma}
\label{lem:justfied_by_unanswered}
Under assumption (A2), if $s$ be a justified sequence of moves satisfying alternation and visibility then any O-move (resp. P-move) in $s$ points to an \emph{unanswered} P question (resp. O-question).
\end{lemma}
\begin{proof}
Suppose that an O-move $c$ points to a P-move $d$ that has already been answered by the O-move $a$. The sequence $s$ as the following form:
$$ s= \ldots \Pstr{(d){d}  \ldots  (a-d,20){a}  \ldots  (c-d,20){c}}$$

By O-visibility, $d$ must belong to $\oview{s_{<c}}$. But since $a$ is an answer, by assumption (A2), it cannot justify any P-move, therefore
$\oview{s_{<q}}$ must contain an OP-arc ``hoping'' over $a$. We name the nodes of this arc $d^1$ and $c^1$:
$$ s = \ldots \Pstr[0.7cm]{(d){d}  \ldots  (d1){d^1} \ldots (a-d,20){a} \ldots
 (c1-d1,20){c^1} \ldots (c-d,25){c}}$$

By P-visibility, $d^1$ must belong to $\pview{s_{<c^1}}$. Consequently, $a$ does not belong to $\pview{s_{<c^1}}$ (otherwise the PO-arc $\Pstr[0.5cm]{(d){d} \quad (a-d,45){a}}$ would cause the P-view to jump over $d^1$).
Therefore there must be a PO-arc $\Pstr[0.5cm]{(d2){d^2} \quad (c2-d2,45){c^2}}$ in $\pview{s_{<c^1}}$ hoping over $a$:
$$ s = \ldots \Pstr[0.7cm]{(d){d}  \ldots
(d1){d^1} \ldots (d2){c^2} \ldots
(a-d,20){a} \ldots
 (c2-d2,20){d^2} \ldots (c1-d1,20){c^1} \ldots (c-d,25){c}}$$

This process can be repeated infinitely often by using alternatively O-visibility and P-visibility. This gives a contradiction since the sequence of moves $s_{<c}$ has finite length.
Hence $d$ cannot point to a question that has already been answered. Since, by assumption (A2), a question is enabled by another question, $d$ is necessarily justified by an unanswered question.
\end{proof}


\begin{lemma}
\label{lem:oq_in_pview_unanswered}
Under assumption (A2), if $s$ is a P-well-bracketed justified sequence of moves of odd length satisfying alternation and visibility then  all O-questions occurring in $\pview{s}$ are unanswered in $s$.
\end{lemma}
\begin{proof}
We proof the first part by induction on $s$.
The base case ($s = q$ with $q$ initial O-move) is trivial.

Suppose $\Pstr[0.4cm]{ s = s' \cdot (n)n \cdot u \cdot (m-n,45){m} }$.
Let $r$ be an O-question in $\pview{s} = \pview{s'} \cdot n \cdot m$.
If $r$ is the last move $m$ then it is necessarily unanswered.
If $r \in \pview{s'}$ then by the induction hypothesis, $r$ is unanswered in $s'$.
Suppose that $r$ is answered in $s$. This implies that some answer move $a$ in $u$ points to $r$:
$$\pstr[0.7cm][5pt]{ s = \underbrace{\cdots\ \nd(r){r}^O \cdots }_{s'} \
\nd(n){n}^P \ \underbrace{\cdots\ \nd(a-r,35){a}^P \cdots }_{u} \
\nd(m-n,30){m}^O } \ .$$

Since $m$ points to $n$, by lemma \ref{lem:justfied_by_unanswered}, $n$ is still unanswered at $s_{\prefixof a}$. Therefore the pending
question at $s_{\prefixof a}$ cannot be $r$. But $a$ is justified by $r$, therefore the well-bracketing condition is violated. Hence $r$ is
unanswered in $s$.
\end{proof}








\subsection{Interaction sequences} Let us first recall the
definition of an interaction sequence. Let $A$,$B$ and $C$ be three
games. We say that $u$  is an
\defname{interaction sequence} of $A$,$B$ and $C$ whenever $u\filter
A,B$ is a valid position of the game $A\lingamear B$ (i.e.~$u\filter
A,B \in P_{A\lingamear B}$) and  $u\filter B,C$ is a valid position
of the game $B\lingamear C$. We write $Int(A,B,C)$ to denote the set
of all such interaction sequences.

Let $\sigma:A\lingamear B$ and $\mu:B\lingamear C$ be two
strategies. We write $\sigma \parallel \mu$ to denote the set of
interaction sequences that unfold according to the strategy $\sigma$
in the $A,B$-projection of the game and to $\mu$ in the
$B,C$-projection:
$$ \sigma \parallel \mu = \{ u \in Int(A,B,C) \ | \ u\filter A,B \in \sigma \wedge u \filter B,C \in \mu \} \ .$$
The composite of $\sigma$ and $\mu$ is then defined as $\sigma ; \mu
= \{ u \filter A,C \ | \ u \in \sigma \parallel \tau \}$.

The diagram below shows the structure of an interaction sequence
from $\sigma \parallel \mu$. There are four states represented by
the rectangular boxes. The content of the state shows who is to play
in each of the game $A\lingamear B$, $B\lingamear C$ and
$A\lingamear C$. For instance in state $OPP$, it is O's turn to play
in $A\lingamear B$ and P's turn to play in $B\lingamear C$ and
$A\lingamear C$. Arrows represent the moves. When specifying
interaction sequence, the following bullet symbols are used to
represent moves: $\pmove$ for P-moves, $\omove$ for O-moves,
$\pomove$ for a move playing the role of P in $A\lingamear B$ and O
in $B\lingamear C$ and $\opmove$ for the symmetric of $\pomove$. We
sometimes add a subscript to the symbols $\pmove$ and $\omove$ to
denote the component in which the moves is played ($A$ or $C$).


\tikzstyle{state}=[rectangle,draw=blue!50,fill=blue!20,thick,minimum
height = 4ex, text width=4cm] \tikzstyle{move}=[->,shorten
<=1pt,>=latex',line width=1pt] \tikzstyle{intmove}=[dashed]
\tikzstyle{extomove}=[color=\extomovecolor]
\tikzstyle{genomove}=[]%[dashed]
\tikzstyle{genpmove}=[color=\genpmovecolor]
\def\sep{1.5cm}
\begin{figure}[htbp]
\begin{center}
\begin{tikzpicture}[node distance=1.7cm]

% the four states
\path
 node(oooT)  [state] {}
 node(opp)   [state, below of=oooT] {}
 node(pop)   [state, below of=opp]  {}
 node(oooB)  [state, below of=pop] {}
 node(title) [anchor=south, at=(oooT.north), minimum height = 4ex, text width=4cm] { };

\path
% text in the title centered in 3 columns
  ([xshift=-\sep]title) node {$A\lingamear B$}
        (title) node {$B\lingamear C$}
        ([xshift=\sep]title) node {$A\lingamear C$}

% text in the states centered in 3 columns
  ([xshift=-\sep]oooT) node {O}
        (oooT) node {O}
        ([xshift=\sep]oooT) node {O}
  ([xshift=-\sep]opp) node {O}
        (opp) node {P}
        ([xshift=\sep]opp) node {P}
  ([xshift=-\sep]pop) node {P}
        (pop) node {O}
        ([xshift=\sep]pop) node {P}
  ([xshift=-\sep]oooB) node {O}
        (oooB) node {O}
        ([xshift=\sep]oooB) node {O}

% text in between two arrows giving the arena of the move
  (oooT) to node {\bf C} (opp)
  (opp) to node {\bf B} (pop)
  (pop) to node {\bf A} (oooB)

% arrows representing the moves
  (opp.20)    edge[move, genpmove]
        node[right] {$\mu$}
        node[left]{$\pmove$} (oooT.-20)
  (oooT.-160) edge[move, extomove, genomove]
        node[left] {$env_\mu$}
        node[right]{$\omove$} (opp.160)
  (pop.20)    edge[move, genomove,genpmove,intmove]
        node[right] {$\sigma$}
        node[left]{$\pomove$} (opp.-20)
  (opp.-160)  edge[move, genomove, genpmove,intmove]
        node[left] {$\mu$}
        node[right]{$\opmove$}  (pop.160)
  (oooB.20)   edge[move, extomove,genomove]
        node[right] {$env_\sigma$}
        node[left]{$\omove$} (pop.-20)
  (pop.-160)  edge[move, genpmove]
        node[left] {$\sigma$}
        node[right]{$\pmove$} (oooB.160);

%\draw[move, genpmove] (3.5cm,-1cm) -- +(1,0) node[right] {Generalised P-move \& External P-move };
%\draw[move, genomove,genpmove] (3.5cm,-2cm) -- +(1,0) node[right] {Generalised O-move \& Generalised P-move};
%\draw[move, genomove,extomove] (3.5cm,-3cm) -- +(1,0) node[right] {Generalised O-move \& External O-move};
\draw[move] (3.5cm,-1cm) -- +(1cm,0cm) node[right] {External move};
\draw[move,intmove] (3.5cm,-2cm) -- +(1cm,0cm) node[right] {Internal
move}; \draw (3.5cm,-3cm) node[anchor=west]
{\textcolor{\extomovecolor}{External O-moves: $\omove$}}; \draw
(3.5cm,-4cm) node[anchor=west]
{\textcolor{\genpmovecolor}Generalised P-move: $\opmove, \pomove,
\pmove$};
\end{tikzpicture}
\end{center}
\caption{Structure of an interaction sequence.} \label{fig:interseq}
\end{figure}

Note that in state OPP, the alternation condition (for each of the
three games involved) prevents the players from playing in A.
Indeed, the O-moves in component $A$ of $A\lingamear B$ are also
$O$-moves in component $A$ of $A\lingamear C$ however the state name
indicates that the next move in $A\lingamear B$ must be an O-move
and the next move in $A\lingamear C$ must be a P-move.

Similarly, in the top state OOO, the players cannot make move in B
since the O-moves in component B of the game $B\lingamear C$
correspond to P-moves in the component B of $A\lingamear B$. However
the state name indicates that the next move in $A\lingamear B$ and
the next move in $B\lingamear C$ must be played by O.


Let $u \in Int(A,B,C)$ and $m$ be a move of $u$. The
\defname{component} of $m$ is $A,B$ if after playing $m$ the game is
under the control of the strategy $\sigma$ and $B,C$ otherwise (if
$\mu$ has control). In other words, the moves $\omove, \pmove \in A$
and $\opmove \in B$ shown on the diagram of Figure
\ref{fig:interseq} have component $A,B$ and $\omove, \pmove \in C$
and $\pomove \in B$ have component $B,C$.


Also we call \defname{generalized O-move in component $A,B$} moves
that play the role of O in the game $A\lingamear B$, that is to say
moves represented by $\opmove$ and $\omove_A$. Similarly $\pomove$
and $\pmove_A$ moves are the \defname{generalized P-moves in
component $A,B$}, $\omove_C$ and $\pomove$ moves are the
\defname{generalized O-moves in component $B,C$} and  $\pmove_C$ and
$\opmove$ moves are the \defname{generalized P-moves in component
$B,C$}.

The P-view (also called \emph{core} in
\cite{McCusker-GamesandFullAbstrac}) of an interaction sequence $u
\in Int(A,B,C)$, written $\overline{u}$ or $\pview{u}$ is defined
as:
\begin{align*}
\pview{u\cdot \extomove{n}} &= \extomove{n} &
\mbox{ if \extomove{$m$} is an \extomove{external O-move} initial in C,}\\
\pview{\Pstr{u\cdot (m)m\cdot v \cdot (n-m,45){\extomove{n}} }} &= \extomove{n} &\mbox{ if \extomove{$m$} is an \extomove{external O-move} non initial in C,}\\
\pview{u \cdot \genpmove{m}} &= \pview{u}\cdot \genpmove{m}  & \mbox{ if \genpmove{$m$} is a \genpmove{generalised P-move}.}\\
\end{align*}

We can show the following property by an easy induction :
\begin{lemma}
\label{lem:pviewAC_eq_ACpview}
 Let $u$ be an interaction sequence in $Int(A,B,C)$ then
$$\pview{u} \filter A,C = \pview{u \filter A,C} \ .$$
\end{lemma}
\begin{proof}
  By induction on $u$. It is trivial for the empty sequence.
Let $b$ be a move in $B$. We have $\pview{u b} \filter A,C =
\pview{u} \filter A,C$. By the I.H.\ this is equal to $\pview{u
\filter A,C} = \pview{u b\filter A,C}$. Let $m$ be a P-move in $A$
or $C$ then $\pview{u m} \filter A,C = (\pview{u} \filter A,C) m$
and by the I.H.\ this is equal to $\pview{u \filter A,C} m =
\pview{(u \filter A,C) m} = \pview{u m \filter A,C}$. Let $c$ be an
initial move in $C$. We have $\pview{u c \filter A,C}  = \pview{(u
\filter A,C) c} = c =  c \filter A,C = \pview{u c} \filter A,C$. Let
$u = \Pstr{u_1 (m){m} u_2 (n-m){n}}$ with $n$ an O-move in
$A\rightarrow C$. Then necessarily $m\in A,C$ and $ \pview{u\filter
A,C} = \pview{\Pstr[0.5cm]{u_1\filter A,C \cdot (m){m} \cdot
u_2\filter A,C \cdot (n-m,30){n}}} =
 \pview{u_1 \filter A,C} \Pstr{(m){m} (n-m){n}}$. By the I.H.\ this is equal to
$(\pview{u_1}\filter A,C) \Pstr{(m){m} (n-m){n}} = (\pview{u_1}
\Pstr{(m){m} (n-m){n}} ) \filter A,C  = \pview{u_1 \Pstr{(m){m} u_2
(n-m){n}}} \filter A,C$
\end{proof}


\subsection{P-incremental justification}


\begin{definition}\rm
A play $s m$ of even length is said to be \defname{P-incrementally
justified}, or \emph{P-i.j.} for short, if $m$ points to the last
unanswered O-question in $\pview{s}$ with order strictly greater
than $\ord{m}$.

 A strategy $\sigma$ is said to be \defname{P-incrementally justified}, if all plays in $\sigma$ ending with a P-question are
P-incrementally justified.
\end{definition}
Let $\sigma$ be a strategy. We write $Pij(\sigma)$ to denote the set of plays of $\sigma$ that are P-i.j.
We can define equivalently P-i.j.\ strategies as those verifying the relation $\sigma = Pij(\sigma)$.
\begin{proposition}
\label{prop:char_pincr}
\rm We make assumption (A1) and (A2).
Let $\sigma$ be a \emph{P-well-bracketed} strategy on an arena $A\neq \bot$.
The following statements are equivalent:
\begin{enumerate}
\item[(i)] $\sigma$ is P-incrementally justified,
\item[(ii)] for $s \, q \in \sigma$ with $q$ a P-question, $q$ points to the last O-question in $\pview{s}$ with order $>\ord{q}$,
\item[(iii)] for $s \, q \in \sigma$ with $q$ a P-question, $q$ points to the last O-move in $\pview{s}$ with order $>\ord{q}$.
\end{enumerate}
\end{proposition}
\begin{proof}
$(i)\iff(ii)$: By lemma \ref{lem:oq_in_pview_unanswered}, O-question occurring in $\pview{s}$ are all unanswered.

$(ii)\iff(iii)$: Because of (A1) and (A2), $\ord{q} \geq 0$ thus an O-move with order $>\ord{q}$ is necessarily an O-question.
\end{proof}

Putting proposition \ref{prop:char_pincr} and
\ref{prop:char_wellbrack} together we obtain:
\begin{proposition}
Under assumption (A1) and (A2).
A strategy $\sigma$ on $A\neq \bot$
is \emph{P-well-bracketed} and
 \emph{P-incrementally justified} if and only if
for $s \, m \in \sigma$, $m$ points to the last O-move in $\pview{s}$ with order $>\ord{m}$.
\end{proposition}




\section{Closed P-i.j.\ strategies}
\label{sec:closedpij}

\subsection{Definition}

\begin{definition}
\label{def:closedpij} Let $s m$ be an even-length play on some game
$A \rightarrow B$. $s m$ is said to be
\defname{closed P-incrementally justified} (closed P-i.j.\ for short)
just if
\begin{itemize}
\item $s m$ is P-incrementally justified;
\item and if $m$ is an initial move in $A$ then its justifier $n$ (initial in
$B$) verifies $\ord_A m \geq \ord_B n$.
\end{itemize}

\noindent A strategy $\sigma$ is \defname{closed P-i.j.} just if all
plays in $\sigma$ ending with a P-questions are closed P-i.j.
\end{definition}
An example of closed P-i.j.\ strategy is the identity strategy $id_A$
for any game $A$.

\begin{lemma}
\label{lem:closedpij_singleBinitmove} Let $\sigma : A \lingamear B$
be a P-i.j.\ strategy.
\begin{enumerate}[i.]
\item If for each initial move $m$ of $A$ occurring in some play of $\sigma$ we have $\ord_A m \geq \ord{B}$, then $\sigma$ is closed P-i.j.
\item Suppose that $A=A_1\times \ldots \times A_n$ where each of the $A_i$ are prime arenas. If for each initial move $m_i$ of $A_i$, for $i \in \{1..n\}$, occurring in some play of $\sigma$ we have $\ord A_i \geq \ord{B}$, then $\sigma$ is closed P-i.j.
\end{enumerate}
\end{lemma}
\begin{proof}
(i) This is a direct consequence of the definition since $\ord B \geq \ord_B b$ for every move $b$ initial in $B$.

(ii) Take an initial move $m$ of $A$. It is necessary an initial move of $A_i$ for some $i$ hence $\ord_A m = \ord_{A_i} m$ which is equal to $\ord A_i$ since $A_i$ is prime. By hypothesis this is in turn greater than $\ord{B}$ hence we can conclude using (i).
\end{proof}



We observe that every P-i.j.\ strategy $\sigma$ on the game $I
\lingamear A$ is closed P-i.j.\ while $\sigma : A$ is not
necessarily closed P-i.j.\footnote{In particular, every P-i.j.\
strategy $\sigma$ on the game $!A_1 \otimes \ldots \otimes !A_n
\lingamear B$, is isomorphic, up to arena-tagging of the moves, to
the closed P-i.j.\ strategy $\Lambda^n(\sigma)$ on the game $I
\lingamear (A_1,\ldots,A_n,B)$, where $\Lambda$ denotes the usual
{\it currying} isomorphism.}; hence the distinction between $I
\lingamear A$ and $A$ matters. This is because the definition of
closed P-i.j.\ strategy specifically refers to the moves of  the
arena in the left-hand side of the function space arrow
$\lingamear$, therefore the property is not valid up to an
isomorphism that retags the moves such as {\it currying}.

Consequently, it is possible to have two isomorphic strategies $\sigma$ and
$\mu$ such that one is closed P-i.j.\ but not the other. In contrast, the ``ordinary'' P-incremental
justification condition is preserved across the  {\it curry} isomorphism. A consequence of this remark is that the category of closed P-i.j.\ strategies
that we will introduce later on, is not closed (neither monoidal closed nor cartesian closed) and
that it only admits a weak form of {\it curry} isomorphism.

\subsection{Compositionality - A semantic proof}

{\bf Notation} In plays representations, the symbol $\omove$ stands
for an O-move and $\pmove$ for a P-move. Suppose the game considered
is $L\lingamear R$ for some game $L$ and $R$ then whenever the
sub-arena in which the move is played is known, it is specified in
subscripts ($\omove_L$, $\pmove_L$, $\omove_R$ or $\pmove_R$). For
interaction sequences in $Int(A,B,C)$ we use the symbols $\omove_A$,
$\pmove_A$, $\omove_C$, $\pmove_C$, $\opmove$ and $\pomove$ as
defined in Figure \ref{fig:interseq}. We use the variable $X$ to
denote one of the component $A,B$ or $B,C$, the variable  $Y$ then
denotes the other component. We write $s \subseqof t$ to say that
$s$ is a subsequence (with pointers) of $t$, $s \prefixof t$ to say
that $s$ is a prefix (with pointers) of $t$ and  $s \suffixof t$ to
say that $s$ is a suffix of $t$.

We now prove several useful lemmas which will become useful when studying compositionality of P-i.j.\ strategies.

\begin{lemma}
\label{lem:interjump}
Let $X$ be a component (either  $A,B$ or  $B,C$).
Let $u$ be an interaction sequence of the form
$ u =
\Pstr[0.5cm][2pt]{ \ldots (b){\stk \beta \pmove}  \ldots
 {n}  \ldots  (a-b,30){\stk \alpha\omove}
\ldots m}$ where:
\begin{itemize}[-]
\item $\alpha,\beta$ are external moves in component $X$ (necessarily both played in $A$ or in $C$),
\item  $m$ is either played in $B$ or an external P-move in $X$,
\item  $\alpha$ is visible at $m$ in $X$ \emph{i.e.}~$\alpha\in \pview{u \filter X}$ (consequently $\beta$ is also visible).
\end{itemize}
Then $n \not\in \pview{u \filter A, C}$.
\end{lemma}
\begin{proof}
Since $\alpha$ is an O-move, $\alpha$ and $\beta$ are necessarily
played in the same arena ($A$ or $C$). Take $v=u$ if $m$ is a
generalized O-move in $X$ and $v=u_{<z}$ otherwise (if $m$ is a
generalized P-move in $X$). The third assumption implies
$\alpha,\beta\in \pview{v}$. The last move in $v$ is necessarily a
generalized O-move in component $X$ (see diagram of Figure
\ref{fig:interseq}) therefore by \cite[Lemma 3.3.1]{Harmer2005} we
have $\pview{v \filter X} = \pview{\overline{v} \filter X} \subseqof
\overline{v} \subseqof \overline{u}$. Thus $\alpha,\beta \in
\overline{u}$ and since $\alpha,\beta$ are played in $A,C$ we have
$\alpha,\beta  \in \overline{u} \filter A,C = \pview{u
\filter A,C}$ (Lemma \ref{lem:pviewAC_eq_ACpview}). Finally
since $n$ lies underneath the $\beta$-$\alpha$ PO-arc it cannot
appear in the P-view  $\pview{u \filter A,C}$.
\end{proof}

\begin{lemma}
\label{lem:in_pviewAC_imp_in_pviewX}
Let $u$ be an interaction sequence in $Int(A,B,C)$ and
$n$ be a move of $u$ such that $n\in\pview{u \filter A,C}$:
\begin{enumerate}[i.]
\item
if all the moves in $u_{\suffixof n}$
are played in $C$  then $n \in \pview{u \filter B,C}$;
\item
if all the moves in $u_{\suffixof n}$ are played in $A$ then $n \in \pview{u \filter A,B}$.
\end{enumerate}
\end{lemma}
\begin{proof}
\begin{enumerate}[(i)]
\item
We show the contrapositive. Suppose that $n \not\in\pview{u \filter B,C}$. This must be due to one of the following  two
reasons:
\begin{itemize}[-]
\item $\pview{u \filter B,C}$ contains an initial move $c_0 \in C$
occurring after $n$ in $u$.


By \cite[Lemma 3.3.1]{Harmer2005}
we have $\pview{u \filter B,C} = \pview{\overline{u} \filter B,C} \subseqof \pview{u}$, thus $c_0$ also occurs in $\pview{u}$.
Since $c_0$ belongs to $C$ we have
$c_0 \in \pview{u} \filter A,C=
\pview{u \filter A,C}$ (Lemma \ref{lem:pviewAC_eq_ACpview}).
Thus the P-view $\pview{u \filter A,C}$
starts with the initial move $c_0$ and
since $n$ occurs before $c_0$, $n$ does not occur in the P-view.

\item $n$ lies underneath a PO-arc $\beta$-$\alpha$ visible
at $ u \filter B,C$.
By assumption, since $\alpha$ occurs after $n$ in $u$, it must belong to $C$. We can therefore apply Lemma \ref{lem:interjump}
with $X\assignar B,C$ which gives
$n \not\in\pview{u \filter A,C}$.
\end{itemize}

\item Suppose that $n \not\in\pview{u \filter A,B}$ then either:
\begin{itemize}[-]
\item $\pview{u \filter A,B}$ contains an initial move $b_0 \in B$
occurring after $n$ in $u$. But this is impossible since by assumption all the moves occurring after $n$ in $u$ belong to $A$.

\item or $n$ lies underneath a PO-arc $\beta$-$\alpha$ in $A,B$.
By assumption, since $\alpha$ occurs after $n$ it must belong to $A$. We can then conclude using
Lemma \ref{lem:interjump} with $X\assignar A,B$.
\end{itemize}
\end{enumerate}
\end{proof}

Note that we cannot completely relax the assumption
which says that moves in $u_{\suffixof n}$ are all in the same component.
For instance take $u = \Pstr[0.5cm]{(co){\omove_C}\thinspace
(b0-co){\opmove} \thinspace
(n){\stk{\pmove_A}{n}} \thinspace
(b1-co){\opmove}}$ then we have $n\in\pview{u\filter A,C}$ but $n\notin\pview{u\filter A,B}$.


%%%%%%%%%%%
% This commented Lemma could be useful be we did not make use of it eventually.
%
% \begin{lemma}
%\label{lem:oviewsegmentinB}
%For any legal sequence $s = \ldots x \cdot r \cdot y$ of a game $A\lingamear B$ if $x, y \in A$ and $x$ is O-visible from $y$ then any move in $r$ occurring in $\oview{s}$ belongs to $A$.
%\end{lemma}
%\begin{proof}
%We proceed by induction on the length of the segment $r$.
%Base case $r=\epsilon$ is trivial. Suppose $r = r' \cdot m$.
%If $y$ is an O-move then by the Switching Condition
%$m$ is necessarily in $A$. Clearly $x$ is O-visible from $m$ thus  by the I.H.\ any move from $r$ occurring in the O-view is in $A$.
%
%If $y$ is a P-move then it cannot point to an initial move in $B$. Indeed, suppose that it points to an initial O-move $b_0 \in B$ then
%we have $\oview{s} = b_0 \cdot y$ which contradicts the fact that $x\in \oview{s}$.
%Thus $y$ points to a move in $A$ and again we can conclude using the induction hypothesis.
%\end{proof}


\begin{lemma}[P-visibility decomposition (from $C$)]
\label{lem:middlepomove}
Let $u = \ldots n' \cdot r \cdot m \in Int(A,B,C)$ where
$n'$ is a $\omove_A$-move verifying $n' \in \pview{u\filter A,C}$ and $m$ is in $\{ \pmove_C, \opmove, \pomove \}$. Then there is a $\pomove$-move $\gamma$ in $r \cdot m$ such that $\gamma \in \pview{u\filter B,C}$ , $n' \in \pview{u_{\leq \gamma} \filter A,B}$ and $\gamma$ is justified by a move occurring before $n'$.
\end{lemma}
\begin{proof}
By induction on $|r|$.
If $r=\epsilon$ then necessarily $u = \ldots \stk{\omove_A}{n'} \thinspace\stk \pomove m$ where $m$ points before $n'$ ($n'$ being played in $A$ cannot justify $m$ played in $B$) so we just need to take $\gamma = m$.
If $|r|=1$ then either
$u = \ldots \stk{\omove_A}{n'} \pomove\thinspace\stk {\pmove_C} m$
or $u = \ldots \stk{\omove_A}{n'} \pomove\thinspace\stk \opmove m$.
In both cases we can take $\gamma$ to be the $\pomove$-move between $n'$ and $m$.
Suppose $|r|>1$. Let $m^-$ denote the move preceding $m$ in $u$.
We proceed by case analysis:
\begin{enumerate}[i.]
\item Suppose $m = \pmove_C$ and $m^- = \omove_C$.
Let $q$ be the external P-move that justifies $m^-$.
Since $n' \in \pview{u\filter A,C}$, $q$ must occur after $n'$ in $u$:
$$
\begin{array}{ccccl}
A & \stackrel\sigma{\longrightarrow} & B & \stackrel\mu{\longrightarrow} & C \\
&\vdots&&\vdots\\
n' \omove\\
&\vdots&&\vdots  \\
&& & &  \rnode{q}{\pmove}q  \\
&\vdots&&\vdots  \\
&& & &  \rnode{mp}{\omove}m^-  \\
&& & &  \rnode{m}{\pmove}m  \\
\end{array}
\ncarc[arcangleA=60,arcangleB=60]{->}{mp}{q}
 $$
Thus we can use the induction hypothesis (with $u\assignar u_{\prefixof q}$): there is a $\pomove$-move $\gamma$
in $u_{]n',q]}$ pointing before $n'$ such that $\gamma \in \pview{u_{\prefixof q} \filter B,C}$, $n' \in \pview{u_{\prefixof \gamma} \filter A,B}$.
Moreover $\pview{u_{\prefixof q} \filter B,C} \prefixof \pview{u_{\prefixof m} \filter B,C}$ (since $q$ is visible from $m$ in $B,C$) thus we have $\gamma \in \pview{u_{\prefixof m} \filter B,C}$ as required.

\item Suppose $m = \pmove_C$ and $m^- = \pomove \in B$.
Again we can conclude using
the induction hypothesis with $u \assignar u_{\prefixof m^-}$.

\item Suppose $m = \pomove \in B$.

Suppose that all the moves in $r$ are in $A$.
Then $r$ is of the form $(\pmove_A \omove_A)^*$ (where $(\cdot)^*$ denotes the Kleenee star operator).
We just need to take $\gamma = m$.
Indeed, moves in $u_{\suffixof m}$ are all in $A$
and by assumption $n'\in\pview{u\filter A,C}$  therefore
Lemma \ref{lem:in_pviewAC_imp_in_pviewX}(ii) gives
$n'\in\pview{u\filter A,B}$.
Also, since $m$ is a $\pomove$-move,
its justifier is a $\opmove$-move but $r$ contains only $\omove$ and $\pmove$ moves hence $m$'s justifier must occur before $n'$.

Suppose that $r$ contains at least one move in $B$. Let $b$ be the last such move, then $u$ is of the form $\ldots n' \cdot \ldots \cdot \stk\opmove  b \cdot (\pmove_A \omove_A)^* \cdot\thinspace\stk\pomove m $. We then have
$u\filter B,C = \ldots n' \cdot \ldots \cdot
\thinspace\stk\opmove b \thinspace\cdot \stk\pomove m $ thus $b \in \pview{u\filter B,C}$. We can then conclude by applying the induction hypothesis with $u \assignar u_{\prefixof b}$.

\item Suppose $m = \pomove \in B$.
If $m^- = \opmove \in B$ then the I.H.\ with $u \assignar u_{\prefixof m^-}$ permits us to conclude.
If $m^- = \omove \in C$ then we conlude by applying  the I.H.\ on $u \assignar u_{\prefixof q}$ where $q$ is the external P-move in $C$ justifying
$m^-$.
\end{enumerate}
\end{proof}

We now show the lemma symmetric to the previous one:
\begin{lemma}[P-visibility decomposition (from $A$)]
\label{lem:middleopmove}
Let $u = \ldots n' \cdot r \cdot m \in Int(A,B,C)$ where
$n'$ is an O-move \emph{non initial} in $C$ verifying $n' \in \pview{u\filter A,C}$ and $m$ is in $\{\pmove_A, \opmove, \pomove\}$. Then there is a $\opmove$-move $\gamma$ in $r \cdot m$ such that $\gamma \in \pview{u\filter A,B}$ , $n' \in \pview{u_{\leq \gamma} \filter B,C}$ and $\gamma$ is justified by a move occurring before $n'$.
\end{lemma}
\begin{proof}
The proof is almost symmetrical to the previous one (Lemma \ref{lem:middlepomove}). We proceed by induction on $|r|$.
If $r=\epsilon$ then necessarily $u = \ldots \stk {\omove_C} {n'} \thinspace\stk \opmove m$ where $m$ points before $n'$ (it cannot point to $n'$
since $n'$ is not initial in $C$). Thus we just need to take $\gamma = m$.

If $|r|=1$ then either
$u = \ldots \stk {\omove_C} {n'} \thinspace\opmove\thinspace\thinspace\stk{\pmove_A} m$
or $u = \ldots \stk {\omove_C} {n'} \thinspace\opmove\thinspace\thinspace\stk \pomove m$.
In both cases we can take $\gamma$ to be the $\opmove$-move between $n'$ and $m$.
Suppose $|r|>1$. Let $m^-$ denote the move preceding $m$ in $u$.
We do a case analysis:
\begin{enumerate}[i.]
\item Suppose $m = \pmove_A$ and $m^- = \omove_A$.
Let $q$ be the external P-move that justifies $m^-$.
Since $n' \in \pview{u\filter A,C}$, $q$ must occur after $n'$ in $u$:
$$
\begin{array}{rcccl}
A & \stackrel\sigma{\longrightarrow} & B & \stackrel\mu{\longrightarrow} & C \\
&\vdots&&\vdots\\
&&&& \omove\ n'\\
&\vdots&&\vdots  \\
q\rnode{q}{\pmove}  \\
&\vdots&&\vdots  \\
m^- \rnode{mp}{\omove}  \\
m \rnode{m}{\pmove}  \\
\end{array}
\ncarc[arcangleA=-45,arcangleB=-45]{->}{mp}{q}
 $$
Thus we can use the induction hypothesis (with $u\assignar u_{\prefixof q}$): there is a $\opmove$-move $\gamma$
in $u_{]n',q]}$ pointing before $n'$ such that $\gamma \in \pview{u_{\prefixof q} \filter A,B}$, $n' \in \pview{u_{\prefixof \gamma} \filter B,C}$.
Moreover $\pview{u_{\prefixof q} \filter A,B} \prefixof \pview{u_{\prefixof m} \filter A,B}$ (since $q$ is visible from $m$ in $A,B$) thus we have $\gamma \in \pview{u_{\prefixof m} \filter A,B}$ as required.

\item Suppose $m = \pmove_A$ and $m^- = \pomove$ then again we can conclude using the I.H.\ with $u \assignar u_{\prefixof m^-}$.

\item Suppose $m = \opmove$.
\begin{itemize}[-]
\item Suppose that $r$ does not contain any move in $B$  then $r$ is of the form $(\pmove_C \omove_C)^*$.

We just need to take $\gamma = m$.
Indeed:
\begin{enumerate}
\item By lemmma \ref{lem:in_pviewAC_imp_in_pviewX}(i)
we have $n'\in \pview{u\filter B,C}$.

\item  $m$ is justified by a move occurring before $n'$.
Indeed, if $m$ is justified by a $\pomove$-move then since $n' \cdot r$ contains only $\omove$ and $\pmove$ moves, $m$'s justifier must occur before $n'$.
If $m$'s justifier is an initial $\omove_C$-move $c_i$, then
by P-visibility we have $c_i \in \pview{u\filter B,C}$
but since the P-view computation ``stops'' when reaching an initial moves, in order to guarantee that $n'$ also belongs to the P-view (as shown in (a)) it must
occurs after $c_i$.
\end{enumerate}


\item Suppose that $r$ contains some move in $B$. Let $b$ be the last such move. Then $u$ is of the form $u = \ldots n' \cdot \ldots \cdot \stk\opmove  b \cdot (\pmove_A \omove_A)^* \cdot\ \stk\pomove m $.
So we have
$u\filter B,C = \ldots n' \cdot \ldots \cdot \stk\opmove  b \cdot \stk\pomove m $ hence $b \in \pview{u\filter B,C}$. We can now
conclude by applying the I.H.\ with $u \assignar u_{\prefixof b}$.
\end{itemize}

\item Suppose $m = \pomove \in B$.
If $m^- = \pomove \in B$ then the I.H.\ with $u \assignar u_{\prefixof m^-}$ permits us to conclude.
If $m^- = \omove \in A$ then we conclude by applying the I.H.\ on $u \assignar u_{\prefixof q}$ where $q$ is the external P-move in $A$ justifying $m^-$.
\end{enumerate}
\end{proof}

We now use the two preceding Lemmas to show
the following useful result:
\begin{lemma}[Increasing order lemma]
\label{lem:increasing_order}
Let $u = \ldots n' \cdot r \cdot m \in Int(A,B,C)$ where
\begin{enumerate}
\item
$n'$ is an external O-move in compoment $X$
($n'=\omove_A$ and $X=A,B$, or $n'=\omove_C$ and $X=B,C$)  non initial in $C$,
\item $n' \in \pview{u\filter A,C}$,
\item $m$ is either played in $B$
($\opmove$ or $\pomove$) or is an external
 P-move in $Y$
($\pmove_C$ if $n'=\omove_A$ and
$\pmove_A$ if $n'=\omove_C$),
\item $m$'s justifier occurs before $n'$,
\item $u\filter X$ is P-i.j.,
\item $u_{\prefixof b}\filter Y$ is P-i.j.\ for all non-initial B-move $b$ occurring in $u$.
\end{enumerate}
Then:
$$ \ord_{Y} m \geq \ord_{A\lingamear C} n' \ .$$
\end{lemma}
\begin{proof}
If $n' =\omove_C$ (resp.~if $n'=\omove_A$)
then by Lemma \ref{lem:middleopmove}
(resp.~Lemma \ref{lem:middlepomove})
there is an occurrence in $r \cdot m$ of a non-initial B-move $\gamma$ of type $\opmove$
(resp.~$\pomove$) such that $\gamma \in \pview{u\filter Y}$ , $n' \in \pview{u_{\leq \gamma} \filter X}$ and $\gamma$ is justified by a move occurring before $n'$. By the $6^{th}$ hypothesis, $u_{\prefixof \gamma}\filter Y$ is P-i.j.

There are six possible cases depending on
the type of the moves $n'$ and $m$:
$(n',m) \in \{ \omove_A \} \times \{\pmove_C,\opmove,\pomove \}
\union \{ \omove_C \} \times \{\pmove_A,\opmove,\pomove \} $).
The following diagram illustrates the cases $(n',m)
 = (\omove_A,\pmove_C)$ (left)
and  $(n',m)
 = (\omove_C,\pmove_A)$  (right):
$$
\begin{array}{ccccc}
A & \longrightarrow & B &
 \longrightarrow & C \\
&\vdots&&\vdots\\
&&&& \rnode{n}{\omove} \\
&\vdots&\rnode{gj}{\opmove}&\vdots\\
n' \omove \\
&\vdots&&\vdots  \\
&&\rnode{g}{\gamma} \pomove \\
&\vdots&&\vdots  \\
&&&&\rnode{m}{m} \pmove \\
\end{array}
\ncarc[arcangleA=30,arcangleB=30]{->}{m}{n}
\ncarc[arcangleA=30,arcangleB=30]{->}{g}{gj}
\hspace{2cm} \begin{array}{ccccc}
A & \longrightarrow & B & \longrightarrow & C \\
&\vdots&&\vdots\\
& \rnode{n}{\omove} \\
&\vdots& &\rnode{gj}\vdots\\
&&&&n' \omove \\
&\vdots&&\vdots  \\
&&\rnode{g}{\gamma} \opmove \\
&\vdots&&\vdots  \\
\rnode{m}{m} \pmove \\
\end{array}
\ncarc[arcangleA=30,arcangleB=30]{->}{m}{n}
\ncarc[arcangleA=30,arcangleB=30]{->}{g}{gj}
 $$

We have:
\begin{equation}
\ord_Y \gamma \geq \ord_X \gamma \label{eqn:gammaorderXY}
\end{equation}
Indeed, if $n' =\omove_C$ then $X=B,C$ and $Y=A,B$ and by Lemma
\ref{lem:compositionorder} we have $\ord_{A\lingamear B} \gamma \geq
\ord_{B\lingamear C} \gamma$. If $n=\omove_A$ then $\gamma$ is a
$\pomove$-move therefore it is not initial in $B$ and Lemma
\ref{lem:compositionorder} gives $\ord_{A\lingamear B} \gamma =
\ord_{B\lingamear C} \gamma$.

Hence:
\begin{align*}
\ord_{A\lingamear C} n'
& = \ord_{X} n' & \mbox{(n' non initial in $C$ \& Lemma \ref{lem:compositionorder})} \\
& \leq \ord_{X} \gamma & \mbox{($u_{\prefixof \gamma}\filter Y$ is P-i.j. \& $\gamma$'s justifier occurs before $n'$)} \\
& \leq \ord_{Y} \gamma & \mbox{(By Eq.\ \ref{eqn:gammaorderXY})} \\
& \leq \ord_{Y} m & \mbox{($u\filter X$ is P-i.j. \&
4$^{th}$ assumption: $m$'s justifier occurs before $\gamma$)}.
\end{align*}
\end{proof}


\begin{lemma}
\label{lem:visibleatprefixofu}
Let $u\in Int(A,B,C)$ such that
$u = \ldots \gamma \ldots \delta \ldots m$
where $m$ is a generalized P-move in $X$,
$\gamma \in \pview{u\filter A,C}$  and $\delta \in \pview{u\filter X}$. Then $\gamma \in \pview{u_{\prefixof \delta} \filter A,C}$.
\end{lemma}
\begin{proof}
First we remark than $\delta$ must occur in $\pview{u}$.
Indeed, $\delta \in \pview{u\filter X} = \pview{u_{< m} \filter X} \cdot m$ therefore $\delta \in \pview{u_{< m} \filter X}$ and since the move preceding $m$ in $u$ is necessarily a generalized O-move in $X$, we can use Lemma 3.3.1 from \cite{Harmer2005}:
\begin{align*}
\delta \in \pview{u_{< m} \filter X}
&= \pview{\pview{u_{<m}}\filter X} & \mbox{(Lemma 3.3.1 from \cite{Harmer2005})}\\
&\subseqof \pview{u_{<m}} \\
&\subseqof \pview{u} \ .
\end{align*}

Clearly, $\pview{u_{\prefixof \delta} \filter A,C}$ is a prefix of $\pview{u \filter A,C}$, indeed:
\begin{align*}
\pview{u_{\prefixof \delta} \filter A,C}
& = \pview{u_{\prefixof \delta}}\filter A,C
  & \mbox{(Lemma \ref{lem:pviewAC_eq_ACpview})}  \\
& \prefixof \pview{u}\filter A,C
  & \mbox{($\delta \in \pview{u}$)} \\
& = \pview{u\filter A,C}
  & \mbox{(Lemma \ref{lem:pviewAC_eq_ACpview})} \ .
\end{align*}

Finally since $\gamma \in \pview{u\filter A,C}$ and $\gamma$ occurs before $\delta$ in $u$, we necessarily have $\gamma \in \pview{u_{\prefixof \delta}\filter A,C}$.
\end{proof}

\begin{lemma}
\label{lem:compos_auxiliary_lemma}
Let $X$ be a component and $u \in Int(A,B,C)$ such that
the projection of $u$ on the component $X$ has the form:
$$ u \filter X =
\Pstr[0.5cm][2pt]{ \ldots (n){n}  \ldots
 {\stk {n'}{\omove}}  \ldots  (m-n,30){\stk m {\pmove}}
}$$
and
\begin{enumerate}
  \item $m$ and $n'$ are external move in $X$ ({\it i.e.}~in $A$ if $X =A,B$ and in $C$ if $X=B,C$);
  \item $u\filter X$ is P-i.j.;
  \item $u_{\prefixof b}\filter Y$ is P-i.j.\ for all non-initial B-move $b$ occurring in $u$.
\end{enumerate}
Then either $\ord_{A\lingamear C} n' \leq \ord_{A\lingamear C} m$ or
$n' \not \in \pview{u\filter A,C}$.
\end{lemma}
\begin{proof}

Suppose that $n'$ occurs in the P-view $\pview{u\filter X}$. Then we have
\begin{equation}
\ord_{A\lingamear C} n'  = \ord_{B\lingamear C} n' \ . \label{eqn:ordnp}
\end{equation}
Indeed, if $X$ is the component $B,C$ then necessarily $n'$ is not initial in $C$ (otherwise it would be the first move in $\pview{u \filter B,C}$, which is not the case since by visibility $n$ must occur before $n'$ in the P-view) and
if $X=A,B$ then $n'$ is in $A$. Thus in both cases, Lemma \ref{lem:compositionorder} gives us the claimed equality.

Hence we have
\begin{align*}
\ord_{A\lingamear C} n'
& = \ord_{X} n' & \mbox{(Eq.\
\ref{eqn:ordnp})} \\
& \leq \ord_{X} m & \mbox{($u\filter X$ is P-i.j.)} \\
& = \ord_{A\lingamear C} m & \mbox{(Lemma \ref{lem:compositionorder} \& $m$ is not initial in $C$)} \ .
\end{align*}

Suppose that $n'$ does not occur in the P-view $\pview{u \filter X}$, then $n'$ lies underneath a PO arc occurring in $\pview{u \filter X}$. Let us denote this arc by $\beta$-$\alpha$ where $\beta$ and $\alpha$ denote the arc's nodes. We have:
$$ u \filter X = \ldots
\Pstr[0.5cm]{
 (n){n} \ldots (b){\stk\beta \pmove} \ldots \stk{n'} {\omove}
\ldots (a-b){\stk\alpha \omove}  \ldots (m-n){\stk m {\pmove} }
} $$
with $\ord_X \alpha \leq \ord_X m$ (by P-i.j.\ of $u \filter X$).

\begin{enumerate}[i.]
\item Suppose $\alpha$ is an external move then so is $\beta$. Indeed, if $X=B,C$ and $\alpha = \omove_C$ then $\alpha$ can only point to another move in $C$ and
if $X=A,B$ and $\alpha = \omove_A$ then since $\alpha$ is an O-move in $A,B$, it is not initial in $A$ and therefore its justifier must also be in $A$.

Then instancing Lemma \ref{lem:interjump} with
$n \assignar n'$ gives us $n' \not\in\pview{u \filter A,C}$.

\item Suppose $\alpha$ is a $B$-move then necessarily so is $\beta$. Indeed, if $X=A,B$ then $\alpha \in B$
can only point to a move in $B$, and if $X=B,C$ then
since $\alpha$ is an O-move in the game $B,C$ it is not initial in $B$ and therefore its justfier must also be in $B$.

Now suppose that $n' \in \pview{u\filter A,C}$,
then by Lemma \ref{lem:visibleatprefixofu}
(with $\delta,\gamma \assignar \alpha,n'$)
we have $n' \in \pview{u_{\prefixof \alpha}\filter A,C}$,
and $u_{\prefixof \alpha}\filter Y$ is P-i.j.\ by hypothesis 3. This permits us to apply Lemma \ref{lem:increasing_order} on $u_{\prefixof \alpha}$:
\begin{align*}
\ord_{A\lingamear C} n'
& \leq \ord_{Y} \alpha & \mbox{(Lemma \ref{lem:increasing_order} with $u\assignar u_{\prefixof \alpha}$)} \\
& = \ord_{X} \alpha & \mbox{(Lemma \ref{lem:compositionorder} \& $\alpha$ non initial in $B$)} \\
& \leq \ord_{X} m & \mbox{($u \filter X$ is P-i.j.)} \\
& = \ord_{A\lingamear C} m & \mbox{(Lemma \ref{lem:compositionorder} \& $m$ is not initial in $C$)} \ .
\end{align*}
\end{enumerate}
\end{proof}


\begin{proposition}
\label{prop:closedpijcompose} Let $\sigma : A \lingamear B$ and $\mu
: B \lingamear C$ be two well-bracketed (P-visible) strategies then
\begin{enumerate}[(I)]
\item $\sigma$ closed P-i.j.\ $\wedge$ $\mu$ P-i.j.
$\implies$ $\sigma ; \mu$  P-i.j.;
\item $\sigma, \mu$ closed P-i.j.
$\implies$ $\sigma ; \mu$ closed P-i.j.
\end{enumerate}
\end{proposition}

\begin{proof}
Well-bracketing is preserved by strategy composition (see \cite[Proposition 2.5]{abramsky94full}) thus
$\sigma ; \mu$ is well-bracketed so we can use the definition of P-i.j.\ from Proposition \ref{prop:char_wellbrack}.

\noindent (I) Let us prove that $\sigma ; \mu$ is P-i.j..
Let $u$ be a play of the interaction $\sigma\ \|\ \mu$ between $\sigma$ and $\mu$
ending with an external P-move $m$
justified by $n$ in $\pview{u \filter A , C}$.
Let $n'$ be an external O-move occurring betweeen $n$ and $m$:
$$ u \filter A,C =
\Pstr[0.5cm][2pt]{ \ldots (n){\stk {n} \omove}  \ldots
 {\stk {n'} \omove}  \ldots  (m-n,30){\stk m \pmove}
}
$$
To show that $u \filter A,C$ is P-incrementally justified, we just
need to prove that either $n'\not\in \pview{u \filter A,C}$ or
$\ord_{A\lingamear C} n' \leq \ord_{A\lingamear C} m$. Note that if
$n'\in \pview{u \filter A,C}$ then necessarily $n'$ is not initial
in $C$ because $n$ occurs before $n'$ in $\pview{u \filter A,C}$.

Let $E$ denote one of the two external arenas ($A$ or $C$), $X$ be
the corresponding component ({\it i.e.}~$X=A,B$ if $E=A$ and $X=B,C$
if $E=C$) and $Y$ denote the other component.
    \begin{enumerate}[1)]
    \item Suppose $m$ and $n$ are two external moves in $E$.

        \begin{enumerate}[{1}.a)]
        \item Suppose $n' \in E$.

        This case corresponds to the situation handled by
        Lemma \ref{lem:compos_auxiliary_lemma}: we have
        either $\ord_{A\lingamear C} n' \leq
        \ord_{A\lingamear C} m$ or $n' \not\in \pview{u
        \filter A,C}$.

        \item Suppose $n' \not\in E$.

        Suppose that $n' \in \pview{u\filter A,C}$, then by
        Lemma \ref{lem:increasing_order} with $X\assignar Y$
        we have $ \ord_{A\lingamear C} n'  \leq \ord_X m$
        and since $m$ is not initial in $C$, Lemma
        \ref{lem:compositionorder} gives $\ord_X m =
        \ord_{A\lingamear C} m$, thus $\ord_{A\lingamear C}
        n' \leq \ord_{A\lingamear C} m$.
        \end{enumerate}

        \item \label{case:mA} Suppose $m \in A$ and $n \in C$.

        Then $m$ is an initial move in $A$
        pointing to a $\opmove$-move
        $b_0$ initial in $B$ which in turn points to the $\omove_C$-move $n$ initial in $C$.

        This situation cannot be handled similarly as the
        previous case. Indeed the pointer associated to the move
        $m$ in the game $A,C$ is not the same as the one
        attached to the corresponding move in the game $A,B$
        (see in \cite{Abr02} for the definition of the
        projection operation over the overall component A,C),
        hence we cannot use Lemma \ref{lem:increasing_order}
        since the condition requiring that $m$ points before
        $n'$ is not necessarily met. A more detailed analysis is
        therefore required.

        Let us assume that $n'\in \pview{u\filter A,C}$ and
        prove that we necessarily have $\ord_{A\lingamear C} n'
        \leq \ord_{A\lingamear C} m$. We do a case analysis:
        \begin{itemize}[-]
        \item Suppose $n'$ occurs before $b_0$.
        Note that we cannot apply Lemma \ref{lem:increasing_order} on $u$
        since $m$ does not point before $b_0$.
        Up to now we have only used the fact that $\sigma$ and $\mu$ are P-i.j. The assumption that $\sigma$ is  \emph{closed} P-i.j.\ now becomes crucial.

        Since $n' \in \pview{u\filter A,C}$ and
        $b_0 \in \pview{u\filter B,C}$, applying Lemma \ref{lem:visibleatprefixofu}
        with $X\assignar B,C$ and $\delta,\gamma \assignar b_0,n'$ gives
        $n' \in \pview{u_{\prefixof b_0}\filter A,C}$. This allows us to apply Lemma \ref{lem:increasing_order} on $u_{\prefixof b_0}$:
            \begin{align*}
            \ord_{A\lingamear C} m
            = \ord_A m
            & \geq \ord_B b_0 & \mbox{($u \filter A,B$ is closed P-i.j., $m$ is initial in $A$)} \\
            & = \ord_{B\lingamear C} b_0  \\
            & \geq \ord_{A\lingamear C} n' & \mbox{(Lemma \ref{lem:increasing_order} on $u_{\prefixof b_0}$ with $X\assignar A,B$)} \ .
            \end{align*}

        \item Suppose $n'$ occurs after $b_0$ (and necessarily before $m$).

            \begin{enumerate}[a.]
            \item Suppose $n'\in C$. Since $m$'s justifier occurs before $n'$ in $u$, we can use Lemma \ref{lem:increasing_order} which gives $\ord_{A\lingamear C} n' \leq \ord_{A\lingamear B} m
                = \ord_{A\lingamear C} m$.

            \item Suppose $n'\in A$.
By Lemma \ref{lem:compos_auxiliary_lemma} with $X
\assignar A,B$, since $n' \in \pview{u \filter A,C}$, we
have $\ord_{A\lingamear C} n' \leq \ord_{A\lingamear C}
m$.
\smallskip

        Note that we could not use Lemma
        \ref{lem:increasing_order} on $u$ directly since
        both $m$ and $n'$ are played in $A$. Also, in
        the ideal case where $n'$ is hereditarily
        enabled by the initial move $m$, we can
        immediately conclude $\ord_{A\lingamear C} n'
        \leq \ord_{A\lingamear C} m$; however this
        argument does not work in general: there may be
        more than one initial move in $A$ in which case
        $n'$ can be hereditarily enabled by an initial
        $A$-move distinct from $m$.
            \end{enumerate}
        \end{itemize}

    \end{enumerate}

\noindent (II) We now show that $\sigma;\mu$ is closed P-i.j.\
provided that both $\sigma$ and $\mu$ are. Take a play $s m \in
\sigma ; \mu$ such that $m$ is initial in $A$ and let $n$ be the
initial move of $C$ justifying $m$. Let $u \in \sigma \ \|\ \mu$ be
the uncovering of $s m$ ($s m = u \filter A,C$) and $b_0$ be the
initial $B$-move justifying $m$ in $u$.
 We have:
\begin{align*}
\ord_A m & \geq \ord_B b_0 & \mbox{($u \filter A,B \in \sigma$ is closed P-i.j.)} \\
 & \geq \ord_C n & \mbox{($u_{\prefixof b_0} \filter B,C \in \mu $ is closed P-i.j.)}.
\end{align*}
\end{proof}

{\it Remark:} The second part of the proposition only gives a
\emph{sufficient} condition for $\sigma ; \mu$ to be closed P-i.j.
In fact it is possible to have that $\sigma ; \mu$ is closed P-i.j.\
although $\mu$ is not.


\subsection{Tensor product}

 Given two strategies $\sigma :\ A
\lingamear B$  and $\tau :\ C\lingamear D$, their tensor product
$\sigma \otimes \tau :\ A\otimes B \lingamear C\otimes D$ is defined
as
$$\sigma \otimes \tau = \{ s \in L_{A\otimes C \lingamear B\otimes
D} \ | \ s \filter A,B \in \sigma \wedge s \filter C,D \in \tau \} $$
 where $A\otimes B$ denotes the tensor product of the games $A$ and $B$ (see \cite{abramsky:game-semantics-tutorial}).
\begin{proposition}
Let $\sigma :\ A \lingamear B$  and $\tau :\ C\lingamear D$.
\begin{enumerate}
\item If $\sigma$ and $\tau$ are P-i.j.\
then so is $\sigma \otimes \tau$;
\item If $\sigma$ and $\tau$ are closed P-i.j.\ then so is $\sigma \otimes \tau$.
\end{enumerate}
\end{proposition}

\begin{proof}
By establishing the state diagram of the game $A\otimes C \lingamear
B\otimes D$ one can show easily that only player O can switch
between the subgames $A\lingamear B$ and $C\lingamear D$.
Consequently, in the P-view of a play of the game $A\otimes C
\lingamear B\otimes D$, all the moves are played in the same subgame
({\it i.e.~} all in $A\lingamear B$ or all in $C\lingamear D$).
Hence if the last move of a play $m$ is played in $A\lingamear B$
then $\pview{s\filter A,B} = \pview{s} \filter A,B = \pview{s}$ (and
reciprocally if $m$ is played in $C\lingamear D$). The first part of
the proposition then follows immediately. The second part is also
straightforward.
\end{proof}


\subsection{Pairing} Given two strategies $\sigma :\ C \lingamear A$
and $\tau :\ C\lingamear B$, the pairing $\langle \sigma , \tau
\rangle :\ C \lingamear A\& B$ is defined as
\begin{align*}
\langle \sigma , \tau \rangle
    &= \{ s \in L_{C \lingamear A\& B} \ | \ s \filter C,A \in \sigma \wedge s \filter B = \epsilon \} \\
    & \union \{ s \in L_{C \lingamear A\& B} \ | \ s \filter C,B \in \tau \wedge s \filter A = \epsilon \}
\ .
\end{align*}
 where $A\& B$ denotes the product of the games $A$ and $B$ (see \cite{abramsky:game-semantics-tutorial}).

\begin{proposition}
\label{prop:pij_paring} Let $\sigma :\ C \lingamear A$  and $\tau :\
C\lingamear B$.
\begin{enumerate}
\item If $\sigma$ and $\tau$ are P-i.j.\ then so is $\langle \sigma , \tau \rangle$;
\item If $\sigma$ and $\tau$ are closed P-i.j.\ then so is $\langle \sigma , \tau \rangle$.
\end{enumerate}
\end{proposition}
The proof is immediate.


\subsection{Promotion} \emph{Notation:} Let $s$ be a play. We call
\defname{thread} a maximal subsequence of $s$ constituted of moves
that are hereditarily justified by the same occurrence of an initial
move. Let $m$ be a move occurring in $s$. We call thread of $m$ the
only thread in $s$ containing $m$.


We recall some definitions. Let $A$ and $B$ be two well-opened
games. Given a strategy  $\sigma :\ !A \lingamear B$, its promotion
$\sigma^\dag :\ !A\lingamear !B$ is defined as
$$ \sigma^\dag = \{ s \in L_{!A\lingamear !B}\ |\ \mbox{for all inital $m$ in $B$, } s\filter m \in \sigma \}$$
and for $\mu :\ !B\lingamear C$ the composite strategy $\sigma
\fatcompos \mu$ is defined as:
$$ \sigma \fatcompos \mu = \sigma^\dag ; \mu \ .$$

Since $B$ is well-opened, plays of $\sigma$ are constituted of a
single thread initiated by some initial $B$-move. Plays of
$\sigma^\dag$ however, are interleaves of potentially infinitely many single-threaded
plays of $\sigma$. One can show easily, using the visibility condition, that the thread of a $P$-move
is always the same as the thread of the preceding $O$-move. Consequently, the P-view of a play is equal to the P-view of the current thread:
if the current thread of a play $s$ is opened by an initial move $b \in B$ then
$\pview{s} = \pview{s \filter b} = \pview{s} \filter b$.


The state of the game is given by an infinite sequence of symbols in $\{O, P\}$, each element of the
sequence indicating who is to play in the corresponding thread.
The diagram on Figure \ref{fig:promotion_state_diagram} illustrates
how the state changes as a play of $\sigma^\dag$ unfolds.
The initial state of the game is $O^\omega$ - an infinite
sequence of O's -- which indicates that O is to play in all the
threads. When O plays an initial move in $B$, it ``opens'' a new
thread so the state of the game becomes $O^k P O^\omega$ where $k$
is the index of the thread being opened. By alternation, $P$ now has to play. His move must be played in a thread
already opened by $O$ and in which $P$ is to play; only one thread is in such state: the $k$th one. Hence after P's move
we are back to state $O^\omega$.

\tikzstyle{state}=[rectangle,draw=blue!50,fill=blue!20,thick,minimum
height = 4ex, text width=1.2cm, text centered]
\tikzstyle{state_nobg}=[thick,minimum
height = 4ex, text width=1.2cm, text centered]
\tikzstyle{omove}=[->,shorten <=1pt,>=latex',line width=0.5pt,bend left=10]
\tikzstyle{pmove}=[->,shorten <=1pt,>=latex',line width=0.5pt,bend left=10, draw=blue!50]
\begin{figure}[htbp]
\begin{center}
\begin{tikzpicture}[node distance=2cm]
% the states
\path
 node(init)  [state, text width=4cm] {$O^\omega$}
 (init)+(-2.8cm,-3cm)
 node(p)     [state, anchor=east,] {$PO^\omega$}
 node(p1)    [state, right of=p]  {$OPO^\omega$}
 node(p2)    [state_nobg, right of=p1] {\ldots}
 node(p3)    [state, right of=p2] {$O^kPO^\omega$}
 node(p4)    [state_nobg, right of=p3] {\ldots} ;
\path
% arrows representing the moves
  ([xshift=-1.4cm]init.south)  edge[omove] node[right]{O} ([xshift=0.2cm]p.north)
  (p.north)    edge[pmove] node[left]{P} ([xshift=-1.5cm]init.south)
  ([xshift=-0cm]init.south)   edge[omove] node[right]{O} ([xshift=0.2cm]p1.north)
  (p1.north)   edge[pmove] node[left]{P} ([xshift=-0.2cm]init.south)
  ([xshift=1cm]init.south)   edge[omove] node[right]{O} ([xshift=0cm]p3.north)
  ([xshift=-0.2cm]p3.north)   edge[pmove] node[left]{P} ([xshift=0.8cm]init.south);
\end{tikzpicture}
\end{center}
\caption{State diagram for plays of $\sigma^\dag$.}
\label{fig:promotion_state_diagram}
\end{figure}



\begin{proposition}
\label{prop:fatcompos_pij} If $A$ and $B$ are two well-opened games
and $\sigma :\ !A \lingamear B$ is a well-bracketed P-i.j.\ strategy
then $\sigma^\dag$ is also well-bracketed and P-i.j. Furthermore if
$\sigma$ is closed P-i.j.\ then so is $\sigma^\dagger$.
\end{proposition}
\begin{proof}
$\sigma^\dag$ is well-bracketed by \cite[Proposition
2.10.]{abramsky94full}. For P-incremental justification, the result is a direct consequence of the
fact that the P-view of a play in $\sigma^\dag$ is equal to the P-view of the current thread.
For closed P-incremental justification, the result is immediate.
\end{proof}

From propositions \ref{prop:closedpijcompose} and
\ref{prop:fatcompos_pij} we obtain:
\begin{corollary}
Let $A$ and $B$ be two well-opened games. Let $\sigma :\ !A
\lingamear B$ and $\mu :\ !B\lingamear C$ be two well-bracketed
strategies then:
\begin{enumerate}
\item If $\sigma$ is closed P-i.j.\ and $\mu$ is P-i.j.\ then $\sigma \fatcompos \mu :\ !A \lingamear
C$ is also P-i.j.;
\item If $\sigma$ and $\mu$ are closed P-i.j.\ then so is $\sigma \fatcompos \mu :\ !A \lingamear C$.
\end{enumerate}
\end{corollary}

\subsection{The category $\mathcal{G}_{Pij}$ of closed P-i.j.\ strategies}

We define the category $\mathcal{G}_{Pij}$ as follows:
\begin{itemize}
\item the objects are games (as defined in \cite{abramsky:game-semantics-tutorial}),
\item the morphisms from $A$ to $B$ are the closed P-incrementally-justified strategies
on the game $A\rightarrow B$,
\item morphisms are composed using the standard game-semantic strategy composition.
\end{itemize}
This indeed defines a category. Indeed we have shown in the previous
section that closed P-i.j.\ compose, strategy composition is
associative (\cite{abramsky94full,hylandong_pcf}) and finally the
identity strategy $id_A$ for any game $A$ is closed P-i.j.

We mentioned before that such category cannot be cartesian closed. Indeed, remember that
a P-i.j.\ strategy from $A$ to $B$ is said to be \emph{closed} P-i.j.\ provided that some condition
on the arena $A\rightarrow B$ holds. This condition refers precisely to the structure of the arenas $A$ and $B$ and consequently relies on the fact the arena considered is exactly $A\rightarrow B$ (and not any other isomorphic arena).



\section{Modeling the Safe Lambda Calculus in $\mathcal{G}_{Pij}^{inn}$}

Consider the category $\mathcal{G}_{Pij}$ defined in section
\ref{sec:closedpij}. In this section we show how the safe lambda
calculus can be modeled in the sub-category
$\mathcal{G}_{Pij}^{inn}$ of innocent P-ij strategies.

\subsection{The language}
We recall the definition of the safe simply-typed lambda calculus
\cite{blumong:safelambdacalculus}.  We use sequents of the form
$\Gamma \vdash_s M : A$ to represent terms-in-context where $\Gamma$
is the context and $A$ is the type of $M$. For simplicity we write
$(A_1, \cdots, A_n, B)$ to mean $A_1 \typear \cdots \typear A_n
\typear B$, where $B$ is not necessarily ground.

\begin{definition}\rm
The \defname{safe lambda calculus}, or Safe $\Lambda^{\rightarrow}$ for short, is a sub-system of the
  simply-typed lambda calculus defined by induction over the
  following rules:
$$ \rulename{var} \ \rulef{}{x : A\vdash_s x : A} \quad
\rulename{wk} \ \rulef{\Gamma \vdash_s s : A}{\Delta \vdash_s s : A} \quad
\Gamma \subset \Delta$$
$$ \rulename{app} \ \rulef{\Gamma \vdash_s s : (A_1,\ldots,A_n,B) \
  \Gamma \vdash_s t_1 : A_1 \; \ldots \; \Gamma \vdash_s t_n : A_n
} {\Gamma \vdash_s s t_1 \ldots t_n : B} \ \min_{y:Y \in \Gamma} \ord Y \geq \ord B$$
$$ \rulename{abs} \ \rulef{\Gamma, x_1 : A_1, \ldots, x_n : A_n
  \vdash_s s : B} {\Gamma \vdash_s \lambda x_1 \ldots x_n . s :
  (A_1, \ldots ,A_n,B)} \ \min_{y:Y \in \Gamma} \ord Y \geq \ord (A_1, \ldots ,A_n,B)$$
%  where $\ord{\Gamma}$ denotes the set $\{ \ord{y} : y
%\in \Gamma \}$ and ``$c \sqsubseteq S$'' means that $c$ is a
%lower-bound of the set $S$.
\end{definition}


\subsection{Game-semantic denotation}

In \cite{blumong:safelambdacalculus} we showed that in the game
semantic model safe lambda terms are denoted by P-i.j.\ strategies.
The argument was syntactic: it is based on the analysis of a special
kind of abstract syntax tree of a term called computation tree
\cite{OngLics2006}. Here we give another proof based on a semantic
argument that uses the results of section \ref{sec:closedpij}.


\begin{proposition}
\label{prop:safe_closepij_sem}
  Safe simply-typed terms are denoted by closed P-i.j.\ strategies.
\end{proposition}
\begin{proof}
  By induction on the formation rules.
  \begin{enumerate}
    \item (var) $\sem{x:A \vdash_s x:A } = id_A$. Clearly the
    identity strategy is closed P-i.j.

    \item (wk) Take $\Gamma \subset \Delta $ and suppose $\sem{\Gamma \vdash_s
    s : A}$ is closed P-i.j. Up to an appropriate retagging of
    the moves the two strategies $\sem{\Delta \vdash_s s : A}$
    and $\sem{\Gamma \vdash_s s : A}$ are isomorphic. Hence
    $\sem{\Delta \vdash_s s : A}$ is P-i.j. It is also closed
    P-i.j.\ since none of the new initial moves introduced by
    $\Delta$ occurs in any play of the strategy.

    \item (app) Suppose that $\sem{\Gamma \vdash_s s :
    (A_1,\ldots,A_n,B)}$ and $\sem{\Gamma \vdash_s t_i : A_i}$
    for $i \in \{1..n\}$ are closed P-i.j.\ and $\ord{B}
    \sqsubseteq \ord{\Gamma}$. We have $\sem{ \Gamma \vdash_s s
    t_1 \ldots t_n : B} = \langle \sem{\Gamma \vdash s},
    \sem{\Gamma \vdash t_1}, \ldots, \sem{\Gamma \vdash t_n}
    \rangle \fatsemi ev_n$ where $ev_n$ is the $n$-parameter
    evaluation strategy. By Proposition \ref{prop:pij_paring},
    $\langle \sem{\Gamma \vdash s}, \sem{\Gamma \vdash t_1} ,
    \ldots, \sem{\Gamma \vdash t_n} \rangle$ is closed P-i.j.
    The evaluation map $ev_n$ is P-i.j.\ (but not necessarily
    closed P-i.j.) therefore by Proposition
    \ref{prop:closedpijcompose} I. $\sem{ \Gamma \vdash_s s t_1
    \ldots t_n : B}$ is P-i.j. The arena of the game
    $\sem{\Gamma}$ is of type $\sem{Y_1} \times \ldots \times
    \sem{Y_n}$ where the $Y_i$s are the types of the variables
    in the context $\Gamma$. The $\sem{Y_i}$s are all prime
    since we work with pure simple types without product.
    Moreover the side-condition of the rule gives $\ord Y_i \geq
    \ord B$ for all $i \in \{1..n\}$. Hence by Lemma
    \ref{lem:closedpij_singleBinitmove}(ii), $\sem{ \Gamma
    \vdash_s s t_1 \ldots t_n : B}$ is closed P-i.j.

    \item (abs) Suppose that $\sem{\Gamma, x_1 : A_1, \ldots, x_n : A_n \vdash_s
    s : B}$ is closed P-i.j. Then the isomorphic strategy $\sigma = \sem{\Gamma \vdash_s \lambda
    x_1 \ldots x_n . s : (A_1,\ldots,A_n,B)}$ is also P-i.j.
    Again, using the side-condition, Lemma \ref{lem:closedpij_singleBinitmove}(ii)
    implies that $\sigma$ is closed P-i.j.
  \end{enumerate}
\end{proof}

\subsection{The safe lambda calculus with product ($\Lambda^{\rightarrow}_\times$)}
We will now show how product types and pairing can be added to the safe lambda calculus.
This can be done trivially as follows. Types are now given by the following grammar:
\begin{align*}
T ::=& \ B,  \quad \mbox{for some base type $B$} \\
   &| \ T \rightarrow T \\
   &| \ T \times T
\end{align*}
and the typing system is extended with the following three rules:
$$ \rulename{\times} \ \rulef{\Gamma \vdash_s s : A \qquad \Gamma \vdash_s t : B}
{\Gamma \vdash_s \langle s, t \rangle : A \times B}
\qquad \rulename{\pi_1} \ \rulef{\Gamma \vdash_s s : A \times B}
{\Gamma \vdash_s \pi_1 s : A} \qquad
 \rulename{\pi_2} \ \rulef{\Gamma \vdash_s s : A \times B}
{\Gamma \vdash_s \pi_2 s : B}$$

One can easily check that most of the good properties of the safe
lambda calculus remains in this extended calculus: the free
variables of a term have order greater than the order of the term
itself; the no-variable-renaming Lemma holds and terms are denoted
by P-i.j.\ strategies. However in general, terms are not denoted by
\emph{closed} P-i.j.\ strategies. Indeed, in the safe lambda calculus
without product - as defined in the previous section - all the
arenas involved are prime {\it i.e.}~they have a single initial
move. In the present case however, since types can be constructed
using the cartesian product, the corresponding arenas can have many
initial moves. Consequently Lemma
\ref{lem:closedpij_singleBinitmove}(ii) cannot be used anymore! Here
is a counter-example: Take the term $x : (o^1\rightarrow o^2)\times
o^3 \vdash \lambda y^o . \pi_2 x : o^4 \rightarrow o^5$ denoted by
some P-i.j.\ strategy $\sigma$ containing the play $q^5 q^3$. We have
$\ord_{(o^1\rightarrow o^2)\times o^3} q^3 = 0 < 1 = \ord_{o^4
\rightarrow o^5} q^5$ therefore $\sigma$ is not closed P-i.j.


There are two different approaches to overcome this problem. The
first one consists in restricting the types of the variables
appearing in the context of a term. More precisely we require that
whenever a variable has product type $A \times B$ we have $\ord A =
\ord B$. One can easily check that this rules out the problem
underlined in the previous counter-example and that it guarantees
that terms are then indeed denoted by closed P-i.j.\ strategies.

The other approach is less restrictive but requires us to modify
slightly the side-conditions of the application and abstraction
rules: instead of requiring that all variables in the context have
order greater than the order of the term, we require that the order
of \emph{any prime sub-type of any variable} in the context has
order greater that the order of the term. The set $Pr(A)$ of prime
sub-types of a type $A$ being defined as follows:
\begin{align*}
Pr(B) &= \{ B \} \qquad \mbox{ for some base type } B \\
Pr(A\rightarrow B) &= \{ A\rightarrow B \} \\
Pr(A\times B) &= Pr(A) \union Pr(B)
\end{align*}

This gives rise the following calculus:
\begin{definition}\rm
The \defname{safe lambda calculus with product}, or Safe
$\Lambda^{\rightarrow}_\times$ for short, is given by induction over
the following rules:
$$ \rulename{var} \ \rulef{}{x : A\vdash_s x : A} \quad
\rulename{wk} \ \rulef{\Gamma \vdash_s s : A}{\Delta \vdash_s s : A} \quad
\Gamma \subset \Delta$$
$$ \rulename{\times} \ \rulef{\Gamma \vdash_s s : A \qquad \Gamma \vdash_s t : B}
{\Gamma \vdash_s \langle s, t \rangle : A \times B}
\qquad \rulename{\pi_1} \ \rulef{\Gamma \vdash_s s : A \times B}
{\Gamma \vdash_s \pi_1 s : A} \qquad
 \rulename{\pi_2} \ \rulef{\Gamma \vdash_s s : A \times B}
{\Gamma \vdash_s \pi_2 s : B}$$
$$ \rulename{app} \ \rulef{\Gamma \vdash_s s : (A_1,\ldots,A_n,B) \
  \Gamma \vdash_s t_1 : A_1 \; \ldots \; \Gamma \vdash_s t_n : A_n
} {\Gamma \vdash_s s t_1 \ldots t_n : B} \ C(\Gamma ; B)$$
$$ \rulename{abs} \ \rulef{\Gamma, x_1 : A_1, \ldots, x_n : A_n
  \vdash_s s : B} {\Gamma \vdash_s \lambda x_1 \ldots x_n . s :
  (A_1, \ldots ,A_n,B)} \ C(\Gamma ; (A_1, \ldots ,A_n,B) )$$

where the side-condition $C(\Gamma ; B)$ expresses that $\forall y:Y
\in \Gamma. \forall Y' \in Pr(Y) . \ord Y' \geq \ord B$.
\end{definition}

One can show by induction on the rules that the game denotations of
terms of this calculus are closed P-i.j.\ (the argument is similar to
the one used in the proof of Proposition
\ref{prop:safe_closepij_sem}). However this syntax does not
completely capture all the closed P-i.j.\ strategies. Take for
instance the simply-typed term $x:(o\rightarrow o)\times o \vdash \lambda z^o. \
(\pi_1 x) : o \rightarrow (o \rightarrow o)$; Its denotation is
closed P-i.j.\ but it is not typable in
$\Lambda^{\rightarrow}_\times$.


\section{Modeling Safe PCF in $\mathcal{G}_{Pij}^{inn}$}

\subsection{Safe PCF}

\notetoself{Insert here the definition of Safe PCF from transfer
thesis or refer to a definition given in a previous chapter.}

\subsection{Game-semantic denotation (semantic argument)}

\begin{proposition}
\label{prop:safepcf_closedpij} Safe PCF terms are denoted by closed
P-incrementally justified strategies.
\end{proposition}
\begin{proof}
We first prove the result for $\pcf_1$ - the fragment of \pcf\
containing terms of the form $\Omega_A = Y (\lambda x : A.x)$ but
where no other use of Y is allowed (see
\cite{abramsky:game-semantics-tutorial}). The proof is by structural
induction over the structure of the term.
\begin{itemize}
\item The strategy $\sem{\Omega_A} = \bot$ is
clearly closed P-i.j.

\item The functional rules are treated the same way as in the
corresponding proof for the safe lambda calculus.

\item For the arithmetic rules, we observe that the strategies
$succ$, $pred$ and $cond$ are all closed P-i.j. The fact that
pairing and strategy composition preserve closed P-incremental
justification permits us to conclude.
\end{itemize}

We now lift the result to full PCF using the technique of
\emph{syntactic approximant} (see
\cite{abramsky:game-semantics-tutorial}). By \cite[lemma
16]{abramsky:game-semantics-tutorial} we have
$$ \sem{M} = \Union_{n\in\omega} \sem{M_n}$$
where $M_n$ is the $\pcf_1$ term obtained from $M$ by replacing each
subterm of the form $Y N$ with $Y^n N_n$, and $Y^n F$ denotes the
$n$th approximant of $Y F$. Since the $M_n$s are $\pcf_1$ terms, by
the previous result each $\sem{M_n}$ is closed P-i.j.\ and since
closed P-incremental justification is clearly a continuous property,
$\sem{M}$ is also closed P-i.j.
\end{proof}


\subsection{Full abstraction}

\subsubsection{O-incremental justification}
 
\defname{O-incremental justification} is the counterpart of P-incremental justification ({\it i.e.}~the role of O and P is exchanged in the definition).

O-incremental justification relates to P-incremental justification very much like O-visibility relates to P-visibility
(see \cite[Sec.~3.6]{Harmer2005}).

Let $\sigma : A$ and $\mu : A \rightarrow o$ be two strategies and $q$ be the initial move of the game $A \rightarrow o$. Then P-views of plays in $A$ correspond to O-views
in the game $A \rightarrow o$. Indeed, for $s\in L_A$ we have $q s \in L_{A \rightarrow o}$ and
due to alternation, $q \pview{s}^A = \oview{q s }_{A \rightarrow o}$.

Consequently, if $\sigma$ is P-i.j.\ then the play involved in the interaction between $\sigma$ and $\mu$
are all O-i.j.\ from $\mu$'s perspective. Indeed, let $u \in \sigma \| \mu$ with $|u|\geq1$. Then $u=q v$
and $u\filter A = v \filter A$ is P-i.j. By the previous remark, this implies that $q (v\filter A) = (q v)\filter (A \rightarrow o) = u \filter (A \rightarrow o)$ is O-i.j.
\smallskip

Now if we regard $\sigma$ as the denotation of some closed term $\vdash M:A$ and $\mu$ as the
denotation of some context $x:A \vdash C[x]:o$ then what the previous remark says is that
non O-i.j.\ plays are useless for the purpose of studying observational equivalence!
This suggests that it is not necessary to include non O-i.j.\ plays in the game denotation of safe terms. However before removing completely those plays from the game model, we have to ensure that this does not prevent us from constructing a category:
\begin{lemma}
\label{lem:oij_decomp}
Let $\sigma : A\rightarrow B$ and $\tau : A\rightarrow B$ be closed P-i.j.\ strategies and suppose
that $u\in \sigma \| \tau$ such that for all external O-moves $o$ of $u$, we have that $u_{\prefixof o} \filter A,C$ satisfies
O-incremental justification. Then, for any generalized O-move $m$ of $u$ in component $X$, we have that
$u_{\prefixof m} \filter X$ satisfies O-incremental justification
\end{lemma}

This lemma states that O-i.j plays cannot be obtained from the interaction of plays that are not O-i.j. In other words, if we write $\mathcal{O}(\sigma)$ for the set of O-i.j.\ plays of $\sigma$, then the Lemma can be restated equivalently as:
\begin{eqnarray}
     \forall \sigma, \tau\ \mbox{closed P-i.j.}: \mathcal{O}(\sigma) ; \mathcal{O}(\tau) \supseteq \mathcal{O}(\sigma ; \tau)
     \label{eqn:oijdecomp_1}
\end{eqnarray}
which in turn is equivalent to
\begin{eqnarray}
    \forall \sigma, \tau\ \mbox{closed P-i.j.}: \mathcal{O}( \mathcal{O}(\sigma) ; \mathcal{O}(\tau) ) = \mathcal{O}(\sigma ; \tau)
    \label{eqn:oijdecomp_2}
\end{eqnarray}
Indeed, Eq.~\ref{eqn:oijdecomp_1} implies the right-to-left inclusion and the other inclusion
is given by the fact that $\mathcal{O}(\sigma) ; \mathcal{O}(\tau) \subseteq \sigma;\tau$.


In some sense, Lemma \ref{lem:oij_decomp} is the dual of the proposition stating that closed P-i.j.\
strategies compose, since  the latter can be reexpressed more succinctly with the relation:
\begin{eqnarray}
     \forall \sigma, \tau .\, \mathcal{P}(\sigma) ; \mathcal{P}(\tau) \subseteq \mathcal{P}(\sigma ; \tau)
     \label{eqn:pijcomp_1}
\end{eqnarray}
where $\mathcal{P}(\sigma)$ is define as be the largest even-length-prefix-closed subset of $\sigma$ consisting of closed P-i.j.\ plays.

\subsubsection{A category of incremental strategies}

\notetoself{
-Incremental strategies means O-i.j.\ and closed P-i.j.

}

\subsubsection{Full abstraction}

The fully-abstract game-model of PCF is also fully-abstract for the
safe fragment of PCF when observational equivalence is defined with
respect to unrestricted ({\it i.e.}~possibly unsafe) PCF contexts.
However one may ask what is a fully abstract model of Safe PCF with
respect to \emph{safe} contexts.




\notetoself{
By the definability results for Safe PCF, it should be possible to
prove that the category $\mathcal{C}^{inn}_{OP-incr}$ of
OP-incrementally justified and innocent strategies is fully abstract
for Safe PCF. We can use the same proof as in the PCF case: we have
a compact test strategy $\alpha:A\rightarrow N$ and by definability,
there must be some context $x:A \vdash C[x] : N$ such that $\sem{x:A
\vdash C[x] : N} = \alpha$. The definability result for Safe PCF
gives us that $\lambda x . C[x] : A \rightarrow N$ is safe which in
turns implies that $x:A \vdash C[x] : N$ is safe since $\ord{N} =
0$.
}

\subsubsection{Algorithmic game semantics}
We recall that Strongly Safe IA $\subseteq$ Safe IA $\subseteq$ IA.
Up to order $3$, it is conservative, with respect to observational equivalence, to add unsafe context to safe ones.
At order $4$, it is not conservative anymore.

\paragraph{Observational equivalence}
\begin{table}
\begin{tabular}{|c|c||c|c|c|c|c|}
    \cline{3-7}
  \multicolumn{2}{c|}{}  & \multicolumn{5}{c|}{Finitary fragments} \\ \hline
  \multirow{2}{*}{$L$} & \multirow{2}{*}{$C[\_]$} &   order 2          &  order 2       & order 3     & order 3 & \multirow{2}{*}{order 4}  \\
                       &                          &    + while         &   + $Y_1$      & + while     & +$Y_0$  &          \\ \hline \hline

  \multirow{4}{*}{IA}  & \multirow{2}{*}{IA}      & \multirow{4}{2cm}{PSPACE$^{(1)}$ \\ {\small $\preccurlyeq$ DFA}} & \multirow{4}{*}{U$^{(2)}$} & \multirow{4}{2.8cm}{EXP-complete$^{(3)}$ \\ {\small $\preccurlyeq$ VPA} }  & \multirow{4}{2cm}{D$^{(4)}$ \\ {\small $\preccurlyeq_{exp}$ DPDA\\ $\succcurlyeq$ DPDA} } & \multirow{2}{*}{U$^{(5)}$}\\
                       &                          &                    &                    &  & & \\
\cline{2-2}\cline{7-7} & \multirow{2}{*}{Safe IA} &                    &                    &  & & \multirow{2}{*}{?} \\
                       &                          &                    &                    &  & & \\ \hline

  \multirow{4}{*}{Safe IA} & \multirow{2}{*}{IA}      & \multirow{4}{2cm}{PSPACE \\ {\small $\preccurlyeq$ DFA}} & \multirow{4}{*}{U} & \multirow{4}{2.3cm}{EXP-complete \\ {\small $\preccurlyeq$ VPA}} & \multirow{4}{2cm}{D \\ {\small $\preccurlyeq_{exp}$ DPDA\\ $\succcurlyeq$ DPDA} } & \multirow{2}{*}{U} \\
                           &                          &                    &                & & & \\
\cline{2-2}\cline{7-7}     & \multirow{2}{*}{Safe IA} &                    &                & & & \multirow{2}{*}{?} \\
                           &                          &                    &                & & & \\ \hline

  \multirow{4}{*}{St. Safe IA} & \multirow{2}{*}{IA}           & \multirow{4}{*}{D} & \multirow{4}{*}{?} & \multirow{4}{*}{D} & \multirow{4}{*}{D} & \multirow{2}{*}{?} \\
                               &                               &                    &                    &                    &                    & \\
\cline{2-2} \cline{7-7}        &  \multirow{2}{*}{St. Safe IA} &                    &                    &                    &                    & \multirow{2}{*}{?} \\
                               &                               &                    &                    &                    &                    & \\ \hline
\end{tabular}
\caption{Decidability (and complexity) of observational equivalence for some finitary fragments of IA}

U stands for Undecidable and D stands for decidable with unknown complexity, $\preccurlyeq P$ means ``reducible to problem $P$''
and $\succcurlyeq P$ means ``at least as hard as problem $P$''.
\begin{asparaenum}
\item[1.] See \cite{ghicamccusker00}.
\item[2.] Showed by Ong in \cite{OngLics2006}.
\item[3.] See \cite{DBLP:conf/fossacs/MurawskiW05}.
\item[4.] See \cite{DBLP:conf/icalp/MurawskiOW05}.
\item[5.] By encoding of $\Sigma$-machine (turing complete) into IA$_4$, see \cite{murawski03program}.

\end{asparaenum}
\end{table}

\paragraph{Observational approximation}

Observational approximation has been shown to be undecidable at order $1$ already, for the fragment $IA_1 + Y_0$ (\cite{DBLP:conf/fossacs/MurawskiW05}).


\notetoself{
- Characterization of the set of complete plays for Safe IA.
(Easy adaptation of the corresponding result for IA. In the present case however, the proof relies
on the fact that plays of the strategy are O-i.j. (in order for $\alpha$ to be P-i.j.)
}


\subsection{What is a model of Safe PCF/Safe IA?}

\notetoself{
- Define the notion of incremental category.

- Show that any incremental category is a model of Safe PCF and that any model of Safe PCF
is an incremental category.

- Show that the category of games and OP-i.j. strategies is an incremental category.

}



\section{Modeling Safe IA in $\mathcal{G}_{Pij}$}
\notetoself{
- I need to merge this section with the other note on Safe IA.

- $\iavar =  \iacom^{\omega}\times \iaexp$

- Any strategy on the game $I \lingamear\ !\iavar$ is P-i.j.\ (and
thus closed P-i.j.) since there is no P-question in the arena
\iavar. Hence the strategy $cell$ is P-i.j. }

\subsubsection{Game-semantic denotation}

In this section, our aim is to extend the game-semantic characterization of safety to Safe IA.

We first observe that the result extends extends trivially from Safe PCF to Strongly Safe IA:
\begin{proposition}
  Strongly Safe IA terms are denoted by closed P-i.j. strategies.
\end{proposition}
\begin{proof}
The proof is an adaptation of the proof for Safe PCF. We first show that the result holds for the
fragment of Strongly Safe IA in which the only allowed uses of $Y$ are in terms of the form $\Omega$.
This is done by induction over the structure of the term:
The functional rules and the arithmetic rules are treated
the same way as in the proof for Safe PCF. For the imperative rules, we
observe that the strategies $assign$, $deref$, $mkvar$, $seq$ and
$cell$ are all closed P-i.j. The fact that pairing, tensor product
and strategy composition all preserve closed P-incremental
justification permits us to conclude.

The result is then lifted to the whole of Strongly Safe IA using syntactic approximants as in the PCF case.
\end{proof}

Now we would like to extend this result to full Safe IA:
\begin{proposition}
\label{prop:safeia_closedpij} Safe IA terms are denoted by closed
P-incrementally justified strategies.
\end{proposition}

This happens to be less trivial than for the previous restricted fragment of Safe IA. We first introduce some definitions:

\begin{definition}[P-i.j. modulo $\mathfrak{M}$]
\label{def:pij_modulo} Let $\sigma$ be a strategy on some game $A$
and $\mathfrak{M}$ be a set of moves. We say that $\sigma$ is P-i.j.
modulo $\mathfrak{M}$ iff for all $s m \in \sigma$ with $m \not\in
\mathfrak{M}$, the play $s m$ is P-i.j.

Similarly we say that $\sigma$ is \emph{closed} P-i.j. modulo
$\mathfrak{M}$ iff for all $s m \in \sigma$ with $m \not\in
\mathfrak{M}$ the play $s m$ is \emph{closed} P-i.j.

Hence a strategy is P-i.j. if and only if it is P-i.j. modulo
$\emptyset$.
\end{definition}

Given a term $\Gamma | \Gamma^{\ianew} \safeentail M : A$, we write
$\sem{\Gamma | \Gamma^{\ianew} \safeentail M : A}$ to denote the
game denotation of the corresponding IA term {\it i.e.}
$\sem{\Gamma, \Gamma^{\ianew} \vdash M : A}$. Instead of showing
Proposition \ref{prop:safeia_closedpij} we will prove the following
more general result:
\begin{proposition}
\label{prop:safeia_closedpijmodulo} Let $\Gamma | \Gamma^{\ianew}
\safeentail M : A $ be a Safe IA term. Its denotation $\sem{\Gamma |
\Gamma^{\ianew} \safeentail M : A}$ is closed P-i.j. modulo
$\mathfrak{M}_{\Gamma^{\ianew}}$ where
$\mathfrak{M}_{\Gamma^{\ianew}}$ is the set of initial moves in
$\Gamma^{\ianew}$.
\end{proposition}

\begin{remark}
Since the context $\Gamma^{\ianew}$ contains variable of type
\iavar\ only, $\mathfrak{M}_{\Gamma^{\ianew}}$ contains only moves
of the form `$read$' or `$write_i$' for some $i\in \nat$.
\end{remark}

\begin{lemma}
\label{lem:leftcompos_preserv_pijmodulo}
 Let $\sigma : A \rightarrow
B$ and $\mu : B \rightarrow C$.
  Let $\mathfrak{M}$ be any set of moves initial in $A$.
  If $\sigma$ is closed  P-i.j. modulo $\mathfrak{M}$ and $\mu$ is
  P-i.j. (resp. closed P-i.j.) then $\sigma \fatsemi \mu$ is P-i.j. (resp. closed P-i.j.) modulo $\mathfrak{M}$.
\end{lemma}
\notetoself{
\begin{proof}
Let us analyze the proof of compositionality for closed P-i.j.
strategies.

\end{proof}
}


\begin{lemma}
\label{lem:cellcomposition_preserve_pijmodulo} Let $\tau : I
\rightarrow C_2$, $\sigma : C_1 \otimes C_2 \rightarrow B$  and
$\mathfrak{M}$ be any set of moves initial in $C_1 \otimes
C_2$.
  If $\tau$ is P-i.j. and
  $\sigma$ is P-i.j. (resp. closed P-i.j.) modulo $\mathfrak{M}$
  then $(id_{C_1} \otimes \tau) \fatsemi \sigma$ is P-i.j. (resp. closed P-i.j.) modulo $\mathfrak{M} \inter C_1$.
\end{lemma}
\notetoself{
\begin{proof}
\end{proof}
}


\notetoself{
\begin{proof}[Proof of Prop.\ \ref{prop:safeia_closedpijmodulo}]
We first prove the result for the fragment of Safe IA where the only allowed uses of the $Y$ combinator is in terms of the form $\Omega$. By induction and and case analysis on the structure of Safe IA terms:
\begin{itemize}
  \item[$\rulename{var}$, $\rulename{var^\ianew}$
  , $\rulename{wk}$, $\rulename{wk^\ianew}$]

  These cases are treated the same way as in the corresponding
  proof for the Safe Lambda Calculus.

  \item[$\rulename{app}$]

  \item[$\rulename{abs}$]


  \item[$\rulename{const}$]
  \item[$\rulename{succ}$, $\rulename{pred}$,$\rulename{cond}$]

  \item[$\rulename{seq}$]
  \item[$\rulename{assign}$]
  \item[$\rulename{deref}$]
  \item[$\rulename{new}$] $\Gamma | \Gamma^\ianew \safeentail \ianewin{x}\ M : B$.

Let $cell : I \rightarrow !\iavar$ denotes the ``storage cell''
strategy (see \cite{abramsky:game-semantics-tutorial}).  Let
$\sigma = \sem{\Gamma | \Gamma^\ianew, x : \iavar \safeentail M
: B}$.  We have $\sem{\Gamma | \Gamma^\ianew \safeentail
\ianewin{x}\  M : B} = (id_{\Gamma,\Gamma^\iavar} \otimes cell)
\fatcompos \sigma$. By induction hypothesis $\sigma$ is closed
P-i.j. modulo $\mathfrak{M}_{\Gamma^{\ianew} \otimes !\iavar}$
and one can easily check that $cell$ is P-incrementally
justified. Instancing Lemma
\ref{lem:cellcomposition_preserve_pijmodulo} with $\tau
\leftarrow cell$, $C_1 \leftarrow \Gamma \otimes  \Gamma^\ianew$
and $C_2\leftarrow !\iavar$ gives us the desired result.
  \item[$\rulename{mkvar}$]

\end{itemize}
The result is then lifted to the whole of Safe IA using the technique of syntactic approximants and using the fact that the ``closed P-i.j.'' property is continuous.
\end{proof}
}


\subsection{Algorithmic game semantics}

There is an important theorem (\cite{AM97a}) in game semantics
which states that two IA terms are equivalent if and only if the set
of complete plays of their game denotations are equal. This result was used in \cite{ghicamccusker00} to show that observational
equivalence for the $IA_2$ fragment of IA is decidable -- the set of
complete plays being representable by regular expressions. In
\cite{Ong02} it was shown that it is still decidable
 for $IA_3+Y_0$. Indeed, for this fragment, the set of complete plays becomes context-free
therefore the problem reduces to the DPDA equivalence problem which
is itself decidable (with an unknown complexity).

Imposing the safety condition should lead to some improvement in
complexity. The complexity of  Safe $IA_3$ (resp. Safe $IA'_3$) for
instance, must be lower than the complexity of the DPDA equivalence
problem. Moreover the fact that Safe $IA_3$ (resp. Safe $IA'_3$)
contains terms whose denotation is context free -- e.g. $\lambda f .
f (\lambda x .x )$ -- strongly suggests that its complexity is
strictly higher than the complexity of regular language equivalence.

Murawski \cite{Murawski2003} has shown that observational
equivalence for $\ialgol_4$ is undecidable. The proofs proceeds by
showing that the computations of $\Gamma$-machine -- some variation
of queue machines that are Turing complete -- are representable
using IA terms.

Does this result extend to the safe fragments? For Safe IA, it does,
simply because the $IA_4$ term exhibited in \cite{Murawski2003} to
represent computations of the $\Gamma$-machine is also a Safe IA
term. The same argument does not carry over to Very Safe IA since
the term is not typable in this language. We do not know whether
observational equivalence is decidable for Strongly Safe $IA_4$.



\subsection{Expressivity of Safe IA/Strongly Safe IA}

Murawski representability : Safe IA representable languages are
exactly the context free languages. For Strongly Safe IA however, we
believe that the representable languages are a proper subclass of
the context free languages.













\section{Remarks}
\subsection{Homogeneity constraint}

Type homogeneity is not preserved after composition. Indeed the
types  $o \typear (o \typear o)$ and $(o \typear o) \typear \left((o
\typear o) \typear o \right)$ are homogeneous but $o \typear
\left((o \typear o) \typear o\right)$ is not.

If $A\typear B$ and $B \typear C$ are homogeneous types then  a
sufficient condition for $A\typear C$ to be homogeneous is
``$\ord{A} \geq \ord{B}$''.




\chapter{Conclusion}
    \label{chap:conclusion}
    \chapter{Further possible developments}

In the previous chapter, we have given an account of the game
semantics of Safe $\lambda$-Calculus. However the nature of this
calculus is still not well known. We propose the following possible
roadmap for further research:
\begin{enumerate}
\item prove or disprove that observational equivalence is decidable for Safe \ialgol;
\item find a categorical interpretation of the Safe $\lambda$-Calculus;
\item study the proof theory obtained by the Curry-Howard isomorphism and determine whether it has nice properties that can be helpful in theorem proving;
\item In \cite{DBLP:conf/tlca/LeivantM93}, the $\lambda$-calculus is used to
give several characterisations of the complexity class P. We would
like to investigate whether, by following similar techniques, we can
obtain a characterisation of a different complexity class using the
Safe $\lambda$-Calculus.
\end{enumerate}


In a more general direction of research, we would like to study the
class of languages for which pointers are uniquely recoverable. We
name this class PUR for ``Pointer Uniquely Recoverable''.

We proved that Safe $\lambda$-Calculus is a PUR-language. Another
example is the Serially Re-entrant Idealized Algol (SRIA) proposed
by Abramsky  in \cite{abramsky:mchecking_ia}. This language allows
multiple occurrences or uses of arguments, as long as they do not
overlap in time. In the game semantics denotation of a SRIA term
there is at most one pending occurrence of a question at any time.
Each move has therefore a unique justifier and consequently
justification pointers may be ignored. Safe \ialgol\ is not a
sublanguage of SRIA. One reason for this is that none of the two
Kierstead terms $\lambda f . f (\lambda x . f (\lambda y .y ))$ and
$\lambda f . f (\lambda x . f (\lambda y .x ))$ are Serially
Re-entrant whereas the first one is safe. Conversely, SRIA is not a
sublanguage of Safe \ialgol\ since the term $\lambda f g. f (\lambda
x . g (\lambda y .x ))$ where $f,g:((o,o),o)$ belongs to SRIA but
not to Safe \ialgol. SRIA and Safe \ialgol\ are therefore two
different examples of languages with pointer-less game semantics.

Finitary $\ialgol_2$ is also an example of PUR-language for which
observational equivalence is decidable. As we indicated in the first
chapter, decidability of observational equivalence is a very
appealing property which has immediate applications in the domain of
program verification. Intuitively, PUR-languages seem to be good
candidates of languages for which observational equivalence is
decidable. It would be interesting to discover classes of PUR
languages having this appealing property.

Another possible way to generate PUR-languages might be to constrain
the types of an existing language. In \cite{DBLP:conf/tlca/Joly01},
a notion of ``complexity'' is defined for $\lambda$-terms. It is
proved that a type $T$ can be generated from a finite set of
combinators if and only if there is a constant bounding the
complexity of every closed normal $\lambda$-term of type $T$;
consequently, the only inhabited finitely generated types are the
type of rank $\leq 2$ and the types $(A_1, A_2, \ldots, A_n, o)$
such that for all $i = 1..n$: $A_i = o$ , $A_i = o \rightarrow o$ or
$A_i = o^k \rightarrow o \rightarrow o$.

We know that imposing the first of these two type restrictions to
Finitary \ialgol\ leads to a PUR language. Is is also the case when
imposing the second type restriction?



\torework[Make the bibliographical entries uniform.]{Make the bibliographical entries uniform: ``Firstname Lastname'' $\rightarrow$ ``F. lastname''}
\torework[Compile the final document with latex without pdfsync]{Compile the final document with latex with pdfsync deactivated.}



\bibliographystyle{plain}
\bibliography{../bib/dphil-all}

\printindex

%adds the bibliography to the table of contents
\addcontentsline{toc}{chapter}
     {\protect\numberline{Bibliography\hspace{-96pt}}}

\end{document}
