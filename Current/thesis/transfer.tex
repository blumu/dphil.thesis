\usepackage{bbm,latexsym}
\usepackage{tabularx}

\usepackage{bigcenter}
\usepackage[draft]{graphicx}
% \CompileMatrices  % this command causes problem with the \justseq macro

%\newcommand\textbfit[1]{{\bf\em #1}\index{#1}}
\newcommand\defeq{\stackrel{\textsf{def}}{=}}

% lambda calculus, reduction
\newcommand\seq[2]{{{#1} \vdash {#2}}}
\newcommand\typear{\rightarrow}
\newcommand{\rulename}[1]{\mathbf{(#1)}}
\newcommand\betasredO{\rightarrow_{\beta_s^1}}
\newcommand\betasredT{\rightarrow_{\beta_s^2}}
%\newcommand\blambda{\hbox{\boldmath $\lambda$}}
%\newcommand\lterm[2]{{\blambda{#1}.{#2}}}

% Safe lambda calculus
\newcommand\funto{\longrightarrow}
\newcommand\rank[1]{{\sf rank}(#1)}
\newcommand\order[1]{{\sf order}(#1)}
\newcommand\slheight[1]{{\sf height}(#1)}
\newcommand\nparam[1]{{\sf nparam}(#1)}
\newcommand\level[1]{{\sf level}(#1)}

% set theory
\newcommand\sthat{\ | \ }  % ``such that'' for set defined by comprehension
\newcommand\natbf{\mathbf{N}}
\newcommand\zset{\mathbb{Z}}

% justified sequence of moves
%%\newcommand\jseq{\stackrel{\curvearrowleft}{=}} %equality of justseq
%  justified sequence links using psttricks
\newcommand{\link}[2][nodesep=0pt]{\ncarc[offset=-2pt,nodesep=0pt,ncurv=1,arcangleA=-#2, arcangleB=-#2,#1]{->}}
%\newcommand{\bkptrb}[1][nodesep=0pt]{\nccurve[offset=-2pt,nodesep=0pt,ncurv=1,angleA=90,angleB=90,#1]{->}}
\newcommand{\lnklabel}[1]{\mput*{\mbox{{\tiny $#1$}}}}
\newcommand{\lnklabelb}[1]{\mput{\mbox{\tiny $#1$}}}
\newcommand{\lnklabelc}[1]{\Bput[1pt]{\mbox{{\tiny $#1$}}}}

% todo symbol
\newcommand\todo{\textdbend}
\newcommand\todomargin[1]{\marginpar{\textdbend \begin{sloppypar} #1 \end{sloppypar}}}
\newcommand\todobox[1]{\colorbox{lightgray}{\parbox[h]{0.9\textwidth}{#1}}\marginpar[\textdbend]{\textdbend}}




The first chapter of this part is devoted to the presentation of the
basics and main results of game semantics. The categorical
interpretation of game semantics is presented as well as the full
abstraction result for \pcf. We also give a brief summary of the
main results in algorithmic game semantics. There is no personal
contribution in this chapter.

In the second chapter we present the \emph{safe $\lambda$-calculus}.
Originally, \emph{safety} has been introduced as a syntactical
restriction on higher-order grammars in order to show a decidability
result about MSO theory of infinite trees \citep{KNU02}. In
\cite{safety-mirlong2004}, Aehlig, de Miranda and Ong  proposed an
adaptation of the safety restriction to the $\lambda$-calculus. This
restriction gives rise to the safe $\lambda$-calculus. We first
present this calculus and then give a more general definition which
does not make any assumption on the types of the terms.

In the third chapter, following ideas described in
\cite{OngLics2006}, we introduce the notions of computation tree of
a simply-typed term and traversal over a computation tree. We prove
a theorem showing a correspondence between traversals of the
computation tree and the game semantics of a term. Based on that
correspondence, we give a characterisation of the game semantics of
safe terms by a property called ``P-incremental-justification''. In
P-incrementally-justified strategies, P-pointers are superfluous (i.e.
they can be recovered uniquely from the underlying sequence of
moves and from O-moves' pointers). This simplification of the game semantics suggests some potential applications in algorithmic game semantics. We finish the
chapter by extending the result to safe \pcf\ and by giving the key
elements for an extension to full Safe Idealized Algol.


% first chapter
\chapter{Game semantics}

The aim of this chapter is to introduce game semantics. It starts
with a history of game semantics and a presentation of the full
abstraction problem for PCF which has been solved using game
semantics. It then goes on by introducing the basic notions of game
semantics and by giving a categorical interpretation of games.
Finally we show how games are used to define a syntax-independent
model of programming languages like PCF and Idealized Algol (IA).

This chapter is largely based on the tutorial by Samson Abramsky
tutorial on Game Semantics \cite{abramsky:game-semantics-tutorial}.
Many details and proofs will be omitted and we refer the reader to
\cite{hylandong_pcf, abramsky94full} for a complete description of
game semantics.

\section{History}

\subsection{Game semantics}

In the 1950s, Paul Lorenzen invented Game semantics as a new
approach to study semantics of intuitionistic logic \citep{lor61}.
In this setting, the notion of logical truth is modeled using game
theoretic concepts such as the existence of winning strategy.

Four decade later, game semantics is used to prove the full
completeness of Multiplicative Linear Logic (MLL)
\citep{abramsky92games,HO93a}. Shortly after, a connection between
games and linear logic has been established. Game semantics has then
been used as a new paradigm to study formal models of programming
languages. The idea is to model the execution of a program as a game
played by two protagonists: the Opponent representing the
environment and the Proponent representing the system. The meaning
of the program is then modeled by a strategy for the Proponent.


Subsequently, these game-based model have been used to give a
solution to the long-standing problem of ``Full abstraction of PCF''
\citep{abramsky94full, hylandong_pcf,Nickau:lfcs94}.

Based on that major result, and in a more applied direction, games
have been used as a new tool for software verification
\cite{ghicamccusker00}. This open-up a new field called Algorithmic
Game Semantics \citep{Abr02}.




\subsection{Model of programming languages}

Before the 1980s, there were many approaches to define models for
programming languages. Among the successful ones, there were the
axiomatic, operational and denotational semantics:
\begin{itemize}
\item Operational semantics gives a meaning to a program by describing the
behaviour of a machine executing the program. It is defined formally
by giving a state transition system.
\item Axiomatic semantics defined the behaviour of the program
with axioms and is used to prove program correctness by static
analysis of the code of the program.
\item The denotational semantics approach consists in mapping a program to a mathematical structure
having good properties such as compositionality. This mapping is
achieved by structural induction on the syntax of the program.
\end{itemize}

In the 1990s, three different independent research groups: Samson
Abramsky, Radhakrishnan Jagadeesan and Pasquale Malacaria
\citep{abramsky94full}, Martin Hyland and Luke Ong
\citep{hylandong_pcf} and Nickau \citep{Nickau:lfcs94} have
introduced game semantics, a new kind of semantics, in order to
solve a long standing problem in the semanticists community :
finding a fully abstract model for PCF.

\subsection{The problem of full abstraction for PCF}

PCF is a simple programming language introduced in a classical paper
by Plotkin ``LCF considered as a programming language''
(\cite{DBLP:journals/tcs/Plotkin77}). PCF is based on LCF, the Logic
of Computable Functions devised by Dana Scott in \cite{scott_lcf}.
It is a simply typed lambda calculus extended with arithmetic
operators, conditional and recursion.

The problem of the Full Abstraction for PCF goes back to the 1970s.
In \citep{scott93}, Scott gave a model for PCF based on domain
theory. This model gives a sound interpretation of observational
equivalence: if two terms have the same domain theoretic
interpretation then they are observationally equivalent. However the
converse is not true: there exist two PCF terms which are
observationally equivalent but have different domain theoretic
denotation. We say that the model is not fully abstract.

The key reason why the domain theoretic model of PCF is not fully
abstract is that the parallel-or operator defined by the following
truth table
\begin{center}
\begin{tabular}{l|lll}
p-or  & $\bot$ & tt & ff \\ \hline
$\bot$ & $\bot$ & tt & $\bot$\\
tt & tt & tt & tt\\
ff & $\bot$ & tt & ff\\
\end{tabular}
\end{center}
is not definable as a PCF term! It is possible to create two
different PCF terms that always behave the same except when they are
apply to a term computing p-or. Since p-or is not definable in PCF,
these two terms will have the same denotation. This implies that the
model is not fully abstract.


It is possible to patch PCF by adding the operator $p-or$, the
resulting language ``PCF+p-or'' becomes fully-abstracted by Scott
domain theoretic model \citep{DBLP:journals/tcs/Plotkin77}. However
the language we are now dealing with is strictly more powerful than
PCF, it allows parallel execution of commands whereas PCF only
permits sequential execution.

Another approach consists in eliminating  the undefinable elements
(like p-or) by strengthening the conditions on the function used in
the model. This approach has been followed by Berry in
\cite{berry-stable,gberry-thesis} where he gives a model based on
stable functions, a class of function smaller than the class of
strict and continuous function. Unfortunately this approach did not
succeed.

The only successful approaches to obtain a fully abstract model for
PCF were the ones taken by Ambramsky, Jagadeesan and Malacaria
\citep{abramsky94full}, Hyland and Ong \citep{hylandong_pcf} and
Nickau \citep{Nickau:lfcs94}, all based on game semantics.

This result has then been adapted to other varieties of programming
paradigm including languages with stores (Idealized Algol),
call-by-value \citep{honda99gametheoretic, abramsky98callbyvalue}
and call-by-name, general referencees
\citep{DBLP:conf/lics/AbramskyHM98}, polymorphism
\citep{DBLP:journals/apal/AbramskyJ05}, control features
(continuation and exception), non determinism, concurrency. In all
these cases, the game semantics model led to a syntax-independent
fully abstract model of the corresponding language.

\section{Games}
\label{sec:catgames}

We now introduce formally the notion of game that will be used in
the following section to give a model of the programming languages
PCF and Idealized Algol. The definitions are taken from
\cite{abramsky:game-semantics-tutorial, hylandong_pcf,
abramsky94full}.


\subsection{Arenas and Games}

The games we are interested in are two-players games. The players are named O for Opponent and P for Proponent.

The game played by O and P is constraint by something called
\emph{arena}. The arena defines the possible moves of the game. By
analogy with real board games, the arena represents the board
together with the rules that tell how players can make their moves
on the board. In fact the analogy with board game stops here. Our
games can be thought as dialog games: one person O interviews
another person P, P tries to answer the initial O-question by
possibly asking O some precisions about its initial question.
Moreover, the notion of winner and winning strategy will not be
relevant in our setting.


More formally, the arena can be seen as a forest of trees whose nodes are possible questions and leaves are possible answers.
The arena is partitioned into two kinds of moves: the moves that can be played by P and the ones that can be played by O.
A move is either a question to the other player or an answer to a question previously asked by the other player.

Each move of the game must be justified by another move that has already been played by the other player. This justification relation
is induced by the edges of the forest arena. Moreover, an answer must always be justified by the question that it answers and a question
is always justified by another question.

\begin{dfn}[Arena]
An arena is a structure $\langle M, \lambda, \vdash \rangle$ where:
\begin{itemize}
\item $M$ is the set of possible moves;
\item $(M,\vdash)$ is a forest of trees;

\item $\lambda : M \rightarrow \{ O, P\} \times \{Q, A\}$ is a labeling functions indicating whether a given move
    is a question or an answer and whether it can be played by O or by P.

    $\lambda = [\lambda^{OP},\lambda^{QA}]$ where $\lambda^{OP} : M \rightarrow  \{ O, P\}$
    and $\lambda^{QA} : M \rightarrow  \{ Q, A\}$.

    \begin{itemize}
    \item If $\lambda^{OP} (m) = O$, we call $m$ and O-move otherwise $m$ is a P-move.
    $\lambda^{QA} (m) = Q$ indicates that $m$ is a question otherwise $m$ is an answer.

    \item For any leaf $l$ of the tree $(M,\vdash)$, $\lambda^{QA} (l) = A$ and for any node
    $n \in (M,\vdash)$, $\lambda^{QA} (n) = Q$.
    \end{itemize}

\item The forest of tree $(M,\vdash)$ respect the following condition:
    \begin{itemize}
    \item[(e1)] The roots are O-moves: for any root $r$ of $(M,\vdash)$, $\lambda^{OP} (r) = O$.
    \item[(e2)] Answers are enabled by questions: $m \vdash n  \zand \lambda^{QA}(n) = A \imp \lambda^{QA}(m) = Q$.
    % Or more succinctly, if we write $\dashv$ the relation $\vdash^-1$: $\lambda^{QA} \left( \dashv( (\lambda^{QA})^{-1}(\{A\}) ) \right) = \{ O \}$
    \item[(e3)] A player move must be justified by a move played by the other player:
         $m\vdash n \imp \lambda^{OP}(m) \neq \lambda^{OP}(n)$.
    \end{itemize}
\end{itemize}
\end{dfn}

For commodity we write the set $\{O,P\} \times \{Q,A\}$ as $\{OQ,OA,PQ,PA\}$.
$\overline{\lambda}$ denotes the labeling function $\lambda$ with the question and answer swapped. For instance:
$$\overline{\lambda(m)} = OQ \iff \lambda(m) = PQ$$

The roots of the forest of tree $(M,\vdash)$ are the \emph{initial moves}.

For example, the simplest possible arena is written $\mathbf{1}$ and
denotes the arena which set of moves $M$ is empty.

\begin{exmp}[The flat arena]
\label{exmp:flatarena}

 Let $A$ be any countable set then the flat arena over $A$
is defined to be the arena $\langle M, \lambda, \vdash \rangle$ such
that $M$ has one move $q$ with $\lambda(q) = OQ$ and for each
element in $A$, there is a corresponding move $a_i$ in $M$ with
$\lambda(a_i) = PA$ for some $i \in \nat$. The enabling relation
$\vdash$ is defined to be $\{ q \vdash a_i \ | i \in \nat \}$.

This arena is represented by the following tree:
\begin{center}
  \pstree[levelsep=6ex]
    { \TR{$q$} }
    {    \TR{$a_1$} \TR{$a_2$} \TR{\ldots} }
\end{center}
The vertices represent the moves and the edges represent the
enabling relation.

The flat arena over $\nat$ and $\mathbb{B}$ is written
$\mathbf{int}$ and  $\mathbf{bool}$ respectively.

\end{exmp}

Once the arena has been defined, the bases of the game are set and the players have something to play with.
We now need to describe the state of the game, for that purpose
we introduced \emph{justified sequences of moves}. Sequence of moves are used to record the history of all the moves that have been
played.

\begin{dfn}[Justified sequence of moves]
A justified sequence is a sequence of moves $s$ together with an associated sequence of pointers. Any
move $m$ in the sequence that is not initial has as pointer that points to a previous move $n$ that justifies it (i.e. $n \vdash m$).
\end{dfn}

The pointers of a justified sequences are represented with arrows.
This is an example of justified sequence of moves:
$$\rnode{q4}{q}^4
\rnode{q3}{q}^3 \rnode{q2}{q}^2 \rnode{q3b}{q}^3 \rnode{q2b}{q}^2
\rnode{q1}{q}^1 \bkptrc{q3}{q4} \bkptrc{q2}{q3}
\bkptrc[ncurv=0.6]{q3b}{q4} \bkptrc{q2b}{q3b}$$

The first move of a justified sequence must be an O-move since
initial moves are all O-moves.

Notation: we write $s t$ or sometimes $s \cdot t$ do denote the
sequences obtain by concatenating $s$ and $t$. The empty sequence is
written $\epsilon$.

 A justified sequence has two particular subsequences which
will be of particular interest later on when we introduce
strategies. These subsequences are called the P-view and the O-view
of the sequence. The idea is that a view describes the local context
of the game. Here is the formal definition:

\begin{dfn}[View]
Given a justified sequence of moves $s$. We define the proponent view (P-view) noted $\pview{s}$ by induction:
\begin{align*}
\pview{\epsilon} &= \epsilon \\
\pview{s \cdot m} &= \pview{s} \cdot \ m && \mbox{ if $m$ is a P-move} \\
\pview{s \cdot m} &= m && \mbox{ if $m$ is initial (O-move) } \\
\pview{ s \cdot \rnode{m}{m} \cdot t \cdot \rnode{n}{n} \bkptra{50}{n}{m} } &=
 \pview{s} \cdot \rnode{mm}{m} \cdot \rnode{nn}{n} \bkptra{70}{nn}{mm} && \mbox{ if $n$ is a non initial O-move }
\end{align*}
The O-view $\oview{s}$ is defined similarly:
\begin{align*}
\oview{\epsilon} &= \epsilon \\
\oview{s \cdot m} &= \oview{s} \cdot \ m && \mbox{ if $m$ is a O-move} \\
\oview{ s \cdot \rnode{m}{m} \cdot t \cdot \rnode{n}{n} \bkptra{50}{n}{m} } &=
 \pview{s} \cdot \rnode{mm}{m} \cdot \rnode{nn}{n} \bkptra{70}{nn}{mm} && \mbox{ if $n$ is a P-move }
\end{align*}
\end{dfn}

In fact not all justified sequences will be of interest for the
games that we will use. We call \emph{legal position} any justified
sequence verifying two additional conditions: alternation and
visibility. Alternation says that players O and P plays
alternatively. Visibility expresses that each non-initial move is
justified by a move situated in the local context at that point.
Intuitively, the visibility condition gives some coherence to the
justification pointers of the sequence.

\begin{dfn}[Legal position]
A legal position is a justified sequence of move $s$ respecting the following constraint:
\begin{itemize}
\item Alternation: For any subsequence $m \cdot n$ of $s$, $\lambda^{OP}(m) \neq \lambda^{OP}(n)$.
\item Visibility: For any subsequence $t m$ of $s$ where $m$ is not initial, if $m$ is a P-move then $m$ points to a move in $\pview{s}$
and if $m$ is a O-move then $m$ points to a move in $\oview{s}$.
\end{itemize}

The set of legal position of an arena $A$ is noted $L_A$.
\end{dfn}

We say that a move $n$ is hereditarily justified by a move $m$ if there is a sequence of move
$m_1, \ldots, m_q$ such that:
$$ m \vdash m_1 \vdash m_2 \vdash \ldots m_q \vdash n$$
If a move has no justification pointer, we says that it is an
\emph{initial move} (in that case it must be a root of the forest
arena).

Suppose that $n$ is an occurrence of a move in the sequence $s$ then
$s \upharpoonright n$ denotes the subsequence of $s$ containing all the moves hereditarily justified by $n$.
Similarly, $s \upharpoonright I$ denotes the
subsequence of $s$ containing all the moves hereditarily justified by the moves in $I$.

\begin{dfn}[Game]
A game is a structure $\langle M, \lambda, \vdash, P \rangle$ such that
\begin{itemize}
\item $ \langle M, \lambda, \vdash \rangle$ is an arena.
\item $P$ is called the set of valid positions, it is:
    \begin{itemize}
    \item a non-empty prefix closed subset of the set of legal position
    \item closed by initial hereditary filtering: if $s$ is a valid position then for any set $I$ of occurrences of initial moves
    in $s$, $s\upharpoonright I$ is also a valid position.
    \end{itemize}
\end{itemize}
\end{dfn}

\begin{exmp}  Consider the flat arena  $\mathbf{int}$.
The set of valid position $P = \{ \epsilon, q \} \union \{ q \cdot
a_i \ | i \in \nat \}$ defines a game on the arena $\mathbf{int}$.
\end{exmp}

\subsection{Constructions on games}
\label{sec:gameconstruction}

We now define game constructors that will be useful later on.

Consider the two functions $f : A \rightarrow C$ and $g : B
\rightarrow C$, we write $[f,g]$ to denote the pairing of $f$ and
$g$ defined on the direct sum $A + B$. Given a game $A$ with a set
of moves $M_A$, we use the filtering operator $s \upharpoonright A$
do denote the subsequence of $s$ consisting of all moves in $M_A$.
Although this notation conflicts with the hereditarily filtering
operator, it should not cause any confusion.

\subsubsection{Tensor product}
Given two games $A$ and $B$ we define the tensor product constructor
$A \otimes B$ as follows:
\begin{eqnarray*}
  M_{A \otimes B} &=& M_A + M_B \\
  \lambda_{A\otimes B} &=& [\lambda_A,\lambda_B] \\
  \vdash_{A\otimes B} & = & \vdash_{A}\ \union\ \vdash_{B} \\
  P_{A\otimes B} & = & \{ s \in L_{A\otimes B} | s \upharpoonright A \in P_A \wedge s \ \upharpoonright B \in P_B  \}.
\end{eqnarray*}

In particular,  $n$ is initial in $A\otimes B$ if and only if $n$ is
initial in A or B. And $m \vdash_{A\otimes B} n$  holds if and only if $m
\vdash_{A} n$ or $m \vdash_{B} n$ holds.

\subsubsection{Function space}
The game $A \otimes B$ is defined as follows:
\begin{eqnarray*}
  M_{A \multimap B} &=& M_A + M_B \\
  \lambda_{A\multimap B} &=& [\overline{\lambda_A},\lambda_B] \\
  \vdash_{A\multimap B} & = & \vdash_{A}\ \union\ \vdash_{B}\ \union\  \{ (m,n) \ |\ m \mbox{ initial in } B \wedge n \mbox{ initial in } A \} \\
  P_{A\otimes B} & = & \{ s \in L_{A\otimes B} | s \upharpoonright A \in P_A \wedge s \ \upharpoonright B \in P_B  \}.
\end{eqnarray*}

Graphically if we draw a triangle to represent an arena $A$ then the
arena for $A \multimap B$ is represented as follows:
\begin{center}
\psset{xunit=.5pt,yunit=.5pt,runit=.5pt}
\begin{pspicture}(150,80)
\rput[tr](150,80){ \pnode(27,40){b} \pstribox{B} } \rput[bl](0,0){
\pnode(27,40){a} \pstribox{A} } \ncline{->}{a}{b}
\end{pspicture}
\end{center}

\subsubsection{Cartesian product}
The game $A \& B$ is defined as follows:
\begin{eqnarray*}
  M_{A \& B} &=& M_A + M_B \\
  \lambda_{A\& B} &=& [\lambda_A,\lambda_B] \\
  \vdash_{A\& B} & = & \vdash_{A}\ \union\ \vdash_{B} \\
  P_{A\& B} & = & \{ s \in L_{A\otimes B} | s \upharpoonright A \in P_A \wedge s \ \upharpoonright B = \epsilon  \} \\
        &&   \union \{ s \in L_{A\otimes B} | s \upharpoonright A \in P_B \wedge s \ \upharpoonright A = \epsilon  \}.
\end{eqnarray*}

A play of the game $A \& B$ is either a play of $A$ or a play of $B$ whether a play
of the game $A \otimes B$ may be an interleaving of plays on $A$ and plays on $B$.

\subsection{Representation of plays}

Plays of the game are usually represented in a table diagram. The
columns of the table correspond to the different components of the
arena and each row corresponds to one move in the play. The first
row always represents an O-move, this is because O is the only
player who can open a game (since roots of the arena are O-moves).

As an example the play
$$\rnode{q1}{q}\
 \rnode{q2}{q}
 \ \rnode{a2}{8}
\  \rnode{a1}{12}
  \bkptrc{a1}{q1}
\bkptrc{a2}{q2} $$
on the
game $\textbf{int} \multimap \textbf{int} $ can be represented by
the following diagram:

\begin{center}
\begin{tabular}{cccc}
\textbf{int} & $\imp$ & \textbf{int} & \\
&& q & O\\
q  &&& P\\
8  &&& O\\
&& 12 & P
\end{tabular}
\end{center}

When it is necessary, the justification pointers of the play can also
be shown on the diagram.


\subsection{Strategy}

\subsubsection{Definition}

During a game, the player who has to play may have several choices
for his next move. The move that he makes is chosen according to a
given strategy.

A strategy is a rule telling the player which move to make when the
game is in a given position. More abstractly, a strategy is a
partial function mapping legal position where Proponent has to move
to P-moves.

\begin{dfn}[Strategy]
A strategy for player P on a given game $\langle M, \lambda, \vdash, P \rangle$ is a
non-empty set of even-length positions from $P$ such that:
\begin{enumerate}
\item (\emph{no unreachable position}) $sab \in \sigma \imp s \in \sigma$
\item (\emph{determinacy}) $sab, sac \in \sigma \quad \imp \quad  b = c$  and $b$ has the same justifier as
$c$.
\end{enumerate}
\end{dfn}

The idea is that the presence of the even-length sequence $s a b$ in
$\sigma$ tells the player P that whenever the game is in position
$s$ and player O plays the move $a$ then it must respond by playing
the move $b$.

The first condition ensures that the strategy $\sigma$ only
considers positions that the strategy itself could have led to in a
previous move. The second condition in the definition requires that
this choice of move is deterministic (i.e. there is a function $f$
from the set of odd length position to the set of moves $M$ such
that $f(s a) = b$).


For any game $A$, the smallest possible strategy is the strategy
that never respond given by $\{ \epsilon \}$. It is called the
\emph{empty strategy} and denoted $\bot$.

\subsubsection{Copy-cat strategy}

For any arena $A$ there is a strategy on the game $A \multimap A$
called the \emph{copy-cat strategy}. We write $A_1$ and $A_2$ to
denote the first and second copy of the arena $A$ in the game $A
\multimap A$. If $A$ is the arena $A_1$ then $A^\perp$ denotes the
arena $A_2$ and reciprocally.

Let $A$ be one of the arena $A_1$ or $A_2$. The copy-cat strategy
operates as follows: whenever P has to respond to an O-move played
in $A$, it replicates the move played by O in the arena $A^{\perp}$
after that $O$ has to respond in $A^{\perp}$ and $P$ replicates this
response in $(A^\perp)^\perp = A$ and so on and so forth.


More formally, the copy-cat strategy is defined by:
$$ \textsf{id}_A = \{ s \in P^{\textsf{even}}_{A \multimap A} \ | \ \forall t \sqsubseteq^{\textsf{even}} s\ .\ t \upharpoonright A_1 = t \upharpoonright A_2 \}$$
where $P^{\textsf{even}}_A$ denotes the valid position of even
length in the game $A$ and $t \sqsubseteq^{\textsf{even}} s$ denotes
that $t$ is an even length prefix of $s$.

The copy-cat strategy is also called \emph{identity strategy} since
it is the identity for strategy composition as we will see in the
next paragraph.

\begin{exmp} The copy-cat strategy on $\textbf{int}$ is:
$$\begin{array}{ccc}
\textbf{int} & \imp & \textbf{int} \\
&& q\\
q \\
n \\
&& n
\end{array}
$$
Note that we introduced this type of diagram to represent plays of
games but, as we can see here, the same diagrams can be used to
represent strategies when the play represented is general enough.

The copy-cat strategy on $\textbf{int} \typar \textbf{int}$ is given
by the following diagram:
$$\begin{array}{ccccccc}
(\textbf{int} & \imp & \textbf{int}) & \imp & (\textbf{int} & \imp & \textbf{int}) \\
&&&& && q\\
&& q\\
q \\
&&&& q \\
&&&& m \\
m\\
&& n \\
&&&& && n
\end{array}$$
\end{exmp}

\subsubsection{Composition}

It is well-known that any model of the simply typed lambda-calculus
is a cartesian closed category \citep{CroleRL:catt}. Games are used
to give a fully-abstract model of PCF, an extended simply typed
lambda calculus, therefore the game model should fit into a
cartesian closed category. This category will have games as objects
and strategies as morphisms. In a category, morphisms should be able
to compose together, therefore there should be an appropriate notion
of strategy composition.

Composition of strategies is an essential feature of game semantics.
As we will see in the following section, in the game model of PCF,
strategies represent programs. Therefore, strategy composition will
prove to be very useful : obtaining the model of a composed program
boils down to composing the strategies of the composing programs.

The way composition is defined for strategies is similar to
``parallel composition plus hiding'' in the trace semantics of CSP
\citep{hoare_csp}. Consider two strategies $\sigma : A \multimap B$
and $\tau : B \multimap C$ that we wish to compose.

For any sequence of moves $u$ on three arenas $A$, $B$, $C$, we call
projection of $s$ on the game $A \multimap B$ and we note $u
\upharpoonright A,B$ the subsequence of $s$ obtained by removing
from $u$ the moves in $C$ and pointers to moves in $C$. The
projection on $B \multimap C$ is defined similarly.

The definition of the projection on $A \multimap B$ differs
slightly: $u \upharpoonright A,C$ is the subsequence of $u$
consisting of the moves from $A$ and $C$ with some additional
pointers: we add a pointer from $a \in A$ to $c\in C$ whenever $a$
points to some move $b \in B$ itself pointing to $c$. All the
pointers to moves in $B$ are removed.


First we remark that for a given legal position $s$ in the game $A
\multimap C$, there is what is called an \emph{uncovering} of $s$.
The uncovering of $s$ is the maximal justified sequence of moves $u$
from the games $A$, $B$ and $C$ such that:
\begin{itemize}
\item The sequence $s$, considered as a pointer-less sequence, is a subsequence of
$u$;
\item the projection of $u$ on the game $A \multimap B$ lies in the
strategy $\sigma$;
\item the projection of $u$ on the game $B \multimap C$
lies in the strategy $\tau$;
\item and the projection of $u$ on the game $A \multimap C$ is a subsequence of $s$ (here the term ``subsequence'' refers to the sequence of nodes together with the auxiliary sequence of pointers).
\end{itemize}
This uncovering, noted $uncover(s, \sigma, \tau)$, is
defined uniquely for given strategies $\sigma$, $\tau$ and legal
position $s$ (this is proved in part II of \cite{hylandong_pcf}).

We define $\sigma \| \tau $ to be the set of uncovering of legal
positions in $A \multimap C$:
$$ \sigma \| \tau = \{ uncover(s, \sigma, \tau) \ | \ s \mbox{ is a legal position in } A \multimap C \}$$

The composition of $\sigma$, $\tau$ is defined to be the set of
projections of uncovering of legal positions in $A \multimap C$:

\begin{dfn}[Strategy composition]
Consider $\sigma : A \multimap B$ and  $\tau : B \multimap C$ two
strategies. We define $\sigma ; \tau$ to be:
$$ \sigma ; \tau = \{ u \upharpoonright A,C \ | \ u \in \sigma \|
\tau \}$$
\end{dfn}

It can be verified that composition is well-defined and associative
\citep{hylandong_pcf} and that the copy-cat strategy $\textsf{id}_A$ is the identity for composition.

\subsubsection{Constraint on strategies}

Different classes of strategies will be considered depending on the
features of the language that we want to model. Here is a list of
common restrictions that we will consider:
\begin{itemize}
\item \emph{Well-bracketing:} In a well-bracketed strategies the players always answer the last unanswered question (called the pending question) first.
If we represent Opponent's question as ``['', Proponent's answer as
``]'', Proponent's question as ``('' and Opponent's answers as ``)''
then requiring that the last pending question is answered first is
the same as requiring that the string representing the play is a
prefix of a well-bracketed sequence.

\item \emph{History-free strategies:} A strategy is history-free if the Proponent's move at any position of the game where he has to play
is determined by the last move of the Opponent. In other words, the
history prior to the last move is ignored by the Proponent when
deciding how to respond.

\item \emph{History-sensitive strategies:} The Proponent follows a history-sensitive strategy if he needs to have access to the full
history of the moves in order to decide which move to make.

\item \emph{Innocence:} a strategy is innocent if it determines Proponent's moves based on a restricted view of the history of the play, mainly the P-view
at that point. Such strategies can be specified by a partial
function mapping P-views to P-moves called the \emph{view function}. However not every partial
function from P-views to P-moves gives rise to an innocent strategy
(a sufficient condition is given in \cite{hylandong_pcf}).
\end{itemize}

The formal definition of innocence follows:
\begin{dfn}[Innocence]
Given positions $sab, ta \in L_A$ where $sab$ has even length and
$\pview{sa} = \pview{ta}$, there is a unique extension of $ta$ by
the move $b$ together with a justification pointer such that
$\pview{sab} = \pview{sa}$. We write this extension
$\textsf{match}(sab,ta)$.

The strategy $\sigma:A$ is \emph{innocent} if and only if:
$$ \left(
     \begin{array}{c}
       \pview{sa} = \pview{ta} \\
       sab \in \sigma \\
       t\in \sigma \wedge ta \in P_A \\
     \end{array}
   \right)
\quad \imp\quad  \textsf{match}(sab,ta) \in \sigma$$

\end{dfn}


\subsection{Categorical interpretation}

In this section we recall some results about the categorical representation of Games.
These results with complete details and proofs can be found in \cite{McC96b,hylandong_pcf,abramsky94full}.
We refer the reader to \cite{CroleRL:catt} for more information about category theory.

We consider the category $\mathcal{G}$ whose objects are games and morphisms are
strategies. A morphism from $A$ to $B$ is a strategy on the game $A \multimap B$.

Three other sub-categories of $\mathcal{G}$ are considered: each of them correspond to some restriction on strategies:
$\mathcal{G}_i$ is the sub-category
of $\mathcal{G}$ whose morphisms are the innocent strategies,
$\mathcal{G}_b$ has only the well-bracketed strategies and $\mathcal{G}_{ib}$ has the innocent and well-bracketed strategies.

\begin{prop}
$\mathcal{G}$, $\mathcal{G}_i$, $\mathcal{G}_b$ and $\mathcal{G}_{ib}$ are categories.
\end{prop}

Proving this requires to prove that composition of strategies is well-defined, associative, has a unit (the copy-cat strategy), preserves innocence and
well-bracketedness. See \cite{hylandong_pcf,abramsky94full} for a proof.


\subsubsection{Monoidal structure}

We have already defined the tensor product on games in section \ref{sec:gameconstruction}.
We now define the corresponding transformation on morphisms:
given two strategies $\sigma : A \multimap B$ and $\tau : C \multimap D$ the strategy
$\sigma \otimes \tau : (A \otimes C) \multimap (B\otimes D)$ is defined by:
$$ \sigma \otimes \tau = \{ s \in L_{A \otimes C \multimap B\otimes D} \ s \upharpoonright A,B \in \sigma
\wedge s \upharpoonright C,D \in \tau \}$$

It can be shown that the tensor product is associative, commutative and has
$I = \langle \emptyset, \emptyset,\emptyset, \{ \epsilon \} \rangle $ as identity.
Hence the game categories $\mathcal{G}$ is a symmetric monoidal categories. Moreover
$\mathcal{G}_i$ and  $\mathcal{G}_b$ are sub-symmetric monoidal categories of $\mathcal{G}$,
and $\mathcal{G}_{ib}$ is a sub-symmetric monoidal category of $\mathcal{G}_i$, $\mathcal{G}_b$ and
$\mathcal{G}$.

\subsubsection{Closed structure}

Given the games $A$, $B$ and $C$, we can transform strategies on $A\otimes B \multimap C$ to
strategies on $A \multimap (B \multimap C)$ by retagging the moves to the appropriate arenas. This transformation
defines an isomorphism noted $\Lambda_B$ and called currying. Therefore the hom-set $\mathcal{G}(A\otimes B, C)$ is isomorphic to the hom-set
$\mathcal{G}(A,B\multimap C)$ which makes $\mathcal{G}$ an autonomous (i.e. symmetric monoidal closed) category.

We write $ev_{A,B} : (A \multimap B) \otimes A \rightarrow B$ to denote the \emph{evaluation strategy} obtained by uncurrying the
identity map on $A \rightarrow B$. $ev_{A,B}$ is in fact the copycat strategy for the game
$(A \multimap B) \otimes A \rightarrow B$.

$\mathcal{G}_i$ and  $\mathcal{G}_b$ are sub-autonomous categories of $\mathcal{G}$,
and $\mathcal{G}_{ib}$ is a sub-autonomous category of $\mathcal{G}_i$, $\mathcal{G}_b$ and
$\mathcal{G}$.

\subsubsection{Cartesian product}
The cartesian product defined in section \ref{sec:gameconstruction} is indeed a cartesian product in the category
$\mathcal{G}$, $\mathcal{G}_i$, $\mathcal{G}_b$ and $\mathcal{G}_{ib}$.

The projections $\pi_1:A \& B \rightarrow A$ and $\pi_1:A \& B \rightarrow B$ are given by the obvious copy-cat strategies.
Given two category morphisms $\sigma :C \rightarrow A$ and $\tau : C \rightarrow B$ the pairing function
$\langle \sigma, \tau \rangle : C \rightarrow A \& B$ is given by:
\begin{eqnarray*}
\langle \sigma, \tau \rangle &=& \{ s \in L_{C\multimap A\&B} \ | \ s \upharpoonright C,A \in \sigma \wedge s \upharpoonright B = \epsilon  \} \\
&\union& \{ s \in L_{C\multimap A\&B} \ | \ s \upharpoonright C,A \in \sigma \wedge s \upharpoonright B = \epsilon  \}
\end{eqnarray*}

\subsubsection{Cartesian closed structure}
Having defined the cartesian product is not enough to turn $\mathcal{G}$ into a cartesian closed category :
we also need to define a terminal object $I$ and the exponential construct $A \imp B$ for any two games $A$ and $B$.
In fact, this cannot be done in the current categories $\mathcal{G}$ and we have to move on to another category
of games noted $\mathcal{C}$ whose objects and morphisms are certain sub-classes of games and strategies.

Before introducing the category $\mathcal{C}$ we need some new definitions:


For any game $A$ we define the exponential game noted $!A$.
The game $!A$ corresponds to a repeated version of the game $A$. Plays of $!A$ are interleaving of plays of
$A$. It is defined as follows:
\begin{eqnarray*}
  M_{!A} &=& M_A \\
  \lambda_{!A} &=& \lambda_A \\
  \vdash_{!A} & = & \vdash_{A} \\
  P_{!A} & = & \{ s \in L_{!A} | \mbox{ for each initial move $m$, } s \upharpoonright m \in P_A \}
\end{eqnarray*}
The following equalities hold:
\begin{eqnarray*}
  !(A \& B) &=& !A \otimes !B\\
  I &=& !I
\end{eqnarray*}

\begin{dfn}[Well-opened games]
A game $A$ is well-opened if for any position $s \in P_A$ the only initial move is the first
one.
\end{dfn}

Well-opened games have single thread of dialog. Then can be turned into games with multiple-thread of dialog
using the promotion operator:

\begin{dfn}[Promotion]
Consider a well-opened game $B$.
Given a strategy on ${!A} \multimap B$, we define it promotion $\sigma^\dagger : {!A} \multimap {!B}$ to be the
strategy which plays several copies of $\sigma$. It is formally defined by:
$$ \sigma^\dagger = \{ s \in L_{{!A} \multimap !B} \ | \ \mbox{ for all initial $m$, } s \upharpoonright m \in \sigma  \}.$$
\end{dfn}

It can be shown that promotion is well-defined (it is indeed a strategy) and that it preserves innocence and
well-bracketedness.


We now introduce the category of well-opened games:
\begin{dfn}[Category of well-opened games]
The category $\mathcal{C}$ of well-opened games is defined as follow:
\begin{enumerate}
\item The objects are the well-opened games,
\item a morphism $\sigma : A \rightarrow B$ is a strategy for the game $!A \multimap B$,
\item the identity map for $A$ is the copy-cat strategy on $!A \multimap A$ (which is well-defined for well-opened games).
It is called dereliction, noted
$\textsf{der}_A$ and defined formally by:
$$ \textsf{der}_A = \{ s \in P^{\textsf{even}}_{{!A} \multimap A} \ | \ \forall t \sqsubseteq^{\textsf{even}} s \ . \ t \upharpoonright {!A} = t \upharpoonright A \},$$
\item composition of morphisms $\sigma : {!A} \multimap B$ and $\tau : {!B} \multimap C$
noted $\sigma \fatsemi \tau : {!A} \multimap C$ is defined as $\sigma^\dagger;\tau$.
\end{enumerate}
\end{dfn}
$\mathcal{C}$ is a well-defined category and the three sub-categories
$\mathcal{C}_i$, $\mathcal{C}_b$, $\mathcal{C}_{ib}$ corresponding to sub-category
with innocent strategies, well-bracketed strategies and innocent and well-bracketed strategies respectively.


The category $\mathcal{C}$ has a terminal object $I$, for any two games $A$ and $B$ a product $A \& B$ and
an exponential $A \imp B$ defined to be $!A \multimap B$. The hom-sets $\mathcal{C}(A \& B,C)$ and
$\mathcal{C}(A,!B \multimap C)$ are isomorphic. Indeed:
\begin{eqnarray*}
\mathcal{C}(A\& B,C) &=& \mathcal{G}(!(A\& B),C) \\
&=& \mathcal{G}({!A}\otimes {!B},C) \\
&\cong& \mathcal{G}({!A}, {!B} \multimap C) \qquad  \mbox{($\mathcal{G}$ is a closed monoidal category)}\\
&=& \mathcal{C}(A, {!B} \multimap C)
\end{eqnarray*}
Hence $\mathcal{C}$ is a cartesian closed category. Moreover $\mathcal{C}_i$ and $\mathcal{C}_b$
are sub-cartesian closed caterogies of $\mathcal{C}$ and $\mathcal{C}_{ib}$ is as sub-cartesian closed category
of each of $\mathcal{C}$, $\mathcal{C}_i$ and $\mathcal{C}_b$.





\subsubsection{Order enrichment}

Strategies can be ordered using the inclusion ordering. Under this
ordering, the set of strategies on a given game $A$ is a pointed
directed complete partial order : the least upper bounds is the
union of two strategies and the least element is the empty strategy
$\{ \epsilon \}$.

Moreover all the operators on strategies that we have defined so far
(composition, tensor product, ...) are continuous. Hence the
category $\mathcal{C}$ and $\mathcal{G}$ are cpo-enriched.

This significant characteristic will prove to be extremely useful
when it comes to model programming languages with recursion such as
PCF.


\subsubsection{Intrinsic preoder}

We now define a pre-ordering on strategies. We assume that we are working in one of the categories
$\mathcal{C}$, $\mathcal{C}_i$, $\mathcal{C}_b$, $\mathcal{C}_{ib}$.

Let $\Sigma$ be the game with a single question $q$ and single answer $a$. There are only two strategies on $\Sigma$:
$\bot = \{ \epsilon \}$ and $\top = \{ \epsilon, q a \}$ which are both innocent and well-bracketed. These strategies are used
to test strategies: for any strategy $\sigma : {\bf 1} \rightarrow A$ and for any test strategy $\alpha : A \rightarrow \Sigma$ we say that $\sigma$
passes the test $\alpha$ if $\sigma \fatsemi \alpha = \top$.

The intrinsic preorder noted $\lesssim$ is then defined as follows:
for any strategy $\sigma,\tau$ on the game $A$, $\sigma \lesssim \tau$ if $\tau$ passes all the test passed by $\sigma$. Formally:
$$ \sigma \lesssim \tau \quad \iff \quad \forall \alpha : A \rightarrow \Sigma. \sigma \fatsemi \tau = \top \imp \tau \fatsemi \alpha = \top$$

One can check that the relation $\lesssim$ is indeed a preorder on the set of strategies of the considered category.
This preorder defines classes of equivalence: two strategies are in the same equivalence class if no test can distinguish them.
The quotiented category is written $\bf C/\lesssim$ where $\bf C$ ranges over $\{ \mathcal{C}_i, \mathcal{C}_i, \mathcal{C}_b, \mathcal{C}_{ib} \}$.

Later on we will state the full abstraction of the game semantics model of PCF. This result will
be proved in the quotiented category.

\subsection{Special case of arenas of order 2 at most}
In this section, we consider a restricted class of arena and prove a
property on the games played on these arenas.

The height of the arena is the length of the longest sequence of moves
$m_1 \ldots m_h$ in $M$ such that $m_1 \vdash m_2 \vdash \ldots \vdash m_h$.

The order of an arena $\langle M, \lambda, \vdash \rangle$ is defined to be
$h-2$ where $h$ is the height of the forest of trees $(M, \vdash)$.


\begin{lem}[Pointers are superfluous up to order 2]
Let $A$ be the arena of order at most 2. Let $s$ be a justified sequence of moves in the arena $A$ satisfying
 alternation, visibility, well-openedness and well-bracketing then
the pointers of the sequence $s$ can be reconstructed uniquely.
\end{lem}



\begin{proof}
We represent an arena graphically as a forest of trees. We choose to display the sub-trees of a given node
from left to right by decreasing order of the sub-arena order. This reordering is harmless since reordering children nodes
produces isomorphic arenas.

Let $A$ be an arena of order 2.
The justified sequence that we consider are well-opened therefore there is only one initial move in the sequence (the first move). Consequently
if $A$ is a forest arena (i.e. with multiple roots), the problem can be reduced to the case of single root arena just by replacing
the arena $A$ by the tree of the forest $A$ whose root is the first move of the justified sequence.
Therefore we assume that $A$ has a single root. $A$ has the following shape:
\begin{center}
\
  \pstree[levelsep=6ex]
    { \TR{$q$} }
    {
\SubTree{$T_1$} \SubTree[linestyle=none]{$\ldots$} \SubTree{$T_n$}
    \TR{$a_1$} \TR{$a_2$} \TR{\ldots} }
\end{center}

where each triangle $T_i$ represents an arena of order 0 or 1.


We write $I_k$, for $k=0$ or $1$, the set of indices $i$ such that the arena $T_i$ has order $k$:
$$I_k = \{ i \in 1.. n\ |\ \order{T_i} = k \}$$

Here is a graphic representation of the arenas $T_i$ for $i \in I_0$ and $T_j$ for $j \in I_1$:
\begin{center}
\
  \pstree[levelsep=6ex]
    {\TR{$q^i$}}
    { \TR{$a_1^i$} \TR{$a_2^i$} \TR{\ldots} }
\hspace{2cm}
  \pstree[levelsep=6ex]
    { \TR{$p^j$} }
    {
      \pstree[levelsep=6ex]
        { \TR{$q^j$} }
        { \TR{$a_1^j$} \TR{$a_2^j$} \TR{\ldots} }
      \TR{$b_1^j$} \TR{$b_2^j$} \TR{\ldots}
    }
\end{center}



For any justified sequence of moves $u$, we write $?(u)$ for the
subsequence of $u$ consisting of the questions in the sequence $u$
that are still pending at the end of the sequence.

Let $L$ be the following language $L = \{\ p^i q^i\ | \ i \in I_1
\}$. We consider the following cases:

\begin{center}
\begin{tabular}{c|c|l|l}
Case & $\lambda_{OP}(m)$ & $?(u) \in$ & condition \\ \hline
0 & O & $\{ \epsilon \}$ \\
A & P & $q$ \\
B & O & $q \cdot L^* \cdot p^i$     & $i \in I_1$ \\
C & P & $q \cdot L^* \cdot p^i q^i$ & $i \in I_1$ \\
D & O & $q \cdot L^* \cdot q^i$      & $i \in I_0$ \\
\end{tabular}
\end{center}

We use the notation $\hat{s}$ to denote a legal and well-bracketed
\emph{justified} sequence of moves and $s$ to denote the same
sequence of moves with pointers removed.

Note that the well-bracketing condition already tells us how to
uniquely recover the pointers for P answer moves: a P-answers points
to the last pending question having the same tag. However for O
answers, we will see that the visibility condition already ensures
the unique recoverability of the pointer and that the
well-bracketing condition is not needed.


We prove by induction on the sequence of moves $u$ that $?(u)$
corresponds to either case 0, A, B, C or D and that the pointers in
$u$ can be recovered uniquely.

\textbf{Base cases:}

If $u$ is the empty sequence $\epsilon$ then there is no pointer to
recover and it corresponds to case 0.

If $u$ is a singleton then it must be the initial question $q$ and
there is not pointer to recover. This corresponds to case A.

\textbf{Step case:}

Consider a legal well-bracketed justified sequence $\hat{s}$ where
$s = u \cdot m$ and $m \in M_A$. The induction hypothesis tells us
that the pointers of $u$ can be recovered (and therefore the P-view
or O-view at that point can be computed) and that $u$ corresponds to
one of the cases 0,A,B,C or D.

We proceed by case analysis on $u$:

\begin{description}

\item[case 0] This case cannot happen because $?(u) = \epsilon$ ($u$ is a complete play) implies that there cannot be any further move $m$.

Indeed the visibility condition implies that $m$ must point to a
P-question in the O-view at that point. But since $u$ is a complete
play, the O-view is $\oview{\hat{u}} = q a$ which does not contain
any P-question. Hence the move $m$ cannot be justified and is not
valid.


\item[case A] $?(u) = q$ and the last move $m$ is played by P.
    There are several cases:
    \begin{itemize}
    \item $m$ is an answer $a_k$ (to the initial question
    $q$) for some $k$, then $m$ points to $q$:

    $\hat{s} = \justseq{ q & \ldots & m \pointto{ll}}$

    and $?(s) = \epsilon$ therefore $s$ correspond to the case 0 (complete play).

    \item $m = q^i$ where $q^i$ is an order 0 question ($i \in I_0$).
    Then $q^i$ points to the initial question $q$ and $s$ falls into category D.

    \item $m = p^i$, a first order question, then $p^i$ points to $q$,

    $?(s)= q p^i$ and it is O's turn after $s$ therefore $s$ falls into category B.

    \end{itemize}


\item[case B] $?(u) \in q \cdot L^* \cdot p^i$ where $i \in I_1$ and O plays the move $m$.

We now analyse the different possible O-moves:
\begin{itemize}
\item Suppose that O gives the (tagged) answer $b^j$ for some $j \in I_1$ then
the visibility condition constraints it to point to a question in
the O-view at that point.

We remark that the last move in $\hat{u}$ must be $p^i$. Indeed,
suppose that there is a move $x \in M_A$ such that $\hat{u} =
\justseq{q & \ldots & p^i\ x \pointto{ll}}$ then by visibility, the
O-move $x$ should points to a move in the O-view a that point. The
O-view is $q p^i$, therefore $x$ can only points to $p^i$. But then,
$p^i$ is not a pending question in $s$ which is a contradiction.


Therefore $\oview{\hat{u}} = \oview{ \justseq{ q & \ldots & p^i
\pointto{ll}} } = q p^i$.

Hence $b^j$ can only point to $p^i$ (and therefore $i=j$).

We then have $?(s) = ?(u \cdot b^i) \in  q \cdot L^*$ which is
covered by case A and C.

\item The only other possible O-move is $q^i$ which, again by the visibility condition, points necessarily
to the previous move $p^i$. We then have $?(s) = ?(u \cdot q^i) \in
q \cdot L^* \cdot p^i q^i$. This falls into category C.

\end{itemize}

\item[case C] $?(u) \in q \cdot L^* \cdot p^i q^i$ where $i \in I_1$ and the move $m$ is played by $P$.

Suppose $m$ is an answer, then the well-bracketing condition imposes
to answer to $q^i$ first. The move $m$ is therefore an integer $a^i$
pointing to $q^i$. We then have $?(s) = ?(u \cdot a^i) \in  q \cdot
L^* \cdot p^i$. This correspond to case B.


Suppose $m$ is a question then there are two cases:
\begin{itemize}
\item $m = q^j$ with $j \in I_0$, the pointer goes to the initial question $q$ and $s$ falls into category D.
\item $m = p^j$ with $j \in I_1$, the pointer goes to the initial question $q$ and $s$ falls into category B.
\end{itemize}

\item[case D] $?(u) \in q \cdot L^* \cdot q^i$ where $i \in I_0$ and the move $m$ is played by $O$.

    The same argument as in case B holds. However there is now another possible move:
    the answer $m = a^i_k$ for some $k$.  This moves can only points to
    $q^i$ (this is the only pending question tagged by $i \in I_0$).

    Then $?(\hat{s}) = ?(\hat{u}\cdot a^i_k) = ?(\justseq{ q & \ldots & q^i \pointto{ll} & \ldots & a^i_k \pointto{ll}}) \in q \cdot L^* $ therefore $s$ falls either into category A or C.

\end{description}

This completes the induction.
\end{proof}


\subsection{Pointers are necessary}
\label{subsec:pointer_necessary}

Up to order 2, the semantics of PCF terms is entirely defined by
pointer-less strategies. In other words, the pointers can be
uniquely reconstructed from any non justified sequence of moves
satisfying the visibility and well-bracketing condition.

At level 3 however, pointers cannot be omitted in general. Here is
an example taken from \cite{abramsky:game-semantics-tutorial}
illustrating this. Consider the following two terms, called the
Kierstead terms, of type $((\nat \typar \nat) \typar \nat) \typar
\nat$:

$$M_1 = \lambda f . f (\lambda x . f (\lambda y .y ))$$
$$M_2 = \lambda f . f (\lambda x . f (\lambda y .x ))$$

We assign tags to the types in order to identify in which arena the
questions are asked: $((\nat^1 \typar \nat^2) \typar \nat^3) \typar
\nat^4$. Consider now the following pointer-less sequence of moves
$s = q^4 q^3 q^2 q^3 q^2 q^1$. It is possible to retrieve the
pointers of the first five moves but there is an ambiguity for the
last move: does it point to the first or second occurrence of $q^2$
in the sequence $s$?

Note that the visibility condition does not eliminate the ambiguity,
since the two occurrences of $q^2$ both appear in the P-view at that
point (after recovering the pointers of $s$ up to the second last
move we get:
$$s = \rnode{q4}{q}^4
\rnode{q3}{q}^3
\rnode{q2}{q}^2
\rnode{q3b}{q}^3
\rnode{q2b}{q}^2
\rnode{q1}{q}^1
\bkptrc{q3}{q4}
\bkptrc{q2}{q3}
\bkptrc[ncurv=0.6]{q3b}{q4}
\bkptrc{q2b}{q3b}$$

 therefore the P-view of $s$ is $s$ itself.)

In fact these two different possibilities correspond to two
different strategies. Suppose that the link goes to the first
occurrence of $q^2$ then it means that the proponent is requesting
the value of the variable $x$ bound in the subterm $\lambda x . f (
\lambda y. ... )$. If P needs to know the value of $x$, this is
because P is in fact following the strategy of the subterm $\lambda
y . x$. And the entire play is part of the strategy $\sem{M_2}$.

Similarly, if the link points to the second occurrence of $q^2$ then
the play belongs to the strategy $\sem{M_1}$.

\section{The fully abstract game model for PCF}

In this section we introduce the functional languages PCF. We then
describe the game model introduced in \cite{abramsky94full} and
finally we will state the full abstraction result.

\subsection{The syntax of PCF}
PCF is a simply-type $\lambda$-calculus with the following
additions: integer constants  (of ground type), first-order
arithmetic operators, if-then-else branching, and the recursion
combinator $Y_A : (A\rightarrow A)\rightarrow A$ for any type $A$.

The types of PCF are given by the following grammar:
$$ T ::= \texttt{exp}\ |\ T \rightarrow T$$

and the structure of terms is given by:
\begin{eqnarray*}
 M ::= x\ |\ \lambda x :A . M \ |\ M M \ |\ \\
\ |\ n \ |\ \texttt{succ } M \ |\  \texttt{pred } M \\
\ |\ \texttt{cond } M M M \ |\ \texttt{Y}_A\ M
\end{eqnarray*}

where $x$ ranges over a set of countably many variables and $n$
ranges over the set of natural numbers.

Terms are generated according to the formation rules given in table
\ref{tab:pcf_formrules} where the judgement is of the form $ \Gamma  \vdash M : A$.

\begin{table}[htbp]
$$ (var) \rulef{}{x_1:A_1, x_2:A_2, \ldots x_n : A_n  \vdash x_i : A_i}\ i \in 1..n$$
$$ (app) \rulef{\Gamma \vdash M : A\rightarrow B \qquad \Gamma \vdash N:A}{\Gamma \vdash M\ N : B}
\qquad (abs) \rulef{\Gamma, x:A \vdash M : B}{\Gamma \vdash \lambda x :A . M : A\rightarrow B}$$

$$ (const) \rulef{}{\Gamma \vdash n :\texttt{exp}}
\qquad (succ) \rulef{\Gamma \vdash M:\texttt{exp} }{\Gamma \vdash \texttt{succ}\ M:\texttt{exp}}
\qquad (pred) \rulef{\Gamma \vdash M:\texttt{exp} }{\Gamma \vdash \texttt{pred}\ M:\texttt{exp}}$$

$$
(cond) \rulef{\Gamma \vdash M : exp \qquad \Gamma \vdash N_1 : exp \qquad \Gamma \vdash N_2 : exp }{\Gamma \vdash \texttt{cond}\ M\ N_1\ N_2}
\qquad  (rec) \rulef{\Gamma \vdash M : A\rightarrow A }{ \Gamma \vdash Y_A M : A}$$

\caption{Formation rules for PCF terms}
\label{tab:pcf_formrules}
\end{table}

\subsection{Operational semantics of PCF}

We give the big-step operational semantics of PCF. The notation $M \eval V$ means
that the closed term $M$ evaluates to the canonical form $V$. The canonical forms are given by the following
grammar:
$$V ::= n\ |\ \lambda x. M$$
In other word, a canonical form is either a number or a function.

The full operational semantics is given in table
\ref{tab:bigstep_pcf}. The evaluation rules are defined for closed
terms only therefore the context $\Gamma$ is not present in the
rules. We write $M \eval$ if $M \eval V$ for some value $V$.

\begin{table}[htbp]
$$\rulef{}{V \eval V} \quad \mbox{ provided that $V$ is in canonical form.} $$

$$ \rulef{M \eval \lambda x. M' \quad M'\subst{x}{N} \eval V}{M N \eval V}$$

$$\rulef{M \eval n}{\texttt{succ}\ M \eval n+1}
\qquad \rulef{M \eval n+1}{\texttt{pred}\ M \eval n}
\qquad \rulef{M \eval 0}{\texttt{pred}\ M \eval 0}$$

$$\rulef{M \eval 0 \quad N_1 \eval V}{\texttt{cond}\ M N_1 N_2  \eval V}
\qquad
 \rulef{M \eval n+1 \quad N_2 \eval V}{\texttt{cond}\ M N_1 N_2  \eval V}$$

$$\rulef{M (\mathrm{Y} M) \eval V }{\texttt{Y} M \eval V}$$
\label{tab:bigstep_pcf}
\caption{Big-step operational semantics of PCF}
\end{table}



\subsection{Game model of PCF}
\label{subsec:pcfgamemodel}

As we have seen in section \ref{sec:catgames}, games and strategies
form a cartesian closed category, therefore games can model the
simply-typed $\lambda$-calculus. We are now about to make this
connection concrete by explicitly giving the strategy corresponding
to a given $\lambda$-term. We will then extend the game model to PCF
and IA.

\subsubsection{Simply-typed $\lambda$-calculus fragment}

In the games that we are considering, the Opponent represents the
environment and the Proponent represents the lambda term. Opponent
opens the game by asking a question such as ``What is the output of
the function?'', the proponent then may then ask further information
such that ``What is the input of the function?'' O can then provide
$P$ with an answer (the value of the input) or can pursue with
another question. The dialog goes on until O gets the answer to his
initial question.

O represents the environment, he is responsible for proving input
values while P plays from the term's point of view: he is
responsible for performing the computation and returning the output
to O. P plays according to the strategy that is associated to the
$\lambda$-term being modeled.

We recall that in the cartesian closed category $\mathcal{C}$, the
objects are the arenas and the morphisms are the strategies. Given a
simple type $A$, we will model it as an arena $\sem{A}$. A context
$\Gamma = x_1 :A_1, \ldots x_n:A_n$ will be mapped to the arena
$\sem{\Gamma} = \sem{A_1} \times \ldots \times \sem{A_n}$ and a term
$\Gamma \vdash M : A$ will be modeled by a strategy on the arena
$\sem{\Gamma} \rightarrow \sem{A}$. Since $\mathcal{C}$ is cartesian
closed, there is is a terminal object $\textbf{1}$ (the empty arena)
that models the empty context ($\sem{\Gamma} = \textbf{1}$).


Let $\omega$ denotes the set of natural numbers. Consider the
following flat arena over $\omega$:
$$  \pstree[levelsep=6ex]
    {\TR[name=R]{q}}
    { \TR{1} \TR{2} \TR{\ldots}
    }
$$
Then the base type \texttt{exp} is interpreted by the flat game
$\nat$ over the previous arena where the set of valid position is:
$$P_N = \{ \epsilon, q \} \union \{ qn \ | \ n \in \omega \}$$


In this arena, there is only one question: the initial O-question, P
can then answer by playing a natural number $i \in \omega$. There
are only two kinds strategy on this arena:
\begin{itemize}
\item the empty strategy where P never answer the initial question. This corresponds to a non terminating computation;
\item the strategies where P answers by playing a number $n$. This models the numerical constants of the language.
\end{itemize}

Given the interpretation of base types, we define the interpretation
of $A\rightarrow B$ by induction:
$$\sem{A \rightarrow B} = \sem{A} \Rightarrow \sem{B}$$

where the operator $\Rightarrow$ denotes the arena construction $!A
\multimap B$, the exponential object of the cartesian closed
category $\mathcal{C}$.



Variables are interpreted by projection:
$$\sem{x_1 : A_1, \ldots, x_n:A_n \vdash x_i : A_i} = \pi_i : \sem{A_i} \times \ldots \times \sem{A_i} \times \ldots \times \sem{A_n} \rightarrow  \sem{A_i}$$

The abstraction $\Gamma \vdash \lambda x :A.M : A \rightarrow B$ is
modeled by a strategy on the arena $\sem{\Gamma} \rightarrow
(\sem{A}\Rightarrow\sem{B})$. This strategy is obtain by using the
currying operator of the cartesian closed category:
$$\sem{\Gamma \vdash \lambda x :A.M : A \rightarrow B} = \Lambda( \sem{\Gamma, x :A \vdash M : B})$$

The application $\Gamma \vdash M N$ is modeled using the evaluation
map $ev_{A,B} : (A\Rightarrow B)\times A \rightarrow B$:

$$\sem{\Gamma \vdash M N} = \langle \sem{\Gamma \vdash M, \Gamma \vdash N} \rangle \fatsemi ev_{A,B}$$


\subsubsection{PCF fragment}

We now show how to model PCF constructs in the game semantics
setting. In the following, the sub-arena of a game are tagged in
order to distinguish identical arenas present in different
components of the game. Moves are also tagged in the exponent in
order to identify the sub-arena in which moves are played. We will
omit the pointers in the play when there is no ambiguity.

The successor arithmetic operator is modeled by the following
strategy on the arena $\nat^1 \Rightarrow \nat^0$:
$$\sem{\texttt{succ}} = \{q^0 \cdot q^1 \cdot n^1 \cdot (n+1)^0\ |\ n \in \nat \}$$

The predecessor arithmetic operator is denoted by the strategy
$$\sem{\texttt{pred}} = \{q^0 \cdot q^1 \cdot n^1 \cdot (n-1)^0\ |\ n >0 \} \union \{ q^0 \cdot q^1 \cdot 0^1 \cdot 0^0 \} $$

Then given a term $\Gamma \vdash \texttt{succ }M : \texttt{exp}$ we
define:
$$\sem{\Gamma \vdash \texttt{succ } M : \texttt{exp}} = \sem{\Gamma \vdash M} \fatsemi \sem{\texttt{succ}} $$
$$\sem{\Gamma \vdash \texttt{pred } M : \texttt{exp}} = \sem{\Gamma \vdash M} \fatsemi \sem{\texttt{pred}} $$


The conditional operator is denoted by the following strategy on the
arena $\nat^3 \times \nat^2 \times \nat ^1 \Rightarrow \nat^0$:
$$\sem{\texttt{cond}} =
    \{ q^0 \cdot q^3 \cdot 0 \cdot q^2 \cdot n^2 \cdot n^0 \ | \ n \in \nat \}
    \union
    \{ q^0 \cdot q^3 \cdot m \cdot q^2 \cdot n^2 \cdot n^0 \ | \ m >0, n \in \nat \}
    $$


Given a term $\Gamma \vdash \texttt{cond}\ M\ N_1\ N_2$ we define:
$$\sem{\Gamma \vdash \texttt{cond}\ M\ N_1\ N_2} =
\langle \sem{\Gamma \vdash M}, \sem{\Gamma \vdash N_1}, \sem{\Gamma
\vdash N_2} \rangle \fatsemi \sem{\texttt{cond}}$$


The interpretation of the \texttt{Y} combinator is a bit more
complicated.

Consider the term $\Gamma \vdash M : A \rightarrow A$, its semantics
$f$ is a strategy on $\sem{\Gamma} \times \sem{A} \rightarrow
\sem{A}$. We define the chain $g_n$ of strategies on the arena
$\sem{\Gamma} \rightarrow \sem{A}$ as follows:
\begin{eqnarray*}
g_0 &=& \perp \\
g_{n+1} &=&  F(g_n) = \langle id_{\sem{\Gamma}}, g_n\rangle \fatsemi f
\end{eqnarray*}

where $\perp$ denotes the empty strategy $\{ \epsilon \}$.

It is easy to see that the $g_n$ forms a chain. We define
$\sem{\texttt{Y } M}$ to be the least upper bound of the chain $g_n$
(i.e. the  least fixed point of $F$). Its existence is guaranteed by
the fact that the category of games is cpo-enriched.

Since all the strategies that we have given are innocent and
well-bracketed, the game model of PCF can be interpreted in any of
the four categories $\mathcal{C}$, $\mathcal{C}_i$, $\mathcal{C}_b$,
$\mathcal{C}_{ib}$.



\subsection{Full-abstraction of PCF}
In this section we state the full abstraction result proved in
\cite{abramsky94full} and \cite{hylandong_pcf}.


\subsubsection{Observational preorder}

A context noted $C[-]$ is a term containing a hole denoted by $-$.
If $C[-]$ is a context then $C[A]$ denotes the term obtained after
replacing the hole by the term $A$.

If $M$ is a PCF term then we write $C[M]$ to denote the term
obtained after replacing the hole by the term $M$. $C[M]$ is
well-formed provided that $M$ has the appropriate type. Remark: this
capture permitting substitution must be distinguished from the
capture-free substitution which is noted $M[N/x]$ for any two terms
$M$ and $N$.


\begin{dfn}[Observational preorder]
We define the relation on terms $\obspre$ as follows: suppose $M$
and $N$ are two closed terms of the same type then:
\begin{eqnarray*}
M \obspre N &\iff& \parbox{10cm}{for all context $C[-]$ such that
                $C[M]$ and $C[N]$ are well-formed closed term of type \texttt{exp},
                    $C[M] \eval$ implies $C[N] \eval$}
\end{eqnarray*}
Observational equivalence is defined as the reflexive closure of
$\obspre$ noted $\obseq$.
\end{dfn}

Said informally, two programs are observationally equivalent if then
can be safely interchanged in any program context.

\subsubsection{Soundness and adequacy}
A model of a programming language is said to be \emph{sound} or
\emph{inequationally sound} if whenever the denotation of two
programs are equal then the two programs are observationally
equivalent, or more formally if for any closed terms $M$ and $N$ of
the same type:
$$ \sem{M} \subseteq \sem{N} \imp M \obspre N.$$

In a way, soundness is the minimum one can require for a model of
programming language: it guarantees that we can reason about the
program by manipulating the object of the denotational model.

It can be shown that the game model of PCF is sound for evaluation
and computationally adequate. These two properties imply the
soundness of the game model:

We said that the evaluation relation $\eval$ is sound if the
denotation is preserved by evaluation:
\begin{lem}[Soundness of evaluation]
\label{lem:evalsoundness}
 Let $M$ be a PCF term then
$$M \eval V \quad \imp \quad \sem{M} = \sem{V}.$$
\end{lem}

\begin{dfn}[Computable terms] \
\begin{itemize}
\item A closed term $\vdash M$ of base type is computable if $\sem{M} \neq \bot$
implies $M \eval$.
\item A higher-order closed term $\vdash M : A\rightarrow B$ is computable if $M N$ is computable for any computable closed term $\vdash  N:A$.
\item An open term $x_1 : A_1, \ldots, x_n : A_n \vdash M : A\rightarrow B$ is computable if $\vdash M [N_1/x_1, \ldots N_n/x_n]$ is computable
for all computable closed terms $N_1:A_1, \ldots, N_n:A_n$.
\end{itemize}
\end{dfn}

A model is \emph{computationally adequate} if all
terms are computable.
\begin{lem}[Computational adequacy]
\label{lem:computadequacy}
The game model of PCF is
computationally adequate.
\end{lem}
We refer the reader to \cite{abramsky:game-semantics-tutorial} for
the proofs.

Inequational soundness follows from the last two lemmas:
\begin{prop}[Inequational soundness]
\label{prop:ineqsoundness} Let $M$ and $N$ be two closed terms then
$$\sem{M} \subseteq \sem{N} \implies  M \obspre N $$
\end{prop}
\begin{proof}
  Suppose that $\sem{M} \subseteq \sem{N}$ and $C[M] \eval$ for some context $C[-]$. Then by compositionality of game semantics we also have
  $C[\sem{M}] \subseteq C[\sem{N}]$.
  Lemma \ref{lem:evalsoundness} gives $\sem{C[M]} \neq \bot$, therefore $\sem{C[N]} \neq \bot$.
  Lemma \ref{lem:computadequacy} then implies that $C[N] \eval$.
  Hence $M \obspre N$.
\end{proof}

\subsubsection{Definability}

We will now consider only strategies that are innocent and
well-bracketed (i.e. we work in the category $\mathcal{C}_{ib}$).

The definability result says that every compact element of the model
is the denotation of some term.
The compact morphisms of the category $\mathcal{C}_{ib}$ are those
with finite view-function.

The economical syntax of PCF prevents us from stating this
result directly: we need to consider an extension of PCF with some additional
constants. Indeed, there are strategies that are not the denotation of any term
in PCF, for instance the ternary conditional strategy : this
strategy denotes the computation that tests the value of its first
parameter, if its equal to zero or one then it returns the value of
the second or third parameter respectively, otherwise it returns the
value of the fourth parameter. This strategy is illustrated by the
following diagram:
$$
\begin{array}{ccccccccc}
!\bf N & \otimes & !\bf N & \otimes & !\bf N & \otimes & !\bf N & \multimap & !\bf N \\
&&&&&&&&q \\
q \\
0 \\
&& q \\
&& n \\
&&&&&&&&n \\
\hline
&&&&&&&&q \\
q \\
1 \\
&&&& q \\
&&&& n \\
&&&&&&&&n \\
\hline
&&&&&&&&q \\
q \\
m>1 \\
&&&&&& q \\
&&&&&& n \\
&&&&&&&&n \\
\end{array}
$$

It is possible to simulate this computation in PCF using the conditional operator, for instance the following term is a potential candidate:
$$ T = \texttt{cond}\ M\  N_1 (\texttt{cond}\  (\texttt{pred } M)\  N_2\  N_3)$$

Unfortunately the game semantics of $T$ is not given by the strategy that we have just defined, it is instead the following one:
$$
\begin{array}{ccccccccc}
!\bf N & \otimes & !\bf N & \otimes & !\bf N & \otimes & !\bf N & \multimap & !\bf N \\
&&&&&&&&q \\
q \\
0 \\
&& q \\
&& n \\
&&&&&&&&n \\
\hline
&&&&&&&&q \\
q \\
1 \\
q \\
0 \\
&&&& q \\
&&&& n \\
&&&&&&&&n \\
\hline
&&&&&&&&q \\
q \\
m>1 \\
q \\
m-1>0 \\
&&&&&& q \\
&&&&&& n \\
&&&&&&&&n \\
\end{array}
$$

To make up for this deficiency we add a family of terms to PCF: the $k$-ary conditionals:
$$ \texttt{case}_k\ N\ N_1\ N_2\ \ldots\ N_k$$
with the desired operational semantics:
$$ \rulef{M \eval i \quad N_{i+1} \eval V}{\texttt{case}_k\ N\ N_1\ N_2\ \ldots\ N_k\ \eval V}\ i \in \{0, \ldots,k-1\}.$$
The denotation of this term is given by the first strategy illustrated above.
The extended language is called PCF'.

We can now prove the definability result:
\begin{prop}[Definability]
\label{prop:definability} Let $A$ be a PCF type and $\sigma$ be a compact innocent and well-bracketed
strategy on $A$. There exists a PCF' term $M$ such that $\sem{M} = \sigma$.
\end{prop}

Note that definability is proved for PCF' and not for PCF.
Nevertheless, PCF' is a conservative extension of PCF: if $M$ and $N$ are terms such that for any PCF-context $C[-]$,
$C[M] \eval \imp C[N] \eval$ then the same is true for any PCF'-context. This is because $\texttt{case}_k$ constructs can be ``simulated''
in PCF, for instance $\texttt{case}_3$ can be replaced by the PCF term $T$ which shares the same operational semantics.

This observation will allow us to use definability in PCF' to
prove the full-abstraction of PCF.


\subsubsection{Full abstraction}

Full abstraction of PCF cannot be stated directly in the category $\mathcal{C}_{ib}$. Instead we need to consider the quotiented category
$\mathcal{C}_{ib}/\lesssim_{ib}$.

First we need to show that $\mathcal{C}_{ib}/\lesssim_{ib}$ is a model of PCF.
$\mathcal{C}_{ib}/\lesssim_{ib}$ is a posset-enriched cartesian closed category. The game semantics of the basic types and constants of PCF
can be transposed from $\mathcal{C}_{ib}$ to $\mathcal{C}_{ib}/\lesssim_{ib}$. Unfortunately it is not know whether $\mathcal{C}_{ib}/\lesssim_{ib}$
is enriched over the category of CPOs. However it can be proved that it is a rational category \citep{abramsky94full}
and this suffices to ensure that $\mathcal{C}_{ib}/\lesssim_{ib}$ is indeed a model of PCF.

The full abstraction of the game model then follows from
proposition \ref{prop:ineqsoundness} and \ref{prop:definability}:
\begin{thm}[Full abstraction]
Let $M$ and $N$ be two closed IA-terms.
$$\sem{M} \precsim_{ib} \sem{N} \ \iff \ M \obspre N$$
where $\precsim_{ib}$ denotes the intrinsic preorder of the category $\mathcal{C}_{ib}$.
\end{thm}

\section{The fully abstract game model for Idealized Algol (\ialgol)}

We now extend the work of the previous section to the language
\ialgol, an imperative extension of PCF. We start by giving the
syntax and operational semantics of the language, we then describe
the game model which was introduced in \cite{abramsky99full}.
Finally we will state the full abstraction result for the game
model.

\subsection{The syntax of \ialgol}
IA is an extension of PCF introduced by J.C. Reynold in
\cite{Reynolds81}. It adds imperative features such as local
variables and sequential composition.

On top of \texttt{exp}, PCF has the following two new types:
\texttt{com} for commands and \texttt{var} for variables. There is a
constant \texttt{skip} of type \texttt{com} which corresponds to the
command that do nothing.

Commands can be composed using the
sequential composition operator $\texttt{seq}_A$: suppose that $M$ and $N$ are of type
\texttt{com} and $A$ respectively then they can be composed to form the term
$S = \texttt{seq}_A M N : \texttt{com}$. $S$ denotes program that executes $M$ until it terminates and then
behave like $N:A$. If $A = \texttt{exp}$ then the expression is allowed to have a side-effect and $S$ returns the expression computed by $N$, if
$A = \texttt{com}$ then the command $N$ is executed after $M$.
We say that the language has \emph{active expressions} to indicate the presence of the
sequential operator $\texttt{seq}_{exp}$ in the language.


Local variable are
declared using the \texttt{new} operator, variable content is altered
using \texttt{assign} and retrieved using \texttt{deref}.

In addition IA has the constant \texttt{mkvar} that can be used to
create a particular kind of variables. The \texttt{mkvar} operator
works in an object oriented fashion: it takes two arguments, the
first one is a function (called the acceptor) that affects a value
to the variable and the second argument is an expression that
returns a value from the variable. This mechanism is very much like
the ``set/get'' object programming paradigm used by C++ programmers.

Variables created with \texttt{mkvar} are less constraint than the
variables created with \texttt{new}. Indeed, variables created with
\texttt{new} act like memory cells, they obey the following rule: the value read
from the variable is always the last value that has been assigned to
it. This rule does not apply to variables created with
\texttt{mkvar}. For instance the variable:
$$\texttt{mkvar}\ (\lambda v.\texttt{skip})\ 0$$
will always return $0$ even if another number has been assigned it.


One may think that this addition to the language is artificial,
however the full abstraction result of the game model of IA relies
upon this addition. At present, it is still an open problem to find
a fully abstract model of IA deprived of \texttt{mkvar}.

Judgement are of the form $\Gamma \vdash M : A$.
If the judgement $\Gamma = \emptyset$ we say that $M$ is a closed term.
The set of additional formations rules completing those of PCF are
given in table \ref{tab:ia_formrules}.

\begin{table}[htbp]
$$ \rulef{\Gamma \vdash M : \texttt{com} \quad \Gamma \vdash N :A}
    {\Gamma \vdash \texttt{seq}_A \ M\ N\ : A} \quad A \in \{ \texttt{com}, \texttt{exp}\}$$

$$ \rulef{\Gamma \vdash M : \texttt{var} \quad \Gamma \vdash N : \texttt{exp}}
    {\Gamma \vdash \texttt{assign}\ M\ N\ : \texttt{com}}
\qquad
 \rulef{\Gamma \vdash M : \texttt{var}}
    {\Gamma \vdash \texttt{deref}\ M\ : \texttt{exp}}$$

$$ \rulef{\Gamma, x : \texttt{var} \vdash M : A}
    {\Gamma \vdash \texttt{new } x \texttt{ in } M} \quad A \in \{ \texttt{com}, \texttt{exp}\}$$

$$ \rulef{\Gamma \vdash M_1 : \texttt{exp} \rightarrow \texttt{com} \quad \Gamma \vdash M_2 : \texttt{exp}}
    {\Gamma \vdash \texttt{mkvar } M_1\ M_2\ : \texttt{var}}$$

\caption{Formation rules for IA terms}
\label{tab:ia_formrules}
\end{table}


\subsection{Operational semantics of \ialgol}

The operational semantics of IA is given in a slightly different form compared to PCF.
Instead of giving the semantics for closed terms we consider terms
whose free variables are all of type \texttt{var}. Terms are
``closed'' by mean of stores. A store is a function mapping free
variables of type \texttt{var} to natural numbers. Suppose $\Gamma$
is a context containing only variables of type \texttt{var}, then we
say that $\Gamma$ is a \texttt{var}-context. A store with domain
$\Gamma$ is called a $\Gamma$-store. The notation $s\ |\ x \mapsto
n$ refers to the store that maps $x$ to $n$ and otherwise maps
variables according to the store $s$.

%%%% The following is poorly written:
%
%In PCF, the evaluation rules were given for closed terms only.
%Suppose that we proceed the same way for IA and consider the
%evaluation rule for the $\texttt{new}$ construct: the conclusion is
%$\texttt{new } x:=0 \texttt{ in } M$ and the premise is an
%evaluation for a certain term constructed from $M$, more precisely
%the term $M$ where \emph{some} occurrences of $x$ are replaced by
%the value $0$. Because of the presence of the \texttt{assign}
%operator, we cannot simply replace all the occurrences of $x$ in $M$
%(the required substitution is  more complicated than the
%substitution used for beta-reduction).


The canonical forms for IA are given by the grammar:
$$ V ::= \texttt{skip}\ |\ n\ |\ \lambda x. M\ |\ x\ |\  \texttt{mkvar}\ M\ N$$

where $n \in \nat$ and $x: \texttt{var}$.


In \ialgol, a program is a term together with a $\Gamma$-store such
that $\Gamma \vdash M : A$. The evaluation semantics is expressed by
the judgment form:
$$s,M \eval s', V$$
where $s$ and $s'$ are $\Gamma$-stores, $V$ is a canonical form and $\Gamma \vdash V : A$.

The operational semantics for IA is given by the rule of PCF (table \ref{tab:bigstep_pcf})
together with the rules of table \ref{tab:bigstep_ia} where the following abbreviation is used:
$$ \rulef{M_1 \eval V_1 \quad M_2 \eval V_2}{M \eval V} \qquad \mbox{for} \qquad
  \rulef{s,M_1 \eval s',V_1 \quad s', M_2 \eval s'',V_2 }{s,M \eval s'',V}
$$


\begin{table}[htbp]
$$\mbox{\textbf{Sequencing }}
    \rulef{M \eval \iaskip \quad N \eval V}{\texttt{seq } M\ N \eval V}
$$

$$\mbox{\textbf{Variables }}
    \rulef{s,N \eval s',n \quad s',M \eval s'',x}{s, \iaassign\ M\ N \eval (s''\ |\ x \mapsto n),\iaskip}
\qquad
    \rulef{s,M \eval s',x }{s, \iaderef\ M \eval s',s'(x)}$$

$$\mbox{\texttt{\textbf{mkvar}}}
    \rulef{N \eval n \quad M \eval \texttt{mkvar}\ M_1\ M_2 \quad M_1\ n \eval \iaskip}
    {\iaassign\ M\ N \eval \iaskip}
\qquad
    \rulef{N \eval \texttt{mkvar } M_1\ M_2 \quad M_2\ \eval n}
    {\iaderef\ M \eval n}
$$

$$\mbox{\textbf{Block}}
    \rulef{(s\ |\ x \mapsto 0),M \eval (s'\ |\ x \mapsto n),V }
    {s, \texttt{new } x \texttt{ in } M \eval s',V}
$$

\label{tab:bigstep_ia}
\caption{Big-step operational semantics of IA}
\end{table}


\subsection{Game model of \ialgol}

All the strategies used to model PCF are well-bracketed and
innocent. On the other hand, to obtain a model of IA we need to
introduce strategies that are not innocent.
This is necessary to model memory cell variable created with the \texttt{new} operator.
The intuition is that a cell needs to
remember what was the last value written in it in order to be able
to return it when it is read, and this can only be done by looking
at the whole history of moves, not only those present in the P-view.

Hence we now consider the categories $\mathcal{C}$ and $\mathcal{C}_b$.

\subsubsection{Base types}

The type \texttt{com} is modeled by the flat game with a single initial question \texttt{run} and a single answer
\texttt{done}. The idea is that O can request the execution of a command by playing \texttt{run}, P then execute the command
and if it terminates acknowledge it by playing \texttt{done}.

The variable type \texttt{var} is modeled by the game $\mathtt{com^{\bf N} \times exp}$ illustrated below:
\begin{center}
\begin{pspicture}(10cm,1.7cm)
$\rput[b]{0}(3cm,0){
\pstree[treemode=U,levelsep=8ex,nodesep=2pt]
    {\TR[name=R]{\mathtt{ok}}}
    { \TR{\mathtt{write}_0} \TR{\mathtt{write}_1} \TR{\mathtt{write}_2} \TR{\ldots}
    }
}
\rput[b]{0}(8cm,0){
\pstree[levelsep=8ex,nodesep=2pt]
    { \TR[name=R]{\mathtt{read}} }
    { \TR{0} \TR{1} \TR{2} \TR{\ldots} }
    }$
\end{pspicture}
\end{center}

\subsubsection{Constants}

\texttt{skip} is interpreted by the strategy $\{ \epsilon, \iarun \cdot \iadone \}$.
The sequential composition $\mathtt{seq_{exp}}$ is interpreted by the following strategy:
$$
\begin{array}{ccccc}
!\mathtt{com} & \otimes & ! \mathtt{exp} & \multimap & \mathtt{exp}\\
&&&&q\\
\iarun\\
\iadone\\
&&q\\
&&n\\
&&&&n
\end{array}
$$

Assignment \iaassign and dereferencing \iaderef are denoted  by the
following strategies (left and right respectively):
$$
\begin{array}{ccccc}
!\mathtt{var} & \otimes & ! \mathtt{exp} & \multimap & \mathtt{com}\\
&&&&q\\
&&q\\
&&n\\
\iawrite_n\\
\iaok\\
&&&&\iadone
\end{array}
\hspace{3cm}
\begin{array}{ccccc}
!\mathtt{var} & \multimap & \mathtt{exp}\\
&&q\\
\iaread\\
n\\
&&n
\end{array}
$$

\iamkvar is modeled by the paired strategy $\langle \iamkvar_{acc} , \iamkvar_{exp}
\rangle$ where $\iamkvar_{acc}$ and $\iamkvar_{exp}$ are the following strategies:
$$
\begin{array}{ccccccc}
(\mathtt{!exp} & \multimap & \mathtt{com}) & \otimes & !\mathtt{exp} & \multimap & \mathtt{com}^\omega\\
&&&&&&\iawrite_n\\
&&\iarun\\
q\\
n\\
&&\iadone \\
&&&&&&\iaok
\end{array}
\hspace{2cm}
\begin{array}{ccccccc}
(\mathtt{!exp} & \multimap & \mathtt{com}) & \otimes & !\mathtt{exp} & \multimap & \mathtt{exp}\\
&&&&&&\iaread\\
&&&&\iaread\\
&&&&n\\
&&&&&&n
\end{array}
$$


The strategies used until now are all innocent. In order to model the \ianew operator, we need to introduce non-innocent strategies, sometimes called
\emph{knowing strategies}. We define the knowing well-bracketed strategy $cell : I \multimap !\mathtt{var}$ that models a storage cell: it responds to \iawrite\
with \iaok\ and responds
to \iaread\ with the last value written or $0$ if no value has yet been written.

Consider the term $\Gamma,x:\mathtt{var} \vdash M : A$ modeled by $\sem{M}$ then the term
 $\Gamma \vdash \ianew\ x \texttt{ in } M : A$  will be modeled by the strategy $(id_{\sem{\Gamma}} \otimes cell) \fatsemi\fatsemi \sem{M}$ on the game
 $!\Gamma \multimap \iacom$.

\subsection{Full abstraction of \ialgol}

We now state the full abstraction result. All the details are omitted, the reader is refered
to \cite{abramsky:game-semantics-tutorial,AM97a} for the proofs.

\subsubsection{Inequational soundness}

The inequational soundness result can be also proved for \ialgol.
Proving soundness of the evaluation requires a bit more work than in the PCF case because
the store needs to be made explicit. Also, an appropriate notion of \emph{computable term} must be defined
that takes into account the presence of stores in the evaluation semantics.
Again it is possible to prove that the model is computational adequate.
The inequational soundness then follows from evaluation soundness and computational adequacy:

%\begin{lem}[Soundness for IA terms] Let $\Gamma \vdash M : A$ be an IA term and a $\Gamma$ store $s$.
%If $s,M \eval s',V$ then the plays of $\sem{s,M} : I \multimap A
%\otimes !\Gamma$ which begin with a move of $A$ are identical to
%those of $\sem{s',V}$.
%\end{lem}

\begin{prop}[Inequational soundness]
\label{prop:ia_ineqsoundness} Let $M$ and $N$ be two \ialgol\ closed terms then
$$\sem{M} \subseteq \sem{N} \implies  M \obspre N $$
\end{prop}

\subsubsection{Definability}

The proof of definability is based on a factoring argument: strategies in
$\mathcal{G}_b$ can all be obtained by composing the non-innocent strategy $cell$ with an innocent strategy.
The strategy $cell$ can therefore be viewed as a generic non-innocent strategy. Using this factorization argument,
it is possible to prove the definability result:
\begin{prop}[Definability]
\label{prop:ia_definability} Let $\sigma$ be a compact well-bracketed
strategy on a game $A$ denoting a IA type. There is an IA-term $M$ such
that $\sem{M} = \sigma$.
\end{prop}

\subsubsection{Full abstraction}

Full abstraction for the model $\mathcal{C}_b$ is a consequence of proposition
\ref{prop:ia_ineqsoundness} and \ref{prop:ia_definability}:
\begin{thm}[Full abstraction]
Let $M$ and $N$ be two closed \ialgol-terms.
$$\sem{M} \precsim_b \sem{N} \ \iff \ M \obspre N$$
where $\precsim_b$ denotes the intrinsic preorder of the category
$\mathcal{C}_b$.
\end{thm}


\section{Algorithmic game semantics}

After the resolution of the ``Full Abstraction of PCF'' problem,
game semantics has become a very successful paradigm in fundamental
computer science. It has permitted to give full abstract semantics
for a variety of programming languages. More recently, game
semantics has emerged as a new approach to program verification and
program analysis. In particular in the paper \cite{ghicamccusker00},
the authors considered a fragment of Idealized Algol for which the
game semantics of programs can be expressed simply using regular
expressions. In this setting, observational equivalence of programs
becomes decidable. Consequently, numbers of interesting verification
problem become solvable. This development opened up a new direction
of research called \emph{Algorithmic game semantics}.

\subsection{Characterization of observational equivalence}

In \citep{AM97a} it is shown that observational equivalence of IA is
characterized by complete plays.

A play of a game is \emph{complete} if it is maximal and all
question have been answered. A game is \emph{simple} if the complete
plays are exactly those in which the initial question has been
answered. It can be shown that for any IA type $T$, $\sem{T}$ is a
simple game. The following characterization theorem holds for simple
games:
\begin{thm}[Characterization Theorem for Simple Game (Abramsky, McCusker 1997)]
Let $\sigma$ and $\tau$ be strategies on a simple game $A$ then:
$$\sigma \leq \tau \iff \textsf{comp}(\sigma) = \textsf{comp}(\tau)$$
\end{thm}
Therefore terms in IA are fully described by the complete plays of
the corresponding strategies.

\subsection{Finitary fragments of Idealized algol}
We introduce
some fragments of the language \ialgol. Firstly, \emph{Finitary
Idealized Algol} denotes the recursion-free sub-fragment of \ialgol\
over finite ground types. A term $\Gamma \vdash M:T$ of finitary
Idealized algol is an $i^{th}$-order term if $T$ is of order $i$ at
most and the variables in $\Gamma$ are of order strictly less than
$i$. $\ialgol_i$ denotes the fragment of finitary Idealized Algol
consisting of terms of $i^{th}$-order terms. $\ialgol_i +
\textsf{while}$ denotes the fragment $\ialgol_i$ augmented with
primitive recursion. Finally $\ialgol_i + \textsf{Y}_j$ where $j
\leq i$ denotes the fragment $\ialgol_i$ augmented with a set of
fixed-point iterators $\textsf{Y}_A : (A\rightarrow A ) \rightarrow
A$ for any type $A$ of order $j$ at most.

We recall the observational equivalence decision problem: given two
$\beta$-normal forms $M$ and $N$ in a given fragment of \ialgol,
does $M \approx N$ hold?

This problem has been investigated and decidability results have
been obtained for a complete class of fragments of Idealized Algol.
These results help us to understand the limits of Algorithmic Game
Semantics. We now present briefly those results.

\subsubsection{$\ialgol_2$ fragment}
In \cite{ghicamccusker00}, Dan R. Ghica and Guy McCusker considered the $\ialgol_2$ fragment.
They show that in $\ialgol_2$ the set of complete plays are
representable by extended regular languages.

\begin{lem}[Ghica and McCusker 2000]
For any $\ialgol_2$-term $\Gamma \vdash M : T$, the set of complete
plays of $\sem{\Gamma \vdash M : T}$ is regular.
\end{lem}
Since equivalence of regular expression is decidable, this shows
decidability of observational equivalence of $\ialgol_2$-terms. In
the same paper they show that the same result holds for the
$\ialgol_2 +\textsf{while}$ fragment.

In \cite{Ong02}, it is shown that observational equivalence is
undecidable for $\ialgol_2 + \textsf{Y}_1$.


\subsubsection{Other fragments of IA}

Observational equivalence is decidable for $\ialgol_3$. This is
proved in \cite{Ong02} by reduction to the \emph{Deterministic
Push-down Automata Equivalence} problem. Unfortunately, this result
does not extend beyond order $3$: Murawski showed in
\cite{murawski03program} that the problem is undecidable for
$\ialgol_i$ with $i\geq4$.

However in $\ialgol_3 + \textsf{while}$ the problem becomes
decidable: it is shown in \cite{C:MW05} that the problem is EXPTIME
in $\ialgol_2 + \textsf{while}$ and $\ialgol_3 + \textsf{while}$.

Moreover in \cite{C:MOW05} it is shown that $\ialgol_i +Y_0$, for $i
= 1, 2, 3$ is as difficult as the DPDA equivalence problem. This
problem is decidable \citep{DBLP:journals/tcs/Senizergues01} but no
complexity result is known about it. We only know that it is
primitive recursive \citep{stirling02}.

\subsubsection{The complete classification}
\begin{center}
\begin{tabular}{rcccc}
Fragment  & pure & +while & +Y0 & +Y1 \\ \hline \hline
$\ialgol_0$ & PTIME & ??? & ??? & $\times$  \\
$\ialgol_1$ & coNP & PSPACE & DPDA EQUIV & ??? \\
$\ialgol_2$ & PSPACE & PSPACE & DPDA EQUIV & undecidable \\
$\ialgol_3$ &EXPTIME & EXPTIME & DPDA EQUIV & undecidable \\
$\ialgol_i, i \geq 4$  & undecidable & undecidable & undecidable
& undecidable
\end{tabular}
\end{center}

The $\times$ symbol denotes undefined \ialgol\ fragments.

The coNP and PSPACE results are due to Murawski \citep{Mur04b}.

%\input dataref


%%%%%%%%%%%%%%%%%%%%%%%%%%%%%%%%%%%%%%%%%%%%%%%%%%%%%%%%%%%%%
% second chapter
\chapter{Safe $\lambda$-Calculus}
In \cite{KNU02}, the authors introduced a restriction on
higher-order grammars called \emph{safety} in order to study the
infinite hierarchy of trees recognized by a higher-order pushdown
automaton. They proved that trees recognized by pushdown automata of
level $n$ coincide with trees generated by safe higher-order
grammars of level $n$. This characterisation permitted them to prove
the decidability of the monadic second-order theory of infinite
trees recognized by a higher-order pushdown automaton of any level.

Safety has also appeared in a different form in \cite{Dam82} under
the name \emph{restriction of derived types}. The forthcoming thesis
of Jolie de Miranda \citep{demirandathesis} contains a comparison of
safety and the restriction of derived types.

More recently, Ong proved in \cite{OngLics2006} that the safety
assumption of \cite{KNU02} is in fact not necessary. More precisely,
the paper shows that the MSO theory of trees generated by order-$n$
recursion schemes is $n$-EXPTIME complete.

For this particular problem, \emph{safety} happens to be an
artificial restriction. However when the \emph{safety} condition is
transposed to the simply-typed $\lambda$-calculus, it gives some
interesting properties. In particular, for safe terms, it becomes
unnecessary to rename variables when performing substitution.

This chapter starts with a presentation of the original version of
the safe $\lambda$-calculus where types are required to satisfy a
condition called homogeneity. We then give a more general definition
which does not require type homogeneity.

\section{Homogeneous Safe $\lambda$-Calculus}
\label{sec:safe_homog}

\subsection{Type homogeneity}
Let $Types$ be the set of simple types generated by the grammar $A
\, ::= \, o \; | \; A \typear A$. Any type different from the base
type $o$ can be written $(A_1, \cdots, A_n, o)$ for some $n \geq 1$,
which is a shorthand for $A_1 \typear \cdots \typear A_n \typear o$ (by
convention, $\rightarrow$ associates to the right). If $T=(A_1,
\cdots, A_n, o)$ then the arity of $T$, written $arity(T)$, is
defined to be $n$.

Suppose that a ranking function ${\sf rank} :
Types \funto (L, \leq)$ is given where $(L, \leq)$ is any linearly ordered
set. Possible candidates for the ranking function are:
\begin{itemize}
\item ${\sf ord} : Types \funto (\nat,\leq)$ with $\ord{o} = 0$
and $\ord{A \typear B} = \max(\ord{A}+1, \ord{B})$;
\item ${\sf height} : Types \funto (\nat,\leq)$ with
$\slheight{A \typear B} = 1 + \max(\slheight{A}, \slheight{B})$ and
$\slheight{o} = 0$ ;
\item ${\sf nparam} : Types \funto (\nat,\leq)$ with $\nparam{o} = 0$
and $\nparam{A_1, \cdots, A_n} = n$;
\item ${\sf ordernp} : Types \funto (\nat \times \nat,\leq)$ with $ {\sf ordernp} (t)  = \langle \order{t}, \nparam{t} \rangle$ for $t \in Types$.
\end{itemize}
Following \cite{KNU02}, we say that a type is {\sf rank}-homogeneous
if it is $o$ or if it is $(A_1, \cdots, A_n, o)$ with the condition
that $\rank{A_1} \geq \rank{A_2}\geq \cdots \geq \rank{A_n}$ and
each $A_1$, \ldots, $A_n$ is {\sf rank}-homogeneous.



Suppose that $\overline{A_1}$, $\overline{A_2}$, \ldots,
$\overline{A_n}$ are $n$ lists of types, where $A_{ij}$ denotes the
$j$th type of list $\overline{A_i}$ and $l_i$ the size of
$\overline{A_i}$, then the notation $A \; = \; (\overline{A_1} \, |
\, \cdots \, | \, \overline{A_r} \, | \, o)$ means that
\begin{itemize}
  \item $A$ is the type $(A_{11},A_{12},\cdots, A_{1l_1}, A_{21}, \cdots,A_{2l_2}, \cdots A_{n1},\cdots, A_{nl_n},o)$
  \item $\forall i: \forall u,v \in A_i : \rank u = \rank v $
  \item $\forall i,j . \forall u \in A_i . \forall v \in A_j . i<j \implies \rank u >
   \rank v $
\end{itemize}
and therefore $A$ is {\sf rank}-homogenous. This notation organises
the $A_{ij}$s into partitions according to their ranks. Suppose $B =
(\overline{B_1} \, | \, \cdots \, | \, \overline{B_m} \, | \, o)$,
we write $(\overline{A_1} \, | \, \cdots \, | \, \overline{A_n} \, |
\, {B})$ to mean
\[(\overline{A_1} \, | \, \cdots \, | \, \overline{A_n} \, | \,
\overline{B_1} \, | \, \cdots \, | \, \overline{B_m} \, | \, o).\]

From now on, we only consider the rank function {\sf ord}. We will
use the term ``homogeneous'' to refer to {\sf ord}-homogeneity.


\subsection{Safe Higher-Order Grammars}
We now present the original notion of safety introduced in \cite{KNU02} as a restriction for higher-order grammars. We present briefly the notion of higher-order grammar. The reader is referred to \cite{KNU02,demirandathesis,safety-mirlong2004}
for more details.

Suppose that $\Gamma$ is a set of typed symbols then the set of \emph{applicative terms} written $\mathcal{T}(\Gamma)$ is the
closure of $\Gamma$ under the application rule i.e. if $s: A\rightarrow B$ and $t:A$ are in $\mathcal{T}(\Gamma)$ then so is $st :B$.

\begin{dfn}[Higher-order grammar]
A \emph{higher-order grammar} is a tuple $\langle \Sigma,
\mathcal{N}, V, \mathcal{R}, S \rangle$, where
\begin{itemize}
\item $\Sigma$ is a ranked alphabet of terminals of order at most 1,
\item $V$ is a finite set of typed variables,
\item $\mathcal{N}$ is a finite set of homogeneously-typed non-terminals,
\item $S$ a distinguished symbol of $\mathcal{N}$ of ground type, called the start symbol,
\item $\mathcal{R}$ is a finite set of production rules, one for each $F : (A_1, \ldots, A_n, o) \in \mathcal{N}$, of the form
    $$ F z_1 \ldots z_m \rightarrow e$$
where $z_i$ is a variable of type $A_i$ and $e$ is an applicative
term of type $o$ in $\mathcal{T}(\Sigma \union \mathcal{N} \union
\{z_1 \ldots z_m \} )$. The $z_i$s are called the \emph{parameters}
of the rule.
\end{itemize}
\end{dfn}
A higher-order recursion scheme is a \emph{deterministic} higher-order grammar i.e. for each non-terminal $F \in \mathcal{N}$ there is exactly one production rule with $F$ on the left hand side.
Higher-order recursion schemes are used as generators
of infinite trees.

The order of a rewrite rule is the order of the non-terminal symbol
appearing on the left hand side of the rule. The order of a grammar
is the highest order of its non-terminals.

Safety is a syntactic restriction on higher-order grammars. It can be formulated as
follows:
\begin{dfn}[Safe higher-order grammar]
  Let $G$ be a higher-order grammar $G$ of order $n$
    whose non-terminals are of homogeneous type.
    $G$ is \emph{unsafe} if and only if there is a rewrite rule $F z_1 \ldots z_m \rightarrow e$ where
   $e$ contains a subterm $t$ such that:
  \begin{enumerate}
    \item $t$ occurs in an operand position in $e$,
    \item $t$ is of order $k>0$,
    \item $t$ contains a parameter of order strictly less than $k$.
  \end{enumerate}
  $G$ is \emph{safe} if it is not unsafe.
\end{dfn}

Let us illustrate the definition with an example taken from \cite{KNU02}:
\begin{exmp} Let $f:(o,o,o)$, $g,h:(o,o)$ and $a,b:o$ be $\Sigma$ constants.
 The grammar of level 3 with non-terminals $S:o$ and $F: ((o,o),o,o,o)$ and production rules:
\begin{eqnarray*}
    S &\rightarrow&  F g a b \\
    F \varphi x y &\rightarrow& f ( F ( F \varphi x ) y (h y)) (f (\varphi x) y)
\end{eqnarray*}
is not safe because the term $F \varphi x : (o,o)$ containing a variable of order $0$
occurs at an operand position in the right-hand side expression of the second rule.

On the other hand, the grammar with the following production rules is safe:
\begin{eqnarray*}
    S &\rightarrow&  G (g a) b \\
    G z y &\rightarrow& f ( G ( G z y) (h y)) (f z y)
\end{eqnarray*}
Moreover it can be shown that these two grammars are equivalent in the sense that they generate the same
infinite tree.
\end{exmp}


\subsection{Rules of the Safe $\lambda$-Calculus}

There is a correspondence between higher-order grammars and the simply-typed $\lambda$-calculus. The non-terminals of a recursion scheme can be interpreted as $\lambda$-abstractions in the
simply-typed $\lambda$-calculus. The $\Sigma$-constants are
interpreted as ``constructors'' constants (in the sense of
constructor used in functional programming languages to represent
abstract data-types such as trees). The notions of variable and
application are directly transposed to the equivalent notions in the
simply-typed $\lambda$-calculus. Using this analogy it is possible
to derive a version of the safety restriction for the
$\lambda$-calculus.

The safe $\lambda$-calculus has been first proposed in
\cite{DBLP:conf/fossacs/AehligMO05}, a corrected definition appeared
in \cite{Ong2005}. The definition that we give here is slightly more
general in the sense that we allow the use of $\Sigma$-constants of
any higher-order type whereas the original definition only allows
first-order constants.


The \textbf{safe $\lambda$-calculus} is a sub-system of the
simply-typed $\lambda$-calculus. Typing judgements (or
terms-in-context) are of the form:
\begin{equation}
\nonumber \seq{\overline{x_1}:\overline{A_1} \, | \, \cdots \, | \,
\overline{x_n} :  \overline{A_n}}{M : B}
\end{equation}
which is shorthand for $\seq{x_{11} : A_{11}, \cdots, x_{1r}:
A_{1r}, A_{21},\ldots }{M : B}$ such that the context variables are listed in decreasing type order and
 with the condition that $\ord{x_{ik}} < \ord{x_{jl}}$ for any $k, l$ and $i<j$.

\emph{Valid typing judgements} of the system are defined by
induction over the following rules, where $\Delta$ is a given
homogeneously-typed alphabet i.e.\ a countable set of symbols such that each
symbol has an homogeneous type, and $\Sigma$ is a set of homogeneously-typed
constants:

$$ \rulename{wk}
    {   \rulef{ \seq{\Gamma}{M:B} \qquad {\Gamma \subset \Delta} }
             { \seq{\Delta }{M : B}}
   }
\qquad
    \rulename{perm}
    {
      \rulef { \seq{\Gamma}{M:B} \qquad \sigma(\Gamma) \hbox{ homogeneous} }
            { \seq{\sigma(\Gamma)}{M : B} }
    }
$$

$$ \rulename{\Sigma\mbox{\textbf{-const}}}  \rulef{}{\seq{}{b : A}}\ b:A \in \Sigma
\qquad
 \rulename{var} \rulef{}{\seq{x_{ij} : A_{ij}\, }{x_{ij} : A_{ij}}}
$$

$$\rulename{abs}
\rulef{
 {\seq{\overline{x_1} : \overline{A_1}\, | \, \cdots\, | \, \overline{x_{n+1}} : \overline{A_{n+1}}}{M : B}}            \qquad \ord{\overline{A_{n+1}}} > \ord{B}
 }
 { \seq{\overline{x_1} : \overline{A_1}\, | \, \cdots\, | \, \overline{x_{n}} : \overline{A_{n}}}
     { \lambda \overline{x_{n+1}} : \overline{A_{n+1}} .M : (\overline{A_{n+1}} \, | \, B)  }
 }
$$

$$ \rulename{app} \rulef{{\seq{\Gamma}{M : (\overline{B_1} \, | \, \cdots \, | \, \overline{B_m} \, | \, o)} \qquad
\seq{\Gamma}{N_1 : B_{11}} \quad \cdots \quad \seq{\Gamma}{N_{l} :
B_{1l}} \qquad l = |\overline{B_1}| }}
    { \seq{\Gamma}{M N_1
\cdots N_{l} : (\overline{B_2} \, | \, \cdots \, | \,
\overline{B_m} \, | \, o)}} $$

$$ \rulename{app^+} \rulef
    {\seq{\Gamma}{M : (\overline{B_1} \, | \, \cdots \, | \, \overline{B_m} \, | \, o)} \qquad
    \seq{\Gamma}{N_1 : B_{11}} \quad \cdots \quad \seq{\Gamma}{N_{l} :
    B_{1l}} \qquad l < |\overline{B_1}| }
    { \seq{\Gamma}{M N_1
    \cdots N_{l} : (\overline{B} \, | \, \overline{B_2} \, | \ \cdots \, | \,
    \overline{B_m} \, | \, o)}} $$

where $\overline{B_1} = B_{11}, \ldots, B_{1l},\overline{B}$ with
the condition that every variable in $\Gamma$ has an order strictly greater
than $\ord{\overline{B_1}}$.






\begin{property}[Basic properties]
\label{proper:safe_basic_prop} Suppose $\Gamma \vdash M : B$ is a
valid judgment then
\begin{itemize}
\item[(i)] $B$ is homogeneous;
\item[(ii)] every free variable of $M$ has order at least $\ord{B}$;
\item[(iii)] $fv(M) \vdash M : B$,
\end{itemize}
where $fv(M) \subseteq \Gamma$ denotes the context constituted of the variables
in $\Gamma$ occurring free in $M$.
\end{property}
\begin{proof}
(i) and (ii) are proved by an easy structural induction. (iii) is
due to the fact that the weakening rule is the only rule which can
introduce a variable not occurring freely in $M$ in the context of a
typing judgement.
\end{proof}

We now define a special kind of substitution that performs
simultaneous substitution and permits variable capture i.e. that
does not rename variables when the substitution is performed on an
abstraction.

\begin{dfn}[Capture-permitting simultaneous substitution (for homogeneous safe terms)]
\label{dnf:safe_simsubst} We use the notation
$\subst{\overline{N}}{\overline{x}}$ for $\subst{N_1 \ldots N_n}{x_1
\ldots x_n}$ and $\overline{y}:\overline{A}$ for $y_1:A_1, \ldots,
y_p:A_p$. A safe term has necessarily one of the forms occurring on
the left-hand side of the following equations, where $M$, $N_1,
\ldots N_l$ are safe terms. The capture-permitting simultaneous
substitution is then defined by:
\begin{eqnarray*}
c \subst{\overline{N}}{\overline{x}} &=& c \quad \mbox{ where $c$ is a $\Sigma$-constant}\\
x_i \subst{\overline{N}}{\overline{x}} &=& N_i\\
 y \subst{\overline{N}}{\overline{x}} &=& y \quad \mbox{ if } y \not \neq x_i \mbox{ for all } i,\\
(M N_1 \ldots N_l) \subst{\overline{N}}{\overline{x}} &=& (M \subst{\overline{N}}{\overline{x}}) (N_1 \subst{\overline{N}}{\overline{x}}) \ldots  (N_l \subst{\overline{N}}{\overline{x}})\\
(\lambda \overline{y} : \overline{A}. M)
\subst{\overline{N}}{\overline{x}} &=& \lambda \overline{y} . M
\subst{\overline{N} \upharpoonright I}{\overline{x} \upharpoonright I} \\
&& \mbox{where } I  = \{ i \in 1..n \ | \ x_i \not \in \overline{y} \}
\end{eqnarray*}

where $ \upharpoonright$ is the index filtering operator: if $s$ is
a sequence and $I$ a set of indices then $s \upharpoonright I$ is
the subsequence of $s$ obtained by keeping only the element in $s$
at positions in $I$.
\end{dfn}

This substitution is well-defined for safe terms in the sense that safety is preserved by substitution:

\begin{lem}[Capture-permitting simultaneous substitution preserves safety]
\label{lem:subst_preserve_safety} Let $\Gamma \union \overline{x}
\vdash M$ be a safe term where $\overline{x}$ denotes a list of
variables (which do not necessarily belong to the same partition).

For any safe terms $\Gamma \vdash N_1, \cdots, \Gamma \vdash N_n$,
the capture-permitting simultaneous substitution $M[N_1 / x_1 ,
\cdots, N_n / x_n]$ is safe. In other words, the following judgment
is valid:
$$ \Gamma \vdash M[N_1 / x_1 , \cdots, N_n / x_n] $$
\end{lem}
\begin{proof}
An easy proof by an induction on the structure of the safe term.
\end{proof}



With the traditional substitution, it is necessary to rename
variables when performing substitution on an abstraction in order to
avoid possible variable capture. As a consequence, in order to
implement substitution one needs to have access to an unbound number
of variable names. An interesting property of the homogeneous Safe
$\lambda$-Calculus is that variable capture never occurs when
performing substitution. In other words, the traditional
substitution can be safely replaced by the capture-permitting
substitution:

\begin{lem}[No variable capture lemma]
\label{lem:homog_nocapture} In the safe $\lambda$-calculus, there is
no variable capture when performing the following capture-permitting
simultaneous substitution:
$$ M[N_1 / x_1 , \cdots, N_n / x_n] $$
provided that $\Gamma \union \overline{x} \vdash M$, $\Gamma \vdash  N_1, \cdots ,\Gamma \vdash  N_n$ are valid judgments.
\end{lem}

\begin{proof}
We prove the result by induction. The variable, constant and
application cases are trivial. For the abstraction case, suppose $M
= \lambda \overline{y} : \overline{A}. P$ where $\overline{y} = y_1
\ldots y_p$. The capture-permitting simultaneous substitution gives:
$$M \subst{\overline{N}}{\overline{x}} = \lambda \overline{y} . P
\subst{\overline{N} \upharpoonright I}{\overline{x} \upharpoonright
I} \mbox{ where } I  = \{ i \in 1..n \ | \ x_i \not \in \overline{y}
\}. $$


By the induction hypothesis there is no variable capture in $P
\subst{\overline{N} \upharpoonright I}{\overline{x} \upharpoonright
I}$. Hence variable capture can only happen when the variable $y_j$
occurs freely in $N_i$ and $x_i$ occurs freely in $P$ for some $i
\in I$ and $j \in 1..p$. In that case, property
\ref{proper:safe_basic_prop} (ii) gives:
$$ \ord{y_j} \geq \ord{N_i} = \ord{x_i}$$

Moreover $i\in I$ therefore $x_i \not \in \overline{y}$ and since $x_i$ occurs freely in $P$, $x_i$ must also occur freely in the safe term
$\lambda \overline{y}. P$. Thus, property \ref{proper:safe_basic_prop} (ii) gives:
$$ \ord{x_i} \geq \ord{\lambda y_1 \ldots y_p . T} \geq 1+ \ord{y_j} > \ord{y_j}$$

which, together with the previous equation, gives a contradiction.
\end{proof}




\subsection{Safe $\beta$-reduction}

We now introduce the notion of safe $\beta$-redex and show how to
reduce them using the capture-permitting simultaneous substitution.
We will then show that a safe $\beta$-reduction reduces to a safe
term.


In the simply-typed lambda calculus a redex is a term of the form $(\lambda x . M) N$.
We generalize this notion to the safe lambda calculus. We call multi-redex a term of the form
$(\lambda x_1 \ldots x_n . M) N_1 \ldots N_l$ (it is not required to have $n=l$).


We say that a multi-redex is safe if it respects the formation rules
of the safe $\lambda$-calculus: the multi-redex $(\lambda x_1 \ldots
x_n . M) N_1 \ldots N_l$ is a safe redex if the variable
$x_1,\ldots,x_n$ are abstracted altogether at once using the
abstraction rule and if the terms $N_1 \ldots N_l$ are applied to
the term $\lambda x_1 \ldots x_n . M$ at once using either the rule
$\rulename{app^+}$ or $\rulename{app}$. The formal definition
follows:

\begin{dfn}[Safe redex]
A safe redex is a term of the form:
$$(\lambda \overline{x} . M) N_1 \ldots N_l$$
such that
\begin{itemize}
  \item variables $\overline{x}=x_1\ldots x_n$ are abstracted
altogether by one occurrence of the rule $\rulename{abs}$ in the
proof tree (possibly followed by the weakening rule). This implies
that:
$$\ord{M} -1 \leq \ord{\overline{x}} = \ord{x_1} = \ldots = \ord{x_n};$$
\item the terms $(\lambda \overline{x} . M)$, $N_1$,
$N_l$ are applied together at once using either:
\begin{itemize}
    \item the rule $\rulename{app}$:
        $$   \rulef{
                    \Sigma \vdash \lambda \overline{x} . M : (\overline{B_1}|\ldots|\overline{B_m}|o)
                    \quad
                    \Sigma \vdash N_1         \quad \ldots \quad \Sigma \vdash N_l
                    \quad l = |\overline{B_1}|
            }
            {
            \Sigma \vdash (\lambda \overline{x} . M) N_1 \ldots N_l
            } (\mathbf{app}),
        $$
        in which case  $n\leq |\overline{B_1}| = l$;

\item or the rule $\rulename{app^+}$:
        $$   \rulef{
                    \Sigma \vdash \lambda \overline{x} . M : (\overline{B_1}|\ldots|\overline{B_m}|o)
                    \quad
                    \Sigma \vdash N_1         \quad \ldots \quad \Sigma \vdash N_l
                    \quad l < |\overline{B_1}|
            }
            {
            \Sigma \vdash (\lambda \overline{x} . L) N_1 \ldots N_l
            } (\mathbf{app^+}),
        $$
      in which case $n \leq |\overline{B_1}|$ and no relation holds between $n$ and $l$.
\end{itemize}
\end{itemize}
It is not required to have $n = |\overline{B_1}|$.
\end{dfn}

Note that there are safe terms of the form $(\lambda x_1 \ldots x_n
. M) N_1 \ldots N_l$ with $l>n$. For instance the term $ (\lambda f
. ((\lambda g h . h) a) ) a a$ of type $o \rightarrow o$ for some
constant $a:o \rightarrow o$ and variables $x : o$ and $f,g,h:o
\rightarrow o$, can be formed using the $\rulename{app}$ rule as
follows:
$$ \rulef{
    \emptyset \vdash (\lambda f . ((\lambda g h . h) a) ) : (o,o),(o,o),o,o
        \quad \emptyset \vdash a : o,o
        \quad \emptyset \vdash a : o,o
    }
    {
       \emptyset \vdash (\lambda f . ((\lambda g h . h) a) ) a a : o,o
    } \rulename{app}
$$


\begin{dfn}[Safe reduction $\beta_s$] \
\label{dfn:safereduction} For the sake of concision, the following
abbreviations are used $\overline{x} = x_1 \ldots x_n$,
$\overline{N} = N_1 \ldots N_l$, and when $n\geq l$, $\overline{x_L}
= x_1 \ldots x_l$, $\overline{x_R} = x_{l+1} \ldots x_n$.
\begin{itemize}
\item The relation $\beta_s$ is defined on the set of safe redex as follows:
\begin{eqnarray*}
\beta_s &=&
\{  \ (\lambda \overline{x} : \overline{A} . T) N_1 \ldots N_l \mapsto \lambda \overline{x_R}. T\subst{\overline{N}}{\overline{x_L}}  \\
&& \mbox{ where $(\lambda \overline{x} : \overline{A} . T) N_1 \ldots N_l$ is a safe redex such that $n> l$}
\} \\
&\union&
\{ \ (\lambda \overline{x} : \overline{A} . T) N_1 \ldots N_l \mapsto T\subst{\overline{N}}{\overline{x}} N_{n+1} \ldots N_l  \\
&& \mbox{ where $(\lambda \overline{x} : \overline{A} . T) N_1 \ldots N_l$ is a safe redex such that $n\leq l$}
\}
\end{eqnarray*}
where the notation $\subst{\overline{N}}{\overline{x}}$ denotes the capture-permitting simultaneous substitution.

\item
The safe $\beta$-reduction, written $\betasred$, is the closure of
the relation $\beta_s$ by compatibility with the formation rules of
the safe $\lambda$-calculus.
\end{itemize}
\end{dfn}



We observe that safe $\beta$-reduction is a certain kind of multi-steps $\beta$-reduction.
\begin{property}
$\betasred \subset \betaredtr$, i.e. the safe
$\beta$-reduction relation is included in the transitive closure of the $\beta$-reduction relation.
\end{property}
\begin{proof}
Suppose that $(M\mapsto N) \in \beta_s$. We show that $M \betared^* N$.
\begin{itemize}
\item Suppose that the safe-redex is
$M \equiv (\lambda \overline{x} : \overline{A} . T) N_1 \ldots N_l$ such that $n\leq l$ then:
\begin{eqnarray*}
 M &=_\alpha& (\lambda z_1 \ldots z_n .T [z_1,\ldots z_n /x_1,\ldots x_n] ) \ N_1  N_2 \ldots N_l
            \\
&& \mbox{where the $z_i$ are fresh variables}  \\
     &\betared& (\lambda z_2 \ldots z_n .T [z_1,\ldots z_n /x_1,\ldots x_n] \subst{N_1}{z_1} ) \ N_2 \ldots N_l \\
&& \mbox{ (because the $z_i$s do not occur freely in $N_1$) }\\
%%    &=_\alpha& (\lambda z_2 \ldots z_n .T [z_2,\ldots z_n /x_2,\ldots x_n] \subst{N_1}{x_1})\  N_2 \ldots N_l  \qquad \mbox{where the $z_i$ are fresh variables}  \\
    &\betared& \ldots \\
    &\betared& (T [z_1,\ldots z_n /x_1,\ldots x_n] \subst{N_1}{z_1}  \ldots \subst{N_n}{z_n})\  N_{n+1} \ldots N_l \\
    &\betared& (T [N_1\ldots N_l/x_1,\ldots x_l])\ N_{n+1} \ldots
    N_l,
\end{eqnarray*}
and since $T$ is safe, the substitution $T [N_1\ldots N_l/x_1,\ldots
x_l]$ in the last equation can be performed using the
capture-permitting substitution. Hence $M \betared^* N$.

\item
 Suppose that $M \equiv (\lambda \overline{x} : \overline{A} . T) N_1 \ldots N_l$ such that $n> l$, then necessarily
the redex must be formed using the $\rulename{app^+}$ rule. The
side-condition of this rules says that the free variables of the
terms $N_1, \ldots N_l$ have all order strictly greater than
$\ord{\overline{x}}$, hence the $x_i$s do not occur freely in $N_1,
\ldots N_l$. Therefore:
\begin{eqnarray*}
 M &=& (\lambda x_1 \ldots x_n .T) \ N_1  N_2 \ldots N_l  \\
     &\betared& (\lambda x_2 \ldots x_n .T \subst{N_1}{x_1} ) \ N_2 \ldots N_l \\
            && \mbox{(for $i \in 2..n$, $x_i$ does not occur freely in $N_1$)}\\
    &\betared& \ldots \\
    &\betared& \lambda x_{l+1} \ldots x_n . T \subst{N_1}{x_1}  \ldots \subst{N_l}{x_l} \\
        && \mbox{(for $i \in (l+1)..n$,  $x_i$ does not occur freely in $N_l$)}\\
    &\betared& \lambda x_{l+1} \ldots x_n . T [N_1\ldots, N_l /  \ x_1,\ldots, x_l] \\
        && \mbox{(the $x_i$ do not occur freely in $N_1, \ldots
        N_l$)},
\end{eqnarray*}
and since $T$ is safe, the substitution $T [N_1\ldots N_l/x_1,\ldots
x_l]$ in the last equation can be performed using the
capture-permitting substitution. Hence $M \betared^* N$.
\end{itemize}
\end{proof}

\begin{property} In the simply-typed $\lambda$-calculus:
\begin{enumerate}
\item $\betasred$ is strongly normalizing.
\item $\beta_s$ has the unique normal form property.
\item $\beta_s$ has the Church-Rosser property.
\end{enumerate}
\end{property}

\begin{proof}
1. This is because $\betasred \subset \betaredtr$ and, $\betared$ is
strongly normalizing in the simply-typed $\lambda$-calculus. 2. A
term has a safe redex iff it has a $\beta$-redex therefore the set
of $\beta_s$ normal forms is equal to the set of $\beta_s$ normal
forms. Hence, the unicity of $\beta$-normal form implies the unicity
of $\beta_s$-normal form. 3. is a consequence of 1 and 2.
\end{proof}


Capture-permitting simultaneous substitution preserves safety (lemma
\ref{lem:subst_preserve_safety}), consequently any safe redex
reduces to a safe term:

\begin{lem}[The safe reduction $\beta_s$ preserves safety]
\label{lem:homoh_safered_preserve_safety}
If $M$ is safe and $M \betasred N$ then $N$ is safe.
\end{lem}

\begin{proof}
It suffices to show that the relation $\beta_s$ preserves safety.
Consider the safe-redex $(s\mapsto t) \in \beta_s$ where $ s \equiv (\lambda x_1 \ldots x_n . M) N_1 \ldots N_l $ .
We proceed by case analysis on the last rule used to form the redex.
\begin{itemize}
\item Suppose the last rule used is $\rulename{app}$, then necessarily $n\leq l$ and the reduction is
$$(\lambda x_1 \ldots x_n . M) N_1 \ldots N_l \qquad \mapsto  \qquad t \equiv M[N_1 / x_1 , \cdots, N_n / x_n]\ N_{n+1} \ldots N_l.$$
The first premise of the rule $\rulename{app}$ tells us that $M$ is safe therefore using lemma \ref{lem:subst_preserve_safety} and
the application rule we obtain that $t$ is safe.

\item Suppose the last rule used is $\rulename{app^+}$ and $n> l$ then the reduction is
$$ (\lambda \overline{x_L} : \overline{A_L} \
\overline{x_R}: \overline{A_R} . T) \overline{N_L} \qquad \mapsto
\qquad t \equiv \lambda \overline{x_R}: \overline{A_R} .
T\subst{\overline{x_L}}{\overline{N_L}}.
$$
By lemma \ref{lem:subst_preserve_safety}, $T\subst{\overline{x_L}}{\overline{N_L}}$ is a safe term.
Using the rule $\rulename{abs}$ we derive that $t$ is safe.

\item Suppose the last rule used is $\rulename{app^+}$ and $n\leq l$ then the reduction is
$$(\lambda x_1 \ldots x_n . M) N_1 \ldots N_l \qquad \mapsto \qquad t \equiv M[N_1 / x_1 , \cdots, N_n / x_n]\ N_{n+1} \ldots N_l$$
We conclude that $t$ is safe similarly to case $\rulename{app}$.

\item Rule $\rulename{wk}$ $\rulename{seq}$: these cases reduce to one of the previous cases.
\end{itemize}
\end{proof}


\begin{rem}
\label{rem:betasred_notpreserv_unsafety} $\betasred$ \emph{does not}
preserves un-safety: given two terms $S$ safe and $U$ unsafe of the
same type, the term $(\lambda x y . y) U S$ is also unsafe but it
$\beta_s$-reduces to $S$ which is safe.
\end{rem}


\subsection{An alternative system of rules}


In this section, we will refine the formation rules
given in the previous section. We say that $\Gamma \vdash M : A$ satisfies $P_i$ for $i \in \zset$ if the
variables in $\Gamma$ all have orders at least $\ord{A}+i$. We introduce the notation $\Gamma \vdash^{i} M : A$ for $i \in
\zset$ to mean that $\Gamma \vdash M : A$ is a valid judgment satisfying $P_i$.


We remark that if $\Gamma \vdash M : A$ then the variables in $\Gamma$ with order
strictly smaller than $M$ cannot occur freely in $M$ and therefore it is possible to restrict
the context to a smaller number of variables:

\begin{lem}[Context reduction]
\label{lem:restriction}

If $\Gamma \vdash^i M : A$ then $\Gamma' \vdash^{0} M : A$
where $$\Gamma' = \{ z \in \Gamma \ |
\ \ord{z} \geq \ord{M} \} = \Gamma \setminus \{ z \in \Gamma \ | \ \ord{M} + i \leq \ord{z} < \ord{M} \}$$
\end{lem}
\begin{proof}
If $i\geq 0$ then the result is trivial. Suppose $i<0$. We proceed
by structural induction and case analysis. We only give the details
for the application cases $\rulename{app}$ and $\rulename{app^+}$:
\begin{itemize}
\item Case of the rule $\rulename{app}$:

    \[ (\mathbf{app}) \
    \rulef
        {\seq{\Gamma}{M : (\overline{B_1} \, | \, \cdots \, | \, \overline{B_m} \, | \, o)} \qquad
            \seq{\Gamma}{N_1 : B_{11}} \quad \cdots \quad \seq{\Gamma}{N_{l} :
            B_{1l}} \qquad l = |\overline{B_1}| }
        { \seq{\Gamma}{M N_1
            \cdots N_{l} : (\overline{B_2} \, | \, \cdots \, | \,
            \overline{B_m} \, | \, o)}}
    \]

    If the conclusion satisfies $P_i$ then, for all $z \in \Gamma$:
    \begin{eqnarray*}
    \ord{z} \geq 1 + \ord{\overline{B_2}} + i
    &=& 1 + \ord{\overline{B_1}} + \ord{\overline{B_2}} - \ord{\overline{B_1}} + i \\
    &=& \ord{M} + (\ord{\overline{B_2}} - \ord{\overline{B_1}} + i)
    \end{eqnarray*}
    Therefore the first premise satisfies $P_j$ where $j={\ord{\overline{B_2}} - \ord{\overline{B_1}} + i}$.
    Hence by the induction hypothesis,
    $$\Gamma' \vdash^{0} M : (\overline{B_1} \, | \, \cdots \, | \, \overline{B_m} \, | \, o)$$
    where $\Gamma' = \Gamma \setminus \{ z \in \Gamma \ | \ \ord{M} + j \leq \ord{z} < \ord{M} \}$.


    Similarly, for all $z \in \Sigma$:
    \begin{eqnarray*}
    \ord{z} \geq 1 + \ord{\overline{B_2}} + i
    &=& \ord{\overline{B_1}} + (1+\ord{\overline{B_2}} - \ord{\overline{B_1}} + i) \\
    &=& \ord{\overline{B_1}} + j+1
    \end{eqnarray*}
    Hence by the induction hypothesis:
    $$\Gamma'' \vdash^0 N_k : B_{1k} \mbox{ for } k \in 1..l$$
    where $\Gamma'' = \Gamma \setminus \{ z \in \Gamma \ | \ \ord{M} + j+1 \leq \ord{z} < \ord{M} \}$.

    Furthermore, $\Gamma'' = \Gamma' \union \{ z \in \Gamma \ | \ \ord{M} + j = \ord{z}\}$ therefore
    the weakening rule gives:
    $$\Gamma'' \vdash^{-1} M : (\overline{B_1} \, | \, \cdots \, | \, \overline{B_m} \, | \, o)$$

    Finally the $\rulename{app}$ rule gives:
    $$\rulef{\Gamma'' \vdash^{-1} M : (\overline{B_1} \, | \, \cdots \, | \, \overline{B_m} \, | \, o)
    \quad \Gamma'' \vdash^0 N_1 : B_{11} \quad \ldots \quad \Gamma'' \vdash^0 N_1 : B_{1l}
    }
        { \Gamma'' \vdash M N_1 \ldots N_l : (\overline{B_2} \, | \, \cdots \, | \,
            \overline{B_m} \, | \, o)}
    $$
    such that for all $z\in \Gamma''$:
    \begin{eqnarray*}
    \ord{z} \geq \ord{\overline{B_1}}
    &\geq& 1 + \ord{\overline{B_2}} = \ord{M N_1 \ldots N_l}
    \end{eqnarray*}

    Therefore:
    $$\Gamma'' \vdash^0 M N_1 \ldots N_l : (\overline{B_2} \, | \, \cdots \, | \,
            \overline{B_m} \, | \, o)$$

\item $\rulename{app^+}$  The side-condition of the rule $\rulename{app^+}$ ensures that the first premise
 satisfies $P_0$. The conclusion of the rule has the same order as the first premise
 therefore the conclusion also satisfies $P_0$.
\end{itemize}
\end{proof}


\begin{lem}
\label{lem:prooftree01only} If $\Gamma \vdash^{0} M : T$ or $\Gamma
\vdash^{-1} M : T$ then there is a valid proof tree showing $\Gamma
\vdash M : T$ such that all the judgments appearing in the proof
tree satisfy either $P_0$ or $P_{-1}$.
\end{lem}


\begin{proof}
Since $P_{-1}$ implies $P_0$, w.l.o.g. we can assume that the
judgment $\Gamma \vdash M : T$ satisfies $P_{-1}$. We show that
there is a proof tree for $\Gamma \vdash M : T$ where all the nodes
of the tree satisfy $P_0$ or $P_{-1}$. We proceed by structural
induction and case analysis on the last rule used to show $\Gamma
\vdash M : T$:
\begin{itemize}
\item Axiom $\rulename{\Sigma\mbox{\textbf{-const}}}$: the context is empty therefore the sequent satisfies $P_{-1}$.

\item Axiom $\rulename{var}$: the context contains only the variable itself therefore the sequent satisfies $P_0$.

\item Rule $\rulename{wk}$: The premise is $\Delta \vdash M : T$ with $\Delta \subset \Gamma$. Since
$\Gamma \vdash M : T$ satisfies $P_{-1}$ and $\Delta \subset \Gamma$ the premise must also satisfy $P_{-1}$. We can conclude using the
induction hypothesis.

\item Rule $\rulename{perm}$: By the induction hypothesis.


\item Rule $\rulename{abs}$: the second premise of the rule guarantees that the first
premise satisfies $P_{-1}$.

\item Rule $\rulename{app^+}$: The first premise has the same order as the
conclusion of the rule therefore the first premise satisfies
$P_0$. The side-condition of the rule $\rulename{app^+}$ ensures that all the other premises satisfy $P_0$.

\item Rule $\rulename{app}$:

$$ \rulename{app} \
    \rulef{
        { \Gamma \vdash M : (\overline{A} \, | B)
        \qquad
        \Gamma \vdash N_1 : A_1 \quad \cdots \quad \Gamma \vdash N_{l} : A_l \qquad l = |\overline{A}|
        }
    }
    {
        \Gamma \vdash^0 M N_1 \cdots N_{l} : B
    }
$$

Applying lemma \ref{lem:restriction} to the first premise we obtain:
\begin{equation}
 \Sigma \vdash^0 M : (\overline{A} \, | B) \label{eq:seq1}
\end{equation}
where $\Sigma = \{ z \in \Gamma \ | \ \ord{z} \geq \ord{(\overline{A} \, | B)} \} = \{ z \in \Gamma \ | \ \ord{z} \geq 1 + \ord{\overline{A}} \}.$

Applying lemma \ref{lem:restriction} to each of the remaining
premises gives  :
$$ \Sigma' \vdash^0 N_i : A_i \quad \mbox{ for all } i \in 1..p$$
where $\Sigma' = \{ z \in \Gamma \ | \ \ord{z} \geq \ord{A_i} =
\ord{\overline{A}} \} \supseteq \Sigma.$

If the inclusion $\Sigma \subseteq \Sigma'$ is strict then we apply the weakening rule to sequent (\ref{eq:seq1}):
$$ \rulef{\Sigma \vdash^0 M : (\overline{A} \, | B)}{\Sigma' \vdash^{-1} M : (\overline{A} \, | B)} \rulename{wk} $$

Finally, we obtain the following proof tree:
$$  \rulef{
        \rulef{
            { \Sigma' \vdash^{-1} M : (\overline{A} \, | B)
            \qquad
            \Sigma' \vdash^0 N_1 : A_1 \quad \cdots \quad \Sigma' \vdash^0 N_{l} : A_l \qquad l = |\overline{A}|
            }
        }
        {
            \Sigma' \vdash^0 M N_1 \cdots N_{l} : B
        } \rulename{app}
    }
    {
         \Gamma \vdash^0 M N_1 \cdots N_{l} : B
    } \rulename{wk}
$$

where the last weakening rules is applied only if the inclusion $\Sigma' \subseteq \Gamma$ is strict.

We can now conclude by applying the induction hypothesis on the
sequents $\Sigma' \vdash^{-1} M$, $\Sigma' \vdash^0 N_1$, \ldots,
$\Sigma' \vdash^0 N_l$ .
\end{itemize}
\end{proof}

\subsubsection{An Alternative Definition of the Homogeneous Safe $\lambda$-Calculus}

Using the observations that we have just made, we will now derive
new rules for the safe $\lambda$-calculus with homogeneous type. We
want a system of rules generating sequents that satisfy $P_0$. Also,
it must be able to generate intermediate sequents that do not
necessarily satisfy $P_0$ provided that they can be used to produce
\emph{in fine} terms satisfying $P_0$.

Because of the lemma \ref{lem:prooftree01only}, we know that the
only necessary intermediate sequents are those that either satisfy
$P_0$ or $P_{-1}$. Hence, we can assume by default that premises of
the rules all satisfy $P_{-1}$ at least.

First we define an additional rule expressing the fact that $P_0$
implies $P_{-1}$:
$$ \rulename{seq} \  \rulef{\Gamma \vdash^{0} M : A}{\Gamma \vdash^{-1} M : A} $$

The weakening rule can be rewritten as follows:
$$ \rulename{wk^{0}} \   \rulef{\Gamma \vdash^{0} M : A}{\Gamma , x : B \vdash^{0} M : A} \quad \ord{B} \geq \ord{A} $$
$$ \rulename{wk^{-1}} \   \rulef{\Gamma \vdash^{-1} M : A}{\Gamma , x : B \vdash^{-1} M : A} \quad \ord{B} \geq \ord{A} -1$$

Because of the context reduction lemma, any sequent satisfying $P_{-1}$ can be obtained
by applying the weakening rule $\rulename{wk^{-1}}$ or the rule $\rulename{seq}$ to another sequent
satisfying $P_0$. Therefore, with the exception of these two rules, we only need to use rules whose conclusion sequents satisfy $P_0$:
\begin{itemize}
\item For the rules $\rulename{perm}$, $\rulename{const}$ and $\rulename{var}$, only the tagging of the sequents
changes:
$$ \rulename{var} \   \rulef{}{x : A\vdash^{0} x : A}
\qquad
\rulename{\Sigma\mbox{\textbf{-const}}}  \  \rulef{}{\vdash^0 b : A} \ b:A \in \Sigma
$$

$$
  \rulename{perm} \  \rulef{
      { \Gamma \vdash^0 M:B \qquad \sigma(\Gamma)  } \hbox{ homogeneous}
    }
      { \sigma(\Gamma) \vdash^0 M : B }
$$

\item $\rulename{abs}$ The abstraction rule has a side condition
expressing the fact that the premise satisfies $P_0$ or $P_{-1}$. Since this is always true for sequents
generated by our new system of rules, we can drop the side condition:
$$ \rulename{abs} \   \rulef{\Gamma | \overline{x} : \overline{A} \vdash^{-1} M : B}
                                   {\Gamma  \vdash^{0} \lambda \overline{x} : \overline{A} . M : (\overline{A},B)}$$


\item $\rulename{app}$ The application rule has the following form:
$$ \rulename{app} \
    \rulef{
        { \Gamma \vdash^{-1} M : (\overline{A} \, | B)
        \qquad
        \Gamma \vdash^{-1} N_1 : A_1 \quad \cdots \quad \Gamma \vdash^{-1} N_{l} : A_l \qquad l = |\overline{A}|
        }
    }
    {
        \Gamma \vdash^0 M N_1 \cdots N_{l} : B
    }
$$

Since the first premise satisfies $P_{-1}$, by property \ref{proper:safe_basic_prop}(ii) we have:
$$\forall z \in \Gamma : \ord{z} \geq 1 + \ord{\overline{A}} -1 = \ord{\overline{A}} = \ord{\overline{N}}$$
Hence, all the sequents of the premises but the first must satisfy $P_0$. The rule (app) is therefore given by:
$$ \rulename{app} \
    \rulef{
        { \Gamma \vdash^{-1} M : (\overline{A} \, | B)
        \qquad
        \Gamma \vdash^0 N_1 : A_1 \quad \cdots \quad \Gamma \vdash^0 N_{l} : A_l \qquad l = |\overline{A}|
        }
    }{
        \Gamma \vdash^0 M N_1 \cdots N_{l} : B
      }
$$

\item For the application rule $\rulename{app^+}$, the type of the sequent in the first premise has the same order
as the type of the conclusion premise, and since the conclusion
satisfies $P_0$, the first premise must also satisfy $P_0$. The
side-condition implies that all the other sequents in the premise
satisfy $P_0$. Moreover since the first premise satisfies $P_0$, the
side-condition must hold. Hence the rule becomes:
$$ \rulename{app^+} \
    \rulef{
        \Gamma \vdash^0 M : (\overline{B_1} \, | \, \cdots \, | \, \overline{B_m} \, | \, o) \qquad
        \Gamma \vdash^0 N_1 : B_{11} \quad \cdots \quad \Gamma \vdash^0 N_{l} : B_{1l} \qquad l < |\overline{B_1}|
    }
    {
        \Gamma \vdash^0 M N_1 \cdots N_{l} : (\overline{B} \, | \, \cdots \, | \, \overline{B_m} \, | \, o)
    }
$$
where $\overline{B_1} = B_{11}, \ldots, B_{1l},\overline{B}$.
Clearly, this rule can be equivalently stated as:
$$ \rulef{\Gamma \vdash^0 M : A\rightarrow B
                                        \qquad \Gamma \vdash^{0} N : A
                                   }
                                   {\Gamma  \vdash^{0} M N : B}$$
\end{itemize}

The full set of rules is given in table \ref{tab:homosafelmd_rules_refined}.

\begin{table}[htbp]
$$  \rulename{perm} \
    \rulef{
      { \Gamma \vdash^0 M:B \qquad \sigma(\Gamma)  } \hbox{ homogeneous}
    }
    { \sigma(\Gamma) \vdash^0 M : B
    }
\qquad
\rulename{seq} \  \rulef{\Gamma \vdash^{0} M : A}{\Gamma \vdash^{-1} M : A}
$$

$$
\rulename{\Sigma\mbox{\textbf{-const}}} \  \rulef{}{\vdash^0 b : A}\ b:A \in \Sigma
\qquad
 \rulename{var} \   \rulef{}{x : A\vdash^{0} x : A} $$

$$ \rulename{wk^{0}} \   \rulef{\Gamma \vdash^{0} M : A}{\Gamma , x : B \vdash^{0} M : A} \quad \ord{B} \geq \ord{A} $$

$$ \rulename{wk^{-1}} \   \rulef{\Gamma \vdash^{-1} M : A}{\Gamma , x : B \vdash^{-1} M : A} \quad \ord{B} \geq \ord{A} -1$$


$$ \rulename{app} \
    \rulef
        {   \Gamma \vdash^{-1} M : (\overline{A} \, | B)
            \qquad
            \Gamma \vdash^0 N_1 : A_1 \quad \cdots \quad \Gamma \vdash^0 N_{l} : A_l \qquad l = |\overline{A}|
        }
        {
            \Gamma \vdash^0 M N_1 \cdots N_{l} : B
        }
$$

$$ \rulename{app^+} \   \rulef{\Gamma \vdash^0 M : A\rightarrow B
                                        \qquad \Gamma \vdash^{0} N : A
                                   }
                                   {\Gamma  \vdash^{0} M N : B}$$

$$ \rulename{abs} \   \rulef{\Gamma| \overline{x} : \overline{A} \vdash^{-1} M : B}
                                   {\Gamma  \vdash^{0} \lambda \overline{x} : \overline{A} . M : (\overline{A}|B)}$$


where $\Gamma| \overline{x} : \overline{A}$ means that the lowest type-partition of the context is
$\overline{x} : \overline{A}$.
\caption{Alternative rules for the homogeneous safe lambda calculus}
\label{tab:homosafelmd_rules_refined}
\end{table}
%%%

\clearpage

\section{Safe $\lambda$-Calculus without the Homogeneity Constraint}
\label{sec:safe_nonhomog}


In section \ref{sec:safe_homog}, we have presented a version of the
safe lambda calculus where types are required to be homogeneous. We
now give a more general version of the safe simply-typed
$\lambda$-calculus where type homogeneity is not required.

\subsection{Rules}

We use a set of sequents of the form $\Gamma \vdash M : A$ where
$\Gamma$ is the context of the term and $A$ is its type. Let
$\Sigma$ be a set of higher-order constants. We call safe terms any
simply-typed lambda term that is typable within the following system
of formation rules:
$$ \rulename{var} \   \rulef{}{x : A\vdash x : A}
\qquad  \rulename{const} \   \rulef{}{\vdash f : A} \quad f \in \Sigma
\qquad  \rulename{wk} \   \rulef{\Gamma \vdash M : A}{\Delta \vdash M : A} \quad \Gamma \subset \Delta$$

$$ \rulename{app} \  \rulef{\Gamma \vdash M : (A,\ldots,A_l,B)
                                        \qquad \Gamma \vdash N_1 : A_1
                                        \quad \ldots \quad \Gamma \vdash N_l : A_l  }
                                   {\Gamma  \vdash M N_1 \ldots N_l : B}
                                    \quad
                                   \forall y \in \Gamma : \ord{y} \geq \ord{B}$$

$$ \rulename{abs} \   \rulef{\Gamma \union \overline{x} : \overline{A} \vdash M : B}
                                   {\Gamma  \vdash \lambda \overline{x} : \overline{A} . M : (\overline{A},B)} \qquad
                                   \forall y \in \Gamma : \ord{y} \geq \ord{\overline{A},B}$$


Remark:
\begin{itemize}
\item $(\overline{A},B)$ denotes the type $(A_1,A_2, \ldots, A_n, B)$;
\item all the types appearing in the rule are not required to be homogeneous (for instance
it is possible to have $\ord{A_l} < \ord{B}$ in rule $\rulename{app}$) ;
\item the environment $\Gamma \union \overline{x}:\overline{A}$ is not stratified, in particular, variables in $\overline{x}$ do not necessarily have the same order;
\item in the abstraction rule, the side-condition imposes that at least all variables of the lowest order
in the context are abstracted. Variables of greater order can also be
abstracted together with the lowest order variables and, in contrast to
the homogeneous safe lambda calculus, there is no constraint on the
order in which these variables are abstracted;
\end{itemize}

\begin{exmp}
For $x:o$, $f:(o,o)$ and $\varphi:((o,o),o)$ the term $$\vdash \lambda x f \varphi .
\varphi : (o , (o, o) , ((o,o),o) , (o,o),o)$$ is
a valid safe term that is not homogeneously typed.
\end{exmp}

\begin{exmp}
For $x:o$, $g:(o,(o,o),o)$, the term $\vdash \lambda g x . g x$ is unsafe and not homogeneously typed
and the term $\lambda g x . g x (\lambda x . x)$ is safe and not homogeneously typed.
\end{exmp}

Side-remark: safety is preserved by full $\eta$-expansion. Indeed,
consider the safe term $\Gamma \vdash M:(A_1,\ldots,A_l,o)$ where
$(A_1,\ldots,A_l,o)$ is not necessarily homogeneous. Its full $\eta$-expansion
is $\lambda x_1 .. x_l . M x_1 \dots x_l$ for some variables
$x_1:A_1, \ldots, x_l:A_l$ fresh in $M$. For all $i \in 1..l$ we
have $\Gamma, \Sigma \vdash x_i :A_i$ where $\Sigma = \{ x_1:A_1,
\cdots x_l :A_l \}$. Applying $\rulename{app}$ we obtain $\Gamma,
\Sigma \vdash M x_1 \ldots x_l$ and by the (abs) rule we get
$$\Gamma \vdash \lambda x_1:A_1 \ldots x_l:A_l .M x_1 \ldots x_l.$$

\begin{lem}[Context reduction]
\label{lem:nonhomosafe_basic_prop}
If $\Gamma \vdash M : B$ is a valid judgment then
\begin{enumerate}
\item $fv(M) \vdash M : B$
\item every variable in $\Gamma$ \emph{occurring free in $M$} has order at
least $ord(M)$.
\end{enumerate}
where $fv(M)$ denotes the context constituted of the free variables occurring in $M$.
\end{lem}
\begin{proof}
(i) Suppose that some variable $x$ in $\Gamma$ does not occur free
in $M$, then necessarily $x$ has been introduced in the context
using the weakening rule. Hence $\Gamma\setminus \{ x \} \vdash M$
must also be typable. (ii) An easy structural induction.
\end{proof}

The converse of this lemma is not true: consider the simply-typed
term $\lambda y z. (\lambda x . y ) z$ with $x,y,z:o$. This term is
closed therefore it satisfies property (i) and (ii) of lemma
\ref{lem:nonhomosafe_basic_prop}. However it is not typable by the
rules of the safe lambda-calculus since the subterm $\lambda x .y$
is not safe.

\subsection{Substitution in the safe lambda calculus}

The traditional notion of substitution, on which the
$\lambda$-calculus is based, is defined as follows:
\begin{dfn}[Substitution]
\label{dfn:subst}
\begin{eqnarray*}
c \subst{t}{x} &=& c \quad \mbox{where $c$ is a $\Sigma$-constant},\\
x \subst{t}{x} &=& t\\
 y\subst{t}{x} &=& y \quad \mbox{for } x \not \neq y,\\
(M_1 M_2) \subst{t}{x} &=& (M_1 \subst{t}{x}) (M_2 \subst{t}{x})\\
(\lambda x . M) \subst{t}{x} &=& \lambda x . M\\
(\lambda y . M) \subst{t}{x} &=& \lambda z . M \subst{z}{y}
\subst{t}{x} \mbox{where $z$ is a fresh variable and $x\not = y$}.
\end{eqnarray*}
\end{dfn}

In the setting of the safe lambda calculus, the notion of
substitution can be simplified. Indeed, similarly to what we observe
in the homogeneous safe $\lambda$-calculus, we remark that for safe
$\lambda$-terms there is no need to rename variables when performing
substitution:

\begin{lem}[No variable capture lemma]
\label{lem:noclash} There is no variable capture when performing
substitution on a safe term.
\end{lem}

This is the counterpart of lemma \ref{lem:homog_nocapture}. The
proof (which does not rely on homogeneity) is the same.
Consequently, in the safe lambda calculus setting, we can omit to
rename variable when performing substitution. The equation
$$(\lambda x . M) \subst{t}{y} = \lambda z . M \subst{z}{x}
\subst{t}{y} \mbox{where $z$ is a fresh variable}$$ becomes
$$(\lambda x . M) \subst{t}{y} = \lambda x . M \subst{t}{y}.$$

Unfortunately, this notion of substitution is still not adequate for
the purpose of the safe simply-typed lambda calculus. The problem is
that performing a single $\beta$-reduction on a safe term will not
necessarily produce another safe term.

The solution consists in reducing several consecutive $\beta$-redex
at the same time until we obtain a safe term. To achieve this, we
introduce the \emph{simultaneous substitution}, a generalization of
the standard substitution given in definition \ref{dfn:subst}.

\begin{dfn}[Simultaneous substitution]
\label{dnf:simsubst}
 The expression $\subst{\overline{N}}{\overline{x}}$ is an abbreviation for $\subst{N_1 \ldots N_n}{x_1
\ldots x_n}$:
\begin{eqnarray*}
c \subst{\overline{N}}{\overline{x}} &=& c \quad \mbox{where $c$ is a $\Sigma$-constant},\\
x_i \subst{\overline{N}}{\overline{x}} &=& N_i\\
 y \subst{\overline{N}}{\overline{x}} &=& y \quad \mbox{ if } y \not \neq x_i \mbox{ for all } i,\\
(M N) \subst{\overline{N}}{\overline{x}} &=& (M \subst{\overline{N}}{\overline{x}}) (N \subst{\overline{N}}{\overline{x}}) \\
(\lambda x_i . M) \subst{\overline{N}}{\overline{x}} &=& \lambda x_i
. M
\subst{N_1 \ldots N_{i-1} N_{i+1}\ldots N_n}{x_1 \ldots x_{i-1} x_{i+1}\ldots x_n} \\
(\lambda y . M)
\subst{\overline{N}}{\overline{x}} &=& \lambda z . M \subst{z}{y} \subst{\overline{N}}{\overline{x}} \\
&& \mbox{where $z$ is a fresh variables and } y \neq x_i \mbox{ for
all } i.
\end{eqnarray*}
\end{dfn}

In general, variable capture should be avoided, this explains why
the definition of simultaneous substitution uses auxiliary fresh
variables. However in the current setting, lemma \ref{lem:noclash}
can clearly be transposed to the simultaneous substitution,
therefore there is no need to rename variables.

The notion of substitution that we need is therefore the
\emph{capture-permitting simultaneous substitution} defined as
follows:

\begin{dfn}[Capture-permitting simultaneous substitution]
 We use the notation
$\subst{\overline{N}}{\overline{x}}$ for $\subst{N_1 \ldots N_n}{x_1
\ldots x_n}$:
\begin{eqnarray*}
c \subst{\overline{N}}{\overline{x}} &=& c \quad \mbox{where $c$ is a $\Sigma$-constant},\\
 x_i \subst{\overline{N}}{\overline{x}} &=& N_i\\
 y \subst{\overline{N}}{\overline{x}} &=& y \quad \mbox{where } x \not \neq y_i \mbox{ for all } i,\\
(M_1 M_2) \subst{\overline{N}}{\overline{x}} &=& (M_1 \subst{\overline{N}}{\overline{x}}) (M_2 \subst{\overline{N}}{\overline{x}})\\
(\lambda x_i . M) \subst{\overline{N}}{\overline{x}} &=& \lambda x_i
. M
\subst{N_1 \ldots N_{i-1} N_{i+1}\ldots N_n}{x_1 \ldots x_{i-1} x_{i+1}\ldots x_n} \\
(\lambda y . M) \subst{\overline{N}}{\overline{x}} &=& \lambda y . M
\subst{\overline{N}}{\overline{x}} \mbox{where $y \not = x_i$ for
all $i$}. \qquad \mathbf{(\star)}
\end{eqnarray*}
The symbol $\mathbf{(\star)}$ identifies the equation which has
changed compared to the previous definition.
\end{dfn}

\begin{lem}[Substitution preserves safety]
\label{lem:subst_preserve_i}
$$ \Gamma\union \overline{x} : \overline{A}\vdash M : T
\quad \mbox{and} \quad \Gamma \vdash N_k : B_k \mbox{, } k \in
1..n \qquad \mbox{ implies } \qquad \Gamma \vdash
M[\overline{N}/\overline{x}] : T$$
\end{lem}

\begin{proof}
Suppose that $\Gamma \union \overline{x}: \overline{A} \vdash M :T$ and
$\Gamma \vdash N_k : B_k$ for $k \in 1..n$.

We prove $\Gamma \vdash M[\overline{N}/\overline{x}]$ by induction
on the size of the proof tree of $\Gamma\union
\overline{x}:\overline{A} \vdash M : T$ and by case analysis on the
last rule used. We only give the proof for the abstraction case. If
$\Gamma \union \overline{x}:\overline{A} \vdash \lambda \overline{y}
: \overline{C}. P : (\overline{C}|D)$ where $\Gamma\union
\overline{x}:\overline{A}\union \overline{y}:\overline{C} \vdash P :
D$, then by the induction hypothesis $\Gamma\union
\overline{y}:\overline{C} \vdash P\subst{\overline{N}}{\overline{x}}
: D$. Applying the rule $\rulename{abs}$ gives $\Gamma \vdash
\lambda \overline{y}:\overline{C} . P
\subst{\overline{N}}{\overline{x}}$.
\end{proof}

\subsection{Safe-redex}
In the simply-typed lambda calculus a redex is a term of the form
$(\lambda x . M) N$. We generalize this definition to the safe
lambda calculus:
\begin{dfn}[Safe redex]
We call safe redex a term of the form $(\lambda \overline{x} . M)
N_1 \ldots N_l$ such that:
\begin{itemize}
\item $ \Gamma \vdash (\lambda \overline{x} . M) N_1 \ldots N_l $;
\item the variable $\overline{x}=x_1\ldots x_n$ are abstracted altogether by one occurrence of the rule $\rulename{abs}$ in the proof
tree;
\item the terms $(\lambda \overline{x} . M)$, $N_1$, $N_l$ are applied together at once using the $\rulename{app}$ rule :
$$   \rulef{
            \Sigma \vdash \lambda \overline{x} . M
            \quad
            \Sigma \vdash N_1         \quad \ldots \quad \Sigma \vdash N_l
    }
    {
       \Sigma \vdash (\lambda \overline{x} . L) N_1 \ldots N_l
    } (\mathbf{app})
$$
and consequently each $N_i$ is safe;
\end{itemize}
\end{dfn}

The relation $\beta_s$ is defined exactly the same way as in the homogeneous safe $\lambda$-calculus. The safe $\beta$-reduction $\betasred$ is defined as the closure of $\beta_s$ by
compatibility with the formation rules of the safe
$\lambda$-calculus.  It is straightforward to show, as we did for the homogeneous safe $\lambda$-calculus, that $\betasred \subset \betaredtr$.


\begin{lem}
\label{lem:safereduction} A safe redex reduces to a safe term.
\end{lem}

This lemma, which is a consequence of lemma
\ref{lem:subst_preserve_i}, is the counterpart of lemma
\ref{lem:homoh_safered_preserve_safety} in the homogeneous safe
lambda calculus. Their proofs are identical.


\subsection{Particular case of homogeneously-safe lambda terms}

In this section, we derive a new set of rules by adding the type-homogeneity restriction to the non-homogenous safe lambda calculus.

We recall the definition of type-homogeneity from section
\ref{sec:safe_homog}: a type $(A_1, A_2, \ldots A_n, o)$ is said to
be homogeneous whenever $\ord{A_1} \geq \ord{A_2} \geq \ldots \geq
\ord{A_n}$ and each of the $A_i$ is homogeneous. A term is said to
be homogeneous if its type is homogeneous.

We now impose type-homogeneity to all the sequents present in the
rules of the safe $\lambda$-calculus: we say that a term is
\emph{homogeneously-safe} if there is a proof tree showing its
safety in which all sequents are of homogenous type. Consequently a
homogeneously-safe term is safe and has an homogenous type.

We say that $\Gamma \vdash M : A$ verifies $P_i$ for $i \in \zset$
if all the variables in $\Gamma$ have order at least $\ord{A}+i$.
Lemma \ref{lem:nonhomosafe_basic_prop} can then be restated as
follows:
\begin{lem}[Context reduction]
\label{lem:context_reduction} If $\Gamma \vdash M : A$ then the sequent $fv(M) \vdash M : A$ is valid and satisfies $P_0$.
\end{lem}


We now prove that if we impose the homogeneity of types, the set of
rules of the non-homogenous safe $\lambda$-calculus and the rules of
table \ref{tab:homosafelmd_rules_refined} are equivalent.  We recall
that in the system of rules of table
\ref{tab:homosafelmd_rules_refined}, if the sequent $\Gamma
\vdash^{i} M : A$ is valid for some $i \in \zset$ then all the
variables in $\Gamma$ have orders at least $\ord{A}+i$.

\begin{prop}[Homogeneity restriction]
\label{prop:nonhomogsafe_homog_restriction}
Let $k \in \{ 0, -1 \}$. The sequent $\Gamma \vdash M : A$ is valid, homogeneously-safe and satisfies $P_k$
if and only if the sequent $\Gamma \vdash^k M : A$ is valid in the system of rules of table \ref{tab:homosafelmd_rules_refined}.
\end{prop}

\begin{proof}
\emph{If}: An easy induction by case analysis on the last rule used to derive $\Gamma \vdash^0 M : A$.

\emph{Only if}:
Consider an homogeneously-safe term $\Gamma \vdash S : T$ satisfying $P_0$.
We proceed by induction and case analysis on the last rule used to derive $\Gamma \vdash S : T$.
We only give the details for the application and abstraction
case:
\begin{itemize}
\item \textbf{Abstraction.} We recall the abstraction rule:
$$ \rulename{abs} \quad  \rulef{\Gamma \union \overline{x} : \overline{A} \vdash M : B}
                                   {\Gamma  \vdash \lambda \overline{x} : \overline{A} . M : (\overline{A},B)} \qquad
                                   \forall y \in \Gamma : \ord{y} \geq \ord{\overline{A},B}$$

Type homogeneity requires that for all $i$: $\ord{x_i} = \ord{A_i} \geq
\ord{B} -1$. Therefore the premise of the rule verifies $P_{-1}$. Using the induction hypothesis we have:
\begin{equation}
\Gamma \union \overline{x} : \overline{A} \vdash^{-1} M : B. \label{eq:prop:nonhomogsafe_homog_restriction:abs1}
\end{equation}

We now partition the context $\Gamma$ according to the order of
the variables. The partitions are written in decreasing order of
type order. The notation $\Gamma | \overline{x}:\overline{A}$ means
that $\overline{x}:\overline{A}$ is the lowest partition of the
context.
We also use the notation $(\overline{A}|B)$ to denote the
homogeneous type $(A_1, A_2, \ldots A_n, B)$ where $\ord{A_1} =
\ord{A_2} =  \ldots \ord{A_n} \geq \ord{B} -1$.


Suppose that we abstract a single variable $x$, then in order to
respect the side condition, we need to abstract all variables of
order less or equal to $\ord{x}$. In particular we need to abstract
the partition of the order of $x$. Moreover to respect type
homogeneity, we need to abstract variables of the lowest order
first.

Hence $\overline{x}$ must contain at least the lowest variable
partition (all the variables of the lowest order). If $\overline{x}$
contains variables of different order, then the instance of the
abstraction rule can be replaced by consecutive instances of the
abstraction rule, one for each of the different variable order in
$\overline{x}$. Therefore, without loss of generality, we can assume
that $\overline{x}$ only contains the lowest partition, that is to
say, $\overline{x}$ \emph{is} the lowest partition.

The sequent \ref{eq:prop:nonhomogsafe_homog_restriction:abs1} therefore becomes:
$$\Gamma | \overline{x} : \overline{A} \vdash^{-1} M : B.$$

We conclude by applying the abstraction rule of table
\ref{tab:homosafelmd_rules_refined}:
$$ \rulename{abs} \quad  \rulef{\Gamma| \overline{x} : \overline{A} \vdash^{-1} M : B}
                                   {\Gamma  \vdash^{0} \lambda \overline{x} : \overline{A} . M : (\overline{A}|B)}$$



\item \textbf{Application.} We recall the application rule:
$$ \rulename{app} \  \rulef{\Gamma \vdash M : (A,\ldots,A_l,B)
                                        \qquad \Gamma \vdash N_1 : A_1
                                        \quad \ldots \quad \Gamma \vdash N_l : A_l  }
                                   {\Gamma  \vdash M N_1 \ldots N_l : B}
                                    \quad
                                   \forall y \in \Gamma : \ord{y} \geq \ord{B}$$

The term in the conclusion is homogeneously safe therefore the term in the first premise must be of homogeneous \
type. This implies that $\ord{A_1} \geq \ldots \geq \ord{A_l}
\geq \ord{B} - 1$.
Furthermore, we can make the assumption that $\ord{A_1} = \ldots = \ord{A_l} = \ord{\overline{A}}$
(it is always possible to replace an instance of the application rule
by several consecutive instances of this kind).

By lemma \ref{lem:context_reduction}, we have for all $i \in 1..l$:
$$fv(N_i) \vdash N_i : A_i \mbox{ is valid and satisfies } P_0.$$

Let $\Sigma = \Union_{i=1..p} fv(N_i)$. Since $\ord{A_1} = \ldots = \ord{A_l}$, by applying the weakening rule we get for all $i\in 1..p$:
$$\Sigma \vdash N_i : A_i \mbox{ is valid and satisfies } P_0.$$


Applying lemma \ref{lem:context_reduction} to the term $M$ we have:
$$fv(M) \vdash M : (A_1,\ldots,A_l,B) \mbox{ is valid and satisfies } P_0.$$

The weakening rule $\rulename{wk}$ then gives:
$fv(M) \union \Sigma \vdash M : (A_1,\ldots,A_l,B)$.
Since $\Sigma \vdash N_i : A_i$ satisfies $P_0$, for any
$z \in \Sigma$ we have $\ord{z} \geq \ord{A_i} = \ord{(A_1,\ldots,A_l,B)} - 1$.
Hence:
\begin{equation}
fv(M) \union \Sigma \vdash M : (A_1,\ldots,A_l,B) \mbox{ is valid
and satisfies } P_{-1}
\label{eq:prop:nonhomogsafe_homog_restriction:m}.
\end{equation}

Similarly, for all $i \in 1..p$, the weakening rule gives $fv(M) \union \Sigma \vdash N_i : A_i$.
Since $fv(M) \vdash M : (A_1,\ldots,A_l,B)$ satisfies $P_0$,
for any $z \in fv(M)$ we have $\ord{z} \geq \ord{M} \geq \ord{A_i}$. Hence:
\begin{equation}
fv(M) \union \Sigma \vdash N_i : A_i \mbox{ is valid and satisfies }
P_0 \label{eq:prop:nonhomogsafe_homog_restriction:ni}.
\end{equation}

Let us define the context $\Sigma' = fv(M) \union \Sigma$. Using the induction hypothesis on equation
\ref{eq:prop:nonhomogsafe_homog_restriction:m} and \ref{eq:prop:nonhomogsafe_homog_restriction:ni} we have:
$$
\Sigma' \vdash^{-1} M : (A_1,\ldots,A_l,B) \qquad \mbox{and} \qquad
\Sigma' \vdash^0 N_i : A_i \mbox{ for all } i \in 1..l.
$$


We consider the following two sub-cases:
\begin{itemize}
\item If $A_1, \ldots, A_l$ forms a type partition then we can apply
rule $\rulename{app}$ of table \ref{tab:homosafelmd_rules_refined}:

$$ \rulef{\Sigma' \vdash^{-1} M : \overline{A} | B
                                        \qquad \Sigma' \vdash^{0} N_1 :
                                        A_1
                                        \quad \ldots \quad \Sigma' \vdash^{0} N_l :
                                        A_l
                                        \quad l = |\overline{A}|
                                        }
                                   {\Sigma'  \vdash^{0} M N_1 \ldots N_l : B} \quad  \rulename{app}
$$
where $\overline{A} = A_1, \ldots, A_l$.

\item  Suppose that $A_1, \ldots, A_l$ does not form a type partition, then we
have $$\ord{A_1} = \ldots = \ord{A_l} = \ord{B} - 1.$$

The side condition in the original instance of the application rule
says that for any variable $y$ in $\Gamma$ we have
$$\ord{y} \geq \ord{B} = 1 + \ord{A_l} = \ord{(A_1,\ldots, A_l,B)} = \ord{M}.$$

In particular the variables in $\Sigma' \subseteq \Gamma$ are of order greater than $\ord{M}$ and consequently
the sequent $\Sigma' \vdash M : (A,\ldots,A_l,B)$ verifies $P_0$. The induction hypothesis then gives:
$$\Sigma' \vdash^0 M : (A,\ldots,A_l,B)$$

By using $l$ consecutive instances of the rules $\rulename{app^+}$ from table \ref{tab:homosafelmd_rules_refined} we get:
$$  \rulef{ \rulef{ \rulef{ \Sigma' \vdash^0 M : (A_1,\ldots, A_l,B)
                    \qquad \Sigma'\vdash^{0} N_1 : A_1
                    }{ \Sigma' \vdash^0 M N_1 : (A_2,\ldots, A_l,B)} \quad \rulename{app^+}
          }
          { \vdots
          }
          \quad \rulename{app^+}
       }
       { \Sigma'  \vdash^{0} M N_1 \ldots N_l : B } \quad \rulename{app^+}
$$
\end{itemize}

In both cases we have proved that $\Sigma'  \vdash^{0} M N_1 \ldots N_l : B$ is a valid sequent.

Clearly $\Sigma' \subseteq \Gamma$ since $fv(M) \subseteq \Gamma$ and $\Sigma' = \Union_{i\in1..l} fv(N_i) \subseteq \Gamma$.
Suppose that $\Sigma' = \Gamma$ then the proof is done.
Suppose that $\Sigma' \subset \Gamma$, then the side condition in the original instance of the application rule says that all
the variables in $\Gamma$ have order
greater or equal to $\ord{B}$, we can therefore apply the weakening rule $\rulename{wk^0}$
of table \ref{tab:homosafelmd_rules_refined} exactly $|\Gamma\setminus \Sigma'|$ times and get:
$$ \rulef{\Sigma'  \vdash^{0} M N_1 \ldots N_l : B}
                                   {\Gamma  \vdash^{0} M N_1 \ldots N_l : B} \quad
                                   \rulename{wk^0}.
$$


\end{itemize}
\end{proof}


\subsection{Examples}
\subsubsection{Example 1}
Let $f,g:o\rightarrow o$, $x,y:o\rightarrow o$, $\Gamma =
g:o\rightarrow o$ and $\Gamma' = g:o\rightarrow o, y:o$. The term
$(\lambda f x . x) g y $ is safe. One possible proof tree is:
$$ \rulef{
        \rulef{
            \rulef{
                \rulef{\vdots}{\Gamma \vdash \lambda f x. x}      \qquad \axiomf{\Gamma \vdash g} }
            {\Gamma \vdash (\lambda f x. x) g} \rulename{app}
        }
        { \Gamma' \vdash (\lambda f x. x) g } \rulename{wk}
        \qquad \axiomf{\Gamma' \vdash y}
    }
    { \Gamma' \vdash (\lambda f x. x) g y } \rulename{app}
$$
Here is another proof for the same judgment:
$$ \rulef{  \rulef{ \rulef{\vdots}{\Gamma \vdash \lambda f x. x} }{\Gamma' \vdash \lambda f x. x} \rulename{wk}    \qquad \rulef{}{\Gamma' \vdash g} \qquad \rulef{}{\Gamma' \vdash y}}
    {\Gamma' \vdash (\lambda f x. x) g y } \rulename{app}$$

We see on this particular example that there may exist different
proof trees deriving the same judgment.

\subsubsection{Example 2 - Damien Sereni's SCT counter-example}
In \cite{serenistypesct05}, the following counter-example is given
to show that not all simply-typed terms are size-change terminating
(see \cite{jones01} for a definition of size-change termination):

$$ E =  (\lambda a . a (\lambda b . a (\lambda c d .d))) (\lambda e . e (\lambda f .f))$$
where:
\begin{eqnarray*}
a &:& \sigma \typear \mu \typear \mu \\
b &:& \tau \typear \tau \\
c &:& \tau \typear \tau \\
d &:& \mu \\
e &:& \sigma = (\tau \typear \tau) \typear \mu \typear \mu \\
f &:& \tau
\end{eqnarray*}
and $\tau$, $\mu$ and $\sigma$ are type variables.

This example shows that the rules of the safe $\lambda$-calculus
without the homogeneity restriction generates a class of terms that
strictly contains the class of terms generated by the rules of the
homogeneous safe $\lambda$-calculus of section \ref{sec:safe_homog}.

Indeed, for $E$ to be an homogeneous safe lambda term, in the sense
of the rules of section \ref{sec:safe_homog}, $\tau$ and $\mu$ must
be homogeneous types and the variables $a,b,c,d,e,f$ must be
homogeneously typed. This implies that $ \ord{\tau} \geq
\ord{\mu}-1$. Conversely, if this condition is met then $\vdash E :
\mu \typear \mu$ is a valid judgement of the \emph{homogeneous} safe $\lambda$-calculus.

In the safe $\lambda$-calculus \emph{without} the homogeneity
constraint, however, the judgement $\vdash E : \mu \typear \mu$ is
always valid whatever the types $\mu$ and $\tau$ are.


%%%%%%%%%%%%%%%%%%%%%%%%%%%%%%%%%%%%%%%%%%%%%%%%%%%%%%%%%%%%

% fourth chapter
\chapter{Game-semantic characterisation of safety}

Safety has been defined as a syntactical constraint. Since Game
Semantics is by essence syntax-independent, it seems difficult at
first sight to characterise Safety in a game-semantic manner.
However, with the help of the tools developed in the previous
chapter and using the Correspondence Theorem, we can interpret plays
of a strategy as sequences of nodes of some AST of the term.
Therefore it is now possible to investigate the impact of the Safety
restriction on Game Semantics.


The main theorem of this chapter (theorem
\ref{thm:safe_ptr_recoverable}) states that pointers in a play of
the strategy denotation of a safe term can be uniquely recovered
from O-questions' pointers and from the underlying sequence of moves. The proof is in several steps. We start by introducing the notion of
\emph{P-incrementally-justified strategies} and prove that for plays
of such strategies, pointers emanating from P-moves can be reconstructed uniquely from the underlying sequences of moves and from O-moves' pointers. We then introduce the notion of \emph{incrementally-bound computation trees} and prove that incremental-binding coincides with P-incremental-justification (proposition \ref{prop:incrbound_imp_incrjustified}). Finally, we show that safe simply-typed terms in $\beta$-normal form have incrementally-bound computation trees, consequently their game denotation is P-incrementally-justified.


The first section of this chapter is concerned only with the safe $\lambda$-calculus without interpreted constants. In the next
section we extend the result by taking into account the interpreted
constants of \pcf\ and \ialgol. We define the language safe \ialgol\
(resp. safe \pcf) to be the fragment of \ialgol\ (resp. \pcf) where
the application and abstraction rules are constrained the same way
as in the safe $\lambda$-calculus. We show that safe \pcf\ terms are
denoted by P-incrementally-justified strategies and we give the key
elements for a possible extension of the result to Safe Idealized
Algol.

\section{Safe $\lambda$-Calculus}
Let us consider the safe $\lambda$-calculus without interpreted
constants. Our aim is to prove that pointers in the game semantics
of safe terms can be uniquely recovered.

The example of section \ref{subsec:pointer_necessary} gives a good
intuition: in order to distinguish the terms $M_1 = \lambda f . f
(\lambda x . f (\lambda y .y ))$ and $M_2 = \lambda f . f (\lambda x
. f (\lambda y .x ))$ we have to keep the pointers in the plays of
strategies. However, if we limit ourselves to the safe
$\lambda$-Calculus then the ambiguity disappears because $M_1$ is
safe whereas $M_2$ is not (in the subterm $f (\lambda y . x)$, the
free variable $x$ has the same order as $y$ but $x$ is not
abstracted together with $y$).

\begin{dfn}[P-incremental-justification]
A strategy $\sigma$ on a game $A$ is
\emph{P-incrementally\-justified} if and only if for any sequence of
moves $s q \in P_A$ we have:
\begin{eqnarray*}
%s q \in \sigma \wedge q \mbox{ is a O-question } &\implies& \parbox[t]{9cm}{$q$ points to the last P-move in $\oview{?(s)}$ with order strictly greater than $\ord{q}$;} \\
s q \in \sigma \wedge q \mbox{ is a P-question } &\implies&
\parbox[t]{9cm}{$q$  points to the last O-move in $\pview{?(s)}$
with order strictly greater than $\ord{q}$.}
\end{eqnarray*}
\end{dfn}

\begin{lem}
\label{lem:incrjustified_pointers_uniqu_recover} Pointers emanating from P-moves are
superfluous for P-incrementally-justified strategies.
\end{lem}
\begin{proof}
Suppose $\sigma$ is a P-incrementally-justified strategy. We prove that pointers attached to P-moves in a play $s\in \sigma$ are uniquely recoverable by induction on the length of $s$. \noindent \emph{Base case}: if $|s| \leq 1$ then there is no pointer to recover.
\noindent \emph{Step case}: suppose $s m \in \sigma$. If $m$ is an answer move then by the well-bracketing condition $m$ points
to the last unanswered question in $s$. If $m$ is a P-question then by  P-incremental-justification of $\sigma$, $m$ points to the last O-move in
$\pview{?(s)}$ with order strictly greater than $\ord{q}$. Since we have access to O-moves' pointers, we can compute the P-view $\pview{?(s)}$.
Hence $m$'s pointer is uniquely recoverable.
\end{proof}

\begin{exmp}
\label{examp:evnotincrjust}
The denotation of the evaluation map $ev$ is not
P-incrementally-justified. Indeed consider the play $s = q_0 q_1 q_2
q_3 \in \sem{ev}$ shown on the diagram below:
$$\begin{array}{cccccccc}
(A & \implies & B) & \times  & A & \stackrel{ev}{\longrightarrow} & B \\
&&&&&& \rnode{q0}{q_0} \\
&& \rnode{q1}{q_1} \\
 \rnode{q2}{q_2} \\
 &&&&\rnode{q3}{q_3}
\end{array}
\ncline[nodesep=3pt,linewidth=0.5pt]{->}{q3}{q0}
\ncline[nodesep=3pt,linewidth=0.5pt]{->}{q1}{q0}
\ncline[nodesep=3pt,linewidth=0.5pt]{->}{q2}{q1}
$$
The order of the moves are as follows:  $\ord{q_3} = \ord{A}$,
$\ord{q_2} = \ord{A}$, $\ord{q_1} = \max( 1+\ord{A}, \ord{B})$ and
$\ord{q_0} = 1 + \ord{q_1}$. The last O-move in $?(\pview{s})= s$
with order strictly greater than $\ord{q_3}$ is $q_1$.
 But since $q_3$ points to $q_0$, $\sem{ev}$ is not P-incrementally-justified.
\end{exmp}


In a computation tree, a binder node always occurs in the path from the bound node to the root. We now introduce a class of computation trees in which binder nodes can be uniquely recovered from the order
of the nodes. We write $[n_1,n_2]$ to denote the path from node
$n_1$ to node $n_2$ if it exists and $]n_1,n_2]$ for the sequence of
nodes obtained by removing $n_1$ from $[n_1,n_2]$.

\begin{dfn}[Incrementally-bound computation tree]
A variable node $x$ of a computation tree is said to be
\emph{incrementally-bound} if either:
\begin{enumerate}
\item $x$ is \emph{bound} by the first $\lambda$-node in the path to the root that has
order strictly greater than $\ord{x}$. Formally:
$$ x \mbox{ bound by } n \quad \imp \quad n \in [r,x] \wedge \ord{n} > \ord{x} \wedge \forall \lambda\mbox{-node } n' \in ]n,x] . \ord{n'} \leq \ord{x},$$

\item $x$ is a \emph{free variable} and all the $\lambda$-nodes in the path to the root except the root have order
smaller or equal to $\ord{x}$. Formally:
$$ x \mbox{ free } \quad \imp \quad  \forall \lambda\mbox{-node } n' \in ]r,x] . \ord{n'} \leq \ord{x}$$
\end{enumerate}
where $r$ denotes the root of the computation tree.

A computation tree is said to be \emph{incrementally-bound} if all
the variable nodes are incrementally-bound.
\end{dfn}

\begin{prop}[Incremental-binding coincides with P-incremental-justification] \
\label{prop:incrbound_imp_incrjustified} Let $M$ be a $\beta$-normal
term.
\begin{enumerate}
\item[(i)] If $\tau(M)$ is incrementally-bound then $\sem{M}$ is P-incrementally-justified.
\item[(ii)] In the $\lambda$-calculus without interpreted constants, conversely, if $\sem{M}$ is P-incrementally-justified then
$\tau(M)$ is incrementally-bound.
\end{enumerate}
\end{prop}

\begin{proof}
Let $\Gamma \vdash M : A$ be a simply-typed term in $\beta$-normal
form and $r$ denotes the root of $\tau(M)$.

\noindent (i) Suppose that $\tau(M)$ is incrementally-bound.
Consider a justified sequence of move $s \in \sem{\Gamma \vdash M}$
ending with a P-question move $q$ (note that $q$ is also the last
question in $?(s)$). By proposition \ref{prop:rel_gamesem_trav},
there is a traversal $t$ of $\tau(M)$ such that $\varphi_{M}(t
\upharpoonright r) = s$. We assume that the last node $n$ of $t$ is
hereditarily justified by $r$ (otherwise we replace $t$ by its
longest prefix verifying this condition). Then $n$ is also the last
node in $?(t \upharpoonright r)$ and $t \upharpoonright r$.
\begin{itemize}
\item First case: $n$ is a variable node $x$ bound by a node $m$ occurring in $t$.

Since $\tau(M)$ is incrementally-bound,
$m$ is the last $\lambda$-node in $[r,n]$ of order strictly greater
than $\ord{n}$. By visibility, $m$ occurs in $\pview{?(t)}$ and
since $m$ is hereditarily justified by $r$ (because $n$ is) 
$m$ must occur in $\pview{?(t)} \upharpoonright r$ which in turn
is equal to $\pview{?(t) \upharpoonright r}$ (by lemma \ref{lem:redtrav_trav}(i), since $M$ is in $\beta$-normal form).

But $\pview{?(t) \upharpoonright r} = \pview{?(t \upharpoonright r)}$ is a subsequence of
$\pview{?(t)}$ which is equal to $[r,n]$ (by proposition \ref{prop:pviewtrav_is_path}), therefore $m$ is also the last $\lambda$-node in
$\pview{?(t \upharpoonright  r)}$ that has order strictly greater
than $\ord{n}$.

By property \ref{proper:phi_pview} (ii), the P-view of $?(s)$ and
the P-view of $?(t \upharpoonright r)$ are computed similarly and
have the same pointers. This means that node $n$ and  move $q$ both
point to the same position in the justified sequence
$\pview{?(t\upharpoonright r)}$ and $\pview{?(s)}$ respectively.

Finally, since $\varphi$ maps nodes of a given order to moves of the
same order (property \ref{proper:phi_conserve_order}), $q$ must
point to the last O-move in $\pview{?(s)}$ whose order is strictly
greater than $\ord{q}$.


\item  Second case: $n$ is a free variable node $x$. Then $n$ is enabled by the root which is the first node in $t$.
By definition of $\varphi$, $\varphi(n) = x$ must be a move enabled
by the initial move $q_0 = \varphi(r)$ in the arena $\sem{\Gamma
\rightarrow A}$. Therefore $\ord{q_0} > \ord{x}$. Since the
computation tree is incrementally-bound, all the $\lambda$-nodes in
$]r,n]$ have order smaller than $\ord{n}$. Therefore by the
correspondence theorem, all the O-moves in $\pview{?(s)}$ have order
smaller than $\ord{x}$.
\end{itemize}



%\item If $|s|$ is odd then $q$ is an O-move:
%
%$M$ is in $\beta$-normal form and $t$ is a traversal of $\tau(M)$
%whose last node $n$ is hereditarily justified by $r$. Therefore by
%lemma \ref{lem:redtrav_trav} (ii), $ \oview{?(t \upharpoonright r)}
%= \oview{?(t)}$.
%
%A lambda-node always points to its parent node in the computation
%tree. For terms in $\beta$-normal form, this parent node must be a
%variable node of order strictly greater than $\ord{n}$.
%
%By inspecting the formation rules for traversals (definition
%\ref{def:traversal}) we remark that a lambda-node occurring in a
%traversal always points to the last node with order strictly greater
%that $\ord{n}$ in the O-view of the sequence of unmatched nodes at
%that point (there are just two cases, $n$ points either to the
%preceding node or to the third previous node in $\oview{?(t)}$).
%
%Similarly, as in the P-move case, we conclude that $q$ points to the
%last question move in $\oview{?(s)}$ of order strictly greater than
%$\ord{q}$.
%\end{itemize}

\noindent (ii) Suppose that $M$ is $\beta$-normal and the strategy
$\sem{M}$ is P-incrementally-justified. Let $x$ be a variable node of
$\tau(M)$. Since $M$ is $\beta$-normal, by lemma
\ref{lem:betaeta_trav}, $x$ is either hereditarily justified by the root $r$ or by a constant in $N_\Sigma$. Since we are working in the simply-typed
$\lambda$-calculus without constants we have $N_\Sigma = \emptyset$ therefore $x$ is
hereditarily justified by $r$.


We remark that for terms in $\beta$-normal form, every variable node
occurring in the computation tree can be visited by some traversal
i.e. there exists a traversal of the form $t \cdot x$ in
$\travset(M)$. The correspondence theorem gives $\varphi((t \cdot x)
\upharpoonright r) = \varphi((t \upharpoonright r) \cdot x) \in
\sem{M}$. Since $\sem{M}$ is P-incrementally-justified, $\varphi(x)$
must point to the last O-move in $\pview{?(\varphi(t \upharpoonright
r))}$ with order strictly greater than $\ord{\varphi(x)}$.
Consequently $x$ points to the last $\lambda$-node in $\pview{?(t
\upharpoonright r)}$ with order strictly greater than $\ord{x}$. Moreover we
have:
\begin{align*}
\pview{?(t \upharpoonright r)} &= \pview{?(t) \upharpoonright r} = \pview{?(t)} \upharpoonright r & (\mbox{by lemma \ref{lem:redtrav_trav}}) \\
& = [r,x[\  \upharpoonright r & (\mbox{by proposition \ref{prop:pviewtrav_is_path}}) \\
& = [r,x[  & (\mbox{$M$ is in $\beta$-nf and $N_\Sigma = \emptyset$}). 
\end{align*}
Therefore if $x$ is a bound variable node then it is bound by the
last $\lambda$-node in $[r,x[$ with order strictly greater than
$\ord{x}$ and if $x$ is a free variable then it points to $r$ and
therefore all the $\lambda$-node in $]r,x[$ have order smaller than
$\ord{x}$. Hence $\tau(M)$ is incrementally-bound.
\end{proof}


\parpic[r]{
    \psset{levelsep=4ex}
    \pstree{\TR{$\lambda x^3$}}{\pstree{\TR{$f^2$}}{ \pstree{\TR{$\lambda y^1$}}{ \TR{$x^0$} }}}
}

\noindent \emph{Examples:} Consider the $\beta$-normal term $\lambda
x . f (\lambda y .x)$ where $x,y:o$ and $f:(o,o),o$. The figure on
the right represents the computation tree with the order of each
node in the exponent part. Since node $x$ of order $0$ is not bound
by the order 1 node $\lambda y$, $\tau(M)$ is not
incrementally-bound and by proposition
\ref{prop:incrbound_imp_incrjustified} $\sem{\lambda x . f (\lambda
y .x)}$ is not P-incrementally-justified. Similarly we can check that
$\sem{f (\lambda y .x)}$ is not P-incrementally-justified
whereas $\sem{\lambda y. x}$ is.
Also for any higher-order variable $x:A$, the computation tree
$\tau(x)$ is incrementally-bound, therefore the projection
strategies $\pi_i$ are P-incrementally-justified. From these examples
we observe that application does not preserve
P-incremental-justification ($\sem{f}$ and $\sem{\lambda y. x}$ are
P-incrementally-justified whereas $\sem{f (\lambda y .x)}$ is not).

These examples suggest that P-incremental-justification is not a compositional property. Since the
 evaluation map $ev$ is not P-incrementally-justified (see example \ref{examp:evnotincrjust}),
application cannot conserve P-incremental-justification in the general case. One interesting problem would be to find some condition under which  the composition of two P-incrementally-justified strategy remains
P-incrementally-justified.
Unfortunately we have not provided an answer to that question yet.


\begin{lem}[Safe terms have incrementally-bound computation trees]
\label{lem:safe_imp_incrbound} Let $\Gamma \vdash M$ be a
simply-typed term.
\begin{itemize}
\item[(i)] If $M$ is a safe term then $\tau(M)$ is incrementally-bound ;
\item[(ii)] conversely, if $M$ is \emph{closed} and $\tau(M)$ is incrementally-bound then the $\eta$-normal form of $M$ is safe.
\end{itemize}
\end{lem}
\begin{proof}
(i) Suppose that $M$ is safe. The safety property is preserved after
taking the $\eta$-long normal form, therefore $\tau(M)$ is the tree representation of a safe term.

In the safe $\lambda$-calculus, the variables in the context with the the lowest order must be all abstracted
at once when using the abstraction rule. Since the computation
tree merges consecutive abstractions into a single node,
any variable $x$ occurring free in the subtree rooted at a $\lambda$-node $\lambda \overline{\xi}$ different from the root
must have order greater or equal to $\ord{\lambda \overline{\xi}}$. Reciprocally, if a lambda node
$\lambda \overline{\xi}$ binds a variable node $x$ then
$\ord{\lambda \overline{\xi}} = 1+\max_{z\in\overline{\xi}} \ord{z} > \ord{x}$.

Let $x$ be a bound variable node. Its binder occurs in the path from $x$
to the root, therefore, according to the previous observation, $x$ must be bound
by the first $\lambda$-node occurring in $[r,x]$ with order strictly
greater than $\ord{x}$. Let $x$ be a free variable node then $x$ is not bound
by any of the $\lambda$-nodes occurring in $[r,x]$. Once again, by the previous observation, all
these $\lambda$-nodes except $r$ have order smaller than $\ord{x}$. Hence
$\tau$ is incrementally-bound.

(ii) Let $M$ be a closed term such that $\tau(M)$ is incrementally-bound.
We assume that $M$ is already in $\eta$-normal form.
We prove that $M$ is safe by induction on its structure. The base case $M =
\lambda \overline{\xi} . \alpha$ for some variable or constant
$\alpha$ is trivial.
\emph{Step case:} If $M = \lambda \overline{\xi} . N_1 \ldots N_p$.
Let $i$ range over $1..p$. $N_i$ can be written $\lambda
\overline{\eta_i} . N'_i$ where $N'_i$ is not an abstraction. By the
induction hypothesis, $\lambda \overline{\xi} . N_i = \lambda
\overline{\xi} \overline{\eta_i} . N'_i$ is safe.
Hence $\vdash \lambda \overline{\xi} \overline{\eta_i} . N'_i$
is a valid judgment of safe $\lambda$-calculus.
But this judgment can only be derived using the (abs) rule on the term $N'_i$. Hence
$N'_i$ is necessarily safe. Let $z$ be a variable occurring free in
$N'_i$. Since $M$ is closed, $z$ is either bound by $\lambda
\overline{\eta_1}$ or $\lambda \overline{\xi}$. If it is bound by
$\lambda \overline{\xi}$ then because $\tau(M)$ is
incrementally-bound we have $\ord{z} \geq \ord{\lambda
\overline{\eta_1}} = \ord{N_i}$. Hence in both case we can abstract the variables
$\overline{\eta_1}$ using the (abs) rule which shows that $N_i$ is safe.

Each of the $N_i$s are safe and $N_1 \ldots N_p$ is of type $o$ therefore
by the (app) rule we have $\overline{\xi} \vdash N_1 \ldots N_p$. Finally, using the (abs) rule we conclude
with the judgement $\vdash M = \lambda \overline{\xi} . N_1 \ldots N_p$.
\end{proof}

Note that the hypothesis that $M$ is closed in (ii) is necessary.
For instance, the two terms $\lambda x y .x$ and $\lambda y . x$,
where $x,y:o$, have (isomorphic) incrementally-bound computation
trees. However $\lambda x y .x$ is safe whereas $\lambda y . x$ is
not.



Putting proposition \ref{prop:incrbound_imp_incrjustified} and lemma
\ref{lem:safe_imp_incrbound} together we obtain a game-semantic
characterisation of safe terms:
\begin{cor}[P-incrementally-justified strategies characterise safe closed $\eta\beta$-normal terms]
Let $M$ be a simply-typed term without interpreted constants. We have:
$$ \sem{M} \mbox{ is P-incrementally-justified if and only if $\etabetanf{M}$ is safe,} $$
where $\etabetanf{M}$ denotes the $\eta$-normal form of the
$\beta$-normal form of $M$.
\end{cor}



\begin{thm}[P's pointers are superfluous for safe terms]
\label{thm:safe_ptr_recoverable} Pointers emanating from P-moves in the game semantics of
safe terms are uniquely recoverable.
\end{thm}
\begin{proof}
Let $M$ be a safe simply-typed term. The $\beta$-normal form of $M$
denoted by $M'$ is also safe. By lemma \ref{lem:safe_imp_incrbound}
(i), $\tau(M')$ is incrementally-bound and by proposition
\ref{prop:incrbound_imp_incrjustified}, $\sem{M'}$ is a
P-incrementally-justified strategy. By lemma
\ref{lem:incrjustified_pointers_uniqu_recover}, P's pointers in
$\sem{M'}$ are uniquely recoverable. Finally, the soundness of the
game model gives $\sem{M} = \sem{M'}$.
\end{proof}


\section{Safe PCF and Safe Idealized Algol}

Safe Idealized Algol, or safe \ialgol\ for short, is Idealized Algol
where the application and abstraction rules are restricted the same
way as in the safe $\lambda$-calculus (see rules of section
\ref{sec:safe_nonhomog}).

The properties of the safe $\lambda$-calculus can be transposed
straightforwardly to safe \ialgol. In particular, it can be shown
that safety is preserved by $\beta$-reduction and that no variable
capture occurs when performing substitution on a safe term.

A natural question to ask is whether we can extend the result about
game semantics of safe $\lambda$-terms to safe \ialgol-terms. In
this section we lay out the key elements permitting to prove that
the pointers in the game semantics of safe IA terms can be recovered
uniquely.

Such result has potential applications in algorithmic game semantics.
For instance, by following the framework of \cite{ghicamccusker00},
it may be possible to give a characterisation of the game semantics
of some higher-order fragments of safe \ialgol\ using extended
regular expressions. Subsequently, this would lead to the
decidability of program equivalence for the considered fragment.


\subsection{Formation rules of Safe \ialgol}
We call safe \ialgol\ term any term that is typable within the
following system of formation rules:
$$ \rulename{var} \   \rulef{}{x : A\vdash x : A}
%\qquad  \rulename{const} \   \rulef{}{\vdash f : A} \quad f \in \Sigma
\qquad  \rulename{wk} \   \rulef{\Gamma \vdash M : A}{\Delta \vdash
M : A} \quad  \Gamma \subset \Delta$$

$$ \rulename{app} \  \rulef{\Gamma \vdash M : (A,\ldots,A_l,B)
                                        \qquad \Gamma \vdash N_1 : A_1
                                        \quad \ldots \quad \Gamma \vdash N_l : A_l  }
                                   {\Gamma  \vdash M N_1 \ldots N_l : B}
                                    \quad
\mbox{\fbox{$\forall y \in \Gamma : \ord{y} \geq \ord{B}$}}$$

$$ \rulename{abs} \   \rulef{\Gamma \union \overline{x} : \overline{A} \vdash M : B}
                                   {\Gamma  \vdash \lambda \overline{x} : \overline{A} . M : (\overline{A},B)} \quad
\mbox{\fbox{$\forall y \in \Gamma : \ord{y} \geq \ord{\overline{A},B}$}}$$

$$ \rulename{num} \rulef{}{\Gamma \vdash n :\texttt{exp}}
\qquad \rulename{succ} \rulef{\Gamma \vdash M:\texttt{exp} }{\Gamma
\vdash \texttt{succ}\ M:\texttt{exp}} \qquad \rulename{pred}
\rulef{\Gamma \vdash M:\texttt{exp} }{\Gamma \vdash \texttt{pred}\
M:\texttt{exp}}$$

$$
\rulename{cond} \rulef{\Gamma \vdash M : \texttt{exp} \qquad \Gamma
\vdash N_1 : \texttt{exp} \qquad \Gamma \vdash N_2 : \texttt{exp}
}{\Gamma \vdash \texttt{cond}\ M\ N_1\ N_2} \qquad  \rulename{rec}
\rulef{\Gamma \vdash M : A\rightarrow A }{ \Gamma \vdash Y_A M :
A}$$

$$ \rulename{seq} \rulef{\Gamma \vdash M : \texttt{com} \quad \Gamma \vdash N :A}
    {\Gamma \vdash \texttt{seq}_A \ M\ N\ : A} \quad A \in \{ \texttt{com}, \texttt{exp}\}$$

$$ \rulename{assign} \rulef{\Gamma \vdash M : \texttt{var} \quad \Gamma \vdash N : \texttt{exp}}
    {\Gamma \vdash \texttt{assign}\ M\ N\ : \texttt{com}}
\qquad
 \rulename{deref} \rulef{\Gamma \vdash M : \texttt{var}}
    {\Gamma \vdash \texttt{deref}\ M\ : \texttt{exp}}$$

$$ \rulename{new} \rulef{\Gamma, x : \texttt{var} \vdash M : A}
    {\Gamma \vdash \texttt{new } x \texttt{ in } M} \quad A \in \{ \texttt{com}, \texttt{exp}\}$$

$$ \rulename{mkvar} \rulef{\Gamma \vdash M_1 : \texttt{exp} \rightarrow \texttt{com} \quad \Gamma \vdash M_2 : \texttt{exp}}
    {\Gamma \vdash \texttt{mkvar } M_1\ M_2\ : \texttt{var}}$$

\subsection{Small-step semantics of Safe \ialgol}
In the first chapter we defined the operational semantics of
\ialgol\ using a big step semantics. The operational semantics of
\ialgol\ can be defined equivalently using a small-step semantics.
The reduction rules of the small-step semantics are of the form $s,e
\rightarrow s',e'$ where $s$ and $s'$ denotes the stores and $e$ and
$e'$ denotes \ialgol\ expressions.

Let us give the rules that tell how to reduce redexes:
\begin{itemize}
\item the reduction of safe-redex (relation $\beta_s$ from definition \ref{dfn:safereduction});
\item reduction rules for \pcf\ constants:
\begin{eqnarray*}
\pcfsucc\ n &\rightarrow& n+1 \\
\pcfpred\ n+1 &\rightarrow& n \\
\pcfpred\ 0 &\rightarrow& 0 \\
\pcfcond\ 0\ N_1 N_2 &\rightarrow& N_1 \\
\pcfcond\ n+1\ N_1 N_2 &\rightarrow& N_2 \\
Y\ M &\rightarrow& M (Y M)
\end{eqnarray*}
\item reduction rules for \ialgol\ constants:
\begin{eqnarray*}
\iaseq\ \iaskip\  M &\rightarrow& M \\
s, \ianewin{x}\ M &\rightarrow& (s|x\mapsto 0), M \\
s, \iaassign\ x\ n &\rightarrow& (s|x\mapsto n), \iaskip \\
s, \iaderef\ x &\rightarrow& s, s(x) \\
\iaassign\ (\iamkvar M N)\ n &\rightarrow& M n \\
\iaderef\ (\iamkvar M N) &\rightarrow& N
\end{eqnarray*}
\end{itemize}

Redex can also be reduced when they occur as subexpressions within a
larger expression. We make use of evaluation contexts to indicate
when such reduction can happen. Evaluation contexts are given by the
following grammar:
\begin{eqnarray*}
E[-] &::=& - |\ E N\ |\ \pcfsucc\ E\ |\ \pcfpred\ E\ |\ \pcfcond\ E\ N_1\ N_2\ |\ \\
&&    \iaseq\ E\ N\ |\ \iaderef\ E\ |\ \iaassign\ E\ n\ |\ \iaassign\ M\ E \ |\ \\
&&    \iamkvar\ M\ E\ |\ \iamkvar\ E\ M\ |\ \ianewin{x}\ E  .
\end{eqnarray*}

The small-step semantics is completed with following rule:
$$ \rulef{M \rightarrow N}{E[M] \rightarrow E[N]} $$

\begin{lem}[Reduction preserves safety]
\label{lem:ia_safety_preserved} Let $M$ be a safe IA term. If
$M \rightarrow N$ then $N$ is also a safe term.
\end{lem}
This can be proved easily by induction on the structure of M.


\subsection{Safe \pcf\ fragment}
In this section, we show how to extend the results obtained for the
safe $\lambda$-calculus to the \pcf\ fragment of safe \ialgol.

The $Y$ combinator needs a special treatment. In order to deal with
it, we follow the idea of \cite{abramsky:game-semantics-tutorial}:
we consider the sublanguage $\pcf_1$ of \pcf\ in which the only
allowed use of the $Y$ combinator is in terms of the form $Y(
\lambda x:A .x )$ for some type $A$. We will write $\Omega_A$ to
denote the non-terminating term $Y(\lambda x:A .x)$ for a given type
$A$.

We introduce the \emph{syntactic approximants} to $Y_A M$:
\begin{eqnarray*}
Y^0_A M &=& \Gamma \vdash \Omega_A : A\\
Y^{n+1}_A M &=& M( Y^n M )
\end{eqnarray*}
For any \pcf\ term $M$ and natural number $n$, we define $M_n$ to be
the $\pcf_1$ term obtained from $M$ by replacing each subterm of the
form $Y N$ with $Y^n N_n$. We have $\sem{M} = \Union_{n\in\omega}
\sem{M_n}$ (\cite{abramsky:game-semantics-tutorial}, lemma 16).


\subsubsection{Computation tree}

We would like to define a unique computation tree for terms that use
the $Y$ combinator.

Let us first define the computation tree for $\pcf_1$ terms. We
introduce a special $\Sigma$-constant $\bot$ representing the
non-terminating computation of ground type $\Omega_o$. Given any
type $A = (A_1, \ldots, A_n, o)$, the computation tree
$\tau(\Omega_A)$ is defined to be the tree representation of
$\lambda x_1:A_1 \ldots x_n:A_n . \bot$. The computation tree of a
$\pcf_1$ term is then computed inductively in the standard way.

We now introduce a partial order on the set of computation trees.

A \emph{tree} $t$ is a labelling function $t:T\rightarrow L$ where
$T$, called the domain of $t$ and written $dom(t)$, is a non-empty
prefix-closed subset of some free monoid $X^*$ and $L$ denotes the
set of possible labels. Intuitively, $T$ represents the structure of
the tree (the set of all paths) and $t$ is the labelling function
mapping paths to labels. Trees can be ordered using the
\emph{approximation ordering} defined in \cite{KNU02}, section 1: we
write $t' \sqsubseteq t$ if the tree $t'$ is obtained from $t$ by
replacing some of its subtrees by $\bot$. Formally:
$$t' \sqsubseteq t \quad \iff dom(t') \subseteq dom(t) \wedge \forall  w \in dom(t'). (t'(w) = t(w) \vee t'(w) = \bot).$$
The set of all trees together with the approximation ordering is a
complete partial order.

We now consider a strict subset of the set of all trees: the set of
computation trees. A computation tree is a tree which represents the
$\eta$-normal form of some (potentially infinite) \pcf\ term. In
other words a tree is a computation tree if it can be written
$\tau(M)$ for some infinite \pcf\ term $M$. The set $L$ of labels is
constituted of the $\Sigma$-constants, @, the special constant
$\bot$, variables and abstractions of any sequence of variables. We
will write $(CT, \sqsubseteq)$ to denote the set of computation
trees ordered by the approximation ordering $\sqsubseteq$ defined
above. $(CT, \sqsubseteq)$ is also a complete partial order.

It is easy to check that the sequence of computation trees
$(\tau(M_n))_{n\in\omega}$ is a chain. We can therefore define the
computation tree of a \pcf\ term $M$ to be the least upper-bound of
the chain of computation trees of its approximants:
$$\tau(M) = \Union_{n\in\omega}(\tau(M_n))_{n\in\omega}.$$

In other words, we construct the computation tree by expanding
infinitely any subterm of the form $Y M$. For instance consider the
term $M = Y (\lambda f x. f x)$ where $f:(o,o)$ and $x:o$. Its
computation tree $\tau(M)$, represented below, is a tree
representation of the $\eta$-normal form of the infinite term
$(\lambda f x. f x) ((\lambda f x. f x) ((\lambda f x. f x)  (
\ldots$.
$$\tau(M) = \tree{\lambda y}{
                \tree{@}{
                        \tree{\lambda f x} { \tree{f}{\tree{\lambda}{\TR{x}} }}
                        \TR{\tau(M)}
                        \tree{\lambda}{\TR{y}}
                }
            }
$$

The remaining operators of \ialgol\ are treated as standard
constants and the corresponding computation tree is constructed from
the $\eta$-normal form of the term in the standard way. For instance
the diagram below shows the computation tree for $\pcfcond\ b\ x\ y$
(left) and $\lambda x . 5$ (right):
$$
\tree{\lambda b x y}
     {  \tree{\pcfcond}
        {   \tree{\lambda} {\TR{b}}
            \tree{\lambda} {\TR{x}}
            \tree{\lambda} {\TR{y}}
        }
    }
\hspace{2cm} \tree{\lambda x}{  \TR{5} }
$$
The node labelled $5$ has, like any other node, children
value-leaves which are not represented on the diagram above for
simplicity.

\subsubsection{Traversal}

New traversal rules accompany the additional constants of \ialgol.
There is one additional rule for natural number constants:
\begin{itemize}
\item (Nat) If $t \cdot n$ is a traversal where $n$ denotes a node labelled with some numeral constant $i\in \nat$ then
            $t \cdot \rnode{n}{n} \cdot \rnode{in}{i_n} \link[nodesep=0pt]{40}{in}{n}$
            is also a traversal where $i_n$ denotes the value-leaf of $m$ corresponding to the value $i\in \nat$.
\end{itemize}

\noindent The traversals rules for \pcfpred\ and \pcfsucc\ are
defined similarly. For instance, the rules for \pcfsucc\ are:
\begin{itemize}
\item (Succ) If $t \cdot \pcfsucc$ is a traversal and $\lambda$ denotes the only child node of \pcfsucc\ then
$t \cdot \rnode{succ}{\pcfsucc} \cdot \rnode{l}{\lambda}
\link[nodesep=1pt]{60}{l}{succ} \lnklabel{1}$ is also a traversal.

\item (Succ') If
$t_1 \cdot \rnode{succ}{\pcfsucc} \cdot \rnode{l}{\lambda} \cdot t_2
\cdot \rnode{lv}{i_{\lambda}} \link[nodesep=1pt]{60}{l}{succ}
\lnklabel{1} \link[nodesep=1pt]{40}{lv}{l}$ is a traversal for some
$i \in \nat$ then $t_1 \cdot \rnode{succ}{\pcfsucc} \cdot
\rnode{l}{\lambda} \cdot t_2 \cdot \rnode{lv}{i_{\lambda}} \cdot
\rnode{succv}{(i+1)_{\pcfsucc}} \link[nodesep=1pt]{60}{l}{succ}
\lnklabel{1} \link[nodesep=1pt]{25}{succv}{succ}
\link[nodesep=1pt]{40}{lv}{l} $ is also a traversal.
\end{itemize}

\noindent In the computation tree, nodes labelled with \pcfcond\
have three children nodes numbered from $1$ to $3$ corresponding to
the three parameters of the operator \pcfcond. The traversal rules
are:
\begin{itemize}
\item (Cond-If) If $t_1 \cdot \pcfcond$ is a traversal and $\lambda$ denotes the first child of \pcfcond\ then
$t_1 \cdot \rnode{cond}{\pcfcond} \cdot \rnode{l}{\lambda}
\link[nodesep=1pt]{60}{l}{cond} \lnklabel{1}$ is also a traversal.

\item (Cond-ThenElse) If
$t_1 \cdot \rnode{cond}{\pcfcond} \cdot \rnode{l}{\lambda} \cdot t_2
\cdot \rnode{lv}{i_{\lambda}} \link[nodesep=1pt]{60}{l}{cond}
\lnklabel{1} \link[nodesep=1pt]{40}{lv}{l}$ then $t_1 \cdot
\rnode{cond}{\pcfcond} \cdot \rnode{l}{\lambda} \cdot t_2 \cdot
\rnode{lv}{i_{\lambda}} \cdot \rnode{condthenelse}{\lambda}
\link[nodesep=1pt]{60}{l}{cond} \lnklabel{1}
\link[nodesep=1pt]{40}{lv}{l}
\link[nodesep=1pt]{35}{condthenelse}{cond} \lnklabelc{2+[i>0]} $ is
also a traversal.



\item (Cond') If
$t_1 \cdot \rnode{cond}{\pcfcond} \cdot t_2 \cdot \rnode{l}{\lambda}
\cdot t_3 \cdot \rnode{lv}{i_{\lambda}}
\link[nodesep=1pt]{40}{l}{cond} \lnklabel{k}
\link[nodesep=1pt]{40}{lv}{l}$ for $k=2$ or $k=3$ then $t_1 \cdot
\rnode{cond}{\pcfcond} \cdot t_2 \cdot \rnode{l}{\lambda} \cdot t_3
\cdot \rnode{lv}{i_{\lambda}} \cdot \rnode{condv}{i_{\pcfcond}}
\link[nodesep=1pt]{40}{l}{cond} \lnklabel{k}
\link[nodesep=1pt]{40}{lv}{l} \link[nodesep=1pt]{20}{condv}{cond}
$ is also a traversal.
\end{itemize}
It is easy to verify that these traversal rules are all well-behaved
and therefore condition (WB) of section \ref{subsec:traversal} is
met. This completes the definition of traversal for the \pcf\ subset
of \ialgol.

\subsubsection{Interaction semantics}
We recall that the interaction semantics defined in section
\ref{sec:interaction_semantics} takes into account the constants
of the language. For any higher-order constant $f : (A_1,\ldots,A_p,B) \in \Sigma$, definition \ref{dfn:interactionstrategy_ofterms} gives the  revealed strategy of a term of the form $\lambda \overline{\xi}. f N_1 \ldots
N_p$ as follows:
$$ \intersem{\lambda \overline{\xi}. f N_1 \ldots N_p} = \langle \intersem{N_1}, \ldots, \intersem{N_p} \rangle \fatsemi^{0..p-1} \sem{f}.$$
where $\sem{f}$ is the standard strategy denotation of the constant $f$.


\subsubsection{Removing $\Sigma$-nodes from the traversals}

To establish the correspondence with the interaction semantics, we
need to remove the superfluous nodes from the traversals. These
nodes are the @-nodes and the constant nodes. We will use the
operation $-@$ (definition \ref{dfn:appnode_filter}) to filter out
the @-nodes and we introduce a similar operation $-\Sigma$ to
eliminate the $\Sigma$-nodes.

\begin{dfn}[Hiding $\Sigma$-constants in the traversals]
Let $t$ be a traversal of $\tau(M)$. We write $t-\Sigma$ for the
sequence of nodes with pointers obtained by
\begin{itemize}
\item removing from $t$ all nodes labelled with a $\Sigma$-constant or value-leaf justified by a $\Sigma$-constant,
\item replacing any link pointing to a $\Sigma$-constant $f$
by a link pointing to the predecessor of $f$ in $t$.
\end{itemize}

Suppose $u = t-\Sigma$ is a sequence of nodes obtained by applying
the previously defined transformation on the traversal $t$, then $t$
can be partially recovered from $u$ by reinserting the
$\Sigma$-nodes as follows. For each $\Sigma$-node $f$, where $p$
denotes the parent node of $f$, do the following:
    \begin{enumerate}
    \item replace every occurrence of the pattern $p \cdot n$ in $u$ where
    $n$ is a $\lambda$-node by $p \cdot f \cdot n$;

    \item replace any link in $u$ starting from a $\lambda$-node and pointing to $p$ by a link pointing to the inserted node $f$;

    \item for each occurrence in $u$ of a value-leaf $v_p$ pointing to $p$, add the value-leaf $v_f$
    immediately before $v_p$. The links of $v_f$ points to the node immediately following $p$.
    \end{enumerate}
We write $u+\Sigma$ for this second transformation.
\end{dfn}
These transformations are well-defined since in a traversal, a
$\Sigma$-node $f$ always follows immediately its parent
$\lambda$-node $p$, and an occurrence of a value-node $v_p$ always
follows immediately a value-node $v_f$. In other words, if $f$
occurs in $t$ then $t$ must be a prefix of a traversal of the
following form for some $v \in \mathcal{D}$:
$$ \ldots \cdot \rnode{p}{p} \cdot \rnode{f}{f} \cdot \ldots \cdot \rnode{vf}{v_f} \cdot \rnode{vp}{v_p} \cdot \ldots
\link[offset=-4pt]{20}{vf}{f} \link[offset=-4pt]{20}{vp}{p}
$$

Remark: $t-\Sigma$ is not a proper traversal since it does not
satisfy alternation. It is not a proper justified sequence either
since after removing a $\Sigma$-node $f$, any $\lambda$-node
justified by $f$ will become justified by the parent of $f$ which is
also a $\lambda$-node.

The following lemma follows directly from the definition:
\begin{lem}
\label{lem:minus_sig_plus_sig} For any traversal $t$ we have
$(t-\Sigma)+\Sigma \sqsubseteq t$ and if $t$ does not end with an
$\Sigma$-node or a value-leaf of a $\Sigma$-node then
$(t-\Sigma)+\Sigma = t$.
\end{lem}

The operations $-@$ and $-\Sigma$ are commutative: $(t-@)-\Sigma =
(t-\Sigma)-@$. We write $t^*$ to denote $(t-@)-\Sigma$ i.e. the
sequence obtained from $t$ by removing all the @-nodes as well as
the constant nodes together with their associated value-leaves. We
introduce the notation $\travset(M)^{*} = \{ t^* \ | \  t \in
\travset(M) \}$.

\begin{lem}[Filtering lemma]
\label{lem:SIGMACONST:varphi_filter} Let $\Gamma \vdash M :T$ be a
term and $r$ be the root of $\tau(M)$. For any traversal $t$ of the
computation tree we have $ \varphi(\travset^*(M)) \upharpoonright
\sem{\Gamma \rightarrow T} = \varphi(\travset^{\upharpoonright
r}(M)) $.
 Consequently,
$$\varphi(t^*) \upharpoonright \sem{\Gamma \rightarrow T} = \varphi(t\upharpoonright r).$$
\end{lem}
\begin{proof}
    From the definition of $\varphi$, the nodes of the computation tree that $\varphi$ maps
    to moves in the arena $\sem{\Gamma \rightarrow T}$ are exactly the nodes that are hereditarily justified by $r$.
    The result follows from the fact that @-nodes, constant nodes and value-leaves of constant nodes
    are not hereditarily justified by the root.
\end{proof}

The following lemma is the counterpart of lemma
\ref{lem:varphiinjective} and it is proved identically.
\begin{lem}[$\varphi$ is injective]
\label{lem:SIGMACONST:varphiinjective} $\varphi$ regarded as a
function defined on the set of sequences of nodes is injective in
the sense that for any two traversals $t_1$ and $t_2$:
\begin{itemize}
\item[(i)] if $\varphi (t_1^* ) = \varphi (t_2^* )$ then $t_1^* =t_2^*$;
\item[(ii)] if $\varphi (t_1 \upharpoonright r ) = \varphi (t_2 \upharpoonright r )$ then $t_1\upharpoonright r = t_2\upharpoonright r$.
\end{itemize}
\end{lem}

\begin{cor} \
\label{cor:SIGMACONST:varphi_bij}
\begin{itemize}
\item[(i)] $\varphi$ defines a bijection from $\travset(M)^*$
to $\varphi(\travset(M)^*)$;
\item[(ii)] $\varphi$ defines a bijection from $\travset(M)^{\upharpoonright r}$ to
$\varphi(\travset(M)^{\upharpoonright r})$.
\end{itemize}
\end{cor}


\subsubsection{Correspondence theorem}
We would like to prove the counterpart of proposition
\ref{prop:rel_gamesem_trav} in the context of the simply-typed
$\lambda$-calculus \emph{with interpreted PCF constants}. The game
model of the language \pcf\ is given by the category $\mathcal{C}_b$
of well-bracketed strategies. Hence the well-bracketing assumption
stated in section \ref{sec:assumptions} is satisfied.

We first prove that $\travset^{\upharpoonright r}$ is continuous.
\begin{lem}
\label{lem:travred_continuous} Let $(S,\subseteq)$ denote the set of
sets of justified sequences of nodes ordered by subset inclusion.
The function $\travset^{\upharpoonright r} : (CT,\sqsubseteq)
\rightarrow (S,\subseteq)$ is continuous.
\end{lem}
\begin{proof} \
    \begin{description}
    \item[Monotonicity:] Let $T$ and $T'$ be two computation trees such that $T \sqsubseteq T'$
    and let $t$ be some traversal of $T$.
    Traversals ending with a node labelled $\bot$ are maximal therefore $\bot$ can only occur
    at the last position in a traversal. Let us prove the following two properties:
        \begin{itemize}
            \item[(i)]  If $t = t \cdot n$ with $n\neq \bot$ then $t$ is a traversal of $T'$;
            \item[(ii)] if $t= t_1 \cdot \bot$ then $t_1\in \travset(T')$.
        \end{itemize}

        (i) By induction on the length of $t$. It is trivial for the empty traversal.
            Suppose that $t = t_1 \cdot n$ is a traversal with $n \neq \bot$.
            By the induction hypothesis, $t_1$ is a traversal of $T'$.

            We observe that for all traversal rules, the traversal produced is of the form $t_1 \cdot n$ where
            $n$ is defined to be a child node or value-leaf of some node $m$ occurring in $t_1$.
            Moreover, the choice of the node $n$ only depends on the traversal $t_1$
            (for the constant rules, this is guaranteed by assumption (WB)).

            Since $T \sqsubseteq T'$, any node $m$ occurring in $t_1$ belongs
            to $T'$ and the children nodes and leaves of $m$ in $T$ also belong to the tree $T'$.
            Hence $n$ is also present in $T'$ and the rule used to produce the traversal $t$ of $T$
            can be used to produce the traversal $t$ of $T'$.

        (ii) $\bot$ can only occur at the last position in a traversal
        therefore $t_1$ does not end with $\bot$ and by (i) we have $t_1\in \travset(T')$.
\vspace{6pt}

        Hence we have:
        \begin{align*}
        \travset(T)^{\upharpoonright r} &= \{ t \upharpoonright r \ | \ t \in \travset(T)     \} \\
        & = \{ (t\cdot n) \upharpoonright r \ | \ t\cdot n \in \travset(T) \wedge n \neq \bot \}
            \union \{ (t \cdot \bot ) \upharpoonright r \ | \ t \cdot \bot \in \travset(T)  \} \\
\mbox{(by (i) and (ii))} \quad        & \subseteq  \{ (t\cdot n)
\upharpoonright r \ | \ t\cdot n \in \travset(T') \wedge n \neq \bot
\}
            \union \{ t \upharpoonright r \ | \ t \in \travset(T')  \} \\
        & = \travset(T')^{\upharpoonright r}
        \end{align*}

        \item[Continuity:] Let $t \in \travset \left( \Union_{n\in\omega} T_n \right)$.
        We write $t_i$ for the finite prefix of $t$ of length $i$.
        The set of traversals is prefix-closed therefore $t_i \in \travset \left( \Union_{n\in\omega} T_n \right)$ for any $i$.
        Since $t_i$ has finite length we have $t_i \in \travset(T_{j_i})$ for some $j_i \in \omega$.
        Therefore we have:
        \begin{align*}
          t \upharpoonright r &= (\bigvee_{i\in\omega} t_i ) \upharpoonright r   & (\mbox{the sequence $(t_i)_{i\in\omega}$ converges to $t$}) \\
          &= \Union_{i\in\omega} ( t_i \upharpoonright r )   & (\_ \upharpoonright r \mbox{ is continuous, lemma \ref{lem:filtercontinous}}) \\
          &\in \Union_{i\in\omega} \travset^{\upharpoonright r}(T_{j_i})   & (t_i \in \travset(T_{j_i})) \\
          &\subseteq \Union_{i\in\omega} \travset^{\upharpoonright r}(T_i)   & (\mbox{since } \{ j_i \sthat i \in \omega \} \subseteq \omega)
        \end{align*}

        Hence $\travset^{\upharpoonright r} (\Union_{n\in\omega} T_n ) \subseteq \Union_{n\in\omega} \travset^{\upharpoonright r}(T_n).$

    \end{description}
\end{proof}

\begin{prop}
Let $\Gamma \vdash M : T$ be a PCF term and $r$ be the root of
$\tau(M)$. Then:
\begin{align*}
(i)  \quad\varphi_M(\travset(M)^*) = \intersem{M},  \\
(ii) \quad \varphi_M(\travset(M)^{\upharpoonright r}) = \sem{M}.
\end{align*}
\end{prop}
\begin{proof}
We first prove the result for $\pcf_1$: (i) The proof is an
induction identical to the proof of proposition
\ref{prop:rel_gamesem_trav}. However we need to complete the case
analysis with the $\Sigma$-constant cases:
\begin{itemize}
\item The cases \pcfsucc, \pcfpred, \pcfcond\ and numeral constants are straightforward.

\item Suppose $M = \Omega_o$ then $\travset(\Omega_o) = \prefset ( \{ \lambda \cdot \bot \} )$ therefore
$\travset(\Omega_o)^{\upharpoonright r} = \prefset( \{ \lambda \} )$
and $\sem{\Omega_o} = \prefset( \{ q \})$ with $\varphi(\lambda) =
q$. Hence $\sem{\Omega_o} = \varphi
(\travset(\Omega_o)^{\upharpoonright r})$.
\end{itemize}
(ii) is a direct consequence of (i) and the filtering lemma (lemma
\ref{lem:SIGMACONST:varphi_filter}). \vspace{10pt}

\noindent We now extend the result to \pcf. Let $M$ be a \pcf\ term,
we have:
\begin{align*}
\sem{M} &= \Union_{n\in\omega} \sem{M_n} & (\mbox{\cite{abramsky:game-semantics-tutorial}, lemma 16})\\
&= \Union_{n\in\omega} \travset^{\upharpoonright r}(\tau(M_n)) & (M_n \mbox{ is a $\pcf_1$ term}) \\
&= \travset^{\upharpoonright r}(\Union_{n\in\omega} \tau(M_n) ) & (\mbox{by continuity of $\travset^{\upharpoonright r}$, lemma \ref{lem:travred_continuous}}) \\
&= \travset^{\upharpoonright r}(\tau(M)) & (\mbox{by definition of } \tau(M)) \\
&= \travset^{\upharpoonright r}(M) & (\mbox{abbreviation}).
\end{align*}
\end{proof}

Hence by corollary \ref{cor:SIGMACONST:varphi_bij}, $\varphi$
defines a bijection from $\travset(M)^{\upharpoonright r}$ to
$\sem{M}$:
$$\varphi : \travset(M)^{\upharpoonright r} \stackrel{\cong}{\longrightarrow} \sem{M}.$$

\subsubsection{Example: \pcfsucc}

Consider the term $M = \pcfsucc\ 5$ whose computation tree is
represented below. The value-leaves are also represented on the
diagram, they are the vertices attached to their parent node with a
dashed line.
$$
\psmatrix[colsep=3ex,rowsep=2ex]
\lambda^0 \\
\pcfsucc & 0 & 1 & \ldots \\
\lambda^1 & 0 & 1 & \ldots \\
5 & 0 & 1 & \ldots \\
  & 0 & 1 & \ldots
\endpsmatrix
\ncline{1,1}{2,1} \ncline{2,1}{3,1} \ncline{3,1}{4,1}
\valueedge{1,1}{2,2} \valueedge{1,1}{2,3} \valueedge{1,1}{2,4}
\valueedge{2,1}{3,2} \valueedge{2,1}{3,3} \valueedge{2,1}{3,4}
\valueedge{3,1}{4,2} \valueedge{3,1}{4,3} \valueedge{3,1}{4,4}
\valueedge{4,1}{5,2} \valueedge{4,1}{5,3} \valueedge{4,1}{5,4}
$$

The following sequence of nodes is a traversal of $\tau(M)$:
\vspace{18pt}
$$ t = \rnode{l0}{\lambda^0} \cdot \rnode{succ}{\pcfsucc} \cdot \rnode{l1}{\lambda^1} \cdot \rnode{c5}{5} \cdot \rnode{55}{5_5} \cdot \rnode{5l1}{5_{\lambda^1}} \cdot \rnode{6succ}{6_\pcfsucc} \cdot \rnode{6l0}{6_{\lambda^0}}.
\link[offset=-4pt]{20}{6l0}{l0} \link[offset=-4pt]{20}{5l1}{l1}
\link[offset=-4pt]{20}{55}{c5} \link[offset=-4pt]{20}{6succ}{succ}
$$
The subsequences $t^*$ and $t \upharpoonright r$ are given by:
$$
t^* = \rnode{l0}{\lambda^0} \cdot \rnode{l1}{\lambda^1} \cdot
\rnode{5l1}{5_{\lambda^1}} \cdot \rnode{6l0}{6_{\lambda^0}}.
\link[offset=-4pt]{20}{6l0}{l0} \link[offset=-4pt]{20}{5l1}{l1}
\link[offset=-4pt]{20}{l1}{l0} \qquad  \mbox{ and } \qquad t
\upharpoonright r = \rnode{l0}{\lambda^0} \cdot
\rnode{6l0}{6_{\lambda^0}}. \link[offset=-4pt]{20}{6l0}{l0}
$$
We have $\varphi(t^*) = q_0 \cdot q_5 \cdot 5_{q_5} \cdot 5_{q_0}$
and $\varphi(t\upharpoonright r) = q_0 \cdot 5_{q_0}$ where $q_0$
and $q_5$ denote the roots of two flat arenas over $\nat$. These two
sequences of moves correspond to some play of the interaction
semantics and the standard semantics respectively. The interaction
play is represented below:
$$\begin{array}{ccccc}
  \textbf{1} & \stackrel{5}{\multimap} & !\nat & \stackrel{\pcfsucc}{\multimap} & \nat \\
&&&&  \rnode{q0}{q_0} \\
&&  \rnode{q5}{q_5} \\
&&  \rnode{a5}{5_{q_5}} \\
&&&&  \rnode{a6}{6_{q_0}}
\end{array}
\nccurve[nodesep=2pt,ncurv=0.9,angleA=180,angleB=180]{->}{a5}{q5}
\nccurve[nodesep=2pt,ncurv=0.9,angleA=180,angleB=210]{->}{a6}{q0}
\ncarc[nodesep=2pt,ncurv=0.9,angleA=180,angleB=180]{->}{q5}{q0}
$$

\subsubsection{Another example : \pcfcond}

Consider the term $M = \lambda x y . \pcfcond\ 1\ x\ y$. Its
computation tree is represented below (without the value-leaves):
    $$ \tree{\lambda x y}
       {
          \tree{\pcfcond}
          {
            \tree{\lambda^1}{ \TR{1} }
            \tree{\lambda^2}{ \TR{x} }
            \tree{\lambda^3}{ \TR{y} }
          }
      }
    $$
For any value $v \in\mathcal{D}$ the following sequence of nodes is
a traversal of $\tau(M)$: \vspace{18pt}
$$ t = \rnode{lxy}{\lambda x y} \cdot \rnode{cond}{\pcfcond} \cdot \rnode{l1}{\lambda^1} \cdot \rnode{1}{1} \cdot \rnode{11}{1_1}
    \cdot \rnode{l3}{\lambda^3} \cdot \rnode{y}{y} \cdot \rnode{vy}{v_y}  \cdot \rnode{vl3}{v_{\lambda^3}} \cdot \rnode{vcond}{v_{\pcfcond}}
    \cdot \rnode{vlxy}{v_{\lambda x y}}.
\link[offset=-4pt]{20}{vlxy}{lxy}
\link[offset=-4pt]{20}{vcond}{cond}
\link[offset=-4pt]{20}{vl3}{l3} \link[offset=-4pt]{20}{vy}{y}
\link[offset=-4pt]{20}{y}{vxy} \link[offset=-4pt]{20}{l2}{cond}
\link[offset=-4pt]{20}{11}{1} \link[offset=-4pt]{20}{l1}{cond}
$$
The subsequences $t^*$ and $t \upharpoonright r$ are given by:
\vspace{13pt}
$$
t^* =  t = \rnode{lxy}{\lambda x y} \cdot
        \rnode{l1}{\lambda^1} \cdot
        \rnode{l3}{\lambda^3} \cdot
        \rnode{y}{y} \cdot
        \rnode{vy}{v_y}  \cdot
        \rnode{vl3}{v_{\lambda^3}} \cdot
        \rnode{vlxy}{v_{\lambda x y}}
\link[offset=-4pt]{20}{vlxy}{lxy} \link[offset=-4pt]{20}{vl3}{l3}
\link[offset=-4pt]{20}{vy}{y} \link[offset=-4pt]{20}{y}{vxy}
\link[offset=-4pt]{20}{l3}{lxy} \link[offset=-4pt]{20}{l1}{lxy}
\qquad  \mbox{ and } \qquad t \upharpoonright r =
\rnode{lxy}{\lambda x y} \cdot \rnode{y}{y} \cdot \rnode{vy}{v_y}
\cdot \rnode{vlxy}{v_{\lambda x y}}.
\link[offset=-4pt]{20}{vlxy}{lxy} \link[offset=-4pt]{20}{vy}{y}
\link[offset=-4pt]{20}{y}{vxy}
$$
The sequence of moves $\varphi(t^*)$ corresponds to some play of the
interaction semantics and the sequence $\varphi(t\upharpoonright r)$
is a play of the standard semantics obtained by hiding the internal
moves of $\varphi(t^*)$. The interaction play $\varphi(t^*)$ is
represented below:
$$\begin{array}{ccccccccccc}
!\nat & \otimes & !\nat & \stackrel{ \langle \sem{1}, \pi_1,
\pi_2\rangle }{\multimap} & !\nat & \otimes & !\nat & \otimes &
!\nat
& \stackrel{ \pcfcond}{\multimap} & \nat \\
&&&&&&&&&&  \rnode{q0}{q_0^{(\lambda x y)}} \\
&&&&  \rnode{qa}{q_a^{(\lambda^1)}} \\
&&&&  \rnode{1}{1} \\
&&&&&&  \rnode{qb}{q_b^{(\lambda^2)}} \\
&&  \rnode{qy}{q_y^{(y)}} \\
&&  \rnode{vqy}{v_{q_y}} \\
&&&&&&  \rnode{vqb}{v_{q_b}} \\
&&&&&&&&&& \rnode{vq0}{v_{q_0}}
\end{array}
\ncarc[nodesep=2pt,ncurv=0.9,angleA=180,angleB=180]{->}{vq0}{q0}
\ncarc[nodesep=2pt,ncurv=0.9,angleA=180,angleB=180]{->}{vqb}{qb}
\nccurve[nodesep=2pt,ncurv=0.9,angleA=180,angleB=180]{->}{vqy}{qy}
\ncarc[nodesep=2pt,ncurv=0.9,angleA=180,angleB=180]{->}{qy}{qb}
\ncarc[nodesep=2pt,ncurv=0.9,angleA=90,angleB=180]{->}{qb}{q0}
\nccurve[nodesepB=2pt,nodesepA=6pt,ncurv=0.9,angleA=180,angleB=180]{->}{1}{qa}
\ncarc[nodesep=2pt,ncurv=0.9,angleA=90,angleB=180]{->}{qa}{q0}
$$


\subsubsection{Game characterisation of safe terms}

A difficulty arises because of the presence of the Y combinator :
computation trees of \pcf\ terms are potentially infinite. Despite
this particularity, lemma \ref{lem:safe_imp_incrbound} still holds
in the \pcf\ setting:
\begin{lem} \label{lem:pcf_safe_imp_incrbound} If $M$ is a safe
PCF term then $\tau(M)$ is incrementally-bound.
\end{lem}
\begin{proof}
Let $i$ denote the number of occurrences of the Y combinator in $M$.
We first prove by induction on $i$ that $M_k$ is safe for any $k\in
\omega$. \emph{Base case:} $i=0$ then $M_k = M$. \emph{Step case:}
$i>0$. Let $Y_A N$ be a subterm of $M$. Since $M$ is safe, $N$ is
also safe. The number of occurrences of the Y combinator in $N$ is
smaller than $i$ therefore by the induction hypothesis $N_k$ is
safe. Consequently the term $Y_A^k N_k = \underbrace{N_k ( \ldots (
N_k}_{k \mbox{ times}} \Omega ) \ldots )$ is also safe and by
compositionality so is $M_k$.

Clearly, lemma \ref{lem:safe_imp_incrbound}(i) is remains valid for infinite 
$\pcf_1$ terms (the subterms of the form $\Omega$ are just represented by
the constant $\bot$ in the computation tree), thus since $M_k$
is a safe $\pcf_1$ term, $\tau(M_k)$ is incrementally-bound.
Now let $z$ be a variable node in $\tau(M) =
\Union_{k\in\omega} \tau(M_k)$. There exists $k\in \omega$ such
that $z$ belongs to $\tau(M_k) \sqsubseteq \tau(M)$. 
If we write $r_k$ to denote the root of the tree $\tau(M_k)$ then the path $[r_k,z]$ in $\tau(M_k)$ is equal to the path $[r,z]$ in $\tau(M)$.
Hence, since the node $z$ is incrementally-bound in $\tau(M_k)$,
it is also incrementally-bound in $\tau(M)$.
\end{proof}


\begin{thm}
Safe PCF terms are denoted by P-incrementally-justified strategies.
\end{thm}
\begin{proof}
Let $M^{\infty}$ be the $\beta$-normal form of $M$ (i.e. the possibly infinite term obtained by reducing all the redexes in $M$). By lemma \ref{lem:ia_safety_preserved}, safety is preserved by small-step reduction therefore, by lemma \ref{lem:pcf_safe_imp_incrbound}, if $M$ is a \pcf\ term then $\tau(M^{\infty})$ is also 
incrementally-bound.

Since condition (WB) is verified ({\it i.e.} \pcf\ constant rules are well-behaved), lemma \ref{lem:redtrav_trav} holds in the safe \pcf\ setting.
Thus proposition \ref{prop:incrbound_imp_incrjustified}(i) remains valid
in \pcf\ for infinite computation trees: infinite terms in $\beta$-nf
with an incrementally-bound computation tree are denoted
by P-incrementally-justified strategies. Consequently, $\sem{M^{\infty}}$
is P-incrementally-justified.
By soundness of the game denotation, $\sem{M^{\infty}} = \sem{M}$, thus $\sem{M}$ is P-incrementally-justified.
\end{proof}

Consequently, P-pointers are superfluous in the game denotation of safe \pcf\ terms {\it i.e.} pointers emanating from P-moves are uniquely recoverable.

\subsection{Safe \ialgol}

We are now in a position to consider the full safe Idealized Algol
language. The general idea is the same as for safe \pcf, however
there are some difficulties caused by the presence of the two new
base types \iavar\ and \iacom. We just give indications on how to
adapt our framework to the particular case of safe \ialgol\ without
giving the complete proofs. However we believe that enough
indications are given to convince the reader that the argument used
in the \pcf\ case can be easily adapted to \ialgol.

\subsubsection{Computation DAG}
In \pcf, arenas have a single initial move, therefore they can be
regarded as trees. In \ialgol, on the other hand, the base type
\iavar\ is represented by the infinite product of games
$\iacom^{\nat} \times \iaexp$ which has an infinite number of
initial moves. In order to preserve the relationship established
between arenas and computation trees, we need to accommodate the
definition of computation tree to reflect this property. The
consequence is that in \ialgol, ``computation trees'' become
``computation directed acyclic graphs (DAG)'': a computation DAG may
have (possibly infinitely) many roots and two nodes of a given level
can share children at the next level.


We use the notations $\mathcal{D}_{\iaexp} = \nat$ and
$\mathcal{D}_{\iacom} = \{ \iadone \}$ to denote the set of value
leaves of type \iaexp\ and \iacom\ respectively. There are two types
of value-leaves in the computation DAG: the value-leaf \iadone\ of
type \iacom\ and the value-leaves labelled in $\mathcal{D}_{\iaexp}$
of type \iaexp.

Let $n$ be a node. If $\kappa(n)$ is of type $(A_1,\ldots A_n,B)$,
we call $B$ the \emph{return type of $n$}. The set of value-leaves
of a node $n$ is given by $\mathcal{D}_{\iaexp}$ if the return type
of $n$ is \iaexp, by $\mathcal{D}_{\iacom}$ if its return type is
\iacom, and by $\mathcal{D}_{\iaexp} \union \{ \iadone \}$ if its
return type is \iavar.


Table \ref{tab:ia_computationdag} shows the computation DAG for each
construct of \ialgol. The value-leaves are represented in the DAGs
using the following abbreviations:
$$ \tree{n}{ \TRV{\mathcal{D}_\iaexp} }  \quad \mbox{ for }\quad
 \tree{n}{ \TRV{0} \TRV{1} \TRV{2} \TRV{\ldots} }
 \qquad \mbox{ and } \qquad
 \tree{n}{ \TRV{\mathcal{D}_\iadone} }  \quad \mbox{ for }\quad
 \tree{n}{ \TRV{\iadone }}.
$$

A term of type \iavar\ has a computation DAG with an infinite number
of root $\lambda$-nodes. Suppose that $M$ is a term of type \iavar,
then the computation DAG for $\lambda \overline{\xi} . M$ is
obtained by relabelling the root $\lambda$-nodes $\lambda^r$,
$\lambda^{w_0}$, $\lambda^{w_1}$, $\lambda^{w_2}$, \ldots into
$\lambda^r \overline{\xi}$, $\lambda^{w_0} \overline{\xi}$,
$\lambda^{w_1} \overline{\xi}$, $\lambda^{w_2} \overline{\xi}$,
\ldots. For a term $M$  of type \iaexp\ or \iacom, the computation
DAG for $\lambda \overline{\xi} . M$ is computed in the same way as
in the safe $\lambda$-calculus.

\begin{table}
\begin{center}
\begin{tabular}{cc}
$M$ & $\tau(M)$ \\ \hline \hline \\
x $: A \in \{ \iacom, \iaexp \}$ &
    $\psmatrix[colsep=3ex,rowsep=2ex] \lambda \\ x & \mathcal{D}_A \\  & \mathcal{D}_A \endpsmatrix
    \ncline{1,1}{2,1} \valueedge{1,1}{2,2} \valueedge{2,1}{3,2} $
\\ \\
x : \iavar &
    $\psmatrix[colsep=3ex,rowsep=3ex]
    \lambda^r & \lambda^{w_0} & \lambda^{w_1}  & \lambda^{w_2} & \lambda^{w_{\ldots}} \\
    \mathcal{D}_\iaexp &  & x & & \iadone \\
    &  &  & \mathcal{D}_\iaexp & \iadone
    \endpsmatrix
    \ncline{1,1}{2,3} \ncline{1,2}{2,3} \ncline{1,3}{2,3} \ncline{1,4}{2,3} \ncline{1,5}{2,3}
    \valueedge{2,3}{3,4} \valueedge{2,3}{3,5}
    \valueedge{1,1}{2,1}
    \valueedge{1,5}{2,5} \valueedge{1,4}{2,5} \valueedge{1,3}{2,5} \valueedge{1,2}{2,5}
    $
\\ \\
\iaskip : \iacom &
    $\psmatrix[colsep=3ex,rowsep=3ex] \lambda \\ \iaskip & \iadone \\  & \iadone \endpsmatrix
    \ncline{1,1}{2,1} \valueedge{1,1}{2,2} \valueedge{2,1}{3,2} $
\\ \\
$\iaassign\ L\ N :\iacom$ &
    $\psmatrix[colsep=3ex,rowsep=3ex] & \lambda \\ & \iaassign & \iadone \\ \tau(N:\iaexp)  & \tau(L:\iavar) & \iadone \endpsmatrix
    \ncline{1,2}{2,2} \ncline{2,2}{3,2} \ncline{2,2}{3,1}
    \valueedge{1,2}{2,3} \valueedge{2,2}{3,3} $
\\ \\
$\iaderef\ L :\iaexp$ &
    $\psmatrix[colsep=3ex,rowsep=3ex] \lambda \\ \iaderef & \iadone \\ \tau(L:\iavar) & \iadone \endpsmatrix
    \ncline{1,1}{2,1} \ncline{2,1}{3,1} \valueedge{1,1}{2,2} \valueedge{2,1}{3,2} $
\\ \\
$\iaseq_{\iaexp}\ N_1\ N_2 :\iacom$ &
    $\psmatrix[colsep=3ex,rowsep=3ex] & \lambda \\ & \iaseq_{\iaexp} & \mathcal{D}_\iaexp \\ \tau(N_1:\iacom)  & \tau(N_2:\iaexp) & \iadone \endpsmatrix
    \ncline{1,2}{2,2} \ncline{2,2}{3,2} \ncline{2,2}{3,1}
    \valueedge{1,2}{2,3} \valueedge{2,2}{3,3} $
\\ \\
$\iamkvar\ N_w\ N_r :\iavar$ &
    $\psmatrix[colsep=3ex,rowsep=3ex]
    \lambda^r & \lambda^{w_0} & \lambda^{w_1}  & \lambda^{w_2} & \lambda^{w_{\ldots}} \\
    \mathcal{D}_\iaexp &  & \iamkvar & & \iadone \\
    & \tau(N_r) & \tau(N_w) & \mathcal{D}_\iaexp & \iadone
    \endpsmatrix
    \ncline{1,1}{2,3} \ncline{1,2}{2,3} \ncline{1,3}{2,3} \ncline{1,4}{2,3} \ncline{1,5}{2,3}
    \ncline{2,3}{3,2} \ncline{2,3}{3,3}
    \valueedge{2,3}{3,4} \valueedge{2,3}{3,5}
    \valueedge{1,1}{2,1}
    \valueedge{1,5}{2,5} \valueedge{1,4}{2,5} \valueedge{1,3}{2,5} \valueedge{1,2}{2,5}
    $
\\ \\
$\ianewin{x}\ N : A \in \{ \iacom, \iaexp \} $ &
   $\psmatrix[colsep=3ex,rowsep=3ex] \lambda \\ \ianewin{x} & \mathcal{D}_A \\ \tau(N:A) & \mathcal{D}_A \endpsmatrix
    \ncline{1,1}{2,1} \ncline{2,1}{3,1} \valueedge{1,1}{2,2} \valueedge{2,1}{3,2} $
\end{tabular}
\end{center}
  \caption{Computation DAGs for the constructs of \ialgol.}
  \label{tab:ia_computationdag}
\end{table}


\subsubsection{Traversals}
Let $p$ be a node and suppose that its $i$th child $n$ has the
return type \iavar. Then $n$ is in fact constituted of several
$\lambda$-nodes : $\lambda^r \overline{\xi}$, $\lambda^{w_0}
\overline{\xi}$, \ldots. From $p$'s point of view, these nodes are
referenced as follows: $i.r$ refers to $\lambda^r \overline{\xi}$
and  $i.w_k$ refers to $\lambda^{w_k} \overline{\xi}$ for $k \in
\omega$.

\begin{itemize}
\item \emph{The application rule}

There are two rules (app$_{\iaexp}$) and (app$_{\iacom}$)
corresponding to traversals ending with an @-node of return type
\iaexp\ and \iacom\ respectively. These rules are identical to the
rule \iaexp\ of section \ref{subsec:traversal}.

The application rule for $@$-nodes with return type \iavar\ is:
$$(\mbox{app}_{\iavar})
\rulefex[5pt]{t \cdot \rnode{l-}{\lambda^k \overline{\xi}} \cdot
\rnode{app-}{@} \in \travset} {t \cdot \rnode{l}{\lambda^k
\overline{\xi}} \cdot \rnode{app}{@} \cdot \rnode{l2}{\lambda^k
\overline{\eta}} \in \travset }
 \ k \in \{ r, w_0, w_1, \ldots \}
\link{40}{app-}{l-} \lnklabelc{0} \link{40}{app}{l} \lnklabelc{0}
\link{40}{l2}{app} \lnklabelc{0.k}
$$


\item \emph{Input-variable rules}

There are two rules (InputVar$^{\iaexp}$) and (InputVar$^{\iacom}$)
which are the counterparts of rule (InputVar$^0$) of section
\ref{subsec:traversal} and are defined identically.

Let $x$ be an input-variable of type \iavar:
$$ (\mbox{InputVar}^{\iavar})
\rulef{t \cdot \lambda^r \overline{\xi} \cdot x \in \travset}
    {t \cdot \lambda^r \overline{\xi} \cdot \rnode{x}{x} \cdot v_x \in \travset }
\hspace{2cm} (\mbox{InputVar}^{' \iavar}) \rulef{t \cdot
\lambda^{w_i} \overline{\xi} \cdot x \in \travset}
    {t \cdot \lambda^{w_i} \overline{\xi} \cdot \rnode{x}{x} \cdot \iadone_x \in \travset }
$$

\item \emph{IA constants rules}

The rules for \ianew\ are purely structural, they are defined the
same way as the rules (app$_{\iaexp}$), (app$_{\iacom}$) and
(app$_{\iadone}$).

The rules for \iaderef\ are:
$$(\mbox{deref}) \rulefex[6pt]{t \cdot \iaderef \in \travset}{t \cdot \rnode{d}{\iaderef} \cdot \rnode{n}{n} \in \travset }
\link{30}{n}{d} \lnklabelc{1.r} \hspace{1.6cm} (\mbox{deref'})
\rulef{t \cdot \iaderef \cdot n \cdot t_2 \cdot v_n \in \travset} {t
\cdot \iaderef \cdot n \cdot t_2 \cdot v_n \cdot v_{\iaderef}\in
\travset }
$$

The rules for \iaassign\ are:
$$(\mbox{assign}) \rulefex[7pt]{t \cdot \iaassign \in \travset}{t \cdot \rnode{ass}{\iaassign} \cdot \rnode{n}{n} \in \travset }
\link{30}{n}{ass} \lnklabelc{1} \hspace{1.6cm} (\mbox{assign'})
\rulefex[10pt]{t \cdot \iaassign \cdot n \cdot t_2 \cdot v_n \in
\travset} {t \cdot \rnode{ass}{\iaassign} \cdot \rnode{n}{n} \cdot
t_2 \cdot v_n \cdot \rnode{m}{m} \in \travset } \link{13}{m}{ass}
\lnklabelc{2.w_n}
$$
$$(\mbox{assign''})  \rulef{t \cdot \rnode{ass-}{\iaassign} \cdot t_2 \cdot \rnode{m-}{m} \cdot t_3 \cdot \iadone_m \in \travset}
{t \cdot \iaassign \cdot t_2 \cdot m \cdot t_3 \cdot \iadone_m \cdot
\iadone_{\iaassign} \in \travset } \link{20}{m-}{ass-}
\lnklabelc{2.w_k}
$$

The rules for $\iaseq_{\iaexp}$ are:
$$(\mbox{seq}) \rulefex[5pt]{t \cdot \iaseq \in \travset}{t \cdot \rnode{seq}{\iaseq} \cdot \rnode{n}{n} \in \travset }
\link{30}{n}{seq} \lnklabelc{1} \hspace{1.6cm} (\mbox{seq'})
\rulefex[5pt]{t \cdot \iaseq \cdot n \cdot t_2 \cdot v_n \in
\travset} {t \cdot \rnode{seq}{\iaseq} \cdot \rnode{n}{n} \cdot t_2
\cdot v_n \cdot \rnode{m}{m} \in \travset } \link{13}{m}{seq}
\lnklabelc{2}
$$
$$(\mbox{seq''})  \rulef{t \cdot \rnode{seq-}{\iaseq} \cdot t_2 \cdot \rnode{m-}{m} \cdot t_3 \cdot v_m \in \travset}
{t \cdot \iaseq \cdot t_2 \cdot m \cdot t_3 \cdot v_m \cdot
v_{\iaseq} \in \travset } \link{20}{m-}{seq-} \lnklabelc{2}
$$




The rules for \iamkvar\ are:
$$(\mbox{mkvar}_r) \rulefex[5pt]{t \cdot \lambda^r \overline{\xi} \cdot \iamkvar \in \travset}{t \cdot \lambda^r \overline{\xi} \cdot \rnode{d}{\iamkvar} \cdot \rnode{n}{n} \in \travset }
\link{30}{n}{d} \lnklabelc{1} \hspace{1cm} (\mbox{mkvar}_r')
\rulef{t \cdot \iamkvar \cdot n \cdot t_2 \cdot v_n \in \travset} {t
\cdot \iamkvar \cdot n \cdot t_2 \cdot v_n \cdot v_{\iamkvar}\in
\travset } $$
$$(\mbox{mkvar}_w) \rulefex[5pt]{t \cdot \lambda^{w_k} \overline{\xi} \cdot \iamkvar \in \travset}{t \cdot \lambda^{w_k} \overline{\xi} \cdot \rnode{mk}{\iamkvar} \cdot \rnode{n}{n} \in \travset }
\link{30}{n}{mk} \lnklabelc{2} $$
$$ (\mbox{mkvar}_w'')  \rulef{t \cdot \lambda^{w_k} \overline{\xi} \cdot \iamkvar \cdot n \cdot t_2 \cdot \iadone_n \in \travset}
{t \cdot \lambda^{w_k} \overline{\xi} \cdot \iamkvar \cdot n \cdot
t_2 \cdot \iadone_n \cdot \iadone_{\iamkvar} \in \travset }
$$
These four rules are not sufficient to model the constant \iamkvar.
Indeed, consider the term $\iaassign\ (\iamkvar\ (\lambda x . M) N)
7$. The rule (\mbox{mkvar}$_w''$) permits to traverse the node
\iamkvar\ and to go on by traversing the computation tree of
$\lambda x . M$. The problem is that when traversing $\tau(M)$, if
we reach a variable $x$, we are not able to relate $x$ to the value
$7$ that is assigned to the variable.

To overcome this problem, we need to define traversal rules for
variable in such a way that a variable node bound by the second
child of a $\iamkvar$-node is treated differently from other
variables.

\item \emph{Variable rules}
Let $x$ be a non input-variable node and let $b$ be the binder of
$x$. $b$ is either a ``$\ianewin{x}$''-node or a $\lambda$-node.

\begin{itemize}
\item Consider the case where $b$ is a $\lambda$-node. Take $b = \lambda \overline{x}$.
In \ialgol, the only constant nodes of order greater than 1 is
\iamkvar, therefore there are two cases: $\lambda \overline{x}$ is
either the child of a node in $N_@ \union N_{var}$ or it is the
second child of a \iamkvar-node.

To handle the first case, we define a rule similar to the (Var) rule
of section \ref{subsec:traversal} with some modification to take
into account variables $x$ of type \iavar (in which case $x$ has
multiple parent $\lambda$-nodes). We do not give the details here
but it is easy to see how to redefine this rule.

To handle the case where $\lambda \overline{x}$ is the child of a
\iamkvar-node, we define the following rule:
$$ (\mbox{Var}_{\iamkvar})  \rulef{t \cdot \lambda^{w_k} \overline{\xi} \cdot \iamkvar \cdot \lambda \overline{x} \cdot t_2 \cdot x \in \travset}
{t \cdot \lambda^{w_k} \overline{\xi} \cdot \iamkvar \cdot \lambda
\overline{x} \cdot t_2 \cdot x \cdot k_{x} \in \travset }
$$

\item The case $b = \ianewin{x}$ is handle by the following rules.

We call \emph{overwrite of $x$ relatively to an occurrence of a ``\ianewin{x}'' node}, any sequence of nodes of the form
\raisebox{0cm}[0.5cm]{$\rnode{decl}{\ianewin{x}}\cdot \ldots \cdot \lambda^{w_k}\overline{\xi} \cdot \rnode{x}{x}$} \link[nodesep=1pt]{17}{x}{decl} for some $k\in \mathcal{D}_{\iaexp}$ and node $\lambda^{w_k}\overline{\xi}$ parent
of $x$.
$$(\mbox{Var}_w)
\rulef{t \cdot \lambda^{w_k} \overline{\xi} \cdot x \in \travset}
{t \cdot \lambda^{w_k} \overline{\xi} \cdot x \cdot \iadone_x \in \travset },$$

$$(\mbox{Var}_r) \rulef{\raisebox{0cm}[0.5cm]{$
t_1 \cdot \rnode{decl}{\ianewin{x}} \cdot t_2 \cdot \lambda^r \overline{\xi} \cdot \rnode{x}{x} \in
 \travset$ \link[nodesep=2pt]{20}{x}{decl}
}}
{t_1 \cdot \ianewin{x} \cdot t_2 \cdot \lambda^r \overline{\xi} \cdot x \cdot 0_x \in \travset
}
\mbox{ if $t_2$ contains no overwrite of $x$},
$$

$$(\mbox{Var'}_r) \rulef{\raisebox{0cm}[0.5cm]{$t_1 \cdot \rnode{decl}{\ianewin{x}} \cdot t_2 \cdot \lambda^r \overline{\xi} \cdot \rnode{x}{x} \in \travset \link[nodesep=2pt]{20}{x}{decl}$}}
{t_1 \cdot \ianewin{x} \cdot t_2 \cdot \lambda^r \overline{\xi} \cdot x \cdot k_x \in \travset }
\mbox{ if $\lambda^{w_k} \cdot x$ is the last overwrite of $x$ in }
t_2. $$
\end{itemize}
\end{itemize}

\subsubsection{Game semantics correspondence}
The properties that we proved for computation trees and traversals
of the safe $\lambda$-calculus with constants can easily be lifted
to computation DAGs of \ialgol. In particular:
\begin{itemize}
\item constant traversal rules are well-behaved;
\item P-view of traversals are paths in the computation DAG;
\item the P-view of the reduction of a traversal is the reduction of the P-view,
and the O-view of a traversal is the O-view of its reduction (lemma
\ref{lem:redtrav_trav});
\item there is a mapping from vertices of the computation DAG to moves in the interaction game semantics;
\item there is a correspondence between traversals of the computation tree and plays in interaction game semantics;
\item consequently, there is a correspondence between the standard game semantics and
the set of justified sequences of nodes $\travset^{\upharpoonright
r}$.
\end{itemize}

\subsubsection{Game-semantic characterisation of safe terms}
Clearly, the computation DAG of a safe term is incrementally-bound.
By using the correspondence between traversals and plays, it is easy
to prove that incrementally-bound computation trees are denoted by
P-incrementally-justified strategies. Consequently, by lemma
\ref{lem:incrjustified_pointers_uniqu_recover}, P's pointers are superfluous in the
game semantics of safe \ialgol\ terms.

Since the game denotation of an \ialgol\ term is fully determined by
the set of complete plays, this pointer economy suggests that the
game denotation of a safe \ialgol\ can be represented in a compact
way. This raises the question of the decidability of observational
equivalence for safe \ialgol.










%%%%%%%%%%%%%%%%%%%%%%%%%%%%%%%%%%%%%%%%%%%%%%%%%%%%%%%%%%%

\chapter{Further possible developments}

In the previous chapter, we have given an account of the game
semantics of safe $\lambda$-calculus. However the nature of this
calculus is still not well known. We propose the following possible
roadmap for further research:
\begin{enumerate}
\item give a detailed account of
P-incrementally-justified strategies that treats the problem of compositionality;
\item find a categorical interpretation of the safe $\lambda$-calculus;
\item study the proof theory obtained by the Curry-Howard isomorphism and determine whether it has nice properties that can be helpful in theorem proving;
\item identify a non-trivial fragment of safe \ialgol\ for which observational equivalence is decidable;
\item in \cite{DBLP:conf/tlca/LeivantM93}, the $\lambda$-calculus is used to
give several characterisations of the complexity class P. We would
like to investigate whether, by following similar techniques, we can
obtain a characterisation of a different complexity class using the
safe $\lambda$-calculus.
\end{enumerate}


More generally, we would like to study the class of languages for
which pointers are uniquely recoverable. We name this class PUR for
``Pointer Uniquely Recoverable''.

An example is the Serially Re-entrant Idealized Algol (SRIA) proposed
by Abramsky  in \cite{abramsky:mchecking_ia}. This language allows
multiple occurrences or uses of arguments, as long as they do not
overlap in time. In the game semantics denotation of a SRIA term
there is at most one pending occurrence of a question at any time.
Each move has therefore a unique justifier and consequently
justification pointers may be ignored. Safe \ialgol\ is not a
sublanguage of SRIA. One reason for this is that none of the two
Kierstead terms $\lambda f . f (\lambda x . f (\lambda y .y ))$ and
$\lambda f . f (\lambda x . f (\lambda y .x ))$ are Serially
Re-entrant whereas the first one is safe. Conversely, SRIA is not a
sublanguage of safe \ialgol\ since the term $\lambda f g. f (\lambda
x . g (\lambda y .x ))$ where $f,g:((o,o),o)$ belongs to SRIA but
not to safe \ialgol.

Finitary $\ialgol_2$ is also an example of PUR-language for which
observational equivalence is decidable. As we indicated in the first
chapter, decidability of observational equivalence is a very
appealing property which has immediate applications in the domain of
program verification. Intuitively, PUR-languages seem to be good
candidates of languages for which observational equivalence is decidable. It would be interesting to discover classes of PUR
languages having this appealing property. Safe $\ialgol_3$ seems to be a good candidate.

Another possible way to generate PUR-languages may be to constrain
the types of an existing language. In \cite{DBLP:conf/tlca/Joly01},
a notion of ``complexity'' is defined for $\lambda$-terms. It is
proved that a type $T$ can be generated from a finite set of
combinators if and only if there is a constant bounding the
complexity of every closed normal $\lambda$-term of type $T$;
consequently, the only inhabited finitely generated types are the
type of rank $\leq 2$ and the types $(A_1, A_2, \ldots, A_n, o)$
such that for all $i = 1..n$: $A_i = o$ , $A_i = o \rightarrow o$ or
$A_i = (o^k \rightarrow o) \rightarrow o$. We know that imposing the
first of these two type restrictions to Finitary \ialgol\ leads to a
PUR language. Is it also the case when imposing the second type
restriction?

