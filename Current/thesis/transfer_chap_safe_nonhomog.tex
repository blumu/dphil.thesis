\section{Safe $\lambda$-Calculus without the Homogeneity Constraint}
\label{sec:safe_nonhomog}


In section \ref{sec:safe_homog}, we have presented a version of the
safe lambda calculus where types are required to be homogeneous. We
now give a more general version of the safe simply-typed
$\lambda$-calculus where type homogeneity is not required.

\subsection{Rules}

We use a set of sequents of the form $\Gamma \vdash M : A$ where
$\Gamma$ is the context of the term and $A$ is its type. Let
$\Sigma$ be a set of higher-order constants. We call safe terms any
simply-typed lambda term that is typable within the following system
of formation rules:
$$ \rulename{var} \   \rulef{}{x : A\vdash x : A}
\qquad  \rulename{const} \   \rulef{}{\vdash f : A} \quad f \in \Sigma
\qquad  \rulename{wk} \   \rulef{\Gamma \vdash M : A}{\Delta \vdash M : A} \quad \Gamma \subset \Delta$$

$$ \rulename{app} \  \rulef{\Gamma \vdash M : (A,\ldots,A_l,B)
                                        \qquad \Gamma \vdash N_1 : A_1
                                        \quad \ldots \quad \Gamma \vdash N_l : A_l  }
                                   {\Gamma  \vdash M N_1 \ldots N_l : B}
                                    \quad
                                   \forall y \in \Gamma : \ord{y} \geq \ord{B}$$

$$ \rulename{abs} \   \rulef{\Gamma \union \overline{x} : \overline{A} \vdash M : B}
                                   {\Gamma  \vdash \lambda \overline{x} : \overline{A} . M : (\overline{A},B)} \qquad
                                   \forall y \in \Gamma : \ord{y} \geq \ord{\overline{A},B}$$


Remark:
\begin{itemize}
\item $(\overline{A},B)$ denotes the type $(A_1,A_2, \ldots, A_n, B)$;
\item all the types appearing in the rule are not required to be homogeneous (for instance
it is possible to have $\ord{A_l} < \ord{B}$ in rule $\rulename{app}$) ;
\item the environment $\Gamma \union \overline{x}:\overline{A}$ is not stratified, in particular, variables in $\overline{x}$ do not necessarily have the same order;
\item in the abstraction rule, the side-condition imposes that at least all variables of the lowest order
in the context are abstracted. Variables of greater order can also be
abstracted together with the lowest order variables and, in contrast to
the homogeneous safe lambda calculus, there is no constraint on the
order in which these variables are abstracted;
\end{itemize}

\begin{example}
For $x:o$, $f:(o,o)$ and $\varphi:((o,o),o)$ the term $$\vdash \lambda x f \varphi .
\varphi : (o , (o, o) , ((o,o),o) , (o,o),o)$$ is
a valid safe term that is not homogeneously typed.
\end{example}

\begin{example}
For $x:o$, $g:(o,(o,o),o)$, the term $\vdash \lambda g x . g x$ is unsafe and not homogeneously typed
and the term $\lambda g x . g x (\lambda x . x)$ is safe and not homogeneously typed.
\end{example}

Side-remark: safety is preserved by full $\eta$-expansion. Indeed,
consider the safe term $\Gamma \vdash M:(A_1,\ldots,A_l,o)$ where
$(A_1,\ldots,A_l,o)$ is not necessarily homogeneous. Its full $\eta$-expansion
is $\lambda x_1 .. x_l . M x_1 \dots x_l$ for some variables
$x_1:A_1, \ldots, x_l:A_l$ fresh in $M$. For all $i \in 1..l$ we
have $\Gamma, \Sigma \vdash x_i :A_i$ where $\Sigma = \{ x_1:A_1,
\cdots x_l :A_l \}$. Applying $\rulename{app}$ we obtain $\Gamma,
\Sigma \vdash M x_1 \ldots x_l$ and by the (abs) rule we get
$$\Gamma \vdash \lambda x_1:A_1 \ldots x_l:A_l .M x_1 \ldots x_l.$$

\begin{lemma}[Context reduction]
\label{lem:nonhomosafe_basic_prop}
If $\Gamma \vdash M : B$ is a valid judgment then
\begin{enumerate}
\item $fv(M) \vdash M : B$
\item every variable in $\Gamma$ \emph{occurring free in $M$} has order at
least $ord(M)$.
\end{enumerate}
where $fv(M)$ denotes the context constituted of the free variables occurring in $M$.
\end{lemma}
\begin{proof}
(i) Suppose that some variable $x$ in $\Gamma$ does not occur free
in $M$, then necessarily $x$ has been introduced in the context
using the weakening rule. Hence $\Gamma\setminus \{ x \} \vdash M$
must also be typable. (ii) An easy structural induction.
\end{proof}

The converse of this lemma is not true: consider the simply-typed
term $\lambda y z. (\lambda x . y ) z$ with $x,y,z:o$. This term is
closed therefore it satisfies property (i) and (ii) of lemma
\ref{lem:nonhomosafe_basic_prop}. However it is not typable by the
rules of the safe lambda-calculus since the subterm $\lambda x .y$
is not safe.

\subsection{Substitution in the safe lambda calculus}

The traditional notion of substitution, on which the
$\lambda$-calculus is based, is defined as follows:
\begin{definition}[Substitution]
\label{dfn:subst}
\begin{eqnarray*}
c \subst{t}{x} &=& c \quad \mbox{where $c$ is a $\Sigma$-constant},\\
x \subst{t}{x} &=& t\\
 y\subst{t}{x} &=& y \quad \mbox{for } x \not \neq y,\\
(M_1 M_2) \subst{t}{x} &=& (M_1 \subst{t}{x}) (M_2 \subst{t}{x})\\
(\lambda x . M) \subst{t}{x} &=& \lambda x . M\\
(\lambda y . M) \subst{t}{x} &=& \lambda z . M \subst{z}{y}
\subst{t}{x} \mbox{where $z$ is a fresh variable and $x\not = y$}.
\end{eqnarray*}
\end{definition}

In the setting of the safe lambda calculus, the notion of
substitution can be simplified. Indeed, similarly to what we observe
in the homogeneous safe $\lambda$-calculus, we remark that for safe
$\lambda$-terms there is no need to rename variables when performing
substitution:

\begin{lemma}[No variable capture lemma]
\label{lem:noclash} There is no variable capture when performing
substitution on a safe term.
\end{lemma}

This is the counterpart of lemma \ref{lem:homog_nocapture}. The
proof (which does not rely on homogeneity) is the same.
Consequently, in the safe lambda calculus setting, we can omit to
rename variable when performing substitution. The equation
$$(\lambda x . M) \subst{t}{y} = \lambda z . M \subst{z}{x}
\subst{t}{y} \mbox{where $z$ is a fresh variable}$$ becomes
$$(\lambda x . M) \subst{t}{y} = \lambda x . M \subst{t}{y}.$$

Unfortunately, this notion of substitution is still not adequate for
the purpose of the safe simply-typed lambda calculus. The problem is
that performing a single $\beta$-reduction on a safe term will not
necessarily produce another safe term.

The solution consists in reducing several consecutive $\beta$-redex
at the same time until we obtain a safe term. To achieve this, we
introduce the \emph{simultaneous substitution}, a generalization of
the standard substitution given in definition \ref{dfn:subst}.

\begin{definition}[Simultaneous substitution]
\label{dnf:simsubst}
 The expression $\subst{\overline{N}}{\overline{x}}$ is an abbreviation for $\subst{N_1 \ldots N_n}{x_1
\ldots x_n}$:
\begin{eqnarray*}
c \subst{\overline{N}}{\overline{x}} &=& c \quad \mbox{where $c$ is a $\Sigma$-constant},\\
x_i \subst{\overline{N}}{\overline{x}} &=& N_i\\
 y \subst{\overline{N}}{\overline{x}} &=& y \quad \mbox{ if } y \not \neq x_i \mbox{ for all } i,\\
(M N) \subst{\overline{N}}{\overline{x}} &=& (M \subst{\overline{N}}{\overline{x}}) (N \subst{\overline{N}}{\overline{x}}) \\
(\lambda x_i . M) \subst{\overline{N}}{\overline{x}} &=& \lambda x_i
. M
\subst{N_1 \ldots N_{i-1} N_{i+1}\ldots N_n}{x_1 \ldots x_{i-1} x_{i+1}\ldots x_n} \\
(\lambda y . M)
\subst{\overline{N}}{\overline{x}} &=& \lambda z . M \subst{z}{y} \subst{\overline{N}}{\overline{x}} \\
&& \mbox{where $z$ is a fresh variables and } y \neq x_i \mbox{ for
all } i.
\end{eqnarray*}
\end{definition}

In general, variable capture should be avoided, this explains why
the definition of simultaneous substitution uses auxiliary fresh
variables. However in the current setting, lemma \ref{lem:noclash}
can clearly be transposed to the simultaneous substitution,
therefore there is no need to rename variables.

The notion of substitution that we need is therefore the
\emph{capture-permitting simultaneous substitution} defined as
follows:

\begin{definition}[Capture-permitting simultaneous substitution]
 We use the notation
$\subst{\overline{N}}{\overline{x}}$ for $\subst{N_1 \ldots N_n}{x_1
\ldots x_n}$:
\begin{eqnarray*}
c \subst{\overline{N}}{\overline{x}} &=& c \quad \mbox{where $c$ is a $\Sigma$-constant},\\
 x_i \subst{\overline{N}}{\overline{x}} &=& N_i\\
 y \subst{\overline{N}}{\overline{x}} &=& y \quad \mbox{where } x \not \neq y_i \mbox{ for all } i,\\
(M_1 M_2) \subst{\overline{N}}{\overline{x}} &=& (M_1 \subst{\overline{N}}{\overline{x}}) (M_2 \subst{\overline{N}}{\overline{x}})\\
(\lambda x_i . M) \subst{\overline{N}}{\overline{x}} &=& \lambda x_i
. M
\subst{N_1 \ldots N_{i-1} N_{i+1}\ldots N_n}{x_1 \ldots x_{i-1} x_{i+1}\ldots x_n} \\
(\lambda y . M) \subst{\overline{N}}{\overline{x}} &=& \lambda y . M
\subst{\overline{N}}{\overline{x}} \mbox{where $y \not = x_i$ for
all $i$}. \qquad \mathbf{(\star)}
\end{eqnarray*}
The symbol $\mathbf{(\star)}$ identifies the equation which has
changed compared to the previous definition.
\end{definition}

\begin{lemma}[Substitution preserves safety]
\label{lem:subst_preserve_i}
$$ \Gamma\union \overline{x} : \overline{A}\vdash M : T
\quad \mbox{and} \quad \Gamma \vdash N_k : B_k \mbox{, } k \in
1..n \qquad \mbox{ implies } \qquad \Gamma \vdash
M[\overline{N}/\overline{x}] : T$$
\end{lemma}

\begin{proof}
Suppose that $\Gamma \union \overline{x}: \overline{A} \vdash M :T$ and
$\Gamma \vdash N_k : B_k$ for $k \in 1..n$.

We prove $\Gamma \vdash M[\overline{N}/\overline{x}]$ by induction
on the size of the proof tree of $\Gamma\union
\overline{x}:\overline{A} \vdash M : T$ and by case analysis on the
last rule used. We only give the proof for the abstraction case. If
$\Gamma \union \overline{x}:\overline{A} \vdash \lambda \overline{y}
: \overline{C}. P : (\overline{C}|D)$ where $\Gamma\union
\overline{x}:\overline{A}\union \overline{y}:\overline{C} \vdash P :
D$, then by the induction hypothesis $\Gamma\union
\overline{y}:\overline{C} \vdash P\subst{\overline{N}}{\overline{x}}
: D$. Applying the rule $\rulename{abs}$ gives $\Gamma \vdash
\lambda \overline{y}:\overline{C} . P
\subst{\overline{N}}{\overline{x}}$.
\end{proof}

\subsection{Safe-redex}
In the simply-typed lambda calculus a redex is a term of the form
$(\lambda x . M) N$. We generalize this definition to the safe
lambda calculus:
\begin{definition}[Safe redex]
We call safe redex a term of the form $(\lambda \overline{x} . M)
N_1 \ldots N_l$ such that:
\begin{itemize}
\item $ \Gamma \vdash (\lambda \overline{x} . M) N_1 \ldots N_l $;
\item the variable $\overline{x}=x_1\ldots x_n$ are abstracted altogether by one occurrence of the rule $\rulename{abs}$ in the proof
tree;
\item the terms $(\lambda \overline{x} . M)$, $N_1$, $N_l$ are applied together at once using the $\rulename{app}$ rule :
$$   \rulef{
            \Sigma \vdash \lambda \overline{x} . M
            \quad
            \Sigma \vdash N_1         \quad \ldots \quad \Sigma \vdash N_l
    }
    {
       \Sigma \vdash (\lambda \overline{x} . L) N_1 \ldots N_l
    } (\mathbf{app})
$$
and consequently each $N_i$ is safe;
\end{itemize}
\end{definition}

The relation $\beta_s$ is defined exactly the same way as in the homogeneous safe $\lambda$-calculus. The safe $\beta$-reduction $\betasred$ is defined as the closure of $\beta_s$ by
compatibility with the formation rules of the safe
$\lambda$-calculus.  It is straightforward to show, as we did for the homogeneous safe $\lambda$-calculus, that $\betasred \subset \betaredtr$.


\begin{lemma}
\label{lem:safereduction} A safe redex reduces to a safe term.
\end{lemma}

This lemma, which is a consequence of lemma
\ref{lem:subst_preserve_i}, is the counterpart of lemma
\ref{lem:homoh_safered_preserve_safety} in the homogeneous safe
lambda calculus. Their proofs are identical.


\subsection{Particular case of homogeneously-safe lambda terms}

In this section, we derive a new set of rules by adding the type-homogeneity restriction to the non-homogenous safe lambda calculus.

We recall the definition of type-homogeneity from section
\ref{sec:safe_homog}: a type $(A_1, A_2, \ldots A_n, o)$ is said to
be homogeneous whenever $\ord{A_1} \geq \ord{A_2} \geq \ldots \geq
\ord{A_n}$ and each of the $A_i$ is homogeneous. A term is said to
be homogeneous if its type is homogeneous.

We now impose type-homogeneity to all the sequents present in the
rules of the safe $\lambda$-calculus: we say that a term is
\emph{homogeneously-safe} if there is a proof tree showing its
safety in which all sequents are of homogenous type. Consequently a
homogeneously-safe term is safe and has an homogenous type.

We say that $\Gamma \vdash M : A$ verifies $P_i$ for $i \in \zset$
if all the variables in $\Gamma$ have order at least $\ord{A}+i$.
Lemma \ref{lem:nonhomosafe_basic_prop} can then be restated as
follows:
\begin{lemma}[Context reduction]
\label{lem:context_reduction} If $\Gamma \vdash M : A$ then the sequent $fv(M) \vdash M : A$ is valid and satisfies $P_0$.
\end{lemma}


We now prove that if we impose the homogeneity of types, the set of
rules of the non-homogenous safe $\lambda$-calculus and the rules of
table \ref{tab:homosafelmd_rules_refined} are equivalent.  We recall
that in the system of rules of table
\ref{tab:homosafelmd_rules_refined}, if the sequent $\Gamma
\vdash^{i} M : A$ is valid for some $i \in \zset$ then all the
variables in $\Gamma$ have orders at least $\ord{A}+i$.

\begin{proposition}[Homogeneity restriction]
\label{prop:nonhomogsafe_homog_restriction}
Let $k \in \{ 0, -1 \}$. The sequent $\Gamma \vdash M : A$ is valid, homogeneously-safe and satisfies $P_k$
if and only if the sequent $\Gamma \vdash^k M : A$ is valid in the system of rules of table \ref{tab:homosafelmd_rules_refined}.
\end{proposition}

\begin{proof}
\emph{If}: An easy induction by case analysis on the last rule used to derive $\Gamma \vdash^0 M : A$.

\emph{Only if}:
Consider an homogeneously-safe term $\Gamma \vdash S : T$ satisfying $P_0$.
We proceed by induction and case analysis on the last rule used to derive $\Gamma \vdash S : T$.
We only give the details for the application and abstraction
case:
\begin{itemize}
\item \textbf{Abstraction.} We recall the abstraction rule:
$$ \rulename{abs} \quad  \rulef{\Gamma \union \overline{x} : \overline{A} \vdash M : B}
                                   {\Gamma  \vdash \lambda \overline{x} : \overline{A} . M : (\overline{A},B)} \qquad
                                   \forall y \in \Gamma : \ord{y} \geq \ord{\overline{A},B}$$

Type homogeneity requires that for all $i$: $\ord{x_i} = \ord{A_i} \geq
\ord{B} -1$. Therefore the premise of the rule verifies $P_{-1}$. Using the induction hypothesis we have:
\begin{equation}
\Gamma \union \overline{x} : \overline{A} \vdash^{-1} M : B. \label{eq:prop:nonhomogsafe_homog_restriction:abs1}
\end{equation}

We now partition the context $\Gamma$ according to the order of
the variables. The partitions are written in decreasing order of
type order. The notation $\Gamma | \overline{x}:\overline{A}$ means
that $\overline{x}:\overline{A}$ is the lowest partition of the
context.
We also use the notation $(\overline{A}|B)$ to denote the
homogeneous type $(A_1, A_2, \ldots A_n, B)$ where $\ord{A_1} =
\ord{A_2} =  \ldots \ord{A_n} \geq \ord{B} -1$.


Suppose that we abstract a single variable $x$, then in order to
respect the side condition, we need to abstract all variables of
order less or equal to $\ord{x}$. In particular we need to abstract
the partition of the order of $x$. Moreover to respect type
homogeneity, we need to abstract variables of the lowest order
first.

Hence $\overline{x}$ must contain at least the lowest variable
partition (all the variables of the lowest order). If $\overline{x}$
contains variables of different order, then the instance of the
abstraction rule can be replaced by consecutive instances of the
abstraction rule, one for each of the different variable order in
$\overline{x}$. Therefore, without loss of generality, we can assume
that $\overline{x}$ only contains the lowest partition, that is to
say, $\overline{x}$ \emph{is} the lowest partition.

The sequent \ref{eq:prop:nonhomogsafe_homog_restriction:abs1} therefore becomes:
$$\Gamma | \overline{x} : \overline{A} \vdash^{-1} M : B.$$

We conclude by applying the abstraction rule of table
\ref{tab:homosafelmd_rules_refined}:
$$ \rulename{abs} \quad  \rulef{\Gamma| \overline{x} : \overline{A} \vdash^{-1} M : B}
                                   {\Gamma  \vdash^{0} \lambda \overline{x} : \overline{A} . M : (\overline{A}|B)}$$



\item \textbf{Application.} We recall the application rule:
$$ \rulename{app} \  \rulef{\Gamma \vdash M : (A,\ldots,A_l,B)
                                        \qquad \Gamma \vdash N_1 : A_1
                                        \quad \ldots \quad \Gamma \vdash N_l : A_l  }
                                   {\Gamma  \vdash M N_1 \ldots N_l : B}
                                    \quad
                                   \forall y \in \Gamma : \ord{y} \geq \ord{B}$$

The term in the conclusion is homogeneously safe therefore the term in the first premise must be of homogeneous \
type. This implies that $\ord{A_1} \geq \ldots \geq \ord{A_l}
\geq \ord{B} - 1$.
Furthermore, we can make the assumption that $\ord{A_1} = \ldots = \ord{A_l} = \ord{\overline{A}}$
(it is always possible to replace an instance of the application rule
by several consecutive instances of this kind).

By lemma \ref{lem:context_reduction}, we have for all $i \in 1..l$:
$$fv(N_i) \vdash N_i : A_i \mbox{ is valid and satisfies } P_0.$$

Let $\Sigma = \Union_{i=1..p} fv(N_i)$. Since $\ord{A_1} = \ldots = \ord{A_l}$, by applying the weakening rule we get for all $i\in 1..p$:
$$\Sigma \vdash N_i : A_i \mbox{ is valid and satisfies } P_0.$$


Applying lemma \ref{lem:context_reduction} to the term $M$ we have:
$$fv(M) \vdash M : (A_1,\ldots,A_l,B) \mbox{ is valid and satisfies } P_0.$$

The weakening rule $\rulename{wk}$ then gives:
$fv(M) \union \Sigma \vdash M : (A_1,\ldots,A_l,B)$.
Since $\Sigma \vdash N_i : A_i$ satisfies $P_0$, for any
$z \in \Sigma$ we have $\ord{z} \geq \ord{A_i} = \ord{(A_1,\ldots,A_l,B)} - 1$.
Hence:
\begin{equation}
fv(M) \union \Sigma \vdash M : (A_1,\ldots,A_l,B) \mbox{ is valid
and satisfies } P_{-1}
\label{eq:prop:nonhomogsafe_homog_restriction:m}.
\end{equation}

Similarly, for all $i \in 1..p$, the weakening rule gives $fv(M) \union \Sigma \vdash N_i : A_i$.
Since $fv(M) \vdash M : (A_1,\ldots,A_l,B)$ satisfies $P_0$,
for any $z \in fv(M)$ we have $\ord{z} \geq \ord{M} \geq \ord{A_i}$. Hence:
\begin{equation}
fv(M) \union \Sigma \vdash N_i : A_i \mbox{ is valid and satisfies }
P_0 \label{eq:prop:nonhomogsafe_homog_restriction:ni}.
\end{equation}

Let us define the context $\Sigma' = fv(M) \union \Sigma$. Using the induction hypothesis on equation
\ref{eq:prop:nonhomogsafe_homog_restriction:m} and \ref{eq:prop:nonhomogsafe_homog_restriction:ni} we have:
$$
\Sigma' \vdash^{-1} M : (A_1,\ldots,A_l,B) \qquad \mbox{and} \qquad
\Sigma' \vdash^0 N_i : A_i \mbox{ for all } i \in 1..l.
$$


We consider the following two sub-cases:
\begin{itemize}
\item If $A_1, \ldots, A_l$ forms a type partition then we can apply
rule $\rulename{app}$ of table \ref{tab:homosafelmd_rules_refined}:

$$ \rulef{\Sigma' \vdash^{-1} M : \overline{A} | B
                                        \qquad \Sigma' \vdash^{0} N_1 :
                                        A_1
                                        \quad \ldots \quad \Sigma' \vdash^{0} N_l :
                                        A_l
                                        \quad l = |\overline{A}|
                                        }
                                   {\Sigma'  \vdash^{0} M N_1 \ldots N_l : B} \quad  \rulename{app}
$$
where $\overline{A} = A_1, \ldots, A_l$.

\item  Suppose that $A_1, \ldots, A_l$ does not form a type partition, then we
have $$\ord{A_1} = \ldots = \ord{A_l} = \ord{B} - 1.$$

The side condition in the original instance of the application rule
says that for any variable $y$ in $\Gamma$ we have
$$\ord{y} \geq \ord{B} = 1 + \ord{A_l} = \ord{(A_1,\ldots, A_l,B)} = \ord{M}.$$

In particular the variables in $\Sigma' \subseteq \Gamma$ are of order greater than $\ord{M}$ and consequently
the sequent $\Sigma' \vdash M : (A,\ldots,A_l,B)$ verifies $P_0$. The induction hypothesis then gives:
$$\Sigma' \vdash^0 M : (A,\ldots,A_l,B)$$

By using $l$ consecutive instances of the rules $\rulename{app^+}$ from table \ref{tab:homosafelmd_rules_refined} we get:
$$  \rulef{ \rulef{ \rulef{ \Sigma' \vdash^0 M : (A_1,\ldots, A_l,B)
                    \qquad \Sigma'\vdash^{0} N_1 : A_1
                    }{ \Sigma' \vdash^0 M N_1 : (A_2,\ldots, A_l,B)} \quad \rulename{app^+}
          }
          { \vdots
          }
          \quad \rulename{app^+}
       }
       { \Sigma'  \vdash^{0} M N_1 \ldots N_l : B } \quad \rulename{app^+}
$$
\end{itemize}

In both cases we have proved that $\Sigma'  \vdash^{0} M N_1 \ldots N_l : B$ is a valid sequent.

Clearly $\Sigma' \subseteq \Gamma$ since $fv(M) \subseteq \Gamma$ and $\Sigma' = \Union_{i\in1..l} fv(N_i) \subseteq \Gamma$.
Suppose that $\Sigma' = \Gamma$ then the proof is done.
Suppose that $\Sigma' \subset \Gamma$, then the side condition in the original instance of the application rule says that all
the variables in $\Gamma$ have order
greater or equal to $\ord{B}$, we can therefore apply the weakening rule $\rulename{wk^0}$
of table \ref{tab:homosafelmd_rules_refined} exactly $|\Gamma\setminus \Sigma'|$ times and get:
$$ \rulef{\Sigma'  \vdash^{0} M N_1 \ldots N_l : B}
                                   {\Gamma  \vdash^{0} M N_1 \ldots N_l : B} \quad
                                   \rulename{wk^0}.
$$


\end{itemize}
\end{proof}


\subsection{Examples}
\subsubsection{Example 1}
Let $f,g:o\rightarrow o$, $x,y:o\rightarrow o$, $\Gamma =
g:o\rightarrow o$ and $\Gamma' = g:o\rightarrow o, y:o$. The term
$(\lambda f x . x) g y $ is safe. One possible proof tree is:
$$ \rulef{
        \rulef{
            \rulef{
                \rulef{\vdots}{\Gamma \vdash \lambda f x. x}      \qquad \axiomf{\Gamma \vdash g} }
            {\Gamma \vdash (\lambda f x. x) g} \rulename{app}
        }
        { \Gamma' \vdash (\lambda f x. x) g } \rulename{wk}
        \qquad \axiomf{\Gamma' \vdash y}
    }
    { \Gamma' \vdash (\lambda f x. x) g y } \rulename{app}
$$
Here is another proof for the same judgment:
$$ \rulef{  \rulef{ \rulef{\vdots}{\Gamma \vdash \lambda f x. x} }{\Gamma' \vdash \lambda f x. x} \rulename{wk}    \qquad \rulef{}{\Gamma' \vdash g} \qquad \rulef{}{\Gamma' \vdash y}}
    {\Gamma' \vdash (\lambda f x. x) g y } \rulename{app}$$

We see on this particular example that there may exist different
proof trees deriving the same judgment.

\subsubsection{Example 2 - Damien Sereni's SCT counter-example}
In \cite{serenistypesct05}, the following counter-example is given
to show that not all simply-typed terms are size-change terminating
(see \cite{jones01} for a definition of size-change termination):

$$ E =  (\lambda a . a (\lambda b . a (\lambda c d .d))) (\lambda e . e (\lambda f .f))$$
where:
\begin{eqnarray*}
a &:& \sigma \typear \mu \typear \mu \\
b &:& \tau \typear \tau \\
c &:& \tau \typear \tau \\
d &:& \mu \\
e &:& \sigma = (\tau \typear \tau) \typear \mu \typear \mu \\
f &:& \tau
\end{eqnarray*}
and $\tau$, $\mu$ and $\sigma$ are type variables.

This example shows that the rules of the safe $\lambda$-calculus
without the homogeneity restriction generates a class of terms that
strictly contains the class of terms generated by the rules of the
homogeneous safe $\lambda$-calculus of section \ref{sec:safe_homog}.

Indeed, for $E$ to be an homogeneous safe lambda term, in the sense
of the rules of section \ref{sec:safe_homog}, $\tau$ and $\mu$ must
be homogeneous types and the variables $a,b,c,d,e,f$ must be
homogeneously typed. This implies that $ \ord{\tau} \geq
\ord{\mu}-1$. Conversely, if this condition is met then $\vdash E :
\mu \typear \mu$ is a valid judgement of the \emph{homogeneous} safe $\lambda$-calculus.

In the safe $\lambda$-calculus \emph{without} the homogeneity
constraint, however, the judgement $\vdash E : \mu \typear \mu$ is
always valid whatever the types $\mu$ and $\tau$ are.
