\section{Higher-order grammars, safety and higher-order PDAs}
% preliminaries: higher-order grammars and higher-order pushdown automata, and the result in KNU??

In this section we introduce higher-order grammars (HOGs) and
higher-order pushdown automata (HOPDAs). Both devices have
appeared previously in the literature \cite{Dam82, DG86, KNU02,
Eng91}, however, under two sets of definitions.

In \cite{Dam82, DG86, Eng91}, HOGs and HOPDAs are introduced as
generators and acceptors of \emph{string languages} respectively.
Thus each HOG (HOPDA) generates (accepts) a language of strings. In contrast, Knapik \emph{et al.} \cite{KNU01, KNU02} introduce
HOGs and HOPDAS as generators and acceptors of \emph{term trees}
respectively. In particular, each HOG (HOPDA) generates (accepts)
exactly one term tree.

As stated in the introduction, our result concerns only the
string-language setting. Therefore, unless otherwise stated, we
will understand higher-order grammars and higher-order pushdown
automata to be definitional devices for string languages. However,
in Section 7 we will relate our result to the term-tree setting of
\cite{KNU01, KNU02}.

%\begin{remark} \label{stringsvtrees} Although the alternative
%definition of higher-order grammars and higher-order pushdown automata
%does not feature much in this paper, we will make references to it in
%future sections.  In particular, we will have to compare the two
%definitions. When such a situation arises, we will add the subscript
%$s$ or $t$ when we want to make it absolutely clear which set of
%definitions we are employing. Thus, we write $HOG_s$ and $HOPDA_s$
%when these devices are used for defining languages of
%strings. Similarly, we write $HOG_t$ and $HOPDA_t$ when these devices
%are used for defining term trees.
%\end{remark}

\subsection{Safe $\lambda$-Calculus}
We shall consider \emph{simple types} (or just \emph{types} for
short) as defined by the grammar $A \, ::= \, o \; | \; A \funsp
A$. Each type $A$, other than the ground type $o$, can be uniquely
written as $(A_1, \cdots, A_n, o)$ for some $n \geq 1$, which is a
shorthand for $A_1 \funsp \cdots \funsp A_n \funsp o$ (by
convention, $\rightarrow$ associates to the right). We define the
\emph{level}\footnote{This is sometimes referred to as
\emph{order} in the literature.} of a type by $\level{o} = 0$ and
$\level{A \funsp B} = \max(\level{A}+1, \level{B}).$ In the
following we shall consider terms-in-context $\seq{\Gamma}{M : A}$
of the simply-typed $\lambda$-calculus. Let $\Delta$ be a simply-typed
alphabet i.e., each symbol in $\Delta$ has a simple type. We
write $\terms{A}{\Delta}$ for the set of terms of type $A$ built
up from the set $\Delta$ understood as constant symbols,
\emph{without} using $\lambda$-abstraction.

Following \cite{KNU02}, we say that $o$ is \textbfit{homogeneous},
and for $n \geq 1$, $A = (A_1, \cdots, A_n, o)$ is
\textbfit{homogeneous} just if $\level{A_1} \geq \level{A_2} \geq
\cdots \geq \level{A_n}$, and each $A_i$ is homogeneous. Assuming
$A = (\underbrace{A_{11}, \cdots, A_{1l_1}}_{\overline{A_1}},
\cdots, \underbrace{A_{r1}, \cdots, A_{rl_r}}_{\overline{A_r}},
o)$ is homogeneous, we write
\[A \; = \;
(\overline{A_1} \, | \, \cdots \, | \, \overline{A_r} \, | \, o)\]
to mean: all types in each partition (or sequence) $\overline{A_i}
= A_{i1}, \cdots, A_{il_i}$ have the same level, and $i < j \iff
\level{A_{ia}}> \level{A_{jb}}$. Thus the notation organises the
$A_{ij}$s into partitions according to their levels. Suppose $B =
(\overline{B_1} \, | \, \cdots \, | \, \overline{B_m} \, | \, o)$.
We write $(\overline{A_1} \, | \, \cdots \, | \, \overline{A_n} \,
| \, {B})$ to mean \[(\overline{A_1} \, | \, \cdots \, | \,
\overline{A_n} \, | \, \overline{B_1} \, | \, \cdots \, | \,
\overline{B_m} \, | \, o).\]

The \textbfit{Safe $\lambda$-Calculus} is a sub-system of the
simply-typed $\lambda$-calculus. Typing judgements (or
terms-in-context) are of the form
\begin{equation}
\nonumber \seq{\overline{x_1}:\overline{A_1} \, | \, \cdots \, | \,
\overline{x_n} :  \overline{A_n}}{M : B} 
\end{equation}
which is shorthand for
$\seq{x_{11} : A_{11}, \cdots, x_{1r}: A_{1r}, \cdots}{M : B}$. (We shall see shortly
that for valid such judgements, $(\overline{A_1} \, | \, \cdots \,
| \, \overline{A_n} \, | \, B)$ is a homogeneous type.)
\emph{Valid typing judgements} of the system are defined by
induction over the following rules, where $\Delta$ is a given
homogeneously-typed alphabet:
\[{{(\overline{A_1}
\, | \, \cdots \, | \, \overline{A_n} \, | \, B) \hbox{
homogeneous} \qquad {b : B} \; \in \; \Delta} \over
{\seq{\overline{x_1} : \overline{A_1}\, | \, \cdots\, | \,
\overline{x_n} : \overline{A_n}}{b : B}}}\]
\[{{(\overline{A_1}
\, | \, \cdots \, | \, \overline{A_n} \, | \, A_{ni}) \hbox{
homogeneous}} \over {\seq{\overline{x_1} : \overline{A_1}\, | \,
\cdots\, | \, \overline{x_n} : \overline{A_n}}{x_{ni} :
A_{ni}}}}\]
\[
{
{\seq{\overline{x_1} :
\overline{A_1}\, | \, \cdots\, | \, \overline{x_{n+1}} : \overline{A_{n+1}}}{M : B}}
\over
{\seq{\overline{x_1} :
\overline{A_1}\, | \, \cdots\, | \, \overline{x_{n}} : \overline{A_{n}}}{\lterm{\overline{x_{n+1}} : \overline{A_{n+1}}}{M} : (\overline{A_{n+1}} \, | \, B)}}
}
\]
\[{\seq{\Gamma}{M : (\overline{B_1} \, | \, \cdots \, | \, \overline{B_m} \, | \, o)} \qquad
\seq{\Gamma}{N_1 : B_{11}} \quad \cdots \quad \seq{\Gamma}{N_{l_1}
: B_{1l_1}} }\over{ \seq{\Gamma}{M N_1 \cdots N_{l_1} :
(\overline{B_2} \, | \, \cdots \, | \, \overline{B_m} \, | \, o)}}
\] In the safe $\lambda$-calculus, when constructing
$\lambda$-abstractions, \emph{all variables} of the relevant
type-partition must be abstracted; when constructing applications,
the operator-term must be applied to \emph{all operand-terms} (one
for each type) of the relevant type-partition. For example $\seq{F
: ((o, o), o, o, o), \varphi : (o, o), x : o}{F \varphi x : (o,
o)}$ is not safe; it follows that \[\seq{F : ((o, o), o, o, o),
\varphi : (o, o), x : o, y : o}{ F(F \varphi x)xy : o}\] is not
safe. But $\seq{F : ((o, o), o, o, o), \varphi : (o, o)}{F \varphi
a: (o, o)}$ is safe for some constant $a$, and so is $\seq{F :
((o, o), o, o, o), \varphi : (o, o), x : o, y : o}{F \varphi x y:
o}$.

In the following, whenever it is clear from the context what the
type of a term $M$ is, we shall write $\level{M}$ to mean the
level of that type.
\begin{lemma}\label{lem:safe}
Suppose $\seq{\overline{x_1} :
\overline{A_1}\, | \, \cdots\, | \, \overline{x_n} : \overline{A_n}}{M
: B}$ is valid, where $B = (\overline{B_1} \, | \,
\cdots \, | \, \overline{B_m} \, | \, o)$.
\begin{itemize}
\item[(i)] $(\overline{A_1} \, | \, \cdots \, | \, \overline{A_n}
\, | \, \overline{B_1} \, | \, \cdots \, | \, \overline{B_m} \, |
\, o)$ is homogeneous. \item[(ii)] Any free variable of $M$ has
level at least $\level{M}$. \item[(iii)] For any subterm
$\lterm{\Phi}{L}$ of $M$, if the variable $\varphi$ occurs in $L$
and $\level{\varphi} < \level{\Phi}$ then $\varphi$ is bound in
$L$. \myendproof
\end{itemize}
\end{lemma}

We omit the straightforward proofs.

\subsubsection*{What does ``safe'' mean?}

Substitution is a fundamental operation in the $\lambda$-calculus.
In the key clause of the definition
\[ (\lterm{x}{M})[N / y] \; \defined \; \lterm{z}{((M[z / x]) [N / y])} \quad
\hbox{where ``$z$ is fresh''}\] bound variables are renamed afresh to
prevent variable capture. In the {safe $\lambda$-calculus}, one can
get away without any renaming.

\begin{lemma}In the safe $\lambda$-calculus, there is no need to
rename bound variables afresh when performing substitution
\[M[N_1 / x_1 , \cdots, N_n / x_n]\]
provided the substitution is performed simultaneously on
\emph{all} free variables of the same level in $M$
i.e.~$\makeset{x_1, \cdots, x_n}$ is the set variables of the
same level as $x_1$ that occur free in $M$.
\end{lemma}

\begin{proof} Suppose $\varphi$ occurs free in $M$, and
bound variables in $M$ are not renamed in the substitution $M[N /
\varphi]$. Further suppose $x$, a variable occurring free in $N$,
is captured as a result of the substitution. I.e.~there is a
subterm $\lterm{x}{L}$ of $M$ such that $\varphi$ occurs free in
$L$. We compare $\level{x}$ with $\level{\varphi}$:

\begin{itemize}
\item {Case 1}: $\level{x} > \level{\varphi}$.

This is impossible: Since $\lterm{x}{L}$ is safe, by
Lemma~\ref{lem:safe}(iii), $L$ can have no free variables of level
less than $\level{x}$.

\item {Case 2}: $\level{x} < \level{\varphi}$.

This is impossible: Since $N$ is safe and of level
$\level{\varphi}$, by Lemma~\ref{lem:safe}(ii), it can have no
free variable of level less than $\level{\varphi}$.

\item {Case 3}: $\level{x} = \level{\varphi}$.

If follows from the formation rule for $\lambda$-abstraction that
$\varphi$ cannot occur free in $M$ since the subterm
$\lterm{x}{L}$ must be in the scope of some subterm
$\lterm{\varphi}{\cdots}$ of $M$, so that $\varphi$ does not occur free
in $M$. Thus this case cannot arise either.
\end{itemize}

\end{proof}

\subsection{Higher-order grammars and the OI-hierarchy}

A \textbfit{higher-order grammar} (or HOG, for short) is a
five-tuple $G = \anglebra{N, V, \Sigma, {\cal R}, S, e}$ such that
\begin{itemize}
\item[(i)] $N$ is a finite set of homogeneously-typed
non-terminals, and $S$, the \emph{start symbol}, is a
distinguished element of $N$ of type $o$

\item[(ii)] $V$ is a finite set of typed variables

\item[(iii)] $\Sigma$ is a finite alphabet

\item[(iv)] $\cal R$ is a finite set of triples, called
\emph{rewrite rules} (also referred to as production rules), of
the form
\[F x_1 \cdots x_n \; \larr{\alpha} \; E\]
where $\alpha \in (\Sigma \cup \makeset{\epsilon})$, ${F : (A_1,
\cdots, A_n, o)} \in N$, each ${x_i : A_i} \in V$, and $E$ is
either a term in ${\cal T}^o(N \cup \{x_1, \cdots, x_m\})$ or is
$e : o$. We say that $F$ has \textbfit{formal parameters} $x_1,
\cdots, x_m$. In the case where there grammar has two or more
rules with the non-terminal $F$ on the lefthand side, then we
assume both rules have the same formal parameters in the same
order. Following \cite{KNU01} we make the assumption that if $F
\in N$ has type $(A_1, \cdots, A_n, o)$ and $n \geq 1$, then $A_n
= o$. Thus, each non-terminal has at least one level-$0$ variable.
Note that this is not really a restriction -- as this variable need
not occur on the righthand side. We also assume that $S$ never occurs
on the righthand side of a rewrite rule. We say that the rewrite rule has
\emph{name} $F$ and \emph{label} $\alpha$; it has level $n$ just
in case the type of its name has level $n$.
\end{itemize}
We say that $G$ is of level $n$ just in case $n$ is the level of
the rewrite rule that has the highest-level. We say that $G$ is
\textbfit{deterministic} if for every $F x_1 \cdots x_n
\larr{\alpha} E$ and $F x_1 \cdots x_n \larr{\alpha'} E'$ in $\cal
R$
%(we identify rewrite rules that are
%$\alpha$-equivalent\footnote{$F x_1 \cdots x_n \larr{\alpha} U$
%and $F y_1 \cdots y_n \larr{\alpha} V$ in are said to be
%$\alpha$-\emph{equivalent} if $U$ and $V[\overline{x_i / y_i}]$
%are syntactically equal.})
\begin{itemize}
\item If $\alpha = \alpha'$ then $E = E'$.

\item If $\alpha = \epsilon$ and $E \not = e$, then $\alpha = \alpha'$
\end{itemize}
We say that a deterministic HOG is \emph{real-time} if no rule has
an $\epsilon$ label.


\subsubsection*{The language of a HOG}
We extend $\cal R$ to a family of binary relations $\larr{\alpha}$
over $\terms{o}{N} \cup \makeset{e}$, where $\alpha$ ranges over
$\Sigma \cup \makeset{\epsilon}$, by the rule: if $F x_1 \cdots
x_n \larr{\alpha} E$ is a rule in $\cal R$ where $x_i : A_i$ then
for each $M_i \in \terms{A_i}{N}$ we have
\[ FM_1 \cdots M_n \larr{\alpha} E\overline{[M_i / x_i]}. \]

A \emph{derivation} of $w \in \Sigma^\ast$ is a sequence $P_1,
P_2, \cdots, P_{n}$ of terms in $\terms{o}{N}$, and a
corresponding sequence $\alpha_1, \cdots, \alpha_n$ of elements in
$\Sigma \cup \makeset{\epsilon}$ such that
\[S = P_1 \larr{\alpha_1} P_2 \larr{\alpha_2} P_3 \larr{\alpha_3}
\quad \cdots \quad \larr{\alpha_{n-1}} P_{n} \larr{\alpha_n} e
\] and $w = \alpha_1 \cdots \alpha_n$. The \emph{language} generated by
$G$, written $L(G)$, is the set of words over $\Sigma$ that have
derivations in $G$. We say that two grammars are \emph{equivalent}
if they generate the same language.

\subsubsection*{Safe Grammars}

A grammar is \textbfit{safe} if for each rewrite rule
$F x_1 \cdots x_n \larr{\alpha} E$ we have that
\[ \seq{x_1 : A_1, \cdots, x_n : A_n}{E :
o}\]
is a valid typing judgement of the safe $\lambda$-calculus, where $E$ is constructed possibly
using symbols from $N$ as constants. Otherwise, the grammar is \textbfit{unsafe}.

\subsubsection*{The OI-hierarchy}

In \cite{Dam82}, Damm introduced the OI-hierarchy. The $n$th level
of the hierarchy is generated by level-$n$ grammars (defined
differently from our grammars). Furthermore,
each level is strictly contained in the one above it. The first
three levels correspond to the regular, the context-free, and the
indexed languages \cite{Aho68}. Damm's definition of a level-$n$
grammar, although packaged somewhat differently from ours, can be
thought of as a special case of our definition. In particular, it
is routine to show that a level-$n$ grammar using his definition
corresponds to a \emph{safe} level-$n$ grammar in our definition
(and the converse holds too). Thus, Damm's definition is such that
the safety restriction -- referred to as derived types in his
paper -- is always ``built-in''. For a note comparing the two
definitions (ours and Damm's) we point the reader to \cite{dMO}.
This note also motivates our preference for our definition.

The reader familiar with the OI-hierarchy should note that our
derivation relation is constructed so that all derivations are
outside-in (which, in the monadic case, corresponds to a leftmost
derivation). Bearing this in mind, the fact that our definition of
a safe level-$n$ grammar coincides with Damm's definition of a
level-$n$ grammar should be almost immediate.

\begin{remark}
We note there is nothing to be gained by allowing the rhs of rules
to be $\lambda$-terms, since any such rule can be transformed to
an equivalent system of rules, all of whose rhs are applicative
terms. E.g.~the level-$3$ rule
\[F \varphi \; \larr{\alpha} \; \varphi (\lterm{x}{ \varphi(\lterm{y}{x})})\]
is equivalent to the following system of rules
\[\begin{array}{rcl}
F \, \varphi & \larr{\alpha} & \varphi \, (G \varphi) \\
G \, \varphi \, x & \larr{\epsilon} & \varphi \, (H x) \\
H \, x \, y & \larr{\epsilon} & x \\
\end{array}\]
\end{remark}

%\begin{remark}\label{rem:otherdefinitions}
%We relate our definition to the various versions of higher-order
%grammar or recursion schemes in the literature.
%
%\medskip
%
%\noindent\emph{(i)} First we note there is nothing to be gained by
%allowing the rhs of rules to be $\lambda$-terms, since any such
%rule can be transformed to an equivalent system of rules, all of
%whose rhs are applicative terms. E.g.~the level-$3$ rule
%\[F \varphi \; \larr{a} \; \varphi (\lterm{x}{ \varphi(\lterm{y}{x})})\]
%is equivalent to the following system of rules
%\[\begin{array}{rcl}
%F \, \varphi & \larr{a} & \varphi \, (G \varphi) \\
%G \, \varphi \, x & \larr{\epsilon} & \varphi \, (H x) \\
%H \, x \, y & \larr{\epsilon} & x \\
%\end{array}\]
%
%\noindent\emph{(ii)} In \cite{KNU02} Knapik \emph{et al.} have
%introduced (deterministic) \emph{level-$n$ grammars} for
%generating $\Sigma$-trees (rather than languages), where $\Sigma$
%is a finite alphabet of typed constants of level at most one. The
%trees that are generated are just the syntax trees of potentially
%infinite applicative terms built up from $\Sigma$: all nodes of
%such a tree are labelled by constants from $\Sigma$, and if a node
%is labelled by a constant $f : o^n \funsp o$ then it has exactly
%$n$ descendants. In Section 7 we will show that we can
%systematically transform such a grammar $G$ (say) into one of ours
%$G'$ (say) in such a way that the language of $G'$ is
%essentially\footnote{but not quite.} the \emph{branch language}
%(in the sense of \cite{Cou83}) of the term-tree determined by the
%grammar $G$. We will also examine the relationship between
%string-language generating grammars (as introduced here) and
%term-tree generating grammars with some pleasing results.
%
%%In their approach, rules are not
%%labelled by symbols (as we do), instead the rhs of each rule is an
%%applicative term that may contain occurrences of $\Sigma$-symbols.  We
%%can systematically transform such a grammar $G$ (say) into one of ours
%%$G'$ (say): the constants that occur in the rhs of $G$-rules determine
%%the symbols that label the $G'$-rules in such a way that the language
%%generated by the transform $G'$ is essentially the \emph{branch
%%language} (in the sense of \cite{Cou83}) of the term-tree determined
%%by the grammar $G$. E.g.~the level-2 grammar in the sense of Knapik
%%\emph{et al.}
%%\[\begin{array}{rll}
%%F \, \varphi \, x & \larr{} & f(F (f(\varphi \, x)) x)\,(\varphi \, x)\\
%%S & \larr{} & F \, g \, a \\
%%\end{array}\]
%%with constants $f : (o, o, o), g : (o, o)$ and $a : o$ can be transformed to
%%the following system of labelled rules:
%%\[\begin{array}{rlllrll}
%%F \, \varphi \, x & \larr{f.1} & F (H (\varphi \, x)) x & \qquad & S & \larr{\epsilon} & F \, G \, A \\
%%F \, \varphi \, x & \larr{f.2} & \varphi \, x & & G \, x & \larr{g.1} & x \\
%%H \, y \, z & \larr{f.1} & y & & A & \larr{a.0} & \epsilon\\
%%H \, y \, z & \larr{f.2} & z\\
%%\end{array}\]
%
%\medskip
%
%\noindent\emph{(iii)} We should mention another, older, definition
%of level-$n$ grammar, which constitutes the \emph{OI-hierarchy} of
%Damm (see \cite{DG86, Dam82}). Damm's definition, although
%packaged differently from ours, can be thought of as a special
%case of ours. In particular, it is routine to show that a
%level-$n$ grammar using his definition corresponds to a
%\emph{safe} level-$n$ grammar in our definition (and the converse
%holds too). Thus, Damm's definition is such that the safety
%restriction (referred to as derived types) is always ``built-in''.
%For a note comparing the two definitions (ours and Damm's) we
%point the reader to \cite{dMO}. This article also motivates our
%preference for our definition.
%
%\medskip
%
%\noindent\emph{(iv)} The reader familiar with the OI-hierarchy
%\cite{Dam82} should note that our derivation relation is
%constructed so that all derivations are outside-in (which, in the
%monadic case, corresponds to a leftmost derivation).
%\end{remark}

\begin{example}\label{ex:ex1}
Consider the following \emph{deterministic} grammar, where
\[\Sigma = \makeset{h_1, h_2, h_3, f_1, f_2, g_1, a, b},\]
the typed non-terminals are
\[D : ((o, o), o, o, o), \quad H : ((o, o), o, o), \quad
F: (o, o, o), \quad G : (o, o), \quad A, B : o\] with rules:
\begin{equation}
\begin{array}{cc}
\begin{array}{rcl}
\nonumber S & \larr{\epsilon} & DGAB\\
\nonumber D\varphi x y & \larr{h_1} & D (\underline{D \varphi x}) y (\varphi y) \\
\nonumber D\varphi x y & \larr{h_2} & H (\underline{F y}) x \\
\nonumber D\varphi x y & \larr{h_3} & \varphi B \\
\nonumber H \varphi x & \larr{\epsilon} & \varphi x
\end{array} &
\begin{array}{rcl}
\nonumber G x & \larr{g_1} & x \\
\nonumber F x y & \larr{f_1} & x \\
\nonumber F x y & \larr{f_2} & y \\
\nonumber A & \larr{a} & e\\
\nonumber B & \larr{b} & e
\end{array}
\end{array}
\end{equation}
This is an unsafe grammar because of the underlined expressions:
both of which are level-$1$ terms, but contain level-$0$
variables.

As this grammar is deterministic \cite{dMO} this means that each
word in the language is generated in a unique way. Hence, it
should be easy for the reader to check by hand that the words
$h_1h_3h_2f_1b$ and $h_1h_3h_2f_2a$ are part of the language,
whereas $h_1h_3h_2f_1a$ is not.
\end{example}

\subsection{Pointer machines}
Pointer machines are a model of computation for string languages
generated by higher-order grammars. The \textbfit{pointer machine}
for an $n$-grammar $G = \anglebra{N, V, \Sigma, {\cal R}, S, e}$ has
a (pushdown) stack.  A stack $\beta$ is a non-empty sequence
$\mkstore{a_1, a_2, \cdots, a_n}$ where each $a_i \in \Gamma$, a
set determined by $G$. The following operations are defined on a
stack: for $a \in \Gamma$
\[\begin{array}{rll}
\head{\mkstore{a_1, a_2, \cdots, a_n}} & = & a_1 \\
\tail{\mkstore{a_1, a_2, \cdots, a_n}} & = & \mkstore{a_2, \cdots, a_n}\\
a : \mkstore{a_1, a_2, \cdots, a_n} & = & \mkstore{a, a_1, a_2, \cdots, a_n}.
\end{array}\]
The stack alphabet $\Gamma$ is a certain finite subset of
$\terms{o}{N \cup V} \cup \{e\}$, which will be defined shortly.
Each $a_i$, which we shall refer to as an \textbfit{item}, will
have exactly one pointer for each variable (from $V$) that occurs
in it. Thus an item in the stack may have several pointers
emanating from it (apart from $a_n$ which has no pointers).
Intuitively, a pointer is a physical link connecting an item $a_i$
to some other item $a_j$ for $j > i$.

%A \textbfit{pointer} is a physical link connecting an item $a_i$ to
%some item $a_j$ for $j > i$. So for example, we could have:
%\begin{equation}
%\nonumber [{\rnode{x}{a_1}}, {\rnode{y}{a_2}}, {\rnode{z}{a_3}},
%{\rnode{A}{a_4}}, {\rnode{B}{a_5}}] \ncarc[arcangle=90]{->}{x}{A}
%\end{equation}
%In this example there is a pointer from $a_1$ to $a_4$. As we will
%see, the alphabet $\Gamma$ will be a finite subset of .

We define the \textbfit{stack transition relation}. The behaviour
of a pointer machine is given in terms of a labelled transition
relation between stacks. A computation of a pointer machine begins
with the stack containing a single item, which is the start symbol
$S$. Thereafter, let $\beta$ be the stack. We use meta-variables
$F$ and $x$ to range over $N$ (non-terminals) and $V$ (variables)
respectively.
\begin{itemize}
\item[1.] Assume $\head{\beta} = F t_1 \cdots t_n$. If $F x_1
\cdots  x_n \; \larr{\alpha} \; E$ is a $G$-rule then
\[\beta \; \larr{\alpha} \; E: \beta \]
and each occurrence of a variable $x_j$ in $E$ points to
$\head{\beta}$, in other words $F t_1 \cdots t_n$. Further, the
pointer structure of $\beta$ is preserved.

\item[2.] Assume $\head\beta = x  t_1  \cdots  t_n$ and
$x$ points to an item $a = D s_1 \cdots s_m$
in $\beta$. Furthermore, suppose that $x$ is the $i$th formal
parameter of $D$. Then
\[ \beta \; \larr{\epsilon} \; s_i t_1  \cdots  t_n : \tail\beta\]
all pointers in $\tail\beta$ are preserved and all pointers from $t_1,
\cdots, t_n$, as well as those from $s_i$, are preserved.

%\item[3.] Assume $\head\beta = f t_1 \cdots t_n$ where
%$\arity{f} > 0$. Then \[ \beta \; \stackrel{f_i}{\rightarrow} \; t_i
%: \tail\beta\] and the pointer structure of $\beta$ is
%preserved, as are any pointers emanating from $t_i$.

\item[3.] Assume $\head\beta = e$, then the pointer machine halts.
\end{itemize}

Let $\stackl S \stackr \larr{\alpha_1} P_1 \larr{\alpha_2} \cdots
\larr{\alpha_{n-1}} P_n$ be a transition sequence of pointer machine
stacks such that $\head P_n = e$. Then, we say that the word,
$\alpha_1 \alpha_2 \cdots \alpha _{n-1}$ (over $\Sigma$), is \textbfit{accepted} by
the pointer machine. We say that a word belongs to the language of
a given pointer machine if and only if there exists a transition
sequence of the pointer machine starting from $\stackl S \stackr$
that results in the word being accepted.

The language accepted by a pointer machine can easily be seen to
be exactly the language of the higher-order grammar. In
particular, it corresponds to linear head reduction in the
$\lambda$-calculus.

\begin{remark}
\begin{enumerate}
\item[(i)] Pointer machines are due to C. Stirling. The second author
first learnt of the idea in \cite{Sti02}.
\item[(ii)] The reader acquainted with \cite{KNU01, KNU02} should
be able to easily modify the definition of a pointer machine to
serve as a model of computation for term-trees generated by
higher-order grammars.
\end{enumerate}
\end{remark}

\subsubsection*{Stack alphabet}
We write $R$ for the set consisting of the rhs of each rule from $\cal
R$.  Let $\xi E_1 \cdots E_m$ be a ground-type term, where $\xi$ is
either a variable or a non-terminal. We define
\[
     \args{\xi E_1 \cdots E_m} \; = \; \makeset{E_1, \cdots, E_m}.
\]
Now define two sets, $\Gamma$ (which is a set of ground-type terms) and
$\exp$, by mutual induction over the following rules:
\begin{center}
\begin{tabular}{p{12cm}}
\prule{}{R \subseteq \Gamma}\\ \prulec{}{\args{r} \subseteq \exp}{$r
\in R$}\\ \prule{E^A \in \exp \quad x^A E_1 \cdots E_m \in \Gamma}{E
\, E_1 \cdots E_m \in \Gamma} \\ \prule{\xi E_1 \cdots E_m \in
\Gamma}{ \args{\xi E_1 \cdots E_m} \subseteq \exp}
\end{tabular}
\end{center}

We can take $\Gamma$ to be the stack alphabet. ($\Gamma$ is a superset
of what we actually need).

\begin{lemma}
$\Gamma$ is finite.
\end{lemma}

\begin{proof}
We define a new set $\Gamma'$ (of ground-type terms) by induction over
the rules: $R \subseteq \Gamma'$ and
\[
{E^A \in \exp' \quad x^A E_1 \cdots E_m \in \Gamma'}\over{E^A E_1
\cdots E_m \in \Gamma'}
\]
where $\exp' = \makeset{ N : \hbox{$N$ is a subterm of some $r \in
R$}}$, which is clearly finite. It is straightforward to see that $\exp
\subseteq \exp'$ and $\Gamma \subseteq \Gamma'$. Therefore it suffices
to prove that $\Gamma'$ is finite. We define a partial order over the
set $\makeset{A : E^A \in \exp' }$ of types by:
\[
    A > B  \; \iff  \; A = A_1 \mor \cdots \mor A_n \mor B
\]
where $n > 0$. Now define a function $H : {\terms{o}{N \cup V}} \mor
\terms{o}{N \cup V}$ as follows:
\[
   H(G) \; = \; R \cup G \cup
      \makeset{ \xi^A U_1\cdots U_n E_1\cdots E_m :
              x^B E_1 \cdots E_m \in G,
              \xi^A U_1\cdots U_n : B \in \exp' }
\]
We observe that $\Gamma'$ is the fixpoint of $H$, which is reached
after at most $j$ iterations of $H$, where $j$ is the length of the
longest chain in the poset $\makeset{ A : E^A \in \exp' }$.
\end{proof}

\subsubsection*{Useful properties}
We consider the operation of a pointer machine. The stack items of
the pointer machine are of two types: \emph{incomplete} and
\emph{complete}. A complete item is one that is headed by a
non-terminal, so for example $F\overline{s}$ is a complete item.
An incomplete item is headed by a variable. We call it incomplete,
because it is, in a sense, work in progress: we will remain at
this stack item (replacing the head variable) until it is finally
headed by a non-terminal, in which case it will become complete.

\begin{lemma}\label{lem:complete}
All items of any reachable pointer machine stack $\beta$ are of
level 0; further, $\tail\beta$ consists only of complete items.
\end{lemma}

\begin{proof} Induction on the number of stack transitions.
\end{proof}

\begin{lemma}\label{lem:pmgroups}
Let $s_0s_1 \cdots s_k$ be an item in a reachable pointer machine
stack $\beta$. Then the following hold:
\begin{enumerate}
\item[(i)] If $x$, a variable occurring in $s_0s_1 \cdots s_k$, points to a stack item $D t_1 \cdots t_m$ then $x$ is a formal
parameter of $D$.

\item[(ii)] For each $i$, the variables in $s_i$ all point to the same stack item.

\item[(iii)] Suppose $s_i$ has a variable with pointer to an item
$a$, and $s_j$ has a variable with pointer to an item $a'$. If $i
< j$ then either $a$ and $a'$ are the same item or $a$ occurs
deeper in the stack than $a'$.
\end{enumerate}
\end{lemma}

\begin{proof}Induction on the number of stack transitions.\\
\end{proof}

Bearing these two observations in mind, we modify the definition of a
pointer machine slightly to obtain a \textbfit{compacting pointer
machine} for a HOG $G = \anglebra{N, V, \Sigma, {\cal R}, S, e}$. As
before, it has a pushdown stack and a computation of a pointer machine
begins with the stack containing only the start symbol
$S$. Afterwards, the stack transition relation is defined as follows
below. Let $\beta$ be the current stack:
\begin{itemize}
\item[1.] Assume $\head\beta = F t_1 \cdots t_n$. If $F x_1
\cdots  x_n \; \larr{\alpha} \; E$ is a $G$-rule then
\[\beta \; \larr{\alpha} \; E: \beta \]
and each occurrence of a variable $x_j$ in $E$ points to
$\head{\beta}$, in other words $F t_1 \cdots t_n$. Further the pointer
structure of $\beta$ is preserved.

\item[2.]
\begin{itemize}
\item[a.] Assume $\head\beta = x  t_1  \cdots  t_n$ and suppose
that $x$ is a variable of level at least $1$, and points to an
item $a = D s_1 \cdots s_m$ in $\beta$. Furthermore, suppose that
$x$ is the $i$th formal parameter of $D$. Then
\[ \beta \; \larr{\epsilon} \; s_i t_1  \cdots  t_n : \tail\beta\]
all pointers in $\tail\beta$ are preserved and all pointers from
$t_1, \cdots, t_n$, as well as those from $s_i$, are preserved.

\item[b.] Assume $\head \beta = x$, $x$ is of level 0, and points
to an item $\head \beta' = D s_1 \cdots s_m$ where $\beta'$ is a
suffix of $\beta$. Furthermore, suppose that $x$ is the $i$th
formal parameter for $D$. By Lemma ~\ref{lem:pmgroups} we know
that all the variables in $s_i$ (if any) point to the \emph{same}
item in $\beta'$, (say) $\head \beta''$ where $\beta''$ is a
suffix of $\beta'$. Then
\[ \beta \; \larr{\epsilon} \; s_i : {\beta''}\]
all pointers in ${\beta''}$ are preserved as well as those from
$s_i$\footnote{In particular, all pointers from variables in $s_i$ will
be to $\head \beta''$.}. In the case where $s_i$ contains no
variables $\beta''$ may be chosen arbitrarily.
\end{itemize}
%
%\item[3.] Assume $\head\beta = f t_1 \cdots t_n$ where
%$\arity{f} > 0$. By Lemma ~\ref{lem:pmgroups} we know that all the
%variables in $t_i$ point to subterms of the \emph{same} item in
%$\beta$, say this item is $\head{\beta'}$ where $\beta'$ is a suffix
%of $\beta$. Then
%\[ \beta \; \stackrel{f_i}{\rightarrow} \; t_i : {\beta'}\] and the pointer structure of $\beta'$ is
%preserved, as are any pointers emanating from $t_i$.

\item[3.] Assume $\head\beta = e$, then the pointer machine halts.
\end{itemize}

The key to understanding the compacting pointer machine is that it
maintains the following invariant. If $\beta$ is the pointer
stack, with $\head \beta = s_0 s_1 \cdots s_n$, then either 1)
$s_n$ contains at least one variable with a pointer to $\tail
\beta$ or 2) $s_n$ contains only non-terminals. As an example,
suppose that
\begin{equation}
\nonumber \stackl {\rnode{phi}{\varphi}} {\rnode{x1}{x_1}}
{\rnode{x2}{x_2}}, {\rnode{A}{A}}, {}\rnode{B}{B}, {\rnode{C}{C}},
{\rnode{D}{D}}, {\rnode{E}{E}}\stackr \ncarc[arcangle=90]{->}{x2}{C}
\end{equation}
is a pointer stack of the original pointer machine, where
$\varphi$ is a variable of level $1$ and $x_1,x_2$ are variables
of level $0$. Here, $A, B, C, \cdots$  are meta-variables for terms in
${\cal T}^o(N \cup V)$. Only one pointer has been shown: from variable $x_2$
in the topmost item to $C$. Other pointers exist, but have not be
indicated. By Lemma~\ref{lem:pmgroups}(iii) we know that $x_2$ and
$\varphi$ must point to item $C$ or items deeper in the stack,
namely, $D$ or $E$. In this instance the items $A$ and $B$ have
now become redundant: for any stack configuration reachable from
here, the topmost item will never contain a pointer to $A$ or $B$.
Thus, for all intents and purposes, they may as well be cut out,
and removing such redundancies is what the compacting pointer
machine strives to achieve. So the above configuration would
become:
\begin{equation}
\nonumber \stackl {\rnode{phi}{\varphi}} {\rnode{x1}{x_1}}
{\rnode{x2}{x_2}}, {\rnode{C}{C}}, {\rnode{D}{D}}, {\rnode{E}{E}}\stackr
\ncarc[arcangle=90]{->}{x2}{C}
\end{equation}
and all other pointers for remaining items are preserved. Hence
the name compacting\footnote{Note that although the compacting
pointer machine does remove such redundant segments, it is not
optimal; in the sense that some such redundant segments may still
exist. This, however, is deliberate.}.

\begin{lemma} A compacting pointer machine is equivalent to a
pointer machine. \myendproof
\end{lemma}


\subsubsection*{Representation}

This section concerns notation for pointer machine stacks. Recall that
each item in the stack may have several pointers emanating from it. In
particular, suppose we have the following PM stack:
\begin{equation}
\nonumber \stackl
{\rnode{x}{\varphi}}{\rnode{y}{y}}{\rnode{z}{z}},\;
{\rnode{A}{F}}G{\rnode{w}{x}}AB,\; {\rnode{B}{D}}A, \; S \stackr
\ncarc[arcangle=90]{->}{x}{B} \ncarc[arcangle=90]{->}{y}{A}
\ncarc[arcangle=80]{->}{z}{A} \ncarc[arcangle=80]{->}{w}{B}
\end{equation}
Note that $\varphi$ points to $DA$, both $y$ and $z$ point to $FGxAB$,
and $x$ points to $DA$. Rather than resorting to having to draw
arrows we can convey this information by subscripting each
variable with a PM stack. We annotate each variable
with a suffix $s$ of the PM stack, such that the pointer points to
$\head s$. Thus, for the above example we would have:
\begin{equation}
\nonumber \stackl x_{p_{\varphi}}y_{p_y}z_{p_z}, FGx_{p_x}AB, DA, S\stackr \mbox{where}
\end{equation}
\begin{eqnarray}
\nonumber p_{\varphi} & = & \stackl DA, S \stackr \\
\nonumber p_y & = & \stackl FGx_{p_x}AB, DA, S \stackr \\
\nonumber p_z & = & \stackl FGx_{p_x}AB, DA, S \stackr \\
\nonumber p_x & = & \stackl DA, S \stackr
\end{eqnarray}
Recall from Lemma~\ref{lem:pmgroups} that if $s_0s_1 \cdots s_n$
is an item of a PM stack, and if $x, x'$ are variables occurring
in $s_i$ for some $i$, then the pointer of $x$ is the same as the
pointer of $x'$. Hence, to avoid having to subscript \emph{every}
single variable with a PM stack, we adopt following convention:
For $p,q$ pointer stacks and $s, t \in \mathcal{T}(N \cup V) \cup
\{e\}$, we have:
\begin{itemize}
\item $(st)_p = (s_p)(t_p)$ for $(st)$ an application \item
$(s_p)_q = s_p$ for all $s$ \item $F_p = F$ for $F$ a non-terminal
\item $e_p = e$
\end{itemize}
Thus, adopting the above convention, our earlier example could be
rewritten as:
\begin{equation}
\nonumber \stackl (\varphi_{p_{\varphi}}yz)_{p_z}, (FGxAB)_{p_x}, DA, S
\stackr \mbox{ where } p_{\varphi},p_z,p_x \mbox{ are as before}
\end{equation}
Note the $yz$.

%Let us consider a more concrete example, using the grammar for
%Urzyczyn's language. It is easy to see that below is reachable
%pointer machine stack.
%\begin{equation}
%\nonumber \stackl (D\varphi_p x_p) y_q x_q,D(D\varphi_p
%x_p)z_p(Fy_p)(Fy_p),DG\epsilon\epsilon\epsilon, S \stackr
%\end{equation}
%where
%\begin{eqnarray}
%\nonumber q & = & \stackl D(D\varphi_p x_p)z_p(Fy_p)(Fy_p),DG\epsilon\epsilon\epsilon, S \stackr \\
%\nonumber p & = & \stackl DG\epsilon\epsilon\epsilon, S \stackr
%\end{eqnarray}
%However, with using our convention, we can express this more
%succinctly:
%\begin{equation}
%\nonumber \stackl ((D\varphi x)_p y x)_q,(D(D\varphi
%x)z(Fy)(Fy))_p,DG\epsilon\epsilon\epsilon, S \stackr
%\end{equation}
%with $q$ and $p$ as before.

\subsubsection*{An example}

\begin{example} We give an example computation of the (non-compacting) pointer
machine for the grammar in \ref{ex:ex1}.
\begin{eqnarray}
\nonumber & & \stackl S \stackr  \\
\nonumber & \larr{\epsilon} & \stackl DGAB, S \stackr  \\
\nonumber & \larr{h_1} & \stackl (D(D\varphi x) y (\varphi y))_{p_1}, DGAB, S \stackr \\
\nonumber & \larr{h_3} & \stackl (\varphi x)_{p_2},(D(D\varphi x) y (\varphi y))_{p_1}, DGAB, S \stackr \\
\nonumber & \larr{\epsilon} & \stackl (D \varphi x)_{p_1} B,(D(D\varphi x) y (\varphi y))_{p_1}, DGAB, S \stackr \\
\nonumber & \larr{h_2} & \stackl (H(Fy)x)_{p_3},(D \varphi x)_{p_1} B,(D(D\varphi x) y (\varphi y))_{p_1}, DGAB, S \stackr \\
\nonumber & \larr{\epsilon} & \stackl (\varphi x)_{p_4},(H(Fy)x)_{p_3},(D \varphi x)_{p_1} B,(D(D\varphi x) y (\varphi y))_{p_1}, DGAB, S \stackr \\
\nonumber & \larr{\epsilon} & \stackl (Fy)_{p_3} x_{p_4},(H(Fy)x)_{p_3},(D \varphi x)_{p_1} B,(D(D\varphi x) y (\varphi y))_{p_1}, DGAB, S \stackr \\
\nonumber & \larr{f_1} & \stackl x_{p_5},(Fy)_{p_3} x_{p_4},(H(Fy)x)_{p_3},(D \varphi x)_{p_1} B,(D(D\varphi x) y (\varphi y))_{p_1}, DGAB, S \stackr \\
\nonumber & \larr{\epsilon} & \stackl y_{p_3},(Fy)_{p_3} x_{p_4},(H(Fy)x)_{p_3},(D \varphi x)_{p_1} B,(D(D\varphi x) y (\varphi y))_{p_1}, DGAB, S \stackr \\
\nonumber & \larr{\epsilon} & \stackl B,(Fy)_{p_3} x_{p_4},(H(Fy)x)_{p_3},(D \varphi x)_{p_1} B,(D(D\varphi x) y (\varphi y))_{p_1}, DGAB, S \stackr \\
\nonumber & \larr{b} & \stackl e,B,(Fy)_{p_3} x_{p_4},(H(Fy)x)_{p_3},(D \varphi x)_{p_1} B,(D(D\varphi x) y (\varphi y))_{p_1}, DGAB, S \stackr
\end{eqnarray}
where
\begin{eqnarray}
\nonumber p_1 & = & \stackl DGAB, S \stackr \\
\nonumber p_2 & = & \stackl D(D\varphi x) y (\varphi y))_{p_1}, DGAB, S \stackr \\
\nonumber p_3 & = & \stackl (D \varphi x)_{p_1} B,(D(D\varphi x) y (\varphi y))_{p_1}, DGAB, S \stackr \\
\nonumber p_4 & = & \stackl H(Fy)x)_{p_3},(D \varphi x)_{p_1} B,(D(D\varphi x) y (\varphi y))_{p_1}, DGAB, S \stackr \\
\nonumber p_5 & = & \stackl (Fy)_{p_3} x_{p_4},(H(Fy)x)_{p_3},(D \varphi x)_{p_1} B,(D(D\varphi x) y (\varphi y))_{p_1}, DGAB, S \stackr
\end{eqnarray}
Note that the pointer machine accepts the word $h_1h_3h_2f_1b$. We
now demonstrate how the compacting version of the pointer machine
is more economical. In particular, the computation proceeds as
above, up to the point:
\begin{eqnarray}
\nonumber & \larr{f_1} & \stackl x_{p_5},(Fy)_{p_3}
x_{p_4},(H(Fy)x)_{p_3},(D \varphi x)_{p_1} B,(D(D\varphi x) y
(\varphi y))_{p_1}, DGAB, S \stackr
\end{eqnarray}
we then proceed as follows:
\begin{eqnarray}
\nonumber & \larr{\epsilon} & \stackl y_{p_3},(D \varphi x)_{p_1} B,(D(D\varphi x) y (\varphi y))_{p_1}, DGAB, S \stackr \\
\nonumber & \larr{\epsilon} & \stackl B,(D(D\varphi x) y (\varphi y))_{p_1}, DGAB, S \stackr \\
\nonumber & \larr{b} & \stackl e, B, (D(D\varphi x) y (\varphi
y))_{p_1}, DGAB, S \stackr
\end{eqnarray}
where $p_1, \cdots, p_5$ are as before.
\end{example}

\subsection{Higher-order pushdown automata}
%\subsection{Higher-order pushdown automata and the Maslov Hierarchy}

Before we can define a higher-order pushdown automaton, we need to
define a \textbfit{level-$n$ store} or simply \emph{$n$-store}.
Fix a finite set $\Gamma$ of \emph{store symbols}, including a
distinguished bottom-of-store symbol $\bot$. A \emph{1-store} is a
finite non-empty sequence $\mkstore{a_1, \cdots, a_m}$ of
$\Gamma$-symbols such that $a_i = \bot$ iff $i = m$. For $n \geq
1$, an \emph{$(n+1)$-store} is a non-empty sequence of $n$-stores.
Inductively we define the \emph{empty $(n+1)$-store} $\bot_{n+1}$
to be $\mkstore{\bot_n}$ where we set $\bot_0 = \bot$. Recall the
following standard operations on $1$-stores: for $a \in ({\Gamma}
\setminus \makeset{\bot})$
\[\begin{array}{rll}
\nonumber \push_1(a) \, \mkstore{a_1,\cdots,a_m} & = & \mkstore{a,a_1,\cdots,a_m}\\
\nonumber \pop_1 \, \mkstore{a_1, a_2, \cdots, a_m} & = & \mkstore{a_2,\cdots,a_m}\\
%\nonumber top_1 ([a_1, a_2, \cdots, a_m]) & = & a_1
\end{array}\]
\renewcommand\arraystretch{1.3}
For $n \geq 2$, the following set $Op_n$ of \emph{level-$n$
operations} are defined over $n$-stores:
\[\left\{\begin{array}{rll}
\push_n \, \mkstore{s_1, \cdots, s_l} & = & \mkstore{s_1, s_1, \cdots, s_l}\\
\push_k \, \mkstore{s_1, \cdots, s_l} & = & \mkstore{\push_k \, s_1,
s_2, \cdots, s_l}, \quad 2 \leq k < n\\
\push_1 (a) \, \mkstore{s_1, \cdots, s_l} & = & \mkstore{\push_1 (a) \, s_1, s_2, \cdots, s_l}\\
\pop_n \, \mkstore{s_1, \cdots, s_l} & = & \mkstore{s_2, \cdots, s_l}\\
\pop_k \, \mkstore{s_1, \cdots, s_l} & = & \mkstore{\pop_k \, s_1,
s_2, \cdots, s_l}, \quad 1 \leq k < n\\
\id \, \mkstore{s_1, \cdots, s_l} & = & \mkstore{s_1, \cdots, s_l}\\
\end{array}\right.\]
In addition we define
\[\begin{array}{rll}
\mytop_n \, \mkstore{s_1, \cdots, s_l} & = & s_1\\
\mytop_k \, \mkstore{s_1, \cdots, s_l} & = & \mytop_k \, s_1, \quad 1
\leq k < n\\
\end{array}\]
Note that $\pop_{k} \, s$ is undefined if $\mytop_k \, s =
\bot_{k-1}$, for $k \geq 1$.

\medskip

A \textbfit{level-$n$ pushdown automaton} (or $n$PDA for short) is a
6-tuple $\anglebra{Q, \Sigma, \Gamma, \delta, q_0, F}$ where:
\begin{enumerate}
\item[(i)] $Q$ is a finite set of states, $q_0 \in Q$ is the start
state, and $F \subseteq Q$ is a set of accepting states
\item[(ii)] $\Sigma$ the finite input alphabet \item[(iii)]
$\Gamma$ the finite store alphabet (which is assumed to contain
$\bot$) \item[(iv)] $\delta \; \subseteq \; Q \times (\Sigma \cup
\makeset{\epsilon}) \times \Gamma \times Q \times Op_n$ the
transition relation.
\end{enumerate}
A \textbfit{configuration}\footnote{This is sometimes called
\emph{total configuration} in the literature.} of an $n$PDA is given
by a triple $(q, w, s)$ where $q$ is the current state, $w$ is the
remaining input, and $s$ is the is an $n$-store over $\Gamma$.

Given a configuration $(q, aw, s)$ (where $a \in \Sigma$ or $a =
\epsilon$, and $w \in \Sigma^*$) where $\mytop_1(s) = Z$, we
define the following relation ${\rightarrow}$:
\begin{equation}
\nonumber (q, aw, s) \rightarrow (p,w,s')
\end{equation}
if $(q,a,Z,p,\theta) \in \delta$ and $s' = \theta(s)$.
Intuitively, this says that if we are in state $q$ reading input
symbol $a$ and the topmost store symbol is $Z$, then we may change
the state to $p$, consume $a$, and perform the operation $\theta$
to the current $n$-store. The transitive closure of $\rightarrow$
is denoted by $\rightarrow^+$, whereas the reflexive and
transitive closure is denoted by $\rightarrow^*$. We say that the
input $w$ is \textbfit{accepted} by the above $n$PDA if $(q_0, w,
\bot_n) \rightarrow^* (q_f, \epsilon, s)$ for some pushdown store
$s$ and some $q_f \in F$.

\begin{remark}\rm We define $0$PDAs to be finite automata. Note that $1$PDAs
are just the standard pushdown automata. For an example of a $2$PDA we
refer the reader to the following section.
\end{remark}

The above defined $n$PDA can be thought of as being
non-deterministic. However, we say that an $n$PDA is
\textbfit{deterministic} if the following hold:
\begin{enumerate}
\item[(i)] If $\delta(q,a,Z) \not = \emptyset$ for some $a \in \Sigma$ then
$\delta(q,\epsilon, Z) = \emptyset$
\item[(ii)] $|\delta(q,a,Z)| \leq 1$
for all $a \in \Sigma \cup \{\epsilon\}$, $q \in Q$ and $Z \in
\Gamma$.
\end{enumerate}

\begin{remark}
In \cite{dMO} it is shown that a language $L$ is generated by a
level-$n$ safe deterministic grammar if and only if it is accepted
by a level-$n$ deterministic pushdown automaton. This is an easy
result to show -- but has not appeared previously in the literature.
\end{remark}
