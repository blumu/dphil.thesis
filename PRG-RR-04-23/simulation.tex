\section{Simulating $2$PDALs by non-deterministic $2$PDAs}

We have just seen that the incorporation of labels (as names of
links) into the stack alphabet will, in general, lead to an
infinite alphabet. In this section, we show how these links and
the way in which they are manipulated can be simulated by a
\emph{non-deterministic} $2$PDA.

Let $G$ be a possibly unsafe level-$2$ grammar and let the
corresponding $2$PDAL be denoted by $2PDAL_G$. Furthermore, let
$\Gamma$ be the stack alphabet described in the preceding section.
Our non-deterministic $2$PDA will have stack alphabet $\Gamma \cup
\{w^+ : w \in \Gamma\} \cup \{w^- : w \in \Gamma\} \cup \{w^{+/-}
: w \in \Gamma\}$. The $+$ and $-$ take on the same role as
before, i.e. the $-$ marks the start point of an instance of a
link, whereas the $+$ the end point. Note however that they are
\emph{anonymous} - they are no longer prefixed with natural
numbers.  We will explain how to use them shortly.

The crux of this proof lies in understanding how links are
manipulated in $2PDAL_G$. We dedicate the following
subsection to exactly this.

\subsection{The use of links in $2PDAL_G$}

Below we list some observations on the way links are manipulated.
Recall that labels are preserved with $\push_2$ operations. For
example, consider the following $2$-store of $2PDAL_G$
\begin{equation}
\nonumber \left [ \begin{array}{lll} \hbox{\tt [}a, & \mklink{b}{m-},
& c\hbox{\tt ]}\\ \hbox{\tt [}\mklink{d}{m+}, & e, & f\hbox{\tt ]}
\end{array} \right ]
\end{equation}
If we perform a $\push_2$, we will have
\begin{equation}
\nonumber \left [ \begin{array}{lll}
\hbox{\tt [}a, & \mklink{b}{m-}, & c\hbox{\tt ]}\\
\hbox{\tt [}a, & \mklink{b}{m-}, & c\hbox{\tt ]}\\
\hbox{\tt [}\mklink{d}{m+}, & e, & f\hbox{\tt ]}
\end{array} \right ]
\end{equation}
We now have two instances of the link $m$. One connects the upper
$b$ to $d$, and the other connects the lower $b$ to $d$. By Lemma
5.4(i), if there are two or more instances of a label $m$, then
although each may have a different start point, they all share the
same end point, as illustrated by the above example.

\begin{lemma} \label{lem:useonceonly} In any run of $2PDAL_G$, if
$(q_j,s)$ is a reachable configuration where $top_1(s) = {\$ t_1
\cdots t_n}^{\langle m- \rangle \cup \lambda}$ and $j > 0$, then,
for all $(q_k,s')$ such that $(q_j,s) \rightarrow_{+} (q_k,s')$,
$top_1(s')$ is not labelled by $m-$.
\end{lemma}

\begin{proof} The key to this proof is recognising that $\pop_2$
is never used, only the parameterised form $\pop_2(m)$ is used.
There are two cases. If $j \leq n$, then the result follows from
Lemma~\ref{lem:order}. However, if $j > n$, note that the next configuration after $(q_j,s)$ is $(q_{j-n}, s'')$ where
$\mytop_1(s'') = {\varphi u_1 \cdots u_{n'}}^{\langle m+ \rangle \cup
\lambda'}$. By Lemma 5.4(i), no item in $s''$ can be labelled by
$m-$. Furthermore, any new labels introduced will be fresh.
\end{proof}

We say that a link $m$ is \textbfit{followed} (at a configuration
$(q_j,s)$) if $\mytop_1(s) = {\$t_1 \cdots t_n}^{\langle m-
\rangle \cup \lambda}$ for some $\lambda$ and some $n \geq 0$ such
that $j > n$. A corollary of the above lemma is that any given
link $m$ is followed at most once.

%Note that whenever the 2PDAL is at such a configuration, it
%performs a $\pop_2(m)$ on (an instance of) the link $m$, jumping
%from the start point to the end point.


%\begin{lemma} \label{lem:useonceonly} In any run of a 2PDAL, any given link
%$m$ is followed at most once. Formally, if $(q_j,s)$ is a reachable
%configuration at which a link $m$ is followed, then, for all $(q,s')$
%such that $(q_j,s) \rightarrow^+ (q,s')$, no items labelled with $m-$
%occurs in $s'$; in particular $\mytop_1(s')$ is not labelled by $m-$.
%\end{lemma}
%
%\begin{proof} Suppose $\mytop_1(s) = {\$t_1 \cdots t_n}^{\langle m- \rangle
%\cup \lambda}$ for some $n \geq 0$ and some $\lambda$, and such that
%$j > n$. The next configuration after $(q_j, s)$ is $(q_{j-n}, s'')$
%where $\mytop_1(s) = \$' u_1 \cdots u_{n'}^{\langle m+ \rangle \cup
%\lambda'}$. By Lemma 5.4(i), no item in $s''$ can be labelled by
%$m-$. Furthermore, any new labels introduced will be fresh.
%\end{proof}

%The above lemma is tantamount to saying that a link $m$ will be
%followed at most once, and if it is followed then, in any subsequent
%configurations, no items with label $m-$ exist.

\subsection{Intuition}
The important thing to note about $2PDAL_G$ is that not all links
are followed. In the example illustrated in Section 5.2, no link
apart from $1$ was followed. In fact, we may as well not have
bothered to label the remaining links! This observation is the key
to our construction.

The simulating non-deterministic $2$PDA will follow the rules in
Fig.~\ref{fig:pdal} almost exactly. The difference is that each time
we are about to generate a link (i.e.~Case (4) of
Fig.~\ref{fig:pdal}), we guess whether it will ever be followed in the
future or not, and we label the start and end points of the link if
and only if we guess that it will be followed. Furthermore, instead of
a fresh label $m$, we simply mark the start point with a $-$ and the
end point with a $+$.


\subsubsection*{A controlled form of guessing}

Now this presents a problem of ambiguity. Suppose we find ourselves in
a configuration $(q,s)$ where $\mytop_1(s)$ is labelled by $-$, how
can we tell which of the stack items labelled by a $+$ is the
\emph{true} end point of this link? (True in the sense that if we did
have the ability to name our links as with $2PDAL_G$, the topmost
item would have label $m-$ for some $m$, and the \emph{real} end point
would have label $m+$ for the same $m$.) The answer lies in the use of
a \emph{controlled} form of guessing: when guessing whether a link
will be followed in the future, the machine will not always be allowed
to guess whichever way it pleases; instead we require the guess to be
subject to some constraints. We shall see that as a consequence the
following invariant can be maintained:
\begin{quote}
\emph{Assume that the topmost $1$-store has at least one item
labelled by $-$. For the leftmost (closest to the top) of these, the
corresponding end point can be found in the first $1$-store beneath it
whose topmost item is marked with a $+$.}\footnote{The invariant is
actually stronger than this, but this is sufficient to ensure that the
simulation works correctly.}
\end{quote}
%(For example in the following
%\begin{equation}
%\nonumber \left [ \begin{array}{lll} \hbox{\tt [}{a}^{-}, & b, &
%c\hbox{\tt ]}\\ \hbox{\tt [}d, & e, & f\hbox{\tt ]}\\ \hbox{\tt
%[}g^{+}, & h, &i\hbox{\tt ]}\\ \hbox{\tt [}j^{+}, & k, &l\hbox{\tt ]}
%\end{array} \right ]
%\end{equation}
%if the real end point of $a$ is $j$, we would have to perform 3
%$\pop_2$'s were we to follow the link at $a$.)

Before formalising the controlled form of guessing, we introduce a
definition. Let $(q_0, s)$ be a reachable configuration of $2PDAL_G$
such that
\begin{equation}
\nonumber \mytop_2(s) = [\varphi_{j_1} t_1 \cdots t_n^\lambda, A_1,
\cdots, A_k, \cdots, A_N]
\end{equation}
where $N \geq 2$. We say that $\varphi_{j_1}$ \textbfit{ultimately
refers to} $A_k$ just if:
\begin{itemize}
\item[(i)] For $i = 1, \cdots, k-1$, the $j_i$th level-$1$ argument of
$A_i$ is a variable $\varphi_{j_{i+1}}$. We remind the reader of the
notational convention for level-$1$ parameters set out in (\ref{convention}).
\item[(ii)] The $j_{k}$th argument of $A_k$ is an application or a
non-terminal.
\end{itemize}
So, for example, using the grammar in Example~\ref{ex:ex1}, the
following is a reachable $2$-store for $2PDAL_G$ on input $h_1h_3h_3$:
\begin{equation}
\nonumber \begin{array}{rl}
(q_0, & \hbox{\tt [}\mkstore{\varphi B,\mklink{(D \varphi x)}{1-},
DGAB, S},\\
& \mkstore{\mklink{\varphi B}{1+}, D (D \varphi x) y (\varphi y),
DGAB,S}\hbox{\tt ]})
\end{array}
\end{equation}
Here the topmost $\varphi$ ultimately refers to (the $G$ in) $DGAB$ (in
the topmost $1$-store).

Suppose that we are in a configuration $(q_0, s)$ of the
non-deterministic $2$PDA where
\begin{equation}
\nonumber \mytop_2(s) = \mkstore{\varphi t_1 \cdots t_n^{?}, A_1,
\cdots, A_j, \cdots, A_n}
\end{equation}
where $?$ may either denote a $-$ or no label at all. Furthermore,
suppose that $\varphi$ ultimately refers to $A_j$. There are two possibilities:
\begin{enumerate}
\item[A.] None of the stack items $\varphi t_1 \cdots t_n^{?}, A_1,
\cdots, A_j$ are labelled by a $-$; or \item[B.] There exists a stack item
in  $\varphi t_1 \cdots t_n^{?}, A_1, \cdots, A_j$ labelled by a
$-$.
\end{enumerate}
In the first case we leave it up to the $2$PDA to
non-deterministically label $\varphi t_1 \cdots t_n$ (with $+$) in the
configuration $(q_0, s)$ as well as its matching partner (with $-$). However,
in the second case, we \emph{insist} that the $2$PDA label
$\varphi t_1 \cdots t_n$ in the configuration $(q_0, s)$ as well
as its matching partner, as given in Case (4) of Fig.~\ref{fig:pdal}.

Let us illustrate why this maintains the above invariant with an
example. Suppose we have the following configuration:
\begin{equation}
\nonumber \left [\begin{array}{l}
\mkstore{\varphi x_1 x_2, {D \varphi x}^{-}, F(F\varphi x)y, {G\varphi
x}^-, E, S}\\
\mkstore{A^+, \cdots}\\
\mkstore{B^+, \cdots}
\end{array} \right ]
\end{equation}
Note that the topmost stack has two items labelled with a $-$, $D
\varphi x$ and $G \varphi x$. By our invariant we know that $D \varphi
x$ has end point $A^+$. And let us suppose that $G \varphi x$ points
to $B^+$. Suppose that the $\varphi$ of the topmost item ultimately
refers to (the subterm $F \varphi x$) in $F(F \varphi x)
y$. Furthermore, suppose we go against our controlled form of guessing
and allow the machine \emph{not} to label $\varphi x_1 x_2$ and its
matching partner. Thus we arrive at
\begin{equation}
\nonumber \left [ \begin{array}{l}
\mkstore{ \varphi, F(F\varphi x)y, {G\varphi x}^-, E, S}\\
\mkstore{\varphi x_1 x_2, {D \varphi x}^{-}, F(F\varphi x)y, {G\varphi
x}^-, E, S}\\
\mkstore{A^+, \cdots}\\
\mkstore{B^+, \cdots}
\end{array} \right ]
\end{equation}
Note that now $G \varphi x$ is the leftmost item labelled with a
$-$. Our invariant has been violated as the real end point of $G
\varphi x$ is not $A^+$.

%In the next section we will prove rigorously that obeying the above
%convention will ensure that the above invariant is mained.

\subsubsection*{Penalty for guessing wrongly}

The cost of using non-determinism (regardless of how controlled it may
be) is that we commit ourselves to following our guesses. When we find
out that we have guessed wrongly, we shall have to abort the
run. There are two cases. Suppose we find ourselves in a configuration
$(q_j,s)$ where $\mytop_1(s) = {\$ x_1 \cdots x_n}^{-}$ and $j \leq
n$. The fact that the topmost item is labelled by $-$ means that at
some point in the past, we guessed that we would follow this link.
%Recall that the non-deterministic $2$PDA only has the ability to label
%items with a $+, -$ or $+/-$ and is not able to add a name to
%links. Hence, when an item is labelled with a $-$, in the
%corresponding $2$PDAL it will actually be labelled with a $m-$, and it
%will have a corresponding end point $m+$ that presents itself as a
%mere $+$ in the non-deterministic $2$PDA.
But, by Lemma~\ref{lem:useonceonly} we have not yet followed the
link nor will we ever follow it in the future.
%But, as $j \leq n$, we have so far not followed the link $m$, nor (by
%Lemma~\ref{lem:useonceonly}) will we ever follow it in future.
The machine has guessed wrongly, and we abort immediately.
Symmetrically if we find ourselves in a configuration $(q_j,s)$
where $\mytop_1(s) = \$ x_1 \cdots x_n$ and $j > n$, then we also
abort. Why? The absence of a $-$ label means that at some point in
the past we guessed that we would \emph{not} follow this link, but
we are now about to turn against our original guess.

%\subsubsection*{Why it will work}
%
%We will show that for each word in the language of the $2$PDAL, there
%is at least one way to guess correctly which stack items will be
%labelled (with $+,-$ or $+/-$) such that the word is
%accepted. Furthermore, these guesses will be consistent with our
%controlled guessing. Conversely, we will show that every set of
%guesses consistent with our controlled guessing will lead to the
%acceptance of a word that belongs to the language of the $2$PDAL.

\subsection{Definition of the non-deterministic $2$PDA, $2PDA_G$}
Let $G$ be a (possibly unsafe) 2-grammar that generates a string
language. The transition rules of the non-deterministic $2$PDA,
$2PDA_G$, are given in Fig.~\ref{fig:2pda}.
%\begin{figure*}[t]
%\begin{center}
%\makebox{
%\begin{shadowbox}[17cm]
%\begin{math}
%\begin{array}{rclr}
%\nonumber (q_0, a,  Dt_1 \cdots t_n^\lambda) & \rightarrow & (q_0, \push_1(E)) \mbox{ if $Dx_1 \cdots x_n \larr{a} E$ } & \rm{(1)}\\
%\nonumber (q_0, \epsilon ,e) & \rightarrow & \mbox{accept} & \rm{(2)}\\
%\nonumber (q_0, \epsilon ,x_j)& \rightarrow & (q_j, \pop_1) & \rm{(3)}\\
%\nonumber \\
%\nonumber (q_0, \epsilon, \varphi_j t_1 \cdots t_n) & \rightarrow
%& \left \{ \begin{array}{l}
%(q_0, \repl_1({\varphi_j t_1 \cdots t_n}^{+}) \compose \push_2 \compose \pop_1 \compose \repl_1({s_j}^{-}))\\
%(q_0, \push_2 \compose \pop_1 \compose \repl_1(s_j))
%\end{array} \right . & \begin{array}{r} \rm{(4.1)} \\ \rm{(4.1')}\end{array}\\
%\nonumber & & \mbox{where Situation A holds and ${Ds_1 \cdots
%s_{n'}}^{\lambda'}$ precedes
%$\varphi_j t_1 \cdots t_n$} & \\
%\nonumber (q_0, \epsilon, {\varphi_j t_1 \cdots t_n}^{\lambda }) &
%\rightarrow & (q_0, \repl_1({\varphi_j t_1 \cdots t_n}^{+ \cup
%\lambda})
%\compose \push_2 \compose \pop_1 \compose \repl_1({s_j}^{-})) & \rm{(4.2)}\\
%\nonumber & & \mbox{where Situation B holds and ${Ds_1 \cdots
%s_{n'}}^{\lambda'}$ precedes
%${\varphi_j t_1 \cdots t_n}^\lambda$ } & \\
%\nonumber \\
%\nonumber 1 \leq j \leq n, (q_j, \epsilon, \$t_1 \cdots t_n^\lambda) &
%\rightarrow & \left \{ \begin{array}{ll}
%(q_0, \repl_1(t_j)) &  - \not \in \lambda \\
%\mbox{abort} &  - \in \lambda \end{array} \right . & \begin{array}{r} \rm{(5.1)} \\ \rm{(5.2)}\end{array} \\
%\nonumber j > n , (q_j, \epsilon, \$t_1 \cdots t_n^\lambda) &
%\rightarrow & \left \{ \begin{array}{ll}
%\mbox{abort} & - \not \in \lambda\\
%(q_{j-n}, \pop^+_2) & - \in \lambda \end{array} \right . & \begin{array}{r} \rm{(6.1)} \\ \rm{(6.2)}\end{array}
%\end{array} \end{math}
%\end{shadowbox}}
%\end{center}
%\caption{Transition rules of the non-deterministic $2$PDA,
%$2PDA_G$ \label{fig:2pda}}
%\end{figure*}
\begin{figure*}[t]
\begin{center}
\makebox{
\begin{shadowbox}[16.5cm]
\begin{eqnarray}
\nonumber (q_0, a,  Dt_1 \cdots t_n^\lambda) & \rightarrow & (q_0, \push_1(E)) \mbox{ if $Dx_1 \cdots x_n \larr{a} E$ }\\
\nonumber (q_0, \epsilon ,e) & \rightarrow & \mbox{accept}\\
\nonumber (q_0, \epsilon ,x_j)& \rightarrow & (q_j, \pop_1)\\
\nonumber \\
\nonumber (q_0, \epsilon, \varphi_j t_1 \cdots t_n) & \rightarrow
& \left \{ \begin{array}{l}
(q_0, \repl_1({\varphi_j t_1 \cdots t_n}^{+}) \compose \push_2 \compose \pop_1 \compose \repl_1({s_j}^{-}))\\
(q_0, \push_2 \compose \pop_1 \compose \repl_1(s_j))
\end{array} \right .\\
\nonumber & & \mbox{where Situation A holds and ${Ds_1 \cdots s_{n'}}^{\lambda'}$ precedes
$\varphi_j t_1 \cdots t_n$}\\
\nonumber (q_0, \epsilon, {\varphi_j t_1 \cdots t_n}^{\lambda }) &
\rightarrow & (q_0, \repl_1({\varphi_j t_1 \cdots t_n}^{+ \cup
\lambda})
\compose \push_2 \compose \pop_1 \compose \repl_1({s_j}^{-}))\\
\nonumber & & \mbox{where Situation B holds and ${Ds_1 \cdots s_{n'}}^{\lambda'}$ precedes
${\varphi_j t_1 \cdots t_n}^\lambda$ }\\
\nonumber \\
\nonumber 1 \leq j \leq n, (q_j, \epsilon, \$t_1 \cdots t_n^\lambda) &
\rightarrow & \left \{ \begin{array}{ll}
(q_0, \repl_1(t_j)) &  \mbox{if } - \not \in \lambda \\
\mbox{abort} & \mbox{if } - \in \lambda \end{array} \right . \\
\nonumber j > n, (q_j, \epsilon, \$t_1 \cdots t_n^\lambda) & \rightarrow &
\left \{ \begin{array}{ll}
\mbox{abort} & \mbox{if } - \not \in \lambda \\
(q_{j-n}, \pop^+_2) & \mbox{if } - \in \lambda
\end{array} \right .
\end{eqnarray}
\end{shadowbox}}
\end{center}
\caption{Transition rules of the non-deterministic $2$PDA, $2PDA_G$
\label{fig:2pda}}
\end{figure*}
Note that, again, we assume that production rules of the grammar
assume the format given in rule (\ref{convention}). The function
$\pop_2^{+}$ performs a $\pop_2$ and then repeats until the topmost
symbol it reads is marked with a $+$. Formally, let $s$ range over 2-stores, we define $\pop_2^+ (s) =
p(\pop_2(s))$ where
\begin{eqnarray}
\nonumber p (s)  & = &  \left \{
\begin{array}{ll}
s & \hbox{if $\mytop_1(s)$ has label $+$}\\
{p(\pop_2 (s))} & \mbox{otherwise}
\end{array} \right .
\end{eqnarray}
Observe that the rules of the simulating non-deterministic $2$PDA
(Fig.~\ref{fig:2pda}) are almost identical to those of the $2$PDAL
(Fig.~\ref{fig:pdal}).

\begin{remark}
In the definition of the transition rules (Fig.~\ref{fig:2pda}), in
case the $\mytop_1$ item of the 2-store is headed by a level-1
variable, the $2$PDA has to work out whether it is situation A or
B. This can be achieved by a little scratch work on the side: do a
$\push_2$, inspect the topmost 1-store for as deep as necessary,
followed by a $\pop_2$. Alternatively we could ask the oracle to tell
us whether it is A or B, taking care to ensure that a wrong
pronouncement will lead to an abort.
\end{remark}

\subsection{Proof of correctness}
First some notations. Items labelled with superscripts $-, +$ or
$+/-$, will be referred to as \textbfit{marked} items. In
particular, if an item is marked with a $-$ or a $+/-$,
then we say it is \textbfit{$-$marked}, indicating that one of its
labels is indeed a $-$. Similarly, if an item is marked with a $+$
or a $+/-$, then we say it is \textbfit{$+$marked}. If the topmost
1-store of $s$ has at least one $-$marked item, then we say that
the leftmost of these (closest to the top) items is the
\textbfit{foremost}. (If the topmost 1-store of $s$ has no
$-$marked items, then $s$ has no foremost item.) Let $\theta$ be a
$-$marked item in a $2$-store $s$. We say that $\theta$ is
$-$\textbfit{reachable} in $s$ just if $\theta$ is foremost in
$s$, or if $s$ has a foremost item and $\theta$ is $-$reachable in
$\pop_2^+(s)$. For example consider the following $2$-store:
%$\pop_2^+ (s)$ is defined and has a foremost item and $\theta$ is
%$-$reachable in $\pop_2^+ (s)$. For example consider the following
%2-store:
\begin{equation}
\nonumber s \; = \; \left [ \begin{array}{lllll}
\hbox{\tt [} a      , & b^- , & c   , & d^- , & e \hbox{\tt ]}\\
\hbox{\tt [} f      , & g   , & h   , & i   , & j \hbox{\tt ]}\\
\hbox{\tt [} k^{+/-}, & l   , & m   , & n   , & o \hbox{\tt ]}\\
\hbox{\tt [} p^+    , & q   , & r^- , & s   , & t \hbox{\tt ]}\\
\hbox{\tt [} u^+    , & v   , & w   , & x   , & y \hbox{\tt ]}
\end{array} \right ]
\end{equation}
The foremost item is $b$. The items $b,k$ and $r$ are the only
$-$reachable ones. Note that if  we perform a \emph{single}
$\pop_2$, then there will be no foremost item. However, with
another $\pop_2$, the foremost item will be $k$.

Finally, we fix a possibly unsafe 2-grammar $G$ that generates a
string language. Let $(q, s)$ and $(q', s')$ be reachable
configurations of the $2PDA_G$ and $2PDAL_G$ respectively.
Intuitively we write $s \bohm s'$ to mean that $s$ simulates $s'$.
Precisely $s \bohm s'$ holds if either both $s$ and $s'$ are the
empty 2-store, or the following hold
\begin{itemize}
\item[(i)] If $\mytop_1(s) = \mytop_1(s') = \bot$, then $pop_2(s)
\bohm s'$, and
\item[(ii)] If $\mytop_1(s) = {a}^{\lambda}$ and $\mytop_1(s') =
{a'}^{\lambda'}$ then 
\begin{itemize}
\item[a.] $a = a'$ and
\item[b.] If $\lambda$ has label $-$ then $\lambda'$ has label $m-$
for some $m$. Similarly, if $\lambda$ has label $+$ then $\lambda'$
has label $n+$ for some $n$; and

%\[
%\begin{array}{rcl}
%\nonumber \mbox{$\lambda$ has label $-$} & \Rightarrow & \mbox{$\lambda'$ has label $m-$ for some $m$}\\
%\nonumber \mbox{$\lambda$ has label $+$} & \Rightarrow &
%\mbox{$\lambda'$ has label $m+$ for some $m$}
%\end{array}
%\], and

\item[c.] $pop_1(s) \bohm pop_1(s')$
\end{itemize}
\end{itemize}

\begin{lemma}\label{lem:consistency}
Let $(q_0, \mkstore{\mkstore{S}}) = (p_0, s_0) \rightarrow \cdots
\rightarrow (p_n, s_n)$ be a transition sequence of $2PDA_G$. Then
there exists a unique (modulo renaming of labels) transition
sequence $(p_0, s_0') \rightarrow \cdots \rightarrow (p_n, s_n')$
of $2PDAL_G$ such that for all $i = 0, \cdots, n$:
\begin{itemize}
\item[(i)] $s_i \bohm s_i'$
\item[(ii)] If $s \sqsubseteq s_i$ such that $\mytop_1(s)$ is
$-$reachable in $s_i$, and if $s' \sqsubseteq s_i'$ such that $s \bohm
s'$, then $\pop^+_2(s) \bohm \pop_2(m)(s')$ where $\mytop_1(s')$ has
label $m-$.
\end{itemize}
\end{lemma}

\begin{proof}
Induction on the number of transitions.
\end{proof}

\begin{proposition} \label{prop:prop1}
If a string is accepted by $2PDA_G$, it is also accepted by $2PDAL_G$.
\end{proposition}

\begin{proof}
This follows from Lemma~\ref{lem:consistency} and the correctness of
the $2$PDAL.
\end{proof}

\begin{proposition} \label{prop:prop2}
If a string is accepted by $2PDAL_G$, it is also accepted by $2PDA_G$.
\end{proposition}

\begin{proof}
Suppose there is an all-knowing oracle that always tells us
\emph{correctly} whether or not we will ever follow a link. We need to
check that the controlled guessing does not restrict the choices of
the oracle.  Recall that when Situation B arises we do not allow our
$2$PDA a choice: we \emph{always} guess that the link we are about to
create will be followed in the future. We must check that the oracle
agrees with this decision.

Thus suppose that the current configuration is $(q_0, s)$ and
$\mytop_2(s)$ is the following:
\begin{equation}
\nonumber \mkstore{A_1^{\lambda_1}, A_2^{\lambda_2}, \cdots, A_k^-,
\cdots, A_{n-1}^{\lambda^{n-1}}, A_n^{\lambda_n}, \cdots}
\end{equation}
where $A_1 = \varphi_j t_1 \cdots t_m$, and $1\leq k < n$.
Furthermore, suppose that $\varphi_j$ in $A_1$ ultimately refers to
$A_n$. According to our rules, after a finite number of transition
steps we reach the following configuration:
\begin{eqnarray}
\nonumber && \mkstore{t_{A_n}^{-}, \cdots } \\
\nonumber && \mkstore{t_{A_n-1}^{+/-}, A_n^{\lambda_n} , \cdots}  \\
\nonumber && \vdots \\
\nonumber && \mkstore{t_{A_k}^{+/-}, A_{k+1}^{\lambda_{k+1}}, \cdots, A_{n-1}^{\lambda_{n-1}}, A_n^{\lambda_n}, \cdots}\\
\nonumber && \vdots \\
\nonumber && \mkstore{t_{A_2}^{+/-},  A_3^{\lambda_3}, \cdots,  A_k^-, \cdots, A_{n-1}^{\lambda^{n-1}}, A_n^{\lambda_n},\cdots}\\
\nonumber && \mkstore{\varphi_j t_1 \cdots t_n^{+ \cup \lambda}, A_2^{\lambda_2}, \cdots, A_k^-, \cdots, A_{n-1}^{\lambda^{n-1}},A_n^{\lambda_n},\cdots}\\
\nonumber && \vdots
\end{eqnarray}
where $t_{A_i}$ is a subterm of $A_i$. We must check that the above
``forced'' choices would tally with those of the oracle. Suppose, for
a contradiction, that it does not, and that $t_{A_i}$ for some $i$ is
not $-$marked. There are two possibilities:
\begin{enumerate}
\item[(1)] $t_{A_n}$ is not $-$marked.
\item[(2)] $t_{A_n}$ is $-$marked,
but there exists an $i$ such that $t_{A_i}$ is not $-$marked for
some $i < n$.
\end{enumerate}
For case (1), the proof of Lemma~\ref{lem:useonceonly} says that
the link represented by $A_k^-$ has not yet been followed.
However, it should not be difficult to see that by not marking
$t_{A_n}$ we have eliminated the possibility of ever returning to
$A_k^-$. In particular, this contradicts that the oracle's
previous information that we would perform a $\pop_2^+$ at $A_k^-$
(or a duplicate) in the future. For case (2), let $I = \max \{i :
\mbox{$A_i$ is not $-$marked}\}$. Then, according to the oracle,
we will perform a $\pop_2^+$ at $t_{A_{I+1}}$, however, the next
configuration will be $(q_k, s')$ for $k>0$ and $\mytop_1(s') =
t_{A_I}$. This is untenable, as by assumption $t_{A_I} = \varphi$
for some level-$1$ variable $\varphi$, thus we must follow a link
here unconditionally or we will be stuck. Again, this contradicts
the oracle's information that we would not need to follow a link
at $t_{A_I}$.
\end{proof}

Combining Propositions~\ref{prop:prop1} and \ref{prop:prop2} we can
conclude:
\begin{theorem}
There is an effective transformation of any (possibly unsafe)
string-language generating 2-grammar to a non-deterministic $2$PDA that
accepts the same language. \myendproof
\end{theorem}

%\begin{remark} In Proposition \re{prop:prop2} it was shown that if
%$w$ is accepted by a $2$PDAL then it is accepted by the
%corresponding non-deterministic $2$PDA. In particular, it showed
%that there is at least one way to guess which links are followed
%in such a way that $w$ is accepted and such that the guesses are
%consistent with our controlled guessing. However, it should also
%be noted that there may be \emph{more} than one such way. Thus,
%perhaps a truer reflection of what it means for a stack item to be
%$-$marked is the following. If a stack item is $-$marked then,
%should we ever visit this item in state $q_j$ for $j>0$ we will
%always perform a $\pop_2^+$. Note that this says nothing about the
%case where we \emph{never} visit this item in a state $q_j$ for
%$j>0$.
%\end{remark}


\subsection{Some examples}

\begin{example}
We demonstrate the non-deterministic $2$PDA for the grammar in
Example~\ref{ex:ex1} on input $h_1h_3h_2f_1a$. Recall that this word
does not belong in the language. As before, the automaton starts in the configuration $(q_0, h_1h_3h_2f_1a,
\mkstore{\mkstore{S}})$. After a few steps the configuration is:
\begin{eqnarray}
\nonumber (q_0, & h_2f_1a, & \mkstore{\mkstore{\varphi B, D(D
\varphi x) y (\varphi y), DGAB,S}})
\end{eqnarray}
It is now up to the automaton to choose whether or not to label the
start and end points of this link. Suppose it chooses to label:
\begin{eqnarray}
\nonumber (q_0, & h_2f_1a, & \hbox{\tt [}\mkstore{{(D \varphi x)}^{-}, DGAB, S},\\
\nonumber & & \mkstore{{\varphi B}^{+}, D (D \varphi x) y (\varphi
y), DGAB,S}\hbox{\tt ]})
\end{eqnarray}
After a few more steps the automaton arrives at another
configuration where the topmost symbol is headed by a level-$1$
variable.
\begin{eqnarray}
\nonumber (q_0, & f_1a, & \hbox{\tt [}\mkstore{\varphi x,H (F y) x,{(D
\varphi x)}^{-}, DGAB, S}, \\
\nonumber & &\mkstore{{\varphi B}^{+}, D (D \varphi x) y (\varphi
y), DGAB,S}\hbox{\tt ]})
\end{eqnarray}
This time suppose it chooses not to label:
\begin{eqnarray}
\nonumber \rightarrow_+ (q_0, & a, & \hbox{\tt [}\mkstore{x,(F y),{(D \varphi x)}^{-}, DGAB, S}\\
\nonumber & &\mkstore{\varphi x,H (F y) x,{(D \varphi x)}^{-}, DGAB, S}, \\
\nonumber & &\mkstore{{\varphi B}^{+}, D (D \varphi x) y (\varphi
y), DGAB,S}\hbox{\tt ]})\\
\nonumber \rightarrow_+ (q_0, & a, & \hbox{\tt [}\mkstore{y,{(D \varphi x)}^{-}, DGAB, S}\\
\nonumber & &\mkstore{\varphi x,H (F y) x,{(D \varphi x)}^{-}, DGAB, S}, \\
\nonumber & &\mkstore{{\varphi B}^{+}, D (D \varphi x) y (\varphi
y), DGAB,S}\hbox{\tt ]})\\
\nonumber \rightarrow_+ (q_3, & a, & \hbox{\tt [}\mkstore{{(D \varphi x)}^{-}, DGAB, S}\\
\nonumber & &\mkstore{\varphi x,H (F y) x,{(D \varphi x)}^{-}, DGAB, S}, \\
\nonumber & &\mkstore{{\varphi B}^{+}, D (D \varphi x) y (\varphi
y), DGAB,S}\hbox{\tt ]})\\
\nonumber \rightarrow_+ (q_1, & a, & \hbox{\tt [}\mkstore{{\varphi
B}^{+}, D (D \varphi x) y (\varphi
y), DGAB,S}\hbox{\tt ]})\\
\nonumber \rightarrow_+ (q_0, & a, & \hbox{\tt [}\mkstore{B, D (D
\varphi x) y (\varphi y), DGAB,S}\hbox{\tt ]})
\end{eqnarray}
and, as the word $h_1h_3h_2f_1a$ is rejected, as required.
\end{example}

\begin{example}As another example using the same grammar, let us
attempt a computation on the same word $h_1h_3h_2f_1a$ but this time
using a different set of guesses.
\begin{eqnarray}
\nonumber (q_0, & h_1h_3h_2f_1a, & \mkstore{\mkstore{S}})\\
\nonumber \rightarrow_+ (q_0, & h_2f_1a, & \hbox{\tt [}\mkstore{{(D \varphi x)}, DGAB, S},\\
\nonumber & & \mkstore{{\varphi B}, D (D \varphi x) y (\varphi
y), DGAB,S}\hbox{\tt ]})\\
\nonumber \rightarrow_+ (q_0, & f_1a, & \hbox{\tt [}\mkstore{\varphi x,H (F y) x,{(D
\varphi x)}, DGAB, S}, \\
\nonumber & &\mkstore{{\varphi B}, D (D \varphi x) y (\varphi
y), DGAB,S}\hbox{\tt ]})\\
\nonumber \rightarrow_+ (q_0, & f_1a, & \hbox{\tt [}\mkstore{{(F y)}^{-},{(D\varphi x)}, DGAB, S}\\
\nonumber & & \mkstore{{\varphi x}^{+},H (F y) x,{(D\varphi x)}, DGAB, S}, \\
\nonumber & & \mkstore{{\varphi B}, D (D \varphi x) y (\varphi y),
DGAB,S}\hbox{\tt ]})\\
\nonumber \rightarrow_+ (q_0, & a, & \hbox{\tt [}\mkstore{x,{(F y)}^{-},{(D\varphi x)}, DGAB, S}\\
\nonumber & & \mkstore{{\varphi x}^{+},H (F y) x,{(D\varphi x)}, DGAB, S}, \\
\nonumber & & \mkstore{{\varphi B}, D (D \varphi x) y (\varphi y),
DGAB,S}\hbox{\tt ]})\\
\nonumber \rightarrow_+ (q_1, & a, & \hbox{\tt [}\mkstore{{(F y)}^{-},{(D\varphi x)}, DGAB, S}\\
\nonumber & & \mkstore{{\varphi x}^{+},H (F y) x,{(D\varphi x)}, DGAB, S}, \\
\nonumber & & \mkstore{{\varphi B}, D (D \varphi x) y (\varphi y),
DGAB,S}\hbox{\tt ]})
\end{eqnarray}
But now the automaton aborts prematurely. The fact the the topmost
symbol is labelled with a $-$ indicates that earlier the automaton
guessed it would follow the link here. However, it is now in state
$q_1$ and the first argument \emph{is} present, hence, the
automaton guessed incorrectly. There are two other combinations of
guesses that can consume the prefix $h_1h_3h_2f_1$; we leave it to
the reader to check that neither results in acceptance of
$h_1h_3h_2f_1a$.
\end{example}

\begin{example}
Finally, as a last example, we use the same grammar, but this time our
input word is $h_1h_3h_3b$. It can easily be checked that this is
indeed a word in the language.
\begin{eqnarray}
\nonumber (q_0, & h_1h_3h_3b, & \mkstore{\mkstore{S}})\\
\nonumber \rightarrow_+ (q_0, & h_3b, & \mkstore{\mkstore{\varphi B, D(D \varphi x) y (\varphi y), DGAB,S}})\\
\nonumber \rightarrow_+ (q_0, & h_3b, &  \hbox{\tt [}\mkstore{ {(D \varphi x)}^{-}, DGAB,S}\\
\nonumber & & \mkstore{{\varphi B}^{+}, D(D \varphi x) y (\varphi
y), DGAB,S}\hbox{\tt ]})\\
\nonumber \rightarrow_+ (q_0, & b, &  \hbox{\tt [}\mkstore{ \varphi B,{(D \varphi x)}^{-}, DGAB,S}\\
\nonumber & & \mkstore{{\varphi B}^{+}, D(D \varphi x) y (\varphi
y), DGAB,S} \hbox{\tt ]})
\end{eqnarray}
At this point, situation B occurs: $\varphi$ in the topmost item
ultimately refers to ($G$ in) $DGAB$, therefore, our controlled form of
guessing insists that the automaton label the link:
\begin{eqnarray}
\nonumber \rightarrow_+ (q_0, &b, &  \hbox{\tt [}\mkstore{ \varphi^{-}, DGAB,S}\\
\nonumber & & \mkstore{{\varphi B}^{+}, {(D \varphi x)}, DGAB,S}\\
\nonumber & & \mkstore{{\varphi B}^{+}, D(D \varphi x) y (\varphi
y), DGAB,S}
\hbox{\tt ]})\\
\nonumber \rightarrow_+ (q_0, &b, &  \hbox{\tt[}\mkstore{ G^{-},S}\\
\nonumber & & \mkstore{ \varphi^{+/-}, DGAB,S}\\
\nonumber & & \mkstore{{\varphi B}^{+},{(D \varphi x)}^{-}, DGAB,S}\\
\nonumber & & \mkstore{{\varphi B}^{+}, D(D \varphi x) y (\varphi y), DGAB,S} \hbox{\tt ]})\\
\nonumber \rightarrow_+ (q_0, &b, &  \hbox{\tt[}\mkstore{x,G^{-},S}\\
\nonumber & & \mkstore{ \varphi^{+/-}, DGAB,S}\\
\nonumber & & \mkstore{{\varphi B}^{+},{(D \varphi x)}^{-}, DGAB,S}\\
\nonumber & & \mkstore{{\varphi B}^{+}, D(D \varphi x) y (\varphi
y), DGAB,S}
\hbox{\tt ]})\\
\nonumber \rightarrow_+ (q_1, &b, &  \hbox{\tt[}\mkstore{G^{-},S}\\
\nonumber & & \mkstore{ \varphi^{+/-}, DGAB,S}\\
\nonumber & & \mkstore{{\varphi B}^{+},{(D \varphi x)}^{-}, DGAB,S}\\
\nonumber & & \mkstore{{\varphi B}^{+}, D(D \varphi x) y (\varphi
y), DGAB,S}
\hbox{\tt ]})\\
\nonumber \rightarrow_+ (q_1, &b, &  \hbox{\tt[}\mkstore{ \varphi^{+/-}, DGAB,S}\\
\nonumber & & \mkstore{{\varphi B}^{+},{(D \varphi x)}^{-}, DGAB,S}\\
\nonumber & & \mkstore{{\varphi B}^{+}, D(D \varphi x) y (\varphi
y), DGAB,S}
\hbox{\tt ]})\\
\nonumber \rightarrow_+ (q_1, &b, &  \hbox{\tt[}\mkstore{{\varphi B}^{+},{(D \varphi x)}^{-}, DGAB,S}\\
\nonumber & & \mkstore{{\varphi B}^{+}, D(D \varphi x) y (\varphi
y), DGAB,S}
\hbox{\tt ]})\\
\nonumber \rightarrow_+ (q_1, &b, &  \hbox{\tt[}\mkstore{B,{(D \varphi x)}^{-}, DGAB,S}\\
\nonumber & & \mkstore{{\varphi B}^{+}, D(D \varphi x) y (\varphi
y), DGAB,S}
\hbox{\tt ]})\\
\nonumber \rightarrow_+ (q_1, &\epsilon, &  \hbox{\tt[}\mkstore{\epsilon,B,{(D \varphi x)}^{-}, DGAB,S}\\
\nonumber & & \mkstore{{\varphi B}^{+}, D(D \varphi x) y (\varphi
y), DGAB,S} \hbox{\tt ]})
\end{eqnarray}
And the word is accepted as required.
\end{example}
