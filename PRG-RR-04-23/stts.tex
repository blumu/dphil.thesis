\section{Unrestricted term trees}
%\section{Unrestricted term trees and Caucal graphs}
As mentioned earlier, grammars can be used for generating term
trees as well as string languages. We now give the definition of a
term-tree generating grammar as introduced by \cite{KNU01, KNU02}.
We then examine the relationship between string languages and term
trees. We conclude with an application of our level-$2$
result.

\subsection{Definitions}

We shall use higher-order grammars as generators of possibly
infinite terms, viewed as trees. As in Section 2.1 we consider
simple types. Fix a typed alphabet $\Sigma$ of symbols of type
level at most $1$ (this is often referred to as a
\textbfit{signature}). Thus we can think of elements of $\Sigma$
as function symbols $f : \underbrace{o \funsp \cdots \funsp
o}_{r\geq 0} \funsp o$ of arity $\arity{f} = r$.

A \textbfit{$\Sigma$-tree} is a map $t : T \mor \Sigma$ where $T$
is a prefix-closed subset of $\makeset{1, 2, 3, \cdots}^\ast$, and
for $k \geq 0$ whenever $t(w)$ has arity $k$ then $w$ has exactly
$k$-successors in $T$, which are $w1, \cdots, wk$. We write
$\terms{\infty}{\Sigma}$ for the set of $\Sigma$-trees. Thus a
$\Sigma$-tree is just a possibly-infinite ``applicative term''
constructed using symbols from $\Sigma$. Henceforth we shall
identify \emph{finite} $\Sigma$-trees with elements of
$\terms{o}{\Sigma}$, and use them interchangeably.

\begin{definition}\rm
A (term-tree generating) \textbfit{higher-order grammar} is a 5-tuple $G =
\anglebra{N, V, \Sigma, {\cal R}, S}$ such that $N$ is a finite
set of \emph{homogeneously} typed non-terminals, with start symbol
${S : o} \in N$; $V$ is a set of typed variables; $\Sigma$ is as
above; and ${\cal R}$ is a finite set of rules of the form:
\[ F x_1 \cdots x_m \; \rightarrow \; E \]
where $F : (A_1, \cdots, A_m, o) \in N$, each ${x_i : A_i} \in V$,
and $E \in T^o(N \cup \Sigma \cup \{x_1, \cdots, x_m\})$. As with
string-language generating grammars, the notions of the level of a
rule and the level of a grammar apply. Furthermore, we say the
grammar is \textbfit{safe} if and only if the righthand side of
each rule is a safe term-in-context. Also, as we did for the
string-language setting we make a proviso that if $F \in N$ has
type $(A_1, \cdots, A_m, o)$ and $m \geq 1$, then $A_m = o$.
Following the setting suggested \cite{KNU02} we also make the
further assumption that for each non-terminal $F \in N$ there
exists \emph{exactly one} production rule with $F$ on the left
hand side.
\end{definition}

\subsubsection*{Infinite term tree or $\Sigma^\bot$-tree generated by a
grammar}

We write $\Sigma^\bot$ to mean $\Sigma \cup
\makeset{\bot}$ for a distinguished symbol $\bot : o$.  For any $t \in
\terms{A}{N \cup
\Sigma}$, we define $t^\bot$, a term in $\terms{A}{\Sigma^\bot}$,
by induction over the following rules:
\begin{itemize}
\item $f^\bot = f$ for $f \in \Sigma$
\item $F^\bot = \bot$ for $F \in N$
\item $(st)^\bot = (s^\bot t^\bot)$ if $s^\bot \not = \bot$, otherwise
$(st)^\bot = \bot$
\end{itemize}
We define ${\rightarrow_G}$, a binary relation over $\terms{}{N \cup
\Sigma}$, by recursion over the structure of $t$:
\begin{itemize}
\item $Ft_1 \cdots t_n \rightarrow_G r[t_1/x_1, \cdots, t_n/x_n]$ if
there is a rule $Fx_1 \cdots x_n \rightarrow r$, with $x_i:A_i$ and
$t_i \in \terms{A_i}{N \cup \Sigma}$ for $i = 1, \cdots, n$. \item If
$t
\rightarrow_G t'$ then $(st) \rightarrow_G (st')$ and $(tr)
\rightarrow_G (t'r)$, for applicative terms $st$ and $tr$ of the
appropriate types.
\end{itemize}

For $t \in \terms{o}{N \cup \Sigma}$ and $s \in
\terms{\infty}{\Sigma^\bot}$, we say that $t \downarrow_G s$ if:
\begin{itemize}
\item $s$ is finite and there is a finite reduction sequence $t = t_0
\rightarrow_G \cdots \rightarrow_G t_n = s$

\item $s$ is infinite and there is an infinite reduction sequence $t =
t_0 \rightarrow_G t_1 \rightarrow_G \cdots$ such that $s = \lim
t^\bot_n$
\end{itemize}

Given a grammar $G$, with start symbol $S$, we define the \textbfit{$\Sigma^\bot$-tree generated by
$G$}, written $[G]$, by
\begin{equation}
\nonumber [G] \; = \; \sup \{t \in \terms{\infty}{\Sigma^\bot} :
S \downarrow_G t\}
\end{equation}
Intuitively, the tree $[G]$ represents all possible computation paths
taken.

\begin{remark} Note that the setting outlined above for a
term-tree generating grammar differs from our definition of a
string-language generating grammar in the following ways:
\begin{enumerate}
\item Rules are no longer labelled. \item Signature symbols in
$\Sigma$ may now appear on the righthand side of a production
rule. \item We insist that for each non-terminal $F$ there is
exactly one production rule with $F$ on the lefthand side.
\end{enumerate}
These are, in fact, not substantial changes. We could easily have
defined our string-language generating grammars to conform to
items (1) and (2) by insisting that $\Sigma = \Sigma' \cup \{e\}$
where $e$ is the end of word marker, and all symbols in $\Sigma'$ are of type
$(o,o)$. We refer the reader to \cite{dMO} for a discussion of the
possible (but equivalent) definitions of string-language
generating higher-order grammars.

However, restriction (3) is necessary in the term-tree setting to
ensure that we produce exactly \emph{one} term tree and not a
\emph{language} of term trees.
\end{remark}

\subsubsection*{Higher-order pushdown tree automata}

We also introduce higher-order pushdown tree automata. The
definition we present here is taken from \cite{KNU02}.

A \textbfit{level-$n$ pushdown tree automaton} (abbreviated to
$n$PDTA) is a tuple $\anglebra{Q, \Sigma, \Gamma, q_0, \delta}$
where $Q$, $\Sigma$, $\Gamma$ and $q_0$ are as in Section 2;
however, the transition function is now defined as:
\begin{equation}
\nonumber \delta : Q \times \Gamma \rightarrow ((Q \times Op_n)
\cup TreeOp_n)
\end{equation}
where $TreeOp_n =  \{f(p_1, \cdots, p_{\arity{f}}) : f \in \Sigma
\wedge p_1, \cdots, p_{\arity{f}} \in Q\}$.\\

\begin{remark}Note that the above insists on an injective transition function: for
each $(q, Z) \in Q \times \Gamma$ there is exactly one operation
to be performed. Hence, a level-$n$ pushdown tree automaton is
always, in a sense, deterministic.
\end{remark}

A configuration is denoted by a pair $(q,s)$ where $q \in Q$ and
$s$ is an $n$-store, with initial configuration given by $(q,
\bot_n)$. We denote by $C$ the set of all configurations. Given a
configuration $(q,s)$ we say that $(q, s) \rightarrow (p, s')$ if
$(q, \mytop_1(s), (p, \theta)) \in \delta$ where $(p, \theta) \in
Q \times Op_n$ and $\theta(s) = s'$. As before, we denote by
$\rightarrow^*$ the reflexive and transitive closure of
$\rightarrow$.

Let $t: T \rightarrow \Sigma$ be a $\Sigma$-tree. A partial
function $\rho : T \rightarrow C$ defined on an initial fragment
of $T$ is referred to as \textbfit{partial run} of the automaton
on $t$ if and only the following hold:
\begin{itemize}
\item $(q_0, \bot_n) \rightarrow^* \rho(\epsilon)$ \item If $w \in
T$ and $\rho(w) = (q,s)$ then $\delta(q, \mytop_1(s)) = f( p_1,
\cdots, p_{\arity{f}})$ where $t(w) = f$ and $p_1, \cdots,
p_{\arity{f}} \in Q$. Furthermore, $(p_i, s) \rightarrow^*
\rho(wi)$ for $1 \leq i \leq \arity{f}$ when $\rho(wi)$ is
defined.
\end{itemize}
In the case that $\rho$ is total, then it is called simply a
\textbfit{run}. The tree $t$ is accepted by an automaton if there
is a run on $t$. Note that any given automaton can accept at most
one tree.\\

In the following two sections we consider conversions from string
languages to term trees and vice versa. However, in preparation
for this, we give a representation of trees by languages.

\subsubsection*{Branch languages}

Let $t: T \rightarrow \Delta$ be a $\Delta$-tree. We say that a
finite word $a_1d_1a_2d_2 \cdots d_{n-1} a_n$ where each $a_i \in
\Delta$ and each $d_i \in \{1, 2, \cdots \}$ is a
\textbfit{finite branch} in $t$ if and only if:
\begin{enumerate}
\item $d_1d_2 \cdots d_{n-1} \in T$ \item $a_i = t(d_1 \cdots
d_{i-1})$ for $i = 1, 2, \cdots, n$ \item $a_n$ has arity $0$.
\end{enumerate}
We say that an infinite word $a_1d_1a_2d_2 \cdots$ is an
\textbfit{infinite branch} in $t$ if and only if:
\begin{enumerate}
\item $d_1d_2 \cdots d_n \in T$ for each $n$ \item $a_i = t(d_1
\cdots d_{i-1})$
\end{enumerate}
Given a branch $a_1d_1a_2d_2 \cdots$ we say that $a_1a_2 \cdots$
is the \textbfit{label} of the branch.

Following Courcelle \cite{Cou83}, given a signature $\Delta$ we
form a new alphabet:
\begin{equation}
\nonumber \overline{\Delta} = \{[f,i] : f \in \Delta \wedge 1 \leq i
\leq \arity{f}\} \cup \{f : f \in \Delta \wedge \arity{f} = 0\}
\end{equation}
and we say that $w \in {\overline{\Delta}}^* \cup
{\overline{\Delta}}^w$ is in the \textbfit{branch language} of $t:
T \rightarrow \Sigma$ if and only if $w = [a_1, d_1][a_2,d_2]
\cdots $ and $a_1d_1a_2d_2 \cdots $ is a (finite or infinite)
branch of $t$. The branch language of a tree $t$ will be denoted
by $Branch^\infty(t)$. For notational convenience, we will often write
$f_i$ instead of $[f,i]$. Thus if $[h,1][h,3][h,2][f,1]b$
is a word in the branch language, we will write this as $h_1h_3h_2f_1b$.

\begin{lemma} \cite{Cou83} A $\Delta$-tree is completely specified by its
branch language.
\end{lemma}

\subsection{From string languages to term trees}

Let $G$ be a (possibly non-deterministic, possibly unsafe) grammar
that generates a string language $L(G)$ over the signature
$\Sigma$. We will convert it into a term-tree generating grammar
$G_t$ over the signature $\Sigma \cup \{2,3, \cdots n\} \cup
\{e,r\}$, where $n$ is the maximum level of branching (to be
defined later). Intuitively, the tree generated by $G_t$ will
represent \emph{all} possible words generated by $G$, and the
nodes labelled by the new signature symbols $\{2,3, \cdots, n\}$
will
represent the points at which choices are allowed to be made.\\
For example, consider the following grammar:
\begin{eqnarray}
\begin{array}{ll}
\begin{array}{rcl}
\nonumber S & \larr{a} & FHAB \\
\nonumber F\varphi x y & \larr{b} & E_1\\
\nonumber F\varphi x y & \larr{c} & E_2 \\
\nonumber F\varphi x y & \larr{d} & E_3 \\
\end{array} &
\begin{array}{rcl}
\nonumber G \varphi x & \larr{j} & D_1\\
\nonumber G \varphi x & \larr{k} & D_2\\
\nonumber H x & \larr{g} & C \\
\nonumber A & \larr{h} & e\\
\nonumber B & \larr{i} & e
\end{array}
\end{array}
\end{eqnarray}
Note there are $3$ production rules with the non-terminal $F$ on
the left hand side, and $2$ for $G$. Thus whenever we have $F s_1
s_2 s_3$ as a term we can \emph{choose} which production rule to
apply. Therefore we say that the level of branching for $F$ is
$3$, (it is $2$ for $G$, and $1$ for $H, A, B$ and $S$). We say
that the level of branching for a grammar is $n$ just in case $n$
is the maximum of the levels of branching for each non-terminal.
In order to convert the example grammar above into a term-tree
generating one, in the sense of \cite{KNU01, KNU02}, $G_t$, we use
the signature $\Sigma \cup \{2,3\} \cup \{e,r\}$, where $2: o^2
\rightarrow o$ and $3: o^3 \rightarrow o$, each symbol in $\Sigma$
is of type $o \rightarrow o$ and $e,r : o$ and we have the
following production rules:
\begin{eqnarray}
\nonumber S & \rightarrow & a(FHAB) \\
\nonumber F \phi x y &\rightarrow  & 3 (bE_1)(c E_2)( dE_3) \\
\nonumber G \phi x & \rightarrow & 2 (jD_1) (kD_2)\\
\nonumber H x & \rightarrow & gC\\
\nonumber A & \rightarrow & he\\
\nonumber B & \rightarrow & ie
\end{eqnarray}
Finally, in the case where there exists one or more non-terminals
$M$ such that $M$ does not occur on the lefthand side of any
production rule, we add the rule $M\overrightarrow{x} \rightarrow
r$, for $\overrightarrow{x}$ of the appropriate type.

It should not be too difficult to see that each ``properly ending"
branch in $[G_t]$ corresponds to a word in $L(G)$. A
\textbfit{properly ending} branch is one that is finite and whose
final node is labelled by $e$. In fact, $[G_t]$ captures exactly
those words in $L(G)$:

\begin{lemma}
$w \in L(G)$ if and only if there exists a finite branch labelled
by $w_0a_1w_1a_2 \cdots a_nw_ne$ in $[G_t]$ such that each $a_i
\in \Sigma$, $w_i \in \{2,3, \cdots m\}^*$ (where $m$ is the
branching level of $G$) and $w = a_1a_2 \cdots a_n$.
\end{lemma}

\begin{proof} Obvious.
\end{proof}

% It should not be too difficult to see that
%$G_t$ captures exactly those words generated by $G$. The nodes
%labelled by $\{2,3, \cdots, n\}$ are referred to as
%\textbfit{branching nodes}. When we encounter a branching node we
%know that depending on which edge we follow we can continue word
%in a different way.

\begin{remark} It should be clear that if the
original grammar $G$ is safe, then so is the resulting one, $G_t$.
\end{remark}

\subsubsection*{Applying KNU's decidability result}

By \cite{KNU02} the term tree generated by a safe term-tree
generating grammar has a decidable monadic second order (MSO)
theory (see \cite{Tho97} for an introduction to MSO logic).

This has obvious repercussions for a \emph{safe} string-language
generating grammar $G$. Because of the above conversion, we obtain
a tree $[G_t]$, such that each properly ending branch corresponds
to a word in $L(G)$ and vice versa. Therefore, we can obtain
several decidability results about the language $L(G)$ by virtue
of the MSO decidability of $[G_t]$.

\begin{proposition}
Some of the decidability results we can obtain are (1)
non-emptiness, (2) membership, and, if $G$ is deterministic (3)
finiteness.
\end{proposition}

\begin{proof} We give only the proof for non-emptiness; the rest can
be similarly argued. We make the conversion from $G$ to $G_t$ as outlined above and
combine it with the following MSO sentence:
\begin{equation}
\nonumber \exists y . p_{e}(y)
\end{equation}
where $p_{a}$ is the predicate ``is labelled by $a$'' for $a \in
\Sigma \cup \{2,3 \cdots m\} \cup \{e, r\}$ where $m$ is the level
of branching.
\end{proof}

Thus, the above conversion from a grammar $G$ that generates a
string language into a grammar $G_t$ that generates a term tree is
quite a useful one. In particular, for languages in the OI-hierarchy (see Section 2), we have a new and simple way to prove
decidability results.

\subsection{From term trees to string language}

Let us first introduce the definition of the
\textbfit{$\infty$-language} of
a string-language generating grammar $G$, ${L^\infty}(G)$.\\

Let $G = \anglebra{N, V, \Sigma, {\cal R}, S, e}$ be level-$n$
grammar that generates a string language. We say that a derivation sequence $S
= P_1 \larr{a_1} P_2 \larr{a_2} \cdots$ is \textbfit{valid} if
\begin{itemize}
\item it is finite and ends in $e$; or \item it is infinite.
\end{itemize}

If $S = P_1 \larr{a_1} P_2 \larr{a_2} \cdots$ is a valid
derivation sequence, then we say that the word $a_1a_2 \cdots$ is
in the $\infty$-language of $G$.

%In particular, we write:
%\begin{equation}
%\nonumber {L^\infty}(G) = \{w : S = P_1 \larr{a_1} P_2
%\larr{a_2} \cdots \mbox{ is valid and } w = a_1a_2 \cdots \}
%\end{equation}

\begin{remark}Thus, we now allow for a string-language generating
grammar to generate $\omega$-words \emph{in addition to} finite
words.

In the literature, $\omega$-languages (consisting only of
infinite words) are more commonly defined in terms of a machine
model with an associated acceptance condition; such as Buchi,
Muller, and so on (see e.g. \cite{GTW02}). Thus, although our definition
seems somewhat of a departure from this more standard notion, we
will see that it is sufficient for our purposes.
\end{remark}

%In \cite{KNU01, KNU02}, given a term-tree generating grammar $G$,
%the resulting tree $[G]$ is a term tree over the signature
%$\Sigma^\bot = \Sigma \cup \{\bot\}$ where $\bot : o$. Thus, we
%introduce the following definition:

\begin{definition} Let $G$ be a term-tree generating grammar, and let
$[G]$ be the resulting term tree over the signature $\Sigma^\bot$.
We say that a word $w$, over the alphabet $\overline{\Sigma}$, is
in the \textbfit{bottomless branch language} of $[G]$ if and only
if:
\begin{itemize}\item $w$ is an infinite word in $Branch^\infty([G])$,
or \item $w\bot$ is a finite word in $Branch^\infty([G])$, or
\item $w$ is a finite word in $Branch^\infty([G])$ that \emph{does
not} end in $\bot$.
\end{itemize}
\end{definition}

\begin{lemma} If $t$ is a tree over $\Sigma^\bot$ (as above),
then $t$ is completely defined by its bottomless branch language.
\end{lemma}

\begin{proof} Obvious.
\end{proof}

\begin{lemma}\label{lem:ts} Given a term-generating grammar $G = \anglebra{N,
V, \Sigma, {\cal R}, S}$, we can convert it into a string-language
generating grammar $G_s$ such that $w \in {L^\infty}(G_s)$ if and
only if $w$ is in the bottomless branch language of $[G]$
%a  string-language generating grammar $G_s$ such that $w \in {L^\infty}(G_s)$ if and only if
%\begin{itemize}
%\item $w$ is an infinite branch in $[G]$, or \item $w\bot$ is a
%finite branch in $[G]$, or \item $w$ is a finite branch in
%$[G]$.
%\end{itemize}
\end{lemma}

\begin{proof} Let $G_s = \anglebra{N_s, V_s, \overline{\Sigma}, {\cal R}_s,
S, e}$, where $N_s = N \cup \{S_f : f \in \Sigma\}$ and the set of
rewrite rules, ${\cal R}_s$, is constructed as follows. For each
existing production rule, $F \overrightarrow{x} \rightarrow E$, in
${\cal R}$ we simply rewrite this as:
\begin{equation}
\nonumber F \overrightarrow{x} \larr{\epsilon} E'
\end{equation}
where $E'$ is $E$ with each occurrence of a signature symbol $f$
replaced by $S_f$. Furthermore, for each signature constant $f$ of
arity $\arity{f} > 0$ we add the production below for each $i = 1,
\cdots, \arity{f}$:
\begin{eqnarray}
\nonumber S_f x_1 \cdots x_{\arity{f}} & \larr{f_i} & x_i
\end{eqnarray}
In case $\arity{f} = 0$, then we simply add
\begin{eqnarray}
\nonumber S_f & \larr{f} & e
\end{eqnarray}
Finally $V_s$ can easily be inferred from the above.
\end{proof}

Note that for a given term-tree generating grammar $G$, the
resulting grammar $G_s$ will always be deterministic. This is
because of the restriction on $G$ that for each non-terminal $N$
there is exactly one production rule with $N$ on the lefthand side.

\begin{example}
As an example, consider the following term-tree generating
grammar, with $\Sigma = \{h,g,f,a,b\}$ where $h: (o,o,o,o), \; f:
(o,o,o), \; g: (o,o)$ and $a, b : o$.
\begin{eqnarray}
\nonumber S & \rightarrow & Dgab \\
\nonumber D\varphi x y & \rightarrow & h(D(D\varphi x)y (\varphi y)) (H (fy) x) (\varphi b)\\
\nonumber H \varphi x & \rightarrow & \varphi x
\end{eqnarray}
Applying the above conversion gives us the grammar given in
Example~\ref{ex:ex1}.
\end{example}

\subsubsection*{Applying our result}

Unfortunately, our level-$2$ result offers little in the way of
solving either of the key open problems regarding the safety
restriction for term-tree generating grammars. In particular, they
are, as stated by Knapik \emph{et al.}:
\begin{enumerate}
\item Can every term tree generated by an unsafe grammar of
level-$n$ be generated by a safe grammar of the same level? Or of
another level? \item Is safety a necessary requirement to ensure
MSO-decidability?
\end{enumerate}

The reason we cannot apply our result is due to the introduced
non-determinism. As indicated above, we \emph{can} convert a
term-tree generating grammar $G$ into a grammar $G_s$ such that
${L^\infty}(G_s)$ is the bottomless branch language of $G$. At
level $2$, by our result, we can even go on to construct a $2$PDA
such that it accepts ${L^\infty}(G_s)$. However, the $2$PDA will,
in general, be non-deterministic and therein lies the problem.

%For example, recall the grammar in Example \ref{ex:ex1}: the set
%of guesses to produce the branch $h_1h_3h_2f_1b$ are incompatible
%with those used to produce the branch $h_1h_3h_2f_2a$. Thus,
%although both words will be present in ${L^\infty}(G)$,
%there will be no way to associate that they share the same prefix
%and thus are actually the same path up
%to the node $f$.\\

However, as a small consolation we mention two possible avenues
that we believe are worthy of investigation. The first is a
conjecture, which, if proven correct will show that safety
\emph{is} a restriction in terms of generating power for the
term-tree setting. The second introduces the temporal logic,
existential LTL, and shows that it is decidable for term trees
generated by unsafe level-$2$ grammars by way of our result.

\subsection{Urzyczyn's Language: a conjecture}

We have shown that the language $U$ is accepted by a
non-deterministic $2$PDA. We conjecture that it cannot be accepted
by a deterministic $2$PDA. If our conjecture is true then we will
have an example of an inherently unsafe term tree, i.e. an unsafe level-$2$ grammar whose term
tree cannot be generated by a safe level-$2$ grammar.

\begin{figure}[h]
\begin{diagram}[height=1.5em, width=1em]
&& & & \pq & & \\
&& & & \dTo{}{}& & \\
&& & & 3 & & \\
&& & \ldTo{}{} & \dTo{}{}& \rdTo{}{} & \\
&& \pq&  & \oa &  &  *\\
&& \dTo{}{} &  & \dTo{}{} &  &  \dTo{}{}\\
&& 3 &  & r &  &  e\\
& \ldTo{}{} & \dTo{}{} & \rdTo{}{} &  &  &  \\
\pq&  & \oa & & * &  &  \\
\dTo{}{} &  & \dTo{}{} & & \dTo{}{} &  &  \\
3 &  & 3 & & * &  &  \\
\vdots &  & \vdots & & \dTo{}{}  &  &  \\
 &  &  & & e  &  &  \
\end{diagram}
\caption{The term tree $[G_{U}]$\label{fig:utree}}
\end{figure}

Let us apply the conversion from Section 7.2 to the unsafe
string-language generating grammar for $U$ in Section 3, to give
us a new grammar, $G_{U}$. In particular, our signature is $\Sigma
= \{\pq : (o,o), \; \oa: (o,o), \; *: (o,o), \; 3: (o,o,o,o), \; e
: o, \; r : o\}$ and we have the following production rules:

\begin{eqnarray}
\nonumber S & \rightarrow & \pq DGEEE \\
\nonumber D \varphi x y z & \rightarrow & 3 (\pq D(D \varphi x) z (Fy) (Fy)) (\oa \varphi y x) (* z)\\
\nonumber Fx & \rightarrow & \ast x \\
\nonumber E & \rightarrow & e\\
\nonumber G & \rightarrow & r
\end{eqnarray}
The term tree generated by $G_U$ is shown in
Figure~\ref{fig:utree}.

\begin{proposition}Suppose that $[G_U]$ can be
generated by a safe level-$2$ (term tree) grammar. Then it must be
the case that the language $U$ can be accepted by a deterministic
$2$PDA.
\end{proposition}

%\begin{proof} By \cite{KNU02}, if the term tree in Fig~\ref{fig:utree}
%is indeed generated by a safe level-$2$ term-tree generating
%grammar then there exists a $2$PDTA $A = \anglebra{Q, \Sigma,
%\Gamma, q_0} $ that accepts $[G_U]$.
%
%We will see that we can easily modify $A$ into $A_s =
%\anglebra{Q_s, \Sigma, \Gamma, q_0, \delta_s, F_s}$ where $A_s$ is
%a deterministic $2$PDA (as described in Section 2.4).
%
%Analysing $[G_U]$ we can easily see that whenever we have
%$\delta(q, a) = (3, p_1, p_2, p_3)$ it must be the case that $p_i
%\not = p_j$ for $1 \leq i < j \leq 3$ as each immediate successor
%of $3$-node is distinct. It is this property that allows us to
%convert $A$ into a deterministic pushdown automaton that accepts
%$U$. We let $Q_s = Q \cup \{q_r, q_e\} \cup \{q^a : q \in Q \wedge
%a \in \Sigma\}$ and $F = \{q_e\}$. Furthermore, for each $(q, Z,
%(p,op)) \in \delta$ we add the rule $(q, \epsilon, Z, p, op)$ to
%$\delta_s$. However, if $f \in \Sigma^{(o,o)}$, then for each $(q,
%Z, f(p_1)) \in \delta$ we add the rule $(q, f, Z, p_1, \id)$ to
%$\delta_s$. What remain are the cases concerning signature symbols
%$3, e, r$.
%
%For every $(q, Z, (3(p_1, p_2, p_n)) \in \delta$, we replace this,
%in $\delta_s$, with the family of rules:
%\begin{eqnarray}
%\nonumber \delta(q, \pq, Z) & = & (p_1^{\pq}, \id) \\
%\nonumber \delta(q, \oa, Z) & = & (p_2^{\oa}, \id)\\
%\nonumber \delta(q, *, Z) & = & (p_3^{*}, \id)
%\end{eqnarray}
%where, for each $a \in \Sigma$,  $(p^a, \epsilon, Z, q^a, op) \in
%\delta_s$ if and only if $(p, Z, (op, q)) \in \delta$ and $(p^a,
%a, Z, q, id) \in \delta_s$ if and only if $(p, Z, (a,q)) \in
%\delta$. Recall that the immediate successors of a $3$-node are
%always $\pq, oa,$ or $*$, all of which are of arity $1$. The
%intuition behind this is that we preempt the $2$PDA.
%
%For each $(q, Z, r) \in \delta$ we convert this into $(q,
%\epsilon, Z, q_R, \id)$ where $q_R$ is a new state not in $F$. In
%particular, $F = \{q_e\}$ where $q_e$ is another new state, and we
%have $(q, \epsilon, Z, q_e, \id) \in \delta_s$ if and only if $(q,
%Z, e) \in \delta$.
%\end{proof}


\begin{proof} By \cite{KNU02}, if $[G_U]$ is indeed generated by a safe level-$2$
grammar then there exists a $2$PDTA $A = \anglebra{Q, \Sigma,
\Gamma, q_0, \delta} $ that accepts $[G_U]$.

Analysing $[G_U]$ we can easily see that whenever $\delta(q, a) =
3(p_1, p_2, p_3)$ it must be the case that $p_i \not = p_j$ for $1
\leq i < j \leq 3$ as each immediate successor of $3$-node is
distinct. It is this property that allows us to convert $A$ into a
deterministic pushdown automaton $A_s$ that accepts the language
$U$. Let $A_s = \anglebra{Q_s, \Sigma - \{3,e,r\}, \Gamma, q_0,
\delta_s, F_s}$ where let $Q_s = Q \cup \{q_r, q_e\} \cup \{q^a :
q \in Q \wedge a \in \Sigma\}$. Furthermore, $\delta_s$ is
constructed from $\delta$ as follows:
\begin{itemize}
\item for each $(q, Z, (p,op)) \in \delta$ we add the rule $(q,
\epsilon, Z, p, op)$ to $\delta_s$. \item if $f \in
\Sigma^{(o,o)}$, then for each $(q, Z, f(p_1)) \in \delta$ we add
the rule $(q, f, Z, p_1, \id)$ to $\delta_s$. \item For every $(q,
Z, 3(p_1, p_2, p_n)) \in \delta$, we add the following the family
of rules to $\delta_s$:
\begin{eqnarray}
\nonumber \delta(q, \pq, Z) & = & (p_1^{\pq}, \id) \\
\nonumber \delta(q, \oa, Z) & = & (p_2^{\oa}, \id)\\
\nonumber \delta(q, \ast, Z) & = & (p_3^{\ast}, \id)
\end{eqnarray}
where, for each $a \in \{\pq, \oa, \ast \}$,  $(p^a, \epsilon, Z, q^a,
op) \in \delta_s$ if and only if $(p, Z, (q, op)) \in \delta$ and
$(p^a, \epsilon, Z, q, id) \in \delta_s$ if and only if $(p, Z,
a(q)) \in \delta$. \item For each $(q, Z, r) \in \delta$ we have
$(q, \epsilon, Z, q_r, \id) \in \delta_s$ where $q_r$ is a new
state not in $F$. In particular, $F = \{q_e\}$ where $q_e$ is
another new state, and we have $(q, \epsilon, Z, q_e, \id) \in
\delta_s$ if and only if $(q, Z, e) \in \delta$.
\end{itemize}
The key to the construction lies in examining what happens in the
$2$PDTA when we are in a configuration $(q,Z)$ such that
$\delta(q,Z) = 3(p_1, p_2, p_3)$. As mentioned, the $p_i$'s are
distinct, thus, in the translation to
a deterministic $2$PDA, we will ignore the $3$, and depending on
the current input symbol we change the state to $p_1^{\pq},
p_2^{\oa}$ or $p_3^{\ast}$. Note the superscripted states: these are
required because we must take care of the fact that we have
consumed input ``prematurely". In particular, the set of states
$Q^{\pq} = \{q^{\pq} : q \in Q\}$ are such that the transitions
between them match exactly the $\epsilon$-transitions of $\delta$.
But we are not allowed to leave $Q^{\pq}$ until we reach a
configuration $(q^{\pq}, w, s)$ where $(q, top_1(s), \pq(p_1)) \in
\delta$, and, in this case, we leave $Q^{\pq}$ via an
$\epsilon$-transition $(p_1, w, s)$.
\end{proof}

\begin{conjecture} $U$ cannot be accepted by a deterministic
$2$PDA, and hence, there exists a term-tree generated by an unsafe
level-$2$ grammar that cannot be generated by any safe level-$2$
grammar.
\end{conjecture}

\subsection{Existential LTL}

Let $G = \anglebra{N, V, \Sigma, {\cal R}, S}$ be a level-$2$
unsafe term-tree generating grammar. By Section 7.3 we can convert
this into a level-$2$ unsafe string-language generating grammar
$G_s$, such that ${L^\infty}(G_s)$ is the bottomless branch
language of $G$.

Now, by our level-$2$ result, we can convert $G_s$ into a safe
grammar $G_{safe}$ that generates this same bottomless branch
language. In general, however, $G_{safe}$ will contain ``dead"
non-terminals, such that for each ``dead" non-terminal $F$, there
will be no production rule that contains $F$ on the left hand
side. These non-terminals arises solely because of abortive
computations of the non-deterministic $2$PDA caused by incorrect
guesses.

Finally, by the conversion\footnote{It should be clear that the
conversion in Section 7.2 also works when we consider
$\infty$-languages instead of languages of finite words.} in
Section 7.2 we obtain a term-tree generating grammar $G_t$ with
the following property. Every infinite or properly ending branch
in $[G_t]$ corresponds to a word in $L^\infty(G_{safe})$. In
particular, each branch in $[G_t]$ that does not end in
$r$\footnote{We intend the $r$ to mean ``reject".}
corresponds to a branch in $[G]$! Summing up (works only for level $2$):\\

\begin{eqnarray}
\nonumber G & & \mbox{unsafe term-tree generating grammar} \\
\nonumber \downarrow & & \mbox{(Section 7.3)}\\
\nonumber G_s &  &\mbox{unsafe string-language generating grammar} \\
\nonumber \downarrow & & \mbox{(Theorem 4.1)}\\
\nonumber G_{safe} & &  \mbox{safe string-language generating grammar} \\
\nonumber \downarrow & & \mbox{(Section 7.2 )}\\
\nonumber G_t & & \mbox{safe term-tree generating grammar}
\end{eqnarray}

%\begin{equation}
%\nonumber\begin{array}{ccccccc}
%G & \longrightarrow & G_s & \longrightarrow & G_{safe} & \longrightarrow & G_t\\
%\mbox{unsafe term-tree grammar} & & \mbox{unsafe string-language grammar} & & \mbox{safe string-language grammar} & & \mbox{safe term-tree grammar}
%\end{array}
%\end{equation}

As an example, suppose that $f1f2g1a$ is a branch in $[G]$, this
will manifest itself as a branch $w_0f_1w_1f_2w_3g_1w_4aw_5e$ in
$[G_t]$, where each $w_i \in \{2, 3, \cdots m\}^*$ for some fixed
$m$. \\

Note that $[G_t]$ \emph{does} possess a decidable MSO theory as it
is generated by a safe grammar. Thus, we can certainly model-check
certain path properties of $[G]$ via $[G_t]$. As an example, we
consider \emph{existential LTL}. In fact, we can probably
model-check more properties than these, but these are sufficient
to show that some useful model-checking is viable.

The formulae of existential LTL are the following. If $\Sigma$ is
the set of signature symbols of $G$, then we have the following
formulae:

\begin{eqnarray}
\nonumber \phi := p_f \; | \; {\tt true} \; | \; {\tt false} \; |
\; \phi \wedge \phi \; | \; \neg \phi \; | \; X \phi \; | \; \phi
U \phi
\end{eqnarray}
where $f \in \Sigma^\bot$. $X$ means ``next" and $U$ means ``until" in
the usual sense (see \cite{Var95}). In particular, we say that
$[G] \models_{\exists} \phi$ if there exists a path $\pi$ in $[G]$
such that $\phi$ holds. Thus, we can state properties such as,
``There exists a path such that $f$ never holds," or ``There
exists a path such that whenever $g$ holds, eventually $h$ holds."
Hence the name \emph{existential} LTL. Note that this is useful
for model checking safety\footnote{This has nothing to do with the
syntactic restriction of safety.} properties.

To show that such a property $\phi$ is indeed decidable for $[G]$
all one needs to do is give a corresponding formulae $\phi_t$ and
show that it holds for $[G_t]$.

For example, suppose that we wish to establish that $[G]
\models_{\exists} X \phi$. All we need do is decide whether the
following MSO formula holds true for $[G_t]$: $\exists X. path(X)
\wedge \exists z . z \not = root(X) \wedge z \in X \wedge \phi(z)
\wedge (\bigvee_{f \in \Sigma, i \leq \arity{f}} p_{f_i}(z))
\wedge \forall y . (y \in X \wedge (root(X) <y < z) \rightarrow
\vee_{i \in \{2, 3, \cdots m\}} p_i(y))$, where the predicate
$path(X)$ means ``$X$ is a prefix-closed maximal path that does
not end in the signature constant $r$" and $root(X)$ is the least
node in $X$ labelled with a signature constant.

\begin{remark} If $\pi$ is a branch in $[G]$ labelled by $a_1 \cdots a_n \bot$ then either
\begin{itemize}
\item There exists a branch labelled by $w_0 a_1 w_1 \cdots
w_{n-1} a_n w_n \bot$ in $[G_t]$ where $w_0, \cdots, w_n$ are
finite words over $\{2, 3, \cdots, m\}$; or \item There exists a
branch labelled by $w_0 a_1 w_1 \cdots w_{n-1} a_n w_n$ in $[G_t]$
where $w_0, \cdots, w_{n-1}$ are finite words over $\{2, 3,
\cdots, m\}$ and $w_n$ is an infinite word over $\{2, 3, \cdots,
m\}$.
\end{itemize}
Thus, it is possible (and easy) to have a predicate $p_\bot$.
\end{remark}

%\begin{remark} If $\pi\bot$ is a branch in $[G]$, then there will
%be a branch $w_0a_1w_1 \cdots w_{n-1}a_nw_n$ in $[G_t]$ where
%$w_0, \cdots, w_{n-1}$ are finite words over $\{2, 3, \cdots, m\}$
%and $w_n$ is an infinite word over $\{2, 3, \cdots, m\}$. Thus, it
%is possible (and easy) to have a predicate $p_\bot$.
%\end{remark}
