\documentclass[RR]{OUCLdocument}
%\documentclass[11pt]{article}
\usepackage{a4wide}
\usepackage{rawfonts}
\usepackage{amsmath}
\usepackage{amsthm}
\usepackage{amssymb}
\usepackage{graphicx}
\usepackage[all]{xy}
\usepackage{pstricks,pst-node}
\usepackage{shadowbox}
\usepackage{diagrams}


\reportnumber{23}
\theyear{2004}

\fboxsep  = 5pt % making the shadow box smaller (default = 10pt)
\shadowwidth = 2pt % making the shadow smaller (default = 4pt)

%ftp://ftp.dcs.qmw.ac.uk/pub/tex/contrib/pt/diagrams/

\newcommand{\diff}{\mathrm{diff}}
\newcommand{\ad}{\mathrm{ad}}
\newcommand{\Leaves}{\mathrm{Leaves}}
\newcommand{\bigF}{\mathcal{F}}
\newcommand{\bigA}{\mathcal{A}}
\newcommand{\bigT}{\mathcal{T}}
\newcommand{\bigB}{\mathcal{B}}
\newcommand{\bigU}{\mathcal{U}}
\newcommand{\bigG}{\mathcal{G}}
\newcommand{\bigS}{\mathcal{S}}
\newcommand{\bigL}{\mathcal{L}}
\newcommand{\Subf}{\mathrm{Sub}}
\newcommand{\MSOL}{\mathrm{MSOL}}
\newcommand{\mrm}[1]{\; \mathrm{#1} \;}
\newcommand{\Force}{\mathrm{Force}}
\newcommand{\underarrow}[1]{\ensuremath{\underset{\rightarrow}{\text{#1}}}}
\newcommand{\anglebra}[1]{\langle\,{#1}\,\rangle}
\newcommand{\privatenote}[1]{\noindent\makebox[\textwidth][l]{\hrulefill}
\noindent\textbf{\small Note.}~~{\small #1}\\
\makebox[\textwidth][l]{\hrulefill}}
\newcommand\textbfit[1]{{\bf\em #1}\index{#1}}
\newcommand\larr[1]{\stackrel{{#1}}{\longrightarrow}}
\newcommand{\makeset}[1]{\{\,{#1}\,\}}
% rule with a lhs label
\newcommand\lprule[3]{$\makebox[1.6cm][l]{(#1)}
                             \displaystyle{\displaystyle #2 \over
                                           \displaystyle #3}$}
% rule with a lhs label and a rhs condition
\newcommand\lprulec[4]{$\makebox[1.6cm][l]{(#1)}
                             \displaystyle{\displaystyle #2 \over
                                           \displaystyle #3}\quad\makebox[1.7cm][l]{#4}$}
\newcommand\prulec[3]{$
                             \displaystyle{\displaystyle #1 \over
                                           \displaystyle #2}\quad\makebox[1.7cm][l]{#3}$}
% rule without a lhs label
\newcommand\prule[2]{$\displaystyle{\displaystyle #1 \over
                                          \displaystyle #2}$}
\newcommand   \myendproof{{%        set up
           \parfillskip=0pt            % so \par doesnt push \square to left
           \widowpenalty=10000         % so we dont break the page before \square
               \displaywidowpenalty=10000  % ditto
               \finalhyphendemerits=0      % TeXbook exercise 14.32
        %
        %                 horizontal
        \leavevmode                 % \nobreak means lines not pages
           \unskip                     % remove previous space or glue
          \nobreak                    % don't break lines
                \hfil                       % ragged right if we spill over
        \penalty50                  % discouragement to do so
        \hskip2pt                   % ensure some space
        \null                       % anchor following \hfill
        \hfill                      % push \square to right
         $\square$                   % the end-of-proof mark
        %
        %                   vertical
        \par                        % build paragraph
        \penalty-200                % prefer proofs with statements
        \smallskip                  % space after
          }
         }
\newcommand\mor{\longrightarrow}
\newcommand\oq{{\sf \textbf{[}}}
\newcommand\oa{{\sf \textbf{)}}}
\newcommand\pq{{\sf \textbf{(}}}
\newcommand\pa{{\sf \textbf{]}}}
\newcommand{\mklink}[2]{{#1}^{\langle{#2}\rangle}}
\newcommand\compose{\mathbin{\hbox{\boldmath{$;$}}}}
\newcommand{\funsp}{\rightarrow}
\newcommand\level[1]{{\sf level}(#1)}
\newcommand\seq[2]{{{#1} \vdash {#2}}}
\newcommand\blambda{\hbox{\boldmath $\lambda$}}
\newcommand\lterm[2]{{\blambda{#1}.{#2}}}
\newcommand\defined{\mathbin{{\buildrel{\rm \scriptscriptstyle
                   def}\over{=}}}}
\newcommand\mkstore[1]{\hbox{\tt [}{#1}\hbox{\tt ]}}
\newcommand\push{{\sf push}}
\newcommand\pop{{\sf pop}}
\newcommand\repl{{\sf repl}}
\newcommand\mytop{{\sf top}}
\newcommand\id{{\sf id}}
\newcommand\arity[1]{{\sf ar}(#1)}
\newcommand\terms[2]{{\cal T}^{#1}(#2)}
\newcommand\head[1]{{\sf head}\,{#1}}
\newcommand\tail[1]{{\sf tail}\,{#1}}
\newcommand\args[1]{\hbox{\it Args}\,(#1)}
\renewcommand\exp{{\it Exp}}
\newcommand\stackl{\hbox{\tt [}}
\newcommand\stackr{\hbox{\tt ]}}
\newcommand  \bohm{\mathrel{\lower.2ex
                \hbox{${\stackrel{\sqsubset}{\scriptscriptstyle \sim}}$}}}



\newtheorem{theorem}{Theorem}[section]
\newtheorem{fact}[theorem]{Fact}
\newtheorem{rules}[theorem]{Rule}
\newtheorem{lemma}[theorem]{Lemma}
\newtheorem{proposition}[theorem]{Proposition}
\newtheorem{corollary}[theorem]{Corollary}
\newtheorem{notation}[theorem]{Notation}
\newtheorem{claim}[theorem]{Claim}
\newtheorem{conjecture}[theorem]{Conjecture}
\newtheorem{question}[theorem]{Question}

\theoremstyle{definition}
\newtheorem{example}[theorem]{Example}
\newtheorem{definition}[theorem]{Definition}

\theoremstyle{remark}
\newtheorem{remark}[theorem]{Remark}

\renewcommand\arraystretch{1.3}

\title{Safety is not a restriction at level 2\\ for string languages}
\date{}
\author{K.~Aehlig \and J.~G.~de Miranda \and C.-H.~L.~Ong}

\begin{document}

\maketitle

%\begin{center}
%{\small
%  \shbox{Draft of \today}}
%\end{center}

\begin{abstract}
Recent work by Knapik, Niwinski and Urzyczyn \cite{KNU02} has revived
interest in the connexions between higher-order grammars and
higher-order pushdown automata. Both devices can be viewed as
definitions for term trees as well as string languages. In the latter
setting we recall the extensive study by Damm \cite{Dam82}, and Damm
and Goerdt \cite{DG86}. There it was shown that the language of a
level-$n$ higher-order grammar is accepted by a level-$n$ higher-order
pushdown automaton subject to the restriction of \emph{derived types},
more recently rebranded as \emph{safety}. We show that at level 2, if
a string language is generated by an unsafe grammar, there is a
(level-2, non-deterministic) safe grammar that generates the same
language.  Thus safety is not a restriction for level-2 string
languages.

\end{abstract}

\tableofcontents

% Section: Introduction
\section{Introduction}

The result we present in this paper concerns safety -- a
syntactic restriction introduced by Knapik \emph{et al.}
\cite{KNU01, KNU02}.  They define higher-order grammars as
generators of (possibly infinite) term trees. They show
that if a grammar is safe then the term
tree produced by the grammar enjoys a decidable monadic second
order (MSO) theory. Moreover, safety ensures that the term tree produced
can also be accepted by a higher-order pushdown automaton of the
same level, and vice versa. By providing two characterisations of a
new infinite hierarchy of infinite trees with decidable MSO theories,
their result pushes the frontier of decidability to new
boundaries. (This hierarchy actually coincides with a Caucal Hierarchy
of deterministic term trees \cite{Cau02}.) It is intriguing that
safety \emph{appears} to be key to such good algorithmic behaviour and
desirable properties. In particular it leaves us with two natural
open questions. First, is the restriction of safety essential for MSO
decidability? Secondly, does safety really reduce the generating power of a grammar?

As indicated by Knapik \emph{et al.}, it is interesting to note
that the restriction of safety has previously appeared in the
literature in another setting and under a different name. In the
earlier papers of \cite{Dam82, DG86} higher-order grammars and
higher-order pushdown automata were introduced not as definitional
devices for term trees, but as generators, respectively acceptors,
of string languages. Thus Damm defined an infinite hierarchy of languages,
called the OI-hierarchy, which some consider a more natural,
infinite alternative to the Chomsky Hierarchy \cite{Cho59}. Damm's
grammars are rewrite relations over expressions that are required to
be objects of ``derived types''. An analysis of
his definition reveals that the constraint of ``derived types" is
equivalent to the requirement that all types be \emph{homogeneous} and
the grammar be \emph{safe}, both in the sense of Knapik \emph{et al}. Assuming the grammar makes use of only homogeneous types (which all
definitions in the literature do), it follows that safety and
derived types are equivalent. From now on, we will only use the
term ``safe".

Damm and Goerdt went on to show that if a language is generated by a
safe grammar, then it can be accepted by a higher-order pushdown
automaton of the same level (and vice versa). Another
``good" property Damm illustrates of his safe grammars is that
each level of the hierarchy is a full AFL. However, to date, no
results exist for unsafe grammars. This is surprising -- but
perhaps less so as Damm's definition of a higher-order grammar was
such that the safety restriction was ``built-in" to it. However, in
this paper our definition of a string-language generating grammar will
follow Knapik \emph{et al.} in that we will allow for both safe and
unsafe grammars.

%Revitalised by Knapik \emph{et al.}'s questions and the increasing
%evidence that safety may be more than an ad-hoc restriction, we
%then pose the same question to the string-language setting. In
%particular, does safety really reduce the generating power of a
%(string-language generating) grammar? Not only is this an
%interesting language-theoretic question in its own right, but we
%believe this may also contribute to the understanding of what
%safety really means for both the string-language and term-tree
%setting.

This paper is concerned with the second of the open problems mentioned
at the end of the opening paragraph. We show that we can effectively
transform a level-$2$ grammar (whether safe or not) into a
(non-deterministic) level-2 pushdown automaton that accepts the same
language. Thus it follows from \cite{DG86} that every level-2 string
language that is generated by an unsafe grammar can also by generated
by a (level-2, non-deterministic) safe grammar. In this sense, we say that
safety is not a restriction at level $2$.  As a bonus, we then apply
our result to extract decidability results for level-$2$ unsafe
term-tree generating grammars.


%In \cite{Dam82} the OI-hierarchy was introduced. It is often
%perceived as a more natural, infinite, alternative to the Chomsky
%hierarchy of languages. The first three classes of the
%OI-hierarchy are the regular, context-free, and indexed languages
%(in the sense of \cite{Aho68}). Several characterisations of this
%hierarchy exist. We mention two: the higher-order grammars,
%subject to the restriction of \emph{derived types}, and the
%higher-order pushdown automata. It was shown \cite{DG86} that the
%language of a level-$n$ grammar subject to the restriction of
%derived types is accepted by a level-$n$ pushdown automaton and
%vice versa.
%
%This hierarchy has recently generated much interest due to results
%by Knapik, Niwinski and Urzyczyn \cite{KNU01, KNU02}. They
%re-introduce higher-order grammars and higher-order pushdown
%automata not as definitional devices for string languages, but as
%(respectively) generators and acceptors of term trees. In
%particular, they have shown that subject to a restriction known as
%\emph{safety}, a higher-order grammar generates a term tree with a
%decidable monadic second order theory. Furthermore, safe
%higher-order grammars are equivalent to higher-order pushdown
%automata of the same level, and vice versa.
%
%It is interesting to note that the safety restriction on
%higher-order grammars as introduced by \cite{KNU02} is similar to
%the restriction of derived types in \cite{Dam82}. This is not
%surprising considering the equivalence result for higher-order
%pushdown automata and the grammars using both sets of definitions.
%In fact, ``derived types" may be thought of as an instance of the
%safety restriction.
%
%A key open question, as recently posed by Knapik, Niwinski, and
%Urzyzczyn is whether safety is a proper restriction for
%higher-order grammars. In other words, can every tree generated by
%an unsafe higher-order grammar be generated by a safe higher-order
%grammar? This question is also unanswered in the language
%theoretic setting. However, in this paper we make a first step
%towards answering this question. In particular, we will show that
%for the language theoretic setting:
%\begin{theorem}
%If $G$ is a higher-order grammar at level $2$ that is not assumed
%to be safe, then there exists a non-deterministic level $2$
%pushdown automaton that accepts the same language. Moreover, the
%conversion from higher-order grammar to higher-order pushdown
%automaton is effective.
%\end{theorem}
%Combining this with the result on the equivalence between
%higher-order grammars and higher-order pushdown automata we then
%have:
%\begin{corollary}
%Safety is not a restriction at level $2$ for string languages.
%\end{corollary}

%Our proof of this result is decomposed into two transformations.  The
%first transformation shows that every level 2 grammar can be accepted
%by a machine we call the level 2 pushdown automaton with pointers
%(2PDAP). We then show that every 2PDAP can be simulated by a
%non-deterministic level 2 pushdown automaton.
%
%We will also introduce another intermediary device known as the
%pointer machine. This machine was originally introduced as an aid to
%understanding higher-order grammars as well as giving a possible
%intuitive implementation of such grammars. We make use of it in our
%proof, but we prove an additional result concerning containment in the
%context sensitive languages for $\epsilon$-free higher-order grammars
%(both safe and unsafe). We feel this is of interest because it is not
%yet clear where the OI-hierarchy (viewed in the language-theoretic
%sense) stands in relation to the context sensitive.

\subsection*{Related work}
In a recent preprint \cite{Blu04}, Blumensath has given the first
``pumping lemma'' for languages accepted by higher-order pushdown
automata. Following intricate surgeries on runs on an automaton, his
work gives conditions under which they can be ``pumped''. These
conditions are expressed in terms of the length of a given run and the
size of the stacks of each configuration.


% Section: Higher-order grammars, safety and higher-order PDAs
\section{Higher-order grammars, safety and higher-order PDAs}
% preliminaries: higher-order grammars and higher-order pushdown automata, and the result in KNU??

In this section we introduce higher-order grammars (HOGs) and
higher-order pushdown automata (HOPDAs). Both devices have
appeared previously in the literature \cite{Dam82, DG86, KNU02,
Eng91}, however, under two sets of definitions.

In \cite{Dam82, DG86, Eng91}, HOGs and HOPDAs are introduced as
generators and acceptors of \emph{string languages} respectively.
Thus each HOG (HOPDA) generates (accepts) a language of strings. In contrast, Knapik \emph{et al.} \cite{KNU01, KNU02} introduce
HOGs and HOPDAS as generators and acceptors of \emph{term trees}
respectively. In particular, each HOG (HOPDA) generates (accepts)
exactly one term tree.

As stated in the introduction, our result concerns only the
string-language setting. Therefore, unless otherwise stated, we
will understand higher-order grammars and higher-order pushdown
automata to be definitional devices for string languages. However,
in Section 7 we will relate our result to the term-tree setting of
\cite{KNU01, KNU02}.

%\begin{remark} \label{stringsvtrees} Although the alternative
%definition of higher-order grammars and higher-order pushdown automata
%does not feature much in this paper, we will make references to it in
%future sections.  In particular, we will have to compare the two
%definitions. When such a situation arises, we will add the subscript
%$s$ or $t$ when we want to make it absolutely clear which set of
%definitions we are employing. Thus, we write $HOG_s$ and $HOPDA_s$
%when these devices are used for defining languages of
%strings. Similarly, we write $HOG_t$ and $HOPDA_t$ when these devices
%are used for defining term trees.
%\end{remark}

\subsection{Safe $\lambda$-Calculus}
We shall consider \emph{simple types} (or just \emph{types} for
short) as defined by the grammar $A \, ::= \, o \; | \; A \funsp
A$. Each type $A$, other than the ground type $o$, can be uniquely
written as $(A_1, \cdots, A_n, o)$ for some $n \geq 1$, which is a
shorthand for $A_1 \funsp \cdots \funsp A_n \funsp o$ (by
convention, $\rightarrow$ associates to the right). We define the
\emph{level}\footnote{This is sometimes referred to as
\emph{order} in the literature.} of a type by $\level{o} = 0$ and
$\level{A \funsp B} = \max(\level{A}+1, \level{B}).$ In the
following we shall consider terms-in-context $\seq{\Gamma}{M : A}$
of the simply-typed $\lambda$-calculus. Let $\Delta$ be a simply-typed
alphabet i.e., each symbol in $\Delta$ has a simple type. We
write $\terms{A}{\Delta}$ for the set of terms of type $A$ built
up from the set $\Delta$ understood as constant symbols,
\emph{without} using $\lambda$-abstraction.

Following \cite{KNU02}, we say that $o$ is \textbfit{homogeneous},
and for $n \geq 1$, $A = (A_1, \cdots, A_n, o)$ is
\textbfit{homogeneous} just if $\level{A_1} \geq \level{A_2} \geq
\cdots \geq \level{A_n}$, and each $A_i$ is homogeneous. Assuming
$A = (\underbrace{A_{11}, \cdots, A_{1l_1}}_{\overline{A_1}},
\cdots, \underbrace{A_{r1}, \cdots, A_{rl_r}}_{\overline{A_r}},
o)$ is homogeneous, we write
\[A \; = \;
(\overline{A_1} \, | \, \cdots \, | \, \overline{A_r} \, | \, o)\]
to mean: all types in each partition (or sequence) $\overline{A_i}
= A_{i1}, \cdots, A_{il_i}$ have the same level, and $i < j \iff
\level{A_{ia}}> \level{A_{jb}}$. Thus the notation organises the
$A_{ij}$s into partitions according to their levels. Suppose $B =
(\overline{B_1} \, | \, \cdots \, | \, \overline{B_m} \, | \, o)$.
We write $(\overline{A_1} \, | \, \cdots \, | \, \overline{A_n} \,
| \, {B})$ to mean \[(\overline{A_1} \, | \, \cdots \, | \,
\overline{A_n} \, | \, \overline{B_1} \, | \, \cdots \, | \,
\overline{B_m} \, | \, o).\]

The \textbfit{Safe $\lambda$-Calculus} is a sub-system of the
simply-typed $\lambda$-calculus. Typing judgements (or
terms-in-context) are of the form
\begin{equation}
\nonumber \seq{\overline{x_1}:\overline{A_1} \, | \, \cdots \, | \,
\overline{x_n} :  \overline{A_n}}{M : B} 
\end{equation}
which is shorthand for
$\seq{x_{11} : A_{11}, \cdots, x_{1r}: A_{1r}, \cdots}{M : B}$. (We shall see shortly
that for valid such judgements, $(\overline{A_1} \, | \, \cdots \,
| \, \overline{A_n} \, | \, B)$ is a homogeneous type.)
\emph{Valid typing judgements} of the system are defined by
induction over the following rules, where $\Delta$ is a given
homogeneously-typed alphabet:
\[{{(\overline{A_1}
\, | \, \cdots \, | \, \overline{A_n} \, | \, B) \hbox{
homogeneous} \qquad {b : B} \; \in \; \Delta} \over
{\seq{\overline{x_1} : \overline{A_1}\, | \, \cdots\, | \,
\overline{x_n} : \overline{A_n}}{b : B}}}\]
\[{{(\overline{A_1}
\, | \, \cdots \, | \, \overline{A_n} \, | \, A_{ni}) \hbox{
homogeneous}} \over {\seq{\overline{x_1} : \overline{A_1}\, | \,
\cdots\, | \, \overline{x_n} : \overline{A_n}}{x_{ni} :
A_{ni}}}}\]
\[
{
{\seq{\overline{x_1} :
\overline{A_1}\, | \, \cdots\, | \, \overline{x_{n+1}} : \overline{A_{n+1}}}{M : B}}
\over
{\seq{\overline{x_1} :
\overline{A_1}\, | \, \cdots\, | \, \overline{x_{n}} : \overline{A_{n}}}{\lterm{\overline{x_{n+1}} : \overline{A_{n+1}}}{M} : (\overline{A_{n+1}} \, | \, B)}}
}
\]
\[{\seq{\Gamma}{M : (\overline{B_1} \, | \, \cdots \, | \, \overline{B_m} \, | \, o)} \qquad
\seq{\Gamma}{N_1 : B_{11}} \quad \cdots \quad \seq{\Gamma}{N_{l_1}
: B_{1l_1}} }\over{ \seq{\Gamma}{M N_1 \cdots N_{l_1} :
(\overline{B_2} \, | \, \cdots \, | \, \overline{B_m} \, | \, o)}}
\] In the safe $\lambda$-calculus, when constructing
$\lambda$-abstractions, \emph{all variables} of the relevant
type-partition must be abstracted; when constructing applications,
the operator-term must be applied to \emph{all operand-terms} (one
for each type) of the relevant type-partition. For example $\seq{F
: ((o, o), o, o, o), \varphi : (o, o), x : o}{F \varphi x : (o,
o)}$ is not safe; it follows that \[\seq{F : ((o, o), o, o, o),
\varphi : (o, o), x : o, y : o}{ F(F \varphi x)xy : o}\] is not
safe. But $\seq{F : ((o, o), o, o, o), \varphi : (o, o)}{F \varphi
a: (o, o)}$ is safe for some constant $a$, and so is $\seq{F :
((o, o), o, o, o), \varphi : (o, o), x : o, y : o}{F \varphi x y:
o}$.

In the following, whenever it is clear from the context what the
type of a term $M$ is, we shall write $\level{M}$ to mean the
level of that type.
\begin{lemma}\label{lem:safe}
Suppose $\seq{\overline{x_1} :
\overline{A_1}\, | \, \cdots\, | \, \overline{x_n} : \overline{A_n}}{M
: B}$ is valid, where $B = (\overline{B_1} \, | \,
\cdots \, | \, \overline{B_m} \, | \, o)$.
\begin{itemize}
\item[(i)] $(\overline{A_1} \, | \, \cdots \, | \, \overline{A_n}
\, | \, \overline{B_1} \, | \, \cdots \, | \, \overline{B_m} \, |
\, o)$ is homogeneous. \item[(ii)] Any free variable of $M$ has
level at least $\level{M}$. \item[(iii)] For any subterm
$\lterm{\Phi}{L}$ of $M$, if the variable $\varphi$ occurs in $L$
and $\level{\varphi} < \level{\Phi}$ then $\varphi$ is bound in
$L$. \myendproof
\end{itemize}
\end{lemma}

We omit the straightforward proofs.

\subsubsection*{What does ``safe'' mean?}

Substitution is a fundamental operation in the $\lambda$-calculus.
In the key clause of the definition
\[ (\lterm{x}{M})[N / y] \; \defined \; \lterm{z}{((M[z / x]) [N / y])} \quad
\hbox{where ``$z$ is fresh''}\] bound variables are renamed afresh to
prevent variable capture. In the {safe $\lambda$-calculus}, one can
get away without any renaming.

\begin{lemma}In the safe $\lambda$-calculus, there is no need to
rename bound variables afresh when performing substitution
\[M[N_1 / x_1 , \cdots, N_n / x_n]\]
provided the substitution is performed simultaneously on
\emph{all} free variables of the same level in $M$
i.e.~$\makeset{x_1, \cdots, x_n}$ is the set variables of the
same level as $x_1$ that occur free in $M$.
\end{lemma}

\begin{proof} Suppose $\varphi$ occurs free in $M$, and
bound variables in $M$ are not renamed in the substitution $M[N /
\varphi]$. Further suppose $x$, a variable occurring free in $N$,
is captured as a result of the substitution. I.e.~there is a
subterm $\lterm{x}{L}$ of $M$ such that $\varphi$ occurs free in
$L$. We compare $\level{x}$ with $\level{\varphi}$:

\begin{itemize}
\item {Case 1}: $\level{x} > \level{\varphi}$.

This is impossible: Since $\lterm{x}{L}$ is safe, by
Lemma~\ref{lem:safe}(iii), $L$ can have no free variables of level
less than $\level{x}$.

\item {Case 2}: $\level{x} < \level{\varphi}$.

This is impossible: Since $N$ is safe and of level
$\level{\varphi}$, by Lemma~\ref{lem:safe}(ii), it can have no
free variable of level less than $\level{\varphi}$.

\item {Case 3}: $\level{x} = \level{\varphi}$.

If follows from the formation rule for $\lambda$-abstraction that
$\varphi$ cannot occur free in $M$ since the subterm
$\lterm{x}{L}$ must be in the scope of some subterm
$\lterm{\varphi}{\cdots}$ of $M$, so that $\varphi$ does not occur free
in $M$. Thus this case cannot arise either.
\end{itemize}

\end{proof}

\subsection{Higher-order grammars and the OI-hierarchy}

A \textbfit{higher-order grammar} (or HOG, for short) is a
five-tuple $G = \anglebra{N, V, \Sigma, {\cal R}, S, e}$ such that
\begin{itemize}
\item[(i)] $N$ is a finite set of homogeneously-typed
non-terminals, and $S$, the \emph{start symbol}, is a
distinguished element of $N$ of type $o$

\item[(ii)] $V$ is a finite set of typed variables

\item[(iii)] $\Sigma$ is a finite alphabet

\item[(iv)] $\cal R$ is a finite set of triples, called
\emph{rewrite rules} (also referred to as production rules), of
the form
\[F x_1 \cdots x_n \; \larr{\alpha} \; E\]
where $\alpha \in (\Sigma \cup \makeset{\epsilon})$, ${F : (A_1,
\cdots, A_n, o)} \in N$, each ${x_i : A_i} \in V$, and $E$ is
either a term in ${\cal T}^o(N \cup \{x_1, \cdots, x_m\})$ or is
$e : o$. We say that $F$ has \textbfit{formal parameters} $x_1,
\cdots, x_m$. In the case where there grammar has two or more
rules with the non-terminal $F$ on the lefthand side, then we
assume both rules have the same formal parameters in the same
order. Following \cite{KNU01} we make the assumption that if $F
\in N$ has type $(A_1, \cdots, A_n, o)$ and $n \geq 1$, then $A_n
= o$. Thus, each non-terminal has at least one level-$0$ variable.
Note that this is not really a restriction -- as this variable need
not occur on the righthand side. We also assume that $S$ never occurs
on the righthand side of a rewrite rule. We say that the rewrite rule has
\emph{name} $F$ and \emph{label} $\alpha$; it has level $n$ just
in case the type of its name has level $n$.
\end{itemize}
We say that $G$ is of level $n$ just in case $n$ is the level of
the rewrite rule that has the highest-level. We say that $G$ is
\textbfit{deterministic} if for every $F x_1 \cdots x_n
\larr{\alpha} E$ and $F x_1 \cdots x_n \larr{\alpha'} E'$ in $\cal
R$
%(we identify rewrite rules that are
%$\alpha$-equivalent\footnote{$F x_1 \cdots x_n \larr{\alpha} U$
%and $F y_1 \cdots y_n \larr{\alpha} V$ in are said to be
%$\alpha$-\emph{equivalent} if $U$ and $V[\overline{x_i / y_i}]$
%are syntactically equal.})
\begin{itemize}
\item If $\alpha = \alpha'$ then $E = E'$.

\item If $\alpha = \epsilon$ and $E \not = e$, then $\alpha = \alpha'$
\end{itemize}
We say that a deterministic HOG is \emph{real-time} if no rule has
an $\epsilon$ label.


\subsubsection*{The language of a HOG}
We extend $\cal R$ to a family of binary relations $\larr{\alpha}$
over $\terms{o}{N} \cup \makeset{e}$, where $\alpha$ ranges over
$\Sigma \cup \makeset{\epsilon}$, by the rule: if $F x_1 \cdots
x_n \larr{\alpha} E$ is a rule in $\cal R$ where $x_i : A_i$ then
for each $M_i \in \terms{A_i}{N}$ we have
\[ FM_1 \cdots M_n \larr{\alpha} E\overline{[M_i / x_i]}. \]

A \emph{derivation} of $w \in \Sigma^\ast$ is a sequence $P_1,
P_2, \cdots, P_{n}$ of terms in $\terms{o}{N}$, and a
corresponding sequence $\alpha_1, \cdots, \alpha_n$ of elements in
$\Sigma \cup \makeset{\epsilon}$ such that
\[S = P_1 \larr{\alpha_1} P_2 \larr{\alpha_2} P_3 \larr{\alpha_3}
\quad \cdots \quad \larr{\alpha_{n-1}} P_{n} \larr{\alpha_n} e
\] and $w = \alpha_1 \cdots \alpha_n$. The \emph{language} generated by
$G$, written $L(G)$, is the set of words over $\Sigma$ that have
derivations in $G$. We say that two grammars are \emph{equivalent}
if they generate the same language.

\subsubsection*{Safe Grammars}

A grammar is \textbfit{safe} if for each rewrite rule
$F x_1 \cdots x_n \larr{\alpha} E$ we have that
\[ \seq{x_1 : A_1, \cdots, x_n : A_n}{E :
o}\]
is a valid typing judgement of the safe $\lambda$-calculus, where $E$ is constructed possibly
using symbols from $N$ as constants. Otherwise, the grammar is \textbfit{unsafe}.

\subsubsection*{The OI-hierarchy}

In \cite{Dam82}, Damm introduced the OI-hierarchy. The $n$th level
of the hierarchy is generated by level-$n$ grammars (defined
differently from our grammars). Furthermore,
each level is strictly contained in the one above it. The first
three levels correspond to the regular, the context-free, and the
indexed languages \cite{Aho68}. Damm's definition of a level-$n$
grammar, although packaged somewhat differently from ours, can be
thought of as a special case of our definition. In particular, it
is routine to show that a level-$n$ grammar using his definition
corresponds to a \emph{safe} level-$n$ grammar in our definition
(and the converse holds too). Thus, Damm's definition is such that
the safety restriction -- referred to as derived types in his
paper -- is always ``built-in''. For a note comparing the two
definitions (ours and Damm's) we point the reader to \cite{dMO}.
This note also motivates our preference for our definition.

The reader familiar with the OI-hierarchy should note that our
derivation relation is constructed so that all derivations are
outside-in (which, in the monadic case, corresponds to a leftmost
derivation). Bearing this in mind, the fact that our definition of
a safe level-$n$ grammar coincides with Damm's definition of a
level-$n$ grammar should be almost immediate.

\begin{remark}
We note there is nothing to be gained by allowing the rhs of rules
to be $\lambda$-terms, since any such rule can be transformed to
an equivalent system of rules, all of whose rhs are applicative
terms. E.g.~the level-$3$ rule
\[F \varphi \; \larr{\alpha} \; \varphi (\lterm{x}{ \varphi(\lterm{y}{x})})\]
is equivalent to the following system of rules
\[\begin{array}{rcl}
F \, \varphi & \larr{\alpha} & \varphi \, (G \varphi) \\
G \, \varphi \, x & \larr{\epsilon} & \varphi \, (H x) \\
H \, x \, y & \larr{\epsilon} & x \\
\end{array}\]
\end{remark}

%\begin{remark}\label{rem:otherdefinitions}
%We relate our definition to the various versions of higher-order
%grammar or recursion schemes in the literature.
%
%\medskip
%
%\noindent\emph{(i)} First we note there is nothing to be gained by
%allowing the rhs of rules to be $\lambda$-terms, since any such
%rule can be transformed to an equivalent system of rules, all of
%whose rhs are applicative terms. E.g.~the level-$3$ rule
%\[F \varphi \; \larr{a} \; \varphi (\lterm{x}{ \varphi(\lterm{y}{x})})\]
%is equivalent to the following system of rules
%\[\begin{array}{rcl}
%F \, \varphi & \larr{a} & \varphi \, (G \varphi) \\
%G \, \varphi \, x & \larr{\epsilon} & \varphi \, (H x) \\
%H \, x \, y & \larr{\epsilon} & x \\
%\end{array}\]
%
%\noindent\emph{(ii)} In \cite{KNU02} Knapik \emph{et al.} have
%introduced (deterministic) \emph{level-$n$ grammars} for
%generating $\Sigma$-trees (rather than languages), where $\Sigma$
%is a finite alphabet of typed constants of level at most one. The
%trees that are generated are just the syntax trees of potentially
%infinite applicative terms built up from $\Sigma$: all nodes of
%such a tree are labelled by constants from $\Sigma$, and if a node
%is labelled by a constant $f : o^n \funsp o$ then it has exactly
%$n$ descendants. In Section 7 we will show that we can
%systematically transform such a grammar $G$ (say) into one of ours
%$G'$ (say) in such a way that the language of $G'$ is
%essentially\footnote{but not quite.} the \emph{branch language}
%(in the sense of \cite{Cou83}) of the term-tree determined by the
%grammar $G$. We will also examine the relationship between
%string-language generating grammars (as introduced here) and
%term-tree generating grammars with some pleasing results.
%
%%In their approach, rules are not
%%labelled by symbols (as we do), instead the rhs of each rule is an
%%applicative term that may contain occurrences of $\Sigma$-symbols.  We
%%can systematically transform such a grammar $G$ (say) into one of ours
%%$G'$ (say): the constants that occur in the rhs of $G$-rules determine
%%the symbols that label the $G'$-rules in such a way that the language
%%generated by the transform $G'$ is essentially the \emph{branch
%%language} (in the sense of \cite{Cou83}) of the term-tree determined
%%by the grammar $G$. E.g.~the level-2 grammar in the sense of Knapik
%%\emph{et al.}
%%\[\begin{array}{rll}
%%F \, \varphi \, x & \larr{} & f(F (f(\varphi \, x)) x)\,(\varphi \, x)\\
%%S & \larr{} & F \, g \, a \\
%%\end{array}\]
%%with constants $f : (o, o, o), g : (o, o)$ and $a : o$ can be transformed to
%%the following system of labelled rules:
%%\[\begin{array}{rlllrll}
%%F \, \varphi \, x & \larr{f.1} & F (H (\varphi \, x)) x & \qquad & S & \larr{\epsilon} & F \, G \, A \\
%%F \, \varphi \, x & \larr{f.2} & \varphi \, x & & G \, x & \larr{g.1} & x \\
%%H \, y \, z & \larr{f.1} & y & & A & \larr{a.0} & \epsilon\\
%%H \, y \, z & \larr{f.2} & z\\
%%\end{array}\]
%
%\medskip
%
%\noindent\emph{(iii)} We should mention another, older, definition
%of level-$n$ grammar, which constitutes the \emph{OI-hierarchy} of
%Damm (see \cite{DG86, Dam82}). Damm's definition, although
%packaged differently from ours, can be thought of as a special
%case of ours. In particular, it is routine to show that a
%level-$n$ grammar using his definition corresponds to a
%\emph{safe} level-$n$ grammar in our definition (and the converse
%holds too). Thus, Damm's definition is such that the safety
%restriction (referred to as derived types) is always ``built-in''.
%For a note comparing the two definitions (ours and Damm's) we
%point the reader to \cite{dMO}. This article also motivates our
%preference for our definition.
%
%\medskip
%
%\noindent\emph{(iv)} The reader familiar with the OI-hierarchy
%\cite{Dam82} should note that our derivation relation is
%constructed so that all derivations are outside-in (which, in the
%monadic case, corresponds to a leftmost derivation).
%\end{remark}

\begin{example}\label{ex:ex1}
Consider the following \emph{deterministic} grammar, where
\[\Sigma = \makeset{h_1, h_2, h_3, f_1, f_2, g_1, a, b},\]
the typed non-terminals are
\[D : ((o, o), o, o, o), \quad H : ((o, o), o, o), \quad
F: (o, o, o), \quad G : (o, o), \quad A, B : o\] with rules:
\begin{equation}
\begin{array}{cc}
\begin{array}{rcl}
\nonumber S & \larr{\epsilon} & DGAB\\
\nonumber D\varphi x y & \larr{h_1} & D (\underline{D \varphi x}) y (\varphi y) \\
\nonumber D\varphi x y & \larr{h_2} & H (\underline{F y}) x \\
\nonumber D\varphi x y & \larr{h_3} & \varphi B \\
\nonumber H \varphi x & \larr{\epsilon} & \varphi x
\end{array} &
\begin{array}{rcl}
\nonumber G x & \larr{g_1} & x \\
\nonumber F x y & \larr{f_1} & x \\
\nonumber F x y & \larr{f_2} & y \\
\nonumber A & \larr{a} & e\\
\nonumber B & \larr{b} & e
\end{array}
\end{array}
\end{equation}
This is an unsafe grammar because of the underlined expressions:
both of which are level-$1$ terms, but contain level-$0$
variables.

As this grammar is deterministic \cite{dMO} this means that each
word in the language is generated in a unique way. Hence, it
should be easy for the reader to check by hand that the words
$h_1h_3h_2f_1b$ and $h_1h_3h_2f_2a$ are part of the language,
whereas $h_1h_3h_2f_1a$ is not.
\end{example}

\subsection{Pointer machines}
Pointer machines are a model of computation for string languages
generated by higher-order grammars. The \textbfit{pointer machine}
for an $n$-grammar $G = \anglebra{N, V, \Sigma, {\cal R}, S, e}$ has
a (pushdown) stack.  A stack $\beta$ is a non-empty sequence
$\mkstore{a_1, a_2, \cdots, a_n}$ where each $a_i \in \Gamma$, a
set determined by $G$. The following operations are defined on a
stack: for $a \in \Gamma$
\[\begin{array}{rll}
\head{\mkstore{a_1, a_2, \cdots, a_n}} & = & a_1 \\
\tail{\mkstore{a_1, a_2, \cdots, a_n}} & = & \mkstore{a_2, \cdots, a_n}\\
a : \mkstore{a_1, a_2, \cdots, a_n} & = & \mkstore{a, a_1, a_2, \cdots, a_n}.
\end{array}\]
The stack alphabet $\Gamma$ is a certain finite subset of
$\terms{o}{N \cup V} \cup \{e\}$, which will be defined shortly.
Each $a_i$, which we shall refer to as an \textbfit{item}, will
have exactly one pointer for each variable (from $V$) that occurs
in it. Thus an item in the stack may have several pointers
emanating from it (apart from $a_n$ which has no pointers).
Intuitively, a pointer is a physical link connecting an item $a_i$
to some other item $a_j$ for $j > i$.

%A \textbfit{pointer} is a physical link connecting an item $a_i$ to
%some item $a_j$ for $j > i$. So for example, we could have:
%\begin{equation}
%\nonumber [{\rnode{x}{a_1}}, {\rnode{y}{a_2}}, {\rnode{z}{a_3}},
%{\rnode{A}{a_4}}, {\rnode{B}{a_5}}] \ncarc[arcangle=90]{->}{x}{A}
%\end{equation}
%In this example there is a pointer from $a_1$ to $a_4$. As we will
%see, the alphabet $\Gamma$ will be a finite subset of .

We define the \textbfit{stack transition relation}. The behaviour
of a pointer machine is given in terms of a labelled transition
relation between stacks. A computation of a pointer machine begins
with the stack containing a single item, which is the start symbol
$S$. Thereafter, let $\beta$ be the stack. We use meta-variables
$F$ and $x$ to range over $N$ (non-terminals) and $V$ (variables)
respectively.
\begin{itemize}
\item[1.] Assume $\head{\beta} = F t_1 \cdots t_n$. If $F x_1
\cdots  x_n \; \larr{\alpha} \; E$ is a $G$-rule then
\[\beta \; \larr{\alpha} \; E: \beta \]
and each occurrence of a variable $x_j$ in $E$ points to
$\head{\beta}$, in other words $F t_1 \cdots t_n$. Further, the
pointer structure of $\beta$ is preserved.

\item[2.] Assume $\head\beta = x  t_1  \cdots  t_n$ and
$x$ points to an item $a = D s_1 \cdots s_m$
in $\beta$. Furthermore, suppose that $x$ is the $i$th formal
parameter of $D$. Then
\[ \beta \; \larr{\epsilon} \; s_i t_1  \cdots  t_n : \tail\beta\]
all pointers in $\tail\beta$ are preserved and all pointers from $t_1,
\cdots, t_n$, as well as those from $s_i$, are preserved.

%\item[3.] Assume $\head\beta = f t_1 \cdots t_n$ where
%$\arity{f} > 0$. Then \[ \beta \; \stackrel{f_i}{\rightarrow} \; t_i
%: \tail\beta\] and the pointer structure of $\beta$ is
%preserved, as are any pointers emanating from $t_i$.

\item[3.] Assume $\head\beta = e$, then the pointer machine halts.
\end{itemize}

Let $\stackl S \stackr \larr{\alpha_1} P_1 \larr{\alpha_2} \cdots
\larr{\alpha_{n-1}} P_n$ be a transition sequence of pointer machine
stacks such that $\head P_n = e$. Then, we say that the word,
$\alpha_1 \alpha_2 \cdots \alpha _{n-1}$ (over $\Sigma$), is \textbfit{accepted} by
the pointer machine. We say that a word belongs to the language of
a given pointer machine if and only if there exists a transition
sequence of the pointer machine starting from $\stackl S \stackr$
that results in the word being accepted.

The language accepted by a pointer machine can easily be seen to
be exactly the language of the higher-order grammar. In
particular, it corresponds to linear head reduction in the
$\lambda$-calculus.

\begin{remark}
\begin{enumerate}
\item[(i)] Pointer machines are due to C. Stirling. The second author
first learnt of the idea in \cite{Sti02}.
\item[(ii)] The reader acquainted with \cite{KNU01, KNU02} should
be able to easily modify the definition of a pointer machine to
serve as a model of computation for term-trees generated by
higher-order grammars.
\end{enumerate}
\end{remark}

\subsubsection*{Stack alphabet}
We write $R$ for the set consisting of the rhs of each rule from $\cal
R$.  Let $\xi E_1 \cdots E_m$ be a ground-type term, where $\xi$ is
either a variable or a non-terminal. We define
\[
     \args{\xi E_1 \cdots E_m} \; = \; \makeset{E_1, \cdots, E_m}.
\]
Now define two sets, $\Gamma$ (which is a set of ground-type terms) and
$\exp$, by mutual induction over the following rules:
\begin{center}
\begin{tabular}{p{12cm}}
\prule{}{R \subseteq \Gamma}\\ \prulec{}{\args{r} \subseteq \exp}{$r
\in R$}\\ \prule{E^A \in \exp \quad x^A E_1 \cdots E_m \in \Gamma}{E
\, E_1 \cdots E_m \in \Gamma} \\ \prule{\xi E_1 \cdots E_m \in
\Gamma}{ \args{\xi E_1 \cdots E_m} \subseteq \exp}
\end{tabular}
\end{center}

We can take $\Gamma$ to be the stack alphabet. ($\Gamma$ is a superset
of what we actually need).

\begin{lemma}
$\Gamma$ is finite.
\end{lemma}

\begin{proof}
We define a new set $\Gamma'$ (of ground-type terms) by induction over
the rules: $R \subseteq \Gamma'$ and
\[
{E^A \in \exp' \quad x^A E_1 \cdots E_m \in \Gamma'}\over{E^A E_1
\cdots E_m \in \Gamma'}
\]
where $\exp' = \makeset{ N : \hbox{$N$ is a subterm of some $r \in
R$}}$, which is clearly finite. It is straightforward to see that $\exp
\subseteq \exp'$ and $\Gamma \subseteq \Gamma'$. Therefore it suffices
to prove that $\Gamma'$ is finite. We define a partial order over the
set $\makeset{A : E^A \in \exp' }$ of types by:
\[
    A > B  \; \iff  \; A = A_1 \mor \cdots \mor A_n \mor B
\]
where $n > 0$. Now define a function $H : {\terms{o}{N \cup V}} \mor
\terms{o}{N \cup V}$ as follows:
\[
   H(G) \; = \; R \cup G \cup
      \makeset{ \xi^A U_1\cdots U_n E_1\cdots E_m :
              x^B E_1 \cdots E_m \in G,
              \xi^A U_1\cdots U_n : B \in \exp' }
\]
We observe that $\Gamma'$ is the fixpoint of $H$, which is reached
after at most $j$ iterations of $H$, where $j$ is the length of the
longest chain in the poset $\makeset{ A : E^A \in \exp' }$.
\end{proof}

\subsubsection*{Useful properties}
We consider the operation of a pointer machine. The stack items of
the pointer machine are of two types: \emph{incomplete} and
\emph{complete}. A complete item is one that is headed by a
non-terminal, so for example $F\overline{s}$ is a complete item.
An incomplete item is headed by a variable. We call it incomplete,
because it is, in a sense, work in progress: we will remain at
this stack item (replacing the head variable) until it is finally
headed by a non-terminal, in which case it will become complete.

\begin{lemma}\label{lem:complete}
All items of any reachable pointer machine stack $\beta$ are of
level 0; further, $\tail\beta$ consists only of complete items.
\end{lemma}

\begin{proof} Induction on the number of stack transitions.
\end{proof}

\begin{lemma}\label{lem:pmgroups}
Let $s_0s_1 \cdots s_k$ be an item in a reachable pointer machine
stack $\beta$. Then the following hold:
\begin{enumerate}
\item[(i)] If $x$, a variable occurring in $s_0s_1 \cdots s_k$, points to a stack item $D t_1 \cdots t_m$ then $x$ is a formal
parameter of $D$.

\item[(ii)] For each $i$, the variables in $s_i$ all point to the same stack item.

\item[(iii)] Suppose $s_i$ has a variable with pointer to an item
$a$, and $s_j$ has a variable with pointer to an item $a'$. If $i
< j$ then either $a$ and $a'$ are the same item or $a$ occurs
deeper in the stack than $a'$.
\end{enumerate}
\end{lemma}

\begin{proof}Induction on the number of stack transitions.\\
\end{proof}

Bearing these two observations in mind, we modify the definition of a
pointer machine slightly to obtain a \textbfit{compacting pointer
machine} for a HOG $G = \anglebra{N, V, \Sigma, {\cal R}, S, e}$. As
before, it has a pushdown stack and a computation of a pointer machine
begins with the stack containing only the start symbol
$S$. Afterwards, the stack transition relation is defined as follows
below. Let $\beta$ be the current stack:
\begin{itemize}
\item[1.] Assume $\head\beta = F t_1 \cdots t_n$. If $F x_1
\cdots  x_n \; \larr{\alpha} \; E$ is a $G$-rule then
\[\beta \; \larr{\alpha} \; E: \beta \]
and each occurrence of a variable $x_j$ in $E$ points to
$\head{\beta}$, in other words $F t_1 \cdots t_n$. Further the pointer
structure of $\beta$ is preserved.

\item[2.]
\begin{itemize}
\item[a.] Assume $\head\beta = x  t_1  \cdots  t_n$ and suppose
that $x$ is a variable of level at least $1$, and points to an
item $a = D s_1 \cdots s_m$ in $\beta$. Furthermore, suppose that
$x$ is the $i$th formal parameter of $D$. Then
\[ \beta \; \larr{\epsilon} \; s_i t_1  \cdots  t_n : \tail\beta\]
all pointers in $\tail\beta$ are preserved and all pointers from
$t_1, \cdots, t_n$, as well as those from $s_i$, are preserved.

\item[b.] Assume $\head \beta = x$, $x$ is of level 0, and points
to an item $\head \beta' = D s_1 \cdots s_m$ where $\beta'$ is a
suffix of $\beta$. Furthermore, suppose that $x$ is the $i$th
formal parameter for $D$. By Lemma ~\ref{lem:pmgroups} we know
that all the variables in $s_i$ (if any) point to the \emph{same}
item in $\beta'$, (say) $\head \beta''$ where $\beta''$ is a
suffix of $\beta'$. Then
\[ \beta \; \larr{\epsilon} \; s_i : {\beta''}\]
all pointers in ${\beta''}$ are preserved as well as those from
$s_i$\footnote{In particular, all pointers from variables in $s_i$ will
be to $\head \beta''$.}. In the case where $s_i$ contains no
variables $\beta''$ may be chosen arbitrarily.
\end{itemize}
%
%\item[3.] Assume $\head\beta = f t_1 \cdots t_n$ where
%$\arity{f} > 0$. By Lemma ~\ref{lem:pmgroups} we know that all the
%variables in $t_i$ point to subterms of the \emph{same} item in
%$\beta$, say this item is $\head{\beta'}$ where $\beta'$ is a suffix
%of $\beta$. Then
%\[ \beta \; \stackrel{f_i}{\rightarrow} \; t_i : {\beta'}\] and the pointer structure of $\beta'$ is
%preserved, as are any pointers emanating from $t_i$.

\item[3.] Assume $\head\beta = e$, then the pointer machine halts.
\end{itemize}

The key to understanding the compacting pointer machine is that it
maintains the following invariant. If $\beta$ is the pointer
stack, with $\head \beta = s_0 s_1 \cdots s_n$, then either 1)
$s_n$ contains at least one variable with a pointer to $\tail
\beta$ or 2) $s_n$ contains only non-terminals. As an example,
suppose that
\begin{equation}
\nonumber \stackl {\rnode{phi}{\varphi}} {\rnode{x1}{x_1}}
{\rnode{x2}{x_2}}, {\rnode{A}{A}}, {}\rnode{B}{B}, {\rnode{C}{C}},
{\rnode{D}{D}}, {\rnode{E}{E}}\stackr \ncarc[arcangle=90]{->}{x2}{C}
\end{equation}
is a pointer stack of the original pointer machine, where
$\varphi$ is a variable of level $1$ and $x_1,x_2$ are variables
of level $0$. Here, $A, B, C, \cdots$  are meta-variables for terms in
${\cal T}^o(N \cup V)$. Only one pointer has been shown: from variable $x_2$
in the topmost item to $C$. Other pointers exist, but have not be
indicated. By Lemma~\ref{lem:pmgroups}(iii) we know that $x_2$ and
$\varphi$ must point to item $C$ or items deeper in the stack,
namely, $D$ or $E$. In this instance the items $A$ and $B$ have
now become redundant: for any stack configuration reachable from
here, the topmost item will never contain a pointer to $A$ or $B$.
Thus, for all intents and purposes, they may as well be cut out,
and removing such redundancies is what the compacting pointer
machine strives to achieve. So the above configuration would
become:
\begin{equation}
\nonumber \stackl {\rnode{phi}{\varphi}} {\rnode{x1}{x_1}}
{\rnode{x2}{x_2}}, {\rnode{C}{C}}, {\rnode{D}{D}}, {\rnode{E}{E}}\stackr
\ncarc[arcangle=90]{->}{x2}{C}
\end{equation}
and all other pointers for remaining items are preserved. Hence
the name compacting\footnote{Note that although the compacting
pointer machine does remove such redundant segments, it is not
optimal; in the sense that some such redundant segments may still
exist. This, however, is deliberate.}.

\begin{lemma} A compacting pointer machine is equivalent to a
pointer machine. \myendproof
\end{lemma}


\subsubsection*{Representation}

This section concerns notation for pointer machine stacks. Recall that
each item in the stack may have several pointers emanating from it. In
particular, suppose we have the following PM stack:
\begin{equation}
\nonumber \stackl
{\rnode{x}{\varphi}}{\rnode{y}{y}}{\rnode{z}{z}},\;
{\rnode{A}{F}}G{\rnode{w}{x}}AB,\; {\rnode{B}{D}}A, \; S \stackr
\ncarc[arcangle=90]{->}{x}{B} \ncarc[arcangle=90]{->}{y}{A}
\ncarc[arcangle=80]{->}{z}{A} \ncarc[arcangle=80]{->}{w}{B}
\end{equation}
Note that $\varphi$ points to $DA$, both $y$ and $z$ point to $FGxAB$,
and $x$ points to $DA$. Rather than resorting to having to draw
arrows we can convey this information by subscripting each
variable with a PM stack. We annotate each variable
with a suffix $s$ of the PM stack, such that the pointer points to
$\head s$. Thus, for the above example we would have:
\begin{equation}
\nonumber \stackl x_{p_{\varphi}}y_{p_y}z_{p_z}, FGx_{p_x}AB, DA, S\stackr \mbox{where}
\end{equation}
\begin{eqnarray}
\nonumber p_{\varphi} & = & \stackl DA, S \stackr \\
\nonumber p_y & = & \stackl FGx_{p_x}AB, DA, S \stackr \\
\nonumber p_z & = & \stackl FGx_{p_x}AB, DA, S \stackr \\
\nonumber p_x & = & \stackl DA, S \stackr
\end{eqnarray}
Recall from Lemma~\ref{lem:pmgroups} that if $s_0s_1 \cdots s_n$
is an item of a PM stack, and if $x, x'$ are variables occurring
in $s_i$ for some $i$, then the pointer of $x$ is the same as the
pointer of $x'$. Hence, to avoid having to subscript \emph{every}
single variable with a PM stack, we adopt following convention:
For $p,q$ pointer stacks and $s, t \in \mathcal{T}(N \cup V) \cup
\{e\}$, we have:
\begin{itemize}
\item $(st)_p = (s_p)(t_p)$ for $(st)$ an application \item
$(s_p)_q = s_p$ for all $s$ \item $F_p = F$ for $F$ a non-terminal
\item $e_p = e$
\end{itemize}
Thus, adopting the above convention, our earlier example could be
rewritten as:
\begin{equation}
\nonumber \stackl (\varphi_{p_{\varphi}}yz)_{p_z}, (FGxAB)_{p_x}, DA, S
\stackr \mbox{ where } p_{\varphi},p_z,p_x \mbox{ are as before}
\end{equation}
Note the $yz$.

%Let us consider a more concrete example, using the grammar for
%Urzyczyn's language. It is easy to see that below is reachable
%pointer machine stack.
%\begin{equation}
%\nonumber \stackl (D\varphi_p x_p) y_q x_q,D(D\varphi_p
%x_p)z_p(Fy_p)(Fy_p),DG\epsilon\epsilon\epsilon, S \stackr
%\end{equation}
%where
%\begin{eqnarray}
%\nonumber q & = & \stackl D(D\varphi_p x_p)z_p(Fy_p)(Fy_p),DG\epsilon\epsilon\epsilon, S \stackr \\
%\nonumber p & = & \stackl DG\epsilon\epsilon\epsilon, S \stackr
%\end{eqnarray}
%However, with using our convention, we can express this more
%succinctly:
%\begin{equation}
%\nonumber \stackl ((D\varphi x)_p y x)_q,(D(D\varphi
%x)z(Fy)(Fy))_p,DG\epsilon\epsilon\epsilon, S \stackr
%\end{equation}
%with $q$ and $p$ as before.

\subsubsection*{An example}

\begin{example} We give an example computation of the (non-compacting) pointer
machine for the grammar in \ref{ex:ex1}.
\begin{eqnarray}
\nonumber & & \stackl S \stackr  \\
\nonumber & \larr{\epsilon} & \stackl DGAB, S \stackr  \\
\nonumber & \larr{h_1} & \stackl (D(D\varphi x) y (\varphi y))_{p_1}, DGAB, S \stackr \\
\nonumber & \larr{h_3} & \stackl (\varphi x)_{p_2},(D(D\varphi x) y (\varphi y))_{p_1}, DGAB, S \stackr \\
\nonumber & \larr{\epsilon} & \stackl (D \varphi x)_{p_1} B,(D(D\varphi x) y (\varphi y))_{p_1}, DGAB, S \stackr \\
\nonumber & \larr{h_2} & \stackl (H(Fy)x)_{p_3},(D \varphi x)_{p_1} B,(D(D\varphi x) y (\varphi y))_{p_1}, DGAB, S \stackr \\
\nonumber & \larr{\epsilon} & \stackl (\varphi x)_{p_4},(H(Fy)x)_{p_3},(D \varphi x)_{p_1} B,(D(D\varphi x) y (\varphi y))_{p_1}, DGAB, S \stackr \\
\nonumber & \larr{\epsilon} & \stackl (Fy)_{p_3} x_{p_4},(H(Fy)x)_{p_3},(D \varphi x)_{p_1} B,(D(D\varphi x) y (\varphi y))_{p_1}, DGAB, S \stackr \\
\nonumber & \larr{f_1} & \stackl x_{p_5},(Fy)_{p_3} x_{p_4},(H(Fy)x)_{p_3},(D \varphi x)_{p_1} B,(D(D\varphi x) y (\varphi y))_{p_1}, DGAB, S \stackr \\
\nonumber & \larr{\epsilon} & \stackl y_{p_3},(Fy)_{p_3} x_{p_4},(H(Fy)x)_{p_3},(D \varphi x)_{p_1} B,(D(D\varphi x) y (\varphi y))_{p_1}, DGAB, S \stackr \\
\nonumber & \larr{\epsilon} & \stackl B,(Fy)_{p_3} x_{p_4},(H(Fy)x)_{p_3},(D \varphi x)_{p_1} B,(D(D\varphi x) y (\varphi y))_{p_1}, DGAB, S \stackr \\
\nonumber & \larr{b} & \stackl e,B,(Fy)_{p_3} x_{p_4},(H(Fy)x)_{p_3},(D \varphi x)_{p_1} B,(D(D\varphi x) y (\varphi y))_{p_1}, DGAB, S \stackr
\end{eqnarray}
where
\begin{eqnarray}
\nonumber p_1 & = & \stackl DGAB, S \stackr \\
\nonumber p_2 & = & \stackl D(D\varphi x) y (\varphi y))_{p_1}, DGAB, S \stackr \\
\nonumber p_3 & = & \stackl (D \varphi x)_{p_1} B,(D(D\varphi x) y (\varphi y))_{p_1}, DGAB, S \stackr \\
\nonumber p_4 & = & \stackl H(Fy)x)_{p_3},(D \varphi x)_{p_1} B,(D(D\varphi x) y (\varphi y))_{p_1}, DGAB, S \stackr \\
\nonumber p_5 & = & \stackl (Fy)_{p_3} x_{p_4},(H(Fy)x)_{p_3},(D \varphi x)_{p_1} B,(D(D\varphi x) y (\varphi y))_{p_1}, DGAB, S \stackr
\end{eqnarray}
Note that the pointer machine accepts the word $h_1h_3h_2f_1b$. We
now demonstrate how the compacting version of the pointer machine
is more economical. In particular, the computation proceeds as
above, up to the point:
\begin{eqnarray}
\nonumber & \larr{f_1} & \stackl x_{p_5},(Fy)_{p_3}
x_{p_4},(H(Fy)x)_{p_3},(D \varphi x)_{p_1} B,(D(D\varphi x) y
(\varphi y))_{p_1}, DGAB, S \stackr
\end{eqnarray}
we then proceed as follows:
\begin{eqnarray}
\nonumber & \larr{\epsilon} & \stackl y_{p_3},(D \varphi x)_{p_1} B,(D(D\varphi x) y (\varphi y))_{p_1}, DGAB, S \stackr \\
\nonumber & \larr{\epsilon} & \stackl B,(D(D\varphi x) y (\varphi y))_{p_1}, DGAB, S \stackr \\
\nonumber & \larr{b} & \stackl e, B, (D(D\varphi x) y (\varphi
y))_{p_1}, DGAB, S \stackr
\end{eqnarray}
where $p_1, \cdots, p_5$ are as before.
\end{example}

\subsection{Higher-order pushdown automata}
%\subsection{Higher-order pushdown automata and the Maslov Hierarchy}

Before we can define a higher-order pushdown automaton, we need to
define a \textbfit{level-$n$ store} or simply \emph{$n$-store}.
Fix a finite set $\Gamma$ of \emph{store symbols}, including a
distinguished bottom-of-store symbol $\bot$. A \emph{1-store} is a
finite non-empty sequence $\mkstore{a_1, \cdots, a_m}$ of
$\Gamma$-symbols such that $a_i = \bot$ iff $i = m$. For $n \geq
1$, an \emph{$(n+1)$-store} is a non-empty sequence of $n$-stores.
Inductively we define the \emph{empty $(n+1)$-store} $\bot_{n+1}$
to be $\mkstore{\bot_n}$ where we set $\bot_0 = \bot$. Recall the
following standard operations on $1$-stores: for $a \in ({\Gamma}
\setminus \makeset{\bot})$
\[\begin{array}{rll}
\nonumber \push_1(a) \, \mkstore{a_1,\cdots,a_m} & = & \mkstore{a,a_1,\cdots,a_m}\\
\nonumber \pop_1 \, \mkstore{a_1, a_2, \cdots, a_m} & = & \mkstore{a_2,\cdots,a_m}\\
%\nonumber top_1 ([a_1, a_2, \cdots, a_m]) & = & a_1
\end{array}\]
\renewcommand\arraystretch{1.3}
For $n \geq 2$, the following set $Op_n$ of \emph{level-$n$
operations} are defined over $n$-stores:
\[\left\{\begin{array}{rll}
\push_n \, \mkstore{s_1, \cdots, s_l} & = & \mkstore{s_1, s_1, \cdots, s_l}\\
\push_k \, \mkstore{s_1, \cdots, s_l} & = & \mkstore{\push_k \, s_1,
s_2, \cdots, s_l}, \quad 2 \leq k < n\\
\push_1 (a) \, \mkstore{s_1, \cdots, s_l} & = & \mkstore{\push_1 (a) \, s_1, s_2, \cdots, s_l}\\
\pop_n \, \mkstore{s_1, \cdots, s_l} & = & \mkstore{s_2, \cdots, s_l}\\
\pop_k \, \mkstore{s_1, \cdots, s_l} & = & \mkstore{\pop_k \, s_1,
s_2, \cdots, s_l}, \quad 1 \leq k < n\\
\id \, \mkstore{s_1, \cdots, s_l} & = & \mkstore{s_1, \cdots, s_l}\\
\end{array}\right.\]
In addition we define
\[\begin{array}{rll}
\mytop_n \, \mkstore{s_1, \cdots, s_l} & = & s_1\\
\mytop_k \, \mkstore{s_1, \cdots, s_l} & = & \mytop_k \, s_1, \quad 1
\leq k < n\\
\end{array}\]
Note that $\pop_{k} \, s$ is undefined if $\mytop_k \, s =
\bot_{k-1}$, for $k \geq 1$.

\medskip

A \textbfit{level-$n$ pushdown automaton} (or $n$PDA for short) is a
6-tuple $\anglebra{Q, \Sigma, \Gamma, \delta, q_0, F}$ where:
\begin{enumerate}
\item[(i)] $Q$ is a finite set of states, $q_0 \in Q$ is the start
state, and $F \subseteq Q$ is a set of accepting states
\item[(ii)] $\Sigma$ the finite input alphabet \item[(iii)]
$\Gamma$ the finite store alphabet (which is assumed to contain
$\bot$) \item[(iv)] $\delta \; \subseteq \; Q \times (\Sigma \cup
\makeset{\epsilon}) \times \Gamma \times Q \times Op_n$ the
transition relation.
\end{enumerate}
A \textbfit{configuration}\footnote{This is sometimes called
\emph{total configuration} in the literature.} of an $n$PDA is given
by a triple $(q, w, s)$ where $q$ is the current state, $w$ is the
remaining input, and $s$ is the is an $n$-store over $\Gamma$.

Given a configuration $(q, aw, s)$ (where $a \in \Sigma$ or $a =
\epsilon$, and $w \in \Sigma^*$) where $\mytop_1(s) = Z$, we
define the following relation ${\rightarrow}$:
\begin{equation}
\nonumber (q, aw, s) \rightarrow (p,w,s')
\end{equation}
if $(q,a,Z,p,\theta) \in \delta$ and $s' = \theta(s)$.
Intuitively, this says that if we are in state $q$ reading input
symbol $a$ and the topmost store symbol is $Z$, then we may change
the state to $p$, consume $a$, and perform the operation $\theta$
to the current $n$-store. The transitive closure of $\rightarrow$
is denoted by $\rightarrow^+$, whereas the reflexive and
transitive closure is denoted by $\rightarrow^*$. We say that the
input $w$ is \textbfit{accepted} by the above $n$PDA if $(q_0, w,
\bot_n) \rightarrow^* (q_f, \epsilon, s)$ for some pushdown store
$s$ and some $q_f \in F$.

\begin{remark}\rm We define $0$PDAs to be finite automata. Note that $1$PDAs
are just the standard pushdown automata. For an example of a $2$PDA we
refer the reader to the following section.
\end{remark}

The above defined $n$PDA can be thought of as being
non-deterministic. However, we say that an $n$PDA is
\textbfit{deterministic} if the following hold:
\begin{enumerate}
\item[(i)] If $\delta(q,a,Z) \not = \emptyset$ for some $a \in \Sigma$ then
$\delta(q,\epsilon, Z) = \emptyset$
\item[(ii)] $|\delta(q,a,Z)| \leq 1$
for all $a \in \Sigma \cup \{\epsilon\}$, $q \in Q$ and $Z \in
\Gamma$.
\end{enumerate}

\begin{remark}
In \cite{dMO} it is shown that a language $L$ is generated by a
level-$n$ safe deterministic grammar if and only if it is accepted
by a level-$n$ deterministic pushdown automaton. This is an easy
result to show -- but has not appeared previously in the literature.
\end{remark}


% Section: Relating hogs and pdas
\section{Relating $n$PDAs and $n$-grammars}

\subsection{Known results}

Historically, the notion of safety has been key to relating
$n$PDAs and level-$n$ grammars. In particular, the following
results are due to Damm and Goerdt\footnote{In their paper, safety
is referred to as the restriction of derived types.}:

\begin{theorem} (Damm and Goerdt \cite{DG86}) If $G$ is a safe grammar of
level $n$, then $L(G)$ is accepted by an $n$PDA.
\end{theorem}

\begin{theorem} (Damm and Goerdt \cite{DG86}) If $L$ is the language
of an $n$PDA then it is generated by a level-$n$ safe grammar.
\end{theorem}

%The same results were subsequently proved by Knapik \emph{et al.}
%in \cite{KNU02} for their term-tree setting: a term-tree is
%accepted by an $n$PDA if and only if it is generated by a
%level-$n$ safe grammar.

However, to date, no results exist for \emph{unsafe} level-$n$
grammars. In particular, if $G$ is an unsafe level-$n$ grammar, it
is not known whether $L(G)$ is accepted by an $n$PDA, or perhaps a
PDA of another level. Thus, we believe that our result, which we
present in the following section, is a first step towards solving
this problem. We will show that at level $2$, every unsafe grammar
can be converted into a safe one that generates the same language.\\

%We should mention that our interest in questions conc was
%initially motivated by attempting to solve the safety problem in
%Knapik \emph{et al.}'s term-tree setting.

% Knapik \emph{et al.} \cite{KNU02} prove the same result in their
%term-tree setting: a term term tree is generated by a safe grammar
%of level $n$ if and only if it is accepted by an $n$PDA (for
%trees). Knapik \emph{et al.} go on to ask whether the same result
%applies to unsafe grammars. Although their question concerns their
%setting, we believe it is important to address this question for
%the string-language setting as well - as this is not something yet
%me
%
%This is the motivation for our result: although our result
%concerns the original setting for string languages (as opposed to
%term trees), we believe that it is a first step towards solving
%this question for both settings.

Before we give our result, however, we present a case study. Below
we introduce a deterministic but unsafe level-$2$ grammar that
generates a language $U$. We will show, via a ``bespoke'' proof,
that $U$ can be accepted by $2$PDA. This effort may seem redundant,
given that $U$ is exactly the type of language that is amenable to
our result. However, we believe there is much to be gained from
our case study:
\begin{itemize}
\item It is a non-trivial example of an unsafe grammar at the
lowest possible level (where safety becomes an issue), hence it is
interesting in its own right. \item It provides us with a good
idea as to the capabilities of a $2$PDA. \item Finally, and
perhaps most importantly, it lays the foundation for a conjecture
we will make in Section 7.
\end{itemize}

%It is interesting to note that both theorems were subsequently
%proved again by Knapik et al. in \cite{KNU02}. Their proofs were
%constructed to work in the term tree setting
%(\ref{rem:otherdefinitions}).
%
%
%However, a slight modification of their proofs also enables them
%to work in our string-language setting. In fact, virtually no
%modification of either proof in \cite{KNU02} is required - apart
%from taking into account the input alphabet, which is trivial. In
%fact, for our purposes it will be more convenient to use Knapik et
%al.'s proof as opposed to the one presented in Damm and Goerdt. In
%an effort to make this document relatively self-contained, we
%contain the modified proofs in the appendix.

% subsection: Urzyczyn's language

\subsection{An example: Urzyczyn's language} The
language $U$ consists of words of the form $w \, \ast^n$ where $w$
is a proper prefix of a well-bracketed word such that no prefix of
$w$ is a well-bracketed expression; each parenthesis in $w$ is
implicitly labelled with a number, and $n$ is the label of the
last parenthesis. The two labelling rules are:
\begin{itemize}
\item[I.] The label of the opening \hbox{\pq} is one; the label of any
subsequent \hbox{\pq} is that of the preceding \hbox{\pq} plus one.

\item[II.] The label of \hbox{\oa} is the label of the parenthesis that
precedes the matching \hbox{\pq}.
\end{itemize}
For example, the following are words in $U$
(together with their respective sequences of labels):
\begin{itemize}
\item[(i)]
\[\begin{array}{cccccccccccccccccccc}
\pq & \pq & \pq & \pq & \oa & \oa & \pq  & \pq & \oa & \pq & \pq & \oa & \oa & \oa & \pq & \pq  & \oa & \oa & \ast & \ast \\
1 & 2 & 3 & 4 & 3 & 2 & 5  & 6 & 5 & 7 & 8 & 7 & 5 & 2 & 9 & 10 & 9 & 2 &   &
\end{array}\]
\item[(ii)]
\[\begin{array}{cccccccccccccccccc}
\pq  & \pq & \pq & \pq & \oa & \oa & \pq & \pq & \oa & \pq & \pq & \oa & \oa & \ast &\ast &\ast &\ast &\ast\\
1 &  2 & 3 & 4 & 3 & 2 & 5 & 6 & 5 & 7 & 8 & 7 & 5 & \\
\end{array}\]
\end{itemize}
We first learnt of the language from \cite{Urz03}, wherein it was
conjectured that $U$ is inherently unsafe i.e.~not acceptable by
any higher-order PDA. This is in fact not the case. We shall first
give an unsafe 2-grammar that generates the language, and then
show that it is accepted by a safe (non-deterministic) $2$PDA.

\medskip

A way to compute the labels is to maintain a \emph{configuration},
which is either a triple $\anglebra{\gamma, y, z}$ such that
\begin{itemize}
\item $\gamma$ is a stack of future $\oa$-{labels} (written as a list
$x : \phi$ whose head $x$ is the top of the stack)

\item $y$ is the number of \hbox{\pq} read thus far

\item $z$ is the label of the last parenthesis read,
\end{itemize}
or a number, which is the number of remaining $\ast$ to be read. Note
that the length of $\gamma$ is equal to the number of as yet unmatched
$\pq$'s at that point. The transitions are as follows:
\[\begin{array}{rll}
\anglebra{x:\phi, y, z} & \larr{\pq} & \anglebra{z: x : \phi, y+1, y+1}\\
\anglebra{x:\phi, y, z} & \larr{\oa} & \anglebra{\phi, y, x}\\
\anglebra{x:\phi, y, z} & \larr{\ast} & z\\
z+1 & \larr{\ast} & z\\
\end{array}\]
By mimicking the transition system, we can define a
\emph{deterministic} (and unsafe) 2-grammar that generates $U$. We set
\[\begin{array}{rll}
\Sigma & = & \makeset{\pq , \oa , \ast }\\
N & = & \makeset{S: o, D : ((o, o, o), o, o, o, o), G : (o, o, o),
F : (o, o), E : o}
\end{array}\]
with production rules as follows:
\[\left\{\begin{array}{rll}
S & \larr{\pq} & D \, G \, E \, E \, E \\
D \, \phi \, x \, y \, z & \larr{\pq} & D \, (D \, \phi \, x) \, z \, (F \, y) \, (F \,  y)\\
D \, \phi \, x \, y \, z & \larr{\oa} & \phi \, y \, x \\
D \, \phi \, x \, y \, z & \larr{\ast} & z \\
F \, x & \larr{\ast} & x \\
E \, & \larr{\epsilon} & e
\end{array}\right.\]
Note that we have simply encoded the configuration $\anglebra{x :
\phi, y, z}$ as the term $D \, \phi \, x \, y \, z$, and
$\ast^{n+1}$ as $\underbrace{F( \cdots (F}_{n} E)$.

\subsubsection*{Unique decomposition of $U$-words}
Consider words over the alphabet $\makeset{\pq, \oa, \ast}$ composed
of three parts as follows
\[
\underbrace{\pq \cdots \pq \cdots \pq}_{(1)}
\;
\underbrace{\pq \cdots \oa \cdots \pq \cdots \oa}_{(2)}
\;
\underbrace{\ast \cdots \ast}_{(3)}
\]
\begin{itemize}
\item[{(1)}] is a prefix of a well-bracketed word such that no
prefix of it (including itself) is a well-bracketed word
\item[{(2)}] is a well-bracketed word \item[{(3)}] has length
equal to the number of $\pq$ in (1).
\end{itemize}
Call the collection of such words $V$. We claim that $U = V$. In the
preceding example, the first two parts of the 3-partition for (i) are
$\pq \, \pq$ and $\pq \, \pq \, \oa \, \oa \, \pq \, \pq \, \oa \, \pq
\, \pq \, \oa \, \oa \, \oa \, \pq \, \pq \, \oa \, \oa$; for (ii)
they are $\pq \, \pq \, \pq \, \pq \, \oa \, \oa \, \pq$ and $\pq \,
\oa \, \pq \, \pq \, \oa \, \oa $.

%As another example, let us take the following:
%\begin{verbatim}
%[  [  [  [  ]  ]  [  [  ]  [  [  ]  ]  ]  [  [  ]  * * * * * * * * *
%1  2  3  4  3  2  5  6  5  7  8  7  5  2  9  10 9
%\end{verbatim}
%then we have: $w = [\;[\;[\;[\;]\;]\;[\;[\;]\;[\;[\;]\;]\;]\;[, x_1 = [\;]$
%As a last example, let us take:
%\begin{verbatim}
%[  [  [  [  ]  ]  [  [  ]  [  [  ]  ]  * * * * *
%1  2  3  4  3  2  5  6  5  7  8  7  5
%\end{verbatim}
%then we have $w = [\;[\;[\;[\;]\;]\;[, x_1 = [\;], x_2 [\;[\;]\;]$\\
%%I'm pretty sure that the ``uniqueness" criteria stipulated in my
%%conjecture is correct.
%This division of a word in the language $U$ into the above format,
%is effective and easy to do: especially if one works ``backwards", i.e.
%starting from the right-hand side of the word (ignoring the stars).

%\begin{proposition} Let $y \in U$. Then there exists a unique
%decomposition of $y$ into $wx_1 \cdots x_m*^n$, such that $w$ is a
%proper prefix of a well-bracketed expression which ends in $[$ and
%such that $|w|_{[} = n$. Furthermore we have $m \geq 0$, and each
%$x_i$ is a well-bracketed expression.
%\end{proposition}

\begin{proposition}\label{prop:decompu} Let $y \in \{\pq, \oa, *\}^*$. Then $y \in U$ if
an only if it has a unique decomposition into $wx*^n$ where $w$, $x$, $n$ satisfy
conditions (1),(2),(3) above respectively.
\end{proposition}

\begin{proof}For convenience, given a word $w \in \Sigma^*$ and $a \in \Sigma$, we
denote by $|w|_a$ the number of occurrences of $a$ in $w$.
\begin{itemize}
\item[${\Rightarrow}:$]  First let us show \emph{existence} of such a
decomposition for $y \in U$. We perform a case analysis.
\begin{enumerate}
\item[(i).] Suppose $y = z\pq *^k$. Then, we simply take $w =
z\pq$, and clearly $|w|_{\pq} = k$ as required. \item[(ii).]
Suppose instead that $y = z\oa *^k$. We give an algorithm which
will decompose $y$ correctly. Clearly, the final $\oa$ must have a
matching $\pq$, say it is the following: $y = z_1 \pq z_2 \oa *^k$
where $\pq z_2\oa$ must be a well-bracketed expression. We know
that $k$ is the label of the final $\oa$. We also know that the
label of the final $\oa$ is the label of the parenthesis which
precedes the matching $\pq $. Now, if $z_1$ has final parenthesis
$\pq$, then we are done as we set $w = z_1$ and $x = \pq z_2 \oa$.

%\begin{eqnarray}
%\nonumber w & = & z_3 \pq \\
%\nonumber x_1 & = & \pq  z_2 \oa
%\end{eqnarray}
If, on the other hand, $z_1$ has final parenthesis $\oa$, then we
repeat this process again as $z_1 = z_3 \pq z_4 \oa$. It is
obvious that this process terminates. Note that if this process
requires $n$ iterations, then $x = \pq z_{2n} \oa \cdots \pq z_4
\oa \pq z_2 \oa$ and $w = z_{2n-1}$.
\end{enumerate}
To show \emph{uniqueness}, suppose that $y \in U$ has two
decompositions: $wx *^k$ and $w' x' *^k$. It suffices to observe
that these decompositions are distinct if and only if $w \not =
w'$. However, note that $|w|_{\pq } = k = |w'|_{\pq }$. Thus, it
must be the case that either $w$ or $w'$ violates the condition
that they must end in a $\pq $.

\item[$\Leftarrow$:]
Again, we perform a case analysis.
\begin{enumerate}
\item[(i).] Suppose that $m=0$. This is obviously true.
\item[(ii).] Suppose that $m>0$ and let $y = w x *^n$. Clearly
$wx$ is the proper prefix of a well-bracketed
expression.

Note that $x = x_1 \cdots x_m$ for some $m \geq 0$, where each
$x_i = \pq  z_i \oa$ and each $z_i$ is well-bracketed. Thus we have $y = w \pq_1 z_1\oa_1 \cdots
\pq_m z_m\oa_m *^n$. Parentheses have been labelled for
demonstrative purposes. For $y$ to be in $U$ the number of stars,
$n$, should be the the label of the parenthesis $\oa_m$. We show
this is indeed the case. The label of $\oa_m$ should be equal to
the label of the parenthesis preceding $\pq_m$, which in this case
would be $\oa_{m-1}$. Continuing in this way, we see that the
label of $\oa_m$ is the label of the last parenthesis of $w$,
which is always a $\pq $, hence, it is the number $|w|_{\pq }$.
\end{enumerate}
\end{itemize}
\end{proof}

\subsubsection*{Constructing a $2$PDA that accepts $U$}

So, how do we construct a $2$PDA that accepts $U$? Let $y$ be the
input. Let us reuse the notation from Proposition \ref{prop:decompu},
so each word in $U$ has a unique decomposition into the form $wx*^n$. The $2$PDA guesses what prefix of $y$ constitutes $w$ - say
it is the first $k$ characters. As it reads the first $k$ symbols
it acts like a $1$PDA checking that $w$ is indeed a prefix of a
well-bracketed word such that no prefix of it is well-bracketed.
However, at the same time we need it to keep track of the number
of $\pq$'s read. In order to do this dual job we need the power of
the $2$PDA. Essentially, the topmost $1$-store behaves like the
stack of a normal PDA which checks for a (proper prefix of a)
well-bracketed word. However, each time we find a $\pq$ we do both
a $\push_1$ and a $\push_2$. Each time we find a $\oa$, we do only
a $\pop_1$. Thus, the number of $1$-stores is the number of
$\pq$'s read in $w$. If it turns out that the first $k$ characters
of the input $y$ do not satisfy the criteria for what $w$ should
be, we immediately reject. Otherwise, we enter phase 2 of the
operation. In phase 2 we check that the next portion of the string, $x$, is
well-bracketed (for this we only need the power of a $1$PDA),
until we come across the first $\ast$. If $x$ was indeed
well-bracketed we proceed to phase $3$, otherwise we abort. In phase
$3$, we perform a $\pop_2$ for each $\ast$ we meet. If we end up with
an empty $2$-store after reading the all the $\ast$'s, we accept. Otherwise, reject.

%\subsubsection*{Example} Using the \emph{last} example on the
%previous page:%

%Phase I: guessing w, and checking it is valid:%

%\begin{verbatim}
%1.[0][S] (initial configuration, here we assume the machine does some "precomputing")
%2.read [, [X0][S]
%3.read [, [XX0][X0][S]
%4.read [, [XXX0][XX0][X0][S]
%5.read [, [XXXX0][XXX0][XX0][X0][S]
%6.read ], [XXX0][XXX0][XX0][X0][S]
%7.read ], [XX0][XXX0][XX0][X0][S]
%8.read [, [XXX0][XX0][XXX0][XX0][X0][S]
%\end{verbatim}
%Now, as the topmost 1-store has never read a $0$ as the topmost
%symbol throughout this computation, and the final config has at
%least one $X$, then we can be sure that what we have read is a
%proper prefix of a well-bracketed expression. Furthermore, we have
%counted the number of ['s read -  i.e. the number of 2-stores
%(other than S).

%Phase II: check the remainder of the expression is valid

%\begin{verbatim}
%9.  silent,  [P][XXX0][XX0][XXX0][XX0][X0][S]
%10. read [,  [XP][XXX0][XX0][XXX0][XX0][X0][S]
%11. read ],  [P][XXX0][XX0][XXX0][XX0][X0][S]
%12. read [,  [XP][XXX0][XX0][XXX0][XX0][X0][S]
%13. read [,  [XXP][XXX0][XX0][XXX0][XX0][X0][S]
%14. read ],  [XP][XXX0][XX0][XXX0][XX0][X0][S]
%15. read ],  [P][XXX0][XX0][XXX0][XX0][X0][S]
%16. read *,  ...
%\end{verbatim}
%As the topmost symbol upon reading the first $*$ is a $P$, and as
%we have never tried to pop a $P$, we know that what we have read
%was a well-bracketed expression. We now are left with the task of
%counting off the stars. We have seen one so far.

%\begin{verbatim}
%17. silent, [XX0][XXX0][XX0][X0][S]
%18. read *, [XXX0][XX0][Xo][S]
%19. read *, [XX0][X0][S]
%20. read *, [X0][S]
%21. read *, [S]
%\end{verbatim}

%If there are any more stars (or any symbols on the input tape),
%reject, otherwise, accept.

%Now the above was, of course, an accepting computation. However,
%had we chosen $w$ incorrectly, the input would have been rejected,
%as the Phase II operation would result in a non-well bracketed
%expression.


% Section: The Theorem: statement and explanation
\section{The Theorem: motivation and explanation}

We now present our main result.

\begin{theorem} If $G$ is a higher-order grammar at level $2$ that is
not assumed to be safe, there exists a non-deterministic $2$PDA
that accepts the same language. Moreover, the conversion is effective.
\end{theorem}

Our proof is split into two transformations. Given a level-$2$ grammar
$G$, we show:
\begin{enumerate} \item The (compacting) pointer machine for $G$ is simulated by a
$2$PDAL, where $2$PDAL is a machine that has yet to be
introduced. (Section 5)
\item We then show that a $2$PDAL for $G$ can be simulated by a
nondeterministic $2$PDA. (Section 6)
\end{enumerate}

Combining our result with that of \cite{DG86}, we have:

\begin{corollary}
Every string language that is generated by an unsafe $2$-grammar can
also be generated by some safe (non-deterministic) $2$-grammar.
\end{corollary}

Thus safety is not really a restriction at level $2$ for string languages.


% Section: Equivalence between 2PDAPs and 2PMs (extension to all orders?)
\section{Simulating $2$PMs by $2$PDALs}

\subsection{Understanding KNU's proof}

In \cite{KNU02} it was shown that a term tree generated by a safe
grammar of level $n$ is accepted by a pushdown automata of the
same level. Their proof, which given a grammar $G$ constructs a
corresponding pushdown automaton, can easily be adapted to work in
the string-language setting: all one needs to do is incorporate
the input alphabet. Recall, however, that this result for the
string-language setting was proved earlier by Damm and Goerdt
\cite{DG86}. Here we present an adaption of the proof in \cite{KNU02},
specialised to (level $2$) string languages.

\begin{theorem} Let $G$ be a $2$-grammar that generates a string
language. Then $L(G)$ is accepted by some $2$PDA.
\end{theorem}

\begin{proof}
We use the same setup as in Section 5.2 of \cite{KNU02}, but now we
incorporate an input string over the alphabet $\Sigma$. Recall that
there are 6 cases for the transition function; they are given in
Fig.~\ref{fig:KNU}.
\end{proof}

\begin{figure*}[t]
\begin{center}
\makebox{
\begin{shadowbox}[16cm]
\begin{eqnarray}
(q_0, a, {Dt_1 \cdots t_n}) & \rightarrow & (q_0,
\push_1(E)) \mbox{ if $Dx_1 \cdots x_n \larr{a} E$ }\\
(q_0, \epsilon, e)& \rightarrow &
accept\\
(q_0, \epsilon, {x_j})& \rightarrow & (q_j, \pop_1) \mbox{ if $x_j: o$}\\
(q_0, \epsilon, {x_j t_1 \cdots t_n}) & \rightarrow & (q_j, \push_2 \compose \pop_1) \mbox{ if $x_j$ has level $> 0$}\\
{1< j \leq n,} (q_j, \epsilon, {\$t_1 \cdots t_n}) & \rightarrow & (q_0, \pop_1 \compose {\push_1 (t_j)})\\
{j > n,} (q_j, \epsilon, {\$t_1 \cdots t_n}) & \rightarrow & (q_{j-n}, \pop_2)\\
\nonumber
\end{eqnarray}
\end{shadowbox}}
\caption{Adapted transition rules from \cite{KNU02}\label{fig:KNU}}
\end{center}
\noindent\emph{Convention.} In the Figure $x_j$ means the $j$-th
formal parameter of the relevant non-terminal.
\end{figure*}

Let us examine why their proof fails if we attempt to apply it
(blindly) to a level-$2$ unsafe grammar. As an example, we
consider the grammar given in Example~\ref{ex:ex1}. Recall that
the word $h_1h_3h_2f_1a$ is \emph{not} in the language.

The automaton starts off in the configuration $(q_0, h_1h_3h_2f_1a,
\mkstore{\mkstore{S}})$, after a few steps we reach the following configuration:
\begin{eqnarray}
\nonumber (q_0, & h_2f_1a, & \mkstore{\mkstore{\varphi B, D(D \varphi x) y (\varphi y), DGAB,S}})
\end{eqnarray}
As the topmost item, $\varphi B$, is headed by an level-$1$
variable, we need to find out what $\varphi$ is in order to
proceed. Note that $\varphi$ is the $1$st formal parameter of the
preceding item: $ D (D \varphi x) y (\varphi y)$, i.e., it refers
to $D \varphi x$. To this end, we perform a $\push_2$ and then
perform a $\pop_1$, and replace the topmost item with $D \varphi
x$. In other words, we have applied rule $4$ followed by rule $5$
to arrive at:
\begin{eqnarray}
\nonumber (q_0, & h_2f_1a, & \hbox{\tt [}\mkstore{\mklink{(D \varphi x)}{1-}, DGAB, S},\\
\nonumber & & \mkstore{\mklink{\varphi B}{1+}, D (D \varphi x) y (\varphi
y), DGAB,S}\hbox{\tt ]})
\end{eqnarray}
Here we have labelled two store items, one with a $1-$ and the
other with a $1+$. These labels are not part of the store
alphabet, they have been added for our benefit: so that we may
identify these two store items later on.

The crux behind their construction is the following. Suppose we
meet the item $\mklink{D\varphi x}{1-}$ later on in the
computation, and suppose that we would like to request its third
argument, meaning we would be in state $q_3$. Note, however, that
$\mklink{D \varphi x}{1-}$ has only 2 arguments. The missing
argument can be found by visiting the item $\mklink{\varphi
B}{1+}$. Hence the labelling. We need to ensure that there is a
systematic way to get from $\mklink{D \varphi x}{1-}$ to
$\mklink{\varphi B}{1+}$ whenever we are in a state $q_n$ for $n >
2$ and we have $\mklink{D \varphi x}{1-}$ as our topmost symbol.
This systematic way suggested by \cite{KNU02} is embodied by rule
$6$ of Fig.~\ref{fig:KNU}. In particular, it says that all we need
to do is perform a $\pop_2$, followed by a change in state to
$q_{n-2}$, and to repeat if necessary. Note that if we applied
rule $6$ to the current configuration, we would indeed be brought
to the right place, $\mklink{\varphi B}{1+}$. We will see,
however, that with an unsafe grammar, this invariant may be
violated.

After a few more steps of the $2$PDA we will arrive at another
configuration where the topmost symbol is headed by a level-$1$
variable:

\begin{eqnarray}
\nonumber (q_0, & f_1a, & \hbox{\tt [}\mkstore{\varphi x,H (F y) x,\mklink{(D
\varphi x)}{1-}, DGAB, S}, \\
\nonumber & &\mkstore{\mklink{\varphi B}{1+}, D (D \varphi x) y (\varphi
y), DGAB,S}\hbox{\tt ]})
\end{eqnarray}
Therefore, we next get:
\begin{eqnarray}
\nonumber (q_0, & f_1a, &  \hbox{\tt [}\mkstore{\mklink{F y}{2-}, \mklink{(D
\varphi x)}{1-}, DGAB, S},\\
\nonumber  & &\mkstore{\mklink{\varphi x}{2+},H (F y) x,\mklink{(D \varphi x)}{1-}, DGAB, S},\\
\nonumber & &\mkstore{\mklink{\varphi B}{1+}, D (D \varphi x) y (\varphi
y), DGAB,S}\hbox{\tt ]})
\end{eqnarray}

Again we have labelled a new pair of store items, so that the same
principle applies: if we want the missing argument of $\mklink{F
y}{2-}$, then we will be able to find it at $\mklink{\varphi
x}{2+}$. The next configuration is now:

\begin{eqnarray}
\nonumber (q_0, & a, &  \hbox{\tt [}\mkstore{x,\mklink{F y}{2-}, \mklink{(D \varphi
x)}{1-}, DGAB, S}, \\
\nonumber & & \mkstore{\mklink{\varphi x}{2+},H (F y) x,\mklink{(D \varphi x)}{1-}, DGAB, S},\\
\nonumber & & \mkstore{\mklink{\varphi B}{1+}, D (D \varphi x) y (\varphi
y), DGAB,S}\hbox{\tt ]})\\
\nonumber \rightarrow (q_1, & a, &  \hbox{\tt [}\mkstore{\mklink{F y}{2-}, \mklink{(D \varphi
x)}{1-}, DGAB, S}, \\
\nonumber & & \mkstore{\mklink{\varphi x}{2+},H (F y) x,\mklink{(D \varphi x)}{1-}, DGAB, S},\\
\nonumber & & \mkstore{\mklink{\varphi B}{1+}, D (D \varphi x) y (\varphi
y), DGAB,S}\hbox{\tt ]})\\
\nonumber \rightarrow (q_0, & a, &  \hbox{\tt [}\mkstore{y, \mklink{(D \varphi
x)}{1-}, DGAB, S}, \\
\nonumber & & \mkstore{\mklink{\varphi x}{2+},H (F y) x,\mklink{(D \varphi x)}{1-}, DGAB, S},\\
\nonumber & & \mkstore{\mklink{\varphi B}{1+}, D (D \varphi x) y (\varphi
y), DGAB,S}\hbox{\tt ]})
\end{eqnarray}

However, at this point $y$ is the $3$rd argument of the preceding
item $\mklink{(D \varphi x)}{1-}$, therefore, we have:

\begin{eqnarray}
\nonumber (q_3, & a, & \hbox{\tt [}\mkstore{\mklink{(D \varphi x)}{1-}, DGAB, S}, \\
\nonumber & & \mkstore{\mklink{\varphi x}{2+},H (F y) x,\mklink{(D \varphi x)}{1-}, DGAB, S},\\
\nonumber & &\mkstore{\mklink{\varphi B}{1+}, D (D \varphi x) y (\varphi
y), DGAB,S}\hbox{\tt ]})
\end{eqnarray}

By rule $6$ we arrive at: (in the following $\rightarrow_n$ means
$n$ steps of $\rightarrow$)
\begin{eqnarray}
\nonumber (q_1, & a, & \hbox{\tt [}\mkstore{\mklink{\varphi x}{2+},H (F y) x,\mklink{(D \varphi x)}{1-}, DGAB, S},\\
\nonumber & & \mkstore{\mklink{\varphi B}{1+}, D (D \varphi x) y (\varphi y),
DGAB,S}\hbox{\tt ]} )\\
\nonumber \rightarrow_1  (q_0, & a, & \hbox{\tt [}
\mkstore{x,H (F y) x,\mklink{(D \varphi x)}{1-}, DGAB, S},\\
\nonumber && \mkstore{\mklink{\varphi B}{1+}, D (D \varphi x) y (\varphi y),
DGAB,S} \hbox{\tt ]})\\
\nonumber \rightarrow_1 (q_2, & a, &\hbox{\tt [}\mkstore{H (F y) x,\mklink{(D \varphi x)}{1-}, DGAB, S},\\
\nonumber & &\mkstore{\mklink{\varphi B}{1+}, D (D \varphi x) y (\varphi
y), DGAB,S}\hbox{\tt ]})\\
\nonumber \rightarrow_2 (q_2, & a, & \hbox{\tt [}\mkstore{\mklink{(D \varphi x)}{1-}, DGAB, S},\\
\nonumber & &\mkstore{\mklink{\varphi B}{1+}, D (D \varphi x) y (\varphi
y), DGAB,S}\hbox{\tt ]}
)\\
\nonumber \rightarrow_2 (q_2, & a, & \hbox{\tt [}\mkstore{DGAB, S},\\
\nonumber && \mkstore{\mklink{\varphi B}{1+}, D (D \varphi x) y (\varphi
y), DGAB,S}\hbox{\tt ]}
)\\
\nonumber \rightarrow_2 (q_0, & \epsilon, & \hbox{\tt [}\mkstore{e, S},\\
\nonumber &&\mkstore{\mklink{\varphi B}{1+}, D (D \varphi x) y (\varphi
y), DGAB,S}\hbox{\tt ]}
)
\end{eqnarray}
Note that we have accepted $h_1h_3h_2f_1a$ which is incorrect! Their
construction only works under the assumption that the grammar is
safe. However, the labels we have used lead us to the construction of
a machine which can remedy this problem.

\subsection{$2$PDAL: $2$PDA with links}

Let us suppose that the labels used in the above example $1+, 1-,
2+, \cdots$ were actually part of the alphabet. This would, in
general, lead to an infinite alphabet, but let us ignore the
finiteness requirement for now. Provided that each time we create
a new pair of labels (the $+$ and $-$ part), we ensure they are
unique, then these labels provide a way of always jumping to the
correct $1$-store when we are looking for missing arguments. Why?
Because each time we want the missing argument of an item labelled
with $n-$, we would simply perform as many $\pop_2$'s as necessary
until our topmost symbol was labelled with the corresponding $n+$!
To see how this would work, let us backtrack to the following
configuration in the above example:

\begin{eqnarray}
\nonumber (q_3, & a, &  \hbox{\tt [}\mkstore{\mklink{(D \varphi x)}{1-}, DGAB, S},\\
\nonumber && \mkstore{\mklink{\varphi x}{2+},H (F y) x,\mklink{(D \varphi x)}{1-}, DGAB, S},\\
\nonumber &&\mkstore{\mklink{\varphi B}{1+}, D (D \varphi x) y (\varphi
y), DGAB,S}\hbox{\tt ]})
\end{eqnarray}

Applying the rule we have just said, we will keep performing a $\pop_2$ until we find a corresponding $1+$. This brings us to:
\begin{eqnarray}
\nonumber (q_1, & a, &  \mkstore{\mkstore{\mklink{\varphi B}{1+}, D (D
\varphi x) y (\varphi y), DGAB}})\\
\nonumber \rightarrow_1 (q_0, & a, &\mkstore{\mkstore{B , D (D
\varphi x) y (\varphi y), DGAB}} )
\end{eqnarray}
which is indeed what we wanted, and the word is rejected, as the
only rule for $B$ is $B \larr{b} e$. In fact, using the same
computation, but on the word $h_1h_3h_2f_1b$ we end up in an
accepting state.\\

Thus, for the remainder of this section, we afford ourselves the
luxury of this embellished $2$PDA, which we call $2$PDA with links, or
simply $2$PDAL. It is a $2$PDA as defined earlier, but we allow
ourselves to adorn items with matching labels, as we have done in the
previous example. We work under the assumption that each time we
create a new pair of labels they are ``fresh" and unique.

\subsection{Formal definition of $2$PDAL}

Formally each item of a $2$PDAL will now be embellished with
labels from the set
\[
\{n{+} : n \geq 1 \} \cup \{n{-} : n \geq 1 \}
\]
It is possible for an item to have zero, one or two labels -- no
other possibilities exist. We write labels as superscripts, as in
$\mklink{a}{}$ (or simply $a$), $\mklink{a}{3+}$ and
$\mklink{a}{3+, 4-}$. These superscripts are sets of at most two
elements, ranged over by $\lambda$; thus we have ${\langle 3+
\rangle} \cup {\langle 4- \rangle} = \langle 3+, 4- \rangle =
\langle 4-, 3+\rangle$. We shall see that in the case where an
item has two labels, one of these will always be a $+$ and the
other a $-$.\\

These labels come in matching pairs; we will see that if there is
an item labelled by $m-$ then there will be one labelled by $m+$
(although the converse is not necessarily true).  Thus, if one
item is labelled by $m-$ and another is labelled by $m+$, this
pair will be referred to an \textbfit{instance} of the
\textbfit{link} $m$. Each link $m$ may have several instances,
i.e.~several such pairs. For such a given pair, the item which
gains the $-$ part will be called the \textbfit{start point},
whereas that which receives the corresponding $+$ part, the
\textbfit{end point}. This terminology should coincide with
intuition.

%Thus, one may take as the
%formal definition of a link $m$ the set of pairs:
%\begin{equation}
%\nonumber \{(s,t) : \mbox{$s$ has label $m-$ and $t$ has label
%$m+$}\}
%\end{equation}
%where $s$ and $t$ refer to the positions of items in the stack.
%Furthermore, an instance of a link $m$ is an element of the above set.\\

In addition to the usual operations of a $2$PDA, a $2$PDAL has an
iterated form of $\pop_2$, parameterised over links $m$, defined
as follows: for $s$ ranging over 2-stores
\begin{eqnarray}
\nonumber \pop_2(m) \, s & = & \left \{
\begin{array}{ll}
s & \mbox{if $\mytop_1(s)$ has label $m+$}\\
\pop_2(m)(\pop_2(s)) & \mbox{otherwise}
%\pop_2(m) \compose \pop_2(s) & \mbox{otherwise.}
\end{array}
\right .
\end{eqnarray}
For convenience we write $\repl_1(a)$ as a shorthand for $\pop_1
\compose \push_1(a)$ i.e.~replacing the $\mytop_1$ item by $a$. \\

Given a 2-grammar $G$, which is not assumed to be safe,
transitions of the corresponding $2$PDAL, written $2PDAL_G$, are
defined by induction over the set of rules in Fig.~\ref{fig:pdal}.
The store alphabet is a subset of the (finite) set of all
subexpressions of the right hand sides of the productions in $G$.
We assume that each production rule of the grammar assumes the
following format:
\begin{equation}
\label{convention} F \varphi_1 \cdots \varphi_m x_{m+1} \cdots x_{m+n} \larr{a} E
\end{equation}
where the $\varphi$'s are used for level-$1$ parameters, and the
$x$'s are used for level-$0$ parameters.  Also note that, in
Fig.~\ref{fig:pdal}, $\$$ is a place holder for either a
non-terminal or variable. As in \cite{KNU02}, the automaton works
in phases beginning and ending in distinguished states $q_i$, with
some auxiliary states in between. We assume, for the sake of
clarity, that these auxiliary states are disjoint from $\{q_i : 0
\leq i \leq m\}$, where $m$ is the maximum of the arities of any
non-terminal or variable occurring in the grammar. If a
non-terminal or a variable has type $A_1 \rightarrow \cdots \rightarrow A_n
\rightarrow o$, then it is said to have arity $n$. Thus, when we
refer to a state $q_i$, it is \emph{not} an auxiliary state.

\begin{figure*}[t]
\begin{center}
\makebox{
\begin{shadowbox}[16.5cm]
\[\begin{array}{rllr}
(q_0, a, {Dt_1 \cdots t_n}^{\lambda}) & \rightarrow & (q_0, \push_1(E)) \mbox{ if $Dx_1 \cdots x_n \larr{a} E$ } & \rm{(1)}\\
(q_0, \epsilon ,e )& \rightarrow & accept & \rm{(2)}\\
(q_0, \epsilon ,{x_j})& \rightarrow & (q_j, \pop_1) & \rm{(3)}\\
(q_0, \epsilon, {\varphi_j t_1 \cdots t_n}^{\lambda}) &
\rightarrow & (q_0, \repl_1( {\varphi_j t_1 \cdots t_n}^{\lambda
\cup \anglebra{m+}})\compose \push_2 \compose \pop_1
\compose \repl_1(\mklink{s_j}{m-}))\\
& & \hbox{where $m$ is fresh and ${D s_1 \cdots
s_{n'}}^{\lambda'}$ precedes ${\varphi_j t_1 \cdots
t_n}^\lambda$.} & \rm{(4)}\\
1 \leq j \leq n, (q_j, \epsilon, {\$t_1 \cdots t_n}^\lambda) & \rightarrow & (q_0, \repl_1(t_j)) & \rm{(5)}\\
j > n, (q_j, \epsilon, {\$t_1 \cdots t_n}^\lambda) & \rightarrow &
(q_{j-n}, \pop_2(m)) \mbox{ if $m- \in \lambda$}& \rm{(6)}\\
\end{array}\]
\end{shadowbox}}
\end{center}
\caption{Transition rules of the $2$PDAL, $2PDAL_G$.
\label{fig:pdal}}
\end{figure*}

%\begin{figure*}[t]
%\begin{center}
%\makebox{
%\begin{shadowbox}[15cm]
%\[\begin{array}{rll}
%(q_0, a, {Dt_1 \cdots t_n}^{\lambda}) & \rightarrow & (q_0, \push_1(E)) \mbox{ if $Dx_1 \cdots x_n \larr{a} E$ }\\
%(q_0, \epsilon ,e )& \rightarrow & accept\\
%(q_0, \epsilon ,{x_j})& \rightarrow & (q_j, \pop_1)\\
%(q_0, \epsilon, {\varphi_j t_1 \cdots t_n}^{\lambda}) &
%\rightarrow & (q_0, \repl_1( {\varphi_j t_1 \cdots t_n}^{\lambda
%\cup \anglebra{m+}})\compose \push_2 \compose \pop_1
%\compose \repl_1(\mklink{s_j}{m-}))\\
%& & \hbox{where $m$ is fresh and ${D s_1 \cdots
%s_{n'}}^{\lambda'}$ precedes ${\varphi_j t_1 \cdots
%t_n}^\lambda$.}\\
%j \geq 1, (q_j, {\$t_1 \cdots t_n}^\lambda) & \rightarrow &
%\left \{ \begin{array}{ll} (q_0, \repl_1(t_j)) & \mbox{if } j \leq n\\
%(q_{j-n}, \epsilon, \pop_2(m)) & \mbox{if $j > n$ and $m{-} \in
%\lambda$}
%\end{array} \right .\\
%\end{array}\]
%\end{shadowbox}}
%\end{center}
%\caption{Transition rules of the $2$PDAL, $2PDAL_G$. \label{fig:pdal}}
%\end{figure*}

\subsection{Proof of correctness}

\begin{theorem}\label{labels}The language generated by a (possibly unsafe)
$2$-grammar is accepted by $2PDAL_G$.
\end{theorem}

\begin{remark} We believe this theorem can be extended to unsafe
grammars of all levels, but we do not prove this here.
\end{remark}

First, some notation. Let $s$ be a 2-store. We define
\[\begin{array}{rll}
s' \sqsubseteq s & \iff & \hbox{$s'$ is obtained from $s$
by performing zero or more $\pop_1$'s and $\pop_2$'s} \\
s' \sqsubseteq_i s & \iff & \hbox{$s'\sqsubseteq s$ and the final action
is a $\pop_1$ (hence the ``$i$'' for internal)}\\
s' \sqsubseteq_1 s & \iff & \hbox{$s' =
\pop_1^n(s)$ for some $n > 0$}\\
s' \sqsubseteq_2 s & \iff & \hbox{$s' = \pop_2^n(s)$ for some $n > 0$.}
\end{array}\]

We define a function $pm$ that transforms a $2$-store of $2PDAL_G$
to an equivalent stack of a $2$PM as follows:
\[\begin{array}{rlll}
pm(s) & = &
\left \{
\begin{array}{ll}
{\mkstore{S}} & \mbox{if } \mytop_1(s) = S\\
\mytop_1(s)_{pm(\pop_1(s))} : pm(\pop_1(s)) & \mbox{elseif } \mytop_1(s) : o\\
join({\mytop_1(s)_{pm(\pop_1(s))}}, split(pm(\pop_2(m)s))) &
\hbox{elseif $\mytop_1(s)$ has label $m{-}$}\\

\end{array}
\right .\\
\end{array}\]
where the auxiliary functions $split$ and $join$ are defined as
follows: where $t$ ranges over 2PM stacks
\[\begin{array}{rlll}
split(t) & = & (\anglebra{u_1, \cdots, u_r}, {\sf tail}(t)) &
\hbox{where ${\sf head}(t) = \varphi u_1 \cdots u_r$} \\
join(\theta, (\anglebra{u_1, \cdots, u_r}, t)) & = & (\theta . u_1
\cdots u_r) \; : \; t \\
\end{array}\]
(Note: The case of $r = 0$ never arises.)

%\begin{example}Here is an example.

%\begin{eqnarray}
%\nonumber q_0, & & [\;x, H\varphi y, \mklink{D\varphi x }{1-}, Dgab, S]\\
%\nonumber       && [\;\mklink{\varphi a }{3+}, \mklink{(L\varphi) }{2-}, K(fx)x, H\varphi y, \mklink{D\varphi x }{1-}, Dgab, S]\\
%\nonumber       && [\;\mklink{\varphi d }{2+},M\varphi x,J(L\varphi)x,K(fx)x, H\varphi y, \mklink{D\varphi x }{1-}, Dgab, S]\\
%\nonumber       && [\;\mklink{\varphi c }{1+}, D(D\varphi x)ab, Dgab, S]
%\end{eqnarray}
%applying $pm$ to the above 2-store gives us:
%\begin{equation}
%[x_{p_1}, H\varphi y_{p_2}, D\varphi x_{p_3}.c_{p_4}, D(D\varphi
%x)ab_{p_5}, Dgab_{p_6}, S]
%\end{equation}
%where
%\begin{eqnarray}
%\nonumber p_1 &=& [H\varphi y, D\varphi x.c, D(D\varphi x)ab, Dgab, S]\\
%\nonumber p_2 &=& [D\varphi x.c, D(D\varphi x)ab, Dgab, S]\\
%\nonumber p_3 &=& [Dgab, S]\\
%\nonumber p_4 &=& [D(D\varphi x)ab, Dgab,S]\\
%\nonumber p_5 &=& [Dgab, S]\\
%\nonumber p_6 &=& [S]
%\end{eqnarray}
%\end{example}

%\subsection{The induction argument}

\begin{lemma}\label{lem:main} If $(q_i,s)$ is a reachable configuration of $2PDAL_G$, then the following hold, where $s = \mkstore{s_1, s_2,
\cdots, s_n}$:
\begin{enumerate}
\item[(i)] If $s' \sqsubseteq s$ and $\mytop_1(s') = u^\lambda$
such that $m- \in \lambda$, and $u : \tau$, then there exists
unique $s'' \sqsubseteq_2 s'$ such that $\mytop_1(s'') = {\varphi
\cdots}^{\lambda'}$ and $m+ \in \lambda$ and $\varphi : \tau$.

%has label $m+$. Furthermore, $\mytop_1(s'') = \varphi \cdots$ such
%that $\varphi : \tau$.

\item[(ii)] If $s' \sqsubseteq_i s$ then $\mytop_1(s')$ is headed by
a non-terminal.

\item[(iii)] \label{level1} If $s' \sqsubseteq s$, and
$\mytop_1(s') = D \cdots^\lambda$ then all level-$1$ arguments of
$D$ are present. Furthermore, $\lambda$ is either $\langle m-
\rangle$ for some $m$ (in the case that $D \cdots $ is of type
level $1$) or it is empty (in the case that $D \cdots : o$).

\item[(iv)] \label{fps} If an item $t$ occurs in a $1$-store
directly atop another of the form $D \cdots^\lambda$, where $D$ is
a non-terminal (by the above), then all the variables occurring in
$t$ are formal parameters of $D$.

\item[(v)] \label{rightnum} If $i > 0$ and $\mytop_1(s) = {\$
\cdots }^\lambda$, then  $\$ : (A_1, \cdots, A_n, o)$ for some $n
\geq i$.
% Furthermore, if $i > k$, then $m- \in \lambda$ for
%some $m$.

%Furthermore, if $\mytop_1(s) = \$ t_1 \cdots t_n^\lambda$ and $n <
%i$ then $m- \in \lambda$ for some $m$.

\item[(vi)] \label{musthavelabel} If $s' \sqsubseteq s$ and
$\mytop_1(s') = {\varphi \cdots}^\lambda$ has type level $1$, then
$m- \in \lambda$ for some $m$.

%\item[(vi)] \label{atleastone} Finally, if $i=0$ and $\mytop_1(s)
%= \varphi \cdots^\lambda $, then $\lambda$ is either $\langle m-
%\rangle$ or empty.

%And, for $j = 2, \cdots, n$, $\mytop_1(s_j)$ is
%either $\mklink{\varphi \cdots}{k_j+}$ or $\mklink{\varphi \cdots
%}{k_j+,l_j-}$.

%\item suppose that $s_j$
%
%\item suppose that $s_j$, for $j = 2, \cdots, n$ is headed by
%$\varphi x_1 \cdots x_n$ of type $\tau$, then the topmost incomplete
%application of $s_{j-1}$ has type $\tau$
\end{enumerate}
\end{lemma}

\begin{proof} By induction on the number of transitions, where a
transition constitutes the application of one rule in
Fig.~\ref{fig:pdal}.
\end{proof}


\begin{lemma}\label{lem:order}
Let $(q_i, s)$ be a reachable configuration of $2PDAL_G$, and let
\[ \mkstore{\cdots,
a^{\lambda \cup \langle m- \rangle}, \cdots , \mklink{b}{k-},
\cdots}\] be a $1$-store in $s$. The end point of the link $m$ is
a $1$-stack strictly above the end point of the link $k$.
\end{lemma}

\begin{proof} Routine induction on the number of
transitions.
\end{proof}

\begin{lemma} \label{lem:pmpop} If $(q_i, s)$ is a reachable
configuration of $2PDAL_G$ and $s' \sqsubseteq s$ and $s'' =
\pop_1(s')$, then $pm(s') = (\mytop_1(s')_{pm(s'')} . \cdots):
\cdots : pm(s'')$.
\end{lemma}

\begin{proof}
By induction on the number of transitions, where a transition
constitutes the application of one rule. For the inductive step
one performs a case analysis on $\mytop_1(s)$. The only case which
requires some care is when we are in state $q_0$ with $\mytop_1(s) = {\varphi_i u_1 \cdots
u_n}^{\lambda}$; we do this here. If $\mytop_1(s) = {\varphi_i u_1
\cdots u_n}^{\lambda}$, then by Lemma~\ref{lem:main}(iv) we must have that $s = \mkstore{s_1, s_2, \cdots, s_k}$
where $s_1 = \mkstore{{\varphi_i u_1 \cdots u_n}^{\lambda}, {D v_1
\cdots v_{n'}}^{ \lambda'}, s_{11}, \cdots, s_{1N}} $. Now, the
next transition results in $(q_0, t)$ where $t =
\mkstore{s_0, s_1', s_2, \cdots s_k}$, where
\begin{eqnarray}
\nonumber s_0 & = & \mkstore{\mklink{v_i}{m-}, s_{11}, \cdots, s_{1N}}\\
\nonumber s_1' &  = & \mkstore{{\varphi_i u_1 \cdots u_n}^{\lambda \cup \langle m+ \rangle}, {D v_1 \cdots v_{n'}}^{\lambda'}, s_{11}, \cdots, s_{1N}}
\end{eqnarray}
where $m$ is a fresh label. We now check that the induction
hypothesis holds. Certainly, it holds for any $s' \sqsubseteq
\pop_2(t)$ (as these have not been affected by the transition).
However, we do need to check the cases where:
\begin{enumerate}\item $s' = t$ and the case where\item  $s' \sqsubseteq_1
t$\end{enumerate}
For (1) we note that:
\begin{equation}
\nonumber pm(t) = join((v_i)_{pm(\pop_1(t))}, split(pm(\pop_2(m)(t))))
\end{equation}
However, $pm(\pop_2(m)(t))$ is equivalent to $pm(s)$, as the only
difference between $\pop_2(m)(t)$ and $s$ is that the topmost
symbol of the former has the extra label $m+$, but it is obvious
from the definition of $pm$, that the extra $m+$ makes no
difference to the output. Hence we have:
\begin{eqnarray}
\nonumber pm(t) & = & join((v_i)_{pm(\pop_1(t))}, split(pm(s)))\\
\nonumber pm(t) & = & ((v_i)_{pm(\pop_1(t))} . \cdots) : ({\sf tail}\;pm(s))
\end{eqnarray}
Now, by the induction hypothesis we have:
\begin{eqnarray}
\nonumber pm(t) & = & ((v_i)_{pm(\pop_1(t))} . \cdots) : ({\sf tail}\;((\varphi_i u_1 \cdots {u_n}_{pm(\pop_1(s))} . \cdots) : \cdots : pm(\pop_1(s)))) \\
\nonumber pm(t) & = & ((v_i)_{pm(\pop_1(t))} . \cdots) : \cdots : pm(\pop_1(s))
\end{eqnarray}
It is easy to check, with the aid of Lemma~\ref{lem:order} that $pm(\pop_1(t)) = pm(\pop_1(\pop_1(s)))$. Furthermore, by the induction hypothesis, $pm(\pop_1(s)) = \cdots : pm(\pop_1(\pop_1(s)))$. Hence we have:
\begin{eqnarray}
\nonumber pm(t) = ((v_i)_{pm(\pop_1(t))} . \cdots) : \cdots :
pm(\pop_1(t))
\end{eqnarray}
As for case (2), this follows easily with the aid of Lemma~\ref{lem:order}.\\
\end{proof}

\begin{lemma}\label{lem:q0}
Let $(q_j,s)$ be a reachable configuration of $2PDAL_G$ for $j>0$
such that $pm(s) = (({s_0})_{p_0}({s_1})_{p_1} \cdots
({s_j})_{p_j} \cdots ({s_n})_{p_n}) : \cdots : p_j$. If $s_j$
After a finite number of transitions we will reach a configuration
$(q_0, s')$ such that $pm(s') = ({s_j})_{p_j}:p_j$. In the case that
$s_j$ contains no variables, $p_j$ is dictated by the function $pm$.
\end{lemma}

\begin{proof}
Induction with respect to $s$. For the inductive step we perform a
case analysis on $\mytop_1(s)$. Suppose that $\mytop_1(s) = {\$
t_1 \cdots t_n}^{\lambda}$ and that $j \leq n$, then the
hypothesis is immediate. If, on the other hand we have $j > n$,
then it must be the case that that $m- \in \lambda$ for some $m$,
and the next configuration is $(q_{j-n},s')$ where $s' =
\pop_2(m)(s)$. The result now follows by the induction hypothesis
and by noting that the ${j-n}$th argument of $\head pm(s')$ equals
the $j$th argument in $\head pm(s)$.
%Note that
%\begin{equation}
%\nonumber pm(s) = (\$t_1 \cdots t_n)_{q}({u_1})_{p_1} \cdots
%({u_r})_{p_r} : {\sf tail}(pm(\pop_2(m)(s)))
%\end{equation}
%for some PM stacks $q, p_1, \cdots, p_r$. Also note that:
%\begin{equation}
%\nonumber pm(s') = \$'_p({u_1})_{p_1} \cdots ({u_r})_{p_r} : p'
%\end{equation}
%where $\mytop_1(s')$ is headed by $\$'$ and for some PM stack $p$,
%where $p' = {\sf tail}(pm(\pop_2(m')(s')))$ or $p' =
%(pm(\pop_1(s')))$ depending on $\mytop_1(s')$. The result now
%follows by the induction hypothesis.
\end{proof}

We say a configuration is \textbfit{stable} if it is of the form
$(q_0, s)$. A stable transition is one from one stable
configuration $(q_0, s)$ to another $(q_0, s')$ such that $(q_0,
s) \rightarrow \cdots \rightarrow (q_0, s')$ and there are no
stable configurations in the $\cdots$'s.

\begin{lemma} Let $G$ be a level $2$ (possibly unsafe) grammar. If $(q_0, s)$ is a reachable configuration of the $2PDAL_G$, then $pm(s)$ is a reachable configuration of the corresponding compacting pointer
machine for $G$.
\end{lemma}

\begin{proof} Induction on the number of stable transition.
With the aid of Lemma~\ref{lem:pmpop} and Lemma~\ref{lem:q0}, it follows almost
immediately.
\end{proof}

%\begin{enumerate}
%\item Each stack item has either no label $()$, a single label
%$(m+)$ or $(m-)$, or a pair $(m+,n-)$ one of which is always a $+$
%and the other a $-$, furthermore, $m \not = n$. \item an item in
%$s_2, \cdots, s_n$ is marked with a $(m+)$ if an only if it is of
%the form $\varphi x_1 \cdots x_n$ and has type $0$. \item an item in
%$s_2, \cdots, s_n$ is marked with $(m+,n-)$ if and only if it is
%of the form $\varphi x_1 \cdots x_n$ and has type $1$ \item an item
%is marked with a $(n-)$ is and only if it is of the form $D t_1
%\cdots t_n x_1 \cdots x_m$ or $ft_1 \cdots t_m$ for $m \geq 0$,
%such that the term is of level $1$. \item All items of terms of
%level $0$ not headed by a variable have no label. \item All
%internal items are headed by a non-terminal, and all of them
%contain all their level 1 parameters \item If an item $t$ occurs
%in a $1$-stack directly atop another of the form $F \cdots$, where
%$F$ is a non-terminal (by the above), then all the variables
%occurring in $t$ are formal parameters of $F$. \item
%\label{rightnum} If $i \not = 0$ then $\mytop_1(s)$ has at least $i$
%formal parameters. \item \label{atleastone} For $j = 2, \cdots,
%n$, $s_j$ is headed by either $\varphi x_1 \cdots x_n (m+)$ or $\varphi
%x_1 \cdots x_n (m+,l-)$ for some $n \geq 1$ \item suppose that
%$s_j$, for $j = 2, \cdots, n$ is headed by $\varphi x_1 \cdots x_n$
%of type $\tau$, then the topmost incomplete application of
%$s_{j-1}$ has type $\tau$ \item The last item in each $1$-stack,
%is always $S$, where $S$ is the start symbol of the higher-order
%grammar. \item $pm(s)$ is a reachable configuration of the
%corresponding pointer machine. \item Finally, if $s' \sqsubseteq_i
%s$ then $pm(s')$ is a suffix of $pm(s)$ and the pointers are in
%agreement.
%\end{enumerate}
%\begin{proof}
%By the induction on the number of stable transitions. The base
%case is obvious. As for the inductive step, suppose that $s=[s_1,
%s_2, \cdots, s_k]$. We perform a case by analysis on the topmost
%symbol. The interesting cases are the following:
%
%\begin{enumerate} \item Suppose that the topmost symbol is $\varphi x_1 \cdots x_n ()$, note the
%absence of a label implies it is of type $0$. In particular, the
%2-store therefore looks something like $[[\varphi x_1 \cdots x_n ,
%s_{11}, \cdots ,s_{1N}], s_2, \cdots, s_k]$. Now, clearly, we
%have:
%\begin{equation} \label{prev}
%pm(s) = [\varphi x_1 \cdots x_n , A_1 , \cdots , A_2 , \cdots, A_3,
%\cdots , A_N ]
%\end{equation}
%where $A_i = s_{1i} \;.\; \cdots$ for each $1 \leq i \leq N$, and
%such that $\varphi$ points to $A_1$, and the $s_{11}$ portion of
%$A_1$ points to $A_2$, and the $s_{12}$ portion of $A_2$ points to
%$A_3$, and so forth, finally the $s_{1n-1}$ portion of $A_{n-1}$
%points to $A_n$. This follows from the induction hypothesis.\\
%Also from the induction hypothesis, we have that each $s_{1i}$
%contains all its level $1$ variables. This guarantees that the
%``search" function terminates (because it would terminate on the
%pointer machine), and has the correct behaviour. In particular,
%the next configuration will be:
%\begin{equation}\label{next}
%s' = [[a (m+), s_{ij}, \cdots ,s_{iN}][\varphi x_1 \cdots x_n (m-),
%s_{11}, \cdots s_{1N}], s_2, \cdots s_k]
%\end{equation}
%where $a$ is a subterm of $s_{ij}$. We now just need to check that
%the invariant is maintained. Note that the previous pointer
%machine configuration was given in Eq. \ref{prev}. Clearly, the
%next step for the pointer machine would be, to uncover what $\varphi$
%with the first application it finds. Of course, this will be $a$.
%Thus we get:
%\begin{equation}
%pm(s) \rightarrow [a . x_1 \cdots x_n, A_1, \cdots, A_2, \cdots,
%A_3, \cdots, A_N]
%\end{equation}
%and the $a$ points to $A_j$. We need to check that this is what we
%get when we apply $pm$ to Eq. \ref{next}. And this is indeed the
%case. Applying $pm$ to any $t \sqsubseteq s'$ will also give the
%correct result for $t \sqsubseteq \pop_2(s')$, but this also holds
%for $t = \pop_1^m(s')$ for $1 \leq m \leq {N-j}$ - easy to check,
%with the help of lemma 4.5 in preceding note. Another interesting
%case involves the top symbol $\varphi x_1 \cdots x_n (m-)$ - this
%follows in a similar fashion. \item Suppose that the topmost item
%is $f t_1 \cdots t_n (m-)$, and that the environment chooses the
%state $q_j$, for some $j > n$. That we eventually reach state
%$q_0$ follows from the induction hypothesis (\ref{rightnum}) and
%(\ref{atleastone}). Suppose that $(q_0, s')$ is the first stable
%configuration we reach - that $pm(s')$ is indeed the next
%configuration for the corresponding pointer machine can be shown
%with an argument similar to the above. The situation if the
%topmost item is $x$, where $x$ is the $j$th argument of the
%preceding item, is similar.
%\end{enumerate}
%\end{proof}
%\end{document}


% Section: Simulation by non-deterministic 2PDAs
\section{Simulating $2$PDALs by non-deterministic $2$PDAs}

We have just seen that the incorporation of labels (as names of
links) into the stack alphabet will, in general, lead to an
infinite alphabet. In this section, we show how these links and
the way in which they are manipulated can be simulated by a
\emph{non-deterministic} $2$PDA.

Let $G$ be a possibly unsafe level-$2$ grammar and let the
corresponding $2$PDAL be denoted by $2PDAL_G$. Furthermore, let
$\Gamma$ be the stack alphabet described in the preceding section.
Our non-deterministic $2$PDA will have stack alphabet $\Gamma \cup
\{w^+ : w \in \Gamma\} \cup \{w^- : w \in \Gamma\} \cup \{w^{+/-}
: w \in \Gamma\}$. The $+$ and $-$ take on the same role as
before, i.e. the $-$ marks the start point of an instance of a
link, whereas the $+$ the end point. Note however that they are
\emph{anonymous} - they are no longer prefixed with natural
numbers.  We will explain how to use them shortly.

The crux of this proof lies in understanding how links are
manipulated in $2PDAL_G$. We dedicate the following
subsection to exactly this.

\subsection{The use of links in $2PDAL_G$}

Below we list some observations on the way links are manipulated.
Recall that labels are preserved with $\push_2$ operations. For
example, consider the following $2$-store of $2PDAL_G$
\begin{equation}
\nonumber \left [ \begin{array}{lll} \hbox{\tt [}a, & \mklink{b}{m-},
& c\hbox{\tt ]}\\ \hbox{\tt [}\mklink{d}{m+}, & e, & f\hbox{\tt ]}
\end{array} \right ]
\end{equation}
If we perform a $\push_2$, we will have
\begin{equation}
\nonumber \left [ \begin{array}{lll}
\hbox{\tt [}a, & \mklink{b}{m-}, & c\hbox{\tt ]}\\
\hbox{\tt [}a, & \mklink{b}{m-}, & c\hbox{\tt ]}\\
\hbox{\tt [}\mklink{d}{m+}, & e, & f\hbox{\tt ]}
\end{array} \right ]
\end{equation}
We now have two instances of the link $m$. One connects the upper
$b$ to $d$, and the other connects the lower $b$ to $d$. By Lemma
5.4(i), if there are two or more instances of a label $m$, then
although each may have a different start point, they all share the
same end point, as illustrated by the above example.

\begin{lemma} \label{lem:useonceonly} In any run of $2PDAL_G$, if
$(q_j,s)$ is a reachable configuration where $top_1(s) = {\$ t_1
\cdots t_n}^{\langle m- \rangle \cup \lambda}$ and $j > 0$, then,
for all $(q_k,s')$ such that $(q_j,s) \rightarrow_{+} (q_k,s')$,
$top_1(s')$ is not labelled by $m-$.
\end{lemma}

\begin{proof} The key to this proof is recognising that $\pop_2$
is never used, only the parameterised form $\pop_2(m)$ is used.
There are two cases. If $j \leq n$, then the result follows from
Lemma~\ref{lem:order}. However, if $j > n$, note that the next configuration after $(q_j,s)$ is $(q_{j-n}, s'')$ where
$\mytop_1(s'') = {\varphi u_1 \cdots u_{n'}}^{\langle m+ \rangle \cup
\lambda'}$. By Lemma 5.4(i), no item in $s''$ can be labelled by
$m-$. Furthermore, any new labels introduced will be fresh.
\end{proof}

We say that a link $m$ is \textbfit{followed} (at a configuration
$(q_j,s)$) if $\mytop_1(s) = {\$t_1 \cdots t_n}^{\langle m-
\rangle \cup \lambda}$ for some $\lambda$ and some $n \geq 0$ such
that $j > n$. A corollary of the above lemma is that any given
link $m$ is followed at most once.

%Note that whenever the 2PDAL is at such a configuration, it
%performs a $\pop_2(m)$ on (an instance of) the link $m$, jumping
%from the start point to the end point.


%\begin{lemma} \label{lem:useonceonly} In any run of a 2PDAL, any given link
%$m$ is followed at most once. Formally, if $(q_j,s)$ is a reachable
%configuration at which a link $m$ is followed, then, for all $(q,s')$
%such that $(q_j,s) \rightarrow^+ (q,s')$, no items labelled with $m-$
%occurs in $s'$; in particular $\mytop_1(s')$ is not labelled by $m-$.
%\end{lemma}
%
%\begin{proof} Suppose $\mytop_1(s) = {\$t_1 \cdots t_n}^{\langle m- \rangle
%\cup \lambda}$ for some $n \geq 0$ and some $\lambda$, and such that
%$j > n$. The next configuration after $(q_j, s)$ is $(q_{j-n}, s'')$
%where $\mytop_1(s) = \$' u_1 \cdots u_{n'}^{\langle m+ \rangle \cup
%\lambda'}$. By Lemma 5.4(i), no item in $s''$ can be labelled by
%$m-$. Furthermore, any new labels introduced will be fresh.
%\end{proof}

%The above lemma is tantamount to saying that a link $m$ will be
%followed at most once, and if it is followed then, in any subsequent
%configurations, no items with label $m-$ exist.

\subsection{Intuition}
The important thing to note about $2PDAL_G$ is that not all links
are followed. In the example illustrated in Section 5.2, no link
apart from $1$ was followed. In fact, we may as well not have
bothered to label the remaining links! This observation is the key
to our construction.

The simulating non-deterministic $2$PDA will follow the rules in
Fig.~\ref{fig:pdal} almost exactly. The difference is that each time
we are about to generate a link (i.e.~Case (4) of
Fig.~\ref{fig:pdal}), we guess whether it will ever be followed in the
future or not, and we label the start and end points of the link if
and only if we guess that it will be followed. Furthermore, instead of
a fresh label $m$, we simply mark the start point with a $-$ and the
end point with a $+$.


\subsubsection*{A controlled form of guessing}

Now this presents a problem of ambiguity. Suppose we find ourselves in
a configuration $(q,s)$ where $\mytop_1(s)$ is labelled by $-$, how
can we tell which of the stack items labelled by a $+$ is the
\emph{true} end point of this link? (True in the sense that if we did
have the ability to name our links as with $2PDAL_G$, the topmost
item would have label $m-$ for some $m$, and the \emph{real} end point
would have label $m+$ for the same $m$.) The answer lies in the use of
a \emph{controlled} form of guessing: when guessing whether a link
will be followed in the future, the machine will not always be allowed
to guess whichever way it pleases; instead we require the guess to be
subject to some constraints. We shall see that as a consequence the
following invariant can be maintained:
\begin{quote}
\emph{Assume that the topmost $1$-store has at least one item
labelled by $-$. For the leftmost (closest to the top) of these, the
corresponding end point can be found in the first $1$-store beneath it
whose topmost item is marked with a $+$.}\footnote{The invariant is
actually stronger than this, but this is sufficient to ensure that the
simulation works correctly.}
\end{quote}
%(For example in the following
%\begin{equation}
%\nonumber \left [ \begin{array}{lll} \hbox{\tt [}{a}^{-}, & b, &
%c\hbox{\tt ]}\\ \hbox{\tt [}d, & e, & f\hbox{\tt ]}\\ \hbox{\tt
%[}g^{+}, & h, &i\hbox{\tt ]}\\ \hbox{\tt [}j^{+}, & k, &l\hbox{\tt ]}
%\end{array} \right ]
%\end{equation}
%if the real end point of $a$ is $j$, we would have to perform 3
%$\pop_2$'s were we to follow the link at $a$.)

Before formalising the controlled form of guessing, we introduce a
definition. Let $(q_0, s)$ be a reachable configuration of $2PDAL_G$
such that
\begin{equation}
\nonumber \mytop_2(s) = [\varphi_{j_1} t_1 \cdots t_n^\lambda, A_1,
\cdots, A_k, \cdots, A_N]
\end{equation}
where $N \geq 2$. We say that $\varphi_{j_1}$ \textbfit{ultimately
refers to} $A_k$ just if:
\begin{itemize}
\item[(i)] For $i = 1, \cdots, k-1$, the $j_i$th level-$1$ argument of
$A_i$ is a variable $\varphi_{j_{i+1}}$. We remind the reader of the
notational convention for level-$1$ parameters set out in (\ref{convention}).
\item[(ii)] The $j_{k}$th argument of $A_k$ is an application or a
non-terminal.
\end{itemize}
So, for example, using the grammar in Example~\ref{ex:ex1}, the
following is a reachable $2$-store for $2PDAL_G$ on input $h_1h_3h_3$:
\begin{equation}
\nonumber \begin{array}{rl}
(q_0, & \hbox{\tt [}\mkstore{\varphi B,\mklink{(D \varphi x)}{1-},
DGAB, S},\\
& \mkstore{\mklink{\varphi B}{1+}, D (D \varphi x) y (\varphi y),
DGAB,S}\hbox{\tt ]})
\end{array}
\end{equation}
Here the topmost $\varphi$ ultimately refers to (the $G$ in) $DGAB$ (in
the topmost $1$-store).

Suppose that we are in a configuration $(q_0, s)$ of the
non-deterministic $2$PDA where
\begin{equation}
\nonumber \mytop_2(s) = \mkstore{\varphi t_1 \cdots t_n^{?}, A_1,
\cdots, A_j, \cdots, A_n}
\end{equation}
where $?$ may either denote a $-$ or no label at all. Furthermore,
suppose that $\varphi$ ultimately refers to $A_j$. There are two possibilities:
\begin{enumerate}
\item[A.] None of the stack items $\varphi t_1 \cdots t_n^{?}, A_1,
\cdots, A_j$ are labelled by a $-$; or \item[B.] There exists a stack item
in  $\varphi t_1 \cdots t_n^{?}, A_1, \cdots, A_j$ labelled by a
$-$.
\end{enumerate}
In the first case we leave it up to the $2$PDA to
non-deterministically label $\varphi t_1 \cdots t_n$ (with $+$) in the
configuration $(q_0, s)$ as well as its matching partner (with $-$). However,
in the second case, we \emph{insist} that the $2$PDA label
$\varphi t_1 \cdots t_n$ in the configuration $(q_0, s)$ as well
as its matching partner, as given in Case (4) of Fig.~\ref{fig:pdal}.

Let us illustrate why this maintains the above invariant with an
example. Suppose we have the following configuration:
\begin{equation}
\nonumber \left [\begin{array}{l}
\mkstore{\varphi x_1 x_2, {D \varphi x}^{-}, F(F\varphi x)y, {G\varphi
x}^-, E, S}\\
\mkstore{A^+, \cdots}\\
\mkstore{B^+, \cdots}
\end{array} \right ]
\end{equation}
Note that the topmost stack has two items labelled with a $-$, $D
\varphi x$ and $G \varphi x$. By our invariant we know that $D \varphi
x$ has end point $A^+$. And let us suppose that $G \varphi x$ points
to $B^+$. Suppose that the $\varphi$ of the topmost item ultimately
refers to (the subterm $F \varphi x$) in $F(F \varphi x)
y$. Furthermore, suppose we go against our controlled form of guessing
and allow the machine \emph{not} to label $\varphi x_1 x_2$ and its
matching partner. Thus we arrive at
\begin{equation}
\nonumber \left [ \begin{array}{l}
\mkstore{ \varphi, F(F\varphi x)y, {G\varphi x}^-, E, S}\\
\mkstore{\varphi x_1 x_2, {D \varphi x}^{-}, F(F\varphi x)y, {G\varphi
x}^-, E, S}\\
\mkstore{A^+, \cdots}\\
\mkstore{B^+, \cdots}
\end{array} \right ]
\end{equation}
Note that now $G \varphi x$ is the leftmost item labelled with a
$-$. Our invariant has been violated as the real end point of $G
\varphi x$ is not $A^+$.

%In the next section we will prove rigorously that obeying the above
%convention will ensure that the above invariant is mained.

\subsubsection*{Penalty for guessing wrongly}

The cost of using non-determinism (regardless of how controlled it may
be) is that we commit ourselves to following our guesses. When we find
out that we have guessed wrongly, we shall have to abort the
run. There are two cases. Suppose we find ourselves in a configuration
$(q_j,s)$ where $\mytop_1(s) = {\$ x_1 \cdots x_n}^{-}$ and $j \leq
n$. The fact that the topmost item is labelled by $-$ means that at
some point in the past, we guessed that we would follow this link.
%Recall that the non-deterministic $2$PDA only has the ability to label
%items with a $+, -$ or $+/-$ and is not able to add a name to
%links. Hence, when an item is labelled with a $-$, in the
%corresponding $2$PDAL it will actually be labelled with a $m-$, and it
%will have a corresponding end point $m+$ that presents itself as a
%mere $+$ in the non-deterministic $2$PDA.
But, by Lemma~\ref{lem:useonceonly} we have not yet followed the
link nor will we ever follow it in the future.
%But, as $j \leq n$, we have so far not followed the link $m$, nor (by
%Lemma~\ref{lem:useonceonly}) will we ever follow it in future.
The machine has guessed wrongly, and we abort immediately.
Symmetrically if we find ourselves in a configuration $(q_j,s)$
where $\mytop_1(s) = \$ x_1 \cdots x_n$ and $j > n$, then we also
abort. Why? The absence of a $-$ label means that at some point in
the past we guessed that we would \emph{not} follow this link, but
we are now about to turn against our original guess.

%\subsubsection*{Why it will work}
%
%We will show that for each word in the language of the $2$PDAL, there
%is at least one way to guess correctly which stack items will be
%labelled (with $+,-$ or $+/-$) such that the word is
%accepted. Furthermore, these guesses will be consistent with our
%controlled guessing. Conversely, we will show that every set of
%guesses consistent with our controlled guessing will lead to the
%acceptance of a word that belongs to the language of the $2$PDAL.

\subsection{Definition of the non-deterministic $2$PDA, $2PDA_G$}
Let $G$ be a (possibly unsafe) 2-grammar that generates a string
language. The transition rules of the non-deterministic $2$PDA,
$2PDA_G$, are given in Fig.~\ref{fig:2pda}.
%\begin{figure*}[t]
%\begin{center}
%\makebox{
%\begin{shadowbox}[17cm]
%\begin{math}
%\begin{array}{rclr}
%\nonumber (q_0, a,  Dt_1 \cdots t_n^\lambda) & \rightarrow & (q_0, \push_1(E)) \mbox{ if $Dx_1 \cdots x_n \larr{a} E$ } & \rm{(1)}\\
%\nonumber (q_0, \epsilon ,e) & \rightarrow & \mbox{accept} & \rm{(2)}\\
%\nonumber (q_0, \epsilon ,x_j)& \rightarrow & (q_j, \pop_1) & \rm{(3)}\\
%\nonumber \\
%\nonumber (q_0, \epsilon, \varphi_j t_1 \cdots t_n) & \rightarrow
%& \left \{ \begin{array}{l}
%(q_0, \repl_1({\varphi_j t_1 \cdots t_n}^{+}) \compose \push_2 \compose \pop_1 \compose \repl_1({s_j}^{-}))\\
%(q_0, \push_2 \compose \pop_1 \compose \repl_1(s_j))
%\end{array} \right . & \begin{array}{r} \rm{(4.1)} \\ \rm{(4.1')}\end{array}\\
%\nonumber & & \mbox{where Situation A holds and ${Ds_1 \cdots
%s_{n'}}^{\lambda'}$ precedes
%$\varphi_j t_1 \cdots t_n$} & \\
%\nonumber (q_0, \epsilon, {\varphi_j t_1 \cdots t_n}^{\lambda }) &
%\rightarrow & (q_0, \repl_1({\varphi_j t_1 \cdots t_n}^{+ \cup
%\lambda})
%\compose \push_2 \compose \pop_1 \compose \repl_1({s_j}^{-})) & \rm{(4.2)}\\
%\nonumber & & \mbox{where Situation B holds and ${Ds_1 \cdots
%s_{n'}}^{\lambda'}$ precedes
%${\varphi_j t_1 \cdots t_n}^\lambda$ } & \\
%\nonumber \\
%\nonumber 1 \leq j \leq n, (q_j, \epsilon, \$t_1 \cdots t_n^\lambda) &
%\rightarrow & \left \{ \begin{array}{ll}
%(q_0, \repl_1(t_j)) &  - \not \in \lambda \\
%\mbox{abort} &  - \in \lambda \end{array} \right . & \begin{array}{r} \rm{(5.1)} \\ \rm{(5.2)}\end{array} \\
%\nonumber j > n , (q_j, \epsilon, \$t_1 \cdots t_n^\lambda) &
%\rightarrow & \left \{ \begin{array}{ll}
%\mbox{abort} & - \not \in \lambda\\
%(q_{j-n}, \pop^+_2) & - \in \lambda \end{array} \right . & \begin{array}{r} \rm{(6.1)} \\ \rm{(6.2)}\end{array}
%\end{array} \end{math}
%\end{shadowbox}}
%\end{center}
%\caption{Transition rules of the non-deterministic $2$PDA,
%$2PDA_G$ \label{fig:2pda}}
%\end{figure*}
\begin{figure*}[t]
\begin{center}
\makebox{
\begin{shadowbox}[16.5cm]
\begin{eqnarray}
\nonumber (q_0, a,  Dt_1 \cdots t_n^\lambda) & \rightarrow & (q_0, \push_1(E)) \mbox{ if $Dx_1 \cdots x_n \larr{a} E$ }\\
\nonumber (q_0, \epsilon ,e) & \rightarrow & \mbox{accept}\\
\nonumber (q_0, \epsilon ,x_j)& \rightarrow & (q_j, \pop_1)\\
\nonumber \\
\nonumber (q_0, \epsilon, \varphi_j t_1 \cdots t_n) & \rightarrow
& \left \{ \begin{array}{l}
(q_0, \repl_1({\varphi_j t_1 \cdots t_n}^{+}) \compose \push_2 \compose \pop_1 \compose \repl_1({s_j}^{-}))\\
(q_0, \push_2 \compose \pop_1 \compose \repl_1(s_j))
\end{array} \right .\\
\nonumber & & \mbox{where Situation A holds and ${Ds_1 \cdots s_{n'}}^{\lambda'}$ precedes
$\varphi_j t_1 \cdots t_n$}\\
\nonumber (q_0, \epsilon, {\varphi_j t_1 \cdots t_n}^{\lambda }) &
\rightarrow & (q_0, \repl_1({\varphi_j t_1 \cdots t_n}^{+ \cup
\lambda})
\compose \push_2 \compose \pop_1 \compose \repl_1({s_j}^{-}))\\
\nonumber & & \mbox{where Situation B holds and ${Ds_1 \cdots s_{n'}}^{\lambda'}$ precedes
${\varphi_j t_1 \cdots t_n}^\lambda$ }\\
\nonumber \\
\nonumber 1 \leq j \leq n, (q_j, \epsilon, \$t_1 \cdots t_n^\lambda) &
\rightarrow & \left \{ \begin{array}{ll}
(q_0, \repl_1(t_j)) &  \mbox{if } - \not \in \lambda \\
\mbox{abort} & \mbox{if } - \in \lambda \end{array} \right . \\
\nonumber j > n, (q_j, \epsilon, \$t_1 \cdots t_n^\lambda) & \rightarrow &
\left \{ \begin{array}{ll}
\mbox{abort} & \mbox{if } - \not \in \lambda \\
(q_{j-n}, \pop^+_2) & \mbox{if } - \in \lambda
\end{array} \right .
\end{eqnarray}
\end{shadowbox}}
\end{center}
\caption{Transition rules of the non-deterministic $2$PDA, $2PDA_G$
\label{fig:2pda}}
\end{figure*}
Note that, again, we assume that production rules of the grammar
assume the format given in rule (\ref{convention}). The function
$\pop_2^{+}$ performs a $\pop_2$ and then repeats until the topmost
symbol it reads is marked with a $+$. Formally, let $s$ range over 2-stores, we define $\pop_2^+ (s) =
p(\pop_2(s))$ where
\begin{eqnarray}
\nonumber p (s)  & = &  \left \{
\begin{array}{ll}
s & \hbox{if $\mytop_1(s)$ has label $+$}\\
{p(\pop_2 (s))} & \mbox{otherwise}
\end{array} \right .
\end{eqnarray}
Observe that the rules of the simulating non-deterministic $2$PDA
(Fig.~\ref{fig:2pda}) are almost identical to those of the $2$PDAL
(Fig.~\ref{fig:pdal}).

\begin{remark}
In the definition of the transition rules (Fig.~\ref{fig:2pda}), in
case the $\mytop_1$ item of the 2-store is headed by a level-1
variable, the $2$PDA has to work out whether it is situation A or
B. This can be achieved by a little scratch work on the side: do a
$\push_2$, inspect the topmost 1-store for as deep as necessary,
followed by a $\pop_2$. Alternatively we could ask the oracle to tell
us whether it is A or B, taking care to ensure that a wrong
pronouncement will lead to an abort.
\end{remark}

\subsection{Proof of correctness}
First some notations. Items labelled with superscripts $-, +$ or
$+/-$, will be referred to as \textbfit{marked} items. In
particular, if an item is marked with a $-$ or a $+/-$,
then we say it is \textbfit{$-$marked}, indicating that one of its
labels is indeed a $-$. Similarly, if an item is marked with a $+$
or a $+/-$, then we say it is \textbfit{$+$marked}. If the topmost
1-store of $s$ has at least one $-$marked item, then we say that
the leftmost of these (closest to the top) items is the
\textbfit{foremost}. (If the topmost 1-store of $s$ has no
$-$marked items, then $s$ has no foremost item.) Let $\theta$ be a
$-$marked item in a $2$-store $s$. We say that $\theta$ is
$-$\textbfit{reachable} in $s$ just if $\theta$ is foremost in
$s$, or if $s$ has a foremost item and $\theta$ is $-$reachable in
$\pop_2^+(s)$. For example consider the following $2$-store:
%$\pop_2^+ (s)$ is defined and has a foremost item and $\theta$ is
%$-$reachable in $\pop_2^+ (s)$. For example consider the following
%2-store:
\begin{equation}
\nonumber s \; = \; \left [ \begin{array}{lllll}
\hbox{\tt [} a      , & b^- , & c   , & d^- , & e \hbox{\tt ]}\\
\hbox{\tt [} f      , & g   , & h   , & i   , & j \hbox{\tt ]}\\
\hbox{\tt [} k^{+/-}, & l   , & m   , & n   , & o \hbox{\tt ]}\\
\hbox{\tt [} p^+    , & q   , & r^- , & s   , & t \hbox{\tt ]}\\
\hbox{\tt [} u^+    , & v   , & w   , & x   , & y \hbox{\tt ]}
\end{array} \right ]
\end{equation}
The foremost item is $b$. The items $b,k$ and $r$ are the only
$-$reachable ones. Note that if  we perform a \emph{single}
$\pop_2$, then there will be no foremost item. However, with
another $\pop_2$, the foremost item will be $k$.

Finally, we fix a possibly unsafe 2-grammar $G$ that generates a
string language. Let $(q, s)$ and $(q', s')$ be reachable
configurations of the $2PDA_G$ and $2PDAL_G$ respectively.
Intuitively we write $s \bohm s'$ to mean that $s$ simulates $s'$.
Precisely $s \bohm s'$ holds if either both $s$ and $s'$ are the
empty 2-store, or the following hold
\begin{itemize}
\item[(i)] If $\mytop_1(s) = \mytop_1(s') = \bot$, then $pop_2(s)
\bohm s'$, and
\item[(ii)] If $\mytop_1(s) = {a}^{\lambda}$ and $\mytop_1(s') =
{a'}^{\lambda'}$ then 
\begin{itemize}
\item[a.] $a = a'$ and
\item[b.] If $\lambda$ has label $-$ then $\lambda'$ has label $m-$
for some $m$. Similarly, if $\lambda$ has label $+$ then $\lambda'$
has label $n+$ for some $n$; and

%\[
%\begin{array}{rcl}
%\nonumber \mbox{$\lambda$ has label $-$} & \Rightarrow & \mbox{$\lambda'$ has label $m-$ for some $m$}\\
%\nonumber \mbox{$\lambda$ has label $+$} & \Rightarrow &
%\mbox{$\lambda'$ has label $m+$ for some $m$}
%\end{array}
%\], and

\item[c.] $pop_1(s) \bohm pop_1(s')$
\end{itemize}
\end{itemize}

\begin{lemma}\label{lem:consistency}
Let $(q_0, \mkstore{\mkstore{S}}) = (p_0, s_0) \rightarrow \cdots
\rightarrow (p_n, s_n)$ be a transition sequence of $2PDA_G$. Then
there exists a unique (modulo renaming of labels) transition
sequence $(p_0, s_0') \rightarrow \cdots \rightarrow (p_n, s_n')$
of $2PDAL_G$ such that for all $i = 0, \cdots, n$:
\begin{itemize}
\item[(i)] $s_i \bohm s_i'$
\item[(ii)] If $s \sqsubseteq s_i$ such that $\mytop_1(s)$ is
$-$reachable in $s_i$, and if $s' \sqsubseteq s_i'$ such that $s \bohm
s'$, then $\pop^+_2(s) \bohm \pop_2(m)(s')$ where $\mytop_1(s')$ has
label $m-$.
\end{itemize}
\end{lemma}

\begin{proof}
Induction on the number of transitions.
\end{proof}

\begin{proposition} \label{prop:prop1}
If a string is accepted by $2PDA_G$, it is also accepted by $2PDAL_G$.
\end{proposition}

\begin{proof}
This follows from Lemma~\ref{lem:consistency} and the correctness of
the $2$PDAL.
\end{proof}

\begin{proposition} \label{prop:prop2}
If a string is accepted by $2PDAL_G$, it is also accepted by $2PDA_G$.
\end{proposition}

\begin{proof}
Suppose there is an all-knowing oracle that always tells us
\emph{correctly} whether or not we will ever follow a link. We need to
check that the controlled guessing does not restrict the choices of
the oracle.  Recall that when Situation B arises we do not allow our
$2$PDA a choice: we \emph{always} guess that the link we are about to
create will be followed in the future. We must check that the oracle
agrees with this decision.

Thus suppose that the current configuration is $(q_0, s)$ and
$\mytop_2(s)$ is the following:
\begin{equation}
\nonumber \mkstore{A_1^{\lambda_1}, A_2^{\lambda_2}, \cdots, A_k^-,
\cdots, A_{n-1}^{\lambda^{n-1}}, A_n^{\lambda_n}, \cdots}
\end{equation}
where $A_1 = \varphi_j t_1 \cdots t_m$, and $1\leq k < n$.
Furthermore, suppose that $\varphi_j$ in $A_1$ ultimately refers to
$A_n$. According to our rules, after a finite number of transition
steps we reach the following configuration:
\begin{eqnarray}
\nonumber && \mkstore{t_{A_n}^{-}, \cdots } \\
\nonumber && \mkstore{t_{A_n-1}^{+/-}, A_n^{\lambda_n} , \cdots}  \\
\nonumber && \vdots \\
\nonumber && \mkstore{t_{A_k}^{+/-}, A_{k+1}^{\lambda_{k+1}}, \cdots, A_{n-1}^{\lambda_{n-1}}, A_n^{\lambda_n}, \cdots}\\
\nonumber && \vdots \\
\nonumber && \mkstore{t_{A_2}^{+/-},  A_3^{\lambda_3}, \cdots,  A_k^-, \cdots, A_{n-1}^{\lambda^{n-1}}, A_n^{\lambda_n},\cdots}\\
\nonumber && \mkstore{\varphi_j t_1 \cdots t_n^{+ \cup \lambda}, A_2^{\lambda_2}, \cdots, A_k^-, \cdots, A_{n-1}^{\lambda^{n-1}},A_n^{\lambda_n},\cdots}\\
\nonumber && \vdots
\end{eqnarray}
where $t_{A_i}$ is a subterm of $A_i$. We must check that the above
``forced'' choices would tally with those of the oracle. Suppose, for
a contradiction, that it does not, and that $t_{A_i}$ for some $i$ is
not $-$marked. There are two possibilities:
\begin{enumerate}
\item[(1)] $t_{A_n}$ is not $-$marked.
\item[(2)] $t_{A_n}$ is $-$marked,
but there exists an $i$ such that $t_{A_i}$ is not $-$marked for
some $i < n$.
\end{enumerate}
For case (1), the proof of Lemma~\ref{lem:useonceonly} says that
the link represented by $A_k^-$ has not yet been followed.
However, it should not be difficult to see that by not marking
$t_{A_n}$ we have eliminated the possibility of ever returning to
$A_k^-$. In particular, this contradicts that the oracle's
previous information that we would perform a $\pop_2^+$ at $A_k^-$
(or a duplicate) in the future. For case (2), let $I = \max \{i :
\mbox{$A_i$ is not $-$marked}\}$. Then, according to the oracle,
we will perform a $\pop_2^+$ at $t_{A_{I+1}}$, however, the next
configuration will be $(q_k, s')$ for $k>0$ and $\mytop_1(s') =
t_{A_I}$. This is untenable, as by assumption $t_{A_I} = \varphi$
for some level-$1$ variable $\varphi$, thus we must follow a link
here unconditionally or we will be stuck. Again, this contradicts
the oracle's information that we would not need to follow a link
at $t_{A_I}$.
\end{proof}

Combining Propositions~\ref{prop:prop1} and \ref{prop:prop2} we can
conclude:
\begin{theorem}
There is an effective transformation of any (possibly unsafe)
string-language generating 2-grammar to a non-deterministic $2$PDA that
accepts the same language. \myendproof
\end{theorem}

%\begin{remark} In Proposition \re{prop:prop2} it was shown that if
%$w$ is accepted by a $2$PDAL then it is accepted by the
%corresponding non-deterministic $2$PDA. In particular, it showed
%that there is at least one way to guess which links are followed
%in such a way that $w$ is accepted and such that the guesses are
%consistent with our controlled guessing. However, it should also
%be noted that there may be \emph{more} than one such way. Thus,
%perhaps a truer reflection of what it means for a stack item to be
%$-$marked is the following. If a stack item is $-$marked then,
%should we ever visit this item in state $q_j$ for $j>0$ we will
%always perform a $\pop_2^+$. Note that this says nothing about the
%case where we \emph{never} visit this item in a state $q_j$ for
%$j>0$.
%\end{remark}


\subsection{Some examples}

\begin{example}
We demonstrate the non-deterministic $2$PDA for the grammar in
Example~\ref{ex:ex1} on input $h_1h_3h_2f_1a$. Recall that this word
does not belong in the language. As before, the automaton starts in the configuration $(q_0, h_1h_3h_2f_1a,
\mkstore{\mkstore{S}})$. After a few steps the configuration is:
\begin{eqnarray}
\nonumber (q_0, & h_2f_1a, & \mkstore{\mkstore{\varphi B, D(D
\varphi x) y (\varphi y), DGAB,S}})
\end{eqnarray}
It is now up to the automaton to choose whether or not to label the
start and end points of this link. Suppose it chooses to label:
\begin{eqnarray}
\nonumber (q_0, & h_2f_1a, & \hbox{\tt [}\mkstore{{(D \varphi x)}^{-}, DGAB, S},\\
\nonumber & & \mkstore{{\varphi B}^{+}, D (D \varphi x) y (\varphi
y), DGAB,S}\hbox{\tt ]})
\end{eqnarray}
After a few more steps the automaton arrives at another
configuration where the topmost symbol is headed by a level-$1$
variable.
\begin{eqnarray}
\nonumber (q_0, & f_1a, & \hbox{\tt [}\mkstore{\varphi x,H (F y) x,{(D
\varphi x)}^{-}, DGAB, S}, \\
\nonumber & &\mkstore{{\varphi B}^{+}, D (D \varphi x) y (\varphi
y), DGAB,S}\hbox{\tt ]})
\end{eqnarray}
This time suppose it chooses not to label:
\begin{eqnarray}
\nonumber \rightarrow_+ (q_0, & a, & \hbox{\tt [}\mkstore{x,(F y),{(D \varphi x)}^{-}, DGAB, S}\\
\nonumber & &\mkstore{\varphi x,H (F y) x,{(D \varphi x)}^{-}, DGAB, S}, \\
\nonumber & &\mkstore{{\varphi B}^{+}, D (D \varphi x) y (\varphi
y), DGAB,S}\hbox{\tt ]})\\
\nonumber \rightarrow_+ (q_0, & a, & \hbox{\tt [}\mkstore{y,{(D \varphi x)}^{-}, DGAB, S}\\
\nonumber & &\mkstore{\varphi x,H (F y) x,{(D \varphi x)}^{-}, DGAB, S}, \\
\nonumber & &\mkstore{{\varphi B}^{+}, D (D \varphi x) y (\varphi
y), DGAB,S}\hbox{\tt ]})\\
\nonumber \rightarrow_+ (q_3, & a, & \hbox{\tt [}\mkstore{{(D \varphi x)}^{-}, DGAB, S}\\
\nonumber & &\mkstore{\varphi x,H (F y) x,{(D \varphi x)}^{-}, DGAB, S}, \\
\nonumber & &\mkstore{{\varphi B}^{+}, D (D \varphi x) y (\varphi
y), DGAB,S}\hbox{\tt ]})\\
\nonumber \rightarrow_+ (q_1, & a, & \hbox{\tt [}\mkstore{{\varphi
B}^{+}, D (D \varphi x) y (\varphi
y), DGAB,S}\hbox{\tt ]})\\
\nonumber \rightarrow_+ (q_0, & a, & \hbox{\tt [}\mkstore{B, D (D
\varphi x) y (\varphi y), DGAB,S}\hbox{\tt ]})
\end{eqnarray}
and, as the word $h_1h_3h_2f_1a$ is rejected, as required.
\end{example}

\begin{example}As another example using the same grammar, let us
attempt a computation on the same word $h_1h_3h_2f_1a$ but this time
using a different set of guesses.
\begin{eqnarray}
\nonumber (q_0, & h_1h_3h_2f_1a, & \mkstore{\mkstore{S}})\\
\nonumber \rightarrow_+ (q_0, & h_2f_1a, & \hbox{\tt [}\mkstore{{(D \varphi x)}, DGAB, S},\\
\nonumber & & \mkstore{{\varphi B}, D (D \varphi x) y (\varphi
y), DGAB,S}\hbox{\tt ]})\\
\nonumber \rightarrow_+ (q_0, & f_1a, & \hbox{\tt [}\mkstore{\varphi x,H (F y) x,{(D
\varphi x)}, DGAB, S}, \\
\nonumber & &\mkstore{{\varphi B}, D (D \varphi x) y (\varphi
y), DGAB,S}\hbox{\tt ]})\\
\nonumber \rightarrow_+ (q_0, & f_1a, & \hbox{\tt [}\mkstore{{(F y)}^{-},{(D\varphi x)}, DGAB, S}\\
\nonumber & & \mkstore{{\varphi x}^{+},H (F y) x,{(D\varphi x)}, DGAB, S}, \\
\nonumber & & \mkstore{{\varphi B}, D (D \varphi x) y (\varphi y),
DGAB,S}\hbox{\tt ]})\\
\nonumber \rightarrow_+ (q_0, & a, & \hbox{\tt [}\mkstore{x,{(F y)}^{-},{(D\varphi x)}, DGAB, S}\\
\nonumber & & \mkstore{{\varphi x}^{+},H (F y) x,{(D\varphi x)}, DGAB, S}, \\
\nonumber & & \mkstore{{\varphi B}, D (D \varphi x) y (\varphi y),
DGAB,S}\hbox{\tt ]})\\
\nonumber \rightarrow_+ (q_1, & a, & \hbox{\tt [}\mkstore{{(F y)}^{-},{(D\varphi x)}, DGAB, S}\\
\nonumber & & \mkstore{{\varphi x}^{+},H (F y) x,{(D\varphi x)}, DGAB, S}, \\
\nonumber & & \mkstore{{\varphi B}, D (D \varphi x) y (\varphi y),
DGAB,S}\hbox{\tt ]})
\end{eqnarray}
But now the automaton aborts prematurely. The fact the the topmost
symbol is labelled with a $-$ indicates that earlier the automaton
guessed it would follow the link here. However, it is now in state
$q_1$ and the first argument \emph{is} present, hence, the
automaton guessed incorrectly. There are two other combinations of
guesses that can consume the prefix $h_1h_3h_2f_1$; we leave it to
the reader to check that neither results in acceptance of
$h_1h_3h_2f_1a$.
\end{example}

\begin{example}
Finally, as a last example, we use the same grammar, but this time our
input word is $h_1h_3h_3b$. It can easily be checked that this is
indeed a word in the language.
\begin{eqnarray}
\nonumber (q_0, & h_1h_3h_3b, & \mkstore{\mkstore{S}})\\
\nonumber \rightarrow_+ (q_0, & h_3b, & \mkstore{\mkstore{\varphi B, D(D \varphi x) y (\varphi y), DGAB,S}})\\
\nonumber \rightarrow_+ (q_0, & h_3b, &  \hbox{\tt [}\mkstore{ {(D \varphi x)}^{-}, DGAB,S}\\
\nonumber & & \mkstore{{\varphi B}^{+}, D(D \varphi x) y (\varphi
y), DGAB,S}\hbox{\tt ]})\\
\nonumber \rightarrow_+ (q_0, & b, &  \hbox{\tt [}\mkstore{ \varphi B,{(D \varphi x)}^{-}, DGAB,S}\\
\nonumber & & \mkstore{{\varphi B}^{+}, D(D \varphi x) y (\varphi
y), DGAB,S} \hbox{\tt ]})
\end{eqnarray}
At this point, situation B occurs: $\varphi$ in the topmost item
ultimately refers to ($G$ in) $DGAB$, therefore, our controlled form of
guessing insists that the automaton label the link:
\begin{eqnarray}
\nonumber \rightarrow_+ (q_0, &b, &  \hbox{\tt [}\mkstore{ \varphi^{-}, DGAB,S}\\
\nonumber & & \mkstore{{\varphi B}^{+}, {(D \varphi x)}, DGAB,S}\\
\nonumber & & \mkstore{{\varphi B}^{+}, D(D \varphi x) y (\varphi
y), DGAB,S}
\hbox{\tt ]})\\
\nonumber \rightarrow_+ (q_0, &b, &  \hbox{\tt[}\mkstore{ G^{-},S}\\
\nonumber & & \mkstore{ \varphi^{+/-}, DGAB,S}\\
\nonumber & & \mkstore{{\varphi B}^{+},{(D \varphi x)}^{-}, DGAB,S}\\
\nonumber & & \mkstore{{\varphi B}^{+}, D(D \varphi x) y (\varphi y), DGAB,S} \hbox{\tt ]})\\
\nonumber \rightarrow_+ (q_0, &b, &  \hbox{\tt[}\mkstore{x,G^{-},S}\\
\nonumber & & \mkstore{ \varphi^{+/-}, DGAB,S}\\
\nonumber & & \mkstore{{\varphi B}^{+},{(D \varphi x)}^{-}, DGAB,S}\\
\nonumber & & \mkstore{{\varphi B}^{+}, D(D \varphi x) y (\varphi
y), DGAB,S}
\hbox{\tt ]})\\
\nonumber \rightarrow_+ (q_1, &b, &  \hbox{\tt[}\mkstore{G^{-},S}\\
\nonumber & & \mkstore{ \varphi^{+/-}, DGAB,S}\\
\nonumber & & \mkstore{{\varphi B}^{+},{(D \varphi x)}^{-}, DGAB,S}\\
\nonumber & & \mkstore{{\varphi B}^{+}, D(D \varphi x) y (\varphi
y), DGAB,S}
\hbox{\tt ]})\\
\nonumber \rightarrow_+ (q_1, &b, &  \hbox{\tt[}\mkstore{ \varphi^{+/-}, DGAB,S}\\
\nonumber & & \mkstore{{\varphi B}^{+},{(D \varphi x)}^{-}, DGAB,S}\\
\nonumber & & \mkstore{{\varphi B}^{+}, D(D \varphi x) y (\varphi
y), DGAB,S}
\hbox{\tt ]})\\
\nonumber \rightarrow_+ (q_1, &b, &  \hbox{\tt[}\mkstore{{\varphi B}^{+},{(D \varphi x)}^{-}, DGAB,S}\\
\nonumber & & \mkstore{{\varphi B}^{+}, D(D \varphi x) y (\varphi
y), DGAB,S}
\hbox{\tt ]})\\
\nonumber \rightarrow_+ (q_1, &b, &  \hbox{\tt[}\mkstore{B,{(D \varphi x)}^{-}, DGAB,S}\\
\nonumber & & \mkstore{{\varphi B}^{+}, D(D \varphi x) y (\varphi
y), DGAB,S}
\hbox{\tt ]})\\
\nonumber \rightarrow_+ (q_1, &\epsilon, &  \hbox{\tt[}\mkstore{\epsilon,B,{(D \varphi x)}^{-}, DGAB,S}\\
\nonumber & & \mkstore{{\varphi B}^{+}, D(D \varphi x) y (\varphi
y), DGAB,S} \hbox{\tt ]})
\end{eqnarray}
And the word is accepted as required.
\end{example}


% Section: Unrestricted term trees and Caucal's graph
\section{Unrestricted term trees}
%\section{Unrestricted term trees and Caucal graphs}
As mentioned earlier, grammars can be used for generating term
trees as well as string languages. We now give the definition of a
term-tree generating grammar as introduced by \cite{KNU01, KNU02}.
We then examine the relationship between string languages and term
trees. We conclude with an application of our level-$2$
result.

\subsection{Definitions}

We shall use higher-order grammars as generators of possibly
infinite terms, viewed as trees. As in Section 2.1 we consider
simple types. Fix a typed alphabet $\Sigma$ of symbols of type
level at most $1$ (this is often referred to as a
\textbfit{signature}). Thus we can think of elements of $\Sigma$
as function symbols $f : \underbrace{o \funsp \cdots \funsp
o}_{r\geq 0} \funsp o$ of arity $\arity{f} = r$.

A \textbfit{$\Sigma$-tree} is a map $t : T \mor \Sigma$ where $T$
is a prefix-closed subset of $\makeset{1, 2, 3, \cdots}^\ast$, and
for $k \geq 0$ whenever $t(w)$ has arity $k$ then $w$ has exactly
$k$-successors in $T$, which are $w1, \cdots, wk$. We write
$\terms{\infty}{\Sigma}$ for the set of $\Sigma$-trees. Thus a
$\Sigma$-tree is just a possibly-infinite ``applicative term''
constructed using symbols from $\Sigma$. Henceforth we shall
identify \emph{finite} $\Sigma$-trees with elements of
$\terms{o}{\Sigma}$, and use them interchangeably.

\begin{definition}\rm
A (term-tree generating) \textbfit{higher-order grammar} is a 5-tuple $G =
\anglebra{N, V, \Sigma, {\cal R}, S}$ such that $N$ is a finite
set of \emph{homogeneously} typed non-terminals, with start symbol
${S : o} \in N$; $V$ is a set of typed variables; $\Sigma$ is as
above; and ${\cal R}$ is a finite set of rules of the form:
\[ F x_1 \cdots x_m \; \rightarrow \; E \]
where $F : (A_1, \cdots, A_m, o) \in N$, each ${x_i : A_i} \in V$,
and $E \in T^o(N \cup \Sigma \cup \{x_1, \cdots, x_m\})$. As with
string-language generating grammars, the notions of the level of a
rule and the level of a grammar apply. Furthermore, we say the
grammar is \textbfit{safe} if and only if the righthand side of
each rule is a safe term-in-context. Also, as we did for the
string-language setting we make a proviso that if $F \in N$ has
type $(A_1, \cdots, A_m, o)$ and $m \geq 1$, then $A_m = o$.
Following the setting suggested \cite{KNU02} we also make the
further assumption that for each non-terminal $F \in N$ there
exists \emph{exactly one} production rule with $F$ on the left
hand side.
\end{definition}

\subsubsection*{Infinite term tree or $\Sigma^\bot$-tree generated by a
grammar}

We write $\Sigma^\bot$ to mean $\Sigma \cup
\makeset{\bot}$ for a distinguished symbol $\bot : o$.  For any $t \in
\terms{A}{N \cup
\Sigma}$, we define $t^\bot$, a term in $\terms{A}{\Sigma^\bot}$,
by induction over the following rules:
\begin{itemize}
\item $f^\bot = f$ for $f \in \Sigma$
\item $F^\bot = \bot$ for $F \in N$
\item $(st)^\bot = (s^\bot t^\bot)$ if $s^\bot \not = \bot$, otherwise
$(st)^\bot = \bot$
\end{itemize}
We define ${\rightarrow_G}$, a binary relation over $\terms{}{N \cup
\Sigma}$, by recursion over the structure of $t$:
\begin{itemize}
\item $Ft_1 \cdots t_n \rightarrow_G r[t_1/x_1, \cdots, t_n/x_n]$ if
there is a rule $Fx_1 \cdots x_n \rightarrow r$, with $x_i:A_i$ and
$t_i \in \terms{A_i}{N \cup \Sigma}$ for $i = 1, \cdots, n$. \item If
$t
\rightarrow_G t'$ then $(st) \rightarrow_G (st')$ and $(tr)
\rightarrow_G (t'r)$, for applicative terms $st$ and $tr$ of the
appropriate types.
\end{itemize}

For $t \in \terms{o}{N \cup \Sigma}$ and $s \in
\terms{\infty}{\Sigma^\bot}$, we say that $t \downarrow_G s$ if:
\begin{itemize}
\item $s$ is finite and there is a finite reduction sequence $t = t_0
\rightarrow_G \cdots \rightarrow_G t_n = s$

\item $s$ is infinite and there is an infinite reduction sequence $t =
t_0 \rightarrow_G t_1 \rightarrow_G \cdots$ such that $s = \lim
t^\bot_n$
\end{itemize}

Given a grammar $G$, with start symbol $S$, we define the \textbfit{$\Sigma^\bot$-tree generated by
$G$}, written $[G]$, by
\begin{equation}
\nonumber [G] \; = \; \sup \{t \in \terms{\infty}{\Sigma^\bot} :
S \downarrow_G t\}
\end{equation}
Intuitively, the tree $[G]$ represents all possible computation paths
taken.

\begin{remark} Note that the setting outlined above for a
term-tree generating grammar differs from our definition of a
string-language generating grammar in the following ways:
\begin{enumerate}
\item Rules are no longer labelled. \item Signature symbols in
$\Sigma$ may now appear on the righthand side of a production
rule. \item We insist that for each non-terminal $F$ there is
exactly one production rule with $F$ on the lefthand side.
\end{enumerate}
These are, in fact, not substantial changes. We could easily have
defined our string-language generating grammars to conform to
items (1) and (2) by insisting that $\Sigma = \Sigma' \cup \{e\}$
where $e$ is the end of word marker, and all symbols in $\Sigma'$ are of type
$(o,o)$. We refer the reader to \cite{dMO} for a discussion of the
possible (but equivalent) definitions of string-language
generating higher-order grammars.

However, restriction (3) is necessary in the term-tree setting to
ensure that we produce exactly \emph{one} term tree and not a
\emph{language} of term trees.
\end{remark}

\subsubsection*{Higher-order pushdown tree automata}

We also introduce higher-order pushdown tree automata. The
definition we present here is taken from \cite{KNU02}.

A \textbfit{level-$n$ pushdown tree automaton} (abbreviated to
$n$PDTA) is a tuple $\anglebra{Q, \Sigma, \Gamma, q_0, \delta}$
where $Q$, $\Sigma$, $\Gamma$ and $q_0$ are as in Section 2;
however, the transition function is now defined as:
\begin{equation}
\nonumber \delta : Q \times \Gamma \rightarrow ((Q \times Op_n)
\cup TreeOp_n)
\end{equation}
where $TreeOp_n =  \{f(p_1, \cdots, p_{\arity{f}}) : f \in \Sigma
\wedge p_1, \cdots, p_{\arity{f}} \in Q\}$.\\

\begin{remark}Note that the above insists on an injective transition function: for
each $(q, Z) \in Q \times \Gamma$ there is exactly one operation
to be performed. Hence, a level-$n$ pushdown tree automaton is
always, in a sense, deterministic.
\end{remark}

A configuration is denoted by a pair $(q,s)$ where $q \in Q$ and
$s$ is an $n$-store, with initial configuration given by $(q,
\bot_n)$. We denote by $C$ the set of all configurations. Given a
configuration $(q,s)$ we say that $(q, s) \rightarrow (p, s')$ if
$(q, \mytop_1(s), (p, \theta)) \in \delta$ where $(p, \theta) \in
Q \times Op_n$ and $\theta(s) = s'$. As before, we denote by
$\rightarrow^*$ the reflexive and transitive closure of
$\rightarrow$.

Let $t: T \rightarrow \Sigma$ be a $\Sigma$-tree. A partial
function $\rho : T \rightarrow C$ defined on an initial fragment
of $T$ is referred to as \textbfit{partial run} of the automaton
on $t$ if and only the following hold:
\begin{itemize}
\item $(q_0, \bot_n) \rightarrow^* \rho(\epsilon)$ \item If $w \in
T$ and $\rho(w) = (q,s)$ then $\delta(q, \mytop_1(s)) = f( p_1,
\cdots, p_{\arity{f}})$ where $t(w) = f$ and $p_1, \cdots,
p_{\arity{f}} \in Q$. Furthermore, $(p_i, s) \rightarrow^*
\rho(wi)$ for $1 \leq i \leq \arity{f}$ when $\rho(wi)$ is
defined.
\end{itemize}
In the case that $\rho$ is total, then it is called simply a
\textbfit{run}. The tree $t$ is accepted by an automaton if there
is a run on $t$. Note that any given automaton can accept at most
one tree.\\

In the following two sections we consider conversions from string
languages to term trees and vice versa. However, in preparation
for this, we give a representation of trees by languages.

\subsubsection*{Branch languages}

Let $t: T \rightarrow \Delta$ be a $\Delta$-tree. We say that a
finite word $a_1d_1a_2d_2 \cdots d_{n-1} a_n$ where each $a_i \in
\Delta$ and each $d_i \in \{1, 2, \cdots \}$ is a
\textbfit{finite branch} in $t$ if and only if:
\begin{enumerate}
\item $d_1d_2 \cdots d_{n-1} \in T$ \item $a_i = t(d_1 \cdots
d_{i-1})$ for $i = 1, 2, \cdots, n$ \item $a_n$ has arity $0$.
\end{enumerate}
We say that an infinite word $a_1d_1a_2d_2 \cdots$ is an
\textbfit{infinite branch} in $t$ if and only if:
\begin{enumerate}
\item $d_1d_2 \cdots d_n \in T$ for each $n$ \item $a_i = t(d_1
\cdots d_{i-1})$
\end{enumerate}
Given a branch $a_1d_1a_2d_2 \cdots$ we say that $a_1a_2 \cdots$
is the \textbfit{label} of the branch.

Following Courcelle \cite{Cou83}, given a signature $\Delta$ we
form a new alphabet:
\begin{equation}
\nonumber \overline{\Delta} = \{[f,i] : f \in \Delta \wedge 1 \leq i
\leq \arity{f}\} \cup \{f : f \in \Delta \wedge \arity{f} = 0\}
\end{equation}
and we say that $w \in {\overline{\Delta}}^* \cup
{\overline{\Delta}}^w$ is in the \textbfit{branch language} of $t:
T \rightarrow \Sigma$ if and only if $w = [a_1, d_1][a_2,d_2]
\cdots $ and $a_1d_1a_2d_2 \cdots $ is a (finite or infinite)
branch of $t$. The branch language of a tree $t$ will be denoted
by $Branch^\infty(t)$. For notational convenience, we will often write
$f_i$ instead of $[f,i]$. Thus if $[h,1][h,3][h,2][f,1]b$
is a word in the branch language, we will write this as $h_1h_3h_2f_1b$.

\begin{lemma} \cite{Cou83} A $\Delta$-tree is completely specified by its
branch language.
\end{lemma}

\subsection{From string languages to term trees}

Let $G$ be a (possibly non-deterministic, possibly unsafe) grammar
that generates a string language $L(G)$ over the signature
$\Sigma$. We will convert it into a term-tree generating grammar
$G_t$ over the signature $\Sigma \cup \{2,3, \cdots n\} \cup
\{e,r\}$, where $n$ is the maximum level of branching (to be
defined later). Intuitively, the tree generated by $G_t$ will
represent \emph{all} possible words generated by $G$, and the
nodes labelled by the new signature symbols $\{2,3, \cdots, n\}$
will
represent the points at which choices are allowed to be made.\\
For example, consider the following grammar:
\begin{eqnarray}
\begin{array}{ll}
\begin{array}{rcl}
\nonumber S & \larr{a} & FHAB \\
\nonumber F\varphi x y & \larr{b} & E_1\\
\nonumber F\varphi x y & \larr{c} & E_2 \\
\nonumber F\varphi x y & \larr{d} & E_3 \\
\end{array} &
\begin{array}{rcl}
\nonumber G \varphi x & \larr{j} & D_1\\
\nonumber G \varphi x & \larr{k} & D_2\\
\nonumber H x & \larr{g} & C \\
\nonumber A & \larr{h} & e\\
\nonumber B & \larr{i} & e
\end{array}
\end{array}
\end{eqnarray}
Note there are $3$ production rules with the non-terminal $F$ on
the left hand side, and $2$ for $G$. Thus whenever we have $F s_1
s_2 s_3$ as a term we can \emph{choose} which production rule to
apply. Therefore we say that the level of branching for $F$ is
$3$, (it is $2$ for $G$, and $1$ for $H, A, B$ and $S$). We say
that the level of branching for a grammar is $n$ just in case $n$
is the maximum of the levels of branching for each non-terminal.
In order to convert the example grammar above into a term-tree
generating one, in the sense of \cite{KNU01, KNU02}, $G_t$, we use
the signature $\Sigma \cup \{2,3\} \cup \{e,r\}$, where $2: o^2
\rightarrow o$ and $3: o^3 \rightarrow o$, each symbol in $\Sigma$
is of type $o \rightarrow o$ and $e,r : o$ and we have the
following production rules:
\begin{eqnarray}
\nonumber S & \rightarrow & a(FHAB) \\
\nonumber F \phi x y &\rightarrow  & 3 (bE_1)(c E_2)( dE_3) \\
\nonumber G \phi x & \rightarrow & 2 (jD_1) (kD_2)\\
\nonumber H x & \rightarrow & gC\\
\nonumber A & \rightarrow & he\\
\nonumber B & \rightarrow & ie
\end{eqnarray}
Finally, in the case where there exists one or more non-terminals
$M$ such that $M$ does not occur on the lefthand side of any
production rule, we add the rule $M\overrightarrow{x} \rightarrow
r$, for $\overrightarrow{x}$ of the appropriate type.

It should not be too difficult to see that each ``properly ending"
branch in $[G_t]$ corresponds to a word in $L(G)$. A
\textbfit{properly ending} branch is one that is finite and whose
final node is labelled by $e$. In fact, $[G_t]$ captures exactly
those words in $L(G)$:

\begin{lemma}
$w \in L(G)$ if and only if there exists a finite branch labelled
by $w_0a_1w_1a_2 \cdots a_nw_ne$ in $[G_t]$ such that each $a_i
\in \Sigma$, $w_i \in \{2,3, \cdots m\}^*$ (where $m$ is the
branching level of $G$) and $w = a_1a_2 \cdots a_n$.
\end{lemma}

\begin{proof} Obvious.
\end{proof}

% It should not be too difficult to see that
%$G_t$ captures exactly those words generated by $G$. The nodes
%labelled by $\{2,3, \cdots, n\}$ are referred to as
%\textbfit{branching nodes}. When we encounter a branching node we
%know that depending on which edge we follow we can continue word
%in a different way.

\begin{remark} It should be clear that if the
original grammar $G$ is safe, then so is the resulting one, $G_t$.
\end{remark}

\subsubsection*{Applying KNU's decidability result}

By \cite{KNU02} the term tree generated by a safe term-tree
generating grammar has a decidable monadic second order (MSO)
theory (see \cite{Tho97} for an introduction to MSO logic).

This has obvious repercussions for a \emph{safe} string-language
generating grammar $G$. Because of the above conversion, we obtain
a tree $[G_t]$, such that each properly ending branch corresponds
to a word in $L(G)$ and vice versa. Therefore, we can obtain
several decidability results about the language $L(G)$ by virtue
of the MSO decidability of $[G_t]$.

\begin{proposition}
Some of the decidability results we can obtain are (1)
non-emptiness, (2) membership, and, if $G$ is deterministic (3)
finiteness.
\end{proposition}

\begin{proof} We give only the proof for non-emptiness; the rest can
be similarly argued. We make the conversion from $G$ to $G_t$ as outlined above and
combine it with the following MSO sentence:
\begin{equation}
\nonumber \exists y . p_{e}(y)
\end{equation}
where $p_{a}$ is the predicate ``is labelled by $a$'' for $a \in
\Sigma \cup \{2,3 \cdots m\} \cup \{e, r\}$ where $m$ is the level
of branching.
\end{proof}

Thus, the above conversion from a grammar $G$ that generates a
string language into a grammar $G_t$ that generates a term tree is
quite a useful one. In particular, for languages in the OI-hierarchy (see Section 2), we have a new and simple way to prove
decidability results.

\subsection{From term trees to string language}

Let us first introduce the definition of the
\textbfit{$\infty$-language} of
a string-language generating grammar $G$, ${L^\infty}(G)$.\\

Let $G = \anglebra{N, V, \Sigma, {\cal R}, S, e}$ be level-$n$
grammar that generates a string language. We say that a derivation sequence $S
= P_1 \larr{a_1} P_2 \larr{a_2} \cdots$ is \textbfit{valid} if
\begin{itemize}
\item it is finite and ends in $e$; or \item it is infinite.
\end{itemize}

If $S = P_1 \larr{a_1} P_2 \larr{a_2} \cdots$ is a valid
derivation sequence, then we say that the word $a_1a_2 \cdots$ is
in the $\infty$-language of $G$.

%In particular, we write:
%\begin{equation}
%\nonumber {L^\infty}(G) = \{w : S = P_1 \larr{a_1} P_2
%\larr{a_2} \cdots \mbox{ is valid and } w = a_1a_2 \cdots \}
%\end{equation}

\begin{remark}Thus, we now allow for a string-language generating
grammar to generate $\omega$-words \emph{in addition to} finite
words.

In the literature, $\omega$-languages (consisting only of
infinite words) are more commonly defined in terms of a machine
model with an associated acceptance condition; such as Buchi,
Muller, and so on (see e.g. \cite{GTW02}). Thus, although our definition
seems somewhat of a departure from this more standard notion, we
will see that it is sufficient for our purposes.
\end{remark}

%In \cite{KNU01, KNU02}, given a term-tree generating grammar $G$,
%the resulting tree $[G]$ is a term tree over the signature
%$\Sigma^\bot = \Sigma \cup \{\bot\}$ where $\bot : o$. Thus, we
%introduce the following definition:

\begin{definition} Let $G$ be a term-tree generating grammar, and let
$[G]$ be the resulting term tree over the signature $\Sigma^\bot$.
We say that a word $w$, over the alphabet $\overline{\Sigma}$, is
in the \textbfit{bottomless branch language} of $[G]$ if and only
if:
\begin{itemize}\item $w$ is an infinite word in $Branch^\infty([G])$,
or \item $w\bot$ is a finite word in $Branch^\infty([G])$, or
\item $w$ is a finite word in $Branch^\infty([G])$ that \emph{does
not} end in $\bot$.
\end{itemize}
\end{definition}

\begin{lemma} If $t$ is a tree over $\Sigma^\bot$ (as above),
then $t$ is completely defined by its bottomless branch language.
\end{lemma}

\begin{proof} Obvious.
\end{proof}

\begin{lemma}\label{lem:ts} Given a term-generating grammar $G = \anglebra{N,
V, \Sigma, {\cal R}, S}$, we can convert it into a string-language
generating grammar $G_s$ such that $w \in {L^\infty}(G_s)$ if and
only if $w$ is in the bottomless branch language of $[G]$
%a  string-language generating grammar $G_s$ such that $w \in {L^\infty}(G_s)$ if and only if
%\begin{itemize}
%\item $w$ is an infinite branch in $[G]$, or \item $w\bot$ is a
%finite branch in $[G]$, or \item $w$ is a finite branch in
%$[G]$.
%\end{itemize}
\end{lemma}

\begin{proof} Let $G_s = \anglebra{N_s, V_s, \overline{\Sigma}, {\cal R}_s,
S, e}$, where $N_s = N \cup \{S_f : f \in \Sigma\}$ and the set of
rewrite rules, ${\cal R}_s$, is constructed as follows. For each
existing production rule, $F \overrightarrow{x} \rightarrow E$, in
${\cal R}$ we simply rewrite this as:
\begin{equation}
\nonumber F \overrightarrow{x} \larr{\epsilon} E'
\end{equation}
where $E'$ is $E$ with each occurrence of a signature symbol $f$
replaced by $S_f$. Furthermore, for each signature constant $f$ of
arity $\arity{f} > 0$ we add the production below for each $i = 1,
\cdots, \arity{f}$:
\begin{eqnarray}
\nonumber S_f x_1 \cdots x_{\arity{f}} & \larr{f_i} & x_i
\end{eqnarray}
In case $\arity{f} = 0$, then we simply add
\begin{eqnarray}
\nonumber S_f & \larr{f} & e
\end{eqnarray}
Finally $V_s$ can easily be inferred from the above.
\end{proof}

Note that for a given term-tree generating grammar $G$, the
resulting grammar $G_s$ will always be deterministic. This is
because of the restriction on $G$ that for each non-terminal $N$
there is exactly one production rule with $N$ on the lefthand side.

\begin{example}
As an example, consider the following term-tree generating
grammar, with $\Sigma = \{h,g,f,a,b\}$ where $h: (o,o,o,o), \; f:
(o,o,o), \; g: (o,o)$ and $a, b : o$.
\begin{eqnarray}
\nonumber S & \rightarrow & Dgab \\
\nonumber D\varphi x y & \rightarrow & h(D(D\varphi x)y (\varphi y)) (H (fy) x) (\varphi b)\\
\nonumber H \varphi x & \rightarrow & \varphi x
\end{eqnarray}
Applying the above conversion gives us the grammar given in
Example~\ref{ex:ex1}.
\end{example}

\subsubsection*{Applying our result}

Unfortunately, our level-$2$ result offers little in the way of
solving either of the key open problems regarding the safety
restriction for term-tree generating grammars. In particular, they
are, as stated by Knapik \emph{et al.}:
\begin{enumerate}
\item Can every term tree generated by an unsafe grammar of
level-$n$ be generated by a safe grammar of the same level? Or of
another level? \item Is safety a necessary requirement to ensure
MSO-decidability?
\end{enumerate}

The reason we cannot apply our result is due to the introduced
non-determinism. As indicated above, we \emph{can} convert a
term-tree generating grammar $G$ into a grammar $G_s$ such that
${L^\infty}(G_s)$ is the bottomless branch language of $G$. At
level $2$, by our result, we can even go on to construct a $2$PDA
such that it accepts ${L^\infty}(G_s)$. However, the $2$PDA will,
in general, be non-deterministic and therein lies the problem.

%For example, recall the grammar in Example \ref{ex:ex1}: the set
%of guesses to produce the branch $h_1h_3h_2f_1b$ are incompatible
%with those used to produce the branch $h_1h_3h_2f_2a$. Thus,
%although both words will be present in ${L^\infty}(G)$,
%there will be no way to associate that they share the same prefix
%and thus are actually the same path up
%to the node $f$.\\

However, as a small consolation we mention two possible avenues
that we believe are worthy of investigation. The first is a
conjecture, which, if proven correct will show that safety
\emph{is} a restriction in terms of generating power for the
term-tree setting. The second introduces the temporal logic,
existential LTL, and shows that it is decidable for term trees
generated by unsafe level-$2$ grammars by way of our result.

\subsection{Urzyczyn's Language: a conjecture}

We have shown that the language $U$ is accepted by a
non-deterministic $2$PDA. We conjecture that it cannot be accepted
by a deterministic $2$PDA. If our conjecture is true then we will
have an example of an inherently unsafe term tree, i.e. an unsafe level-$2$ grammar whose term
tree cannot be generated by a safe level-$2$ grammar.

\begin{figure}[h]
\begin{diagram}[height=1.5em, width=1em]
&& & & \pq & & \\
&& & & \dTo{}{}& & \\
&& & & 3 & & \\
&& & \ldTo{}{} & \dTo{}{}& \rdTo{}{} & \\
&& \pq&  & \oa &  &  *\\
&& \dTo{}{} &  & \dTo{}{} &  &  \dTo{}{}\\
&& 3 &  & r &  &  e\\
& \ldTo{}{} & \dTo{}{} & \rdTo{}{} &  &  &  \\
\pq&  & \oa & & * &  &  \\
\dTo{}{} &  & \dTo{}{} & & \dTo{}{} &  &  \\
3 &  & 3 & & * &  &  \\
\vdots &  & \vdots & & \dTo{}{}  &  &  \\
 &  &  & & e  &  &  \
\end{diagram}
\caption{The term tree $[G_{U}]$\label{fig:utree}}
\end{figure}

Let us apply the conversion from Section 7.2 to the unsafe
string-language generating grammar for $U$ in Section 3, to give
us a new grammar, $G_{U}$. In particular, our signature is $\Sigma
= \{\pq : (o,o), \; \oa: (o,o), \; *: (o,o), \; 3: (o,o,o,o), \; e
: o, \; r : o\}$ and we have the following production rules:

\begin{eqnarray}
\nonumber S & \rightarrow & \pq DGEEE \\
\nonumber D \varphi x y z & \rightarrow & 3 (\pq D(D \varphi x) z (Fy) (Fy)) (\oa \varphi y x) (* z)\\
\nonumber Fx & \rightarrow & \ast x \\
\nonumber E & \rightarrow & e\\
\nonumber G & \rightarrow & r
\end{eqnarray}
The term tree generated by $G_U$ is shown in
Figure~\ref{fig:utree}.

\begin{proposition}Suppose that $[G_U]$ can be
generated by a safe level-$2$ (term tree) grammar. Then it must be
the case that the language $U$ can be accepted by a deterministic
$2$PDA.
\end{proposition}

%\begin{proof} By \cite{KNU02}, if the term tree in Fig~\ref{fig:utree}
%is indeed generated by a safe level-$2$ term-tree generating
%grammar then there exists a $2$PDTA $A = \anglebra{Q, \Sigma,
%\Gamma, q_0} $ that accepts $[G_U]$.
%
%We will see that we can easily modify $A$ into $A_s =
%\anglebra{Q_s, \Sigma, \Gamma, q_0, \delta_s, F_s}$ where $A_s$ is
%a deterministic $2$PDA (as described in Section 2.4).
%
%Analysing $[G_U]$ we can easily see that whenever we have
%$\delta(q, a) = (3, p_1, p_2, p_3)$ it must be the case that $p_i
%\not = p_j$ for $1 \leq i < j \leq 3$ as each immediate successor
%of $3$-node is distinct. It is this property that allows us to
%convert $A$ into a deterministic pushdown automaton that accepts
%$U$. We let $Q_s = Q \cup \{q_r, q_e\} \cup \{q^a : q \in Q \wedge
%a \in \Sigma\}$ and $F = \{q_e\}$. Furthermore, for each $(q, Z,
%(p,op)) \in \delta$ we add the rule $(q, \epsilon, Z, p, op)$ to
%$\delta_s$. However, if $f \in \Sigma^{(o,o)}$, then for each $(q,
%Z, f(p_1)) \in \delta$ we add the rule $(q, f, Z, p_1, \id)$ to
%$\delta_s$. What remain are the cases concerning signature symbols
%$3, e, r$.
%
%For every $(q, Z, (3(p_1, p_2, p_n)) \in \delta$, we replace this,
%in $\delta_s$, with the family of rules:
%\begin{eqnarray}
%\nonumber \delta(q, \pq, Z) & = & (p_1^{\pq}, \id) \\
%\nonumber \delta(q, \oa, Z) & = & (p_2^{\oa}, \id)\\
%\nonumber \delta(q, *, Z) & = & (p_3^{*}, \id)
%\end{eqnarray}
%where, for each $a \in \Sigma$,  $(p^a, \epsilon, Z, q^a, op) \in
%\delta_s$ if and only if $(p, Z, (op, q)) \in \delta$ and $(p^a,
%a, Z, q, id) \in \delta_s$ if and only if $(p, Z, (a,q)) \in
%\delta$. Recall that the immediate successors of a $3$-node are
%always $\pq, oa,$ or $*$, all of which are of arity $1$. The
%intuition behind this is that we preempt the $2$PDA.
%
%For each $(q, Z, r) \in \delta$ we convert this into $(q,
%\epsilon, Z, q_R, \id)$ where $q_R$ is a new state not in $F$. In
%particular, $F = \{q_e\}$ where $q_e$ is another new state, and we
%have $(q, \epsilon, Z, q_e, \id) \in \delta_s$ if and only if $(q,
%Z, e) \in \delta$.
%\end{proof}


\begin{proof} By \cite{KNU02}, if $[G_U]$ is indeed generated by a safe level-$2$
grammar then there exists a $2$PDTA $A = \anglebra{Q, \Sigma,
\Gamma, q_0, \delta} $ that accepts $[G_U]$.

Analysing $[G_U]$ we can easily see that whenever $\delta(q, a) =
3(p_1, p_2, p_3)$ it must be the case that $p_i \not = p_j$ for $1
\leq i < j \leq 3$ as each immediate successor of $3$-node is
distinct. It is this property that allows us to convert $A$ into a
deterministic pushdown automaton $A_s$ that accepts the language
$U$. Let $A_s = \anglebra{Q_s, \Sigma - \{3,e,r\}, \Gamma, q_0,
\delta_s, F_s}$ where let $Q_s = Q \cup \{q_r, q_e\} \cup \{q^a :
q \in Q \wedge a \in \Sigma\}$. Furthermore, $\delta_s$ is
constructed from $\delta$ as follows:
\begin{itemize}
\item for each $(q, Z, (p,op)) \in \delta$ we add the rule $(q,
\epsilon, Z, p, op)$ to $\delta_s$. \item if $f \in
\Sigma^{(o,o)}$, then for each $(q, Z, f(p_1)) \in \delta$ we add
the rule $(q, f, Z, p_1, \id)$ to $\delta_s$. \item For every $(q,
Z, 3(p_1, p_2, p_n)) \in \delta$, we add the following the family
of rules to $\delta_s$:
\begin{eqnarray}
\nonumber \delta(q, \pq, Z) & = & (p_1^{\pq}, \id) \\
\nonumber \delta(q, \oa, Z) & = & (p_2^{\oa}, \id)\\
\nonumber \delta(q, \ast, Z) & = & (p_3^{\ast}, \id)
\end{eqnarray}
where, for each $a \in \{\pq, \oa, \ast \}$,  $(p^a, \epsilon, Z, q^a,
op) \in \delta_s$ if and only if $(p, Z, (q, op)) \in \delta$ and
$(p^a, \epsilon, Z, q, id) \in \delta_s$ if and only if $(p, Z,
a(q)) \in \delta$. \item For each $(q, Z, r) \in \delta$ we have
$(q, \epsilon, Z, q_r, \id) \in \delta_s$ where $q_r$ is a new
state not in $F$. In particular, $F = \{q_e\}$ where $q_e$ is
another new state, and we have $(q, \epsilon, Z, q_e, \id) \in
\delta_s$ if and only if $(q, Z, e) \in \delta$.
\end{itemize}
The key to the construction lies in examining what happens in the
$2$PDTA when we are in a configuration $(q,Z)$ such that
$\delta(q,Z) = 3(p_1, p_2, p_3)$. As mentioned, the $p_i$'s are
distinct, thus, in the translation to
a deterministic $2$PDA, we will ignore the $3$, and depending on
the current input symbol we change the state to $p_1^{\pq},
p_2^{\oa}$ or $p_3^{\ast}$. Note the superscripted states: these are
required because we must take care of the fact that we have
consumed input ``prematurely". In particular, the set of states
$Q^{\pq} = \{q^{\pq} : q \in Q\}$ are such that the transitions
between them match exactly the $\epsilon$-transitions of $\delta$.
But we are not allowed to leave $Q^{\pq}$ until we reach a
configuration $(q^{\pq}, w, s)$ where $(q, top_1(s), \pq(p_1)) \in
\delta$, and, in this case, we leave $Q^{\pq}$ via an
$\epsilon$-transition $(p_1, w, s)$.
\end{proof}

\begin{conjecture} $U$ cannot be accepted by a deterministic
$2$PDA, and hence, there exists a term-tree generated by an unsafe
level-$2$ grammar that cannot be generated by any safe level-$2$
grammar.
\end{conjecture}

\subsection{Existential LTL}

Let $G = \anglebra{N, V, \Sigma, {\cal R}, S}$ be a level-$2$
unsafe term-tree generating grammar. By Section 7.3 we can convert
this into a level-$2$ unsafe string-language generating grammar
$G_s$, such that ${L^\infty}(G_s)$ is the bottomless branch
language of $G$.

Now, by our level-$2$ result, we can convert $G_s$ into a safe
grammar $G_{safe}$ that generates this same bottomless branch
language. In general, however, $G_{safe}$ will contain ``dead"
non-terminals, such that for each ``dead" non-terminal $F$, there
will be no production rule that contains $F$ on the left hand
side. These non-terminals arises solely because of abortive
computations of the non-deterministic $2$PDA caused by incorrect
guesses.

Finally, by the conversion\footnote{It should be clear that the
conversion in Section 7.2 also works when we consider
$\infty$-languages instead of languages of finite words.} in
Section 7.2 we obtain a term-tree generating grammar $G_t$ with
the following property. Every infinite or properly ending branch
in $[G_t]$ corresponds to a word in $L^\infty(G_{safe})$. In
particular, each branch in $[G_t]$ that does not end in
$r$\footnote{We intend the $r$ to mean ``reject".}
corresponds to a branch in $[G]$! Summing up (works only for level $2$):\\

\begin{eqnarray}
\nonumber G & & \mbox{unsafe term-tree generating grammar} \\
\nonumber \downarrow & & \mbox{(Section 7.3)}\\
\nonumber G_s &  &\mbox{unsafe string-language generating grammar} \\
\nonumber \downarrow & & \mbox{(Theorem 4.1)}\\
\nonumber G_{safe} & &  \mbox{safe string-language generating grammar} \\
\nonumber \downarrow & & \mbox{(Section 7.2 )}\\
\nonumber G_t & & \mbox{safe term-tree generating grammar}
\end{eqnarray}

%\begin{equation}
%\nonumber\begin{array}{ccccccc}
%G & \longrightarrow & G_s & \longrightarrow & G_{safe} & \longrightarrow & G_t\\
%\mbox{unsafe term-tree grammar} & & \mbox{unsafe string-language grammar} & & \mbox{safe string-language grammar} & & \mbox{safe term-tree grammar}
%\end{array}
%\end{equation}

As an example, suppose that $f1f2g1a$ is a branch in $[G]$, this
will manifest itself as a branch $w_0f_1w_1f_2w_3g_1w_4aw_5e$ in
$[G_t]$, where each $w_i \in \{2, 3, \cdots m\}^*$ for some fixed
$m$. \\

Note that $[G_t]$ \emph{does} possess a decidable MSO theory as it
is generated by a safe grammar. Thus, we can certainly model-check
certain path properties of $[G]$ via $[G_t]$. As an example, we
consider \emph{existential LTL}. In fact, we can probably
model-check more properties than these, but these are sufficient
to show that some useful model-checking is viable.

The formulae of existential LTL are the following. If $\Sigma$ is
the set of signature symbols of $G$, then we have the following
formulae:

\begin{eqnarray}
\nonumber \phi := p_f \; | \; {\tt true} \; | \; {\tt false} \; |
\; \phi \wedge \phi \; | \; \neg \phi \; | \; X \phi \; | \; \phi
U \phi
\end{eqnarray}
where $f \in \Sigma^\bot$. $X$ means ``next" and $U$ means ``until" in
the usual sense (see \cite{Var95}). In particular, we say that
$[G] \models_{\exists} \phi$ if there exists a path $\pi$ in $[G]$
such that $\phi$ holds. Thus, we can state properties such as,
``There exists a path such that $f$ never holds," or ``There
exists a path such that whenever $g$ holds, eventually $h$ holds."
Hence the name \emph{existential} LTL. Note that this is useful
for model checking safety\footnote{This has nothing to do with the
syntactic restriction of safety.} properties.

To show that such a property $\phi$ is indeed decidable for $[G]$
all one needs to do is give a corresponding formulae $\phi_t$ and
show that it holds for $[G_t]$.

For example, suppose that we wish to establish that $[G]
\models_{\exists} X \phi$. All we need do is decide whether the
following MSO formula holds true for $[G_t]$: $\exists X. path(X)
\wedge \exists z . z \not = root(X) \wedge z \in X \wedge \phi(z)
\wedge (\bigvee_{f \in \Sigma, i \leq \arity{f}} p_{f_i}(z))
\wedge \forall y . (y \in X \wedge (root(X) <y < z) \rightarrow
\vee_{i \in \{2, 3, \cdots m\}} p_i(y))$, where the predicate
$path(X)$ means ``$X$ is a prefix-closed maximal path that does
not end in the signature constant $r$" and $root(X)$ is the least
node in $X$ labelled with a signature constant.

\begin{remark} If $\pi$ is a branch in $[G]$ labelled by $a_1 \cdots a_n \bot$ then either
\begin{itemize}
\item There exists a branch labelled by $w_0 a_1 w_1 \cdots
w_{n-1} a_n w_n \bot$ in $[G_t]$ where $w_0, \cdots, w_n$ are
finite words over $\{2, 3, \cdots, m\}$; or \item There exists a
branch labelled by $w_0 a_1 w_1 \cdots w_{n-1} a_n w_n$ in $[G_t]$
where $w_0, \cdots, w_{n-1}$ are finite words over $\{2, 3,
\cdots, m\}$ and $w_n$ is an infinite word over $\{2, 3, \cdots,
m\}$.
\end{itemize}
Thus, it is possible (and easy) to have a predicate $p_\bot$.
\end{remark}

%\begin{remark} If $\pi\bot$ is a branch in $[G]$, then there will
%be a branch $w_0a_1w_1 \cdots w_{n-1}a_nw_n$ in $[G_t]$ where
%$w_0, \cdots, w_{n-1}$ are finite words over $\{2, 3, \cdots, m\}$
%and $w_n$ is an infinite word over $\{2, 3, \cdots, m\}$. Thus, it
%is possible (and easy) to have a predicate $p_\bot$.
%\end{remark}


% Further directions
\section{Further directions}
Let us recall our main result. We have shown that the string
language of every level-$2$ grammar (whether safe of unsafe) can
be accepted by a $2$PDA. Combining this with \cite{DG86} we have
proved that every string language that is generated by an unsafe
$2$-grammar can also be generated by a safe (non-deterministic)
$2$-grammar. Thus, there are no \emph{inherently} unsafe string
languages at level $2$. Hence we arrive at the title of our paper:
safety is not a restriction at level $2$ (at least for string
languages). We have also given a small application of our result in the term-tree setting. However, our
result leaves many questions unanswered. Some of the most obvious
are listed here. 
\begin{itemize}
\item Does our result extend to levels $3$ and beyond?
\item What is the relationship between deterministic unsafe grammars and
deterministic safe grammars? In particular, can $U$ -- which is
generated by a deterministic unsafe level-$2$ grammar -- be
generated by a deterministic safe grammar and hence accepted by a
deterministic $2$PDA?
\item Is homogeneity a necessary restriction as well?
\item Can we extract more decidability results for
the term-tree setting? More specifically, is safety a requirement for
MSO decidability?
\end{itemize}

%\subsection*{Extension to level 3?}

%Let us recall our main result. We have shown that given a possibly
%unsafe level-$2$ grammar, we can convert it into a non-deterministic
%$2$PDA that accepts the same language. Our conversion is split into
%two transformations:
%\begin{equation}
%\nonumber \mbox{$2$-grammar} \stackrel{(1)}{\longrightarrow}
%\mbox{$2$PDAL} \stackrel{(2)}{\longrightarrow} \mbox{2$PDA$}
%\end{equation}
%where transformation $(1)$ was defined in Section 5 and transformation
%$(2)$ in Section 6.
%
%\medskip
%
%It should not be difficult to see transformation (1) can easily be
%extended to all levels. However, at present we are unable to extend
%(2) to levels 3 and beyond. We hope, however, that our current
%inability to extend (2) to higher levels is symptomatic of our lack of
%understanding as opposed to level-$2$ grammars being a special
%case. We outline our difficulties below.
%
%The key feature of the $2$PDAL that allows us to simulate links and
%their manipulation via non-determinism is the fact that a link $m$ is
%\textbfit{examined} at most once during the computation. By the term
%examined we mean it to indicate a configuration $(q_j, s)$ where
%$top_1(s) = {\$t_1 \cdots t_n}^{\langle m- \rangle \cup \lambda}$ and
%$j > 0$. Recall from Definition \ref{def:followed} if $j > n$ then we
%say the link is followed. Thus, by the term examined, we refer to a
%configuration where there is a possibility where we may either
%\begin{itemize}
%\item follow a link
%\item not follow a link
%\end{itemize}
%both of these depend on what $j$ and $n$ are. However, the important
%thing to note is, regardless of which action is taken, we will examine
%a link $m$ \emph{at most once}. This follows from Lemmas
%\ref{lem:useonceonly} and \ref{lem:notfollowed}. Thus, if we examine a
%link $m$ for the first time in a run of the pushdown automaton, we can be guaranteed that we will never meet
%$m$ again in the context of an examination. This is what allows us to
%summon an oracle and tell us, for each link $m$, if we will ever
%examine it, and, if so, whether we will follow it or not.\\
%
%This is not, however, the case with a $3$PDAL. For example,
%consider the configuration below. Here, we adopt the convention
%that the $\Psi$'s are of level $2$ and the $\phi$'s are of level
%$1$ and the $x$'s are of level 0.
%\begin{equation}
%\left ( q_0, \mkstore{\mkstore{\mkstore{\Psi \phi_1 x, \cdots}}}
%\right )
%\end{equation}
%As the topmost item is headed by a level-$2$ variable we perform a
%$\push_2$.  Suppose that we get:
%\begin{equation}
%\nonumber \left ( q_0, \left [ \left [ \begin{array}{l}
%\mkstore{\mklink{D s_1 \cdots s_n t_1}{m-}, \cdots}\\
%\mkstore{\mklink{\Psi \phi_1 x}{m+}, \cdots} \end{array} \right ]
%\right ] \right )
%\end{equation}
%where the $s_i$'s are terms of level $2$ and $t_1$ is a term of
%level $1$. Suppose also that the rule for $D$ is $D \Psi_1 \cdots
%\Psi_n \phi_1 \cdots \phi_m x_1 \cdots x_k \rightarrow \phi_1
%(\phi_2 x)$. We next have:
%\begin{equation}
%\nonumber \left ( q_0, \left [ \left [
%\begin{array}{l}
%\mkstore{\phi_1 (\phi_2 x), \mklink{D s_1 \cdots s_n t_1}{m-} , \cdots}\\
%\mkstore{\mklink{\Psi \phi_1 x}{m+}, \cdots}
%\end{array}\right ] \right ] \right )
%\end{equation}
%Now, as the topmost item is headed by a level-$1$ variable, we
%perform a $\push_3$, and we get:
%\begin{equation}
%\nonumber \left ( q_{n+1}, \left [
%\begin{array}{l}
%\left [ \begin{array}{l}
%\mkstore{\mklink{D s_1 \cdots s_n t_1}{m-} , \cdots}\\
%\mkstore{\mklink{\Psi \phi_1 x}{m+}, \cdots}
%\end{array} \right ] \\
%\left [ \begin{array}{l}
%\mkstore{\phi_1 (\phi_2 x), \mklink{D s_1 \cdots s_n t_1}{m-} , \cdots}\\
%\mkstore{\mklink{\Psi \phi_1 x}{m+}, \cdots} \end{array} \right ]
%\end{array} \right ] \right )
%\end{equation}
%Note that we are examining the link $m$. In this instance, we need
%not perform a $\pop_2$; i.e. we do not follow the link.
%However, suppose that after some time we eventually return to the
%configuration below, but in a new state, i.e.
%\begin{equation}
%\nonumber \left ( q_1, \left [ \left [
%\begin{array}{l}
%\mkstore{\phi_1 (\phi_2 x), \mklink{D s_1 \cdots s_n t_1}{m-} , \cdots}\\
%\mkstore{\mklink{\Psi \phi_1 x}{m+}, \cdots}
%\end{array}\right ] \right ] \right )
%\end{equation}
%then we get:
%\begin{equation}
%\nonumber \left ( q_0 \left [ \left [
%\begin{array}{l}
%\mkstore{(\phi_2 x), \mklink{D s_1 \cdots s_n t_1}{m-} , \cdots}\\
%\mkstore{\mklink{\Psi \phi_1 x}{m+}, \cdots}
%\end{array}\right ] \right ] \right )
%\end{equation}
%Again, topmost item is headed by a level-$1$ variable, perform a
%$\push_3$, and we get:
%\begin{equation}
%\nonumber\left ( q_{n+2} \left [
%\begin{array}{l}
%\left [ \begin{array}{l}
%\mkstore{\mklink{D s_1 \cdots s_n t_1}{m-} , \cdots}\\
%\mkstore{\mklink{\Psi \phi_1 x}{m+}, \cdots}
%\end{array} \right ] \\
%\left [ \begin{array}{l}
%\mkstore{(\phi_2 x), \mklink{D s_1 \cdots s_n t_1}{m-} , \cdots}\\
%\mkstore{\mklink{\Psi \phi_1 x}{m+}, \cdots} \end{array} \right ]
%\end{array} \right ] \right )
%\end{equation}
%Note that we are examining the link $m$ again, and this time we do
%want to $\pop_2$.
%
%Thus, we have examined the link $m$ twice (at least)! We no longer
%have the luxury of asking the oracle whether or not a link will be
%followed or not, because, as we have just seen: we may follow it in
%one instance but not in another.


\bibliography{ong00-88,ong89-90}
\bibliographystyle{alpha}

\end{document}
