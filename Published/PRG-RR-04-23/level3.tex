Let us recall our main result. We have shown that the string
language of every level-$2$ grammar (whether safe of unsafe) can
be accepted by a $2$PDA. Combining this with \cite{DG86} we have
proved that every string language that is generated by an unsafe
$2$-grammar can also be generated by a safe (non-deterministic)
$2$-grammar. Thus, there are no \emph{inherently} unsafe string
languages at level $2$. Hence we arrive at the title of our paper:
safety is not a restriction at level $2$ (at least for string
languages). We have also given a small application of our result in the term-tree setting. However, our
result leaves many questions unanswered. Some of the most obvious
are listed here. 
\begin{itemize}
\item Does our result extend to levels $3$ and beyond?
\item What is the relationship between deterministic unsafe grammars and
deterministic safe grammars? In particular, can $U$ -- which is
generated by a deterministic unsafe level-$2$ grammar -- be
generated by a deterministic safe grammar and hence accepted by a
deterministic $2$PDA?
\item Is homogeneity a necessary restriction as well?
\item Can we extract more decidability results for
the term-tree setting? More specifically, is safety a requirement for
MSO decidability?
\end{itemize}

%\subsection*{Extension to level 3?}

%Let us recall our main result. We have shown that given a possibly
%unsafe level-$2$ grammar, we can convert it into a non-deterministic
%$2$PDA that accepts the same language. Our conversion is split into
%two transformations:
%\begin{equation}
%\nonumber \mbox{$2$-grammar} \stackrel{(1)}{\longrightarrow}
%\mbox{$2$PDAL} \stackrel{(2)}{\longrightarrow} \mbox{2$PDA$}
%\end{equation}
%where transformation $(1)$ was defined in Section 5 and transformation
%$(2)$ in Section 6.
%
%\medskip
%
%It should not be difficult to see transformation (1) can easily be
%extended to all levels. However, at present we are unable to extend
%(2) to levels 3 and beyond. We hope, however, that our current
%inability to extend (2) to higher levels is symptomatic of our lack of
%understanding as opposed to level-$2$ grammars being a special
%case. We outline our difficulties below.
%
%The key feature of the $2$PDAL that allows us to simulate links and
%their manipulation via non-determinism is the fact that a link $m$ is
%\textbfit{examined} at most once during the computation. By the term
%examined we mean it to indicate a configuration $(q_j, s)$ where
%$top_1(s) = {\$t_1 \cdots t_n}^{\langle m- \rangle \cup \lambda}$ and
%$j > 0$. Recall from Definition \ref{def:followed} if $j > n$ then we
%say the link is followed. Thus, by the term examined, we refer to a
%configuration where there is a possibility where we may either
%\begin{itemize}
%\item follow a link
%\item not follow a link
%\end{itemize}
%both of these depend on what $j$ and $n$ are. However, the important
%thing to note is, regardless of which action is taken, we will examine
%a link $m$ \emph{at most once}. This follows from Lemmas
%\ref{lem:useonceonly} and \ref{lem:notfollowed}. Thus, if we examine a
%link $m$ for the first time in a run of the pushdown automaton, we can be guaranteed that we will never meet
%$m$ again in the context of an examination. This is what allows us to
%summon an oracle and tell us, for each link $m$, if we will ever
%examine it, and, if so, whether we will follow it or not.\\
%
%This is not, however, the case with a $3$PDAL. For example,
%consider the configuration below. Here, we adopt the convention
%that the $\Psi$'s are of level $2$ and the $\phi$'s are of level
%$1$ and the $x$'s are of level 0.
%\begin{equation}
%\left ( q_0, \mkstore{\mkstore{\mkstore{\Psi \phi_1 x, \cdots}}}
%\right )
%\end{equation}
%As the topmost item is headed by a level-$2$ variable we perform a
%$\push_2$.  Suppose that we get:
%\begin{equation}
%\nonumber \left ( q_0, \left [ \left [ \begin{array}{l}
%\mkstore{\mklink{D s_1 \cdots s_n t_1}{m-}, \cdots}\\
%\mkstore{\mklink{\Psi \phi_1 x}{m+}, \cdots} \end{array} \right ]
%\right ] \right )
%\end{equation}
%where the $s_i$'s are terms of level $2$ and $t_1$ is a term of
%level $1$. Suppose also that the rule for $D$ is $D \Psi_1 \cdots
%\Psi_n \phi_1 \cdots \phi_m x_1 \cdots x_k \rightarrow \phi_1
%(\phi_2 x)$. We next have:
%\begin{equation}
%\nonumber \left ( q_0, \left [ \left [
%\begin{array}{l}
%\mkstore{\phi_1 (\phi_2 x), \mklink{D s_1 \cdots s_n t_1}{m-} , \cdots}\\
%\mkstore{\mklink{\Psi \phi_1 x}{m+}, \cdots}
%\end{array}\right ] \right ] \right )
%\end{equation}
%Now, as the topmost item is headed by a level-$1$ variable, we
%perform a $\push_3$, and we get:
%\begin{equation}
%\nonumber \left ( q_{n+1}, \left [
%\begin{array}{l}
%\left [ \begin{array}{l}
%\mkstore{\mklink{D s_1 \cdots s_n t_1}{m-} , \cdots}\\
%\mkstore{\mklink{\Psi \phi_1 x}{m+}, \cdots}
%\end{array} \right ] \\
%\left [ \begin{array}{l}
%\mkstore{\phi_1 (\phi_2 x), \mklink{D s_1 \cdots s_n t_1}{m-} , \cdots}\\
%\mkstore{\mklink{\Psi \phi_1 x}{m+}, \cdots} \end{array} \right ]
%\end{array} \right ] \right )
%\end{equation}
%Note that we are examining the link $m$. In this instance, we need
%not perform a $\pop_2$; i.e. we do not follow the link.
%However, suppose that after some time we eventually return to the
%configuration below, but in a new state, i.e.
%\begin{equation}
%\nonumber \left ( q_1, \left [ \left [
%\begin{array}{l}
%\mkstore{\phi_1 (\phi_2 x), \mklink{D s_1 \cdots s_n t_1}{m-} , \cdots}\\
%\mkstore{\mklink{\Psi \phi_1 x}{m+}, \cdots}
%\end{array}\right ] \right ] \right )
%\end{equation}
%then we get:
%\begin{equation}
%\nonumber \left ( q_0 \left [ \left [
%\begin{array}{l}
%\mkstore{(\phi_2 x), \mklink{D s_1 \cdots s_n t_1}{m-} , \cdots}\\
%\mkstore{\mklink{\Psi \phi_1 x}{m+}, \cdots}
%\end{array}\right ] \right ] \right )
%\end{equation}
%Again, topmost item is headed by a level-$1$ variable, perform a
%$\push_3$, and we get:
%\begin{equation}
%\nonumber\left ( q_{n+2} \left [
%\begin{array}{l}
%\left [ \begin{array}{l}
%\mkstore{\mklink{D s_1 \cdots s_n t_1}{m-} , \cdots}\\
%\mkstore{\mklink{\Psi \phi_1 x}{m+}, \cdots}
%\end{array} \right ] \\
%\left [ \begin{array}{l}
%\mkstore{(\phi_2 x), \mklink{D s_1 \cdots s_n t_1}{m-} , \cdots}\\
%\mkstore{\mklink{\Psi \phi_1 x}{m+}, \cdots} \end{array} \right ]
%\end{array} \right ] \right )
%\end{equation}
%Note that we are examining the link $m$ again, and this time we do
%want to $\pop_2$.
%
%Thus, we have examined the link $m$ twice (at least)! We no longer
%have the luxury of asking the oracle whether or not a link will be
%followed or not, because, as we have just seen: we may follow it in
%one instance but not in another.
