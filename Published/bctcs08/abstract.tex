\documentclass[a4paper,12pt]{article}
\begin{document}

% Please capitalize the title!

\begin{center}
{\large A Concrete Presentation of Game Semantics}\\[3mm]
\textbf{William Blum}\\[3mm]
(joint work with Luke Ong)\\[3mm]
\end{center}
\bigskip

\noindent  We briefly present a new representation theory for game semantics which is very concrete: instead of playing in an arena game in which P plays the innocent strategy given by a term, the same game is played out over (a souped up version of) the abstract syntax tree of the term itself. The plays that are thus traced out are called \emph{traversals}. More abstractly, traversals are the justified sequences that are obtained by performing parallel-composition \emph{less} the hiding. After stating and explaining a number of Path-Traversal Correspondence Theorems, we present a tool for game semantics based on the new representation.


% Your affiliation, example:
\vspace{5mm}\vfill\noindent
William Blum\\[.5ex]
Oxford University Computing Laboratory, UK\\
\texttt{William.Blum@comlab.ox.ac.uk}\\[2ex]

\end{document}
