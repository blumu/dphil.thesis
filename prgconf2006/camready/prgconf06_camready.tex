\documentstyle{ouclprgsc}

\newtheorem{thm}{Theorem}
\newtheorem{cor}{Corollary}

\begin{document}

\title{Game semantics of the Safe $\lambda$-calculus}

\author{William Blum}

\institute{Oxford University, Computing Laboratory\\ Wolfson Building, Parks Road, Oxford, OX1 3QD, UK.\\
\email{William.Blum@comlab.ox.ac.uk}}
\maketitle

\vspace{0.5cm}

\section{The safety condition}

\noindent The \emph{safety condition} has been introduced in
\cite{KNU02} as a syntactic restriction for higher-order recursion
schemes (grammars) that constrains the order of the variables
occurring in the grammar equations. The authors of \cite{KNU02} were
able to prove that the Monadic Second Order (MSO) theory of the term
tree generated by a safe recursion scheme of any order is
decidable.\footnote{In fact it has been shown in \cite{OngLics2006}
that it is also true for unsafe recursion schemes.}

When transposed to the $\lambda$-calculus \cite{Barendregt84}, the
safety condition gives rise to the \emph{Safe $\lambda$-calculus}, a
strict sub-language of the $\lambda$-calculus. A first version
appeared in the technical report \cite{safety-mirlong2004}. We
propose a more general and simpler version where term types are not
required to be homogeneous \cite{blumtransfer}. A noteworthy feature
of the Safe $\lambda$-calculus is that no variable capture can occur
when performing substitution and therefore it is unnecessary to
rename variables when computing $\beta$-reductions.

Little is known about the Safe $\lambda$-calculus and there are many
 problems that have yet to be studied concerning its
computational power, the complexity classes that it characterises,
its interpretation under the Curry-Howard isomorphism, and its
game-semantic characterisation
\cite{abramsky:game-semantics-tutorial}. Our contribution concerns
the last problem.

\section{The correspondence theorem}

The difficulty in giving a game-semantic account of Safety lies in
the fact that it is a syntactic restriction whereas Game Semantics
is by essence a syntax-independent semantics. The solution consists
in finding a particular syntactical representation of terms on which
the plays of the game denotation can be represented.
 To achieve
this, we use ideas recently introduced in \cite{OngLics2006}: a term
is canonically represented by the abstract syntax tree of its
$\eta$-long normal form, referred as the \emph{computation tree}. A
computation is described by a justified sequence of nodes of the
computation tree respecting some formation rules and called a
\emph{traversal}. Traversals permit us to model $\beta$-reductions
without altering the structure of the computation tree via
substitution.
%
%A notable property is that the \emph{P-view} (in the game-semantic
%sense) of a traversal is a path in the computation tree.
%
We prove the following result:
\begin{thm}[Correspondence theorem]
\label{thm:corresp} The set of traversals of the computation tree is
isomorphic to the set of uncovered plays of the game denotation of
the term.
\end{thm}

In other words, traversals are just representations of the
uncovering of plays of the strategy denoting the term. By defining
an appropriate \emph{reduction} operation which eliminates traversal
nodes that are ``internal'' to the computation, we obtain an
isomorphism between the strategy denotation of a term and the set of
reductions of traversals over its computation tree.

\section{Game-semantic characterisation}

We introduce the notions of \emph{incrementally-justified
strategies} and \emph{incrementally-bound computation trees} and we
show that these two notions coincide using the Correspondence
Theorem. We have the following theorem:

\begin{thm}[Game-semantic characterisation of safety]
\label{thm:gamesem_charact} Safe simply-typed terms in
$\beta$-normal form have incrementally-bound computation trees.
Reciprocally, a closed term in $\eta$-long normal form with an
incrementally-bound computation trees is safe.
\end{thm}

Since pointers in the plays of \emph{incrementally-justified
strategies} are by definition uniquely reconstructible, we obtain
the following corollary:

\begin{cor}
\label{cor:safeptrrecover} The pointers in the game semantics of
safe simply-typed terms can be recovered uniquely from the
underlying sequences of moves.
\end{cor}

\section{Extension to Safe Idealized Algol}

We define Safe \textsf{IA} to be the Safe $\lambda$-calculus
augmented with the constants of Idealized Algol (\textsf{IA})
\cite{Reynolds81} as well as a family of combinators $Y_A$ for every
type $A$. We show that terms of the Safe \textsf{PCF}
\cite{DBLP:journals/tcs/Plotkin77} fragment are denoted by
incrementally-justified strategies and we give the key elements for
a possible extension to full Safe \textsf{IA}.

\bibliographystyle{plain}
\bibliography{prgconf}

\end{document}
