\documentstyle{ouclprgsc}
%
\begin{document}

\title{Hamiltonian Mechanics}

\author{Ivar Ekeland}

\institute{Oxford University, Computing Laboratory\\ Wolfson Building, Parks Road, Oxford, OX1 3QD, UK.\\
\email{Ivar.Ekeland@comlab.ox.ac.uk}}


\maketitle



\section{Fixed-Period Problems: The Sublinear Case}
%
With this chapter, the preliminaries are over, and we begin the search
for periodic solutions to Hamiltonian systems. All this will be done in
the convex case; that is, we shall study the boundary-value problem
\begin{eqnarray*}
  \dot{x}&=&JH' (t,x)\\
  x(0) &=& x(T)
\end{eqnarray*}
with $H(t,\cdot)$ a convex function of $x$, going to $+\infty$ when
$\left\|x\right\| \to \infty$.

%
\subsection{Autonomous Systems}
%
In this section, we will consider the case when the Hamiltonian $H(x)$
is autonomous. For the sake of simplicity, we shall also assume that it
is $C^{1}$.

We shall first consider the question of nontriviality, within the
general framework of
$\left(A_{\infty},B_{\infty}\right)$-subquadratic Hamiltonians. In
the second subsection, we shall look into the special case when $H$ is
$\left(0,b_{\infty}\right)$-subquadratic,
and we shall try to derive additional information.
%
\subsubsection{ The General Case: Nontriviality.}
%
We assume that $H$ is
$\left(A_{\infty},B_{\infty}\right)$-sub\-qua\-dra\-tic at infinity,
for some constant symmetric matrices $A_{\infty}$ and $B_{\infty}$,
with $B_{\infty}-A_{\infty}$ positive definite. Set:
\begin{eqnarray}
\gamma :&=&{\rm smallest\ eigenvalue\ of}\ \ B_{\infty} - A_{\infty} \\
  \lambda : &=& {\rm largest\ negative\ eigenvalue\ of}\ \
  J \frac{d}{dt} +A_{\infty}\ .
\end{eqnarray}


We recall once more that by the integer part $E [\alpha ]$ of
$\alpha \in \bbbr$, we mean the $a\in \bbbz$
such that $a< \alpha \le a+1$. For instance,
if we take $a_{\infty} = 0$, Corollary 2 tells
us that $\overline{x}$ exists and is
non-constant provided that:

\begin{proof}
The spectrum of $\Lambda$ is $\frac{2\pi}{T} \bbbz +a_{\infty}$. The
largest negative eigenvalue $\lambda$ is given by
$\frac{2\pi}{T}k_{o} +a_{\infty}$,
where
\begin{equation}
  \frac{2\pi}{T}k_{o} + a_{\infty} < 0
  \le \frac{2\pi}{T} (k_{o} +1) + a_{\infty}\ .
\end{equation}
Hence:
\begin{equation}
  k_{o} = E \left[- \frac{T}{2\pi} a_{\infty}\right] \ .
\end{equation}

The condition $\gamma < -\lambda < \delta$ now becomes:
\begin{equation}
  b_{\infty} - a_{\infty} <
  - \frac{2\pi}{T} k_{o} -a_{\infty} < \omega -a_{\infty}
\end{equation}
which is precisely condition (\ref{eq:three}).\qed
\end{proof}
%




Now if $\widetilde{x}$ has a lower period, $T/k$ say,
we would have, by Corollary 31:
\begin{equation}
  i_{T} (\widetilde{x} ) =
  i_{kT/k}(\widetilde{x} ) \ge
  ki_{T/k} (\widetilde{x} ) + k-1 \ge k-1 \ge 1\ .
\end{equation}


%
\paragraph{Notes and Comments.}
The results in this section are a
refined version of \cite{clar:eke};
the minimality result of Proposition
14 was the first of its kind.


\begin{table}
\caption{This is the example table taken out of {\it The
\TeX{}book,} p.\,246}
\vspace{2pt}
\begin{tabular}{r@{\quad}rl}
\hline
\multicolumn{1}{l}{\rule{0pt}{12pt}
                   Year}&\multicolumn{2}{l}{World population}\\[2pt]
\hline\rule{0pt}{12pt}
8000 B.C.  &     5,000,000& \\
  50 A.D.  &   200,000,000& \\
1650 A.D.  &   500,000,000& \\
1945 A.D.  & 2,300,000,000& \\
1980 A.D.  & 4,400,000,000& \\[2pt]
\hline
\end{tabular}
\end{table}
%


\paragraph{Notes and Comments.}
The first results on subharmonics were
obtained by Rabinowitz in \cite{rab}, who showed the existence of
infinitely many subharmonics both in the subquadratic and superquadratic
case, with suitable growth conditions on $H'$. Again the duality
approach enabled Clarke and Ekeland in \cite{clar:eke:2} to treat the
same problem in the convex-subquadratic case, with growth conditions on
$H$ only.

Recently, Michalek and Tarantello (see \cite{mich:tar} and \cite{tar})
have obtained lower bound on the number of subharmonics of period $kT$,
based on symmetry considerations and on pinching estimates, as in
Sect.~5.2 of this article.

%
% ---- Bibliography ----
%
\begin{thebibliography}{5}
%
\bibitem {clar:eke}
Clarke, F., Ekeland, I.:
Nonlinear oscillations and
boundary-value problems for Hamiltonian systems.
Arch. Rat. Mech. Anal. {\bf 78} (1982) 315--333
%
\bibitem {clar:eke:2}
Clarke, F., Ekeland, I.:
Solutions p\'{e}riodiques, du
p\'{e}riode donn\'{e}e, des \'{e}quations hamiltoniennes.
Note CRAS Paris {\bf 287} (1978) 1013--1015
%
\bibitem {mich:tar}
Michalek, R., Tarantello, G.:
Subharmonic solutions with prescribed minimal
period for nonautonomous Hamiltonian systems.
J. Diff. Eq. {\bf 72} (1988) 28--55
%
\bibitem {tar}
Tarantello, G.:
Subharmonic solutions for Hamiltonian
systems via a $\bbbz_{p}$ pseudoindex theory.
Annali di Matematica Pura (to appear)
%
\bibitem {rab}
Rabinowitz, P.:
On subharmonic solutions of a Hamiltonian system.
Comm. Pure Appl. Math. {\bf 33} (1980) 609--633
\end{thebibliography}
%
\end{document}
