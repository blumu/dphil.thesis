\documentclass[12pt]{article}

\newcommand\ialgol{\textsf{IA}}
\newcommand\pcf{\textsf{PCF}}
\newtheorem{thm}{Theorem}
\pagestyle{empty}

\begin{document}
\begin{center}
{\Large \bf Game semantics of the Safe $\lambda$-calculus}
\vspace{0.2cm}

{\small Abstract for the PRG Conference 2006} \vspace{0.4cm}

William Blum \vspace{0.4cm}

\today \vspace{0.5cm}
\end{center}


The \emph{safety condition} has been introduced in \cite{KNU02} as a
syntactic restriction for higher-order recursion schemes (grammars)
that constrains the order of the variables occurring in the grammar
equations. The authors were able to prove that the Monadic Second
Order (MSO) theory of the term tree generated by a safe recursion
scheme of any order is decidable\footnote{In fact it has been shown
recently in \cite{OngLics2006} that it is also true for unsafe
recursion schemes.}.

When transposed to the $\lambda$-calculus, the safety condition
gives rise to the \emph{Safe $\lambda$-calculus}, a strict
sub-language of the $\lambda$-calculus. A first version appeared in
the technical report \cite{safety-mirlong2004}. We propose a more
general and simpler version where term types are not required to be
homogeneous (\cite{blumtransfer}). A noteworthy feature of the Safe
$\lambda$-calculus is that no variable capture can occur when
performing substitution and therefore it is unnecessary to rename
variables when computing $\beta$-reductions.

Little is known about the Safe $\lambda$-calculus and there are many
 problems that have yet to be studied concerning its
computational power, the complexity classes that it characterises,
its interpretation under the Curry-Howard isomorphism and its
repercussion on game semantics
(\cite{abramsky:game-semantics-tutorial}). We have contributed to
give an answer to the latter by proving the following result:
\begin{thm}
\label{thm:safeptrrecover} The pointers in the game semantics of
safe simply-typed terms can be recovered uniquely from the
underlying sequences of moves.
\end{thm}

Giving a game-semantic account of Safety is complicated by the fact
that it is a syntactic restriction whereas Game Semantics is by
essence a syntax-independent semantics. The solution consists in
making an explicit correspondence between some syntactical
representation of a term and its game denotation. Our approach is
based on ideas recently introduced in \cite{OngLics2006}: a term is
represented by the abstract syntax tree of its $\eta$-long normal
form which we call \emph{computation tree}. A computation is
represented by a justified sequence of nodes of the computation tree
respecting some formation rules and called \emph{traversal}. A
traversal permits to achieve a \emph{local computation} of
$\beta$-reductions as opposed to global approaches that compute
$\beta$-redexes by performing substitutions. A notable property is
that the \emph{P-view} (in the game-semantic sense) of a traversal
is a path in the computation tree.

We prove the \emph{Correspondence Theorem} stating that traversals
are just representations of the uncovering of plays of the game
denotation of the term. The nodes of the traversals that are
``internal'' to the computation are eliminated by an appropriate
\emph{reduction} operation. This leads to an isomorphism between the
strategy denotation of a term and the set of reductions of
traversals over its computation tree.
% This put us in a position to
%investigate the impact of a syntactic restriction over Game
%Semantics.

To prove Theorem \ref{thm:safeptrrecover} we first introduce the
notion of \emph{incrementally-justified strategies} and show that
pointers are uniquely reconstructible for such strategies. We then
introduce the notion of \emph{incrementally-bound computation trees}
and prove, using the Correspondence Theorem, that
incremental-binding coincides with incremental-justification.
Finally, we show that safe simply-typed terms in $\beta$-normal form
have incrementally-bound computation trees.

We define Safe \ialgol\ to be the Safe $\lambda$-calculus augmented
with the constants of Idealized Algol (\ialgol) as well as a family
of combinators $Y_A$ for every type $A$. We show that terms of the
Safe \pcf\ fragment are denoted by incrementally-justified
strategies and we give the key elements for a possible extension to
full Safe \ialgol.

\bibliographystyle{plain}
\bibliography{prgconf}

\end{document}
