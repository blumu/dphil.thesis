\documentclass[12pt]{article}
\usepackage{natbib}

\author{William Blum}
\title{Game semantics of the Safe $\lambda$-calculus\\ {\small Abstract for the PRG Conference 06}}
\begin{document}
\maketitle

\section{Safe $\lambda$-calculus}

My work is concerned with the study of the ``Safe
$\lambda$-calculus'', a strict restriction of the $\lambda$-calculus
where terms are required to satisfy a property called \emph{safety}.
The safety condition has been introduced in \cite{KNU02} as a
syntactic restriction for higher-order recursion schemes (grammars).
In their paper they proved that the MSO theory of the term tree
generated by a safe recursion scheme of level $n$ is decidable (it
has been shown more recently in \cite{OngLics2006} that the safety
assumption is in fact not necessary for the decidability of MSO
theories).

Transposing the safety condition to the context of the
$\lambda$-calculus gives rise to the Safe $\lambda$-calculus. A
first definition occurred in the technical report
\cite{safety-mirlong2004}. One interesting property of this calculus
is that it is unnecessary to rename variables when performing
$\beta$-reduction. In other words, no variable capture can occur
during substitution.

I have proposed a more general notion of safety (where
type-homogeneity is not assumed) for which this ``no variable
capture'' property still hold.

\section{Game-semantic characterisation}

The nature of this calculus is not well known. We propose to
investigate the effect of the safety restriction on the game
semantics of a term.


Inspired by my reading on game semantics
\citep{abramsky:game-semantics-tutorial} and by the techniques
developed by Luke Ong in \citep{OngLics2006}, I have proved a result
on the game semantics of safe terms: the pointers in the game
semantics of safe simply-typed terms can be recovered uniquely from
the sequence of moves. This result is similar to the standard result
in game semantics which says that pointers of strategies can be
recovered uniquely for arena of order 2 at most.


\bibliographystyle{plainnat}
\bibliography{../transfer/gamesem,../transfer/modelchecking,../transfer/proganalys,../transfer/higherorder}

\end{document}
