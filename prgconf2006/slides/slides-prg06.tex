%&slides-prg06
\ifx\preambleread\undefined
    \input preamble.tex
\fi



\newcommand\rulef[2]{\frac{\dps #1}{#2}}
\newcommand{\rulename}[1]{\textcolor{blue}{(#1)} }
\newcommand\dps{\displaystyle}
\newcommand\ord[1]{{\sf ord}(#1)}
\newcommand\union{\cup}
\newcommand{\typear}{\rightarrow}
\newcommand{\tree}[2][levelsep=4ex]{\pstree[levelsep=4ex,#1]{\TR{#2}}}
\newcommand{\bkptr}[2][nodesep=0pt]{\ncarc[offset=-2pt,nodesep=0pt,ncurv=1,arcangleA=-#2, arcangleB=-#2,#1]{->}}
\newcommand{\lsem}{[\![} % \llbracket
\newcommand{\rsem}{]\!]} % \rrbracket
\newcommand{\sem}[1]{{\lsem #1 \rsem}}
\newcommand\travset{\mathcal{T}rav}

\usepackage{pst-tree}


\title{Game semantics of the Safe $\lambda$-calculus}
\author[W. Blum]{William Blum}

\institute[University Of Oxford]{ \color{red}{Oxford University Computing Laboratory}
\vspace{20pt}

\textcolor{black}{\small PRG Student Conference}}
\date{\small 13 October 2006}



\begin{document}

\frame{\titlepage}

\section<presentation>*{Outline}

%\begin{frame}
%  \frametitle{Outline}
%  \tableofcontents[part=1]
%\end{frame}

%\AtBeginSection[] {
%   \begin{frame}<beamer>
%     \frametitle{Outline}
%     \tableofcontents[currentpart,currentsection]
%   \end{frame}
% }
%

\part<presentation>{Main Talk}


\section{The safety condition}

\frame{
\frametitle{Overview}
\begin{itemize}
\item
\end{itemize}
}

\frame{
\frametitle{Outline for this talk}
\begin{itemize}
\item The safety constraint
\item Definition of the Safe $\lambda$-calculus
\item Game semantics
\item Computation trees, traversals
\item The correspondence theorem
\item Application:
\end{itemize}
}

\frame{
\frametitle{Types}
\begin{itemize}
\item Simple types: $A := o\ |\ A \rightarrow A$
\item The order of a type is given by $\textsf{order}(o) = 0$,
$\textsf{order}(A \rightarrow B) = \max(\textsf{order}(A) + 1; \textsf{order}(B))$.
\end{itemize}
}

% ==================================================================================

\begin{frame}[fragile]
\frametitle{The Safe $\lambda$-calculus}
Jugdements of the form $ \Gamma \vdash M : T $ where $\Gamma$ is the context, $M$ is the term and $T$ is the type.

$$ \rulename{var} \   \rulef{}{x : A\vdash x : A}
\quad  \rulename{wk} \   \rulef{\Gamma \vdash M : A}{\Delta \vdash M : A} \ \Gamma \subset \Delta$$

$$ \rulename{app} \  \rulef{\Gamma \vdash M : (A,\ldots,A_l,B)
                                        \quad \Gamma \vdash N_1 : A_1
                                        \quad \ldots \quad \Gamma \vdash N_l : A_l  }
                                   {\Gamma  \vdash M N_1 \ldots N_l : B}$$
\hfill with  $\textcolor{red}{\forall y \in \Gamma : \ord{y} \geq \ord{B}}$

$$ \rulename{abs} \   \rulef{\Gamma \union \overline{x} : \overline{A} \vdash M : B}
                                   {\Gamma  \vdash \lambda \overline{x} : \overline{A} . M : (\overline{A},B)} \quad
                                   \textcolor{red}{\forall y \in \Gamma : \ord{y} \geq \ord{\overline{A},B}}$$

%    \small
% \begin{columns}
%   \column{7cm}
%\vspace{0.4cm}
%  \column{4.3cm}
%\begin{block}{Example}
%\end{block}
%\vspace{0.1cm}
%\end{columns}
\end{frame}

%%%%%%%%%%%%%%%%%%%%%%%%%%%%%%%%%%%%%%%%%%%%%%%%%


%%%%%%%%%%%%%%%%%%%%%%%%%%%%%%%%%%%%%%%%%%%%%%%%%


%%%%%%%%%%%%%%%%%%%%%%%%%%%%%%%%%%%%%%%%%%%%%%%%%

\section{Computation trees and traversals}

%%%%%%%%%%%%%%%%%%%%%%%%%%%%%%%%%%%%%%%%%%%%%%%%%


\frame{ \frametitle{Computation trees and traversals}
\textcolor{blue}{Computation tree} = tree representation of the $\eta$-long normal form of a term.

Take $M \equiv \lambda f z . (\lambda g x . f x) (\lambda y. y) z$ of type $(o \typear o) \typear o \typear o$.
Its $\eta$-long normal form is $\lambda f z . (\lambda g x . f x) (\lambda y. y) (\lambda .z)$).
Its computation tree is
\begin{center}
$\tree{\lambda f z}
{ \tree{@}
    {
        \tree{\lambda g x}
            {  \tree{f}{  \tree{\lambda}{\TR{x}}}
            }
        \tree{\lambda y}{\TR{y}}
        \tree{\lambda}{\TR{z}}
    }
}$
\end{center}

Traversals:

The set of traversals of the computation tree of $M$ is $\travset(M)$

Reduction of traversals.

}

\frame{
\frametitle{Game semantics}
}

\frame{
\frametitle{Correspondence Theorem (example)}
It is possible to define a
a partial function $\textcolor{green}{\varphi}$ from the nodes of the \textcolor{blue}{computation tree}
to the moves of the \textcolor{red}{arena}.
\begin{center}
$\textcolor{blue}{
\tree{ \Rnode{root} {\lambda f z w}^{[1]} }
     {  \tree{@^{[2]}}
        {   \tree{\lambda g x ^{[3]}}
                {
                 \tree{\Rnode{f2}{f^{[6]}}} {\tree{\Rnode{lmd2}\lambda^{[7]}}{\TR{x^{[8]}}}}
                }
            \tree{\lambda y ^{[4]}}{\TR{y}}
            \tree{\lambda ^{[5]}}{\TR{\Rnode{z}z}}
        }
    }}
\hspace{3cm}
\textcolor{red}{
  \tree[levelsep=12ex]{ \Rnode{q1}q^1 }
    {   \pstree[levelsep=4ex]{\TR{\Rnode{q3}q^3}}{\TR{\Rnode{q4}q^4}}
        \TR{\Rnode{q2}q^2}
        \TR{\Rnode{q5}q^5}
    }
}
\psset{linecolor=green,nodesep=1pt,arrows=->,arcangle=-20,arrowsize=2pt 1,linewidth=0.3pt}
\ncline{->}{root}{q1} \aput*{:U}{\textcolor{green}{\varphi}}
\ncarc{->}{z}{q2}
\ncline{->}{f}{q3}
\ncline{->}{lmd}{q4}
\ncline{->}{f2}{q3}
\ncline{->}{lmd2}{q4}
$\end{center}

The following justified sequence of moves $s \in \sem{M}$:
\vspace{0.2cm}
\begin{center}
 $s =
\rnode{q1}{q}^1\ \rnode{q3}{q}^3\ \rnode{q4}{q}^4\ \rnode{q2}{q}^2
\bkptr[offset=-3pt]{60}{q3}{q1}
\bkptr[offset=-3pt]{60}{q4}{q3}
\bkptr[offset=-3pt,ncurv=0.5]{60}{q2}{q1}
\in \sem{M}$
\end{center}

there is a justified sequence of nodes $r$ such that $s = \varphi(r)$:
\vspace{0.1cm}
\begin{center}
$r =
\rnode{q1}{\lambda f z} \cdot
\rnode{q3}{f}^{[6]} \cdot
\rnode{q4}{\lambda^{[7]}} \cdot
\rnode{q2}{z}
\bkptr[ncurv=1]{60}{q3}{q1}
\bkptr[ncurv=1]{50}{q4}{q3}
\bkptr[ncurv=0.4]{80}{q2}{q1}$
\end{center}
such that $r$ is the reduction of a traversal: \vspace*{1cm}
\begin{center}
$t = \rnode{q1}{\lambda f
z w} \cdot \rnode{n2}{@^{[2]}} \cdot \rnode{n3}{\lambda g x^{[3]}}
\cdot \rnode{q3}{f}^{[6]} \cdot \rnode{q4}{\lambda^{[7]}} \cdot
\rnode{n8}{x^{[8]}} \cdot \rnode{n9}{\lambda^{[5]}} \cdot
\rnode{q2}{z} \bkptr[ncurv=0.6]{60}{q3}{q1}
\bkptr[ncurv=1]{60}{q4}{q3}
\bkptr[ncurv=0.4]{80}{q2}{q1}
\bkptr[ncurv=0.4]{60}{n3}{n2} \bkptr[ncurv=0.4]{60}{n8}{n3}
\bkptr[ncurv=0.4]{60}{n9}{n2}.
$
\end{center}

\note{
By representing side-by-side the computation tree and the type arena of a term, we
see that (some) nodes of the computation tree can be mapped to question moves of the arena.
}
}

%%%%%%%%%%%%%%%%
\section{Game-semantic characterisation of Safety}


\frame{ \frametitle{Game-semantic characterisation of Safety}

}


%%%%%%%%%%%%%%%%%%%%%%%%%%%%%


\frame{ \frametitle{Results}

}


\section*{Summary}

\frame{ \frametitle{Conclusion and Further Work}

\begin{itemize}
\item
\item
\item
\end{itemize}
}

\subsection<presentation>*{Bibliography}

\begin{frame}
  \frametitle<presentation>{Bibliography}

  \begin{thebibliography}{10}
  \beamertemplatearticlebibitems
  \bibitem{jones01}
    Chin~Soon~Lee, Neil~D.~Jones, and Amir~Ben-Amram.
    \newblock The Size-Change Principle for Program Termination.
    \newblock {\em Principles of Programming Languages}, pp. 81-92. Volume 28 of Principles of Programming Languages. ACM press 2001
  \end{thebibliography}
\end{frame}

\end{document}
