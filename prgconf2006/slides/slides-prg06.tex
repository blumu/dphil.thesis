\ifx\incrcompilation\undefined \input preamble.tex \fi

\def\highlight{\textcolor{blue}}


\title{\texorpdfstring{Game semantics of the Safe $\lambda$-calculus}{Game semantics of the Safe lambda-calculus}}
\author[W. Blum]{William Blum}

\institute[University Of Oxford]{ \color{red}{Oxford University Computing Laboratory}
\vspace{20pt} 

\textcolor{black}{\small PRG Student Conference}}
\date{\small 13 October 2006}
 

\begin{document}

\frame{\titlepage}

%\section<presentation>*{Outline}
%\begin{frame}
%  \frametitle{Outline}
%  \tableofcontents[part=1]
%\end{frame}
%\AtBeginSection[] {
%   \begin{frame}<beamer>
%     \frametitle{Outline}
%     \tableofcontents[currentpart,currentsection]
%   \end{frame}
% }
%
%\part<presentation>{Main Talk}

%%%%%%%%%%%%%%%%%%%%%%%%%%%%%%%%%%%%%%%%%%%%%%%%%

\frame{
\frametitle{Overview}
\begin{itemize}
\item \textcolor{blue}{Safe $\lambda$-calculus} is a sub-language of the simply typed $\lambda$-calculus.
A term is safe if it satisfies some constraints on the order of the variables occurring in it.
\pause

\item \textcolor{blue}{Game semantics} provides an abstract model for the $\lambda$-calculus.
\pause
\end{itemize}

\begin{block}{}
We want to study the game model of the Safe $\lambda$-calculus.
\end{block}
}

%%%%%%%%%%%%%%%%%%%%%%%%%%%%%%%%%%%%%%%%%%%%%%%%%
 
\frame{
\frametitle{Outline for this talk}
\begin{enumerate}
\item The simply typed $\lambda$-calculus
\item The Safe $\lambda$-calculus
\item Game semantics
\item The correspondence theorem
\item Game semantic-characterisation of Safety
\end{enumerate}
}

%%%%%%%%%%%%%%%%%%%%%%%%%%%%%%%%%%%%%%%%%%%%%%%%%

\frame{
\frametitle{Simply  typed $\lambda$ calculus}
\begin{itemize}
\item \highlight{Simple types} $A := o\ |\ A \rightarrow A$.
%We write $(A_1,\ldots, A_n)$ for $A_1\rightarrow \ldots \rightarrow A_n$.
\pause
\item The \highlight{order} of a type is given by $\textsf{order}(o) = 0$,
$\textsf{order}(A \rightarrow B) = \max(\textsf{order}(A) + 1; \textsf{order}(B))$.
\pause
\item Jugdements of the form $ \Gamma \vdash M : T $ where $\Gamma$ is the context, $M$ is the term and $T$ is the type.
\pause
\item The formation rules:
$$ \rulename{var} \   \rulef{}{x : A\vdash x : A}
\qquad  \rulename{wk} \   \rulef{\Gamma \vdash M : A}{\Delta \vdash M : A} \ \Gamma \subset \Delta$$
$$ \rulename{app} \  \rulef{\Gamma \vdash M : A \rightarrow B \quad \Gamma \vdash N : A }
                           {\Gamma  \vdash M N : B}
\quad \rulename{abs} \   \rulef{\Gamma, x : A \vdash M : B}
                                {\Gamma  \vdash \lambda x^A. M : A \rightarrow B}$$
\pause
\item Example: $f:o\rightarrow o\rightarrow o, x:o \vdash (\lambda \varphi^{o \rightarrow o} x^o . \varphi x) (f x)$
\end{itemize}
}

%%%%%%%%%%%%%%%%%%%%%%%%%%%%%%%%%%%%%%%%%%%%%%%%%
 
\begin{frame}
\frametitle{The Safe $\lambda$-calculus}
\begin{block}{The rules}
$$ \rulename{var} \   \rulef{}{x : A\vdash x : A}
\qquad  \rulename{wk} \   \rulef{\Gamma \vdash M : A}{\Delta \vdash M : A} \ \Gamma \subset \Delta$$
$$ \rulename{app} \  \rulef{\Gamma \vdash M : (A,\ldots,A_l,B)
                                        \quad \Gamma \vdash N_1 : A_1
                                        \quad \ldots \quad \Gamma \vdash N_l : A_l  }
                                   {\Gamma  \vdash M N_1 \ldots N_l : B}$$
\hfill with the side-condition $\textcolor{red}{\forall y \in \Gamma : \ord{y} \geq \ord{B}}$
$$ \rulename{abs} \   \rulef{\Gamma, x_1:A_1 \ldots x_n : A_n \vdash M : B}
                                   {\Gamma  \vdash \lambda x_1:A_1 \ldots x_n : A_n . M : A_1 \rightarrow \ldots \rightarrow A_n \rightarrow B}$$
\hfill with the side-condition $\textcolor{red}{\forall y \in \Gamma : \ord{y} \geq \ord{A_1 \rightarrow \ldots \rightarrow A_n \rightarrow B}}$
\end{block} 

\begin{block}{Property} 
There is no variable capture in the Safe $\lambda$-calculus when performing $\beta$-reduction.
\end{block} 
\end{frame}


%%%%%%%%%%%%%%%%%%%%%%%%%%%%%%%%%%%%%%%%%%%%%%%%%

\frame{
\frametitle{Game semantics}
}

%%%%%%%%%%%%%%%%%%%%%%%%%%%%%%%%%%%%%%%%%%%%%%%%%

\def\highlightat#1#2{\expandafter\alt<#1->{\textcolor{DarkGreen}{#2}}{#2}}
\def\visibleat#1#2{\expandafter\visible<#1->{#2}}

\frame{ \frametitle{Computation trees and traversals}

%    \small
% \begin{columns}
%   \column{7cm}
%\vspace{0.4cm}
%  \column{4.3cm}
%\begin{block}{Example}
%\end{block}
%\vspace{0.1cm}
%\end{columns}

\textcolor{blue}{Computation tree} = tree representation of the $\eta$-long normal form of a term.

\uncover<2->{
\textcolor{DarkGreen}{Traversal} = justified sequence of nodes respecting some formation rules.
}

\uncover<11->{
\textcolor{red}{Traversal reduction} is obtained by removing the nodes not hereditarily justified by the root of the tree.
}

\vspace{0.5cm}
\textcolor{blue}{Example} $M \equiv \lambda f z . (\lambda g x . f x) (\lambda y. y) z$ of type $(o \typear o) \typear o \typear o$.
\vspace{0.5cm}
\begin{columns}
\column{4cm}
\textcolor{blue}{
$ \tree{\highlightat{3}{\lambda f z} }
{ \tree{\highlightat{4}{@}}
    {
        \tree{\highlightat{5}{\lambda g x}}
            {  \tree{\highlightat{6}{f}}{  \tree{\highlightat{7}{\lambda}}{\TR{\highlightat{8}{x}}}}
            }
        \tree{\lambda y}{\TR{y}}
         \tree{\highlightat{9}{\lambda}}{\TR{\highlightat{10}{z}}}
    }
}$
}

\column{7cm}
$\textcolor{DarkGreen}{
\visible<2->{t } =
\visible<3->{\rnode{q1}{\lambda f z}}
\visible<4->{\ \cdot\ \rnode{n2}{@} \bkptr[ncurv=0.6,linecolor=DarkGreen]{60}{n2}{q1} }
\visible<5->{\ \cdot\ \rnode{n3}{\lambda g x} \bkptr[ncurv=0.6,linecolor=DarkGreen]{60}{n3}{n2} }
\visible<6->{\ \cdot\ \rnode{q3}{f}  \bkptr[ncurv=0.6,linecolor=DarkGreen]{50}{q3}{q1} }
\visible<7->{\ \cdot\ \rnode{q4}{\lambda} \bkptr[ncurv=0.6,linecolor=DarkGreen]{60}{q4}{q3} } 
\visible<8->{\ \cdot\ \rnode{n8}{x}  \bkptr[ncurv=0.4,linecolor=DarkGreen]{60}{n8}{n3} }
\visible<9->{\ \cdot\ \rnode{n9}{\lambda} \bkptr[ncurv=0.4,linecolor=DarkGreen]{60}{n9}{n2} }
\visible<10->{\ \cdot\ \rnode{q2}{z} \bkptr[ncurv=0.4,linecolor=DarkGreen]{80}{q2}{q1}}
}$
 
\vspace*{0.8cm}
\visible<11->{
$\textcolor{Red}{
t \upharpoonright r = \rnode{q1}{\lambda f z}
\cdot \rnode{q3}{f} \bkptr[ncurv=1,linecolor=red]{50}{q3}{q1}
\cdot \rnode{q4}{\lambda} \bkptr[ncurv=1,linecolor=red]{50}{q4}{q3}
\cdot \rnode{q2}{z}  \bkptr[ncurv=0.7,linecolor=red]{80}{q2}{q1}
}$
}
\vspace*{0.8cm}
\visible<12->{
$t - @ = 
\rnode{q1}{\lambda f z}
\ \cdot\ \rnode{n3}{\lambda g x} \bkptr[ncurv=0.6,linecolor=DarkGreen]{60}{n3}{q1}
\ \cdot\ \rnode{q3}{f}  \bkptr[ncurv=0.6,linecolor=DarkGreen]{50}{q3}{q1}
\ \cdot\ \rnode{q4}{\lambda} \bkptr[ncurv=0.6,linecolor=DarkGreen]{60}{q4}{q3}
\ \cdot\ \rnode{n8}{x}  \bkptr[ncurv=0.4,linecolor=DarkGreen]{60}{n8}{n3}
\ \cdot\ \rnode{n9}{\lambda} \bkptr[ncurv=0.4,linecolor=DarkGreen]{60}{n9}{q1}
\ \cdot\ \rnode{q2}{z} \bkptr[ncurv=0.4,linecolor=DarkGreen]{80}{q2}{q1}
$}
\end{columns}
}

%%%%%%%%%%%%%%%%%%%%%%%%%%%%%%%%%%%%%%%%%%%%%%%%%

\frame{ \frametitle{The Correspondence Theorem}
Let $M$ be a pure simply typed term of type $T$.
\begin{itemize}
\item $\textcolor{blue}{\travset(M)}$ = set of traversals of the computation tree of $M$
\item $\textcolor{blue}{\travset(M)}^{\upharpoonright r} = \{ t \upharpoonright r \ | \  t \in \travset(M) \}$
\item $\textcolor{blue}{\travset(M)}^{-@} = \{ t - @ \ | \  t \in \travset(M) \}$
\item $\textcolor{red}{\sem{M}}$ = game-semantic denotation of the term $M$
\item $\textcolor{red}{\intersem{M}}$ = revealed game-semantic denotion (i.e. internal moves are not hidden)
\end{itemize}

There exists a partial function $\textcolor{DarkGreen}{\varphi}$ from the nodes of the \textcolor{blue}{computation tree}
to the moves of the \textcolor{red}{arena \sem{T}} such that
$$ \textcolor{DarkGreen}{\varphi}  : \textcolor{blue}{\travset(M)}^{-@} \textcolor{DarkGreen}{\stackrel{\cong}{\longrightarrow}} \textcolor{red}{\intersem{M}} $$
$$ \textcolor{DarkGreen}{\varphi}  : \textcolor{blue}{\travset(M)}^{\upharpoonright r}  \textcolor{DarkGreen}{\stackrel{\cong}{\longrightarrow}} \textcolor{red}{\sem{M}}.$$
}

%%%%%%%%%%%%%%%%%%%%%%%%%%%%%%%%%%%%%%%%%%%%%%%%%

\frame{
\frametitle{Correspondence Theorem (example)}
\begin{center}
$\textcolor{blue}{
\tree{ \Rnode{root} {\lambda f z} }
     {  \tree{@}
        {   \tree{\lambda g x }
                {
                 \tree{\Rnode{f2}{f}} {\tree{\Rnode{lmd2}\lambda}{\TR{x}}}
                }
            \tree{\lambda y }{\TR{y}}
            \tree{\lambda}{\TR{\Rnode{z}z}}
        }
    }}
\hspace{3cm}
\textcolor{red}{
  \tree[levelsep=12ex]{ \Rnode{q1}q^1 }
    {   \pstree[levelsep=4ex]{\TR{\Rnode{q3}q^3}}{\TR{\Rnode{q4}q^4}}
        \TR{\Rnode{q2}q^2}
        \TR{\Rnode{q5}q^5}
    }
}
\psset{linecolor=DarkGreen,nodesep=1pt,arrows=->,arcangle=-20,arrowsize=2pt 1,linewidth=0.3pt}
\ncline{->}{root}{q1} \aput{:U}{\textcolor{DarkGreen}{\varphi}}
\ncarc{->}{z}{q2}
\ncline{->}{f}{q3}
\ncline{->}{lmd}{q4}
\ncline{->}{f2}{q3}
\ncline{->}{lmd2}{q4}
$\end{center}

\vspace{1cm}
Take the traversal $t = \rnode{q1}{\lambda f z} \cdot
\rnode{n2}{@} \cdot \rnode{n3}{\lambda g x} \cdot
\rnode{q3}{f} \cdot \rnode{q4}{\lambda} \cdot
\rnode{n8}{x} \cdot \rnode{n9}{\lambda} \cdot
\rnode{q2}{z} \bkptr[ncurv=0.6]{60}{q3}{q1}
\bkptr[ncurv=1]{60}{q4}{q3} \bkptr[ncurv=0.4]{80}{q2}{q1}
\bkptr[ncurv=0.4]{60}{n3}{n2} \bkptr[ncurv=0.4]{60}{n8}{n3}
\bkptr[ncurv=0.4]{60}{n9}{n2}. $

\vspace{0.5cm}
The play $\rnode{q1}{q}^1\
\rnode{q3}{q}^3\ \rnode{q4}{q}^4\ \rnode{q2}{q}^2
\bkptr[offset=-3pt]{60}{q3}{q1} \bkptr[offset=-3pt]{60}{q4}{q3}
\bkptr[offset=-3pt,ncurv=0.5]{60}{q2}{q1} \in \sem{M}$ 
is the image by $\varphi$ of the reduction of t : 
$t \upharpoonright r = \rnode{q1}{\lambda f z} \cdot
\rnode{q3}{f} \cdot \rnode{q4}{\lambda} \cdot
\rnode{q2}{z} \bkptr[ncurv=1]{60}{q3}{q1}
\bkptr[ncurv=1]{50}{q4}{q3} \bkptr[ncurv=0.4]{80}{q2}{q1}$.
%\note{
%By representing side-by-side the computation tree and the type arena of a term, we
%see that (some) nodes of the computation tree can be mapped to question moves of the arena.
%}
}


%%%%%%%%%%%%%%%%%%%%%%%%%%%%%%%%%%%%%%%%%%%%%%%%%


\frame{ \frametitle{Game-semantic characterisation of Safety}

}


%%%%%%%%%%%%%%%%%%%%%%%%%%%%%%%%%%%%%%%%%%%%%%%%%


\frame{ \frametitle{Results}

}


%%%%%%%%%%%%%%%%%%%%%%%%%%%%%%%%%%%%%%%%%%%%%%%%%

\frame{ \frametitle{Conclusion and Further Work}

\begin{itemize}
\item
\item
\item
\end{itemize}
}

%%%%%%%%%%%%%%%%%%%%%%%%%%%%%%%%%%%%%%%%%%%%%%%%%

\begin{frame}
  \frametitle<presentation>{Bibliography}

  \begin{thebibliography}{10}
  \beamertemplatearticlebibitems
  \bibitem{jones01}
    Chin~Soon~Lee, Neil~D.~Jones, and Amir~Ben-Amram.
    \newblock The Size-Change Principle for Program Termination.
    \newblock {\em Principles of Programming Languages}, pp. 81-92. Volume 28 of Principles of Programming Languages. ACM press 2001
  \end{thebibliography}
\end{frame}

\end{document}
