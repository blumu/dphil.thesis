\documentclass{article}

\documentclass[nocenter,sfbold]{thesis}
        
%%% preambule f  f                     

\usepackage[draft]{graphicx}
\usepackage{a4wide}
\usepackage{pst-tree}
\usepackage[all]{xy}
\usepackage{amsmath, amsthm, amssymb}
\usepackage{bbm,latexsym}
\usepackage{manfnt}
\usepackage{natbib}
\usepackage{url}
\usepackage{stmaryrd}
\usepackage{tabularx}
%\usepackage{floatflt}
%\usepackage{wrapfig}
\usepackage{picins}
\usepackage{bigcenter}
% \CompileMatrices  % this command causes problem with the \justseq macro


\newtheorem{thm}{Theorem}[section]
\newtheorem{cor}[thm]{Corollary}
\newtheorem{lem}[thm]{Lemma}
\newtheorem{prop}[thm]{Proposition}

\theoremstyle{remark}
\newtheorem{rem}[thm]{Remark}
\newtheorem{exmp}[thm]{Example}
\newtheorem{property}[thm]{Property}

\theoremstyle{definition}
\newtheorem{dfn}[thm]{Definition}
\newtheorem{algo}[thm]{Algorithm}


%\newcommand\textbfit[1]{{\bf\em #1}\index{#1}}
\newcommand\defeq{\stackrel{\textsf{def}}{=}}


% lambda calculus, reduction
\newcommand\seq[2]{{{#1} \vdash {#2}}}
\newcommand\typear{\rightarrow}
\newcommand{\rulename}[1]{\mathbf{(#1)}}
\newcommand\betared{\rightarrow_\beta}
\newcommand\betaredtr{\twoheadrightarrow_\beta} % transitive closure of the beta reduction
\newcommand\betasred{\rightarrow_{\beta_s}}
\newcommand\betasredO{\rightarrow_{\beta_s^1}}
\newcommand\betasredT{\rightarrow_{\beta_s^2}}
\newcommand\subst[2]{\left[ #1/#2 \right]}
%\newcommand\blambda{\hbox{\boldmath $\lambda$}}
%\newcommand\lterm[2]{{\blambda{#1}.{#2}}}

% Safe lambda calculus
\newcommand\funto{\longrightarrow}
\newcommand\ord[1]{{\sf ord}(#1)}
\newcommand\rank[1]{{\sf rank}(#1)}
\newcommand\order[1]{{\sf order}(#1)}
\newcommand\slheight[1]{{\sf height}(#1)}
\newcommand\nparam[1]{{\sf nparam}(#1)}
\newcommand\level[1]{{\sf level}(#1)}

% computation tree, eta normal form, traversals
\def\aux#1{\lceil #1\rceil}
\def\etanf#1{\eta_{\sf nf}(#1)}
\def\etabetanf#1{\eta\beta_{\sf nf}(#1)}
\newcommand\travset{\mathcal{T}rav}

% axiom and rules
\newcommand\dps{\displaystyle}
\newcommand\rulef[2]{\frac{\dps #1}{#2}}
\newcommand\rulefex[3][5pt]{\frac{\dps #2}{\stackrel{\rule{0pt}{#1}}{#3}}}
\newcommand\axiomf[1]{\frac{\dps}{#1}}

% set theory
\newcommand\inter{\cap}
\newcommand\union{\cup}
\newcommand\Union{\bigcup}
\newcommand\prefset{\textsf{Pref}}
\newcommand\sthat{\ | \ }  % ``such that'' for set defined by comprehension
\newcommand{\relimg}[1]{{(\!| #1 |\!)}}
\newcommand{\makeset}[1]{\{\,{#1}\,\}}
\newcommand\natbf{\mathbf{N}}
\newcommand\zset{\mathbb{Z}}
\newcommand\nat{\mathbb{N}}


% game semantics
%\newcommand\bot{\perp}
\newcommand\eval{\Downarrow}
\newcommand\obspre{\sqsubseteq}
\newcommand\obseq{\approx}

\newcommand{\lsem}{[\![} % \llbracket
\newcommand{\rsem}{]\!]} % \rrbracket
\newcommand{\sem}[1]{{\lsem #1 \rsem}}
\newcommand{\intersem}[1]{{\langle\!\langle #1 \rangle\!\rangle}}

\newcommand\intercomp{\fatsemi{^\|}}

% logic
\newcommand\imp{\Rightarrow}
\newcommand\zand{\wedge}

% ia
\newcommand\ialgol{\textsf{IA}}
\newcommand\ialgolmnew{\ialgol-$\{\ianew\}$}
\newcommand\iaseqcom{$\tt{seq_{com}}$}
\newcommand\iaseqexp{$\tt{seq_{exp}}$}
\newcommand\iaseq{\texttt{seq}}
\newcommand\iaskip{\texttt{skip}}
\newcommand\iaderef{\texttt{deref}}
\newcommand\iaassign{\texttt{assign}}
\newcommand\iadone{\texttt{done}}
\newcommand\iarun{\texttt{run}}
\newcommand\iawrite{\texttt{write}}
\newcommand\iaread{\texttt{read}}
\newcommand\iaok{\texttt{ok}}
\newcommand\iamkvar{\texttt{mkvar}}
\newcommand\ianew{\texttt{new}}
\newcommand{\ianewin}[1]{\texttt{new}\ #1\ \texttt{in}}
\newcommand\iabool{\texttt{bool}}
\newcommand\iawhile{\texttt{while}}
\newcommand\iado{\texttt{do}}
\newcommand\iacom{\texttt{com}}
\newcommand\iaexp{\texttt{exp}}
\newcommand\iavar{\texttt{var}}

% pcf
\newcommand\pcf{\textsf{PCF}}
\newcommand\pcfcond{\texttt{cond}}
\newcommand\pcfsucc{\texttt{succ}}
\newcommand\pcfpred{\texttt{pred}}

% trees
\newcommand{\SubTree}[2][]{\Tr[ref=t]{\pstribox[#1]{#2}}}
\newcommand{\SubTreeE}[2][]{\Tr[ref=t]{\pstribox[#1]{#2}}}
\def\dedge{\ncline[linestyle=dashed]}
\def\dotedge{\ncline[linestyle=dotted]}
\def\valueedge{\ncline[linestyle=dashed,linewidth=0.5pt]}
\newcommand{\TRV}[1][edge=\valueedge]{\TR[edge=\valueedge,#1]}
\newcommand{\tree}[2][levelsep=4ex]{\pstree[levelsep=4ex,#1]{\TR{#2}}}
\newcommand\treelabel[1]{\mput*{\mbox{{\small $#1$}}}}
% \tvput*{1}


% justified sequence of moves
\newcommand{\oview}[1]{\llcorner #1 \lrcorner}
\newcommand{\pview}[1]{\ulcorner #1 \urcorner}
%%\newcommand\jseq{\stackrel{\curvearrowleft}{=}} %equality of justseq

%   back pointer using psttricks
\newcommand{\bkptr}[2][nodesep=0pt]{\ncarc[offset=-2pt,nodesep=0pt,ncurv=1,arcangleA=-#2, arcangleB=-#2,#1]{->}}
%\newcommand{\bkptrb}[1][nodesep=0pt]{\nccurve[offset=-2pt,nodesep=0pt,ncurv=1,angleA=90,angleB=90,#1]{->}}
\newcommand{\bklabel}[1]{\mput*{\mbox{{\tiny $#1$}}}}
\newcommand{\bklabelb}[1]{\mput{\mbox{\tiny $#1$}}}
\newcommand{\bklabelc}[1]{\Bput[1pt]{\mbox{{\tiny $#1$}}}}

%   backpointer using xypic
%\newcommand{\justseq}[2][12pt]{\xymatrix @=#1@M=0pt{ #2 }}
%\newcommand{\pointto}[1]{\ar@/_/[#1]}
%\newcommand{\apointto}[1]{\ar@/_1pc/[#1]}


% model checking
\newcommand\entail{\vdash}

% todo symbol
\newcommand\todo{\textdbend}
\newcommand\todomargin[1]{\marginpar{\textdbend \begin{sloppypar} #1 \end{sloppypar}}}
\newcommand\todobox[1]{\colorbox{lightgray}{\parbox[h]{0.9\textwidth}{#1}}\marginpar[\textdbend]{\textdbend}}


\def\funto{\longrightarrow}
\newcommand\rank[1]{{\sf rank}(#1)}
\newcommand\order[1]{{\sf order}(#1)}
\newcommand\height[1]{{\sf height}(#1)}
\newcommand\nparam[1]{{\sf nparam}(#1)}
\def\nat{\mathbb{N}}

\title{Safe $\lambda$-Calculus}

\begin{document}

\maketitle

\section{Homogenous type}

Let $Types$ be the set of simple types generated by the grammar $A
\, ::= \, o \; | \; A \funsp A$. Any type different from the base
type $o$ can be written $(A_1, \cdots, A_n, o)$ for some $n \geq 1$,
which is a shorthand for $A_1 \funsp \cdots \funsp A_n \funsp o$ (by
convention, $\rightarrow$ associates to the right).

We suppose that a ranking function has been defined: ${\sf rank} :
Types \funto (L, \leq)$ where $(L, \leq)$ is any linearly ordered
set. Possible candidates for the ranking function are:
\begin{itemize}
\item ${\sf order} : Types \funto (\nat,\leq)$ with $\order{o} = 0$
and $\order{A \funsp B} = \max(\order{A}+1, \order{B})$.
\item ${\sf height} : Types \funto (\nat,\leq)$ with $\height{o} = 0$
and $\height{A \funsp B} = 1 + \max(\height{A}, \height{B})$.
\item ${\sf nparam} : Types \funto (\nat,\leq)$ with $\nparam{o} = 0$
and $\nparam{A_1, \cdots, A_n} = n$.
\item ${\sf ordernp} : Types \funto (\nat \times \nat,\leq)$ with $ {\sf ordern} (t)  = (\order{t}, \nparam{t})$ for $t \in Types$.
\end{itemize}


Following \cite{KNU02}, a type is rank-homogeneous if it is $o$ or
if it is $(A_1, \cdots, A_n, o)$ with the condition that $rank(A_1)
\geq rank(A_2)\geq \cdots \geq rank(A_n)$ and each $A_1$, \ldots,
$A_n$ is rank-homogeneous.



Suppose that $\overline{A_1}$, $\overline{A_2}$, \ldots,
$\overline{A_n}$ are $n$ lists of types, where $A_{ij}$ denotes the
$j^{th}$ type of list $\overline{A_i}$ and $l_i$ the size of
$\overline{A_i}$. Then the notation $A \; = \; (\overline{A_1} \, |
\, \cdots \, | \, \overline{A_r} \, | \, o)$ means that
\begin{itemize}
  \item $A$ is the type $(A_{11},A_{12},\cdots, A_{1l_1}, A_{21}, \cdots,A_{2l_2}, \cdots A_{n1},\cdots, A_{nl_n},o)$
  \item $\forall i: \forall u,v \in A_i : \rank u = \rank v $
  \item $\forall i,j . \forall u \in A_i . \forall v \in A_j . i<j \implies \rank u >
   \rank v $
\end{itemize}
Consequently, $A$ is rank-homogenous. This notation organises the
$A_{ij}$s into partitions according to their ranks. Suppose $B =
(\overline{B_1} \, | \, \cdots \, | \, \overline{B_m} \, | \, o)$.
We write $(\overline{A_1} \, | \, \cdots \, | \, \overline{A_n} \, |
\, {B})$ to mean
\[(\overline{A_1} \, | \, \cdots \, | \, \overline{A_n} \, | \,
\overline{B_1} \, | \, \cdots \, | \, \overline{B_m} \, | \, o).\]


\section{Safe $\lambda$-calculus -- correction of \cite{DBLP:conf/fossacs/AehligMO05}}

In the following we shall consider terms-in-context $\seq{\Gamma}{M
: A}$ of the simply-typed $\lambda$-calculus. Let $\Delta$ be a
simply-typed alphabet i.e., each symbol in $\Delta$ has a simple
type. We write $\terms{A}{\Delta}$ for the set of terms of type $A$
built up from the set $\Delta$ understood as constant symbols,
\emph{without} using $\lambda$-abstraction.


The \textbfit{Safe $\lambda$-Calculus} is a sub-system of the
simply-typed $\lambda$-calculus. Typing judgements (or
terms-in-context) are of the form
\begin{equation}
\nonumber \seq{\overline{x_1}:\overline{A_1} \, | \, \cdots \, | \,
\overline{x_n} :  \overline{A_n}}{M : B}
\end{equation}
which is shorthand for $\seq{x_{11} : A_{11}, \cdots, x_{1r}:
A_{1r}, \cdots}{M : B}$. \emph{Valid typing judgements} of the
system are defined by induction over the following rules, where
$\Delta$ is a given homogeneously-typed alphabet:
\[{{(\overline{A_1} \, | \, \cdots \, | \, \overline{A_n} \, | \, B)
\hbox{ homogeneous} \qquad {b : B} \; \in \; \Delta} \over
{\seq{\overline{x_1} : \overline{A_1}\, | \, \cdots\, | \,
\overline{x_n} : \overline{A_n}}{b : B}}}(\mbox{const})\]

\[{{(\overline{A_1}
\, | \, \cdots \, | \, \overline{A_n}) \hbox{ homogeneous}} \over
{\seq{\overline{x_1} : \overline{A_1}\, | \, \cdots\, | \,
\overline{x_n} : \overline{A_n}}{x_{ij} : A_{ij}}}}(\mbox{var})\]

\[
{ {\seq{\overline{x_1} : \overline{A_1}\, | \, \cdots\, | \,
\overline{x_{n+1}} : \overline{A_{n+1}}}{M : B}} \qquad
\rank{\overline{A_{n+1}}} \geq \rank{B} \over {\seq{\overline{x_1} :
\overline{A_1}\, | \, \cdots\, | \, \overline{x_{n}} :
\overline{A_{n}}}{\lterm{\overline{x_{n+1}} : \overline{A_{n+1}}}{M}
: (\overline{A_{n+1}} \, | \, B)}} } (\mbox{$\lambda$-abs})\]

\[ {{\seq{\Gamma}{M : (\overline{B_1} \, | \, \cdots \, | \, \overline{B_m} \, | \, o)} \qquad
\seq{\Gamma}{N_1 : B_{11}} \quad \cdots \quad \seq{\Gamma}{N_{l_1} :
B_{1l_1}} }\over{ \seq{\Gamma}{M N_1 \cdots N_{l_1} :
(\overline{B_2} \, | \, \cdots \, | \, \overline{B_m} \, | \, o)}}}
(\mbox{app})\]



\begin{lemma}\label{lem:safe}
Suppose $\seq{\overline{x_1} : \overline{A_1}\, | \, \cdots\, | \,
\overline{x_n} : \overline{A_n}}{M : B}$ is valid. Then
\begin{itemize}
\item[(i)] B is homogeneous.
\item[(ii)] Any free variable of $M$ has rank
at least $\rank{M}$.
\item[(iii)] For any subterm $\lterm{\Phi}{L}$
of $M$, if the variable $\varphi$ occurs in $L$ and $\rank{\varphi}
< \rank{\Phi}$ then $\varphi$ is bound in $L$. \myendproof
\end{itemize}
\end{lemma}

We omit the straightforward proofs.

\subsubsection*{What does ``safe'' mean?}

Substitution is a fundamental operation in the $\lambda$-calculus.
In the key clause of the definition
\[ (\lterm{x}{M})[N / y] \; \defined \; \lterm{z}{((M[z / x]) [N / y])} \quad
\hbox{where ``$z$ is fresh''}\] bound variables are renamed afresh
to prevent variable capture. In the {safe $\lambda$-calculus}, one
can get away without any renaming.

\begin{lemma}In the safe $\lambda$-calculus, there is no need to
rename bound variables afresh when performing substitution
\[M[N_1 / x_1 , \cdots, N_n / x_n]\]
provided the substitution is performed simultaneously on \emph{all}
free variables of the same rank in $M$ i.e.~$\makeset{x_1, \cdots,
x_n}$ is the set variables of the same rank as $x_1$ that occur free
in $M$.
\end{lemma}

\begin{proof} Suppose $\varphi$ occurs free in $M$, and
bound variables in $M$ are not renamed in the substitution $M[N /
\varphi]$. Further suppose $x$, a variable occurring free in $N$, is
captured as a result of the substitution. I.e.~there is a subterm
$\lterm{x}{L}$ of $M$ such that $\varphi$ occurs free in $L$. We
compare $\rank{x}$ with $\rank{\varphi}$:

\begin{itemize}
\item {Case 1}: $\rank{x} > \rank{\varphi}$.

This is impossible: Since $\lterm{x}{L}$ is safe, by
Lemma~\ref{lem:safe}(iii), $L$ can have no free variables of rank
less than $\rank{x}$.

\item {Case 2}: $\rank{x} < \rank{\varphi}$.

This is impossible: Since $N$ is safe and of level $\rank{\varphi}$,
by Lemma~\ref{lem:safe}(ii), it can have no free variable of rank
less than $\rank{\varphi}$.

\item {Case 3}: $\rank{x} = \rank{\varphi}$.

If follows from the formation rule for $\lambda$-abstraction that
$\varphi$ cannot occur free in $M$ since the subterm $\lterm{x}{L}$
must be in the scope of some subterm $\lterm{\varphi}{\cdots}$ of
$M$, so that $\varphi$ does not occur free in $M$. Thus this case
cannot arise either.
\end{itemize}

\end{proof}

\bibliographystyle{plain}
\bibliography{safety}

\end{document}
