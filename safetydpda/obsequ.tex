\documentclass{article}
\usepackage{a4wide}
\usepackage{amsmath, amsthm, amssymb}

\newcommand\nat{\mathbb{N}}
\newcommand\eop{{\sf eop}}
\newcommand{\betared}{\rightarrow_\beta}
\newcommand{\encode}[1]{\ulcorner #1 \urcorner}

%pcf
\newcommand\ialgol{{\textsl{IA}}}
\newcommand\pcf{\textsf{PCF}}
\newcommand\pcfcond{\texttt{cond}}
\newcommand\pcfsucc{\texttt{succ}}
\newcommand\pcfpred{\texttt{pred}}

\newtheorem{theorem}{Theorem}[section]
%\newtheorem{cor}[thm]{Corollary}
\newtheorem{lemma}[theorem]{Lemma}
%\newtheorem{prop}[thm]{Proposition}

\begin{document}


In \cite{Ong02}, Ong shows that finitary second-order \ialgol\ with recursion ($\ialgol_2 + Y_1$) is undecidable by reducing the Queue-Halting problem to the obvservational equivalence of two $\ialgol_2 + Y_1$ programs. The same proof can be reused to show that observational equivalence for the Safe fragment of $\ialgol_2 + Y_1$ is also undecidable.


We fix a finite alphabet $\Sigma = \{ a_1, \ldots, a_p \}$. A queue program is a finite sequence of instructions that manipulate a queue data-structure. Let $n$ be the number of instruction in the program. There are four possible kinds of instruction: halting, enqueuing, dequeuing and branching. The set of instructions is formally defined as:
$$ \mathcal{I} = \{ {\tt halt} \} \cup
\{ {\tt enqueue \ } a \ | \ a \in \Sigma \}
\cup \{ {\tt dequeue} \} \cup \{ {\tt goto\ } l {\tt\ if\ first =\ } a \ | \ l \in 1..n, a \in \Sigma \}
 $$

A program $P$ is a sequence of $n$ instruction for some $n\in \nat$. For $1 \leq i \leq n$, $P i$ denotes the $i^{\sf th}$ instruction of the sequence $P$.

The operational semantics is described using a set of states $ \{ \eop \} \cup 1..n \times \Sigma^* $. The state $\eop$ corresponds to the final state at which the program stops. A state of the form $(i,x) \in 1..n \times \Sigma^*$ indicates that the queue's content is given by the sequence $x$ and that the next instruction to be executed by the machine is $P i$. The empty queue is denoted by $\epsilon$ and for any sequence $x \in \Sigma^*$, the first element of $x$ corresponds to the element that has been \emph{first} enqueued (i.e. the queue is fed at the right-end side and consumed at the left-end side). The rules defining the operational semantics are:
\begin{eqnarray*}
(i,x) \mbox{ with } P i = {\tt halt} &\longrightarrow& \eop \\
(i,x) \mbox{ with } P i = {\tt enqueue\ } a &\longrightarrow& (i+1,x \cdot a) \\
(i,\epsilon) \mbox{ with } P i = {\tt dequeue } &\longrightarrow& \eop \\
(i,a \cdot x) \mbox{ with } P i = {\tt dequeue } &\longrightarrow& (i+1,x) \\
(i,\epsilon) \mbox{ with } P i =  {\tt goto\ } l {\tt\ if\ first =\ } a &\longrightarrow& (i+1,\epsilon) \\
(i,b \cdot x) \mbox{ with } a \neq b \mbox{ and } P i =  {\tt goto\ } l {\tt\ if\ first =\ } a &\longrightarrow& (i+1,b \cdot x) \\
(i,a \cdot x) \mbox{ with } P i =  {\tt goto\ } l {\tt\ if\ first =\ } a &\longrightarrow& (l,a \cdot x)
\end{eqnarray*}


We now describe a transformation which, given a queue-program $P$, produces an \ialgol\ term $M_P$ that simulates $P$ in the sense that $P \Downarrow$ if and only if $M_P \rightarrow^* \eop$.

We fix a distinguished element $\bot$ denoting the end of the queue. Let $\Sigma^\bot = \Sigma \cup \{ \bot \}$. We assume that an injective function $\encode{\underline{\ }}$ is defined that encodes  the
elements of $\Sigma^\bot$ into $\nat$. It can for instance be defined as follows:
\begin{eqnarray*}
\encode{\underline{\ }} : \Sigma^\bot &\longrightarrow& 0..p \\
 \bot &\mapsto& 0 \\
 a_k &\mapsto& k \ , \mbox{for } k \in 1..p \ .
\end{eqnarray*}

We say that a $\ialgol_2$-term $M$ computes a sequence $s \in {(\Sigma^\bot)}^\omega$ if and only if 
$M k \betared^* \encode{s k}$ for all $k\in \nat$.

We now identify a queue content $s \in \Sigma^*$ with the infinite sequence $s \bot^\omega \in \Sigma^\omega$. In the \ialgol\ program that we are about to define, a queue content $s \in \Sigma^*$ will be represented by a term $\hat{s}$ that computes $s \bot^\omega$.

Each instruction $c$ of $\mathcal{I}$ is encoded uniquely into a natural number $\encode{ c }$. The following table give an example of such a (injective) function:
\begin{center}
\begin{tabular}{c|c|c|c|c}
$c \in \mathcal{I}$ & {\tt halt} & {\tt dequeue } & {\tt enqueue} $a_i$ &  {\tt goto} l {\tt if first=} $a_i$ \\ \hline
$\encode{ c } \in \nat$ & $0$ & $1$ & $1+i$ & $1+p + n.l +i$ \\
\end{tabular} 
\end{center}
where $1 \leq i \leq p, 1 \leq l \leq n$.

Let us fix a queue-program $P$ with $n$ instructions. 
The code of $P$ is compiled into the \ialgol\ program $\hat{P}$
which maps an index $i$ to the encoding of the $i^{\sf th}$ instruction of $P$:
\begin{equation}
\hat{P} = \lambda i .  {\sf match\ } i {\sf \ with\ } 
   0 \rightarrow \encode{ P 0 } 
  \ | \ldots \ | n \rightarrow \encode{ P n }  | \_ \rightarrow 0
\end{equation} 

We now define an \ialgol\ program which interprets the queue-program $P$ given  its compiled form $\hat{P}$.
\begin{eqnarray*}
SIM = Y ( \lambda f i z . {\sf match\ } \hat{P} i  {\sf \ with\ }   \\
 \encode{ \tt halt }&\rightarrow& \encode{\eop} \\
| \encode{ {\tt enqueue\ } a_1 } &\rightarrow& f (i+1) ( \lambda j . {\sf if} j = \underline{\sf length}\ x\ {\sf then\ } \encode{a_1} {\sf\ else\ } x j ) \\
| \ldots \\
| \encode{ {\tt enqueue\ } a_p } &\rightarrow& f (i+1) ( \lambda j . {\sf if} j = \underline{\sf length}\ x\ {\sf then\ } \encode{a_p} {\sf\ else\ } x j ) \\
| \encode{ {\tt dequeue } } &\rightarrow& f (i+1) (\lambda j . x (j+1) ) \\
| \encode{ {\tt goto\ } l {\tt\ if\ first=\ } a_1 } &\rightarrow&  
{\sf if\ } \underline{\sf length}\ x = 0 {\ \sf then\ } f (i+1) x \\
&&{\sf else\ if\ } \encode{a_1} = x 0 {\sf\ then\ } f l x \\
&&{\sf else\ } f (i+1) x \\
x j ) \\
| \ldots \\
| \encode{{\tt goto\ } l {\tt\ if\ first=\ } a_p} &\rightarrow&  
{\sf if\ } \underline{\sf length}\ x = 0 {\ \sf then\ } f (i+1) x \\
&&{\sf else\ if\ } \encode{a_p} = x 0 {\sf\ then\ } f l x \\
&&{\sf else\ } f (i+1) x \\
x j ) \\
\\
)\ 0\ (\lambda j . 0)
\end{eqnarray*}

where $f: (o,(o,o),o)$, $i:o$, $x:(o,o)$ and $\encode{\eop}$ is defined to be some \ialgol\ constant natural number .
$$ \underline{\sf  length}  = Y( \lambda f x k . {\sf if\ } x k = \encode{\bot} {\sf\ then\ } 0 {\sf\ else\ } 1 + (f x (k+1)))$$

Note that the test of equality of two natural numbers is definable in $\ialgol_2 + Y_1$ since the substraction operation is  also definable (by recursion). Moreover, it is easy to verify that the corresponding term is safe.


It is easy to show that $SIM \betared^* \encode{\eop}$ if and only if $P \rightarrow^* \eop$. Therefore since $SIM$ is a safe $\ialgol_2 + Y_1$ term we have the following result:
\begin{theorem}
Observational equivalence for Safe $\ialgol_2 + Y_1$ is undecidable.
\end{theorem}
\bibliographystyle{plain}
\bibliography{../transfer/current/gamesem}

\end{document}
