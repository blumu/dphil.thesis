\documentclass{article}

% computation tree, eta normal form, traversals
\newcommand{\aux}[1]{\lceil #1\rceil}
\newcommand{\betared}{\rightarrow_\beta}
\newcommand{\syneq}{\equiv}

\begin{document}

\section{Safe PCF}

We define the computation tree of a term to be the least upper bound of the computation trees of its approximants. Equivalently, it can be defined as follows. Consider a term of the form $Y_A F$ where $F:A\rightarrow A$ and $A = (A_1,\ldots,A_n,o)$.
We have $Y_A F \betared F (Y_A F)$.
We define the eta-normal form $\aux{Y_A F}$ of $Y_A F$ to be the unique infinite term satisfying the recursive equation $X \syneq \lambda \overline{x}: \overline{A} . \aux{F} X \overline{x}$
where $\overline{x}$ is a list of fresh variables.
The computation tree $\tau(Y_A F)$ is the infinite tree verifying the equation 
$\tau =  \lambda \overline{x}: \overline{A} . \tau X \overline{x}$ (consequently we have $\tau(Y_A F) = \tau(F (Y_A F))$).

For instance take $F \syneq \lambda f x. f x$ then 
$\aux{Y F} \syneq \lambda x . \aux{F} \aux{Y F} x \syneq \lambda x. (\lambda f x. f x) \aux{Y F} x$ therefore its computation tree $\tau$ is defined by the recursive equation $\tau = \lambda x . (\lambda f x. f x)  \tau x$.


A term containing a Y combinator may have an infinite computation tree. This is an undesired feature since we would like to derive a DPDA with finite states from the computation tree where each state corresponds to a node of the computation tree.


We therefore need to use an alternative definition of computation tree for Safe PCF terms.  We introduce a new type of abstraction node, written $\lambda_{rec}(f) x_1 \ldots x_n$ for some variable names $f, x_1, \ldots x_n$.




The disadvantage of this definition is that we have to redefine the function $\varphi$ that maps nodes to moves. We also need to define a new traversal rules to handle the new type of node.


\end{document}
