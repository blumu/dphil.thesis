\documentclass{article}
\usepackage{amssymb}
\usepackage{pst-tree}



% computation tree, eta normal form, traversals
\newcommand{\aux}[1]{\lceil #1\rceil}
\newcommand{\betared}{\rightarrow_\beta}
\newcommand{\syneq}{\equiv}

% back pointer using psttricks
\newcommand{\bkptr}[2][nodesep=0pt]{\ncarc[linewidth=0.4pt,offset=-2pt,nodesep=0pt,ncurv=1,arcangleA=-#2, arcangleB=-#2,#1]{->}}
\newcommand{\bklabel}[1]{\mput*{\mbox{{\tiny $#1$}}}}
\newcommand{\bklabelc}[1]{\Bput[1pt]{\mbox{{\tiny $#1$}}}}
\newcommand\treelabel[1]{\mput*{\mbox{{\small $#1$}}}}

% ia
\newcommand\ialgol{\textsf{IA}}
\newcommand\ialgolmnew{\ialgol-$\{\ianew\}$}
\newcommand\iaseqcom{$\tt{seq_{com}}$}
\newcommand\iaseqexp{$\tt{seq_{exp}}$}
\newcommand\iaseq{\texttt{seq}}
\newcommand\iaskip{\texttt{skip}}
\newcommand\iaderef{\texttt{deref}}
\newcommand\iaassign{\texttt{assign}}
\newcommand\iadone{\texttt{done}}
\newcommand\iarun{\texttt{run}}
\newcommand\iawrite{\texttt{write}}
\newcommand\iaread{\texttt{read}}
\newcommand\iaok{\texttt{ok}}
\newcommand\iamkvar{\texttt{mkvar}}
\newcommand\ianew{\texttt{new}}
\newcommand{\ianewin}[1]{\texttt{new}\ #1\ \texttt{in}}
\newcommand\iabool{\texttt{bool}}
\newcommand\iawhile{\texttt{while}}
\newcommand\iado{\texttt{do}}
\newcommand\iacom{\texttt{com}}
\newcommand\iaexp{\texttt{exp}}
\newcommand\iavar{\texttt{var}}

% pcf
\newcommand\pcf{\textsf{PCF}}
\newcommand\pcfcond{\texttt{cond}}
\newcommand\pcfsucc{\texttt{succ}}
\newcommand\pcfpred{\texttt{pred}}


\begin{document}

\section{The correspondence theorem for PCF}

We define the computation tree of a term to be the least upper bound of the computation trees of its approximants. Equivalently, it can be defined as follows. Consider a term of the form $Y_A F$ where $F:A\rightarrow A$ and $A = (A_1,\ldots,A_n,o)$.
We have $Y_A F \betared F (Y_A F)$.
We define the eta-normal form $\aux{Y_A F}$ of $Y_A F$ to be the unique infinite term satisfying the recursive equation $X \syneq \lambda \overline{x}: \overline{A} . \aux{F} X \overline{x}$
where $\overline{x}$ is a list of fresh variables.
The computation tree $\tau(Y_A F)$ is the infinite tree verifying the equation 
$\tau =  \lambda \overline{x}: \overline{A} . \tau X \overline{x}$ (consequently we have $\tau(Y_A F) = \tau(F (Y_A F))$).

For instance take $F \syneq \lambda f x. f x$ then 
$\aux{Y F} \syneq \lambda x . \aux{F} \aux{Y F} x \syneq \lambda x. (\lambda f x. f x) \aux{Y F} x$ therefore its computation tree $\tau$ is defined by the recursive equation $\tau = \lambda x . (\lambda f x. f x)  \tau x$.


A term containing a Y combinator may have an infinite computation tree. This is an undesired feature since we would like to derive from the computation tree a DPDA with finite states where each state corresponds to a node of the computation tree.

To remedy this problem, we introduce an alternative definition of computation tree for Safe PCF terms.  In order to handle the Y combinators, we introduce a new kind of abstraction node. Such node is labelled $\lambda \overline{x}$ where each variable occurring in $\overline{x}$ is either a standard variable or a ``recursion variable''.
Recursion variables are surrounded by parenthesis in order to distinguish them from standard ones. The set of labels $L$ is therefore given by the following grammar:
\begin{eqnarray*}
L &::=& \lambda\ V_{abs}^*\ |\ @\ |\ \mathcal{V} \\
V_{abs} &::=& \mathcal{V}\ |\ (\mathcal{V})
\end{eqnarray*}
where $\mathcal{V}$ is a countable set of variables.


A term of the form $Y_A (\lambda f. M)$ where $f,M:A$ will be written  $\lambda (f) . M$. Ground type recursion can also be represented using this notation. For instance the \iawhile operator of Idealized Algol 
$\iawhile\ C\ \iado\ I \syneq Y( \lambda f. \pcfcond\ C\ \iaskip\ (\iaseq I f))$ will be written $\lambda (f) . \pcfcond\ C\ \iaskip\ (\iaseq I f)$.

To construct the computation tree of a PCF term, we first remove all the $Y$ combinators occurring in the term by converting them into $\lambda$ abstraction as described above. Then we compute the $\eta$-normal form in the usual way, ignoring the recursion variables in the abstraction. Finally the computation tree is obtained from the eta normal form in the standard way.

The definition of the  order of an abstraction node does not change: it is the order of the term represented by the subtree rooted at this node. In other word, the order 
of $\lambda \overline{x}$ is the same as order of node $\lambda \overline{y}$ where $\overline{y}$ is the sublist of $\overline{x}$ obtained by filtering out the recursion variables.

The numbering of variables bound in a $\lambda$-node labelled $\lambda \overline{x}$ is as follows: the $i^{\sf th}$ standard variable in $\overline{x}$ is denoted by $i$ and the
the $i^{\sf th}$ recursion variable in $\overline{x}$ 
is denoted by $(i)$. The links in a justified sequence of nodes are labelled accordingly.

We can now define a notion of traversals adapted to this new kind of computation tree as follows. Firstly, all the traversal rules are kept unmodified. The recursion variables in the $\lambda$-nodes are automatically ignored by each rule since such variables are numbered differently from standard variables. In particular, the (Var) rules only applies to non-recursion variables. 
Secondly we add a new rule to handle the case of recursion variable:
{\bf ($Var_{rec}$)}
If  \raisebox{0cm}[0.5cm]{$t' \cdot \rnode{n}{n} \cdot
    \rnode{l}{\lambda \overline{x}}  \ldots
    \rnode{f}{f_i}  \bkptr[ncurv=0.6]{50}{f}{l} \bklabel{(i)}$} is a traversal for some \emph{recursion} variable $f_i$ bound by $\lambda \overline{x}$ then
    so is
\raisebox{0cm}[0.55cm]{
    $t' \cdot \rnode{n}{n} \cdot
    \rnode{l}{\lambda \overline{x}}  \ldots
    \rnode{f}{f_i} \cdot
    \rnode{letai}{\lambda \overline{x}}
     \bkptr[ncurv=0.6]{50}{f}{l} \bklabel{(i)}
    $}.

The node visited by this rule has no pointer associated to it (strictly speaking the resulting sequence is therefore not a justified sequence of nodes).




We should point out that the original definition of computation tree for PCF was not useless: it permitted us to prove the Correspondence Theorem straightforwardly by reducing it to the special case of the simply typed lambda-calculus (without recursion) through the use of approximants. 
The counterpart of this is that traversals were defined over infinite computation trees.

On the other hand, our new finite version of computation tree does not permit to show immediately a correspondence between 

 from $\lambda$-calclto PCF in a simple way as

The disadvantage of this definition is that we have to redefine the function $\varphi$ wich maps nodes of the computation tree to moves of the arena. We also need to define a new traversal rules to handle the new type of node.


\end{document}
