\input{compos.pre}

\psset{linecolor=darkGreen,linewidth=0.5pt}


\author{William Blum}
\title{P-incrementally justified strategies}

\begin{document}
\maketitle 

\section{Well-bracketing and P-incremental justification}

We consider an arena $A$ and make the following two assumptions on it:
\begin{itemize}
\item (A1) For $A \neq \bot$ (the arena with a single initial question), each question move in the arena enables at least one answer move.
\item (A2) Answer moves do not enable any other move.
\end{itemize}

We define the \defname{order} of a move $m$ in the arena $A$, written $\ord_A{m}$ (or just $\ord{m}$ where there is no ambiguity), as the length of the path from $m$ to its furthest leaf in $A$ minus 1
({\it i.e.}~the height of the subarena rooted at $m$ minus 2.).
Because of assumptions (A1) and (A2),
for any move $m$ of $A \neq \bot$, $m$ is a question move if and only if $\ord{m} \geq 0$, and $m$ is an answer move if and only if $\ord{m} = -1$.





We call \defname{pending question} of a sequence of moves $s \in L_A$ the last unanswered question in $s$.

\begin{definition}\rm
A strategy $\sigma$ is said to be \defname{P-well-bracketed} if for any play $s \, a \in \sigma$ where $a$ is a  P-answer, $a$ points to the pending question in $s$.
\end{definition}



P-well-bracketing can be restated differently as the following proposition shows:
\begin{proposition}
\label{prop:char_wellbrack}
\rm We make assumption (A1) and (A2).
Let $\sigma$ be a strategy on an arena $A\neq \bot$.
The following statements are equivalent:
\begin{enumerate}
\item[(i)] $\sigma$ is P-well-bracketed,
\item[(ii)] for $s \, a \in \sigma$ with $a$ a P-answer, $a$ points to the pending question in $\pview{s}$,
\item[(iii)] for $s \, a \in \sigma$ with $a$ a P-answer, $a$ points to the last O-question in $\pview{s}$,
\item[(iv)] for $s \, a \in \sigma$ with $a$ a P-answer, $a$ points to the last O-move in $\pview{s}$ with order $>\ord{a}$.
\end{enumerate}
\end{proposition}
\begin{proof}
$(i)\iff(ii)$: \cite[Lemma 2.1]{McC96b} states that if P is to move then the pending question in $s$ is the same as that of $\pview{s}$.

$(ii)\iff(iii)$: Assumption (A2) implies that the pending question in $\pview{s}$ is also the last O-question occurring in $\pview{s}$.

$(iii)\iff(iv)$: Because of assumption (A1) and (A2),
for any move $m$, we have $m$ is a question move
if and only if $\ord{m} \geq 0$ if and only if $\ord{m} > \ord{a} = -1$.
\end{proof}




\begin{lemma}
\label{lem:justfied_by_unanswered}
Under assumption (A2), if $s$ be a justified sequence of moves satisfying alternation and visibility then any O-move (resp. P-move) in $s$ points to an \emph{unanswered} P question (resp. O-question).
\end{lemma}
\begin{proof}
Suppose that an O-move $c$ points to a P-move $d$ that has already been answered by the O-move $a$. The sequence $s$ as the following form:
$$ s= \ldots \Pstr{(d){d}  \ldots  (a-d,20){a}  \ldots  (c-d,20){c}}$$

By O-visibility, $d$ must belong to $\oview{s_{<c}}$. But since $a$ is an answer, by assumption (A2), it cannot justify any P-move, therefore
$\oview{s_{<q}}$ must contain an OP-arc ``hoping'' over $a$. We name the nodes of this arc $d^1$ and $c^1$:
$$ s = \ldots \Pstr[0.7cm]{(d){d}  \ldots  (d1){d^1} \ldots (a-d,20){a} \ldots
 (c1-d1,20){c^1} \ldots (c-d,25){c}}$$

By P-visibility, $d^1$ must belong to $\pview{s_{<c^1}}$. Consequently, $a$ does not belong to $\pview{s_{<c^1}}$ (otherwise the PO-arc $\Pstr[0.5cm]{(d){d} \quad (a-d,45){a}}$ would cause the P-view to jump over $d^1$).
Therefore there must be a PO-arc $\Pstr[0.5cm]{(d2){d^2} \quad (c2-d2,45){c^2}}$ in $\pview{s_{<c^1}}$ hoping over $a$:
$$ s = \ldots \Pstr[0.7cm]{(d){d}  \ldots
(d1){d^1} \ldots (d2){c^2} \ldots
(a-d,20){a} \ldots
 (c2-d2,20){d^2} \ldots (c1-d1,20){c^1} \ldots (c-d,25){c}}$$

This process can be repeated infinitely often by using alternatively O-visibility and P-visibility. This gives a contradiction since the sequence of moves $s_{<c}$ has finite length.
Hence $d$ cannot point to a question that has already been answered. Since, by assumption (A2), a question is enabled by another question, $d$ is necessarily justified by an unanswered question.
\end{proof}


\begin{lemma}
\label{lem:oq_in_pview_unanswered}
Under assumption (A2), if $s$ is a P-well-bracketed justified sequence of moves of odd length satisfying alternation and visibility then  all O-questions occurring in $\pview{s}$ are unanswered in $s$.
\end{lemma}
\begin{proof}
We proof the first part by induction on $s$.
The base case ($s = q$ with $q$ initial O-move) is trivial.

Suppose $\Pstr[0.4cm]{ s = s' \cdot (n)n \cdot u \cdot (m-n,45){m} }$.
Let $r$ be an O-question in $\pview{s} = \pview{s'} \cdot n \cdot m$.
If $r$ is the last move $m$ then it is necessarily unanswered.
If $r \in \pview{s'}$ then by the induction hypothesis, $r$ is unanswered in $s'$.
Suppose that $r$ is answered in $s$. This implies that some answer move $a$ in $u$ points to $r$:
$$\pstr[0.7cm][5pt]{ s = \underbrace{\cdots\ \nd(r){r}^O \cdots }_{s'} \
\nd(n){n}^P \ \underbrace{\cdots\ \nd(a-r,35){a}^P \cdots }_{u} \
\nd(m-n,30){m}^O } \ .$$
 
Since $m$ points to $n$, by lemma \ref{lem:justfied_by_unanswered}, $n$ is still unanswered at $s_{\prefixof a}$. Therefore the pending
question at $s_{\prefixof a}$ cannot be $r$. But $a$ is justified by $r$, therefore the well-bracketing condition is violated. Hence $r$ is
unanswered in $s$.
\end{proof}





\begin{definition}\rm
A play $s m$ of even length is said to be \defname{P-incrementally justified}, or {\emph P-i.j.} for short, if $m$ points to the last unanswered O-question in $\pview{s}$ with order strictly greater than $\ord{m}$.

 A strategy $\sigma$ is said to be \defname{P-incrementally justified}, if all plays in $\sigma$ ending with a P-question are
P-incrementally justified.
\end{definition}

\begin{proposition}
\label{prop:char_pincr}
\rm We make assumption (A1) and (A2).
Let $\sigma$ be a \emph{P-well-bracketed} strategy on an arena $A\neq \bot$.
The following statements are equivalent:
\begin{enumerate}
\item[(i)] $\sigma$ is P-incrementally justified,
\item[(ii)] for $s \, q \in \sigma$ with $q$ a P-question, $q$ points to the last O-question in $\pview{s}$ with order $>\ord{q}$,
\item[(iii)] for $s \, q \in \sigma$ with $q$ a P-question, $q$ points to the last O-move in $\pview{s}$ with order $>\ord{q}$.
\end{enumerate}
\end{proposition}
\begin{proof}
$(i)\iff(ii)$: By lemma \ref{lem:oq_in_pview_unanswered}, O-question occurring in $\pview{s}$ are all unanswered.

$(ii)\iff(iii)$: Because of (A1) and (A2), $\ord{q} \geq 0$ thus an O-move with order $>\ord{q}$ is necessarily an O-question.
\end{proof}

Putting proposition \ref{prop:char_pincr} and
\ref{prop:char_wellbrack} together we obtain:
\begin{proposition}
Under assumption (A1) and (A2).
A strategy $\sigma$ on $A\neq \bot$
is \emph{P-well-bracketed} and
 \emph{P-incrementally justified} if and only if
for $s \, m \in \sigma$, $m$ points to the last O-move in $\pview{s}$ with order $>\ord{m}$.
\end{proposition}


\section{Remarks}
\subsection{Homogeneity constraint}

Type homogeneity is not preserved after composition. Indeed the types  $o \typear (o \typear o)$ and $(o \typear o) \typear \left((o \typear o) \typear o \right)$ are homogeneous
but $o \typear \left((o \typear o) \typear o\right)$ is not.

If $A\typear B$ and $B \typear C$ are homogeneous types then  a sufficient condition for $A\typear C$ to be homogeneous is  ``$\ord{A} \geq \ord{B}$''.


\section{Compositionality - A semantic proof}

\subsection{Preliminaries}
 
\subsubsection{Nodes order after composition}

Consider the arena $X\fngamear Y$
and let $m$ be a move of $X\fngamear Y$.
We write $\ord_{X\fngamear Y}{m}$
to denote the order of 
$m$ in the arena ${X\fngamear Y}$.
If $m$ belongs to $X$ (resp.~$Y$) then
we write $\ord_X{m}$ 
(resp.~$\ord_Y{m}$) to denote the order of the move $m$ in the arena $X$ (resp.~$Y$).

\begin{lemma}
\label{lem:compositionorder}
Let $A$, $B$ and $C$ be three arenas. We have:
$$\begin{array}{lll}
\forall m \in A:
    &  \ord_{A\fngamear B}{m} = \ord_{A\fngamear C}{m} \ ,\\
\forall m \in B:
    & \ord_{A\fngamear B}{m} \geq \ord_{B\fngamear C}{m}  & \mbox{for $m$ initial,}\\
    & \ord_{A\fngamear B}{m} = \ord_{B\fngamear C}{m} & \mbox{for $m$ non initial,} \\
\forall m \in C:
    & \ord_{A\fngamear C}{m} \geq \ord_{B\fngamear C}{m} \iff
\ord{A} \geq \ord{B}\ & \mbox{for $m$ initial,}\\
    & \ord_{A\fngamear C}{m} = \ord_{B\fngamear C}{m}   & \mbox{for $m$ non initial.}
\end{array}
$$
\end{lemma}





\subsubsection{Interaction sequences}
Let us first recall the definition of an interaction sequence.
Let $A$,$B$ and $C$ be three games. 
We say that $u$  is an \defname{interaction sequence} of $A$,$B$ and $C$ whenever $u\filter A,B$ is a valid position of the game $A\fngamear B$
(i.e.~$u\filter A,B \in P_{A\fngamear B}$) 
and  $u\filter B,C$ is a valid position of the game
$B\fngamear C$. We write $Int(A,B,C)$ to denote
the set of all such interaction sequences.

Let $\sigma:A\fngamear B$ and $\mu:B\fngamear C$
be two strategies. We write $\sigma \parallel \mu$ to denote the 
set of interaction sequences that unfold according to the strategy $\sigma$ in the $A,B$-projection of the game and 
to $\mu$ in the $B,C$-projection:
$$ \sigma \parallel \mu = \{ u\filter A,B \in \sigma \vee u \filter B,C \in \mu \} \ .$$
The composite of $\sigma$ and $\mu$ is then defined as $\sigma ; \mu = \{ u \filter A,C \ | \ u \in \sigma \parallel \tau \}$.

The diagram below shows the structure of an interaction sequence
from $\sigma \parallel \mu$. There are four states represented by the rectangular boxes. The content of the state shows who is to play in each of the game $A\fngamear B$, $B\fngamear C$ and $A\fngamear C$.
For instance in state $OPP$, it is O's turn to play in 
$A\fngamear B$ and P's turn to play in $B\fngamear C$ and $A\fngamear C$. Arrows represent the moves.
When specifying interaction sequence,
the following bullet symbols are used to represent moves:
$\pmove$ for P-moves, $\omove$ for O-moves, $\pomove$ 
for a move playing the role of P in $A\fngamear B$
and O in $B\fngamear C$ and $\opmove$ for
the symmetric of $\pomove$.
We sometimes add a subscript to the symbols $\pmove$ and $\omove$ to denote the component in which the moves is played ($A$ or $C$).


\tikzstyle{state}=[rectangle,draw=blue!50,fill=blue!20,thick,minimum height = 4ex, text width=4cm]
\tikzstyle{move}=[->,shorten <=1pt,>=latex',line width=1pt]
\tikzstyle{intmove}=[dashed] 
\tikzstyle{extomove}=[color=\extomovecolor] 
\tikzstyle{genomove}=[]%[dashed]
\tikzstyle{genpmove}=[color=\genpmovecolor]
\def\sep{1.5cm} 
\begin{table}[htbp]
\begin{center}
\begin{tikzpicture}[node distance=1.7cm]

% the four states 
\path 
 node(oooT)  [state] {}
 node(opp)   [state, below of=oooT] {}
 node(pop)   [state, below of=opp]  {}
 node(oooB)  [state, below of=pop] {}
 node(title) [anchor=south, at=(oooT.north), minimum height = 4ex, text width=4cm] { };

\path
% text in the title centered in 3 columns
  ([xshift=-\sep]title) node {$A\fngamear B$}
        (title) node {$B\fngamear C$}
        ([xshift=\sep]title) node {$A\fngamear C$}

% text in the states centered in 3 columns
  ([xshift=-\sep]oooT) node {O}
        (oooT) node {O}
        ([xshift=\sep]oooT) node {O}
  ([xshift=-\sep]opp) node {O}
        (opp) node {P}
        ([xshift=\sep]opp) node {P}
  ([xshift=-\sep]pop) node {P}
        (pop) node {O}
        ([xshift=\sep]pop) node {P}
  ([xshift=-\sep]oooB) node {O}
        (oooB) node {O}
        ([xshift=\sep]oooB) node {O}

% text in between two arrows giving the arena of the move
  (oooT) to node {\bf C} (opp)
  (opp) to node {\bf B} (pop)
  (pop) to node {\bf A} (oooB)

% arrows representing the moves
  (opp.20)    edge[move, genpmove]
		node[right] {$\mu$}
		node[left]{$\pmove$} (oooT.-20)
  (oooT.-160) edge[move, extomove, genomove]
		node[left] {$env_\mu$}
		node[right]{$\omove$} (opp.160)
  (pop.20)    edge[move, genomove,genpmove,intmove]
		node[right] {$\sigma$}
		node[left]{$\pomove$} (opp.-20)
  (opp.-160)  edge[move, genomove, genpmove,intmove]
		node[left] {$\mu$} 
		node[right]{$\opmove$}  (pop.160)
  (oooB.20)   edge[move, extomove,genomove]
		node[right] {$env_\sigma$}
		node[left]{$\omove$} (pop.-20)
  (pop.-160)  edge[move, genpmove]
		node[left] {$\sigma$}
		node[right]{$\pmove$} (oooB.160);

%\draw[move, genpmove] (3.5cm,-1cm) -- +(1,0) node[right] {Generalised P-move \& External P-move };
%\draw[move, genomove,genpmove] (3.5cm,-2cm) -- +(1,0) node[right] {Generalised O-move \& Generalised P-move};
%\draw[move, genomove,extomove] (3.5cm,-3cm) -- +(1,0) node[right] {Generalised O-move \& External O-move};
\draw[move] (3.5cm,-1cm) -- +(1cm,0cm) node[right] {External move};
\draw[move,intmove] (3.5cm,-2cm) -- +(1cm,0cm) node[right] {Internal move};
\draw (3.5cm,-3cm) node[anchor=west] {\textcolor{\extomovecolor}{External O-moves: $\omove$}};
\draw (3.5cm,-4cm) node[anchor=west] {\textcolor{\genpmovecolor}Generalised P-move: $\opmove, \pomove, \pmove$};
\end{tikzpicture} 
\end{center}
\caption{Structure of an interaction sequence.}
\label{tab:interseq}
\end{table}

Note that in state OPP, the alternation condition (for each of the three games involved) prevents the players from playing in A. Indeed, the O-moves in component $A$ of $A\fngamear B$ are also $O$-moves in component $A$ of $A\fngamear C$ however the state name indicates that the next move in $A\fngamear B$ must be an O-move and the next move in $A\fngamear C$ must be a P-move.

Similarly, in the top state OOO, the players cannot make move in B since the O-moves in component B of the game $B\fngamear C$ correspond to P-moves in the component B of $A\fngamear B$. However the state name indicates that the next move in $A\fngamear B$ and the next move in $B\fngamear C$ must be played by O.


Let $u \in Int(A,B,C)$ and $m$ be a move of $u$.
The \defname{component} of $m$ is $A,B$ if 
after playing $m$ the game is under the control 
of the strategy $\sigma$ and $B,C$ otherwise (if $\mu$ has control).
In other words, the moves $\omove, \pmove \in A$
and $\opmove \in B$ shown on the diagram of Table \ref{tab:interseq}
have component $A,B$ and 
$\omove, \pmove \in C$ and $\pomove \in B$
have component $B,C$.


Also we call \defname{generalized O-move in component $A,B$}
moves that play the role of O in the game $A\fngamear B$, that is to say moves represented by $\opmove$ and $\omove_A$.
Similarly $\pomove$ and $\pmove_A$ moves are the \defname{generalized P-moves in component $A,B$},
$\omove_C$ and $\pomove$ moves are
the \defname{generalized O-moves in component $B,C$}
and  $\pmove_C$ and $\opmove$ moves are the \defname{generalized P-moves in component $B,C$}.

The P-view (also called {\emph core} in \cite{McCusker-GamesandFullAbstrac}) of an interaction sequence $u \in Int(A,B,C)$, written $\overline{u}$ or $\pview{u}$ is defined as:
\begin{align*}
\pview{u\cdot \extomove{n}} &= \extomove{n} &
\mbox{ if \extomove{$m$} is an \extomove{external O-move} initial in C,}\\
\pview{\Pstr{u\cdot (m)m\cdot v \cdot (n-m,45){\extomove{n}} }} &= \extomove{n} &\mbox{ if \extomove{$m$} is an \extomove{external O-move} non initial in C,}\\
\pview{u \cdot \genpmove{m}} &= \pview{u}\cdot \genpmove{m}  & \mbox{ if \genpmove{$m$} is a \genpmove{generalised P-move}.}\\ 
\end{align*}

We can show the following property by an easy induction :
\begin{lemma}
\label{lem:pviewAC_eq_ACpview}
 Let $u$ be an interaction sequence in $Int(A,B,C)$ then
$$\pview{u} \upharpoonright A,C = \pview{u \upharpoonright A,C} \ .$$
\end{lemma}

\subsection{Closed P-i.j. strategies and compositionality}

\subsubsection{Closed P-i.j strategy}

\begin{definition}
\label{def:safe_strategy}
A P-incrementally justified strategy $\sigma : A \fngamear B$ is said to be \defname{closed P-incrementally justified} if for
every initial move $m$ of $A$ such that some play $s\in\sigma$ contains $m$, we have $\ord_A{m} \geq \ord{B}$. 
\end{definition}
We observe that every P-i.j. strategy $\sigma$ on the game $I \fngamear A$ (and not $A$) is closed P-i.j..\footnote{In particular, every P-i.j. strategy $\sigma$ on the game $!A_1 \otimes \ldots \otimes !A_n \fngamear B$, is isomorphic, up to arena-tagging of the moves, to the closed P-i.j. strategy $\Lambda^n(\sigma)$ on the game $I \fngamear (A_1,\ldots,A_n,B)$, where $\Lambda$ denotes the usual currying isomorphism.}
However $\sigma : A$ is not necessarily closed P-i.j.. The reason is that the definition of closed P-i.j. strategy specifically refers to the moves of  the arena in the left-hand side of the function space arrow $\fngamear$, therefore the definition does not survive an isomorphism that retags the moves such as {\it currying}.

Consequently, for two strategies $\sigma$ and $\mu$ verifying $\sigma \cong \mu$, if $\sigma$ is closed P-i.j. then it does not necessarily imply that $\mu$ is. This contrasts with ``ordinary'' P-incremental justification condition which is preserved across any isomorphism.

Later on we will define a category of closed P-i.j. strategy. A consequence of the previous remark is that this category cannot be a closed category (neither monoidal closed nor cartesian closed).
In particular this category has only a weak form of curry isomorphism.

\subsubsection{Compositionality of closed P-i.j strategy}

{\bf Notation} In plays representations,
the symbol $\omove$ stands for an
O-move and $\pmove$ for
a P-move. 
Suppose the considered game is $L\fngamear R$ 
for some game $L$ and $R$, if we know the sub arena in which the move is played
then we specify it in subscripts ($\omove_L$, $\pmove_L$, $\omove_R$ or $\pmove_R$). For interaction sequences in $Int(A,B,C)$ we use
the symbols $\omove_A$, $\pmove_A$, $\omove_C$, $\pmove_C$, $\opmove$ and $\pomove$ as defined in Table \ref{tab:interseq}. We use the variable $X$ to denote one of the component $A,B$ or $B,C$, the variable  $Y$
then denotes the other component.
We write $s \subseqof t$ to say that $s$ is a subsequence (with pointers) of $t$, $s \prefixof t$ to say that $s$ is a prefix (with pointers)
of $t$ and  $s \suffixof t$ to say that $s$ is a suffix of $t$.

We now prove several useful lemmas which will become useful when studying compositionality of P-i.j. strategies.

\begin{lemma}
\label{lem:interjump}
Let $X$ be a component (either  $A,B$ or  $B,C$).
Let $u$ be an interaction sequence of the form
$ u =  
\Pstr[0.5cm][2pt]{ \ldots (b){\stk \beta \pmove}  \ldots
 {n}  \ldots  (a-b,30){\stk \alpha\omove}
\ldots m}$ where:
\begin{itemize}[-]
\item $\alpha,\beta$ are external moves in component $X$ (necessarily both played in $A$ or in $C$),
\item  $m$ is either played in $B$ or an external P-move in $X$,
\item  $\alpha$ is visible at $m$ in $X$ \emph{i.e.}~$\alpha\in \pview{u \upharpoonright X}$ (consequently $\beta$ is also visible).
\end{itemize}
Then $n \not\in \pview{u \upharpoonright A, C}$.
\end{lemma}
\begin{proof}
Since $\alpha$ is an O-move, $\alpha$ and $\beta$ are necessarily played in the same arena ($A$ or $C$).
Take $v=u$ if $m$ is a generalized O-move in $X$
and $v=u_{<z}$ otherwise (if $m$ is a generalized P-move in $X$).
The third assumption implies 
$\alpha,\beta\in \pview{v}$.
The last move in $v$ is necessarily a generalized O-move in component $X$ (see diagram of Table \ref{tab:interseq}) 
therefore by \cite[Lemma 3.3.1]{Harmer2005}
we have $\pview{v \filter X} = \pview{\overline{v} \filter X} \subseqof \overline{v} \subseqof \overline{u}$.
Thus $\alpha,\beta \in \overline{u}$ and
since $\alpha,\beta$ are played in $A,C$ we have 
$\alpha,\beta  \in \overline{u} \upharpoonright A,C 
= \pview{u \upharpoonright A,C}$ (Lemma \ref{lem:pviewAC_eq_ACpview}).
Finally since $n$ lies underneath the $\beta$-$\alpha$ PO-arc 
it cannot appear in the P-view  $\pview{u \upharpoonright A,C}$.
\end{proof}

\begin{lemma}
\label{lem:in_pviewAC_imp_in_pviewX}
Let $u$ be an interaction sequence in $Int(A,B,C)$ and
$n$ be a move of $u$ such that $n\in\pview{u \filter A,C}$:
\begin{enumerate}[i.]
\item 
if all the moves in $u_{\suffixof n}$ 
are played in $C$  then $n \in \pview{u \filter B,C}$;
\item 
if all the moves in $u_{\suffixof n}$ are played in $A$ then $n \in \pview{u \filter A,B}$.
\end{enumerate}
\end{lemma}
\begin{proof}
\begin{enumerate}[(i)]
\item
We show the contrapositive. Suppose that $n \not\in\pview{u \filter B,C}$. This must be due to one of the following  two
reasons:
\begin{itemize}[-]
\item $\pview{u \filter B,C}$ contains an initial move $c_0 \in C$
occurring after $n$ in $u$.


By \cite[Lemma 3.3.1]{Harmer2005}
we have $\pview{u \filter B,C} = \pview{\overline{u} \filter B,C} \subseqof \pview{u}$, thus $c_0$ also occurs in $\pview{u}$.
Since $c_0$ belongs to $C$ we have
$c_0 \in \pview{u} \filter A,C=
\pview{u \filter A,C}$ (Lemma \ref{lem:pviewAC_eq_ACpview}).
Thus the P-view $\pview{u \filter A,C}$
starts with the initial move $c_0$ and
since $n$ occurs before $c_0$, $n$ does not occur in the P-view.

\item $n$ lies underneath a PO-arc $\beta$-$\alpha$ visible 
at $ u \filter B,C$.
By assumption, since $\alpha$ occurs after $n$ in $u$, it must belong to $C$. We can therefore apply Lemma \ref{lem:interjump}
with $X\assignar B,C$ which gives
$n \not\in\pview{u \filter A,C}$.
\end{itemize}

\item Suppose that $n \not\in\pview{u \filter A,B}$ then either:
\begin{itemize}[-]
\item $\pview{u \filter A,B}$ contains an initial move $b_0 \in B$
occurring after $n$ in $u$. But this is impossible since by assumption all the moves occurring after $n$ in $u$ belong to $A$.

\item or $n$ lies underneath a PO-arc $\beta$-$\alpha$ in $A,B$.
By assumption, since $\alpha$ occurs after $n$ it must belong to $A$. We can then conclude using 
Lemma \ref{lem:interjump} with $X\assignar A,B$.
\end{itemize}
\end{enumerate}
\end{proof}

Note that we cannot completely relax the assumption 
which says that moves in $u_{\suffixof n}$ are all in the same component.
For instance take $u = \Pstr[0.5cm]{(co){\omove_C}\thinspace 
(b0-co){\opmove} \thinspace 
(n){\stk{\pmove_A}{n}} \thinspace 
(b1-co){\opmove}}$ then we have $n\in\pview{u\filter A,C}$ but $n\notin\pview{u\filter A,B}$.


%%%%%%%%%%%
% This commented Lemma could be useful be we did not make use of it eventually.
%
% \begin{lemma}
%\label{lem:oviewsegmentinB}
%For any legal sequence $s = \ldots x \cdot r \cdot y$ of a game $A\fngamear B$ if $x, y \in A$ and $x$ is O-visible from $y$ then any move in $r$ occurring in $\oview{s}$ belongs to $A$.
%\end{lemma}
%\begin{proof}
%We proceed by induction on the length of the segment $r$.
%Base case $r=\epsilon$ is trivial. Suppose $r = r' \cdot m$.
%If $y$ is an O-move then by the Switching Condition
%$m$ is necessarily in $A$. Clearly $x$ is O-visible from $m$ thus  by the I.H. any move from $r$ occurring in the O-view is in $A$.
%
%If $y$ is a P-move then it cannot point to an initial move in $B$. Indeed, suppose that it points to an initial O-move $b_0 \in B$ then
%we have $\oview{s} = b_0 \cdot y$ which contradicts the fact that $x\in \oview{s}$.
%Thus $y$ points to a move in $A$ and again we can conclude using the induction hypothesis.
%\end{proof}


\begin{lemma}[P-visibility decomposition (from $C$)]
\label{lem:middlepomove}
Let $u = \ldots n' \cdot r \cdot m \in Int(A,B,C)$ where
$n'$ is a $\omove_A$-move verifying $n' \in \pview{u\filter A,C}$ and $m$ is in $\{ \pmove_C, \opmove, \pomove \}$. Then there is a $\pomove$-move $\gamma$ in $r \cdot m$ such that $\gamma \in \pview{u\filter B,C}$ , $n' \in \pview{u_{\leq \gamma} \filter A,B}$ and $\gamma$ is justified by a move occurring before $n'$.
\end{lemma}
\begin{proof}
By induction on $|r|$.
If $r=\epsilon$ then necessarily $u = \ldots \stk{\omove_A}{n'} \thinspace\stk \pomove m$ where $m$ points before $n'$ ($n'$ being played in $A$ cannot justify $m$ played in $B$) so we just need to take $\gamma = m$.
If $|r|=1$ then either 
$u = \ldots \stk{\omove_A}{n'} \pomove\thinspace\stk {\pmove_C} m$
or $u = \ldots \stk{\omove_A}{n'} \pomove\thinspace\stk \opmove m$.
In both cases we can take $\gamma$ to be $\pomove$, the move between $n'$ and $m$.
Suppose $|r|>1$. Let $m^-$ denote the move preceding $m$ in $u$.
We proceed by case analysis:
\begin{enumerate}[i.]
\item Suppose $m = \pmove_C$ and $m^- = \omove_C$.
Let $q$ be the external P-move that justifies $m^-$.
Since $n' \in \pview{u\filter A,C}$, $q$ must occur after $n'$ in $u$:
$$ 
\begin{array}{ccccl}
A & \stackrel\sigma{\longrightarrow} & B & \stackrel\mu{\longrightarrow} & C \\
&\vdots&&\vdots\\
n' \omove\\
&\vdots&&\vdots  \\
&& & &  \rnode{q}{\pmove}q  \\
&\vdots&&\vdots  \\
&& & &  \rnode{mp}{\omove}m^-  \\
&& & &  \rnode{m}{\pmove}m  \\
\end{array}
\ncarc[arcangleA=60,arcangleB=60]{->}{mp}{q}
 $$  
Thus we can use the induction hypothesis (with $u\assignar u_{\prefixof q}$): there is a $\pomove$-move $\gamma$ 
in $u_{]n',q]}$ pointing before $n'$ such that $\gamma \in \pview{u_{\prefixof q} \filter B,C}$, $n' \in \pview{u_{\prefixof \gamma} \filter A,B}$.
Moreover $\pview{u_{\prefixof q} \filter B,C} \prefixof \pview{u_{\prefixof m} \filter B,C}$ (since $q$ is visible from $m$ in $B,C$) thus we have $\gamma \in \pview{u_{\prefixof m} \filter B,C}$ as required.

\item Suppose $m = \pmove_C$ and $m^- = \pomove \in B$.
Again we can conclude using 
the induction hypothesis with $u \assignar u_{\prefixof m^-}$.

\item Suppose $m = \pomove \in B$.

Suppose that all the moves in $r$ are in $A$.
Then $r$ is of the form $(\pmove_A \omove_A)^*$ (where $(\cdot)^*$ denotes the Kleenee star operator). 
We just need to take $\gamma = m$. 
Indeed, moves in $u_{\suffixof m}$ are all in $A$
and by assumption $n'\in\pview{u\filter A,C}$  therefore
Lemma \ref{lem:in_pviewAC_imp_in_pviewX}(ii) gives
$n'\in\pview{u\filter A,B}$.
Also, since $m$ is a $\pomove$-move, 
its justifier is a $\opmove$-move but $r$ contains only $\omove$ and $\pmove$ moves hence $m$'s justifier must occur before $n'$.

Suppose that $r$ contains at least one move in $B$. Let $b$ be the last such move, then $u$ is of the form $\ldots n' \cdot \ldots \cdot \stk\opmove  b \cdot (\pmove_A \omove_A)^* \cdot\thinspace\stk\pomove m $. We then have
$u\filter B,C = \ldots n' \cdot \ldots \cdot 
\thinspace\stk\opmove b \thinspace\cdot \stk\pomove m $ thus $b \in \pview{u\filter B,C}$. We can then conclude by applying the induction hypothesis with $u \assignar u_{\prefixof b}$.

\item Suppose $m = \pomove \in B$.
If $m^- = \opmove \in B$ then the I.H. with $u \assignar u_{\prefixof m^-}$ permits us to conclude.
If $m^- = \omove \in C$ then we conlude by applying  the I.H. on $u \assignar u_{\prefixof q}$ where $q$ is the external P-move in $C$ justifying
$m^-$.
\end{enumerate}
\end{proof}

We now show the lemma symmetric to the previous one:
\begin{lemma}[P-visibility decomposition (from $A$)]
\label{lem:middleopmove}
Let $u = \ldots n' \cdot r \cdot m \in Int(A,B,C)$ where
$n'$ is an O-move \emph{non initial} in $C$ verifying $n' \in \pview{u\filter A,C}$ and $m$ is in $\{\pmove_A, \opmove, \pomove\}$. Then there is a $\pomove$-move $\gamma$ in $r \cdot m$ such that $\gamma \in \pview{u\filter A,B}$ , $n' \in \pview{u_{\leq \gamma} \filter B,C}$ and $\gamma$ is justified by a move occurring before $n'$.
\end{lemma}
\begin{proof}
The proof is almost symmetrical to the previous one (Lemma \ref{lem:middlepomove}). We proceed by induction on $|r|$.
If $r=\epsilon$ then necessarily $u = \ldots \stk {\omove_C} {n'} \thinspace\stk \opmove m$ where $m$ points before $n'$ (it cannot point to $n'$
since $n'$ is not initial in $C$). Thus we just need to take $\gamma = m$.

If $|r|=1$ then either 
$u = \ldots \stk {\omove_C} {n'} \thinspace\opmove\thinspace\thinspace\stk{\pmove_A} m$
or $u = \ldots \stk {\omove_C} {n'} \thinspace\opmove\thinspace\thinspace\stk \pomove m$.
In both cases we can take $\gamma$ to be $\pomove$, the move between $n'$ and $m$.
Suppose $|r|>1$. Let $m^-$ denote the move preceding $m$ in $u$.
We do a case analysis:
\begin{enumerate}[i.]
\item Suppose $m = \pmove_A$ and $m^- = \omove_A$.
Let $q$ be the external P-move that justifies $m^-$.
Since $n' \in \pview{u\filter A,C}$, $q$ must occur after $n'$ in $u$:
$$ 
\begin{array}{rcccl}
A & \stackrel\sigma{\longrightarrow} & B & \stackrel\mu{\longrightarrow} & C \\
&\vdots&&\vdots\\
&&&& \omove\ n'\\
&\vdots&&\vdots  \\
q\rnode{q}{\pmove}  \\
&\vdots&&\vdots  \\
m^- \rnode{mp}{\omove}  \\
m \rnode{m}{\pmove}  \\
\end{array}
\ncarc[arcangleA=-45,arcangleB=-45]{->}{mp}{q}
 $$  
Thus we can use the induction hypothesis (with $u\assignar u_{\prefixof q}$): there is a $\opmove$-move $\gamma$ 
in $u_{]n',q]}$ pointing before $n'$ such that $\gamma \in \pview{u_{\prefixof q} \filter A,B}$, $n' \in \pview{u_{\prefixof \gamma} \filter B,C}$.
Moreover $\pview{u_{\prefixof q} \filter A,B} \prefixof \pview{u_{\prefixof m} \filter A,B}$ (since $q$ is visible from $m$ in $A,B$) thus we have $\gamma \in \pview{u_{\prefixof m} \filter A,B}$ as required.

\item Suppose $m = \pmove_A$ and $m^- = \pomove$ then again we can conclude using the I.H. with $u \assignar u_{\prefixof m^-}$.

\item Suppose $m = \opmove$.
\begin{itemize}[-]
\item Suppose that $r$ does not contain any move in $B$  then $r$ is of the form $(\pmove_C \omove_C)^*$. 

We just need to take $\gamma = m$. 
Indeed:
\begin{enumerate}
\item By lemmma \ref{lem:in_pviewAC_imp_in_pviewX}(i)
we have $n'\in \pview{u\filter B,C}$.

\item  $m$ is justified by a move occurring before $n'$. 
Indeed, if $m$ is justified by a $\pomove$-move then since $n' \cdot r$ contains only $\omove$ and $\pmove$ moves, $m$'s justifier must occur before $n'$.
If $m$'s justifier is an initial $\omove_C$-move $c_i$, then 
by P-visibility we have $c_i \in \pview{u\filter B,C}$
but since the P-view computation ``stops'' when reaching an initial moves, in order to guarantee that $n'$ also belongs to the P-view (as shown in (a)) it must
occurs after $c_i$.
\end{enumerate}


\item Suppose that $r$ contains some move in $B$. Let $b$ be the last such move. Then $u$ is of the form $u = \ldots n' \cdot \ldots \cdot \stk\opmove  b \cdot (\pmove_A \omove_A)^* \cdot\ \stk\pomove m $. 
So we have
$u\filter B,C = \ldots n' \cdot \ldots \cdot \stk\opmove  b \cdot \stk\pomove m $ hence $b \in \pview{u\filter B,C}$. We can now 
conclude by applying the I.H. with $u \assignar u_{\prefixof b}$.
\end{itemize}

\item Suppose $m = \pomove \in B$.
If $m^- = \pomove \in B$ then the I.H. with $u \assignar u_{\prefixof m^-}$ permits us to conclude.
If $m^- = \omove \in A$ then we conclude by applying the I.H. on $u \assignar u_{\prefixof q}$ where $q$ is the external P-move in $A$ justifying $m^-$.
\end{enumerate}
\end{proof}

We now use the two preceding Lemmas to show
the following useful result:
\begin{lemma}[Increasing order lemma]
\label{lem:increasing_order}
Let $u = \ldots n' \cdot r \cdot m \in Int(A,B,C)$ where
\begin{enumerate}
\item 
$n'$ is an external O-move in compoment $X$ 
($n'=\omove_A$ and $X=A,B$, or $n'=\omove_C$ and $X=B,C$)  non initial in $C$,
\item $n' \in \pview{u\filter A,C}$,
\item $m$ is either played in $B$ 
($\opmove$ or $\pomove$) or is an external
 P-move in $Y$
($\pmove_C$ if $n'=\omove_A$ and 
$\pmove_A$ if $n'=\omove_C$),
\item $m$'s justifier occurs before $n'$,
\item $u\filter X$ is P-i.j.,
\item $u_{\prefixof b}\filter Y$ is P-i.j. for all B-move $b$.
\end{enumerate}
Then:
$$ \ord_{Y} m \geq \ord_{A\fngamear C} n' \ .$$
\end{lemma}
\begin{proof}
If $n' =\omove_C$ (resp.~if $n'=\omove_A$)
then by Lemma \ref{lem:middleopmove} 
(resp.~Lemma \ref{lem:middlepomove})
there is a $\opmove$
(resp.~$\pomove$) $\gamma$ 
in $r \cdot m$ such that $\gamma \in \pview{u\filter Y}$ , $n' \in \pview{u_{\leq \gamma} \filter X}$ and $\gamma$ is justified by a move occurring before $n'$. 

There are six possible cases depending on 
the type of the moves $n'$ and $m$:
$(n',m) \in \{ \omove_A \} \times \{\pmove_C,\opmove,\pomove \} 
\union \{ \omove_C \} \times \{\pmove_A,\opmove,\pomove \} $).
The following diagram illustrates the cases $(n',m)
 = (\omove_A,\pmove_C)$ (left)
and  $(n',m)
 = (\omove_C,\pmove_A)$  (right):
$$ 
\begin{array}{ccccc}
A & \longrightarrow & B &
 \longrightarrow & C \\
&\vdots&&\vdots\\
&&&& \rnode{n}{\omove} \\
&\vdots&\rnode{gj}{\opmove}&\vdots\\
n' \omove \\
&\vdots&&\vdots  \\
&&\rnode{g}{\gamma} \pomove \\
&\vdots&&\vdots  \\
&&&&\rnode{m}{m} \pmove \\
\end{array}
\ncarc[arcangleA=30,arcangleB=30]{->}{m}{n}
\ncarc[arcangleA=30,arcangleB=30]{->}{g}{gj}
\hspace{2cm} \begin{array}{ccccc}
A & \longrightarrow & B & \longrightarrow & C \\
&\vdots&&\vdots\\
& \rnode{n}{\omove} \\
&\vdots&\rnode{gj}{\pomove}&\vdots\\
&&&&n' \omove \\
&\vdots&&\vdots  \\
&&\rnode{g}{\gamma} \opmove \\
&\vdots&&\vdots  \\
\rnode{m}{m} \pmove \\
\end{array}
\ncarc[arcangleA=30,arcangleB=30]{->}{m}{n}
\ncarc[arcangleA=30,arcangleB=30]{->}{g}{gj}
 $$  

We have:
\begin{equation}
\ord_Y \gamma \geq \ord_X \gamma \label{eqn:gammaorderXY}
\end{equation}
Indeed, if $n' =\omove_C$ then $X=B,C$ and $Y=A,B$ and
by Lemma \ref{lem:compositionorder} we have
$\ord_{A\fngamear B} \gamma \geq \ord_{B\fngamear C} \gamma$.
If $n=\omove_A$ then $\gamma$ is a $\pomove$-move therefore it is not initial in $B$ and Lemma \ref{lem:compositionorder} gives
$\ord_{A\fngamear B} \gamma = \ord_{B\fngamear C} \gamma$.

Hence:
\begin{align*}
\ord_{A\fngamear C} n' 
& = \ord_{X} n' & \mbox{(n' non initial in $C$ \& Lemma \ref{lem:compositionorder})} \\
& \leq \ord_{X} \gamma & \mbox{($u_{\prefixof \gamma}\filter Y$ is P-i.j. \& $\gamma$'s justifier occurs before $n'$)} \\
& = \ord_{Y} \gamma & \mbox{(By Eq. \ref{eqn:gammaorderXY})} \\
& \leq \ord_{Y} m & \mbox{($u\filter X$ is P-i.j. \& 
4$^{th}$ assumption: $m$'s justifier occurs before $\gamma$)}. 
\end{align*}
\end{proof}


\begin{proposition}
\label{prop:pijcompose_when_orda_geq_ordb}
Let $\sigma : A \fngamear B$ and $\mu : B \fngamear C$
be two well-bracketed (P-visible) strategies then
\begin{enumerate}[(I)]
\item if $\sigma$ is closed P-i.j. and $\mu$ is P-i.j.
then $\sigma ; \mu : A \fngamear C$ is P-i.j.,
\item if $\sigma$ and $\mu$ are closed P-i.j.
then $\sigma ; \mu : A \fngamear C$ is closed P-i.j.
\end{enumerate}
\end{proposition}

\begin{proof}
Well-bracketing is preserved by strategy composition (see \cite[Proposition 2.5]{abramsky94full}) thus
$\sigma ; \mu$ is well-bracketed so we can use the definition of P-i.j. from Proposition \ref{prop:char_wellbrack}.

\noindent (I) Let us prove that $\sigma ; \mu$ is P-i.j..
Let $u$ be a play of the interaction $\sigma\ \|\ \mu$ between $\sigma$ and $\mu$
ending with an external P-move $m$
justified by $n$ in $\pview{u \upharpoonright A , C}$.
Let $n'$ be an external O-move occurring betweeen $n$ and $m$:
$$ u \filter A,C =  
\Pstr[0.5cm][2pt]{ \ldots (n){\stk {n} \omove}  \ldots
 {\stk {n'} \omove}  \ldots  (m-n,30){\stk m \pmove}
}
$$
To show that $u \filter A,C$ is P-incrementally justified, we just need to prove that either $n'\not\in \pview{u \filter A,C}$ or $\ord_{A\fngamear C} n' \leq \ord_{A\fngamear C} m$. 
Note that if $n'\in \pview{u \filter A,C}$ 
then necessarily $n'$ is not initial 
in $C$ because $n$ occurs before $n'$ in
$\pview{u \filter A,C}$.

\begin{enumerate}[1)]
\item \label{case:mC}
Suppose $m =\pmove_C$, then necessarily $n=\omove_C$.

\begin{enumerate}[{\ref{case:mC}}.a)]
\item \label{case:mCnpC} Suppose $n' = \omove_C$. The projection of $u$ on the game $B\fngamear C$ has the following form:
$$ u \filter B,C =  
\Pstr[0.5cm][2pt]{ \ldots (n){\stk {n} {\omove_C}}  \ldots
 {\stk {n'}{\omove_C}}  \ldots  (m-n,30){\stk m {\pmove_C}}
}$$

Suppose that $n'$ occurs in the P-view $\pview{u\filter B,C}$ then necessarily $n'$ is not initial 
in $C$ (otherwise it would be the first move in
$\pview{u \filter B,C}$, which is not the case since $n$ occurs before $n'$ in the P-view) and we have:
\begin{align*}
\ord_{A\fngamear C} n' 
& = \ord_{B\fngamear C} n' & \mbox{(Lemma \ref{lem:compositionorder} \& $n'$ non initial in $C$)} \\
& \leq \ord_{B\fngamear C} m & \mbox{($\mu$ is P-i.j.)} \\
& = \ord_{A\fngamear C} m & \mbox{($m$ is not initial in $C$)} \ .
\end{align*}

Suppose that $n'$ does not occur in the P-view $\pview{u \filter B,C}$, then $n'$ lies underneath a PO arc occurring in $\pview{u \filter B,C}$. Let us denote this arc by $\beta$-$\alpha$ where $\beta$ and $\alpha$ denote the arc's nodes. We have:
$$ u \filter B,C = \ldots  
\Pstr[0.5cm]{
 (n){\stk{n} {\omove_C} } \ldots (b){\stk\beta \pmove} \ldots \stk{n'} {\omove_C}
\ldots (a-b){\stk\alpha \omove}  \ldots (m-n){\stk m {\pmove_C} }
} $$
with $\ord_{B\fngamear C} \alpha \leq \ord_{B\fngamear C} m$ (by P-i.j. of $\mu$).
  
\begin{enumerate}[i.]
\item Suppose $\alpha$ is a $\omove_C$-move in $u$ then necessarily
$\beta$ is a $\pmove_C$-move (since $\alpha$ is an O-move).

By instancing Lemma \ref{lem:interjump} with
$X\assignar B,C$ and $n \assignar n'$ we obtain that $n' \not\in\pview{u \filter A,C}$.

\item Suppose $\alpha$ is a $\pomove$-move in $u$ then necessarily $\beta$ is a $\opmove$-move.

We have $\alpha \in \pview{u \filter B,C}
= \pview{\pview{u} \filter B,C} \subseqof
\pview{u}$ (\cite[Lemma 3.3.1]{Harmer2005}).

Suppose that $n' \in \pview{u\filter A,C}$, then 
\begin{align*}
n'  \in \pview{u\filter A,C} 
& = \pview{u}\filter A,C   
& \mbox{(Lemma \ref{lem:pviewAC_eq_ACpview})} \\
& \suffixof \pview{u_{\prefixof \alpha}}\filter A,C
& \mbox{($\alpha \in \pview{u}$, $n'$ occurs before $\alpha$ in $u$)} \\
& = \pview{u_{\prefixof \alpha} \filter A,C} 
& \mbox{(Lemma \ref{lem:pviewAC_eq_ACpview})}\ .
\end{align*}
But since $n'$ occurs before $\alpha$ in $u$, the previous equation 
implies that $n' \in \pview{u_{\prefixof \alpha}\filter A,C}$
so we can apply Lemma \ref{lem:increasing_order}
on $u_{\prefixof \alpha}$: 
\begin{align*}
\ord_{A\fngamear C} n' 
& \leq \ord_{A\fngamear B} \alpha & \mbox{(Lemma \ref{lem:increasing_order} with $u\assignar u_{\prefixof \alpha}$)} \\
& = \ord_{B\fngamear C} \alpha & \mbox{($\alpha$ non initial in $B$)} \\
& \leq \ord_{B\fngamear C} m & \mbox{($\mu$ is P-i.j.)} \\
& = \ord_{A\fngamear C} m & \mbox{($m$ is not initial in $C$)} \ .
\end{align*}
\end{enumerate}


\item Suppose $n' =\omove_A$.

Suppose that $n' \in \pview{u\filter A,C}$, then by
Lemma \ref{lem:increasing_order} we have
$ \ord_{A\fngamear C} n'  \leq \ord_{B\fngamear C} m$.
 By Lemma \ref{lem:compositionorder}, since
 $m$ is not initial in $C$ we have $\ord_{B\fngamear C} m = \ord_{A\fngamear C} m$ thus $\ord_{A\fngamear C} n' \leq \ord_{A\fngamear C} m$.
\end{enumerate}

\item \label{case:mA} Suppose $m = \pmove_A$.
\begin{enumerate}[{\ref{case:mA}}.1)]

\item Suppose $n=\omove_A$.
This case is symmetrical to case
\ref{case:mC}) where $m=\pmove_C$ and $n=\omove_C$.

\item Suppose  $n$ is an $\omove_C$-move initial in $C$.
Then $m$ is initial in $A$ and points to a move
$b_0 = \opmove$ initial in $B$ which in turn points to the
move $n =\omove_C$ initial in $C$.

Let us assume that $n'\in \pview{u\filter A,C}$ and let us prove that necessarily $\ord_{A\fngamear C} n' \leq \ord_{A\fngamear C} m$.

	\begin{enumerate}[a.]
	\item Suppose $n'=\omove_A$. Then
since $m$ is initial in $A$ we clearly have 
$\ord_{A\fngamear C} n' \leq \ord_{A\fngamear C} m$.

	\item Suppose $n'=\omove_C$.
	If $n'$ occurs between $b_0$ and $m$ then $m$ points before $n'$ in 		$u$ so we can use Lemma \ref{lem:increasing_order} which gives $\ord_{A\fngamear C} n' \leq \ord_{A\fngamear B} m
		= \ord_{A\fngamear C} m$.
		\smallskip
		
		
		If $n'$ occurs before $b_0$ then 
		we cannot use Lemma \ref{lem:increasing_order}
		since $m$ does not point before $b_0$.
Up to now we have only used the fact that $\sigma$ and $\mu$ are P-i.j. The assumption that $\sigma$ is  \emph{closed} P-i.j. now becomes crucial. We have:
		\begin{align}
		\ord_{A\fngamear B} b_0 
		& = \max(\ord A +1, \ord_B b_0) & \mbox{($b_0$ initial in $B$)} \nonumber \\
		& \geq \ord_B b_0 \nonumber  \\
		& \geq \ord C & \mbox{($\sigma$ is closed P-i.j. \& $b_0 \in u \filter  B,C \in \mu$)} \nonumber  \\
		& \geq \ord_C n' +1 & \mbox{($n'$ not initial in $C$)} \nonumber  \\
		& = \ord_{A\fngamear C} n' +1 & \mbox{(Lemma \ref{lem:compositionorder} \& $n'$ not initial)}. \label{eq:b0gtnp}
		\end{align}
Thus:
		\begin{align*}
		\ord_{A\fngamear C} m 
		& = \ord_{A\fngamear B} m & \mbox{(Lemma \ref{lem:compositionorder})} \\
		& = \ord_{A\fngamear B} b_0 -1  & \mbox{($m$ justified by $b_0$ initial in $B$)} \\
		& \geq \ord_{B\fngamear C} b_0 -1 & \mbox{($b_0$ initial in $B$)} \\
		& \geq \ord_{A\fngamear C} n' & \mbox{(By Eq. \ref{eq:b0gtnp})} \ .
		\end{align*}
		\end{enumerate}

\end{enumerate} 
\end{enumerate} 

\noindent (II)
To show that $\sigma;\mu$ is closed P-i.j., we need to verify that for all initial move in $A$ occurring in some play of $\sigma ; \mu$ has order in $A$ greater or equal to $\ord C$.
Take an initial move $m \in A$ such that $m$ occurs in $s \in \sigma ; \mu$. Then $s = u \filter A,C$ for some $u \in \sigma 
\ \|\ \mu$. Let $n$ be the initial move of $B$ justifying $m$.
 We have:
\begin{align*}
\ord_A m & \geq \ord B & \mbox{($\sigma$ is closed P-i.j. \& $m \in u
\filter A,B \in \sigma$)} \\
 & \geq \ord_B n & \mbox{($n 
\in B$)} \\
 & \geq \ord C & \mbox{($\mu$ is closed P-i.j. \& $n \in u
\filter B,C \in \mu$)}.
\end{align*}
\end{proof}
We recall some definitions. Let $A$ and $B$ be two well-opened games. Given a strategy  $\sigma :\ !A \fngamear B$, its promotion $\sigma^\dag :\ !A\fngamear !B$ is defined as
$$ \sigma^\dag = \{ s \in L_{!A\fngamear !B}\ |\ \mbox{for all inital $m$ in $B$, } s\filter m \in \sigma \}$$
and for $\mu :\ !B\fngamear C$ the composite strategy $\sigma \fatcompos \mu$ is defined as:
$$ \sigma \fatcompos \mu = \sigma^\dag ; \mu \ .$$

\begin{proposition}
If $A$ and $B$ are two well-opened games 
and $\sigma :\ !A \fngamear B$ is a well-bracketed P-incrementally justified strategy then $\sigma^\dag$ is also well-bracketed and P-incrementally justified.
Furthermore if $\sigma$ is closed P-i.j. then so is
$\mu$.
\end{proposition}
\begin{proof}
$\sigma^\dag$ is well-bracketed by \cite[Proposition 2.10.]{abramsky94full}.

By the visibility condition,
\end{proof}

From the last two propositions we obtain:
\begin{corollary}
Let $A$ and $B$ be two well-opened games. Let
$\sigma :\ !A \fngamear B$ and $\mu :\ !B\fngamear C$ be two well-bracketed strategies then:
\begin{enumerate}
\item if $\sigma$ is closed P-i.j. 
and $\mu$ is P-i.j. then $\sigma \fatcompos \mu :\ !A \fngamear C$ is also P-i.j.
\item if $\sigma$ and $\mu$ are closed P-i.j. then so is $\sigma \fatcompos \mu :\ !A \fngamear C$.
\end{enumerate}
\end{corollary}

\section{Compositionnality - Syntactic approach}

\subsection{Definability result for Safe PCF}

We say that a PCF term is \defname{semi-safe} if it is of the form $N_0 N_1 \ldots N_k$ for $k\geq 1$ where each of the $N_i$ is safe or if it can be written $\lambda \overline{x} . N$ for some safe term $N$. Semi-safe terms are either safe or ``almost safe'' in the sense that they can be turned into an equivalent (i.e.~with isomorphic game semantics) safe term  by performing $\eta$-expansions. Indeed, let $M$ be an semi-safe term that is unsafe.
If $M$ is of the first form $N_0 N_1 \ldots N_k : (A_1,\ldots,A_n)$ with $k\geq 1$ then let $\varphi_i:A_i$ for $i\in\{1..n\}$ be fresh variables, using the (app) and (abs) rules we can build the safe term $\lambda \varphi_1 \ldots \varphi_n . N_0 N_1 \ldots N_k \varphi_1 \ldots \varphi_n$. If $M$ is of the second form $\lambda \overline{x} . N$ then using the abstraction rule we can build the equivalent safe term $\lambda \overline{y} \overline{x}. N$  where $\overline{y} = fv(\lambda \overline{x}. N)$.

The $\beta$-normal form of a \pcf\ term is the possibly infinite term obtained by reducing all the redexes in $M$.

The correspondence between safety and P-incremental justification in the context of the simply typed lambda calculus was shown
in \cite[Theorem 3(ii)]{blumong:safelambdacalculus}:

\begin{theorem}[\cite{blumong:safelambdacalculus},Theorem 3(ii)]
\label{thm:safeincrejust} In the simply typed lambda calculus:
\begin{enumerate}[(i)]
\item If $M$ is safe then $\sem{M}$ is P-incrementally justified.
\item If $M$ is a closed term and $\sem{M}$ is
  P-incrementally justified then the $\eta$-long form of the
  $\beta$-normal form of $M$ is safe.
\end{enumerate}
\end{theorem}
In fact the proof of this theorem can be easily adapted to show a more precise result:
\begin{theorem}[Semi-safety and P-incremental justification]
\label{thm:semisafeincrejust} Let $\Gamma \vdash M : A$ be a simply typed term. Then:
\begin{enumerate}[(i)]
\item If $\Gamma \vdash M : A$ is semi-safe then $\sem{\Gamma \vdash M : A}$ is P-incrementally justified.
\item If $\sem{\Gamma \vdash M : A}$ is
  P-incrementally justified then 
$\etanf{\betanf{M}}$ is semi-safe if $M$ is open
and safe if $M$ is closed.
\end{enumerate}
\end{theorem}



In the context of \pcf\, only the first part of the theorem holds (see \cite{blumtransfer} for the proof). However (ii) does not hold. Indeed, take the closed \pcf\ term $M = \lambda f x y. f (\lambda z. \pcfcond (\pcfsucc\ x) y z )$ where $x,y,z:o$ and $f:((o,o),o)$. $M$ is in normal form (the conditional  could  only be reduced if $x$ were first evaluated). The $\eta$-long form of the $\beta$-normal form of $M$ is therefore $M$ itself which is unsafe.
But clearly we have $\sem{M} = \sem{\lambda f x y. f (\lambda z. z)}$, and since  $\lambda f x y. f (\lambda z. z)$ is safe, by (i), $\sem{M}$ is P-incrementally justified.

Such counter-example arises because the conditional operator of \pcf\ permits us to build terms in normal form containing ``dead code'' {\it i.e.}~some subterm that will never be evaluated for any value of M's parameters. In the example given above, the dead code consists in the subterm $y$. In general, if the dead code part of the computation tree contains a variable that is not incrementally bound then the resulting term will be unsafe even if the rest of the tree is incrementally bound.
In the example above, it was possible to turn $M$ into the equivalent safe term $\lambda f x y. f (\lambda z. z)$ by eliminating the dead code from $M$.
We shall see how to generalise this to any \pcf\ term with a P-incrementally justified denotation.

Dead code elimination can be difficult to achieve in practice but the formal definition is not difficult to formulate. We say that a subterm $N$ occurring
in a context $C[-]$ in $M : (A_1, \ldots, A_n,o)$ is part of the \defname{dead code} of $M$ if for any term $T_0$ of the form $M M_1 \ldots M_n$,
any reduction sequence starting from $T_0$ does not involve a reduction of the subterm $N$ {\it i.e.}~for any reduction sequence $T_0 \redar T_1 \redar \ldots \redar T_k$, there is no $j\in \{0.. k-1\}$ such that $T_j = C[N]$ and $T_{j+1} = C[N']$ for some term $N'$.
 

Let $M$  be a \pcf\ term in $\eta$-nf.
An occurrence of a variable $x$ in $M$ is said to be a \defname{dead occurrence}
if it occurs in the dead code of $M$. In other words, it is a
dead occurrence of $x$ if the corresponding node in the computation tree does not appear in any traversal of $\travset(M)$. Equivalently, thanks to the Correspondence Theorem, an occurrence of $x:B$ is dead if and only if the initial move
of the arena $\sem{B}$ does not appear in any play of $\sem{M}$.


We define $M^*$ as the term obtained from $M$ after substituting all subterms of the form  $x N_1 \dots N_k$ for some dead variable occurrence $x:(B_1,\ldots, B_k, o)$ by the constant $0$. This process is called \defname{dead variable elimination}.
Note that if $M$ is in $\eta\beta$-nf then so is $M^*$.

We also write $\tau(M)^*$ to denote the equivalent transformation on the computation tree. Since the computation tree is constructed from the $\eta$-nf of $M$, we will use this notation even when $M$ is not in $\eta$-nf.



\begin{proposition}[Incremental-binding and P-incremental justification coincide] \
\label{prop:incrbound_imp_incrjustified_pcf} Let $\Gamma \vdash M : A$ be a PCF term in $\beta$-normal form.
\begin{enumerate}[(i)]
\item  If $\tau(\Gamma \vdash M : A)$ is incrementally-bound then $\sem{\Gamma \vdash M : A}$ is P-incrementally justified,
\item  if $\sem{\Gamma \vdash M : A}$ is P-incrementally justified
then $\tau(\Gamma \vdash M : A)^*$ is incrementally-bound.
\end{enumerate}
\end{proposition}
\begin{proof}
(i) The proof is exactly the same as in the simply typed lambda calculus case,
see \cite[Proposition 4.1.5(i)]{blumtransfer}.

\noindent (ii)
Take $\Gamma \vdash M : A$ a \pcf\ term in $\beta$-normal form denoted by $\sem{\Gamma \vdash M : A}$ P-incrementally justified. Let $r$ denote the root of $\tau(M)^*$.
Let $n$ be a node of $\tau(M)^*$ labelled by the variable $x$.
$\tau(M)^*$ is free from dead code therefore $n$ is not a dead occurrence of $x$ and there exists a traversal of $\tau(M)^*$ of the form $t \cdot x$.

\pcf\ constants are of order $1$ at most therefore they cannot hereditarily justify a variable node, thus $x$ is necessarily hereditarily justified by the root $r$ of the computation tree.


By considering $t\cdot x$ as a traversal of $\tau(M)$,  the correspondence theorem gives $\varphi((t \cdot x) \upharpoonright r) = \varphi((t \upharpoonright r) \cdot x) \in \sem{M}$. Since $\sem{M}$ is P-incrementally justified, $\varphi(x)$ must point to the last O-move in $\pview{?(\varphi(t \upharpoonright
r))}$ with order strictly greater than $\ord{\varphi(x)}$.
Consequently $x$ points to the last node in $\pview{?(t
\upharpoonright r)} \inter N^{\lambda}$ with order strictly greater than $\ord{x}$. We have:
\begin{align*}
\pview{?(t \upharpoonright r)} &= \pview{?(t) \upharpoonright r} = \pview{?(t)} \upharpoonright r & (\mbox{by \cite[lemma 3.1.23]{blumtransfer}}) \\
& = [r,x[ \ \upharpoonright r & (\mbox{by \cite[proposition 3.1.20]{blumtransfer}})
\end{align*}
Since $M$ is in $\beta$-nf, the set of nodes not hereditarily justified by $r$ is exactly the set of nodes hereditarily justified by $N_{\Sigma}$ thus
$[r,x[ \ \upharpoonright r = [r,x[\ \setminus\  N^{\upharpoonright \Sigma}$.
Moreover \pcf\ constants are of order $1$ at most therefore $N^{\upharpoonright \Sigma} = N_{\Sigma} \union N^c_{\Sigma}$
where $N^c_{\Sigma}$ is the set of children nodes of $N_{\Sigma}$.
Thus $(\pview{?(t \upharpoonright r)}\upharpoonright r) \inter N^{\lambda} =
([r,x[\ \setminus\  N_{\Sigma} \setminus N^c_{\Sigma} ) \inter N^{\lambda} =
([r,x[\ \setminus\  N^c_{\Sigma} )  \inter N^{\lambda}$, and
since $N^c_{\Sigma}$ is constituted of order $0$ lambda-nodes only we have that
$x$ points to the last node in $[r,x[ \inter N^{\lambda}$ with order strictly greater than $\ord{x}$.

Hence if $x$ is a bound variable node then it is bound by the
last $\lambda$-node in $[r,x[$ with order strictly greater than
$\ord{x}$ and if $x$ is a free variable then it points to $r$ and
therefore all the $\lambda$-node in $]r,x[$ have order smaller than
$\ord{x}$. Thus $\tau(M)^*$ is incrementally-bound.
\end{proof}

The counterpart of Lemma 4.1.6 from
\cite{blumtransfer} can be stated as follows in the context of PCF:
\begin{lemma}[Semi-safety and incrementally-binding]
\label{lem:safe_imp_incrbound_pcf} Let $\Gamma \vdash M : A$ be a PCF term.
\begin{itemize}
\item[(i)] If $\Gamma \vdash M : A$ is a semi-safe term then $\tau(\Gamma \vdash M : A)$ is incrementally-bound ;
\item[(ii)] conversely, if $\tau(\Gamma \vdash M : A)$ is incrementally-bound then the $\eta$-normal form of $\Gamma \vdash M : A$ is semi-safe if $M$ is open and safe if $M$ is closed.
\end{itemize}
\end{lemma}
The proof can be obtained by adapting the proof 
of Lemma 4.1.6 from \cite{blumtransfer}.

\begin{theorem}[Semi-safety and P-incremental justification]
\label{thm:semisafeincrejust_pcf} Let $\Gamma \vdash M : A$ be a PCF term. Then:
\begin{enumerate}[(i)]
\item If $\Gamma \vdash M : A$ is semi-safe then $\sem{\Gamma \vdash M : A}$ is P-incrementally justified.
\item If $\sem{\Gamma \vdash M : A}$ is
  P-incrementally justified then $\etanf{\betanf{M}}^*$ is semi-safe  if $M$ is open, and safe if $M$ is closed.
\end{enumerate}
\end{theorem}

\begin{proof}
\noindent(i)
A proof of this is given in the proof of Theorem 4.2.10 in \cite{blumtransfer}.

\noindent(ii) 
Suppose $M$ is a \pcf\ term with a P-incrementally justified strategy denotation. By Proposition \ref{prop:incrbound_imp_incrjustified_pcf}(ii), $\tau(\betanf{M})^* = \tau(\etanf{\betanf{M}}^*)$ is incrementally-bound.
If $M$ is closed then so is $\etanf{\betanf{M}}^*$ therefore by Lemma \ref{lem:safe_imp_incrbound_pcf}, $\etanf{\etanf{\betanf{M}}^*} = \etanf{\betanf{M}}^*$ is safe. If $M$ is open then so is $\etanf{\betanf{M}}^*$ and by Lemma \ref{lem:safe_imp_incrbound_pcf}, $\etanf{\etanf{\betanf{M}}^*} = \etanf{\betanf{M}}^*$ is semi-safe.
\end{proof}


We write \pcf' to denote the language obtained by extending \pcf\
with the $\pcfcase_k$ construct (see \cite{Abr02}).
The $\pcfcase_k$ construct is the obvious generalisation of the
conditional operator \pcfcond\ to $k$ branches instead of $2$. All the results obtained so far concerning Safe \pcf\ (including those
cited from \cite{blumtransfer}) can clearly be transposed to \pcf'.

The previous theorem leads to the following definability result for safe \pcf':
\begin{proposition}[Definability for safe \pcf' terms]
\label{prop:safetydefinability}
Let $\overline{A}=(A_1,\ldots, A_i)$ and $B =(B_1, \ldots, B_l,o)$ be two PCF types for some $i,l\geq 0$ and $\sigma$ be a well-bracketed innocent
P-i.j. strategy with finite view function defined on the game $!A_1 \otimes \ldots \otimes !A_i \fngamear (!B_1 \fngamear \ldots \fngamear !B_l \fngamear o) $. There exists a \emph{semi-safe} PCF' term $\overline{x} : \overline{A} \vdash M : B$ in $\eta$-long normal form such that:
$$ \sem{\overline{x} : \overline{A} \vdash M_\sigma : B} = \sigma $$
and a safe closed PCF' term $\vdash_s M'_\sigma : (\overline{A},B)$ in $\eta$-long normal form such that:
$$ \sem{\vdash M'_\sigma : (\overline{A},B)} \cong \sigma \ .$$
\end{proposition}
\begin{proof}
By the standard definability result for PCF', there is a term $\overline{x} : \overline{A} \vdash N : B$ such that $\sem{\overline{x} :\overline{A} \vdash N : B} = \sigma$.
Take $M_\sigma$ to be $\etanf{\betanf{N}}^* $. We have $\sem{\overline{x} : \overline{A} \vdash M_\sigma : B} =  \sem{\overline{x} :\overline{A} \vdash N : B} = \sigma$ and by Theorem  \ref{thm:semisafeincrejust_pcf}(ii), $M_\sigma$ is semi-safe.
For the second part we just need to take $M'_\sigma = \lambda \overline{x}. M_\sigma$.
\end{proof}


\subsection{Composition of P-i.j. strategies}

Let $\overline{A} = (A_1, \ldots, A_i)$,
$B = (B_1, \ldots, B_l,o)$ 
and $C=(C_1,\ldots,C_k,o)$ be three PCF types
for some $i\geq 1,l,k\geq 0$. Let
$f:\ !A_1 \otimes \ldots \otimes !A_i \fngamear B$ and $g:\ !B\fngamear C$ be two innocent well-bracketed and P-incrementally justified strategies with finite view function.
We would like to find under which conditions the composition $f\fatcompos g$ is also P-incrementally justified.

By the definability result, there are two closed safe terms (in $\eta$-nf) $\vdash M_f :(\overline{A},B)$  and $\vdash M_g :B \typear C$ such that $\sem{M_f} = f$
and $\sem{M_f} = g$.
We define the term $M_{f\fatcompos g} = \lambda \overline{x} . M_g (M_f \overline{x})$ for some fresh variables $\overline{x} : \overline{A}$. Clearly we have $\sem{M_{f\fatcompos g}} = \sem{M_f} \fatcompos \sem{M_g} = f\fatcompos g$.

By Theorem \ref{thm:semisafeincrejust_pcf}, we know that $f\fatcompos g$ is P-incrementally justified just when $\etanf{\betanf{M_{f\fatcompos g}}}^*$ is safe. 
We will now exploit this fact to extract a sufficient condition on the types $A$ and $B$ for 
the composition of $f$ and $g$ to be P-incrementally justified.

The term $M_f$ and $M_g$, being in $\eta$-nf, are of the following forms:
\begin{eqnarray*}
\vdash M_f &=& \lambda x_1^{A_1} \ldots x_i^{A_i} \varphi_1^{B_1} \ldots \varphi_l^{B_l} . N_f^o\\
\vdash  M_g &=& \lambda y^{ (B_1, \ldots, B_l,o)} \phi_1^{C_1} \ldots \phi_k^{C_k} . N_g^o
\end{eqnarray*}
for some distinct variables $x_1, \ldots, x_i$, $y$, $\varphi_1, \dots \varphi_l$, $\phi_1, \dots \phi_k$  and $\eta$-normal terms $N_f$ and $N_g$:
\begin{eqnarray*}
x_1:A, \ldots, x_i:A_i, \varphi_1:B_1, \dots, \varphi_l:B_l &\vdash& N_f :o \\
y: (B_1, \ldots, B_l,o), \phi_1:C_1, \dots, \phi_l:C_l &\vdash& N_g :o
\end{eqnarray*}


 
The fact that $M_f$ and $M_g$ are safe does not imply that $M_{f\fatcompos g}$ is: take $M_f = \lambda x^o z^o.x$ and $M_g = \lambda y^{(o,o)} . y a$ for some constant $a\in \Sigma$, then $\lambda x:A . M_g (M_f x) = \lambda x . (\lambda y . y a) ( \underline{(\lambda x z.x) x} )$ is unsafe because of the underlined subterm. However we have:
\begin{align*}
f\fatcompos g &= \sem{\lambda \overline{x} . M_g (M_f  \overline{x})} \\
 &= \sem{\lambda \overline{x} . (\lambda \phi_1\ldots \phi_k . N_g) [(M_f \overline{x}) / y]} \\
&= \sem{\lambda \overline{x} \phi_1 \dots \phi_k. N_g [(M_f  \overline{x}) / y]}
& \mbox{(the $x_j$'s and $\phi_j$'s are disjoint)}.
\end{align*}

We now concentrate on the term  $\lambda \overline{x} \phi_1 \dots \phi_k. N_g [(M_f  \overline{x}) / y]$ and try to find a sufficient condition guaranteeing its safety.
\subsubsection{A sufficient condition}
\begin{lemma}
Suppose that $\Gamma,y:B \vdash M$ is a safe term in $\eta$-nf and $\Gamma \vdash R : B$ is an almost safe application. Let $N$ denote the set of nodes of the computation tree $\tau(M)$. We have:
\begin{align*}
\Gamma \vdash M[R/y] :A \mbox{ safe } 
\iff&  \forall x \in fv(R) . \\
    & \forall n_y \in N_{fv} \mbox{ labelled $y$}.
      \forall m \in N_{\lambda} \inter ]r,n_y] : \ord{m} \leq \ord{x}
\end{align*}
\end{lemma}
\begin{proof}
Since $M$ is in $\eta$-nf, all the application to the variable $y$ are total (i.e.~of the form $y P_1 \ldots P_l :o$). Hence after substituting the safe term $N$ for $y$ in $M$, the only possible cause of unsafety is when
some variable free in $N$ becomes not safely bound in $\tau(M)$.
\end{proof}

Applying this lemma with $R= M_f \overline{x}$ gives us a sufficient condition (in the right-hand side of the equivalence) 
for $\lambda x \phi_1 \dots \phi_k. N_g [(M_f \overline{x}) / y]$ to be safe , and hence for $f\fatcompos g$ to be P-incrementally justified. Of course it is not a necessary condition since the 
eta-beta normal form of  $N_g[(M_f \overline{x}) /y]$ can be safe even though it is unsafe.

\subsubsection{A simpler sufficient condition}
\begin{lemma}
If $y:B, \Sigma \vdash N : T$ and $\vdash M : (\overline{A}, B)$ 
are safe terms with $\ord{A_i} \geq \ord{B}$ for all $i\in 1..n$
then $\overline{x}:\overline{A}, \Sigma \vdash N[(M \overline{x})/y] :T$ is also safe.
\end{lemma}
\begin{proof}
Since $\ord{x_i} = \ord{A_i} \geq \ord{B} = \ord{M \overline{x}}$, we can use the application 
rule of the safe lambda calculus to form the safe term $\overline{x}:\overline{A} \vdash M \overline{x}$.
Using the substitution lemma we have that $N[(M \overline{x})/y]$ is safe.
\end{proof}

Hence if we have $\ord{A_i}\geq\ord{B}$
for all $1 \leq i \leq n$,
then by the previous lemma the term $\vdash \lambda \overline{x} \phi_1 \dots \phi_k. N_g [(M_f \overline{x}) / y]$
is safe and therefore its denotation $\sem{\vdash \lambda \overline{x} \phi_1 \dots \phi_k. N_g [(M_f \overline{x}) / y]} = f\fatcompos g$ is P-incrementally justified.
This gives us a sufficient condition for $f\fatcompos g$ to be P-incrementally justified.

This condition is not necessary: Take $A=o$, $B=(o,o)$, $C=(o,o)$ and consider the two safe terms $M_f = \lambda x^A u^o.u$ and $M_g = \lambda y^B . y a$ for  some constant $a:o$. Then we have $M_{f\fatcompos g} = \lambda x . a$ which is safe hence $f\fatcompos g$ is P-incrementally justified although $\ord{A} < \ord{B}$.





\subsection{Two P-i.j. strategies whose composition is not P-i.j.}
\subsubsection{First attempt}

Take the types $A=o$, $B=(o,o)$, $C=o$, the variables
$x,u,v:o$, $y:B$ and $\varphi:((o,o),o)$ and $\Sigma$-constant $a:o$.
Consider the two safe terms $\vdash_s  M_f = \lambda xv.x : A\typear B$ and $\vdash_s M_g = \lambda y . \varphi (\lambda u . y a) : B\typear C$.
The $\eta\beta$-nf of $M_{f\fatcompos g}$ is $\vdash \lambda x . \varphi (\underline{\lambda u . x})$ which is unsafe because of the underlined term. It is then tempting to use
Theorem \ref{thm:safeincrejust}(ii) to conclude that
$\sem{M_{f\fatcompos g}}$ is not P-incrementally justified. However we cannot use it since $M_g$ contains an order $2$ constants ($\varphi$) and therefore
$M_{f\fatcompos g}$ is not a valid simply typed $\lambda$-term (nor a \pcf-term).

\subsubsection{Second attempt}
The previous example can be easily changed into a working counter-example: we just need to elevate $\varphi$ from the status of constant to variable.

Take $A=o$, $B=(o,o)$, $C=(((o,o),o),o)$, the variables
$x,u,v:o$, $y:B$ and $\varphi:((o,o),o)$ and the $\Sigma$-constant $a:o$. Consider the two safe terms $\vdash_s  M_f = \lambda xv.x : A\typear B$ and  $\vdash_s M_g = \lambda y \varphi. \varphi (\lambda u . y a) : B\typear C$.
The $\eta\beta$-nf of $M_{f\fatcompos g}$ is $\vdash \lambda x \varphi. \varphi (\underline{\lambda u . x})$ which is unsafe because of the underlined term, thus by Theorem \ref{thm:safeincrejust}(ii), $\sem{M_{f\fatcompos g}}=\sem{M_f} \fatcompos
\sem{M_g}$ is not P-incrementally justified. The following diagram illustrates a play that is not P-i.j.:
\begingroup
\def\sigcol#1{{\color{gray} #1}}
\def\mucol#1{{\color{red} #1}}
$$\begin{array}{ccccccccc}
A &  & \multicolumn{2}{c}{B} && \multicolumn{4}{c}{C}\\
\cline{1-1} \cline{3-4} \cline{6-9}
o & \stackrel{\sigcol{\sem{M_f}}}\longrightarrow & o, & o & \stackrel{\mucol{\sem{M_g}}}\longrightarrow & ((o, &o),& o),& o \\ \\
&&&&&&&&\rnode{n0}{\lambda x \varphi \omove  \mucol {\lambda y \varphi}}\\
&&&&&&&\rnode{n1}{\varphi  \pmove \mucol \varphi}\\
&&&&&&\rnode{n2}{\lambda u \omove  \mucol {\lambda u}} \\
&&&  \rnode{n3}{\omove \sigcol {\lambda x v} \pmove \mucol y} \\
\rnode{n4}{x \pmove \sigcol x}
\end{array}
\ncarc[arcangleA=20,arcangleB=20,linecolor=black]{->}{n4}{n0}
\ncarc[arcangleA=30,arcangleB=20,linecolor=red]{->}{n2}{n1}
\ncarc[arcangleA=30,arcangleB=20,linecolor=red]{->}{n1}{n0}
\ncarc[arcangleA=20,arcangleB=20,linecolor=red]{->}{n3}{n0}
\ncarc[arcangleA=20,arcangleB=20,linecolor=gray]{->}{n4}{n3}
$$
\endgroup

\subsubsection{Another counter-example where $\ord{B} = \ord{C}$}

Let $A=o$, $B=C=(((o,o),o),o)$ and let $x:A$, $y:B$, $u:o$, $v,\varphi:((o,o),o)$
and $g:(o,o)$ be variables and  $a:o$ be a $\Sigma$-constant. Take the two safe terms $\vdash  M_f = \lambda x v.x$ and $\vdash M_g = \lambda y \varphi. \varphi (\lambda u . y (\lambda g. a))$.
The $\eta\beta$-nf of $M_{f\fatcompos g}$ is $\vdash \lambda x \varphi. \varphi (\underline{\lambda u . x})$ which is unsafe because of the underlined term, so
$f\fatcompos g$ is not P-incrementally justified.
 
\bibliographystyle{plain}
\bibliography{../bib/higherorder,../bib/gamesem,../bib/lambdacalculus}


\section{P-i.j. and IA}
$var = acc \times exp = com^{\omega}\times exp$

Any strategy on the game $I \fngamear\ !var$ is P-i.j. since there is no P-question in the arena $var$.


\end{document}


\psset{linecolor=darkGreen,linewidth=0.5pt}


\author{William Blum}
\title{P-incrementally justified strategies}

\begin{document}
\maketitle 

\section{Well-bracketing and P-incremental justification}

We consider an arena $A$ and make the following two assumptions on it:
\begin{itemize}
\item (A1) For $A \neq \bot$ (the arena with a single initial question), each question move in the arena enables at least one answer move.
\item (A2) Answer moves do not enable any other move.
\end{itemize}

We define the \defname{order} of a move $m$ in the arena $A$, written $\ord_A{m}$ (or just $\ord{m}$ where there is no ambiguity), as the length of the path from $m$ to its furthest leaf in $A$ minus 1
({\it i.e.}~the height of the subarena rooted at $m$ minus 2.).
Because of assumptions (A1) and (A2),
for any move $m$ of $A \neq \bot$, $m$ is a question move if and only if $\ord{m} \geq 0$, and $m$ is an answer move if and only if $\ord{m} = -1$.





We call \defname{pending question} of a sequence of moves $s \in L_A$ the last unanswered question in $s$.

\begin{definition}\rm
A strategy $\sigma$ is said to be \defname{P-well-bracketed} if for any play $s \, a \in \sigma$ where $a$ is a  P-answer, $a$ points to the pending question in $s$.
\end{definition}



P-well-bracketing can be restated differently as the following proposition shows:
\begin{proposition}
\label{prop:char_wellbrack}
\rm We make assumption (A1) and (A2).
Let $\sigma$ be a strategy on an arena $A\neq \bot$.
The following statements are equivalent:
\begin{enumerate}
\item[(i)] $\sigma$ is P-well-bracketed,
\item[(ii)] for $s \, a \in \sigma$ with $a$ a P-answer, $a$ points to the pending question in $\pview{s}$,
\item[(iii)] for $s \, a \in \sigma$ with $a$ a P-answer, $a$ points to the last O-question in $\pview{s}$,
\item[(iv)] for $s \, a \in \sigma$ with $a$ a P-answer, $a$ points to the last O-move in $\pview{s}$ with order $>\ord{a}$.
\end{enumerate}
\end{proposition}
\begin{proof}
$(i)\iff(ii)$: \cite[Lemma 2.1]{McC96b} states that if P is to move then the pending question in $s$ is the same as that of $\pview{s}$.

$(ii)\iff(iii)$: Assumption (A2) implies that the pending question in $\pview{s}$ is also the last O-question occurring in $\pview{s}$.

$(iii)\iff(iv)$: Because of assumption (A1) and (A2),
for any move $m$, we have $m$ is a question move
if and only if $\ord{m} \geq 0$ if and only if $\ord{m} > \ord{a} = -1$.
\end{proof}




\begin{lemma}
\label{lem:justfied_by_unanswered}
Under assumption (A2), if $s$ be a justified sequence of moves satisfying alternation and visibility then any O-move (resp. P-move) in $s$ points to an \emph{unanswered} P question (resp. O-question).
\end{lemma}
\begin{proof}
Suppose that an O-move $c$ points to a P-move $d$ that has already been answered by the O-move $a$. The sequence $s$ as the following form:
$$ s= \ldots \Pstr{(d){d}  \ldots  (a-d,20){a}  \ldots  (c-d,20){c}}$$

By O-visibility, $d$ must belong to $\oview{s_{<c}}$. But since $a$ is an answer, by assumption (A2), it cannot justify any P-move, therefore
$\oview{s_{<q}}$ must contain an OP-arc ``hoping'' over $a$. We name the nodes of this arc $d^1$ and $c^1$:
$$ s = \ldots \Pstr[0.7cm]{(d){d}  \ldots  (d1){d^1} \ldots (a-d,20){a} \ldots
 (c1-d1,20){c^1} \ldots (c-d,25){c}}$$

By P-visibility, $d^1$ must belong to $\pview{s_{<c^1}}$. Consequently, $a$ does not belong to $\pview{s_{<c^1}}$ (otherwise the PO-arc $\Pstr[0.5cm]{(d){d} \quad (a-d,45){a}}$ would cause the P-view to jump over $d^1$).
Therefore there must be a PO-arc $\Pstr[0.5cm]{(d2){d^2} \quad (c2-d2,45){c^2}}$ in $\pview{s_{<c^1}}$ hoping over $a$:
$$ s = \ldots \Pstr[0.7cm]{(d){d}  \ldots
(d1){d^1} \ldots (d2){c^2} \ldots
(a-d,20){a} \ldots
 (c2-d2,20){d^2} \ldots (c1-d1,20){c^1} \ldots (c-d,25){c}}$$

This process can be repeated infinitely often by using alternatively O-visibility and P-visibility. This gives a contradiction since the sequence of moves $s_{<c}$ has finite length.
Hence $d$ cannot point to a question that has already been answered. Since, by assumption (A2), a question is enabled by another question, $d$ is necessarily justified by an unanswered question.
\end{proof}


\begin{lemma}
\label{lem:oq_in_pview_unanswered}
Under assumption (A2), if $s$ is a P-well-bracketed justified sequence of moves of odd length satisfying alternation and visibility then  all O-questions occurring in $\pview{s}$ are unanswered in $s$.
\end{lemma}
\begin{proof}
We proof the first part by induction on $s$.
The base case ($s = q$ with $q$ initial O-move) is trivial.

Suppose $\Pstr[0.4cm]{ s = s' \cdot (n)n \cdot u \cdot (m-n,45){m} }$.
Let $r$ be an O-question in $\pview{s} = \pview{s'} \cdot n \cdot m$.
If $r$ is the last move $m$ then it is necessarily unanswered.
If $r \in \pview{s'}$ then by the induction hypothesis, $r$ is unanswered in $s'$.
Suppose that $r$ is answered in $s$. This implies that some answer move $a$ in $u$ points to $r$:
$$\pstr[0.7cm][5pt]{ s = \underbrace{\cdots\ \nd(r){r}^O \cdots }_{s'} \
\nd(n){n}^P \ \underbrace{\cdots\ \nd(a-r,35){a}^P \cdots }_{u} \
\nd(m-n,30){m}^O } \ .$$
 
Since $m$ points to $n$, by lemma \ref{lem:justfied_by_unanswered}, $n$ is still unanswered at $s_{\prefixof a}$. Therefore the pending
question at $s_{\prefixof a}$ cannot be $r$. But $a$ is justified by $r$, therefore the well-bracketing condition is violated. Hence $r$ is
unanswered in $s$.
\end{proof}





\begin{definition}\rm
A play $s m$ of even length is said to be \defname{P-incrementally justified}, or {\emph P-i.j.} for short, if $m$ points to the last unanswered O-question in $\pview{s}$ with order strictly greater than $\ord{m}$.

 A strategy $\sigma$ is said to be \defname{P-incrementally justified}, if all plays in $\sigma$ ending with a P-question are
P-incrementally justified.
\end{definition}

\begin{proposition}
\label{prop:char_pincr}
\rm We make assumption (A1) and (A2).
Let $\sigma$ be a \emph{P-well-bracketed} strategy on an arena $A\neq \bot$.
The following statements are equivalent:
\begin{enumerate}
\item[(i)] $\sigma$ is P-incrementally justified,
\item[(ii)] for $s \, q \in \sigma$ with $q$ a P-question, $q$ points to the last O-question in $\pview{s}$ with order $>\ord{q}$,
\item[(iii)] for $s \, q \in \sigma$ with $q$ a P-question, $q$ points to the last O-move in $\pview{s}$ with order $>\ord{q}$.
\end{enumerate}
\end{proposition}
\begin{proof}
$(i)\iff(ii)$: By lemma \ref{lem:oq_in_pview_unanswered}, O-question occurring in $\pview{s}$ are all unanswered.

$(ii)\iff(iii)$: Because of (A1) and (A2), $\ord{q} \geq 0$ thus an O-move with order $>\ord{q}$ is necessarily an O-question.
\end{proof}

Putting proposition \ref{prop:char_pincr} and
\ref{prop:char_wellbrack} together we obtain:
\begin{proposition}
Under assumption (A1) and (A2).
A strategy $\sigma$ on $A\neq \bot$
is \emph{P-well-bracketed} and
 \emph{P-incrementally justified} if and only if
for $s \, m \in \sigma$, $m$ points to the last O-move in $\pview{s}$ with order $>\ord{m}$.
\end{proposition}


\section{Remarks}
\subsection{Homogeneity constraint}

Type homogeneity is not preserved after composition. Indeed the types  $o \typear (o \typear o)$ and $(o \typear o) \typear \left((o \typear o) \typear o \right)$ are homogeneous
but $o \typear \left((o \typear o) \typear o\right)$ is not.

If $A\typear B$ and $B \typear C$ are homogeneous types then  a sufficient condition for $A\typear C$ to be homogeneous is  ``$\ord{A} \geq \ord{B}$''.


\section{Compositionality - A semantic proof}

\subsection{Preliminaries}
 
\subsubsection{Nodes order after composition}

Consider the arena $X\fngamear Y$
and let $m$ be a move of $X\fngamear Y$.
We write $\ord_{X\fngamear Y}{m}$
to denote the order of 
$m$ in the arena ${X\fngamear Y}$.
If $m$ belongs to $X$ (resp.~$Y$) then
we write $\ord_X{m}$ 
(resp.~$\ord_Y{m}$) to denote the order of the move $m$ in the arena $X$ (resp.~$Y$).

\begin{lemma}
\label{lem:compositionorder}
Let $A$, $B$ and $C$ be three arenas. We have:
$$\begin{array}{lll}
\forall m \in A:
    &  \ord_{A\fngamear B}{m} = \ord_{A\fngamear C}{m} \ ,\\
\forall m \in B:
    & \ord_{A\fngamear B}{m} \geq \ord_{B\fngamear C}{m}  & \mbox{for $m$ initial,}\\
    & \ord_{A\fngamear B}{m} = \ord_{B\fngamear C}{m} & \mbox{for $m$ non initial,} \\
\forall m \in C:
    & \ord_{A\fngamear C}{m} \geq \ord_{B\fngamear C}{m} \iff
\ord{A} \geq \ord{B}\ & \mbox{for $m$ initial,}\\
    & \ord_{A\fngamear C}{m} = \ord_{B\fngamear C}{m}   & \mbox{for $m$ non initial.}
\end{array}
$$
\end{lemma}





\subsubsection{Interaction sequences}
Let us first recall the definition of an interaction sequence.
Let $A$,$B$ and $C$ be three games. 
We say that $u$  is an \defname{interaction sequence} of $A$,$B$ and $C$ whenever $u\filter A,B$ is a valid position of the game $A\fngamear B$
(i.e.~$u\filter A,B \in P_{A\fngamear B}$) 
and  $u\filter B,C$ is a valid position of the game
$B\fngamear C$. We write $Int(A,B,C)$ to denote
the set of all such interaction sequences.

Let $\sigma:A\fngamear B$ and $\mu:B\fngamear C$
be two strategies. We write $\sigma \parallel \mu$ to denote the 
set of interaction sequences that unfold according to the strategy $\sigma$ in the $A,B$-projection of the game and 
to $\mu$ in the $B,C$-projection:
$$ \sigma \parallel \mu = \{ u\filter A,B \in \sigma \vee u \filter B,C \in \mu \} \ .$$
The composite of $\sigma$ and $\mu$ is then defined as $\sigma ; \mu = \{ u \filter A,C \ | \ u \in \sigma \parallel \tau \}$.

The diagram below shows the structure of an interaction sequence
from $\sigma \parallel \mu$. There are four states represented by the rectangular boxes. The content of the state shows who is to play in each of the game $A\fngamear B$, $B\fngamear C$ and $A\fngamear C$.
For instance in state $OPP$, it is O's turn to play in 
$A\fngamear B$ and P's turn to play in $B\fngamear C$ and $A\fngamear C$. Arrows represent the moves.
When specifying interaction sequence,
the following bullet symbols are used to represent moves:
$\pmove$ for P-moves, $\omove$ for O-moves, $\pomove$ 
for a move playing the role of P in $A\fngamear B$
and O in $B\fngamear C$ and $\opmove$ for
the symmetric of $\pomove$.
We sometimes add a subscript to the symbols $\pmove$ and $\omove$ to denote the component in which the moves is played ($A$ or $C$).


\tikzstyle{state}=[rectangle,draw=blue!50,fill=blue!20,thick,minimum height = 4ex, text width=4cm]
\tikzstyle{move}=[->,shorten <=1pt,>=latex',line width=1pt]
\tikzstyle{intmove}=[dashed] 
\tikzstyle{extomove}=[color=\extomovecolor] 
\tikzstyle{genomove}=[]%[dashed]
\tikzstyle{genpmove}=[color=\genpmovecolor]
\def\sep{1.5cm} 
\begin{table}[htbp]
\begin{center}
\begin{tikzpicture}[node distance=1.7cm]

% the four states 
\path 
 node(oooT)  [state] {}
 node(opp)   [state, below of=oooT] {}
 node(pop)   [state, below of=opp]  {}
 node(oooB)  [state, below of=pop] {}
 node(title) [anchor=south, at=(oooT.north), minimum height = 4ex, text width=4cm] { };

\path
% text in the title centered in 3 columns
  ([xshift=-\sep]title) node {$A\fngamear B$}
        (title) node {$B\fngamear C$}
        ([xshift=\sep]title) node {$A\fngamear C$}

% text in the states centered in 3 columns
  ([xshift=-\sep]oooT) node {O}
        (oooT) node {O}
        ([xshift=\sep]oooT) node {O}
  ([xshift=-\sep]opp) node {O}
        (opp) node {P}
        ([xshift=\sep]opp) node {P}
  ([xshift=-\sep]pop) node {P}
        (pop) node {O}
        ([xshift=\sep]pop) node {P}
  ([xshift=-\sep]oooB) node {O}
        (oooB) node {O}
        ([xshift=\sep]oooB) node {O}

% text in between two arrows giving the arena of the move
  (oooT) to node {\bf C} (opp)
  (opp) to node {\bf B} (pop)
  (pop) to node {\bf A} (oooB)

% arrows representing the moves
  (opp.20)    edge[move, genpmove]
		node[right] {$\mu$}
		node[left]{$\pmove$} (oooT.-20)
  (oooT.-160) edge[move, extomove, genomove]
		node[left] {$env_\mu$}
		node[right]{$\omove$} (opp.160)
  (pop.20)    edge[move, genomove,genpmove,intmove]
		node[right] {$\sigma$}
		node[left]{$\pomove$} (opp.-20)
  (opp.-160)  edge[move, genomove, genpmove,intmove]
		node[left] {$\mu$} 
		node[right]{$\opmove$}  (pop.160)
  (oooB.20)   edge[move, extomove,genomove]
		node[right] {$env_\sigma$}
		node[left]{$\omove$} (pop.-20)
  (pop.-160)  edge[move, genpmove]
		node[left] {$\sigma$}
		node[right]{$\pmove$} (oooB.160);

%\draw[move, genpmove] (3.5cm,-1cm) -- +(1,0) node[right] {Generalised P-move \& External P-move };
%\draw[move, genomove,genpmove] (3.5cm,-2cm) -- +(1,0) node[right] {Generalised O-move \& Generalised P-move};
%\draw[move, genomove,extomove] (3.5cm,-3cm) -- +(1,0) node[right] {Generalised O-move \& External O-move};
\draw[move] (3.5cm,-1cm) -- +(1cm,0cm) node[right] {External move};
\draw[move,intmove] (3.5cm,-2cm) -- +(1cm,0cm) node[right] {Internal move};
\draw (3.5cm,-3cm) node[anchor=west] {\textcolor{\extomovecolor}{External O-moves: $\omove$}};
\draw (3.5cm,-4cm) node[anchor=west] {\textcolor{\genpmovecolor}Generalised P-move: $\opmove, \pomove, \pmove$};
\end{tikzpicture} 
\end{center}
\caption{Structure of an interaction sequence.}
\label{tab:interseq}
\end{table}

Note that in state OPP, the alternation condition (for each of the three games involved) prevents the players from playing in A. Indeed, the O-moves in component $A$ of $A\fngamear B$ are also $O$-moves in component $A$ of $A\fngamear C$ however the state name indicates that the next move in $A\fngamear B$ must be an O-move and the next move in $A\fngamear C$ must be a P-move.

Similarly, in the top state OOO, the players cannot make move in B since the O-moves in component B of the game $B\fngamear C$ correspond to P-moves in the component B of $A\fngamear B$. However the state name indicates that the next move in $A\fngamear B$ and the next move in $B\fngamear C$ must be played by O.


Let $u \in Int(A,B,C)$ and $m$ be a move of $u$.
The \defname{component} of $m$ is $A,B$ if 
after playing $m$ the game is under the control 
of the strategy $\sigma$ and $B,C$ otherwise (if $\mu$ has control).
In other words, the moves $\omove, \pmove \in A$
and $\opmove \in B$ shown on the diagram of Table \ref{tab:interseq}
have component $A,B$ and 
$\omove, \pmove \in C$ and $\pomove \in B$
have component $B,C$.


Also we call \defname{generalized O-move in component $A,B$}
moves that play the role of O in the game $A\fngamear B$, that is to say moves represented by $\opmove$ and $\omove_A$.
Similarly $\pomove$ and $\pmove_A$ moves are the \defname{generalized P-moves in component $A,B$},
$\omove_C$ and $\pomove$ moves are
the \defname{generalized O-moves in component $B,C$}
and  $\pmove_C$ and $\opmove$ moves are the \defname{generalized P-moves in component $B,C$}.

The P-view (also called {\emph core} in \cite{McCusker-GamesandFullAbstrac}) of an interaction sequence $u \in Int(A,B,C)$, written $\overline{u}$ or $\pview{u}$ is defined as:
\begin{align*}
\pview{u\cdot \extomove{n}} &= \extomove{n} &
\mbox{ if \extomove{$m$} is an \extomove{external O-move} initial in C,}\\
\pview{\Pstr{u\cdot (m)m\cdot v \cdot (n-m,45){\extomove{n}} }} &= \extomove{n} &\mbox{ if \extomove{$m$} is an \extomove{external O-move} non initial in C,}\\
\pview{u \cdot \genpmove{m}} &= \pview{u}\cdot \genpmove{m}  & \mbox{ if \genpmove{$m$} is a \genpmove{generalised P-move}.}\\ 
\end{align*}

We can show the following property by an easy induction :
\begin{lemma}
\label{lem:pviewAC_eq_ACpview}
 Let $u$ be an interaction sequence in $Int(A,B,C)$ then
$$\pview{u} \upharpoonright A,C = \pview{u \upharpoonright A,C} \ .$$
\end{lemma}

\subsection{Closed P-i.j. strategies and compositionality}

\subsubsection{Closed P-i.j strategy}

\begin{definition}
\label{def:safe_strategy}
A P-incrementally justified strategy $\sigma : A \fngamear B$ is said to be \defname{closed P-incrementally justified} if for
every initial move $m$ of $A$ such that some play $s\in\sigma$ contains $m$, we have $\ord_A{m} \geq \ord{B}$. 
\end{definition}
We observe that every P-i.j. strategy $\sigma$ on the game $I \fngamear A$ (and not $A$) is closed P-i.j..\footnote{In particular, every P-i.j. strategy $\sigma$ on the game $!A_1 \otimes \ldots \otimes !A_n \fngamear B$, is isomorphic, up to arena-tagging of the moves, to the closed P-i.j. strategy $\Lambda^n(\sigma)$ on the game $I \fngamear (A_1,\ldots,A_n,B)$, where $\Lambda$ denotes the usual currying isomorphism.}
However $\sigma : A$ is not necessarily closed P-i.j.. The reason is that the definition of closed P-i.j. strategy specifically refers to the moves of  the arena in the left-hand side of the function space arrow $\fngamear$, therefore the definition does not survive an isomorphism that retags the moves such as {\it currying}.

Consequently, for two strategies $\sigma$ and $\mu$ verifying $\sigma \cong \mu$, if $\sigma$ is closed P-i.j. then it does not necessarily imply that $\mu$ is. This contrasts with ``ordinary'' P-incremental justification condition which is preserved across any isomorphism.

Later on we will define a category of closed P-i.j. strategy. A consequence of the previous remark is that this category cannot be a closed category (neither monoidal closed nor cartesian closed).
In particular this category has only a weak form of curry isomorphism.

\subsubsection{Compositionality of closed P-i.j strategy}

{\bf Notation} In plays representations,
the symbol $\omove$ stands for an
O-move and $\pmove$ for
a P-move. 
Suppose the considered game is $L\fngamear R$ 
for some game $L$ and $R$, if we know the sub arena in which the move is played
then we specify it in subscripts ($\omove_L$, $\pmove_L$, $\omove_R$ or $\pmove_R$). For interaction sequences in $Int(A,B,C)$ we use
the symbols $\omove_A$, $\pmove_A$, $\omove_C$, $\pmove_C$, $\opmove$ and $\pomove$ as defined in Table \ref{tab:interseq}. We use the variable $X$ to denote one of the component $A,B$ or $B,C$, the variable  $Y$
then denotes the other component.
We write $s \subseqof t$ to say that $s$ is a subsequence (with pointers) of $t$, $s \prefixof t$ to say that $s$ is a prefix (with pointers)
of $t$ and  $s \suffixof t$ to say that $s$ is a suffix of $t$.

We now prove several useful lemmas which will become useful when studying compositionality of P-i.j. strategies.

\begin{lemma}
\label{lem:interjump}
Let $X$ be a component (either  $A,B$ or  $B,C$).
Let $u$ be an interaction sequence of the form
$ u =  
\Pstr[0.5cm][2pt]{ \ldots (b){\stk \beta \pmove}  \ldots
 {n}  \ldots  (a-b,30){\stk \alpha\omove}
\ldots m}$ where:
\begin{itemize}[-]
\item $\alpha,\beta$ are external moves in component $X$ (necessarily both played in $A$ or in $C$),
\item  $m$ is either played in $B$ or an external P-move in $X$,
\item  $\alpha$ is visible at $m$ in $X$ \emph{i.e.}~$\alpha\in \pview{u \upharpoonright X}$ (consequently $\beta$ is also visible).
\end{itemize}
Then $n \not\in \pview{u \upharpoonright A, C}$.
\end{lemma}
\begin{proof}
Since $\alpha$ is an O-move, $\alpha$ and $\beta$ are necessarily played in the same arena ($A$ or $C$).
Take $v=u$ if $m$ is a generalized O-move in $X$
and $v=u_{<z}$ otherwise (if $m$ is a generalized P-move in $X$).
The third assumption implies 
$\alpha,\beta\in \pview{v}$.
The last move in $v$ is necessarily a generalized O-move in component $X$ (see diagram of Table \ref{tab:interseq}) 
therefore by \cite[Lemma 3.3.1]{Harmer2005}
we have $\pview{v \filter X} = \pview{\overline{v} \filter X} \subseqof \overline{v} \subseqof \overline{u}$.
Thus $\alpha,\beta \in \overline{u}$ and
since $\alpha,\beta$ are played in $A,C$ we have 
$\alpha,\beta  \in \overline{u} \upharpoonright A,C 
= \pview{u \upharpoonright A,C}$ (Lemma \ref{lem:pviewAC_eq_ACpview}).
Finally since $n$ lies underneath the $\beta$-$\alpha$ PO-arc 
it cannot appear in the P-view  $\pview{u \upharpoonright A,C}$.
\end{proof}

\begin{lemma}
\label{lem:in_pviewAC_imp_in_pviewX}
Let $u$ be an interaction sequence in $Int(A,B,C)$ and
$n$ be a move of $u$ such that $n\in\pview{u \filter A,C}$:
\begin{enumerate}[i.]
\item 
if all the moves in $u_{\suffixof n}$ 
are played in $C$  then $n \in \pview{u \filter B,C}$;
\item 
if all the moves in $u_{\suffixof n}$ are played in $A$ then $n \in \pview{u \filter A,B}$.
\end{enumerate}
\end{lemma}
\begin{proof}
\begin{enumerate}[(i)]
\item
We show the contrapositive. Suppose that $n \not\in\pview{u \filter B,C}$. This must be due to one of the following  two
reasons:
\begin{itemize}[-]
\item $\pview{u \filter B,C}$ contains an initial move $c_0 \in C$
occurring after $n$ in $u$.


By \cite[Lemma 3.3.1]{Harmer2005}
we have $\pview{u \filter B,C} = \pview{\overline{u} \filter B,C} \subseqof \pview{u}$, thus $c_0$ also occurs in $\pview{u}$.
Since $c_0$ belongs to $C$ we have
$c_0 \in \pview{u} \filter A,C=
\pview{u \filter A,C}$ (Lemma \ref{lem:pviewAC_eq_ACpview}).
Thus the P-view $\pview{u \filter A,C}$
starts with the initial move $c_0$ and
since $n$ occurs before $c_0$, $n$ does not occur in the P-view.

\item $n$ lies underneath a PO-arc $\beta$-$\alpha$ visible 
at $ u \filter B,C$.
By assumption, since $\alpha$ occurs after $n$ in $u$, it must belong to $C$. We can therefore apply Lemma \ref{lem:interjump}
with $X\assignar B,C$ which gives
$n \not\in\pview{u \filter A,C}$.
\end{itemize}

\item Suppose that $n \not\in\pview{u \filter A,B}$ then either:
\begin{itemize}[-]
\item $\pview{u \filter A,B}$ contains an initial move $b_0 \in B$
occurring after $n$ in $u$. But this is impossible since by assumption all the moves occurring after $n$ in $u$ belong to $A$.

\item or $n$ lies underneath a PO-arc $\beta$-$\alpha$ in $A,B$.
By assumption, since $\alpha$ occurs after $n$ it must belong to $A$. We can then conclude using 
Lemma \ref{lem:interjump} with $X\assignar A,B$.
\end{itemize}
\end{enumerate}
\end{proof}

Note that we cannot completely relax the assumption 
which says that moves in $u_{\suffixof n}$ are all in the same component.
For instance take $u = \Pstr[0.5cm]{(co){\omove_C}\thinspace 
(b0-co){\opmove} \thinspace 
(n){\stk{\pmove_A}{n}} \thinspace 
(b1-co){\opmove}}$ then we have $n\in\pview{u\filter A,C}$ but $n\notin\pview{u\filter A,B}$.


%%%%%%%%%%%
% This commented Lemma could be useful be we did not make use of it eventually.
%
% \begin{lemma}
%\label{lem:oviewsegmentinB}
%For any legal sequence $s = \ldots x \cdot r \cdot y$ of a game $A\fngamear B$ if $x, y \in A$ and $x$ is O-visible from $y$ then any move in $r$ occurring in $\oview{s}$ belongs to $A$.
%\end{lemma}
%\begin{proof}
%We proceed by induction on the length of the segment $r$.
%Base case $r=\epsilon$ is trivial. Suppose $r = r' \cdot m$.
%If $y$ is an O-move then by the Switching Condition
%$m$ is necessarily in $A$. Clearly $x$ is O-visible from $m$ thus  by the I.H. any move from $r$ occurring in the O-view is in $A$.
%
%If $y$ is a P-move then it cannot point to an initial move in $B$. Indeed, suppose that it points to an initial O-move $b_0 \in B$ then
%we have $\oview{s} = b_0 \cdot y$ which contradicts the fact that $x\in \oview{s}$.
%Thus $y$ points to a move in $A$ and again we can conclude using the induction hypothesis.
%\end{proof}


\begin{lemma}[P-visibility decomposition (from $C$)]
\label{lem:middlepomove}
Let $u = \ldots n' \cdot r \cdot m \in Int(A,B,C)$ where
$n'$ is a $\omove_A$-move verifying $n' \in \pview{u\filter A,C}$ and $m$ is in $\{ \pmove_C, \opmove, \pomove \}$. Then there is a $\pomove$-move $\gamma$ in $r \cdot m$ such that $\gamma \in \pview{u\filter B,C}$ , $n' \in \pview{u_{\leq \gamma} \filter A,B}$ and $\gamma$ is justified by a move occurring before $n'$.
\end{lemma}
\begin{proof}
By induction on $|r|$.
If $r=\epsilon$ then necessarily $u = \ldots \stk{\omove_A}{n'} \thinspace\stk \pomove m$ where $m$ points before $n'$ ($n'$ being played in $A$ cannot justify $m$ played in $B$) so we just need to take $\gamma = m$.
If $|r|=1$ then either 
$u = \ldots \stk{\omove_A}{n'} \pomove\thinspace\stk {\pmove_C} m$
or $u = \ldots \stk{\omove_A}{n'} \pomove\thinspace\stk \opmove m$.
In both cases we can take $\gamma$ to be $\pomove$, the move between $n'$ and $m$.
Suppose $|r|>1$. Let $m^-$ denote the move preceding $m$ in $u$.
We proceed by case analysis:
\begin{enumerate}[i.]
\item Suppose $m = \pmove_C$ and $m^- = \omove_C$.
Let $q$ be the external P-move that justifies $m^-$.
Since $n' \in \pview{u\filter A,C}$, $q$ must occur after $n'$ in $u$:
$$ 
\begin{array}{ccccl}
A & \stackrel\sigma{\longrightarrow} & B & \stackrel\mu{\longrightarrow} & C \\
&\vdots&&\vdots\\
n' \omove\\
&\vdots&&\vdots  \\
&& & &  \rnode{q}{\pmove}q  \\
&\vdots&&\vdots  \\
&& & &  \rnode{mp}{\omove}m^-  \\
&& & &  \rnode{m}{\pmove}m  \\
\end{array}
\ncarc[arcangleA=60,arcangleB=60]{->}{mp}{q}
 $$  
Thus we can use the induction hypothesis (with $u\assignar u_{\prefixof q}$): there is a $\pomove$-move $\gamma$ 
in $u_{]n',q]}$ pointing before $n'$ such that $\gamma \in \pview{u_{\prefixof q} \filter B,C}$, $n' \in \pview{u_{\prefixof \gamma} \filter A,B}$.
Moreover $\pview{u_{\prefixof q} \filter B,C} \prefixof \pview{u_{\prefixof m} \filter B,C}$ (since $q$ is visible from $m$ in $B,C$) thus we have $\gamma \in \pview{u_{\prefixof m} \filter B,C}$ as required.

\item Suppose $m = \pmove_C$ and $m^- = \pomove \in B$.
Again we can conclude using 
the induction hypothesis with $u \assignar u_{\prefixof m^-}$.

\item Suppose $m = \pomove \in B$.

Suppose that all the moves in $r$ are in $A$.
Then $r$ is of the form $(\pmove_A \omove_A)^*$ (where $(\cdot)^*$ denotes the Kleenee star operator). 
We just need to take $\gamma = m$. 
Indeed, moves in $u_{\suffixof m}$ are all in $A$
and by assumption $n'\in\pview{u\filter A,C}$  therefore
Lemma \ref{lem:in_pviewAC_imp_in_pviewX}(ii) gives
$n'\in\pview{u\filter A,B}$.
Also, since $m$ is a $\pomove$-move, 
its justifier is a $\opmove$-move but $r$ contains only $\omove$ and $\pmove$ moves hence $m$'s justifier must occur before $n'$.

Suppose that $r$ contains at least one move in $B$. Let $b$ be the last such move, then $u$ is of the form $\ldots n' \cdot \ldots \cdot \stk\opmove  b \cdot (\pmove_A \omove_A)^* \cdot\thinspace\stk\pomove m $. We then have
$u\filter B,C = \ldots n' \cdot \ldots \cdot 
\thinspace\stk\opmove b \thinspace\cdot \stk\pomove m $ thus $b \in \pview{u\filter B,C}$. We can then conclude by applying the induction hypothesis with $u \assignar u_{\prefixof b}$.

\item Suppose $m = \pomove \in B$.
If $m^- = \opmove \in B$ then the I.H. with $u \assignar u_{\prefixof m^-}$ permits us to conclude.
If $m^- = \omove \in C$ then we conlude by applying  the I.H. on $u \assignar u_{\prefixof q}$ where $q$ is the external P-move in $C$ justifying
$m^-$.
\end{enumerate}
\end{proof}

We now show the lemma symmetric to the previous one:
\begin{lemma}[P-visibility decomposition (from $A$)]
\label{lem:middleopmove}
Let $u = \ldots n' \cdot r \cdot m \in Int(A,B,C)$ where
$n'$ is an O-move \emph{non initial} in $C$ verifying $n' \in \pview{u\filter A,C}$ and $m$ is in $\{\pmove_A, \opmove, \pomove\}$. Then there is a $\pomove$-move $\gamma$ in $r \cdot m$ such that $\gamma \in \pview{u\filter A,B}$ , $n' \in \pview{u_{\leq \gamma} \filter B,C}$ and $\gamma$ is justified by a move occurring before $n'$.
\end{lemma}
\begin{proof}
The proof is almost symmetrical to the previous one (Lemma \ref{lem:middlepomove}). We proceed by induction on $|r|$.
If $r=\epsilon$ then necessarily $u = \ldots \stk {\omove_C} {n'} \thinspace\stk \opmove m$ where $m$ points before $n'$ (it cannot point to $n'$
since $n'$ is not initial in $C$). Thus we just need to take $\gamma = m$.

If $|r|=1$ then either 
$u = \ldots \stk {\omove_C} {n'} \thinspace\opmove\thinspace\thinspace\stk{\pmove_A} m$
or $u = \ldots \stk {\omove_C} {n'} \thinspace\opmove\thinspace\thinspace\stk \pomove m$.
In both cases we can take $\gamma$ to be $\pomove$, the move between $n'$ and $m$.
Suppose $|r|>1$. Let $m^-$ denote the move preceding $m$ in $u$.
We do a case analysis:
\begin{enumerate}[i.]
\item Suppose $m = \pmove_A$ and $m^- = \omove_A$.
Let $q$ be the external P-move that justifies $m^-$.
Since $n' \in \pview{u\filter A,C}$, $q$ must occur after $n'$ in $u$:
$$ 
\begin{array}{rcccl}
A & \stackrel\sigma{\longrightarrow} & B & \stackrel\mu{\longrightarrow} & C \\
&\vdots&&\vdots\\
&&&& \omove\ n'\\
&\vdots&&\vdots  \\
q\rnode{q}{\pmove}  \\
&\vdots&&\vdots  \\
m^- \rnode{mp}{\omove}  \\
m \rnode{m}{\pmove}  \\
\end{array}
\ncarc[arcangleA=-45,arcangleB=-45]{->}{mp}{q}
 $$  
Thus we can use the induction hypothesis (with $u\assignar u_{\prefixof q}$): there is a $\opmove$-move $\gamma$ 
in $u_{]n',q]}$ pointing before $n'$ such that $\gamma \in \pview{u_{\prefixof q} \filter A,B}$, $n' \in \pview{u_{\prefixof \gamma} \filter B,C}$.
Moreover $\pview{u_{\prefixof q} \filter A,B} \prefixof \pview{u_{\prefixof m} \filter A,B}$ (since $q$ is visible from $m$ in $A,B$) thus we have $\gamma \in \pview{u_{\prefixof m} \filter A,B}$ as required.

\item Suppose $m = \pmove_A$ and $m^- = \pomove$ then again we can conclude using the I.H. with $u \assignar u_{\prefixof m^-}$.

\item Suppose $m = \opmove$.
\begin{itemize}[-]
\item Suppose that $r$ does not contain any move in $B$  then $r$ is of the form $(\pmove_C \omove_C)^*$. 

We just need to take $\gamma = m$. 
Indeed:
\begin{enumerate}
\item By lemmma \ref{lem:in_pviewAC_imp_in_pviewX}(i)
we have $n'\in \pview{u\filter B,C}$.

\item  $m$ is justified by a move occurring before $n'$. 
Indeed, if $m$ is justified by a $\pomove$-move then since $n' \cdot r$ contains only $\omove$ and $\pmove$ moves, $m$'s justifier must occur before $n'$.
If $m$'s justifier is an initial $\omove_C$-move $c_i$, then 
by P-visibility we have $c_i \in \pview{u\filter B,C}$
but since the P-view computation ``stops'' when reaching an initial moves, in order to guarantee that $n'$ also belongs to the P-view (as shown in (a)) it must
occurs after $c_i$.
\end{enumerate}


\item Suppose that $r$ contains some move in $B$. Let $b$ be the last such move. Then $u$ is of the form $u = \ldots n' \cdot \ldots \cdot \stk\opmove  b \cdot (\pmove_A \omove_A)^* \cdot\ \stk\pomove m $. 
So we have
$u\filter B,C = \ldots n' \cdot \ldots \cdot \stk\opmove  b \cdot \stk\pomove m $ hence $b \in \pview{u\filter B,C}$. We can now 
conclude by applying the I.H. with $u \assignar u_{\prefixof b}$.
\end{itemize}

\item Suppose $m = \pomove \in B$.
If $m^- = \pomove \in B$ then the I.H. with $u \assignar u_{\prefixof m^-}$ permits us to conclude.
If $m^- = \omove \in A$ then we conclude by applying the I.H. on $u \assignar u_{\prefixof q}$ where $q$ is the external P-move in $A$ justifying $m^-$.
\end{enumerate}
\end{proof}

We now use the two preceding Lemmas to show
the following useful result:
\begin{lemma}[Increasing order lemma]
\label{lem:increasing_order}
Let $u = \ldots n' \cdot r \cdot m \in Int(A,B,C)$ where
\begin{enumerate}
\item 
$n'$ is an external O-move in compoment $X$ 
($n'=\omove_A$ and $X=A,B$, or $n'=\omove_C$ and $X=B,C$)  non initial in $C$,
\item $n' \in \pview{u\filter A,C}$,
\item $m$ is either played in $B$ 
($\opmove$ or $\pomove$) or is an external
 P-move in $Y$
($\pmove_C$ if $n'=\omove_A$ and 
$\pmove_A$ if $n'=\omove_C$),
\item $m$'s justifier occurs before $n'$,
\item $u\filter X$ is P-i.j.,
\item $u_{\prefixof b}\filter Y$ is P-i.j. for all B-move $b$.
\end{enumerate}
Then:
$$ \ord_{Y} m \geq \ord_{A\fngamear C} n' \ .$$
\end{lemma}
\begin{proof}
If $n' =\omove_C$ (resp.~if $n'=\omove_A$)
then by Lemma \ref{lem:middleopmove} 
(resp.~Lemma \ref{lem:middlepomove})
there is a $\opmove$
(resp.~$\pomove$) $\gamma$ 
in $r \cdot m$ such that $\gamma \in \pview{u\filter Y}$ , $n' \in \pview{u_{\leq \gamma} \filter X}$ and $\gamma$ is justified by a move occurring before $n'$. 

There are six possible cases depending on 
the type of the moves $n'$ and $m$:
$(n',m) \in \{ \omove_A \} \times \{\pmove_C,\opmove,\pomove \} 
\union \{ \omove_C \} \times \{\pmove_A,\opmove,\pomove \} $).
The following diagram illustrates the cases $(n',m)
 = (\omove_A,\pmove_C)$ (left)
and  $(n',m)
 = (\omove_C,\pmove_A)$  (right):
$$ 
\begin{array}{ccccc}
A & \longrightarrow & B &
 \longrightarrow & C \\
&\vdots&&\vdots\\
&&&& \rnode{n}{\omove} \\
&\vdots&\rnode{gj}{\opmove}&\vdots\\
n' \omove \\
&\vdots&&\vdots  \\
&&\rnode{g}{\gamma} \pomove \\
&\vdots&&\vdots  \\
&&&&\rnode{m}{m} \pmove \\
\end{array}
\ncarc[arcangleA=30,arcangleB=30]{->}{m}{n}
\ncarc[arcangleA=30,arcangleB=30]{->}{g}{gj}
\hspace{2cm} \begin{array}{ccccc}
A & \longrightarrow & B & \longrightarrow & C \\
&\vdots&&\vdots\\
& \rnode{n}{\omove} \\
&\vdots&\rnode{gj}{\pomove}&\vdots\\
&&&&n' \omove \\
&\vdots&&\vdots  \\
&&\rnode{g}{\gamma} \opmove \\
&\vdots&&\vdots  \\
\rnode{m}{m} \pmove \\
\end{array}
\ncarc[arcangleA=30,arcangleB=30]{->}{m}{n}
\ncarc[arcangleA=30,arcangleB=30]{->}{g}{gj}
 $$  

We have:
\begin{equation}
\ord_Y \gamma \geq \ord_X \gamma \label{eqn:gammaorderXY}
\end{equation}
Indeed, if $n' =\omove_C$ then $X=B,C$ and $Y=A,B$ and
by Lemma \ref{lem:compositionorder} we have
$\ord_{A\fngamear B} \gamma \geq \ord_{B\fngamear C} \gamma$.
If $n=\omove_A$ then $\gamma$ is a $\pomove$-move therefore it is not initial in $B$ and Lemma \ref{lem:compositionorder} gives
$\ord_{A\fngamear B} \gamma = \ord_{B\fngamear C} \gamma$.

Hence:
\begin{align*}
\ord_{A\fngamear C} n' 
& = \ord_{X} n' & \mbox{(n' non initial in $C$ \& Lemma \ref{lem:compositionorder})} \\
& \leq \ord_{X} \gamma & \mbox{($u_{\prefixof \gamma}\filter Y$ is P-i.j. \& $\gamma$'s justifier occurs before $n'$)} \\
& = \ord_{Y} \gamma & \mbox{(By Eq. \ref{eqn:gammaorderXY})} \\
& \leq \ord_{Y} m & \mbox{($u\filter X$ is P-i.j. \& 
4$^{th}$ assumption: $m$'s justifier occurs before $\gamma$)}. 
\end{align*}
\end{proof}


\begin{proposition}
\label{prop:pijcompose_when_orda_geq_ordb}
Let $\sigma : A \fngamear B$ and $\mu : B \fngamear C$
be two well-bracketed (P-visible) strategies then
\begin{enumerate}[(I)]
\item if $\sigma$ is closed P-i.j. and $\mu$ is P-i.j.
then $\sigma ; \mu : A \fngamear C$ is P-i.j.,
\item if $\sigma$ and $\mu$ are closed P-i.j.
then $\sigma ; \mu : A \fngamear C$ is closed P-i.j.
\end{enumerate}
\end{proposition}

\begin{proof}
Well-bracketing is preserved by strategy composition (see \cite[Proposition 2.5]{abramsky94full}) thus
$\sigma ; \mu$ is well-bracketed so we can use the definition of P-i.j. from Proposition \ref{prop:char_wellbrack}.

\noindent (I) Let us prove that $\sigma ; \mu$ is P-i.j..
Let $u$ be a play of the interaction $\sigma\ \|\ \mu$ between $\sigma$ and $\mu$
ending with an external P-move $m$
justified by $n$ in $\pview{u \upharpoonright A , C}$.
Let $n'$ be an external O-move occurring betweeen $n$ and $m$:
$$ u \filter A,C =  
\Pstr[0.5cm][2pt]{ \ldots (n){\stk {n} \omove}  \ldots
 {\stk {n'} \omove}  \ldots  (m-n,30){\stk m \pmove}
}
$$
To show that $u \filter A,C$ is P-incrementally justified, we just need to prove that either $n'\not\in \pview{u \filter A,C}$ or $\ord_{A\fngamear C} n' \leq \ord_{A\fngamear C} m$. 
Note that if $n'\in \pview{u \filter A,C}$ 
then necessarily $n'$ is not initial 
in $C$ because $n$ occurs before $n'$ in
$\pview{u \filter A,C}$.

\begin{enumerate}[1)]
\item \label{case:mC}
Suppose $m =\pmove_C$, then necessarily $n=\omove_C$.

\begin{enumerate}[{\ref{case:mC}}.a)]
\item \label{case:mCnpC} Suppose $n' = \omove_C$. The projection of $u$ on the game $B\fngamear C$ has the following form:
$$ u \filter B,C =  
\Pstr[0.5cm][2pt]{ \ldots (n){\stk {n} {\omove_C}}  \ldots
 {\stk {n'}{\omove_C}}  \ldots  (m-n,30){\stk m {\pmove_C}}
}$$

Suppose that $n'$ occurs in the P-view $\pview{u\filter B,C}$ then necessarily $n'$ is not initial 
in $C$ (otherwise it would be the first move in
$\pview{u \filter B,C}$, which is not the case since $n$ occurs before $n'$ in the P-view) and we have:
\begin{align*}
\ord_{A\fngamear C} n' 
& = \ord_{B\fngamear C} n' & \mbox{(Lemma \ref{lem:compositionorder} \& $n'$ non initial in $C$)} \\
& \leq \ord_{B\fngamear C} m & \mbox{($\mu$ is P-i.j.)} \\
& = \ord_{A\fngamear C} m & \mbox{($m$ is not initial in $C$)} \ .
\end{align*}

Suppose that $n'$ does not occur in the P-view $\pview{u \filter B,C}$, then $n'$ lies underneath a PO arc occurring in $\pview{u \filter B,C}$. Let us denote this arc by $\beta$-$\alpha$ where $\beta$ and $\alpha$ denote the arc's nodes. We have:
$$ u \filter B,C = \ldots  
\Pstr[0.5cm]{
 (n){\stk{n} {\omove_C} } \ldots (b){\stk\beta \pmove} \ldots \stk{n'} {\omove_C}
\ldots (a-b){\stk\alpha \omove}  \ldots (m-n){\stk m {\pmove_C} }
} $$
with $\ord_{B\fngamear C} \alpha \leq \ord_{B\fngamear C} m$ (by P-i.j. of $\mu$).
  
\begin{enumerate}[i.]
\item Suppose $\alpha$ is a $\omove_C$-move in $u$ then necessarily
$\beta$ is a $\pmove_C$-move (since $\alpha$ is an O-move).

By instancing Lemma \ref{lem:interjump} with
$X\assignar B,C$ and $n \assignar n'$ we obtain that $n' \not\in\pview{u \filter A,C}$.

\item Suppose $\alpha$ is a $\pomove$-move in $u$ then necessarily $\beta$ is a $\opmove$-move.

We have $\alpha \in \pview{u \filter B,C}
= \pview{\pview{u} \filter B,C} \subseqof
\pview{u}$ (\cite[Lemma 3.3.1]{Harmer2005}).

Suppose that $n' \in \pview{u\filter A,C}$, then 
\begin{align*}
n'  \in \pview{u\filter A,C} 
& = \pview{u}\filter A,C   
& \mbox{(Lemma \ref{lem:pviewAC_eq_ACpview})} \\
& \suffixof \pview{u_{\prefixof \alpha}}\filter A,C
& \mbox{($\alpha \in \pview{u}$, $n'$ occurs before $\alpha$ in $u$)} \\
& = \pview{u_{\prefixof \alpha} \filter A,C} 
& \mbox{(Lemma \ref{lem:pviewAC_eq_ACpview})}\ .
\end{align*}
But since $n'$ occurs before $\alpha$ in $u$, the previous equation 
implies that $n' \in \pview{u_{\prefixof \alpha}\filter A,C}$
so we can apply Lemma \ref{lem:increasing_order}
on $u_{\prefixof \alpha}$: 
\begin{align*}
\ord_{A\fngamear C} n' 
& \leq \ord_{A\fngamear B} \alpha & \mbox{(Lemma \ref{lem:increasing_order} with $u\assignar u_{\prefixof \alpha}$)} \\
& = \ord_{B\fngamear C} \alpha & \mbox{($\alpha$ non initial in $B$)} \\
& \leq \ord_{B\fngamear C} m & \mbox{($\mu$ is P-i.j.)} \\
& = \ord_{A\fngamear C} m & \mbox{($m$ is not initial in $C$)} \ .
\end{align*}
\end{enumerate}


\item Suppose $n' =\omove_A$.

Suppose that $n' \in \pview{u\filter A,C}$, then by
Lemma \ref{lem:increasing_order} we have
$ \ord_{A\fngamear C} n'  \leq \ord_{B\fngamear C} m$.
 By Lemma \ref{lem:compositionorder}, since
 $m$ is not initial in $C$ we have $\ord_{B\fngamear C} m = \ord_{A\fngamear C} m$ thus $\ord_{A\fngamear C} n' \leq \ord_{A\fngamear C} m$.
\end{enumerate}

\item \label{case:mA} Suppose $m = \pmove_A$.
\begin{enumerate}[{\ref{case:mA}}.1)]

\item Suppose $n=\omove_A$.
This case is symmetrical to case
\ref{case:mC}) where $m=\pmove_C$ and $n=\omove_C$.

\item Suppose  $n$ is an $\omove_C$-move initial in $C$.
Then $m$ is initial in $A$ and points to a move
$b_0 = \opmove$ initial in $B$ which in turn points to the
move $n =\omove_C$ initial in $C$.

Let us assume that $n'\in \pview{u\filter A,C}$ and let us prove that necessarily $\ord_{A\fngamear C} n' \leq \ord_{A\fngamear C} m$.

	\begin{enumerate}[a.]
	\item Suppose $n'=\omove_A$. Then
since $m$ is initial in $A$ we clearly have 
$\ord_{A\fngamear C} n' \leq \ord_{A\fngamear C} m$.

	\item Suppose $n'=\omove_C$.
	If $n'$ occurs between $b_0$ and $m$ then $m$ points before $n'$ in 		$u$ so we can use Lemma \ref{lem:increasing_order} which gives $\ord_{A\fngamear C} n' \leq \ord_{A\fngamear B} m
		= \ord_{A\fngamear C} m$.
		\smallskip
		
		
		If $n'$ occurs before $b_0$ then 
		we cannot use Lemma \ref{lem:increasing_order}
		since $m$ does not point before $b_0$.
Up to now we have only used the fact that $\sigma$ and $\mu$ are P-i.j. The assumption that $\sigma$ is  \emph{closed} P-i.j. now becomes crucial. We have:
		\begin{align}
		\ord_{A\fngamear B} b_0 
		& = \max(\ord A +1, \ord_B b_0) & \mbox{($b_0$ initial in $B$)} \nonumber \\
		& \geq \ord_B b_0 \nonumber  \\
		& \geq \ord C & \mbox{($\sigma$ is closed P-i.j. \& $b_0 \in u \filter  B,C \in \mu$)} \nonumber  \\
		& \geq \ord_C n' +1 & \mbox{($n'$ not initial in $C$)} \nonumber  \\
		& = \ord_{A\fngamear C} n' +1 & \mbox{(Lemma \ref{lem:compositionorder} \& $n'$ not initial)}. \label{eq:b0gtnp}
		\end{align}
Thus:
		\begin{align*}
		\ord_{A\fngamear C} m 
		& = \ord_{A\fngamear B} m & \mbox{(Lemma \ref{lem:compositionorder})} \\
		& = \ord_{A\fngamear B} b_0 -1  & \mbox{($m$ justified by $b_0$ initial in $B$)} \\
		& \geq \ord_{B\fngamear C} b_0 -1 & \mbox{($b_0$ initial in $B$)} \\
		& \geq \ord_{A\fngamear C} n' & \mbox{(By Eq. \ref{eq:b0gtnp})} \ .
		\end{align*}
		\end{enumerate}

\end{enumerate} 
\end{enumerate} 

\noindent (II)
To show that $\sigma;\mu$ is closed P-i.j., we need to verify that for all initial move in $A$ occurring in some play of $\sigma ; \mu$ has order in $A$ greater or equal to $\ord C$.
Take an initial move $m \in A$ such that $m$ occurs in $s \in \sigma ; \mu$. Then $s = u \filter A,C$ for some $u \in \sigma 
\ \|\ \mu$. Let $n$ be the initial move of $B$ justifying $m$.
 We have:
\begin{align*}
\ord_A m & \geq \ord B & \mbox{($\sigma$ is closed P-i.j. \& $m \in u
\filter A,B \in \sigma$)} \\
 & \geq \ord_B n & \mbox{($n 
\in B$)} \\
 & \geq \ord C & \mbox{($\mu$ is closed P-i.j. \& $n \in u
\filter B,C \in \mu$)}.
\end{align*}
\end{proof}
We recall some definitions. Let $A$ and $B$ be two well-opened games. Given a strategy  $\sigma :\ !A \fngamear B$, its promotion $\sigma^\dag :\ !A\fngamear !B$ is defined as
$$ \sigma^\dag = \{ s \in L_{!A\fngamear !B}\ |\ \mbox{for all inital $m$ in $B$, } s\filter m \in \sigma \}$$
and for $\mu :\ !B\fngamear C$ the composite strategy $\sigma \fatcompos \mu$ is defined as:
$$ \sigma \fatcompos \mu = \sigma^\dag ; \mu \ .$$

\begin{proposition}
If $A$ and $B$ are two well-opened games 
and $\sigma :\ !A \fngamear B$ is a well-bracketed P-incrementally justified strategy then $\sigma^\dag$ is also well-bracketed and P-incrementally justified.
Furthermore if $\sigma$ is closed P-i.j. then so is
$\mu$.
\end{proposition}
\begin{proof}
$\sigma^\dag$ is well-bracketed by \cite[Proposition 2.10.]{abramsky94full}.

By the visibility condition,
\end{proof}

From the last two propositions we obtain:
\begin{corollary}
Let $A$ and $B$ be two well-opened games. Let
$\sigma :\ !A \fngamear B$ and $\mu :\ !B\fngamear C$ be two well-bracketed strategies then:
\begin{enumerate}
\item if $\sigma$ is closed P-i.j. 
and $\mu$ is P-i.j. then $\sigma \fatcompos \mu :\ !A \fngamear C$ is also P-i.j.
\item if $\sigma$ and $\mu$ are closed P-i.j. then so is $\sigma \fatcompos \mu :\ !A \fngamear C$.
\end{enumerate}
\end{corollary}

\section{Compositionnality - Syntactic approach}

\subsection{Definability result for Safe PCF}

We say that a PCF term is \defname{semi-safe} if it is of the form $N_0 N_1 \ldots N_k$ for $k\geq 1$ where each of the $N_i$ is safe or if it can be written $\lambda \overline{x} . N$ for some safe term $N$. Semi-safe terms are either safe or ``almost safe'' in the sense that they can be turned into an equivalent (i.e.~with isomorphic game semantics) safe term  by performing $\eta$-expansions. Indeed, let $M$ be an semi-safe term that is unsafe.
If $M$ is of the first form $N_0 N_1 \ldots N_k : (A_1,\ldots,A_n)$ with $k\geq 1$ then let $\varphi_i:A_i$ for $i\in\{1..n\}$ be fresh variables, using the (app) and (abs) rules we can build the safe term $\lambda \varphi_1 \ldots \varphi_n . N_0 N_1 \ldots N_k \varphi_1 \ldots \varphi_n$. If $M$ is of the second form $\lambda \overline{x} . N$ then using the abstraction rule we can build the equivalent safe term $\lambda \overline{y} \overline{x}. N$  where $\overline{y} = fv(\lambda \overline{x}. N)$.

The $\beta$-normal form of a \pcf\ term is the possibly infinite term obtained by reducing all the redexes in $M$.

The correspondence between safety and P-incremental justification in the context of the simply typed lambda calculus was shown
in \cite[Theorem 3(ii)]{blumong:safelambdacalculus}:

\begin{theorem}[\cite{blumong:safelambdacalculus},Theorem 3(ii)]
\label{thm:safeincrejust} In the simply typed lambda calculus:
\begin{enumerate}[(i)]
\item If $M$ is safe then $\sem{M}$ is P-incrementally justified.
\item If $M$ is a closed term and $\sem{M}$ is
  P-incrementally justified then the $\eta$-long form of the
  $\beta$-normal form of $M$ is safe.
\end{enumerate}
\end{theorem}
In fact the proof of this theorem can be easily adapted to show a more precise result:
\begin{theorem}[Semi-safety and P-incremental justification]
\label{thm:semisafeincrejust} Let $\Gamma \vdash M : A$ be a simply typed term. Then:
\begin{enumerate}[(i)]
\item If $\Gamma \vdash M : A$ is semi-safe then $\sem{\Gamma \vdash M : A}$ is P-incrementally justified.
\item If $\sem{\Gamma \vdash M : A}$ is
  P-incrementally justified then 
$\etanf{\betanf{M}}$ is semi-safe if $M$ is open
and safe if $M$ is closed.
\end{enumerate}
\end{theorem}



In the context of \pcf\, only the first part of the theorem holds (see \cite{blumtransfer} for the proof). However (ii) does not hold. Indeed, take the closed \pcf\ term $M = \lambda f x y. f (\lambda z. \pcfcond (\pcfsucc\ x) y z )$ where $x,y,z:o$ and $f:((o,o),o)$. $M$ is in normal form (the conditional  could  only be reduced if $x$ were first evaluated). The $\eta$-long form of the $\beta$-normal form of $M$ is therefore $M$ itself which is unsafe.
But clearly we have $\sem{M} = \sem{\lambda f x y. f (\lambda z. z)}$, and since  $\lambda f x y. f (\lambda z. z)$ is safe, by (i), $\sem{M}$ is P-incrementally justified.

Such counter-example arises because the conditional operator of \pcf\ permits us to build terms in normal form containing ``dead code'' {\it i.e.}~some subterm that will never be evaluated for any value of M's parameters. In the example given above, the dead code consists in the subterm $y$. In general, if the dead code part of the computation tree contains a variable that is not incrementally bound then the resulting term will be unsafe even if the rest of the tree is incrementally bound.
In the example above, it was possible to turn $M$ into the equivalent safe term $\lambda f x y. f (\lambda z. z)$ by eliminating the dead code from $M$.
We shall see how to generalise this to any \pcf\ term with a P-incrementally justified denotation.

Dead code elimination can be difficult to achieve in practice but the formal definition is not difficult to formulate. We say that a subterm $N$ occurring
in a context $C[-]$ in $M : (A_1, \ldots, A_n,o)$ is part of the \defname{dead code} of $M$ if for any term $T_0$ of the form $M M_1 \ldots M_n$,
any reduction sequence starting from $T_0$ does not involve a reduction of the subterm $N$ {\it i.e.}~for any reduction sequence $T_0 \redar T_1 \redar \ldots \redar T_k$, there is no $j\in \{0.. k-1\}$ such that $T_j = C[N]$ and $T_{j+1} = C[N']$ for some term $N'$.
 

Let $M$  be a \pcf\ term in $\eta$-nf.
An occurrence of a variable $x$ in $M$ is said to be a \defname{dead occurrence}
if it occurs in the dead code of $M$. In other words, it is a
dead occurrence of $x$ if the corresponding node in the computation tree does not appear in any traversal of $\travset(M)$. Equivalently, thanks to the Correspondence Theorem, an occurrence of $x:B$ is dead if and only if the initial move
of the arena $\sem{B}$ does not appear in any play of $\sem{M}$.


We define $M^*$ as the term obtained from $M$ after substituting all subterms of the form  $x N_1 \dots N_k$ for some dead variable occurrence $x:(B_1,\ldots, B_k, o)$ by the constant $0$. This process is called \defname{dead variable elimination}.
Note that if $M$ is in $\eta\beta$-nf then so is $M^*$.

We also write $\tau(M)^*$ to denote the equivalent transformation on the computation tree. Since the computation tree is constructed from the $\eta$-nf of $M$, we will use this notation even when $M$ is not in $\eta$-nf.



\begin{proposition}[Incremental-binding and P-incremental justification coincide] \
\label{prop:incrbound_imp_incrjustified_pcf} Let $\Gamma \vdash M : A$ be a PCF term in $\beta$-normal form.
\begin{enumerate}[(i)]
\item  If $\tau(\Gamma \vdash M : A)$ is incrementally-bound then $\sem{\Gamma \vdash M : A}$ is P-incrementally justified,
\item  if $\sem{\Gamma \vdash M : A}$ is P-incrementally justified
then $\tau(\Gamma \vdash M : A)^*$ is incrementally-bound.
\end{enumerate}
\end{proposition}
\begin{proof}
(i) The proof is exactly the same as in the simply typed lambda calculus case,
see \cite[Proposition 4.1.5(i)]{blumtransfer}.

\noindent (ii)
Take $\Gamma \vdash M : A$ a \pcf\ term in $\beta$-normal form denoted by $\sem{\Gamma \vdash M : A}$ P-incrementally justified. Let $r$ denote the root of $\tau(M)^*$.
Let $n$ be a node of $\tau(M)^*$ labelled by the variable $x$.
$\tau(M)^*$ is free from dead code therefore $n$ is not a dead occurrence of $x$ and there exists a traversal of $\tau(M)^*$ of the form $t \cdot x$.

\pcf\ constants are of order $1$ at most therefore they cannot hereditarily justify a variable node, thus $x$ is necessarily hereditarily justified by the root $r$ of the computation tree.


By considering $t\cdot x$ as a traversal of $\tau(M)$,  the correspondence theorem gives $\varphi((t \cdot x) \upharpoonright r) = \varphi((t \upharpoonright r) \cdot x) \in \sem{M}$. Since $\sem{M}$ is P-incrementally justified, $\varphi(x)$ must point to the last O-move in $\pview{?(\varphi(t \upharpoonright
r))}$ with order strictly greater than $\ord{\varphi(x)}$.
Consequently $x$ points to the last node in $\pview{?(t
\upharpoonright r)} \inter N^{\lambda}$ with order strictly greater than $\ord{x}$. We have:
\begin{align*}
\pview{?(t \upharpoonright r)} &= \pview{?(t) \upharpoonright r} = \pview{?(t)} \upharpoonright r & (\mbox{by \cite[lemma 3.1.23]{blumtransfer}}) \\
& = [r,x[ \ \upharpoonright r & (\mbox{by \cite[proposition 3.1.20]{blumtransfer}})
\end{align*}
Since $M$ is in $\beta$-nf, the set of nodes not hereditarily justified by $r$ is exactly the set of nodes hereditarily justified by $N_{\Sigma}$ thus
$[r,x[ \ \upharpoonright r = [r,x[\ \setminus\  N^{\upharpoonright \Sigma}$.
Moreover \pcf\ constants are of order $1$ at most therefore $N^{\upharpoonright \Sigma} = N_{\Sigma} \union N^c_{\Sigma}$
where $N^c_{\Sigma}$ is the set of children nodes of $N_{\Sigma}$.
Thus $(\pview{?(t \upharpoonright r)}\upharpoonright r) \inter N^{\lambda} =
([r,x[\ \setminus\  N_{\Sigma} \setminus N^c_{\Sigma} ) \inter N^{\lambda} =
([r,x[\ \setminus\  N^c_{\Sigma} )  \inter N^{\lambda}$, and
since $N^c_{\Sigma}$ is constituted of order $0$ lambda-nodes only we have that
$x$ points to the last node in $[r,x[ \inter N^{\lambda}$ with order strictly greater than $\ord{x}$.

Hence if $x$ is a bound variable node then it is bound by the
last $\lambda$-node in $[r,x[$ with order strictly greater than
$\ord{x}$ and if $x$ is a free variable then it points to $r$ and
therefore all the $\lambda$-node in $]r,x[$ have order smaller than
$\ord{x}$. Thus $\tau(M)^*$ is incrementally-bound.
\end{proof}

The counterpart of Lemma 4.1.6 from
\cite{blumtransfer} can be stated as follows in the context of PCF:
\begin{lemma}[Semi-safety and incrementally-binding]
\label{lem:safe_imp_incrbound_pcf} Let $\Gamma \vdash M : A$ be a PCF term.
\begin{itemize}
\item[(i)] If $\Gamma \vdash M : A$ is a semi-safe term then $\tau(\Gamma \vdash M : A)$ is incrementally-bound ;
\item[(ii)] conversely, if $\tau(\Gamma \vdash M : A)$ is incrementally-bound then the $\eta$-normal form of $\Gamma \vdash M : A$ is semi-safe if $M$ is open and safe if $M$ is closed.
\end{itemize}
\end{lemma}
The proof can be obtained by adapting the proof 
of Lemma 4.1.6 from \cite{blumtransfer}.

\begin{theorem}[Semi-safety and P-incremental justification]
\label{thm:semisafeincrejust_pcf} Let $\Gamma \vdash M : A$ be a PCF term. Then:
\begin{enumerate}[(i)]
\item If $\Gamma \vdash M : A$ is semi-safe then $\sem{\Gamma \vdash M : A}$ is P-incrementally justified.
\item If $\sem{\Gamma \vdash M : A}$ is
  P-incrementally justified then $\etanf{\betanf{M}}^*$ is semi-safe  if $M$ is open, and safe if $M$ is closed.
\end{enumerate}
\end{theorem}

\begin{proof}
\noindent(i)
A proof of this is given in the proof of Theorem 4.2.10 in \cite{blumtransfer}.

\noindent(ii) 
Suppose $M$ is a \pcf\ term with a P-incrementally justified strategy denotation. By Proposition \ref{prop:incrbound_imp_incrjustified_pcf}(ii), $\tau(\betanf{M})^* = \tau(\etanf{\betanf{M}}^*)$ is incrementally-bound.
If $M$ is closed then so is $\etanf{\betanf{M}}^*$ therefore by Lemma \ref{lem:safe_imp_incrbound_pcf}, $\etanf{\etanf{\betanf{M}}^*} = \etanf{\betanf{M}}^*$ is safe. If $M$ is open then so is $\etanf{\betanf{M}}^*$ and by Lemma \ref{lem:safe_imp_incrbound_pcf}, $\etanf{\etanf{\betanf{M}}^*} = \etanf{\betanf{M}}^*$ is semi-safe.
\end{proof}


We write \pcf' to denote the language obtained by extending \pcf\
with the $\pcfcase_k$ construct (see \cite{Abr02}).
The $\pcfcase_k$ construct is the obvious generalisation of the
conditional operator \pcfcond\ to $k$ branches instead of $2$. All the results obtained so far concerning Safe \pcf\ (including those
cited from \cite{blumtransfer}) can clearly be transposed to \pcf'.

The previous theorem leads to the following definability result for safe \pcf':
\begin{proposition}[Definability for safe \pcf' terms]
\label{prop:safetydefinability}
Let $\overline{A}=(A_1,\ldots, A_i)$ and $B =(B_1, \ldots, B_l,o)$ be two PCF types for some $i,l\geq 0$ and $\sigma$ be a well-bracketed innocent
P-i.j. strategy with finite view function defined on the game $!A_1 \otimes \ldots \otimes !A_i \fngamear (!B_1 \fngamear \ldots \fngamear !B_l \fngamear o) $. There exists a \emph{semi-safe} PCF' term $\overline{x} : \overline{A} \vdash M : B$ in $\eta$-long normal form such that:
$$ \sem{\overline{x} : \overline{A} \vdash M_\sigma : B} = \sigma $$
and a safe closed PCF' term $\vdash_s M'_\sigma : (\overline{A},B)$ in $\eta$-long normal form such that:
$$ \sem{\vdash M'_\sigma : (\overline{A},B)} \cong \sigma \ .$$
\end{proposition}
\begin{proof}
By the standard definability result for PCF', there is a term $\overline{x} : \overline{A} \vdash N : B$ such that $\sem{\overline{x} :\overline{A} \vdash N : B} = \sigma$.
Take $M_\sigma$ to be $\etanf{\betanf{N}}^* $. We have $\sem{\overline{x} : \overline{A} \vdash M_\sigma : B} =  \sem{\overline{x} :\overline{A} \vdash N : B} = \sigma$ and by Theorem  \ref{thm:semisafeincrejust_pcf}(ii), $M_\sigma$ is semi-safe.
For the second part we just need to take $M'_\sigma = \lambda \overline{x}. M_\sigma$.
\end{proof}


\subsection{Composition of P-i.j. strategies}

Let $\overline{A} = (A_1, \ldots, A_i)$,
$B = (B_1, \ldots, B_l,o)$ 
and $C=(C_1,\ldots,C_k,o)$ be three PCF types
for some $i\geq 1,l,k\geq 0$. Let
$f:\ !A_1 \otimes \ldots \otimes !A_i \fngamear B$ and $g:\ !B\fngamear C$ be two innocent well-bracketed and P-incrementally justified strategies with finite view function.
We would like to find under which conditions the composition $f\fatcompos g$ is also P-incrementally justified.

By the definability result, there are two closed safe terms (in $\eta$-nf) $\vdash M_f :(\overline{A},B)$  and $\vdash M_g :B \typear C$ such that $\sem{M_f} = f$
and $\sem{M_f} = g$.
We define the term $M_{f\fatcompos g} = \lambda \overline{x} . M_g (M_f \overline{x})$ for some fresh variables $\overline{x} : \overline{A}$. Clearly we have $\sem{M_{f\fatcompos g}} = \sem{M_f} \fatcompos \sem{M_g} = f\fatcompos g$.

By Theorem \ref{thm:semisafeincrejust_pcf}, we know that $f\fatcompos g$ is P-incrementally justified just when $\etanf{\betanf{M_{f\fatcompos g}}}^*$ is safe. 
We will now exploit this fact to extract a sufficient condition on the types $A$ and $B$ for 
the composition of $f$ and $g$ to be P-incrementally justified.

The term $M_f$ and $M_g$, being in $\eta$-nf, are of the following forms:
\begin{eqnarray*}
\vdash M_f &=& \lambda x_1^{A_1} \ldots x_i^{A_i} \varphi_1^{B_1} \ldots \varphi_l^{B_l} . N_f^o\\
\vdash  M_g &=& \lambda y^{ (B_1, \ldots, B_l,o)} \phi_1^{C_1} \ldots \phi_k^{C_k} . N_g^o
\end{eqnarray*}
for some distinct variables $x_1, \ldots, x_i$, $y$, $\varphi_1, \dots \varphi_l$, $\phi_1, \dots \phi_k$  and $\eta$-normal terms $N_f$ and $N_g$:
\begin{eqnarray*}
x_1:A, \ldots, x_i:A_i, \varphi_1:B_1, \dots, \varphi_l:B_l &\vdash& N_f :o \\
y: (B_1, \ldots, B_l,o), \phi_1:C_1, \dots, \phi_l:C_l &\vdash& N_g :o
\end{eqnarray*}


 
The fact that $M_f$ and $M_g$ are safe does not imply that $M_{f\fatcompos g}$ is: take $M_f = \lambda x^o z^o.x$ and $M_g = \lambda y^{(o,o)} . y a$ for some constant $a\in \Sigma$, then $\lambda x:A . M_g (M_f x) = \lambda x . (\lambda y . y a) ( \underline{(\lambda x z.x) x} )$ is unsafe because of the underlined subterm. However we have:
\begin{align*}
f\fatcompos g &= \sem{\lambda \overline{x} . M_g (M_f  \overline{x})} \\
 &= \sem{\lambda \overline{x} . (\lambda \phi_1\ldots \phi_k . N_g) [(M_f \overline{x}) / y]} \\
&= \sem{\lambda \overline{x} \phi_1 \dots \phi_k. N_g [(M_f  \overline{x}) / y]}
& \mbox{(the $x_j$'s and $\phi_j$'s are disjoint)}.
\end{align*}

We now concentrate on the term  $\lambda \overline{x} \phi_1 \dots \phi_k. N_g [(M_f  \overline{x}) / y]$ and try to find a sufficient condition guaranteeing its safety.
\subsubsection{A sufficient condition}
\begin{lemma}
Suppose that $\Gamma,y:B \vdash M$ is a safe term in $\eta$-nf and $\Gamma \vdash R : B$ is an almost safe application. Let $N$ denote the set of nodes of the computation tree $\tau(M)$. We have:
\begin{align*}
\Gamma \vdash M[R/y] :A \mbox{ safe } 
\iff&  \forall x \in fv(R) . \\
    & \forall n_y \in N_{fv} \mbox{ labelled $y$}.
      \forall m \in N_{\lambda} \inter ]r,n_y] : \ord{m} \leq \ord{x}
\end{align*}
\end{lemma}
\begin{proof}
Since $M$ is in $\eta$-nf, all the application to the variable $y$ are total (i.e.~of the form $y P_1 \ldots P_l :o$). Hence after substituting the safe term $N$ for $y$ in $M$, the only possible cause of unsafety is when
some variable free in $N$ becomes not safely bound in $\tau(M)$.
\end{proof}

Applying this lemma with $R= M_f \overline{x}$ gives us a sufficient condition (in the right-hand side of the equivalence) 
for $\lambda x \phi_1 \dots \phi_k. N_g [(M_f \overline{x}) / y]$ to be safe , and hence for $f\fatcompos g$ to be P-incrementally justified. Of course it is not a necessary condition since the 
eta-beta normal form of  $N_g[(M_f \overline{x}) /y]$ can be safe even though it is unsafe.

\subsubsection{A simpler sufficient condition}
\begin{lemma}
If $y:B, \Sigma \vdash N : T$ and $\vdash M : (\overline{A}, B)$ 
are safe terms with $\ord{A_i} \geq \ord{B}$ for all $i\in 1..n$
then $\overline{x}:\overline{A}, \Sigma \vdash N[(M \overline{x})/y] :T$ is also safe.
\end{lemma}
\begin{proof}
Since $\ord{x_i} = \ord{A_i} \geq \ord{B} = \ord{M \overline{x}}$, we can use the application 
rule of the safe lambda calculus to form the safe term $\overline{x}:\overline{A} \vdash M \overline{x}$.
Using the substitution lemma we have that $N[(M \overline{x})/y]$ is safe.
\end{proof}

Hence if we have $\ord{A_i}\geq\ord{B}$
for all $1 \leq i \leq n$,
then by the previous lemma the term $\vdash \lambda \overline{x} \phi_1 \dots \phi_k. N_g [(M_f \overline{x}) / y]$
is safe and therefore its denotation $\sem{\vdash \lambda \overline{x} \phi_1 \dots \phi_k. N_g [(M_f \overline{x}) / y]} = f\fatcompos g$ is P-incrementally justified.
This gives us a sufficient condition for $f\fatcompos g$ to be P-incrementally justified.

This condition is not necessary: Take $A=o$, $B=(o,o)$, $C=(o,o)$ and consider the two safe terms $M_f = \lambda x^A u^o.u$ and $M_g = \lambda y^B . y a$ for  some constant $a:o$. Then we have $M_{f\fatcompos g} = \lambda x . a$ which is safe hence $f\fatcompos g$ is P-incrementally justified although $\ord{A} < \ord{B}$.





\subsection{Two P-i.j. strategies whose composition is not P-i.j.}
\subsubsection{First attempt}

Take the types $A=o$, $B=(o,o)$, $C=o$, the variables
$x,u,v:o$, $y:B$ and $\varphi:((o,o),o)$ and $\Sigma$-constant $a:o$.
Consider the two safe terms $\vdash_s  M_f = \lambda xv.x : A\typear B$ and $\vdash_s M_g = \lambda y . \varphi (\lambda u . y a) : B\typear C$.
The $\eta\beta$-nf of $M_{f\fatcompos g}$ is $\vdash \lambda x . \varphi (\underline{\lambda u . x})$ which is unsafe because of the underlined term. It is then tempting to use
Theorem \ref{thm:safeincrejust}(ii) to conclude that
$\sem{M_{f\fatcompos g}}$ is not P-incrementally justified. However we cannot use it since $M_g$ contains an order $2$ constants ($\varphi$) and therefore
$M_{f\fatcompos g}$ is not a valid simply typed $\lambda$-term (nor a \pcf-term).

\subsubsection{Second attempt}
The previous example can be easily changed into a working counter-example: we just need to elevate $\varphi$ from the status of constant to variable.

Take $A=o$, $B=(o,o)$, $C=(((o,o),o),o)$, the variables
$x,u,v:o$, $y:B$ and $\varphi:((o,o),o)$ and the $\Sigma$-constant $a:o$. Consider the two safe terms $\vdash_s  M_f = \lambda xv.x : A\typear B$ and  $\vdash_s M_g = \lambda y \varphi. \varphi (\lambda u . y a) : B\typear C$.
The $\eta\beta$-nf of $M_{f\fatcompos g}$ is $\vdash \lambda x \varphi. \varphi (\underline{\lambda u . x})$ which is unsafe because of the underlined term, thus by Theorem \ref{thm:safeincrejust}(ii), $\sem{M_{f\fatcompos g}}=\sem{M_f} \fatcompos
\sem{M_g}$ is not P-incrementally justified. The following diagram illustrates a play that is not P-i.j.:
\begingroup
\def\sigcol#1{{\color{gray} #1}}
\def\mucol#1{{\color{red} #1}}
$$\begin{array}{ccccccccc}
A &  & \multicolumn{2}{c}{B} && \multicolumn{4}{c}{C}\\
\cline{1-1} \cline{3-4} \cline{6-9}
o & \stackrel{\sigcol{\sem{M_f}}}\longrightarrow & o, & o & \stackrel{\mucol{\sem{M_g}}}\longrightarrow & ((o, &o),& o),& o \\ \\
&&&&&&&&\rnode{n0}{\lambda x \varphi \omove  \mucol {\lambda y \varphi}}\\
&&&&&&&\rnode{n1}{\varphi  \pmove \mucol \varphi}\\
&&&&&&\rnode{n2}{\lambda u \omove  \mucol {\lambda u}} \\
&&&  \rnode{n3}{\omove \sigcol {\lambda x v} \pmove \mucol y} \\
\rnode{n4}{x \pmove \sigcol x}
\end{array}
\ncarc[arcangleA=20,arcangleB=20,linecolor=black]{->}{n4}{n0}
\ncarc[arcangleA=30,arcangleB=20,linecolor=red]{->}{n2}{n1}
\ncarc[arcangleA=30,arcangleB=20,linecolor=red]{->}{n1}{n0}
\ncarc[arcangleA=20,arcangleB=20,linecolor=red]{->}{n3}{n0}
\ncarc[arcangleA=20,arcangleB=20,linecolor=gray]{->}{n4}{n3}
$$
\endgroup

\subsubsection{Another counter-example where $\ord{B} = \ord{C}$}

Let $A=o$, $B=C=(((o,o),o),o)$ and let $x:A$, $y:B$, $u:o$, $v,\varphi:((o,o),o)$
and $g:(o,o)$ be variables and  $a:o$ be a $\Sigma$-constant. Take the two safe terms $\vdash  M_f = \lambda x v.x$ and $\vdash M_g = \lambda y \varphi. \varphi (\lambda u . y (\lambda g. a))$.
The $\eta\beta$-nf of $M_{f\fatcompos g}$ is $\vdash \lambda x \varphi. \varphi (\underline{\lambda u . x})$ which is unsafe because of the underlined term, so
$f\fatcompos g$ is not P-incrementally justified.
 
\bibliographystyle{plain}
\bibliography{../bib/higherorder,../bib/gamesem,../bib/lambdacalculus}


\section{P-i.j. and IA}
$var = acc \times exp = com^{\omega}\times exp$

Any strategy on the game $I \fngamear\ !var$ is P-i.j. since there is no P-question in the arena $var$.


\end{document}


\psset{linecolor=darkGreen,linewidth=0.5pt}


\author{William Blum}
\title{P-incrementally justified strategies}

\begin{document}
\maketitle 

\section{Well-bracketing and P-incremental justification}

We consider an arena $A$ and make the following two assumptions on it:
\begin{itemize}
\item (A1) For $A \neq \bot$ (the arena with a single initial question), each question move in the arena enables at least one answer move.
\item (A2) Answer moves do not enable any other move.
\end{itemize}

We define the \defname{order} of a move $m$ in the arena $A$, written $\ord_A{m}$ (or just $\ord{m}$ where there is no ambiguity), as the length of the path from $m$ to its furthest leaf in $A$ minus 1
({\it i.e.}~the height of the subarena rooted at $m$ minus 2.).
Because of assumptions (A1) and (A2),
for any move $m$ of $A \neq \bot$, $m$ is a question move if and only if $\ord{m} \geq 0$, and $m$ is an answer move if and only if $\ord{m} = -1$.





We call \defname{pending question} of a sequence of moves $s \in L_A$ the last unanswered question in $s$.

\begin{definition}\rm
A strategy $\sigma$ is said to be \defname{P-well-bracketed} if for any play $s \, a \in \sigma$ where $a$ is a  P-answer, $a$ points to the pending question in $s$.
\end{definition}



P-well-bracketing can be restated differently as the following proposition shows:
\begin{proposition}
\label{prop:char_wellbrack}
\rm We make assumption (A1) and (A2).
Let $\sigma$ be a strategy on an arena $A\neq \bot$.
The following statements are equivalent:
\begin{enumerate}
\item[(i)] $\sigma$ is P-well-bracketed,
\item[(ii)] for $s \, a \in \sigma$ with $a$ a P-answer, $a$ points to the pending question in $\pview{s}$,
\item[(iii)] for $s \, a \in \sigma$ with $a$ a P-answer, $a$ points to the last O-question in $\pview{s}$,
\item[(iv)] for $s \, a \in \sigma$ with $a$ a P-answer, $a$ points to the last O-move in $\pview{s}$ with order $>\ord{a}$.
\end{enumerate}
\end{proposition}
\begin{proof}
$(i)\iff(ii)$: \cite[Lemma 2.1]{McC96b} states that if P is to move then the pending question in $s$ is the same as that of $\pview{s}$.

$(ii)\iff(iii)$: Assumption (A2) implies that the pending question in $\pview{s}$ is also the last O-question occurring in $\pview{s}$.

$(iii)\iff(iv)$: Because of assumption (A1) and (A2),
for any move $m$, we have $m$ is a question move
if and only if $\ord{m} \geq 0$ if and only if $\ord{m} > \ord{a} = -1$.
\end{proof}




\begin{lemma}
\label{lem:justfied_by_unanswered}
Under assumption (A2), if $s$ be a justified sequence of moves satisfying alternation and visibility then any O-move (resp. P-move) in $s$ points to an \emph{unanswered} P question (resp. O-question).
\end{lemma}
\begin{proof}
Suppose that an O-move $c$ points to a P-move $d$ that has already been answered by the O-move $a$. The sequence $s$ as the following form:
$$ s= \ldots \Pstr{(d){d}  \ldots  (a-d,20){a}  \ldots  (c-d,20){c}}$$

By O-visibility, $d$ must belong to $\oview{s_{<c}}$. But since $a$ is an answer, by assumption (A2), it cannot justify any P-move, therefore
$\oview{s_{<q}}$ must contain an OP-arc ``hoping'' over $a$. We name the nodes of this arc $d^1$ and $c^1$:
$$ s = \ldots \Pstr[0.7cm]{(d){d}  \ldots  (d1){d^1} \ldots (a-d,20){a} \ldots
 (c1-d1,20){c^1} \ldots (c-d,25){c}}$$

By P-visibility, $d^1$ must belong to $\pview{s_{<c^1}}$. Consequently, $a$ does not belong to $\pview{s_{<c^1}}$ (otherwise the PO-arc $\Pstr[0.5cm]{(d){d} \quad (a-d,45){a}}$ would cause the P-view to jump over $d^1$).
Therefore there must be a PO-arc $\Pstr[0.5cm]{(d2){d^2} \quad (c2-d2,45){c^2}}$ in $\pview{s_{<c^1}}$ hoping over $a$:
$$ s = \ldots \Pstr[0.7cm]{(d){d}  \ldots
(d1){d^1} \ldots (d2){c^2} \ldots
(a-d,20){a} \ldots
 (c2-d2,20){d^2} \ldots (c1-d1,20){c^1} \ldots (c-d,25){c}}$$

This process can be repeated infinitely often by using alternatively O-visibility and P-visibility. This gives a contradiction since the sequence of moves $s_{<c}$ has finite length.
Hence $d$ cannot point to a question that has already been answered. Since, by assumption (A2), a question is enabled by another question, $d$ is necessarily justified by an unanswered question.
\end{proof}


\begin{lemma}
\label{lem:oq_in_pview_unanswered}
Under assumption (A2), if $s$ is a P-well-bracketed justified sequence of moves of odd length satisfying alternation and visibility then  all O-questions occurring in $\pview{s}$ are unanswered in $s$.
\end{lemma}
\begin{proof}
We proof the first part by induction on $s$.
The base case ($s = q$ with $q$ initial O-move) is trivial.

Suppose $\Pstr[0.4cm]{ s = s' \cdot (n)n \cdot u \cdot (m-n,45){m} }$.
Let $r$ be an O-question in $\pview{s} = \pview{s'} \cdot n \cdot m$.
If $r$ is the last move $m$ then it is necessarily unanswered.
If $r \in \pview{s'}$ then by the induction hypothesis, $r$ is unanswered in $s'$.
Suppose that $r$ is answered in $s$. This implies that some answer move $a$ in $u$ points to $r$:
$$\pstr[0.7cm][5pt]{ s = \underbrace{\cdots\ \nd(r){r}^O \cdots }_{s'} \
\nd(n){n}^P \ \underbrace{\cdots\ \nd(a-r,35){a}^P \cdots }_{u} \
\nd(m-n,30){m}^O } \ .$$
 
Since $m$ points to $n$, by lemma \ref{lem:justfied_by_unanswered}, $n$ is still unanswered at $s_{\prefixof a}$. Therefore the pending
question at $s_{\prefixof a}$ cannot be $r$. But $a$ is justified by $r$, therefore the well-bracketing condition is violated. Hence $r$ is
unanswered in $s$.
\end{proof}





\begin{definition}\rm
A play $s m$ of even length is said to be \defname{P-incrementally justified}, or {\emph P-i.j.} for short, if $m$ points to the last unanswered O-question in $\pview{s}$ with order strictly greater than $\ord{m}$.

 A strategy $\sigma$ is said to be \defname{P-incrementally justified}, if all plays in $\sigma$ ending with a P-question are
P-incrementally justified.
\end{definition}

\begin{proposition}
\label{prop:char_pincr}
\rm We make assumption (A1) and (A2).
Let $\sigma$ be a \emph{P-well-bracketed} strategy on an arena $A\neq \bot$.
The following statements are equivalent:
\begin{enumerate}
\item[(i)] $\sigma$ is P-incrementally justified,
\item[(ii)] for $s \, q \in \sigma$ with $q$ a P-question, $q$ points to the last O-question in $\pview{s}$ with order $>\ord{q}$,
\item[(iii)] for $s \, q \in \sigma$ with $q$ a P-question, $q$ points to the last O-move in $\pview{s}$ with order $>\ord{q}$.
\end{enumerate}
\end{proposition}
\begin{proof}
$(i)\iff(ii)$: By lemma \ref{lem:oq_in_pview_unanswered}, O-question occurring in $\pview{s}$ are all unanswered.

$(ii)\iff(iii)$: Because of (A1) and (A2), $\ord{q} \geq 0$ thus an O-move with order $>\ord{q}$ is necessarily an O-question.
\end{proof}

Putting proposition \ref{prop:char_pincr} and
\ref{prop:char_wellbrack} together we obtain:
\begin{proposition}
Under assumption (A1) and (A2).
A strategy $\sigma$ on $A\neq \bot$
is \emph{P-well-bracketed} and
 \emph{P-incrementally justified} if and only if
for $s \, m \in \sigma$, $m$ points to the last O-move in $\pview{s}$ with order $>\ord{m}$.
\end{proposition}


\section{Remarks}
\subsection{Homogeneity constraint}

Type homogeneity is not preserved after composition. Indeed the types  $o \typear (o \typear o)$ and $(o \typear o) \typear \left((o \typear o) \typear o \right)$ are homogeneous
but $o \typear \left((o \typear o) \typear o\right)$ is not.

If $A\typear B$ and $B \typear C$ are homogeneous types then  a sufficient condition for $A\typear C$ to be homogeneous is  ``$\ord{A} \geq \ord{B}$''.


\section{Compositionality - A semantic proof}

\subsection{Preliminaries}
 
\subsubsection{Nodes order after composition}

Consider the arena $X\fngamear Y$
and let $m$ be a move of $X\fngamear Y$.
We write $\ord_{X\fngamear Y}{m}$
to denote the order of 
$m$ in the arena ${X\fngamear Y}$.
If $m$ belongs to $X$ (resp.~$Y$) then
we write $\ord_X{m}$ 
(resp.~$\ord_Y{m}$) to denote the order of the move $m$ in the arena $X$ (resp.~$Y$).

\begin{lemma}
\label{lem:compositionorder}
Let $A$, $B$ and $C$ be three arenas. We have:
$$\begin{array}{lll}
\forall m \in A:
    &  \ord_{A\fngamear B}{m} = \ord_{A\fngamear C}{m} \ ,\\
\forall m \in B:
    & \ord_{A\fngamear B}{m} \geq \ord_{B\fngamear C}{m}  & \mbox{for $m$ initial,}\\
    & \ord_{A\fngamear B}{m} = \ord_{B\fngamear C}{m} & \mbox{for $m$ non initial,} \\
\forall m \in C:
    & \ord_{A\fngamear C}{m} \geq \ord_{B\fngamear C}{m} \iff
\ord{A} \geq \ord{B}\ & \mbox{for $m$ initial,}\\
    & \ord_{A\fngamear C}{m} = \ord_{B\fngamear C}{m}   & \mbox{for $m$ non initial.}
\end{array}
$$
\end{lemma}





\subsubsection{Interaction sequences}
Let us first recall the definition of an interaction sequence.
Let $A$,$B$ and $C$ be three games. 
We say that $u$  is an \defname{interaction sequence} of $A$,$B$ and $C$ whenever $u\filter A,B$ is a valid position of the game $A\fngamear B$
(i.e.~$u\filter A,B \in P_{A\fngamear B}$) 
and  $u\filter B,C$ is a valid position of the game
$B\fngamear C$. We write $Int(A,B,C)$ to denote
the set of all such interaction sequences.

Let $\sigma:A\fngamear B$ and $\mu:B\fngamear C$
be two strategies. We write $\sigma \parallel \mu$ to denote the 
set of interaction sequences that unfold according to the strategy $\sigma$ in the $A,B$-projection of the game and 
to $\mu$ in the $B,C$-projection:
$$ \sigma \parallel \mu = \{ u\filter A,B \in \sigma \vee u \filter B,C \in \mu \} \ .$$
The composite of $\sigma$ and $\mu$ is then defined as $\sigma ; \mu = \{ u \filter A,C \ | \ u \in \sigma \parallel \tau \}$.

The diagram below shows the structure of an interaction sequence
from $\sigma \parallel \mu$. There are four states represented by the rectangular boxes. The content of the state shows who is to play in each of the game $A\fngamear B$, $B\fngamear C$ and $A\fngamear C$.
For instance in state $OPP$, it is O's turn to play in 
$A\fngamear B$ and P's turn to play in $B\fngamear C$ and $A\fngamear C$. Arrows represent the moves.
When specifying interaction sequence,
the following bullet symbols are used to represent moves:
$\pmove$ for P-moves, $\omove$ for O-moves, $\pomove$ 
for a move playing the role of P in $A\fngamear B$
and O in $B\fngamear C$ and $\opmove$ for
the symmetric of $\pomove$.
We sometimes add a subscript to the symbols $\pmove$ and $\omove$ to denote the component in which the moves is played ($A$ or $C$).


\tikzstyle{state}=[rectangle,draw=blue!50,fill=blue!20,thick,minimum height = 4ex, text width=4cm]
\tikzstyle{move}=[->,shorten <=1pt,>=latex',line width=1pt]
\tikzstyle{intmove}=[dashed] 
\tikzstyle{extomove}=[color=\extomovecolor] 
\tikzstyle{genomove}=[]%[dashed]
\tikzstyle{genpmove}=[color=\genpmovecolor]
\def\sep{1.5cm} 
\begin{table}[htbp]
\begin{center}
\begin{tikzpicture}[node distance=1.7cm]

% the four states 
\path 
 node(oooT)  [state] {}
 node(opp)   [state, below of=oooT] {}
 node(pop)   [state, below of=opp]  {}
 node(oooB)  [state, below of=pop] {}
 node(title) [anchor=south, at=(oooT.north), minimum height = 4ex, text width=4cm] { };

\path
% text in the title centered in 3 columns
  ([xshift=-\sep]title) node {$A\fngamear B$}
        (title) node {$B\fngamear C$}
        ([xshift=\sep]title) node {$A\fngamear C$}

% text in the states centered in 3 columns
  ([xshift=-\sep]oooT) node {O}
        (oooT) node {O}
        ([xshift=\sep]oooT) node {O}
  ([xshift=-\sep]opp) node {O}
        (opp) node {P}
        ([xshift=\sep]opp) node {P}
  ([xshift=-\sep]pop) node {P}
        (pop) node {O}
        ([xshift=\sep]pop) node {P}
  ([xshift=-\sep]oooB) node {O}
        (oooB) node {O}
        ([xshift=\sep]oooB) node {O}

% text in between two arrows giving the arena of the move
  (oooT) to node {\bf C} (opp)
  (opp) to node {\bf B} (pop)
  (pop) to node {\bf A} (oooB)

% arrows representing the moves
  (opp.20)    edge[move, genpmove]
		node[right] {$\mu$}
		node[left]{$\pmove$} (oooT.-20)
  (oooT.-160) edge[move, extomove, genomove]
		node[left] {$env_\mu$}
		node[right]{$\omove$} (opp.160)
  (pop.20)    edge[move, genomove,genpmove,intmove]
		node[right] {$\sigma$}
		node[left]{$\pomove$} (opp.-20)
  (opp.-160)  edge[move, genomove, genpmove,intmove]
		node[left] {$\mu$} 
		node[right]{$\opmove$}  (pop.160)
  (oooB.20)   edge[move, extomove,genomove]
		node[right] {$env_\sigma$}
		node[left]{$\omove$} (pop.-20)
  (pop.-160)  edge[move, genpmove]
		node[left] {$\sigma$}
		node[right]{$\pmove$} (oooB.160);

%\draw[move, genpmove] (3.5cm,-1cm) -- +(1,0) node[right] {Generalised P-move \& External P-move };
%\draw[move, genomove,genpmove] (3.5cm,-2cm) -- +(1,0) node[right] {Generalised O-move \& Generalised P-move};
%\draw[move, genomove,extomove] (3.5cm,-3cm) -- +(1,0) node[right] {Generalised O-move \& External O-move};
\draw[move] (3.5cm,-1cm) -- +(1cm,0cm) node[right] {External move};
\draw[move,intmove] (3.5cm,-2cm) -- +(1cm,0cm) node[right] {Internal move};
\draw (3.5cm,-3cm) node[anchor=west] {\textcolor{\extomovecolor}{External O-moves: $\omove$}};
\draw (3.5cm,-4cm) node[anchor=west] {\textcolor{\genpmovecolor}Generalised P-move: $\opmove, \pomove, \pmove$};
\end{tikzpicture} 
\end{center}
\caption{Structure of an interaction sequence.}
\label{tab:interseq}
\end{table}

Note that in state OPP, the alternation condition (for each of the three games involved) prevents the players from playing in A. Indeed, the O-moves in component $A$ of $A\fngamear B$ are also $O$-moves in component $A$ of $A\fngamear C$ however the state name indicates that the next move in $A\fngamear B$ must be an O-move and the next move in $A\fngamear C$ must be a P-move.

Similarly, in the top state OOO, the players cannot make move in B since the O-moves in component B of the game $B\fngamear C$ correspond to P-moves in the component B of $A\fngamear B$. However the state name indicates that the next move in $A\fngamear B$ and the next move in $B\fngamear C$ must be played by O.


Let $u \in Int(A,B,C)$ and $m$ be a move of $u$.
The \defname{component} of $m$ is $A,B$ if 
after playing $m$ the game is under the control 
of the strategy $\sigma$ and $B,C$ otherwise (if $\mu$ has control).
In other words, the moves $\omove, \pmove \in A$
and $\opmove \in B$ shown on the diagram of Table \ref{tab:interseq}
have component $A,B$ and 
$\omove, \pmove \in C$ and $\pomove \in B$
have component $B,C$.


Also we call \defname{generalized O-move in component $A,B$}
moves that play the role of O in the game $A\fngamear B$, that is to say moves represented by $\opmove$ and $\omove_A$.
Similarly $\pomove$ and $\pmove_A$ moves are the \defname{generalized P-moves in component $A,B$},
$\omove_C$ and $\pomove$ moves are
the \defname{generalized O-moves in component $B,C$}
and  $\pmove_C$ and $\opmove$ moves are the \defname{generalized P-moves in component $B,C$}.

The P-view (also called {\emph core} in \cite{McCusker-GamesandFullAbstrac}) of an interaction sequence $u \in Int(A,B,C)$, written $\overline{u}$ or $\pview{u}$ is defined as:
\begin{align*}
\pview{u\cdot \extomove{n}} &= \extomove{n} &
\mbox{ if \extomove{$m$} is an \extomove{external O-move} initial in C,}\\
\pview{\Pstr{u\cdot (m)m\cdot v \cdot (n-m,45){\extomove{n}} }} &= \extomove{n} &\mbox{ if \extomove{$m$} is an \extomove{external O-move} non initial in C,}\\
\pview{u \cdot \genpmove{m}} &= \pview{u}\cdot \genpmove{m}  & \mbox{ if \genpmove{$m$} is a \genpmove{generalised P-move}.}\\ 
\end{align*}

We can show the following property by an easy induction :
\begin{lemma}
\label{lem:pviewAC_eq_ACpview}
 Let $u$ be an interaction sequence in $Int(A,B,C)$ then
$$\pview{u} \upharpoonright A,C = \pview{u \upharpoonright A,C} \ .$$
\end{lemma}

\subsection{Closed P-i.j. strategies and compositionality}

\subsubsection{Closed P-i.j strategy}

\begin{definition}
\label{def:safe_strategy}
A P-incrementally justified strategy $\sigma : A \fngamear B$ is said to be \defname{closed P-incrementally justified} if for
every initial move $m$ of $A$ such that some play $s\in\sigma$ contains $m$, we have $\ord_A{m} \geq \ord{B}$. 
\end{definition}
We observe that every P-i.j. strategy $\sigma$ on the game $I \fngamear A$ (and not $A$) is closed P-i.j..\footnote{In particular, every P-i.j. strategy $\sigma$ on the game $!A_1 \otimes \ldots \otimes !A_n \fngamear B$, is isomorphic, up to arena-tagging of the moves, to the closed P-i.j. strategy $\Lambda^n(\sigma)$ on the game $I \fngamear (A_1,\ldots,A_n,B)$, where $\Lambda$ denotes the usual currying isomorphism.}
However $\sigma : A$ is not necessarily closed P-i.j.. The reason is that the definition of closed P-i.j. strategy specifically refers to the moves of  the arena in the left-hand side of the function space arrow $\fngamear$, therefore the definition does not survive an isomorphism that retags the moves such as {\it currying}.

Consequently, for two strategies $\sigma$ and $\mu$ verifying $\sigma \cong \mu$, if $\sigma$ is closed P-i.j. then it does not necessarily imply that $\mu$ is. This contrasts with ``ordinary'' P-incremental justification condition which is preserved across any isomorphism.

Later on we will define a category of closed P-i.j. strategy. A consequence of the previous remark is that this category cannot be a closed category (neither monoidal closed nor cartesian closed).
In particular this category has only a weak form of curry isomorphism.

\subsubsection{Compositionality of closed P-i.j strategy}

{\bf Notation} In plays representations,
the symbol $\omove$ stands for an
O-move and $\pmove$ for
a P-move. 
Suppose the considered game is $L\fngamear R$ 
for some game $L$ and $R$, if we know the sub arena in which the move is played
then we specify it in subscripts ($\omove_L$, $\pmove_L$, $\omove_R$ or $\pmove_R$). For interaction sequences in $Int(A,B,C)$ we use
the symbols $\omove_A$, $\pmove_A$, $\omove_C$, $\pmove_C$, $\opmove$ and $\pomove$ as defined in Table \ref{tab:interseq}. We use the variable $X$ to denote one of the component $A,B$ or $B,C$, the variable  $Y$
then denotes the other component.
We write $s \subseqof t$ to say that $s$ is a subsequence (with pointers) of $t$, $s \prefixof t$ to say that $s$ is a prefix (with pointers)
of $t$ and  $s \suffixof t$ to say that $s$ is a suffix of $t$.

We now prove several useful lemmas which will become useful when studying compositionality of P-i.j. strategies.

\begin{lemma}
\label{lem:interjump}
Let $X$ be a component (either  $A,B$ or  $B,C$).
Let $u$ be an interaction sequence of the form
$ u =  
\Pstr[0.5cm][2pt]{ \ldots (b){\stk \beta \pmove}  \ldots
 {n}  \ldots  (a-b,30){\stk \alpha\omove}
\ldots m}$ where:
\begin{itemize}[-]
\item $\alpha,\beta$ are external moves in component $X$ (necessarily both played in $A$ or in $C$),
\item  $m$ is either played in $B$ or an external P-move in $X$,
\item  $\alpha$ is visible at $m$ in $X$ \emph{i.e.}~$\alpha\in \pview{u \upharpoonright X}$ (consequently $\beta$ is also visible).
\end{itemize}
Then $n \not\in \pview{u \upharpoonright A, C}$.
\end{lemma}
\begin{proof}
Since $\alpha$ is an O-move, $\alpha$ and $\beta$ are necessarily played in the same arena ($A$ or $C$).
Take $v=u$ if $m$ is a generalized O-move in $X$
and $v=u_{<z}$ otherwise (if $m$ is a generalized P-move in $X$).
The third assumption implies 
$\alpha,\beta\in \pview{v}$.
The last move in $v$ is necessarily a generalized O-move in component $X$ (see diagram of Table \ref{tab:interseq}) 
therefore by \cite[Lemma 3.3.1]{Harmer2005}
we have $\pview{v \filter X} = \pview{\overline{v} \filter X} \subseqof \overline{v} \subseqof \overline{u}$.
Thus $\alpha,\beta \in \overline{u}$ and
since $\alpha,\beta$ are played in $A,C$ we have 
$\alpha,\beta  \in \overline{u} \upharpoonright A,C 
= \pview{u \upharpoonright A,C}$ (Lemma \ref{lem:pviewAC_eq_ACpview}).
Finally since $n$ lies underneath the $\beta$-$\alpha$ PO-arc 
it cannot appear in the P-view  $\pview{u \upharpoonright A,C}$.
\end{proof}

\begin{lemma}
\label{lem:in_pviewAC_imp_in_pviewX}
Let $u$ be an interaction sequence in $Int(A,B,C)$ and
$n$ be a move of $u$ such that $n\in\pview{u \filter A,C}$:
\begin{enumerate}[i.]
\item 
if all the moves in $u_{\suffixof n}$ 
are played in $C$  then $n \in \pview{u \filter B,C}$;
\item 
if all the moves in $u_{\suffixof n}$ are played in $A$ then $n \in \pview{u \filter A,B}$.
\end{enumerate}
\end{lemma}
\begin{proof}
\begin{enumerate}[(i)]
\item
We show the contrapositive. Suppose that $n \not\in\pview{u \filter B,C}$. This must be due to one of the following  two
reasons:
\begin{itemize}[-]
\item $\pview{u \filter B,C}$ contains an initial move $c_0 \in C$
occurring after $n$ in $u$.


By \cite[Lemma 3.3.1]{Harmer2005}
we have $\pview{u \filter B,C} = \pview{\overline{u} \filter B,C} \subseqof \pview{u}$, thus $c_0$ also occurs in $\pview{u}$.
Since $c_0$ belongs to $C$ we have
$c_0 \in \pview{u} \filter A,C=
\pview{u \filter A,C}$ (Lemma \ref{lem:pviewAC_eq_ACpview}).
Thus the P-view $\pview{u \filter A,C}$
starts with the initial move $c_0$ and
since $n$ occurs before $c_0$, $n$ does not occur in the P-view.

\item $n$ lies underneath a PO-arc $\beta$-$\alpha$ visible 
at $ u \filter B,C$.
By assumption, since $\alpha$ occurs after $n$ in $u$, it must belong to $C$. We can therefore apply Lemma \ref{lem:interjump}
with $X\assignar B,C$ which gives
$n \not\in\pview{u \filter A,C}$.
\end{itemize}

\item Suppose that $n \not\in\pview{u \filter A,B}$ then either:
\begin{itemize}[-]
\item $\pview{u \filter A,B}$ contains an initial move $b_0 \in B$
occurring after $n$ in $u$. But this is impossible since by assumption all the moves occurring after $n$ in $u$ belong to $A$.

\item or $n$ lies underneath a PO-arc $\beta$-$\alpha$ in $A,B$.
By assumption, since $\alpha$ occurs after $n$ it must belong to $A$. We can then conclude using 
Lemma \ref{lem:interjump} with $X\assignar A,B$.
\end{itemize}
\end{enumerate}
\end{proof}

Note that we cannot completely relax the assumption 
which says that moves in $u_{\suffixof n}$ are all in the same component.
For instance take $u = \Pstr[0.5cm]{(co){\omove_C}\thinspace 
(b0-co){\opmove} \thinspace 
(n){\stk{\pmove_A}{n}} \thinspace 
(b1-co){\opmove}}$ then we have $n\in\pview{u\filter A,C}$ but $n\notin\pview{u\filter A,B}$.


%%%%%%%%%%%
% This commented Lemma could be useful be we did not make use of it eventually.
%
% \begin{lemma}
%\label{lem:oviewsegmentinB}
%For any legal sequence $s = \ldots x \cdot r \cdot y$ of a game $A\fngamear B$ if $x, y \in A$ and $x$ is O-visible from $y$ then any move in $r$ occurring in $\oview{s}$ belongs to $A$.
%\end{lemma}
%\begin{proof}
%We proceed by induction on the length of the segment $r$.
%Base case $r=\epsilon$ is trivial. Suppose $r = r' \cdot m$.
%If $y$ is an O-move then by the Switching Condition
%$m$ is necessarily in $A$. Clearly $x$ is O-visible from $m$ thus  by the I.H. any move from $r$ occurring in the O-view is in $A$.
%
%If $y$ is a P-move then it cannot point to an initial move in $B$. Indeed, suppose that it points to an initial O-move $b_0 \in B$ then
%we have $\oview{s} = b_0 \cdot y$ which contradicts the fact that $x\in \oview{s}$.
%Thus $y$ points to a move in $A$ and again we can conclude using the induction hypothesis.
%\end{proof}


\begin{lemma}[P-visibility decomposition (from $C$)]
\label{lem:middlepomove}
Let $u = \ldots n' \cdot r \cdot m \in Int(A,B,C)$ where
$n'$ is a $\omove_A$-move verifying $n' \in \pview{u\filter A,C}$ and $m$ is in $\{ \pmove_C, \opmove, \pomove \}$. Then there is a $\pomove$-move $\gamma$ in $r \cdot m$ such that $\gamma \in \pview{u\filter B,C}$ , $n' \in \pview{u_{\leq \gamma} \filter A,B}$ and $\gamma$ is justified by a move occurring before $n'$.
\end{lemma}
\begin{proof}
By induction on $|r|$.
If $r=\epsilon$ then necessarily $u = \ldots \stk{\omove_A}{n'} \thinspace\stk \pomove m$ where $m$ points before $n'$ ($n'$ being played in $A$ cannot justify $m$ played in $B$) so we just need to take $\gamma = m$.
If $|r|=1$ then either 
$u = \ldots \stk{\omove_A}{n'} \pomove\thinspace\stk {\pmove_C} m$
or $u = \ldots \stk{\omove_A}{n'} \pomove\thinspace\stk \opmove m$.
In both cases we can take $\gamma$ to be $\pomove$, the move between $n'$ and $m$.
Suppose $|r|>1$. Let $m^-$ denote the move preceding $m$ in $u$.
We proceed by case analysis:
\begin{enumerate}[i.]
\item Suppose $m = \pmove_C$ and $m^- = \omove_C$.
Let $q$ be the external P-move that justifies $m^-$.
Since $n' \in \pview{u\filter A,C}$, $q$ must occur after $n'$ in $u$:
$$ 
\begin{array}{ccccl}
A & \stackrel\sigma{\longrightarrow} & B & \stackrel\mu{\longrightarrow} & C \\
&\vdots&&\vdots\\
n' \omove\\
&\vdots&&\vdots  \\
&& & &  \rnode{q}{\pmove}q  \\
&\vdots&&\vdots  \\
&& & &  \rnode{mp}{\omove}m^-  \\
&& & &  \rnode{m}{\pmove}m  \\
\end{array}
\ncarc[arcangleA=60,arcangleB=60]{->}{mp}{q}
 $$  
Thus we can use the induction hypothesis (with $u\assignar u_{\prefixof q}$): there is a $\pomove$-move $\gamma$ 
in $u_{]n',q]}$ pointing before $n'$ such that $\gamma \in \pview{u_{\prefixof q} \filter B,C}$, $n' \in \pview{u_{\prefixof \gamma} \filter A,B}$.
Moreover $\pview{u_{\prefixof q} \filter B,C} \prefixof \pview{u_{\prefixof m} \filter B,C}$ (since $q$ is visible from $m$ in $B,C$) thus we have $\gamma \in \pview{u_{\prefixof m} \filter B,C}$ as required.

\item Suppose $m = \pmove_C$ and $m^- = \pomove \in B$.
Again we can conclude using 
the induction hypothesis with $u \assignar u_{\prefixof m^-}$.

\item Suppose $m = \pomove \in B$.

Suppose that all the moves in $r$ are in $A$.
Then $r$ is of the form $(\pmove_A \omove_A)^*$ (where $(\cdot)^*$ denotes the Kleenee star operator). 
We just need to take $\gamma = m$. 
Indeed, moves in $u_{\suffixof m}$ are all in $A$
and by assumption $n'\in\pview{u\filter A,C}$  therefore
Lemma \ref{lem:in_pviewAC_imp_in_pviewX}(ii) gives
$n'\in\pview{u\filter A,B}$.
Also, since $m$ is a $\pomove$-move, 
its justifier is a $\opmove$-move but $r$ contains only $\omove$ and $\pmove$ moves hence $m$'s justifier must occur before $n'$.

Suppose that $r$ contains at least one move in $B$. Let $b$ be the last such move, then $u$ is of the form $\ldots n' \cdot \ldots \cdot \stk\opmove  b \cdot (\pmove_A \omove_A)^* \cdot\thinspace\stk\pomove m $. We then have
$u\filter B,C = \ldots n' \cdot \ldots \cdot 
\thinspace\stk\opmove b \thinspace\cdot \stk\pomove m $ thus $b \in \pview{u\filter B,C}$. We can then conclude by applying the induction hypothesis with $u \assignar u_{\prefixof b}$.

\item Suppose $m = \pomove \in B$.
If $m^- = \opmove \in B$ then the I.H. with $u \assignar u_{\prefixof m^-}$ permits us to conclude.
If $m^- = \omove \in C$ then we conlude by applying  the I.H. on $u \assignar u_{\prefixof q}$ where $q$ is the external P-move in $C$ justifying
$m^-$.
\end{enumerate}
\end{proof}

We now show the lemma symmetric to the previous one:
\begin{lemma}[P-visibility decomposition (from $A$)]
\label{lem:middleopmove}
Let $u = \ldots n' \cdot r \cdot m \in Int(A,B,C)$ where
$n'$ is an O-move \emph{non initial} in $C$ verifying $n' \in \pview{u\filter A,C}$ and $m$ is in $\{\pmove_A, \opmove, \pomove\}$. Then there is a $\pomove$-move $\gamma$ in $r \cdot m$ such that $\gamma \in \pview{u\filter A,B}$ , $n' \in \pview{u_{\leq \gamma} \filter B,C}$ and $\gamma$ is justified by a move occurring before $n'$.
\end{lemma}
\begin{proof}
The proof is almost symmetrical to the previous one (Lemma \ref{lem:middlepomove}). We proceed by induction on $|r|$.
If $r=\epsilon$ then necessarily $u = \ldots \stk {\omove_C} {n'} \thinspace\stk \opmove m$ where $m$ points before $n'$ (it cannot point to $n'$
since $n'$ is not initial in $C$). Thus we just need to take $\gamma = m$.

If $|r|=1$ then either 
$u = \ldots \stk {\omove_C} {n'} \thinspace\opmove\thinspace\thinspace\stk{\pmove_A} m$
or $u = \ldots \stk {\omove_C} {n'} \thinspace\opmove\thinspace\thinspace\stk \pomove m$.
In both cases we can take $\gamma$ to be $\pomove$, the move between $n'$ and $m$.
Suppose $|r|>1$. Let $m^-$ denote the move preceding $m$ in $u$.
We do a case analysis:
\begin{enumerate}[i.]
\item Suppose $m = \pmove_A$ and $m^- = \omove_A$.
Let $q$ be the external P-move that justifies $m^-$.
Since $n' \in \pview{u\filter A,C}$, $q$ must occur after $n'$ in $u$:
$$ 
\begin{array}{rcccl}
A & \stackrel\sigma{\longrightarrow} & B & \stackrel\mu{\longrightarrow} & C \\
&\vdots&&\vdots\\
&&&& \omove\ n'\\
&\vdots&&\vdots  \\
q\rnode{q}{\pmove}  \\
&\vdots&&\vdots  \\
m^- \rnode{mp}{\omove}  \\
m \rnode{m}{\pmove}  \\
\end{array}
\ncarc[arcangleA=-45,arcangleB=-45]{->}{mp}{q}
 $$  
Thus we can use the induction hypothesis (with $u\assignar u_{\prefixof q}$): there is a $\opmove$-move $\gamma$ 
in $u_{]n',q]}$ pointing before $n'$ such that $\gamma \in \pview{u_{\prefixof q} \filter A,B}$, $n' \in \pview{u_{\prefixof \gamma} \filter B,C}$.
Moreover $\pview{u_{\prefixof q} \filter A,B} \prefixof \pview{u_{\prefixof m} \filter A,B}$ (since $q$ is visible from $m$ in $A,B$) thus we have $\gamma \in \pview{u_{\prefixof m} \filter A,B}$ as required.

\item Suppose $m = \pmove_A$ and $m^- = \pomove$ then again we can conclude using the I.H. with $u \assignar u_{\prefixof m^-}$.

\item Suppose $m = \opmove$.
\begin{itemize}[-]
\item Suppose that $r$ does not contain any move in $B$  then $r$ is of the form $(\pmove_C \omove_C)^*$. 

We just need to take $\gamma = m$. 
Indeed:
\begin{enumerate}
\item By lemmma \ref{lem:in_pviewAC_imp_in_pviewX}(i)
we have $n'\in \pview{u\filter B,C}$.

\item  $m$ is justified by a move occurring before $n'$. 
Indeed, if $m$ is justified by a $\pomove$-move then since $n' \cdot r$ contains only $\omove$ and $\pmove$ moves, $m$'s justifier must occur before $n'$.
If $m$'s justifier is an initial $\omove_C$-move $c_i$, then 
by P-visibility we have $c_i \in \pview{u\filter B,C}$
but since the P-view computation ``stops'' when reaching an initial moves, in order to guarantee that $n'$ also belongs to the P-view (as shown in (a)) it must
occurs after $c_i$.
\end{enumerate}


\item Suppose that $r$ contains some move in $B$. Let $b$ be the last such move. Then $u$ is of the form $u = \ldots n' \cdot \ldots \cdot \stk\opmove  b \cdot (\pmove_A \omove_A)^* \cdot\ \stk\pomove m $. 
So we have
$u\filter B,C = \ldots n' \cdot \ldots \cdot \stk\opmove  b \cdot \stk\pomove m $ hence $b \in \pview{u\filter B,C}$. We can now 
conclude by applying the I.H. with $u \assignar u_{\prefixof b}$.
\end{itemize}

\item Suppose $m = \pomove \in B$.
If $m^- = \pomove \in B$ then the I.H. with $u \assignar u_{\prefixof m^-}$ permits us to conclude.
If $m^- = \omove \in A$ then we conclude by applying the I.H. on $u \assignar u_{\prefixof q}$ where $q$ is the external P-move in $A$ justifying $m^-$.
\end{enumerate}
\end{proof}

We now use the two preceding Lemmas to show
the following useful result:
\begin{lemma}[Increasing order lemma]
\label{lem:increasing_order}
Let $u = \ldots n' \cdot r \cdot m \in Int(A,B,C)$ where
\begin{enumerate}
\item 
$n'$ is an external O-move in compoment $X$ 
($n'=\omove_A$ and $X=A,B$, or $n'=\omove_C$ and $X=B,C$)  non initial in $C$,
\item $n' \in \pview{u\filter A,C}$,
\item $m$ is either played in $B$ 
($\opmove$ or $\pomove$) or is an external
 P-move in $Y$
($\pmove_C$ if $n'=\omove_A$ and 
$\pmove_A$ if $n'=\omove_C$),
\item $m$'s justifier occurs before $n'$,
\item $u\filter X$ is P-i.j.,
\item $u_{\prefixof b}\filter Y$ is P-i.j. for all B-move $b$.
\end{enumerate}
Then:
$$ \ord_{Y} m \geq \ord_{A\fngamear C} n' \ .$$
\end{lemma}
\begin{proof}
If $n' =\omove_C$ (resp.~if $n'=\omove_A$)
then by Lemma \ref{lem:middleopmove} 
(resp.~Lemma \ref{lem:middlepomove})
there is a $\opmove$
(resp.~$\pomove$) $\gamma$ 
in $r \cdot m$ such that $\gamma \in \pview{u\filter Y}$ , $n' \in \pview{u_{\leq \gamma} \filter X}$ and $\gamma$ is justified by a move occurring before $n'$. 

There are six possible cases depending on 
the type of the moves $n'$ and $m$:
$(n',m) \in \{ \omove_A \} \times \{\pmove_C,\opmove,\pomove \} 
\union \{ \omove_C \} \times \{\pmove_A,\opmove,\pomove \} $).
The following diagram illustrates the cases $(n',m)
 = (\omove_A,\pmove_C)$ (left)
and  $(n',m)
 = (\omove_C,\pmove_A)$  (right):
$$ 
\begin{array}{ccccc}
A & \longrightarrow & B &
 \longrightarrow & C \\
&\vdots&&\vdots\\
&&&& \rnode{n}{\omove} \\
&\vdots&\rnode{gj}{\opmove}&\vdots\\
n' \omove \\
&\vdots&&\vdots  \\
&&\rnode{g}{\gamma} \pomove \\
&\vdots&&\vdots  \\
&&&&\rnode{m}{m} \pmove \\
\end{array}
\ncarc[arcangleA=30,arcangleB=30]{->}{m}{n}
\ncarc[arcangleA=30,arcangleB=30]{->}{g}{gj}
\hspace{2cm} \begin{array}{ccccc}
A & \longrightarrow & B & \longrightarrow & C \\
&\vdots&&\vdots\\
& \rnode{n}{\omove} \\
&\vdots&\rnode{gj}{\pomove}&\vdots\\
&&&&n' \omove \\
&\vdots&&\vdots  \\
&&\rnode{g}{\gamma} \opmove \\
&\vdots&&\vdots  \\
\rnode{m}{m} \pmove \\
\end{array}
\ncarc[arcangleA=30,arcangleB=30]{->}{m}{n}
\ncarc[arcangleA=30,arcangleB=30]{->}{g}{gj}
 $$  

We have:
\begin{equation}
\ord_Y \gamma \geq \ord_X \gamma \label{eqn:gammaorderXY}
\end{equation}
Indeed, if $n' =\omove_C$ then $X=B,C$ and $Y=A,B$ and
by Lemma \ref{lem:compositionorder} we have
$\ord_{A\fngamear B} \gamma \geq \ord_{B\fngamear C} \gamma$.
If $n=\omove_A$ then $\gamma$ is a $\pomove$-move therefore it is not initial in $B$ and Lemma \ref{lem:compositionorder} gives
$\ord_{A\fngamear B} \gamma = \ord_{B\fngamear C} \gamma$.

Hence:
\begin{align*}
\ord_{A\fngamear C} n' 
& = \ord_{X} n' & \mbox{(n' non initial in $C$ \& Lemma \ref{lem:compositionorder})} \\
& \leq \ord_{X} \gamma & \mbox{($u_{\prefixof \gamma}\filter Y$ is P-i.j. \& $\gamma$'s justifier occurs before $n'$)} \\
& = \ord_{Y} \gamma & \mbox{(By Eq. \ref{eqn:gammaorderXY})} \\
& \leq \ord_{Y} m & \mbox{($u\filter X$ is P-i.j. \& 
4$^{th}$ assumption: $m$'s justifier occurs before $\gamma$)}. 
\end{align*}
\end{proof}


\begin{proposition}
\label{prop:pijcompose_when_orda_geq_ordb}
Let $\sigma : A \fngamear B$ and $\mu : B \fngamear C$
be two well-bracketed (P-visible) strategies then
\begin{enumerate}[(I)]
\item if $\sigma$ is closed P-i.j. and $\mu$ is P-i.j.
then $\sigma ; \mu : A \fngamear C$ is P-i.j.,
\item if $\sigma$ and $\mu$ are closed P-i.j.
then $\sigma ; \mu : A \fngamear C$ is closed P-i.j.
\end{enumerate}
\end{proposition}

\begin{proof}
Well-bracketing is preserved by strategy composition (see \cite[Proposition 2.5]{abramsky94full}) thus
$\sigma ; \mu$ is well-bracketed so we can use the definition of P-i.j. from Proposition \ref{prop:char_wellbrack}.

\noindent (I) Let us prove that $\sigma ; \mu$ is P-i.j..
Let $u$ be a play of the interaction $\sigma\ \|\ \mu$ between $\sigma$ and $\mu$
ending with an external P-move $m$
justified by $n$ in $\pview{u \upharpoonright A , C}$.
Let $n'$ be an external O-move occurring betweeen $n$ and $m$:
$$ u \filter A,C =  
\Pstr[0.5cm][2pt]{ \ldots (n){\stk {n} \omove}  \ldots
 {\stk {n'} \omove}  \ldots  (m-n,30){\stk m \pmove}
}
$$
To show that $u \filter A,C$ is P-incrementally justified, we just need to prove that either $n'\not\in \pview{u \filter A,C}$ or $\ord_{A\fngamear C} n' \leq \ord_{A\fngamear C} m$. 
Note that if $n'\in \pview{u \filter A,C}$ 
then necessarily $n'$ is not initial 
in $C$ because $n$ occurs before $n'$ in
$\pview{u \filter A,C}$.

\begin{enumerate}[1)]
\item \label{case:mC}
Suppose $m =\pmove_C$, then necessarily $n=\omove_C$.

\begin{enumerate}[{\ref{case:mC}}.a)]
\item \label{case:mCnpC} Suppose $n' = \omove_C$. The projection of $u$ on the game $B\fngamear C$ has the following form:
$$ u \filter B,C =  
\Pstr[0.5cm][2pt]{ \ldots (n){\stk {n} {\omove_C}}  \ldots
 {\stk {n'}{\omove_C}}  \ldots  (m-n,30){\stk m {\pmove_C}}
}$$

Suppose that $n'$ occurs in the P-view $\pview{u\filter B,C}$ then necessarily $n'$ is not initial 
in $C$ (otherwise it would be the first move in
$\pview{u \filter B,C}$, which is not the case since $n$ occurs before $n'$ in the P-view) and we have:
\begin{align*}
\ord_{A\fngamear C} n' 
& = \ord_{B\fngamear C} n' & \mbox{(Lemma \ref{lem:compositionorder} \& $n'$ non initial in $C$)} \\
& \leq \ord_{B\fngamear C} m & \mbox{($\mu$ is P-i.j.)} \\
& = \ord_{A\fngamear C} m & \mbox{($m$ is not initial in $C$)} \ .
\end{align*}

Suppose that $n'$ does not occur in the P-view $\pview{u \filter B,C}$, then $n'$ lies underneath a PO arc occurring in $\pview{u \filter B,C}$. Let us denote this arc by $\beta$-$\alpha$ where $\beta$ and $\alpha$ denote the arc's nodes. We have:
$$ u \filter B,C = \ldots  
\Pstr[0.5cm]{
 (n){\stk{n} {\omove_C} } \ldots (b){\stk\beta \pmove} \ldots \stk{n'} {\omove_C}
\ldots (a-b){\stk\alpha \omove}  \ldots (m-n){\stk m {\pmove_C} }
} $$
with $\ord_{B\fngamear C} \alpha \leq \ord_{B\fngamear C} m$ (by P-i.j. of $\mu$).
  
\begin{enumerate}[i.]
\item Suppose $\alpha$ is a $\omove_C$-move in $u$ then necessarily
$\beta$ is a $\pmove_C$-move (since $\alpha$ is an O-move).

By instancing Lemma \ref{lem:interjump} with
$X\assignar B,C$ and $n \assignar n'$ we obtain that $n' \not\in\pview{u \filter A,C}$.

\item Suppose $\alpha$ is a $\pomove$-move in $u$ then necessarily $\beta$ is a $\opmove$-move.

We have $\alpha \in \pview{u \filter B,C}
= \pview{\pview{u} \filter B,C} \subseqof
\pview{u}$ (\cite[Lemma 3.3.1]{Harmer2005}).

Suppose that $n' \in \pview{u\filter A,C}$, then 
\begin{align*}
n'  \in \pview{u\filter A,C} 
& = \pview{u}\filter A,C   
& \mbox{(Lemma \ref{lem:pviewAC_eq_ACpview})} \\
& \suffixof \pview{u_{\prefixof \alpha}}\filter A,C
& \mbox{($\alpha \in \pview{u}$, $n'$ occurs before $\alpha$ in $u$)} \\
& = \pview{u_{\prefixof \alpha} \filter A,C} 
& \mbox{(Lemma \ref{lem:pviewAC_eq_ACpview})}\ .
\end{align*}
But since $n'$ occurs before $\alpha$ in $u$, the previous equation 
implies that $n' \in \pview{u_{\prefixof \alpha}\filter A,C}$
so we can apply Lemma \ref{lem:increasing_order}
on $u_{\prefixof \alpha}$: 
\begin{align*}
\ord_{A\fngamear C} n' 
& \leq \ord_{A\fngamear B} \alpha & \mbox{(Lemma \ref{lem:increasing_order} with $u\assignar u_{\prefixof \alpha}$)} \\
& = \ord_{B\fngamear C} \alpha & \mbox{($\alpha$ non initial in $B$)} \\
& \leq \ord_{B\fngamear C} m & \mbox{($\mu$ is P-i.j.)} \\
& = \ord_{A\fngamear C} m & \mbox{($m$ is not initial in $C$)} \ .
\end{align*}
\end{enumerate}


\item Suppose $n' =\omove_A$.

Suppose that $n' \in \pview{u\filter A,C}$, then by
Lemma \ref{lem:increasing_order} we have
$ \ord_{A\fngamear C} n'  \leq \ord_{B\fngamear C} m$.
 By Lemma \ref{lem:compositionorder}, since
 $m$ is not initial in $C$ we have $\ord_{B\fngamear C} m = \ord_{A\fngamear C} m$ thus $\ord_{A\fngamear C} n' \leq \ord_{A\fngamear C} m$.
\end{enumerate}

\item \label{case:mA} Suppose $m = \pmove_A$.
\begin{enumerate}[{\ref{case:mA}}.1)]

\item Suppose $n=\omove_A$.
This case is symmetrical to case
\ref{case:mC}) where $m=\pmove_C$ and $n=\omove_C$.

\item Suppose  $n$ is an $\omove_C$-move initial in $C$.
Then $m$ is initial in $A$ and points to a move
$b_0 = \opmove$ initial in $B$ which in turn points to the
move $n =\omove_C$ initial in $C$.

Let us assume that $n'\in \pview{u\filter A,C}$ and let us prove that necessarily $\ord_{A\fngamear C} n' \leq \ord_{A\fngamear C} m$.

	\begin{enumerate}[a.]
	\item Suppose $n'=\omove_A$. Then
since $m$ is initial in $A$ we clearly have 
$\ord_{A\fngamear C} n' \leq \ord_{A\fngamear C} m$.

	\item Suppose $n'=\omove_C$.
	If $n'$ occurs between $b_0$ and $m$ then $m$ points before $n'$ in 		$u$ so we can use Lemma \ref{lem:increasing_order} which gives $\ord_{A\fngamear C} n' \leq \ord_{A\fngamear B} m
		= \ord_{A\fngamear C} m$.
		\smallskip
		
		
		If $n'$ occurs before $b_0$ then 
		we cannot use Lemma \ref{lem:increasing_order}
		since $m$ does not point before $b_0$.
Up to now we have only used the fact that $\sigma$ and $\mu$ are P-i.j. The assumption that $\sigma$ is  \emph{closed} P-i.j. now becomes crucial. We have:
		\begin{align}
		\ord_{A\fngamear B} b_0 
		& = \max(\ord A +1, \ord_B b_0) & \mbox{($b_0$ initial in $B$)} \nonumber \\
		& \geq \ord_B b_0 \nonumber  \\
		& \geq \ord C & \mbox{($\sigma$ is closed P-i.j. \& $b_0 \in u \filter  B,C \in \mu$)} \nonumber  \\
		& \geq \ord_C n' +1 & \mbox{($n'$ not initial in $C$)} \nonumber  \\
		& = \ord_{A\fngamear C} n' +1 & \mbox{(Lemma \ref{lem:compositionorder} \& $n'$ not initial)}. \label{eq:b0gtnp}
		\end{align}
Thus:
		\begin{align*}
		\ord_{A\fngamear C} m 
		& = \ord_{A\fngamear B} m & \mbox{(Lemma \ref{lem:compositionorder})} \\
		& = \ord_{A\fngamear B} b_0 -1  & \mbox{($m$ justified by $b_0$ initial in $B$)} \\
		& \geq \ord_{B\fngamear C} b_0 -1 & \mbox{($b_0$ initial in $B$)} \\
		& \geq \ord_{A\fngamear C} n' & \mbox{(By Eq. \ref{eq:b0gtnp})} \ .
		\end{align*}
		\end{enumerate}

\end{enumerate} 
\end{enumerate} 

\noindent (II)
To show that $\sigma;\mu$ is closed P-i.j., we need to verify that for all initial move in $A$ occurring in some play of $\sigma ; \mu$ has order in $A$ greater or equal to $\ord C$.
Take an initial move $m \in A$ such that $m$ occurs in $s \in \sigma ; \mu$. Then $s = u \filter A,C$ for some $u \in \sigma 
\ \|\ \mu$. Let $n$ be the initial move of $B$ justifying $m$.
 We have:
\begin{align*}
\ord_A m & \geq \ord B & \mbox{($\sigma$ is closed P-i.j. \& $m \in u
\filter A,B \in \sigma$)} \\
 & \geq \ord_B n & \mbox{($n 
\in B$)} \\
 & \geq \ord C & \mbox{($\mu$ is closed P-i.j. \& $n \in u
\filter B,C \in \mu$)}.
\end{align*}
\end{proof}
We recall some definitions. Let $A$ and $B$ be two well-opened games. Given a strategy  $\sigma :\ !A \fngamear B$, its promotion $\sigma^\dag :\ !A\fngamear !B$ is defined as
$$ \sigma^\dag = \{ s \in L_{!A\fngamear !B}\ |\ \mbox{for all inital $m$ in $B$, } s\filter m \in \sigma \}$$
and for $\mu :\ !B\fngamear C$ the composite strategy $\sigma \fatcompos \mu$ is defined as:
$$ \sigma \fatcompos \mu = \sigma^\dag ; \mu \ .$$

\begin{proposition}
If $A$ and $B$ are two well-opened games 
and $\sigma :\ !A \fngamear B$ is a well-bracketed P-incrementally justified strategy then $\sigma^\dag$ is also well-bracketed and P-incrementally justified.
Furthermore if $\sigma$ is closed P-i.j. then so is
$\mu$.
\end{proposition}
\begin{proof}
$\sigma^\dag$ is well-bracketed by \cite[Proposition 2.10.]{abramsky94full}.

By the visibility condition,
\end{proof}

From the last two propositions we obtain:
\begin{corollary}
Let $A$ and $B$ be two well-opened games. Let
$\sigma :\ !A \fngamear B$ and $\mu :\ !B\fngamear C$ be two well-bracketed strategies then:
\begin{enumerate}
\item if $\sigma$ is closed P-i.j. 
and $\mu$ is P-i.j. then $\sigma \fatcompos \mu :\ !A \fngamear C$ is also P-i.j.
\item if $\sigma$ and $\mu$ are closed P-i.j. then so is $\sigma \fatcompos \mu :\ !A \fngamear C$.
\end{enumerate}
\end{corollary}

\section{Compositionnality - Syntactic approach}

\subsection{Definability result for Safe PCF}

We say that a PCF term is \defname{semi-safe} if it is of the form $N_0 N_1 \ldots N_k$ for $k\geq 1$ where each of the $N_i$ is safe or if it can be written $\lambda \overline{x} . N$ for some safe term $N$. Semi-safe terms are either safe or ``almost safe'' in the sense that they can be turned into an equivalent (i.e.~with isomorphic game semantics) safe term  by performing $\eta$-expansions. Indeed, let $M$ be an semi-safe term that is unsafe.
If $M$ is of the first form $N_0 N_1 \ldots N_k : (A_1,\ldots,A_n)$ with $k\geq 1$ then let $\varphi_i:A_i$ for $i\in\{1..n\}$ be fresh variables, using the (app) and (abs) rules we can build the safe term $\lambda \varphi_1 \ldots \varphi_n . N_0 N_1 \ldots N_k \varphi_1 \ldots \varphi_n$. If $M$ is of the second form $\lambda \overline{x} . N$ then using the abstraction rule we can build the equivalent safe term $\lambda \overline{y} \overline{x}. N$  where $\overline{y} = fv(\lambda \overline{x}. N)$.

The $\beta$-normal form of a \pcf\ term is the possibly infinite term obtained by reducing all the redexes in $M$.

The correspondence between safety and P-incremental justification in the context of the simply typed lambda calculus was shown
in \cite[Theorem 3(ii)]{blumong:safelambdacalculus}:

\begin{theorem}[\cite{blumong:safelambdacalculus},Theorem 3(ii)]
\label{thm:safeincrejust} In the simply typed lambda calculus:
\begin{enumerate}[(i)]
\item If $M$ is safe then $\sem{M}$ is P-incrementally justified.
\item If $M$ is a closed term and $\sem{M}$ is
  P-incrementally justified then the $\eta$-long form of the
  $\beta$-normal form of $M$ is safe.
\end{enumerate}
\end{theorem}
In fact the proof of this theorem can be easily adapted to show a more precise result:
\begin{theorem}[Semi-safety and P-incremental justification]
\label{thm:semisafeincrejust} Let $\Gamma \vdash M : A$ be a simply typed term. Then:
\begin{enumerate}[(i)]
\item If $\Gamma \vdash M : A$ is semi-safe then $\sem{\Gamma \vdash M : A}$ is P-incrementally justified.
\item If $\sem{\Gamma \vdash M : A}$ is
  P-incrementally justified then 
$\etanf{\betanf{M}}$ is semi-safe if $M$ is open
and safe if $M$ is closed.
\end{enumerate}
\end{theorem}



In the context of \pcf\, only the first part of the theorem holds (see \cite{blumtransfer} for the proof). However (ii) does not hold. Indeed, take the closed \pcf\ term $M = \lambda f x y. f (\lambda z. \pcfcond (\pcfsucc\ x) y z )$ where $x,y,z:o$ and $f:((o,o),o)$. $M$ is in normal form (the conditional  could  only be reduced if $x$ were first evaluated). The $\eta$-long form of the $\beta$-normal form of $M$ is therefore $M$ itself which is unsafe.
But clearly we have $\sem{M} = \sem{\lambda f x y. f (\lambda z. z)}$, and since  $\lambda f x y. f (\lambda z. z)$ is safe, by (i), $\sem{M}$ is P-incrementally justified.

Such counter-example arises because the conditional operator of \pcf\ permits us to build terms in normal form containing ``dead code'' {\it i.e.}~some subterm that will never be evaluated for any value of M's parameters. In the example given above, the dead code consists in the subterm $y$. In general, if the dead code part of the computation tree contains a variable that is not incrementally bound then the resulting term will be unsafe even if the rest of the tree is incrementally bound.
In the example above, it was possible to turn $M$ into the equivalent safe term $\lambda f x y. f (\lambda z. z)$ by eliminating the dead code from $M$.
We shall see how to generalise this to any \pcf\ term with a P-incrementally justified denotation.

Dead code elimination can be difficult to achieve in practice but the formal definition is not difficult to formulate. We say that a subterm $N$ occurring
in a context $C[-]$ in $M : (A_1, \ldots, A_n,o)$ is part of the \defname{dead code} of $M$ if for any term $T_0$ of the form $M M_1 \ldots M_n$,
any reduction sequence starting from $T_0$ does not involve a reduction of the subterm $N$ {\it i.e.}~for any reduction sequence $T_0 \redar T_1 \redar \ldots \redar T_k$, there is no $j\in \{0.. k-1\}$ such that $T_j = C[N]$ and $T_{j+1} = C[N']$ for some term $N'$.
 

Let $M$  be a \pcf\ term in $\eta$-nf.
An occurrence of a variable $x$ in $M$ is said to be a \defname{dead occurrence}
if it occurs in the dead code of $M$. In other words, it is a
dead occurrence of $x$ if the corresponding node in the computation tree does not appear in any traversal of $\travset(M)$. Equivalently, thanks to the Correspondence Theorem, an occurrence of $x:B$ is dead if and only if the initial move
of the arena $\sem{B}$ does not appear in any play of $\sem{M}$.


We define $M^*$ as the term obtained from $M$ after substituting all subterms of the form  $x N_1 \dots N_k$ for some dead variable occurrence $x:(B_1,\ldots, B_k, o)$ by the constant $0$. This process is called \defname{dead variable elimination}.
Note that if $M$ is in $\eta\beta$-nf then so is $M^*$.

We also write $\tau(M)^*$ to denote the equivalent transformation on the computation tree. Since the computation tree is constructed from the $\eta$-nf of $M$, we will use this notation even when $M$ is not in $\eta$-nf.



\begin{proposition}[Incremental-binding and P-incremental justification coincide] \
\label{prop:incrbound_imp_incrjustified_pcf} Let $\Gamma \vdash M : A$ be a PCF term in $\beta$-normal form.
\begin{enumerate}[(i)]
\item  If $\tau(\Gamma \vdash M : A)$ is incrementally-bound then $\sem{\Gamma \vdash M : A}$ is P-incrementally justified,
\item  if $\sem{\Gamma \vdash M : A}$ is P-incrementally justified
then $\tau(\Gamma \vdash M : A)^*$ is incrementally-bound.
\end{enumerate}
\end{proposition}
\begin{proof}
(i) The proof is exactly the same as in the simply typed lambda calculus case,
see \cite[Proposition 4.1.5(i)]{blumtransfer}.

\noindent (ii)
Take $\Gamma \vdash M : A$ a \pcf\ term in $\beta$-normal form denoted by $\sem{\Gamma \vdash M : A}$ P-incrementally justified. Let $r$ denote the root of $\tau(M)^*$.
Let $n$ be a node of $\tau(M)^*$ labelled by the variable $x$.
$\tau(M)^*$ is free from dead code therefore $n$ is not a dead occurrence of $x$ and there exists a traversal of $\tau(M)^*$ of the form $t \cdot x$.

\pcf\ constants are of order $1$ at most therefore they cannot hereditarily justify a variable node, thus $x$ is necessarily hereditarily justified by the root $r$ of the computation tree.


By considering $t\cdot x$ as a traversal of $\tau(M)$,  the correspondence theorem gives $\varphi((t \cdot x) \upharpoonright r) = \varphi((t \upharpoonright r) \cdot x) \in \sem{M}$. Since $\sem{M}$ is P-incrementally justified, $\varphi(x)$ must point to the last O-move in $\pview{?(\varphi(t \upharpoonright
r))}$ with order strictly greater than $\ord{\varphi(x)}$.
Consequently $x$ points to the last node in $\pview{?(t
\upharpoonright r)} \inter N^{\lambda}$ with order strictly greater than $\ord{x}$. We have:
\begin{align*}
\pview{?(t \upharpoonright r)} &= \pview{?(t) \upharpoonright r} = \pview{?(t)} \upharpoonright r & (\mbox{by \cite[lemma 3.1.23]{blumtransfer}}) \\
& = [r,x[ \ \upharpoonright r & (\mbox{by \cite[proposition 3.1.20]{blumtransfer}})
\end{align*}
Since $M$ is in $\beta$-nf, the set of nodes not hereditarily justified by $r$ is exactly the set of nodes hereditarily justified by $N_{\Sigma}$ thus
$[r,x[ \ \upharpoonright r = [r,x[\ \setminus\  N^{\upharpoonright \Sigma}$.
Moreover \pcf\ constants are of order $1$ at most therefore $N^{\upharpoonright \Sigma} = N_{\Sigma} \union N^c_{\Sigma}$
where $N^c_{\Sigma}$ is the set of children nodes of $N_{\Sigma}$.
Thus $(\pview{?(t \upharpoonright r)}\upharpoonright r) \inter N^{\lambda} =
([r,x[\ \setminus\  N_{\Sigma} \setminus N^c_{\Sigma} ) \inter N^{\lambda} =
([r,x[\ \setminus\  N^c_{\Sigma} )  \inter N^{\lambda}$, and
since $N^c_{\Sigma}$ is constituted of order $0$ lambda-nodes only we have that
$x$ points to the last node in $[r,x[ \inter N^{\lambda}$ with order strictly greater than $\ord{x}$.

Hence if $x$ is a bound variable node then it is bound by the
last $\lambda$-node in $[r,x[$ with order strictly greater than
$\ord{x}$ and if $x$ is a free variable then it points to $r$ and
therefore all the $\lambda$-node in $]r,x[$ have order smaller than
$\ord{x}$. Thus $\tau(M)^*$ is incrementally-bound.
\end{proof}

The counterpart of Lemma 4.1.6 from
\cite{blumtransfer} can be stated as follows in the context of PCF:
\begin{lemma}[Semi-safety and incrementally-binding]
\label{lem:safe_imp_incrbound_pcf} Let $\Gamma \vdash M : A$ be a PCF term.
\begin{itemize}
\item[(i)] If $\Gamma \vdash M : A$ is a semi-safe term then $\tau(\Gamma \vdash M : A)$ is incrementally-bound ;
\item[(ii)] conversely, if $\tau(\Gamma \vdash M : A)$ is incrementally-bound then the $\eta$-normal form of $\Gamma \vdash M : A$ is semi-safe if $M$ is open and safe if $M$ is closed.
\end{itemize}
\end{lemma}
The proof can be obtained by adapting the proof 
of Lemma 4.1.6 from \cite{blumtransfer}.

\begin{theorem}[Semi-safety and P-incremental justification]
\label{thm:semisafeincrejust_pcf} Let $\Gamma \vdash M : A$ be a PCF term. Then:
\begin{enumerate}[(i)]
\item If $\Gamma \vdash M : A$ is semi-safe then $\sem{\Gamma \vdash M : A}$ is P-incrementally justified.
\item If $\sem{\Gamma \vdash M : A}$ is
  P-incrementally justified then $\etanf{\betanf{M}}^*$ is semi-safe  if $M$ is open, and safe if $M$ is closed.
\end{enumerate}
\end{theorem}

\begin{proof}
\noindent(i)
A proof of this is given in the proof of Theorem 4.2.10 in \cite{blumtransfer}.

\noindent(ii) 
Suppose $M$ is a \pcf\ term with a P-incrementally justified strategy denotation. By Proposition \ref{prop:incrbound_imp_incrjustified_pcf}(ii), $\tau(\betanf{M})^* = \tau(\etanf{\betanf{M}}^*)$ is incrementally-bound.
If $M$ is closed then so is $\etanf{\betanf{M}}^*$ therefore by Lemma \ref{lem:safe_imp_incrbound_pcf}, $\etanf{\etanf{\betanf{M}}^*} = \etanf{\betanf{M}}^*$ is safe. If $M$ is open then so is $\etanf{\betanf{M}}^*$ and by Lemma \ref{lem:safe_imp_incrbound_pcf}, $\etanf{\etanf{\betanf{M}}^*} = \etanf{\betanf{M}}^*$ is semi-safe.
\end{proof}


We write \pcf' to denote the language obtained by extending \pcf\
with the $\pcfcase_k$ construct (see \cite{Abr02}).
The $\pcfcase_k$ construct is the obvious generalisation of the
conditional operator \pcfcond\ to $k$ branches instead of $2$. All the results obtained so far concerning Safe \pcf\ (including those
cited from \cite{blumtransfer}) can clearly be transposed to \pcf'.

The previous theorem leads to the following definability result for safe \pcf':
\begin{proposition}[Definability for safe \pcf' terms]
\label{prop:safetydefinability}
Let $\overline{A}=(A_1,\ldots, A_i)$ and $B =(B_1, \ldots, B_l,o)$ be two PCF types for some $i,l\geq 0$ and $\sigma$ be a well-bracketed innocent
P-i.j. strategy with finite view function defined on the game $!A_1 \otimes \ldots \otimes !A_i \fngamear (!B_1 \fngamear \ldots \fngamear !B_l \fngamear o) $. There exists a \emph{semi-safe} PCF' term $\overline{x} : \overline{A} \vdash M : B$ in $\eta$-long normal form such that:
$$ \sem{\overline{x} : \overline{A} \vdash M_\sigma : B} = \sigma $$
and a safe closed PCF' term $\vdash_s M'_\sigma : (\overline{A},B)$ in $\eta$-long normal form such that:
$$ \sem{\vdash M'_\sigma : (\overline{A},B)} \cong \sigma \ .$$
\end{proposition}
\begin{proof}
By the standard definability result for PCF', there is a term $\overline{x} : \overline{A} \vdash N : B$ such that $\sem{\overline{x} :\overline{A} \vdash N : B} = \sigma$.
Take $M_\sigma$ to be $\etanf{\betanf{N}}^* $. We have $\sem{\overline{x} : \overline{A} \vdash M_\sigma : B} =  \sem{\overline{x} :\overline{A} \vdash N : B} = \sigma$ and by Theorem  \ref{thm:semisafeincrejust_pcf}(ii), $M_\sigma$ is semi-safe.
For the second part we just need to take $M'_\sigma = \lambda \overline{x}. M_\sigma$.
\end{proof}


\subsection{Composition of P-i.j. strategies}

Let $\overline{A} = (A_1, \ldots, A_i)$,
$B = (B_1, \ldots, B_l,o)$ 
and $C=(C_1,\ldots,C_k,o)$ be three PCF types
for some $i\geq 1,l,k\geq 0$. Let
$f:\ !A_1 \otimes \ldots \otimes !A_i \fngamear B$ and $g:\ !B\fngamear C$ be two innocent well-bracketed and P-incrementally justified strategies with finite view function.
We would like to find under which conditions the composition $f\fatcompos g$ is also P-incrementally justified.

By the definability result, there are two closed safe terms (in $\eta$-nf) $\vdash M_f :(\overline{A},B)$  and $\vdash M_g :B \typear C$ such that $\sem{M_f} = f$
and $\sem{M_f} = g$.
We define the term $M_{f\fatcompos g} = \lambda \overline{x} . M_g (M_f \overline{x})$ for some fresh variables $\overline{x} : \overline{A}$. Clearly we have $\sem{M_{f\fatcompos g}} = \sem{M_f} \fatcompos \sem{M_g} = f\fatcompos g$.

By Theorem \ref{thm:semisafeincrejust_pcf}, we know that $f\fatcompos g$ is P-incrementally justified just when $\etanf{\betanf{M_{f\fatcompos g}}}^*$ is safe. 
We will now exploit this fact to extract a sufficient condition on the types $A$ and $B$ for 
the composition of $f$ and $g$ to be P-incrementally justified.

The term $M_f$ and $M_g$, being in $\eta$-nf, are of the following forms:
\begin{eqnarray*}
\vdash M_f &=& \lambda x_1^{A_1} \ldots x_i^{A_i} \varphi_1^{B_1} \ldots \varphi_l^{B_l} . N_f^o\\
\vdash  M_g &=& \lambda y^{ (B_1, \ldots, B_l,o)} \phi_1^{C_1} \ldots \phi_k^{C_k} . N_g^o
\end{eqnarray*}
for some distinct variables $x_1, \ldots, x_i$, $y$, $\varphi_1, \dots \varphi_l$, $\phi_1, \dots \phi_k$  and $\eta$-normal terms $N_f$ and $N_g$:
\begin{eqnarray*}
x_1:A, \ldots, x_i:A_i, \varphi_1:B_1, \dots, \varphi_l:B_l &\vdash& N_f :o \\
y: (B_1, \ldots, B_l,o), \phi_1:C_1, \dots, \phi_l:C_l &\vdash& N_g :o
\end{eqnarray*}


 
The fact that $M_f$ and $M_g$ are safe does not imply that $M_{f\fatcompos g}$ is: take $M_f = \lambda x^o z^o.x$ and $M_g = \lambda y^{(o,o)} . y a$ for some constant $a\in \Sigma$, then $\lambda x:A . M_g (M_f x) = \lambda x . (\lambda y . y a) ( \underline{(\lambda x z.x) x} )$ is unsafe because of the underlined subterm. However we have:
\begin{align*}
f\fatcompos g &= \sem{\lambda \overline{x} . M_g (M_f  \overline{x})} \\
 &= \sem{\lambda \overline{x} . (\lambda \phi_1\ldots \phi_k . N_g) [(M_f \overline{x}) / y]} \\
&= \sem{\lambda \overline{x} \phi_1 \dots \phi_k. N_g [(M_f  \overline{x}) / y]}
& \mbox{(the $x_j$'s and $\phi_j$'s are disjoint)}.
\end{align*}

We now concentrate on the term  $\lambda \overline{x} \phi_1 \dots \phi_k. N_g [(M_f  \overline{x}) / y]$ and try to find a sufficient condition guaranteeing its safety.
\subsubsection{A sufficient condition}
\begin{lemma}
Suppose that $\Gamma,y:B \vdash M$ is a safe term in $\eta$-nf and $\Gamma \vdash R : B$ is an almost safe application. Let $N$ denote the set of nodes of the computation tree $\tau(M)$. We have:
\begin{align*}
\Gamma \vdash M[R/y] :A \mbox{ safe } 
\iff&  \forall x \in fv(R) . \\
    & \forall n_y \in N_{fv} \mbox{ labelled $y$}.
      \forall m \in N_{\lambda} \inter ]r,n_y] : \ord{m} \leq \ord{x}
\end{align*}
\end{lemma}
\begin{proof}
Since $M$ is in $\eta$-nf, all the application to the variable $y$ are total (i.e.~of the form $y P_1 \ldots P_l :o$). Hence after substituting the safe term $N$ for $y$ in $M$, the only possible cause of unsafety is when
some variable free in $N$ becomes not safely bound in $\tau(M)$.
\end{proof}

Applying this lemma with $R= M_f \overline{x}$ gives us a sufficient condition (in the right-hand side of the equivalence) 
for $\lambda x \phi_1 \dots \phi_k. N_g [(M_f \overline{x}) / y]$ to be safe , and hence for $f\fatcompos g$ to be P-incrementally justified. Of course it is not a necessary condition since the 
eta-beta normal form of  $N_g[(M_f \overline{x}) /y]$ can be safe even though it is unsafe.

\subsubsection{A simpler sufficient condition}
\begin{lemma}
If $y:B, \Sigma \vdash N : T$ and $\vdash M : (\overline{A}, B)$ 
are safe terms with $\ord{A_i} \geq \ord{B}$ for all $i\in 1..n$
then $\overline{x}:\overline{A}, \Sigma \vdash N[(M \overline{x})/y] :T$ is also safe.
\end{lemma}
\begin{proof}
Since $\ord{x_i} = \ord{A_i} \geq \ord{B} = \ord{M \overline{x}}$, we can use the application 
rule of the safe lambda calculus to form the safe term $\overline{x}:\overline{A} \vdash M \overline{x}$.
Using the substitution lemma we have that $N[(M \overline{x})/y]$ is safe.
\end{proof}

Hence if we have $\ord{A_i}\geq\ord{B}$
for all $1 \leq i \leq n$,
then by the previous lemma the term $\vdash \lambda \overline{x} \phi_1 \dots \phi_k. N_g [(M_f \overline{x}) / y]$
is safe and therefore its denotation $\sem{\vdash \lambda \overline{x} \phi_1 \dots \phi_k. N_g [(M_f \overline{x}) / y]} = f\fatcompos g$ is P-incrementally justified.
This gives us a sufficient condition for $f\fatcompos g$ to be P-incrementally justified.

This condition is not necessary: Take $A=o$, $B=(o,o)$, $C=(o,o)$ and consider the two safe terms $M_f = \lambda x^A u^o.u$ and $M_g = \lambda y^B . y a$ for  some constant $a:o$. Then we have $M_{f\fatcompos g} = \lambda x . a$ which is safe hence $f\fatcompos g$ is P-incrementally justified although $\ord{A} < \ord{B}$.





\subsection{Two P-i.j. strategies whose composition is not P-i.j.}
\subsubsection{First attempt}

Take the types $A=o$, $B=(o,o)$, $C=o$, the variables
$x,u,v:o$, $y:B$ and $\varphi:((o,o),o)$ and $\Sigma$-constant $a:o$.
Consider the two safe terms $\vdash_s  M_f = \lambda xv.x : A\typear B$ and $\vdash_s M_g = \lambda y . \varphi (\lambda u . y a) : B\typear C$.
The $\eta\beta$-nf of $M_{f\fatcompos g}$ is $\vdash \lambda x . \varphi (\underline{\lambda u . x})$ which is unsafe because of the underlined term. It is then tempting to use
Theorem \ref{thm:safeincrejust}(ii) to conclude that
$\sem{M_{f\fatcompos g}}$ is not P-incrementally justified. However we cannot use it since $M_g$ contains an order $2$ constants ($\varphi$) and therefore
$M_{f\fatcompos g}$ is not a valid simply typed $\lambda$-term (nor a \pcf-term).

\subsubsection{Second attempt}
The previous example can be easily changed into a working counter-example: we just need to elevate $\varphi$ from the status of constant to variable.

Take $A=o$, $B=(o,o)$, $C=(((o,o),o),o)$, the variables
$x,u,v:o$, $y:B$ and $\varphi:((o,o),o)$ and the $\Sigma$-constant $a:o$. Consider the two safe terms $\vdash_s  M_f = \lambda xv.x : A\typear B$ and  $\vdash_s M_g = \lambda y \varphi. \varphi (\lambda u . y a) : B\typear C$.
The $\eta\beta$-nf of $M_{f\fatcompos g}$ is $\vdash \lambda x \varphi. \varphi (\underline{\lambda u . x})$ which is unsafe because of the underlined term, thus by Theorem \ref{thm:safeincrejust}(ii), $\sem{M_{f\fatcompos g}}=\sem{M_f} \fatcompos
\sem{M_g}$ is not P-incrementally justified. The following diagram illustrates a play that is not P-i.j.:
\begingroup
\def\sigcol#1{{\color{gray} #1}}
\def\mucol#1{{\color{red} #1}}
$$\begin{array}{ccccccccc}
A &  & \multicolumn{2}{c}{B} && \multicolumn{4}{c}{C}\\
\cline{1-1} \cline{3-4} \cline{6-9}
o & \stackrel{\sigcol{\sem{M_f}}}\longrightarrow & o, & o & \stackrel{\mucol{\sem{M_g}}}\longrightarrow & ((o, &o),& o),& o \\ \\
&&&&&&&&\rnode{n0}{\lambda x \varphi \omove  \mucol {\lambda y \varphi}}\\
&&&&&&&\rnode{n1}{\varphi  \pmove \mucol \varphi}\\
&&&&&&\rnode{n2}{\lambda u \omove  \mucol {\lambda u}} \\
&&&  \rnode{n3}{\omove \sigcol {\lambda x v} \pmove \mucol y} \\
\rnode{n4}{x \pmove \sigcol x}
\end{array}
\ncarc[arcangleA=20,arcangleB=20,linecolor=black]{->}{n4}{n0}
\ncarc[arcangleA=30,arcangleB=20,linecolor=red]{->}{n2}{n1}
\ncarc[arcangleA=30,arcangleB=20,linecolor=red]{->}{n1}{n0}
\ncarc[arcangleA=20,arcangleB=20,linecolor=red]{->}{n3}{n0}
\ncarc[arcangleA=20,arcangleB=20,linecolor=gray]{->}{n4}{n3}
$$
\endgroup

\subsubsection{Another counter-example where $\ord{B} = \ord{C}$}

Let $A=o$, $B=C=(((o,o),o),o)$ and let $x:A$, $y:B$, $u:o$, $v,\varphi:((o,o),o)$
and $g:(o,o)$ be variables and  $a:o$ be a $\Sigma$-constant. Take the two safe terms $\vdash  M_f = \lambda x v.x$ and $\vdash M_g = \lambda y \varphi. \varphi (\lambda u . y (\lambda g. a))$.
The $\eta\beta$-nf of $M_{f\fatcompos g}$ is $\vdash \lambda x \varphi. \varphi (\underline{\lambda u . x})$ which is unsafe because of the underlined term, so
$f\fatcompos g$ is not P-incrementally justified.
 
\bibliographystyle{plain}
\bibliography{../bib/higherorder,../bib/gamesem,../bib/lambdacalculus}


\section{P-i.j. and IA}
$var = acc \times exp = com^{\omega}\times exp$

Any strategy on the game $I \fngamear\ !var$ is P-i.j. since there is no P-question in the arena $var$.


\end{document}


\definecolor{darkGreen}{rgb}{0.03,0.35,0.05}

% game semantics
\newcommand{\sem}[1]{{[\![ #1 ]\!]}}

% reduction, substitution
\newcommand\betared{\rightarrow_\beta}
\newcommand\betaredtr{\twoheadrightarrow_\beta} % transitive closure of the beta reduction
\newcommand\betaredtrref{\rightarrow^*_\beta}

% pcf
\newcommand\pcf{\textsf{PCF}}
\newcommand\pcfcond{\texttt{cond}}
\newcommand\pcfsucc{\texttt{succ}}
\newcommand\pcfpred{\texttt{pred}}
\newcommand\pcfcase{\texttt{case}} % part of PCF'

% computation tree, eta normal form, traversals
\def\elnf#1{\lceil #1\rceil}
\def\etanf#1{\eta_{\sf nf}(#1)}
\def\betanf#1{\beta_{\sf nf}(#1)}
\def\etabetanf#1{\eta\beta_{\sf nf}(#1)}
\newcommand\travset{\mathcal{T}rav}


% justified sequence of moves
\newcommand{\oview}[1]{\llcorner #1 \lrcorner}
\newcommand{\pview}[1]{\ulcorner #1 \urcorner}
\newcommand{\extomove}{\textcolor{orange}}
\newcommand{\extpmove}{\textcolor{darkGreen}}

\newcommand{\ord}[1]{{\sf ord}(#1)}
\newcommand{\ordx}[2]{{\sf ord}_{#1}(#2)}

% highlight for definition names
\newcommand\defname[1]{{\bf\em #1}\index{#1}}

%\newcommand{\iff}{\Leftrightarrow}

% set theory
\newcommand{\makeset}[1]{\{\,{#1}\,\}}
\newcommand\inter{\cap}
\newcommand\union{\cup}
\newcommand\Union{\bigcup}
\newcommand\prefset{\textsf{Pref}}
\newcommand{\relimg}[1]{{(\!| #1 |\!)}}
\newcommand\nat{\mathbb{N}}


\newtheorem{theorem}{Theorem}[section]
\newtheorem{corollary}[theorem]{Corollary}

\newtheorem{lemma}{Lemma}[section]
\newtheorem{proposition}{Proposition}[section]

\theoremstyle{remark}
\newtheorem{remark}{Remark}[section]
\newtheorem{example}{Example}[section]
\newtheorem{property}{Property}[section]

\theoremstyle{definition}
\newtheorem{definition}{Definition}[section]
\newtheorem{algorithm}{Algorithm}[section]


\author{William Blum}
\title{Compositionality of P-incrementally justified strategies}

\begin{document}
\maketitle

\section{Well-bracketing and incremental-justification}

We consider an arena $A$ and make the following two assumptions on it:
\begin{itemize}
\item (A1) For $A \neq \bot$ (the arena with a single initial question), each question move in the arena enables at least one answer move.
\item (A2) Answer moves do not enable any other move.
\end{itemize}

We define the \defname{order} of a move $m$, written $\ord{m}$, as
the length of the path from $m$ to its furthest leaf in the arena minus 1
({\it i.e.} the height of the subarena rooted at $m$ minus 2.).
Because of assumptions (A1) and (A2),
for any move $m$ of $A \neq \bot$, $m$ is question move if and only if $\ord{m} \geq 0$, and $m$ is an answer move if and only if $\ord{m} = -1$.





We call \defname{pending question} of a sequence of moves $s \in L_A$ the last unanswered question in $s$.

\begin{definition}\rm
A strategy $\sigma$ is said to be \defname{P-well-bracketed} if for any play $s \, a \in \sigma$ where $a$ is a  P-answer, $a$ points to the pending question in $s$.
\end{definition}



P-well-bracketing can be restated differently as the following proposition shows:
\begin{proposition}
\label{prop:char_wellbrack}
\rm We make assumption (A1) and (A2).
Let $\sigma$ be a strategy on an arena $A\neq \bot$.
The following statements are equivalent:
\begin{enumerate}
\item[(i)] $\sigma$ is P-well-bracketed,
\item[(ii)] for $s \, a \in \sigma$ with $a$ a P-answer, $a$ points to the pending question in $\pview{s}$,
\item[(iii)] for $s \, a \in \sigma$ with $a$ a P-answer, $a$ points to the last O-question in $\pview{s}$.
\item[(iv)] for $s \, a \in \sigma$ with $a$ a P-answer, $a$ points to the last O-move in $\pview{s}$ with order $>\ord{a}$.
\end{enumerate}
\end{proposition}
\begin{proof}
$(i)\iff(ii)$: \cite[Lemma 2.1]{McC96b} states that if P is to move then the pending question in $s$ is the same as that of $\pview{s}$.

$(ii)\iff(iii)$: Assumption (A2) implies that the pending question in $\pview{s}$ is also the last O-question occurring in $\pview{s}$.

$(iii)\iff(iv)$: Because of assumption (A1) and (A2),
for any move $m$, we have $m$ is a question move
if and only if $\ord{m} \geq 0$ if and only if $\ord{m} > \ord{a} = -1$.
\end{proof}




\begin{lemma}
\label{lem:justfied_by_unanswered}
Under assumption (A2), if $s$ be a justified sequence of moves satisfying alternation and visibility then any O-move (resp. P-move) in $s$ points to an \emph{unanswered} P question (resp. O-question).
\end{lemma}
\begin{proof}
Suppose that an O-move $c$ points to a P-move $d$ that has already been answered by the O-move $a$. The sequence $s$ as the following form:
$$ s= \ldots \Pstr{(d){d}  \ldots  (a-d,20){a}  \ldots  (c-d,20){c}}$$

By O-visibility, $d$ must belong to $\oview{s_{<c}}$. But since $a$ is an answer, by assumption (A2), it cannot justify any P-move, therefore
$\oview{s_{<q}}$ must contain an OP-arc ``hoping'' over $a$. We name the nodes of this arc $d^1$ and $c^1$:
$$ s = \ldots \Pstr{(d){d}  \ldots  (d1){d^1} \ldots (a-d,20){a} \ldots
 (c1-d1,20){c^1} \ldots (c-d,25){c}}$$


By P-visibility, $d^1$ must belong to $\pview{s_{<c^1}}$.
Consequently, $a$ does not belong to $\pview{s_{<c^1}}$ (otherwise the PO-arc
$\Pstr[0.5cm]{(d){d} \quad (a-d,45){a}}$ would cause the P-view to jump over $d^1$).
Therefore there must be a PO-arc $\Pstr[0.5cm]{(d2){d^2} \quad (c2-d2,45){c^2}}$ in
$\pview{s_{<c^1}}$ hoping over $a$:
$$ s = \ldots \Pstr[0.7cm]{(d){d}  \ldots
(d1){d^1} \ldots (d2){c^2} \ldots
(a-d,20){a} \ldots
 (c2-d2,20){d^2} \ldots (c1-d1,20){c^1} \ldots (c-d,25){c}}$$

This process can be repeated infinitely often by using alternatively O-visibility and P-visibility. This gives a contradiction since the sequence of moves $s_{<c}$ has finite length.
Hence $d$ cannot point to a question that has already been answered. Since, by assumption (A2), a question is enabled by another question, $d$ is necessarily justified by an unanswered question.
\end{proof}


\begin{lemma}
\label{lem:oq_in_pview_unanswered}
Under assumption (A2), if $s$ is a P-well-bracketed justified sequence of moves of odd length satisfying alternation and visibility then  all O-questions occurring in $\pview{s}$ are unanswered in $s$.
\end{lemma}
\begin{proof}
We proof the first part by induction on $s$.
The base case ($s = q$ with $q$ initial O-move) is trivial.

Suppose $\Pstr[0.4cm]{ s = s' \cdot (n)n \cdot u \cdot (m-n,45){m} }$.
Let $r$ be an O-question in $\pview{s} = \pview{s'} \cdot n \cdot m$.
If $r$ is the last move $m$ then it is necessarily unanswered.
If $r \in \pview{s'}$ then by the induction hypothesis, $r$ is unanswered in $s'$.
Suppose that $r$ is answered in $s$. This implies that some answer move $a$ in $u$ points to $r$:
$$\pstr[0.5cm][5pt]{ s = \underbrace{\cdots\ \nd(r){r}^O \cdots }_{s'} \
\nd(n){n}^P \ \underbrace{\cdots\ \nd(a-r,35){a}^P \cdots }_{u} \
\nd(m-n,30){m}^O } \ .$$

Since $m$ points to $n$, by lemma \ref{lem:justfied_by_unanswered},
$n$ is still unanswered at $s_{\leq a}$. Therefore the pending
question at $s_{\leq a}$ cannot be $r$. But $a$ is justified by $r$,
therefore the well-bracketing condition is violated. Hence $r$ is
unanswered in $s$.
\end{proof}





\begin{definition}\rm
  A strategy $\sigma$ is said to be \defname{P-incrementally
    justified} if for any play $s \, q \in \sigma$ where $q$ is a
  P-question, $q$ points to the last unanswered O-question in $\pview{s}$ with
  order strictly greater than $\ord{q}$.
\end{definition}

\begin{proposition}
\label{prop:char_pincr}
\rm We make assumption (A1) and (A2).
Let $\sigma$ be a \emph{P-well-bracketed} strategy on an arena $A\neq \bot$.
The following statements are equivalent:
\begin{enumerate}
\item[(i)] $\sigma$ is P-incrementally-justified,
\item[(ii)] for $s \, q \in \sigma$ with $q$ a P-question, $q$ points to the last O-question in $\pview{s}$ with order $>\ord{q}$,
\item[(iii)] for $s \, q \in \sigma$ with $q$ a P-question, $q$ points to the last O-move in $\pview{s}$ with order $>\ord{q}$.
\end{enumerate}
\end{proposition}
\begin{proof}
$(i)\iff(ii)$: By lemma \ref{lem:oq_in_pview_unanswered}, O-question occurring in $\pview{s}$ are all unanswered.

$(ii)\iff(iii)$: Because of (A1) and (A2), $\ord{q} \geq 0$ thus an O-move with order $>\ord{q}$ is necessarily an O-question.
\end{proof}

Putting proposition \ref{prop:char_pincr} and
\ref{prop:char_wellbrack} together we obtain:
\begin{proposition}
Under assumption (A1) and (A2).
A strategy $\sigma$ on $A\neq \bot$
is \emph{P-well-bracketed} and
 \emph{P-incrementally-justified} if and only if
for $s \, m \in \sigma$, $m$ points to the last O-move in $\pview{s}$ with order $>\ord{m}$.
\end{proposition}

\section{Compositionality}


Let $X$ and $Y$ be two arenas.
For any move $m$ of the arena $X$ we write $m^{X\rightarrow Y}$ to denote the same move seen as move of the component $X$ of the arena $X\rightarrow Y$.
Similarly we use the notation $m^{Y\rightarrow X}$.
We write $\ordx{X}{m}$ to denote $\ord{m}$, the order of the move $m$ in the arena $X$, $\ordx{X\rightarrow Y}{m}$ to denote the order of $m^{X\rightarrow Y}$ (in the arena $X\rightarrow Y$) and $\ordx{Y\rightarrow X}{m}$ to denote the order of $m^{Y\rightarrow X}$.

\begin{lemma}
%Let $A \stackrel{\sigma}{\rightarrow} B$ and $B \stackrel{\mu}{\rightarrow} C$ be two strategies for some games $A$, $B$ and $C$.
Let $A$, $B$ and $C$ be three arenas. We have:
$$\begin{array}{lll}
\forall m \in A:
    &  \ord{m}_{A\rightarrow B} = \ord{m}_{A\rightarrow C} \ ,\\
\forall m \in B:
    & \ord{m}_{A\rightarrow B} \geq \ord{m}_{B\rightarrow C}  & \mbox{for $m$ initial,}\\
    & \ord{m}_{A\rightarrow B} = \ord{m}_{B\rightarrow C} & \mbox{for $m$ non initial,} \\
\forall m \in C:
    & \ord{m}_{A\rightarrow C} \geq \ord{m}_{B\rightarrow C} \iff
\ord{A} \geq \ord{B}\ & \mbox{for $m$ initial,}\\
    & \ord{m}_{A\rightarrow C} = \ord{m}_{B\rightarrow C}   & \mbox{for $m$ non initial.}
\end{array}
$$
\end{lemma}





\section{Homogeneity constraint}

Type homogeneity is not preserved after composition. Indeed the types  $o \longrightarrow (o \rightarrow o)$ and $(o \rightarrow o) \longrightarrow \left((o \rightarrow o) \rightarrow o \right)$ are homogeneous
but $o \longrightarrow \left((o \rightarrow o) \rightarrow o\right)$ is not.



\section{unfinished section}

We can show the following property by an easy induction :
\begin{lemma}
\label{lem:interaction_projection}
 Let $u$ be an interaction sequence in $Int(A,B,C)$ then
$$\pview{u} \upharpoonright A,C = \pview{u \upharpoonright A,C} \ .$$
\end{lemma}

The P-view of an interaction sequence $u \in Int(A,B,C)$ is defined as:
\begin{align*}
\pview{u\cdot \extomove{n}} &= \extomove{n} &
\mbox{ if \extomove{$m$} is an \extomove{external O-move} initial in C,}\\
\pview{\Pstr{u\cdot (m)m\cdot v \cdot (n-m,45){\extomove{n}} }} &= \extomove{n} &\mbox{ if \extomove{$m$} is an \extomove{external O-move} non initial in C,}\\
\pview{u \cdot \extpmove{m}} &= \pview{u}\cdot \extpmove{m}  & \mbox{ if \extpmove{$m$} is a \extpmove{generalized P-move}.}\\
\end{align*}


\section{Syntactic approach}

We say that a PCF term is \defname{semi-safe} if
it is of the form $N_0 N_1 \ldots N_k$ for $k\geq 1$ where each of the
$N_i$ is safe or if it can be written $\lambda \overline{x} . N$ for some
safe term $N$. Semi-safe terms are either safe or ``almost safe'' in the sense that they can be turned into an equivalent (i.e. preserving
the semantics) safe term  by performing $\eta$-expansions. Indeed, let $M$ be an semi-safe term that is unsafe.
If $M$ is of the first form $N_0 N_1 \ldots N_k : (A_1,\ldots,A_n)$ with $k\geq 1$ then let $\varphi_i:A_i$ for $i\in\{1..n\}$ be fresh variables , using the (app) and (abs) rules we can build the safe term $\lambda \varphi_1 \ldots \varphi_n . N_0 N_1 \ldots N_k \varphi_1 \ldots \varphi_n$. If $M$ is of the second form $\lambda \overline{x} . N$ then using the abstraction rule we can build the equivalent safe term $\lambda \overline{y} \overline{x}. N$  where $\overline{y} = fv(\lambda \overline{x}. N)$.

The $\beta$-normal form of a \pcf\ term is the possibly infinite term obtained by reducing all the redexes in $M$.

%We say that a \pcf\ term is in $\beta$-normal form it does not contain
%any $\beta$-redex but may contain other redexes (in particular it may contain
%an occurrence of the Y combinator, which would produce a $\beta$-redex if it was reduced). The $\beta$-normal form of a \pcf\ term can be obtained by
%computing the $\beta$-normal form of the term treated as a simply typed term  where Y, \pcfcond, \pcfsucc and \pcfpred\ are treated as ordinary constants.


The correspondence between safety and P-incremental-justification for the simply typed lambda calculus was shown
in \cite{blumong:safelambdacalculus}, Theorem 3(ii):

\begin{theorem}[Safety and P-incremental justification]
\label{thm:safeincrejust} In the simply typed lambda calculus:
\begin{enumerate}[(i)]
\item If $M$ is safe then $\sem{M}$ is P-incrementally justified.
\item If $M$ is a closed term and $\sem{M}$ is
  P-incrementally justified then the $\eta$-long form of the
  $\beta$-normal form of $M$ is safe.
\end{enumerate}
\end{theorem}

In the context of \pcf\, only the first part of the theorem holds (see \cite{blumtransfer} for the proof). However (ii) does not hold. Indeed, take the closed \pcf\ term $M = \lambda f x y. f (\lambda z. \pcfcond (\pcfsucc\ x) y z )$ where $x,y,z:o$ and $f:((o,o),o)$. $M$ is in normal form (the conditional  could  only be reduced if $x$ were first evaluated). The $\eta$-long form of the $\beta$-normal form of $M$ is therefore $M$ itself which is unsafe.
But clearly we have $\sem{M} = \sem{\lambda f x y. f (\lambda z. z)}$, and since  $\lambda f x y. f (\lambda z. z)$ is safe, by (i), $\sem{M}$ is P-incrementally justified.

Such counter-example arise because the conditional operator
of \pcf\ permits us to build terms in normal form containing ``dead code'' {\it i.e.} some  subterm that will never be evaluated for any value of M's parameters. In the example given above, the dead code consists in the subterm $y$. In general, if the dead code part of the computation tree contains a variable that is not incrementally bound then the resulting term will be unsafe even if the rest of the tree is incrementally bound.
In the example above, it was possible to turn $M$ into the equivalent safe term $\lambda f x y. f (\lambda z. z)$ by eliminating the dead code from $M$.
We shall see how to generalise this to any \pcf\ term with a P-incrementally justified denotation.

Dead code elimination can be difficult to achieve in practice but the formal definition is not difficult to formulate. We say that a subterm $N$ occurring
in a context $C[-]$ in $M : (A_1, \ldots, A_n,o)$ is
part of the \defname{dead code} of $M$ if for any term $T_0$ of the form $M M_1 \ldots M_n$,
any reduction sequence starting from $T_0$ does not involve a reduction of the subterm $N$ {\it i.e.} for any reduction sequence $T_0 \rightarrow T_1 \rightarrow \ldots \rightarrow T_k$, there is no $j\in \{0.. k-1\}$ such that $T_j = C[N]$ and $T_{j+1} = C[N']$ for some term $N'$.


Let $M$  be a \pcf\ term in $\eta$-nf.
An occurrence of a variable $x$ in $M$ is said to be a \defname{dead occurrence}
if it occurs in the dead code of $M$. In other words, it is a
dead occurrence of $x$ if the corresponding node in the computation tree does not appear in any traversal of $\travset(M)$. Equivalently, thanks to the Correspondence Theorem, an occurrence of $x:B$ is dead if and only if the initial move
of the arena $\sem{B}$ does not appear in any play of $\sem{M}$.


We define $M^*$ as the term obtained from $M$ after substituting
all subterms of the form  $x N_1 \dots N_k$ for some
dead variable occurrence $x:(B_1,\ldots, B_k, o)$ by the constant $0$. This process is called \defname{dead variable elimination}.
Note that if $M$ is in $\eta\beta$-nf then so is $M^*$.

We also write $\tau(M)^*$ to denote the equivalent transformation
on the computation tree. Since the computation tree is constructed from the $\eta$-nf of $M$, we will use this notation even
when $M$ is not in $\eta$-nf.



\begin{proposition}[Incremental-binding and P-incremental-justification coincide] \
\label{prop:incrbound_imp_incrjustified_pcf} Let $M$ be a \pcf\ term in $\beta$-normal form.
\begin{enumerate}[(i)]
\item  If $\tau(M)$ is incrementally-bound then $\sem{M}$ is P-incrementally-justified,
\item  if $\sem{M}$ is P-incrementally-justified
then $\tau(M)^*$ is incrementally-bound.
\end{enumerate}
\end{proposition}
\begin{proof}
(i) The proof is exactly the same as in the simply typed lambda calculus case,
see \cite[Proposition 4.1.5(i)]{blumtransfer}.

\noindent (ii)
Take $M$ a \pcf\ term in $\beta$-normal form denoted by
$\sem{M}$ P-incrementally-justified. Let $r$ denote the root of $\tau(M)^*$.
Let $n$ be a node of $\tau(M)^*$ labelled by the variable $x$.
$\tau(M)^*$ is free from dead code therefore $n$ is not a dead occurrence of $x$ and there exists a traversal of $\tau(M)^*$ of the form $t \cdot x$.

\pcf\ constants are of order $1$ at most therefore they cannot hereditarily justify a variable node, thus $x$ is necessarily hereditarily justified by the root $r$ of the computation tree.


By considering $t\cdot x$ as a traversal of $\tau(M)$,  the correspondence theorem gives $\varphi((t \cdot x) \upharpoonright r) = \varphi((t \upharpoonright r) \cdot x) \in \sem{M}$. Since $\sem{M}$ is P-incrementally-justified, $\varphi(x)$
must point to the last O-move in $\pview{?(\varphi(t \upharpoonright
r))}$ with order strictly greater than $\ord{\varphi(x)}$.
Consequently $x$ points to the last node in $\pview{?(t
\upharpoonright r)} \inter N^{\lambda}$ with order strictly greater than $\ord{x}$. We have:
\begin{align*}
\pview{?(t \upharpoonright r)} &= \pview{?(t) \upharpoonright r} = \pview{?(t)} \upharpoonright r & (\mbox{by \cite[lemma 3.1.23]{blumtransfer}}) \\
& = [r,x[ \ \upharpoonright r & (\mbox{by \cite[proposition 3.1.20]{blumtransfer}})
\end{align*}
Since $M$ is in $\beta$-nf, the set of nodes
not hereditarily justified by $r$ is exactly the set of nodes hereditarily justified by $N_{\Sigma}$ thus
$[r,x[ \ \upharpoonright r = [r,x[\ \setminus\  N^{\upharpoonright \Sigma}$.
Moreover \pcf\ constants are of order $1$ at most therefore
$N^{\upharpoonright \Sigma} = N_{\Sigma} \union N^c_{\Sigma}$
where $N^c_{\Sigma}$ is the set of children nodes of $N_{\Sigma}$.
Thus $(\pview{?(t \upharpoonright r)}\upharpoonright r) \inter N^{\lambda} =
([r,x[\ \setminus\  N_{\Sigma} \setminus N^c_{\Sigma} ) \inter N^{\lambda} =
([r,x[\ \setminus\  N^c_{\Sigma} )  \inter N^{\lambda}$, and
since $N^c_{\Sigma}$ is constituted of order $0$ lambda-nodes only we have that
$x$ points to the last node in $[r,x[ \inter N^{\lambda}$ with order strictly greater than $\ord{x}$.

Hence if $x$ is a bound variable node then it is bound by the
last $\lambda$-node in $[r,x[$ with order strictly greater than
$\ord{x}$ and if $x$ is a free variable then it points to $r$ and
therefore all the $\lambda$-node in $]r,x[$ have order smaller than
$\ord{x}$. Thus $\tau(M)^*$ is incrementally-bound.

\end{proof}

\begin{theorem}[Safety and P-incremental justification]
\label{thm:safeincrejust_pcf} In \pcf:
\begin{enumerate}[(i)]
\item If $M$ is safe then $\sem{M}$ is P-incrementally justified;
\item if $M$ is a closed term and $\sem{M}$ is
  P-incrementally justified then $\etanf{\betanf{M}}^*$ is safe.
\end{enumerate}
\end{theorem}
\begin{proof}
\noindent(i)
This is proved in  \cite[Theorem 4.2.10]{blumtransfer}.

\noindent(ii) Suppose $M$ is a closed \pcf\ term with a P-incrementally justified strategy denotation. By Proposition \ref{prop:incrbound_imp_incrjustified_pcf}(ii), $\tau(\betanf{M})^* = \tau(\etanf{\betanf{M}}^*)$ is incrementally-bound.

Lemma 4.1.6(ii) from \cite{blumtransfer} states that in the simply typed $\lambda$ calculus, if $M$ is closed and $\tau(M)$ is incrementally-bound then the $\etanf{M}$ is safe.
This lemma remains valid for infinite closed \pcf\ terms in normal form.

Thus $\etanf{\etanf{\betanf{M}}^*} = \etanf{\betanf{M}}^*$ is safe.
\end{proof}


We write \pcf' to denote the language obtained by extending \pcf\
with the $\pcfcase_k$ construct (see \cite{Abr02}).
The $\pcfcase_k$ construct is the obvious generalisation of the
conditional operator \pcfcond\ to $k$ branches instead of $2$. All the results obtained so far concerning Safe \pcf\ (including those
cited from \cite{blumtransfer}) can clearly be transposed to \pcf'.

The previous theorem leads to the following definability result for safe \pcf':
\begin{proposition}[Definability for safe \pcf\' terms]
\label{prop:safetydefinability}
Let $\sigma$ be a well-bracketed innocent
P-incrementally-justified strategy with finite view function defined on the game $A$. There exists a \emph{safe} closed PCF' term $\vdash_s M : A$ in $\eta$-long normal form such that:
$$ \sem{M_\sigma} = \sigma $$
\end{proposition}
\begin{proof}
By the standard definability result for PCF', there is a closed term $\vdash N : A$ such that $\sem{N} = \sigma$.
Take $M_\sigma = \etanf{\betanf{N}}^*$.
Clearly $\sem{ M_\sigma} = \sem{N} = \sigma$, and by Theorem \ref{thm:safeincrejust_pcf}(ii), $M_\sigma$ is safe.
\end{proof}


\subsection{Beginning of a proof of compositionality}

Let $f:A\rightarrow B$ and $g:B\rightarrow C$ be two innocent well-bracketed and P-incrementally-justified strategies with finite view function.
We would like to prove that $f;g$ is also P-incrementally-justified.

By the definability result, there are two closed terms (in $\eta$-nf) $\vdash M_f :A\rightarrow B$  and $\vdash M_g :A\rightarrow B$ such that $\sem{M_f} = f$
and $\sem{M_f} = g$.

Suppose $A=I$, where $I$ denotes the empty arena, then $f;g = \sem{M_g M_f}$ and
since $M_g$ and $M_f$ are safe, $M_g M_f$ is semi-safe therefore $f;g$ is P-incrementally-justified.
If $B=I$ then $f;g = g$ which is P-incrementally-justified.
If $C=I$ then $f;g = \{ \epsilon \}$.

We now assume that $A$,$B$ and $C$ are not the empty arena.

We have $B=(B_1,\ldots,B_l,o)$ and $C=(C_1,\ldots,C_k,o)$ for some $l,k\geq 0$.
The term $M_f$ and $M_g$ being in $\eta$-nf are of the following forms:
\begin{eqnarray*}
\vdash M_f &=& \lambda x^A \varphi_1^{B_1} \ldots \varphi_l^{B_l} . N_f^o\\
\vdash  M_g &=& \lambda y^B \phi_1^{C_1} \ldots \phi_k^{C_k} . N_g^o
\end{eqnarray*}
for some distinct variables $x,y,\varphi_1, \dots \varphi_l, \phi_1, \dots \phi_k$ and
terms $N_f$ and $N_g$ in $\eta$-nf:
\begin{eqnarray*}
x:A, \varphi_1:B_1, \dots, \varphi_l:B_l &\vdash& N_f :o \\
y:B, \phi_1:C_1, \dots, \phi_l:C_l &\vdash& N_g :o
\end{eqnarray*}

Let us define the term $M_{f;g} = \lambda x . M_g (M_f x)$. Clearly we have $\sem{M_{f;g}} = \sem{M_f} ; \sem{M_g} = f;g$.

Unfortunately, the term $M_{f;g}$ is not necessarily safe even if $M_f$ and $M_g$ are. Take $M_f = \lambda x^o z^o.x$ and
$M_g = \lambda y^{(o,o)} . y a$ for some constant $a\in \Sigma$.
Then $\lambda x:A . M_g (M_f x) = \lambda x . (\lambda y . y a) ( \underline{(\lambda x z.x) x} )$ which is unsafe because of the underlined subterm. Therefore we cannot conclude at this point that $f;g$ is P-incrementally-bound.

However we have:
\begin{align*}
f;g &= \sem{\lambda x . M_g (M_f x)} \\
 &= \sem{\lambda x . (\lambda \phi_1\ldots \phi_k . N_g) [(M_f x) / y]} \\
&= \sem{\lambda x \phi_1 \dots \phi_k. N_g [(M_f x) / y]}
& \mbox{($x\neq\phi_i$ for all $i$)}.
\end{align*}

\begin{lemma}
If $y:B, \Sigma \vdash N : T$ is a safe term in $\eta$-nf
and $\vdash M : A \rightarrow B$ is a safe term with $\ord{A} \geq \ord{B}$
then $x:A, \Sigma \vdash N[(M x)/y] :T$ is also safe.
\end{lemma}
\begin{proof}
By induction on the $\eta$-normal structure of $N$.

Suppose $N= y N_1 \dots N_l :o$ where  for $1\leq i \leq l$, $y:B, \Sigma \vdash N_i$ is safe.
By the induction hypothesis, $x:A, \Sigma \vdash N_i[(M x)/y]$ is safe for $1\leq i \leq l$ and since $M$ is safe, using the (app) rule we have that
$M x N_1[(M x)/y] \dots N_l[(M x)/y] = N[(M x)/y]$ is safe.

The case $N= z N_1 \dots N_l :o$ for $z\neq y$ is treated identically.

Suppose $N =\lambda \overline{\xi} . S$ with $S:o$. We suppose that
$y\notin  \overline{\xi}$ and $y\in fv(S)$ (otherwise no substitution happens
and the proof becomes trivial).
By the induction hypothesis
$x:A,\Sigma, \overline{\xi} \vdash S [(M x)/y]$ is safe.
By safety of $N$, since $y$ occurs free in $S$, we have
$\ord{C} \leq \ord{y}$, moreover by assumption, $\ord{y} = \ord{B} \leq \ord{A} = \ord{x}$ thus $\ord{C} \leq \ord{x}$ which permits us to use the (abs) rule
to form the safe term $x:A,\Sigma \vdash \lambda \overline{\xi} . S [(M x)/y]$.
\end{proof}

Hence by the previous lemma, provided that $\ord{A}\geq\ord{B}$, the term  $\lambda x \phi_1 \dots \phi_k. N_g [(M_f x) / y]$ is safe and therefore $f;g = \sem{\vdash \lambda x \phi_1 \dots \phi_k. N_g [(M_f x) / y]}$ is P-incrementally justified.


\subsection{Counter-example}
\subsubsection{First attempt:}

Take $A=o$, $B=(o,o)$, $C=o$. Let $x,u,v:o$, $y:B$ be variables and $\varphi:((o,o),o)$ and $a:o$ be $\Sigma$-constants.

Remark that we use of a constant $\varphi$ of order greater than $2$ therefore we are not working in the context of the simply typed lambda calculus, nor \pcf.

Take the two safe terms $\vdash  M_f = \lambda xv.x$ and $\vdash M_g = \lambda y . \varphi (\lambda u . y a)$.
The $\eta\beta$-nf of $M_{f;g}$ is
$\vdash \lambda x . \varphi (\underline{\lambda u . x})$ which is unsafe
because of the underlined term. It is then tempting to use
Theorem \ref{thm:safeincrejust}(ii) (or Theorem \ref{thm:safeincrejust_pcf}(ii) in the case of \pcf) to conclude that
$f;g$ is not P-incrementally justified. However as we remarked before,
the term we are dealing with are neither part of the
simply typed $\lambda$-calculus nor \pcf\ because of the presence of the
order $2$ constants $\varphi$.

\subsubsection{Second attempt:}
The previous example can be easily changed into a working counter-example: we just need to elevate $\varphi$ from the status of constant to variable.

We then have $A=o$, $B=(o,o)$, $C=(((o,o),o),o)$, variables
$x,u,v:o$, $y:B$
and $\varphi:((o,o),o)$ and the $\Sigma$-constant $a:o$.

Take the two safe terms $\vdash  M_f = \lambda xv.x$ and
 $\vdash M_g = \lambda y \varphi. \varphi (\lambda u . y a)$.
The $\eta\beta$-nf of $M_{f;g}$ is
$\vdash \lambda x \varphi. \varphi (\underline{\lambda u . x})$ which is unsafe because of the underlined term. By
Theorem \ref{thm:safeincrejust}(ii) we conclude that
$f;g$ is not P-incrementally justified.

\subsection{Another counter-example where $\ord{B} = \ord{C}$}

Let $A=o$, $B=C=(((o,o),o),o)$ and
let $x:A$, $y:B$, $u:o$, $v,\varphi:((o,o),o)$
and $g:(o,o)$ be variables
and  $a:o$ be a $\Sigma$-constant. Take the two safe terms $\vdash  M_f = \lambda x v.x$ and $\vdash M_g = \lambda y \varphi. \varphi (\lambda u . y (\lambda g. a))$.
The $\eta\beta$-nf of $M_{f;g}$ is
$\vdash \lambda x \varphi. \varphi (\underline{\lambda u . x})$
which is unsafe because of the underlined term, so
$f;g$ is not P-incrementally justified.


\bibliographystyle{plain}
\bibliography{../bib/higherorder,../bib/gamesem,../bib/lambdacalculus}


\end{document}
