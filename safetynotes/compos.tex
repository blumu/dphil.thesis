\documentclass{article}
\usepackage{pstricks}  
\usepackage{pstring}

\usepackage{amssymb}
\usepackage{amsmath}
%\usepackage{qsymbols}

\definecolor{darkGreen}{rgb}{0.03,0.35,0.05}

% justified sequence of moves
\newcommand{\oview}[1]{\llcorner #1 \lrcorner}
\newcommand{\pview}[1]{\ulcorner #1 \urcorner}
\newcommand{\extomove}{\textcolor{orange}}
\newcommand{\extpmove}{\textcolor{darkGreen}}

\title{Compositionality of P-incrementally justified strategies}

\begin{document}

The P-view of an interaction sequence $u \in Int(A,B,C)$ is defined as:
\begin{align*}
\pview{u\cdot \extomove{n}} &= \extomove{n} &
\mbox{ if \extomove{$m$} is an \extomove{external O-move} initial in C}\\
\pview{\Pstr{u\cdot (m)m\cdot v \cdot (n-m,45){\extomove{n}} }} &= \extomove{n} &\mbox{ if \extomove{$m$} is an \extomove{external O-move} non initial in C}\\
\pview{u \cdot \extpmove{m}} &= \pview{u}\cdot \extpmove{m} & \mbox{ if \extpmove{$m$} is a \extpmove{generalized P-move}}\\
\end{align*}

We can show the following property by an easy induction :
$$ \pview{u} \upharpoonright A,C = \pview{u \upharpoonright A,C}$$

\end{document}
