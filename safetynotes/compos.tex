\documentclass{article}
\usepackage{pstricks}  
\usepackage{pst-node}

\usepackage{amssymb}
\usepackage{amsmath}
\usepackage{qsymbols}

\newcommand{\oview}[1]{\llcorner #1 \lrcorner}
\newcommand{\pview}[1]{\ulcorner #1 \urcorner}

\edef\TheAtCode{\the\catcode`\@}
\catcode`\@=11
\def\ncArc{\pst@object{ncArc}}
\def\ncArc@i{\check@arrow{\ncArc@ii}}
\def\ncArc@ii#1#2{\nc@object{Open}{#1}{#2}{.5}{%
%yB yA sub xB xA sub \tx@Atan
  180 
\psk@arcangleA\space sub /AngleA ED
\psk@arcangleB\space %sub 180 add
/AngleB ED
\psk@ncurvB\space \psk@ncurvA\space
\tx@NCCurve}}

% links using psttricks
\newcommand{\link}{\@ifstar
                     \linkStar%
                     \linkNoStar%
}
\newcommand{\linkNoStar}[2][nodesep=0pt]{\ncarc[linewidth=0.4pt,offset=-2pt,nodesep=0pt,arcangleA=-#2, arcangleB=-#2,#1]{->}}
% the starred version uses ncArc instead of ncarc.
\newcommand{\linkStar}[2][nodesep=0pt]{\ncArc[linewidth=0.4pt,offset=-2pt,nodesep=0pt,arcangleA=#2, arcangleB=#2,#1]{->}}

\newcommand{\lnklabel}{\@ifstar
                     \lnklabelStar%
                     \lnklabelNoStar%
}
\newcommand{\lnklabelStar}[1]{\mput*{\mbox{{\tiny $#1$}}}}
\newcommand{\lnklabelNoStar}[1]{\Bput[1pt]{\mbox{{\tiny $#1$}}}}
\newcommand{\arclabel}[1]{\mput*{\mbox{{\small $#1$}}}}
\catcode`\@=\TheAtCode\relax

\psset{arrowlength=1,arrowinset=.4}


\begin{document}

$$
 \pview{ s \rnode{m}{m} \cdot \ldots \cdot \rnode{lmd}{\lambda \overline{\xi}}} = \pview{s} \cdot \rnode{m2}{m} \cdot \rnode{lmd2}{\lambda \overline{\xi}}   \link[nodesep=0pt]{30}{lmd}{m}    \link[nodesep=0pt]{35}{lmd2}{m2}
%  \psbezier[linewidth=0.8mm,linecolor=red,showpoints=true]{|->}%
%           {m}{m}(2,2)(0,0)
$$


\begin{itemize}
  \item  \raisebox{0cm}[0.7cm]{$
t = \rnode{n}{\lambda f z} \
\rnode{n2}{@} \
\rnode{n3}{\lambda g x} \
\rnode{n4}{f^{[1]}} \
\rnode{n5}{\lambda^{[2]}} \
\rnode{n6}{x} \
\rnode{n7}{\lambda^{[3]}} \
\rnode{n8}{f^{[4]}} \
\rnode{n9}{\lambda^{[5]}} \
\rnode{n10}{z}
\psset{ncurv=0.4}
\link*{60}{n3}{n2}
\link*{50}{n4}{n}
\link*{45}{n5}{n4}
\link*{55}{n6}{n3}
\link*{35}{n7}{n2}
\link*{35}{n8}{n}
\link*{45}{n9}{n8}
\link*{40}{n10}{n}$}

\item \raisebox{0cm}[0.8cm]{$
t\upharpoonright r = \rnode{n}{\lambda f z} \
\rnode{n4}{f}^{[1]} \
\rnode{n5}{\lambda}^{[2]} \
\rnode{n8}{f}^{[4]} \
\rnode{n9}{\lambda}^{[5]} \
\rnode{n10}{z}
\psset{ncurv=0.4}
\link*{50}{n4}{n}
\link*{60}{n5}{n4}
\link*{50}{n8}{n}
\link*{60}{n9}{n8}
\link*{50}{n10}{n}$}
\item \raisebox{0cm}[0.8cm]{$
\psi_M(t\upharpoonright r) = \rnode{n}{q^0}\ \rnode{n4}{q^1}\ \rnode{n5}{q^2}\ \rnode{n8}{q^1}\ \rnode{n9}{q^2}\ \rnode{n10}{q^3}
\psset{ncurv=0.4}
\link*{50}{n4}{n}
\link*{50}{n5}{n4}
\link*{50}{n8}{n}
\link*{60}{n9}{n8}
\link*{50}{n10}{n}\ .$}
\end{itemize}

\end{document}
