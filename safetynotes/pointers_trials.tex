\documentclass{article}
\usepackage{pstricks}  
\usepackage{pst-node}
\usepackage[all]{xy} 
\xyoption{tips}
\usepackage{amssymb}
\usepackage{amsmath}
\usepackage{qsymbols}

\newcommand{\oview}[1]{\llcorner #1 \lrcorner}
\newcommand{\pview}[1]{\ulcorner #1 \urcorner}

\edef\TheAtCode{\the\catcode`\@}
\catcode`\@=11
\def\ncArc{\pst@object{ncArc}}
\def\ncArc@i{\check@arrow{\ncArc@ii}}
\def\ncArc@ii#1#2{\nc@object{Open}{#1}{#2}{.5}{%
%yB yA sub xB xA sub \tx@Atan
  180 
\psk@arcangleA\space sub /AngleA ED
\psk@arcangleB\space %sub 180 add
/AngleB ED
\psk@ncurvB\space \psk@ncurvA\space
\tx@NCCurve}}

% links using psttricks
\newcommand{\link}{\@ifstar
                     \linkStar%
                     \linkNoStar%
}
\newcommand{\linkNoStar}[2][nodesep=0pt]{\ncarc[linewidth=0.4pt,offset=-2pt,nodesep=0pt,arcangleA=-#2, arcangleB=-#2,#1]{->}}
% the starred version uses ncArc instead of ncarc.
\newcommand{\linkStar}[2][nodesep=0pt]{\ncArc[linewidth=0.4pt,offset=-2pt,nodesep=0pt,arcangleA=#2, arcangleB=#2,#1]{->}}

\newcommand{\lnklabel}{\@ifstar
                     \lnklabelStar%
                     \lnklabelNoStar%
}
\newcommand{\lnklabelStar}[1]{\mput*{\mbox{{\tiny $#1$}}}}
\newcommand{\lnklabelNoStar}[1]{\Bput[1pt]{\mbox{{\tiny $#1$}}}}
\newcommand{\arclabel}[1]{\mput*{\mbox{{\small $#1$}}}}
\catcode`\@=\TheAtCode\relax

\psset{arrowlength=1,arrowinset=.4}


\begin{document}

$$
 \pview{ s \rnode{m}{m} \cdot \ldots \cdot \rnode{lmd}{\lambda \overline{\xi}}} = \pview{s} \cdot \rnode{m2}{m} \cdot \rnode{lmd2}{\lambda \overline{\xi}}   \link[nodesep=0pt]{30}{lmd}{m}    \link[nodesep=0pt]{35}{lmd2}{m2}
%  \psbezier[linewidth=0.8mm,linecolor=red,showpoints=true]{|->}%
%           {m}{m}(2,2)(0,0)
$$


\begin{itemize}
  \item  \raisebox{0cm}[0.7cm]{$
t = \rnode{n}{\lambda f z} \
\rnode{n2}{@} \
\rnode{n3}{\lambda g x} \
\rnode{n4}{f^{[1]}} \
\rnode{n5}{\lambda^{[2]}} \
\rnode{n6}{x} \
\rnode{n7}{\lambda^{[3]}} \
\rnode{n8}{f^{[4]}} \
\rnode{n9}{\lambda^{[5]}} \
\rnode{n10}{z}
%\psset{ncurv=0.4}
\link*{60}{n3}{n2}
\link*{50}{n4}{n}
\link*{45}{n5}{n4}
\link*{55}{n6}{n3}
\link*{35}{n7}{n2}
\link*{35}{n8}{n}
\link*{45}{n9}{n8}
\link*{35}{n10}{n}$}

\item \raisebox{0cm}[0.8cm]{$
t\upharpoonright r = \rnode{n}{\lambda f z} \
\rnode{n4}{f}^{[1]} \
\rnode{n5}{\lambda}^{[2]} \
\rnode{n8}{f}^{[4]} \
\rnode{n9}{\lambda}^{[5]} \
\rnode{n10}{z}
\link*{50}{n4}{n}
\link*{60}{n5}{n4}
\link*{35}{n8}{n}
\link*{60}{n9}{n8}
\link*{33}{n10}{n}$}
\item \raisebox{0cm}[0.8cm]{$
\psi_M(t\upharpoonright r) = \rnode{n}{q^0}\ \rnode{n4}{q^1}\ \rnode{n5}{q^2}\ \rnode{n8}{q^1}\ \rnode{n9}{q^2}\ \rnode{n10}{q^3}
\link*{60}{n4}{n}
\link*{60}{n5}{n4}
\link*{50}{n8}{n}
\link*{60}{n9}{n8}
\link*{40}{n10}{n}\ .$}
\end{itemize}

%\SelectTips{eu}{10}
%\UseTips
$\psi_M(t\upharpoonright r) =
\xymatrix@1@=0pt @C=1pt @R=0pt {
q^0 & q^1 \ar@/_2pt/+U-(1,0);[l]+U+(1,0) & 
q^2 \ar@/_2pt/+U-(1,0);[l]+U+(1,0) &
 q^1 \ar@/_7pt/+U-(1,0);[lll]+U+(1,0) &
 q^2 \ar@/_2pt/+U-(1,0);[l]+U+(1,0)& 
q^3 \ar@/_7pt/+U-(1,0);[lllll]+U+(1,0) \\
}$


\xymatrix{ \bullet
\ar[r]^x
\ar@/^3pc/[rr]^{\quad}_{}="0"
& \bullet
\ar[r]^{\bar{x}}
\ar@/_3pc/[rr]^{\quad}_{}="2"
\ar@{=>}"0"; {} ^{i_x}
& \bullet
\ar[r]^x
\ar@{=>}{}; "2" ^{e_x}
& \bullet } \qquad = \qquad
 \xymatrix{
\bullet
\ar@/^2pc/[rr]_{\quad}^{x}="1"
\ar@/_2pc/[rr]_{x}="2"
&& \bullet
\ar@{}"1";"2"|(.2){\,}="7"
\ar@{}"1";"2"|(.8){\,}="8"
\ar@{=>}"7" ;"8"^{1_x} } \quad 

\xy \ldots !R-l dd
\endxy

\xy %FIG.19. Natural transformation between functors.
(0,0)*+{A d {\frac{\frac{\frac{\frac{A}{V}}{V}}{V}}{V}} }="a";
 (10,0)*+{B}="b";
%(-16,-5)*+{\bullet}="c";
{\ar@/^.25pc/ "a"+U;"b"+U};
%{\ar@/_.25pc/"a";"c"};
%{\ar@/_.15pc/ "b";"c"}; 
\endxy


\edef\TheAtCode{\the\catcode`\@}
\edef\OpenSqBrCode{\the\catcode`\[}
\edef\CloseSqBrCode{\the\catcode`\]}

\newtoks\lastsrc
\newtoks\lastdst
\newtoks\content
\newcount\lastangle
\lastsrc={@}
\lastdst={@}
\content={@}
\begingroup
\catcode`\@=11
\catcode`&=\active
%\catcode`\[=\active
%\catcode`\]=\active
%\catcode`-=\active

\gdef\ptrstr{%    note the global \gdef
\begingroup
%  \catcode`\[=\active
%  \catcode`\]=\active
  \catcode`&=\active
 % \catcode`-=\active
  \def&{\newlink}%
%  \def-{\newlink}%
%  \def[{\newlink}%
%  \def]{B}%
\newlink
}

\gdef\newlink#1[#2-#3,#4]{ 
\creatependinglink
%\lastangle = #4
%\lastsrc={#2}
%\lastdst={#3}
\rnode{#2}{#1} 
\edef\next{#4}
\ifx\next\@empty
\else 
{ \link*{#4}{#2}{#3} }
\fi
}

\gdef\creatependinglink{
%\ifx\lastsrc @
% \number\lastangle  
%\link*{\number\lastangle}{\the\lastsrc }{\the\lastdst} 
}

\gdef\endptrstr{%    note the global \gdef
\creatependinglink
\endgroup
%  \catcode`\[ = \OpenSqBrCode
%  \catcode`\] = \CloseSqBrCode
  \catcode`\&=4
%  \catcode`-=7
}

\endgroup


%\def\toto#1{\rnode{#1}\hbox\bgroup}
%\def\tutu{{\egroup}}
%
%\toto{n0}3truc  asd \tutu

$\psi = \ptrstr 
q^0 [n-,] \ & q^1 [n4-n,60]\ & q^2 [n5-n4,60]\  & q^1 [n8-n,50]\  & q^2[n9-n8,60]\  &  q^3 [n10-n,40]\ .
\endptrstr$

%\def\ncArc@i{\check@arrow{\ncArc@ii}}
%\def\ncArc@ii#1#2{\nc@object{Open}{#1}{#2}{.5}{%
%yB yA sub xB xA sub \tx@Atan
%  180 
%\psk@arcangleA\space sub /AngleA ED
%\psk@arcangleB\space %sub 180 add
%/AngleB ED
%\psk@ncurvB\space \psk@ncurvA\space
%\tx@NCCurve}}

\catcode`\@=\TheAtCode\relax




%%%%%desired syntax:
%\ptrstr
%\psi_M(t\upharpoonright r) = 
%& [n] q^0 & [n4->n] q^1 & [n5->n4] q^2 & [n8->n] q^1 & [n9->n8] q^2 & [n10->n] q^3.$
%\endptrstr
%
% or 
%\ptrstr
%\psi_M(t\upharpoonright r) = 
%& [n] q^0 & [n4->n]{30} q^1 & [n5->n4]{30} q^2 & [n8->n]{30} q^1 & [n9->n8]{30} q^2 & [n10->n]{30} q^3.$
%\endptrstr

\end{document}
