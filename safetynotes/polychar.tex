\documentclass{article}
\usepackage{a4wide}
\usepackage{amsmath, amsthm, amssymb}

\newtheorem{theorem}{Theorem}[section]
\newtheorem{corollary}[theorem]{Corollary}

\newcommand{\encode}[1]{\ulcorner #1 \urcorner}
\newcommand{\nat}{\mathbb{N}}
\newcommand{\union}{\cup}
\newcommand{\Union}{\bigcup}

\author{William Blum}
\title{The Numerical Functions representable in the Safe Simply Typed Lambda Calculus are exactly the polynomials}

\begin{document}
\maketitle

In \cite{citeulike:622637}, Schwichtenberg showed that 
numerical functions representable in the Simply Typed Lambda Calculus are exactly the multivariate polynomials \emph{extended with the conditional function}:
\begin{theorem}
The functions representable by $\lambda$-expressions of type $I\rightarrow I \rightarrow \ldots \rightarrow I$ where $I = (o\rightarrow o)\rightarrow o\rightarrow o$ are exactly the functions generated by the constants $0$ and $1$ using the operations addition, multiplication and conditional.
\end{theorem}

We prove that if we limit ourselves to Safe terms, the representable functions are exactly the multivariate polynomials:
\begin{theorem}
The functions representable by Safe $\lambda$-expressions of type $I\rightarrow I \rightarrow \ldots \rightarrow I$ where $I = (o\rightarrow o)\rightarrow o\rightarrow o$ are exactly the functions generated by the constants $0$ and $1$ using the operations addition and multiplication.
\end{theorem}
\begin{proof}
Numbers constants are encoded using Church Numerals: $\encode{n} = \lambda s z. s^n z$. 
Addition: for $n,m \in \nat$, $\encode{n+m} = \lambda \alpha^{(o,o)} x^o . (\encode{n} \alpha) (\encode{m} \alpha x)$. Multiplication: $\encode{n . m} = \lambda \alpha^{(o,o)} . \encode{n} (\encode{m} \alpha)$.
All these terms are safe and clearly any multivariate polynomial $P(n_1, \ldots, n_k)$ can be computed by composing the addition and multiplication terms as appropriate.

We now prove the converse. We consider safe lambda terms of type $I\rightarrow I  \rightarrow I$ which corresponds to polynomials with 2 variables. The generalisation to terms of type $I^n \rightarrow I$ for $n>2$ is immediate (they correspond to polynomials with $n$ variables).
 
Consider a Safe $\lambda$-term $U:I\rightarrow I\rightarrow I$. Without lose of generality we can assume that $U = \lambda x y \alpha z. u$ where $u$ is a safe term of ground type in $\beta$-normal form with $fv(u) \subseteq \{ x, y : I, z :o, \alpha : o\rightarrow o \}$.

\emph{Notation:} Let $T$ be a set of terms of type $\tau \rightarrow \tau$ and $T'$ be a set of terms of type $\tau$ then $T \cdot T'$ denotes the set of terms $\{ s s' : \tau \ | \ s \in T \wedge s' \in T' \}$. We also define 
$T^k \cdot T'$ recursively as follows:  $T^0 \cdot T' = T'$ and
for $k\geq 0$, $T^{k+1} \cdot T' = T \cdot (T^k \cdot T')$ (i.e. $T^k \cdot T'$ denotes $\{ s_1( \ldots (s_k s'))  \ | \ s_1, \ldots, s_k \in T \wedge s' \in T' \}$). We define $T^+\cdot T' = \Union_{k > 0} T^k \cdot T'$ and 
$T^*\cdot T' = (T^+\cdot T') \union T'$.

Let us write $\mathcal{N}^\tau$ for the set of $\beta$-normal terms of type $\tau$ where $\tau$ ranges in $\{ o, o\rightarrow o, I  \}$ and with free variables in $\{ x,y:I, z:o, \alpha:o\rightarrow o\}$. We write $\mathcal{A}^\tau$ for the subset of $\mathcal{N}^\tau$ consisting of applicative terms only (i.e. not abstractions).

Let $B$ be the set of terms of type $o\rightarrow o$ defined by $B = \{ \alpha \} \union \{ \lambda a.b \ | \ b \in \{a,z\}, a \neq z \}$.
It is easy to see that the following equations hold:
\begin{eqnarray*}
\mathcal{A}^I &=& \{ x,y \} \\
\mathcal{N}^{(o,o)} &=& B \union \mathcal{A}^I \cdot
\mathcal{N}^{(o,o)} = (\mathcal{A}^I)^* \cdot B \\
\mathcal{A}^{(o,o)} &=& \{ \alpha \} \union (\mathcal{A}^I)^+ \cdot B \\
\mathcal{N}^o &=& \{ z \} \union \mathcal{A}^{(o,o)} \cdot \mathcal{N}^o = (\mathcal{A}^{(o,o)})^* \cdot \{ z \} \\
\mathcal{A}^o &=& \mathcal{N}^{o}
\end{eqnarray*}

Consequently, $$\mathcal{A}^o = \left( \{\alpha \} \union \{x,y\}^+ \cdot \left( \{\alpha \} \union \{\lambda a.b \ | \ b \in \{a,z\}, a \neq z \} \right) \right)^* \cdot \{ z \}$$

We have $u \in \mathcal{A}^o$. Moreover since $u$ is a safe term,
terms of the form $\lambda a . z$ with $a \neq z$ cannot occur at an
operand position in $u$. Therefore:
\begin{equation}
u \in \left( \{\alpha\} \union \{x,y\}^+ \cdot \{\alpha,
\underline{i} \} \right)^* \cdot \{ z \} \label{eqn:u}
\end{equation}
where $\underline{i}$ is the identity term of type $o\rightarrow o$.

For two sets of terms $T$ and $T'$, we introduce the notation $T =_\beta T'$ meaning that for all $t \in T$, there is a $t' \in T'$ such that $t =_\beta t'$ and reciprocally.

We observe that for all $m \in \nat$, we have $\encode{m} \underline{i} =_\beta \underline{i}$ and for $l\geq 1$, for all $k_1, \ldots k_l \in \nat$, 
$\encode{k_1}\ldots \encode{k_l} \alpha =_\beta
\encode{k_1\times \ldots \times k_l} \alpha$. Hence
\begin{align}
\{\encode{m},\encode{n}\}^+ \cdot \{\alpha, \underline{i} \} &=_\beta
\{ \underline{i} \} \union
\{ \encode{m^i n^j} \alpha \ |\ i+j \geq 1 \} \nonumber \\
&= \{ \encode{m^i n^j} \alpha \ |\ i,j \geq 0 \} & ( \mbox{since } \encode{0} \alpha = \underline{i}) \label{eqn:intermediate}
\end{align}
and we have:
\begin{align*}
u[\encode{m} \encode{n}/x,y] &\in \left( \{ \alpha \} \union \{\encode{m},\encode{n}\}^+ \cdot \{\alpha, \underline{i} \} \right)^* \cdot \{ z \}  & \mbox{(by eq. \ref{eqn:u})} \\
&=_\beta \left( \{\alpha \} \union \{ \encode{m^i n^j}
\alpha \ | \ i,j \geq 0 \} \right)^* \cdot \{ z \} & \mbox{(by eq. \ref{eqn:intermediate})} 
\end{align*} 

Furthemore, the following equalities hold for all $m,n,r\in \nat,i,j\geq 0$:
\begin{eqnarray*}
\encode{m^i n^j} \alpha (\alpha^r z) &=_\beta&
\alpha^{r + m^i n^j} z \\
\alpha ( \alpha^r z ) &=_\beta& \alpha^{r+1} z
\end{eqnarray*}

Hence $u[\encode{m} \encode{n}/x,y] =_\beta \alpha^p z$ where
$$ p = \sum_{0\leq k \leq d} m^{i_k} n^{j_k} + C$$
for some constant $C \in \nat$, $d\geq 0$ and $i_k,j_k \geq 0$ for $0 \leq k \leq d$. Thus $U \encode{m} \encode{n} =_\beta \encode{p}$. 

Consequently, safe terms of type $I \rightarrow I \rightarrow I$ computes
exactly the polynomials of two variables.
\end{proof}

\begin{corollary}
The conditional operator $C:I\rightarrow I\rightarrow I \rightarrow I$ verifying the following equations:
\begin{eqnarray*}
C t y z &\rightarrow_\beta& y \mbox{, if } t \rightarrow_\beta \encode{0} \\
C t y z &\rightarrow_\beta& z \mbox{, if } t \rightarrow_\beta \encode{n+1}
\end{eqnarray*}
is not definable in the Safe Simply Typed $\lambda$-Calculus.
\end{corollary}



\bibliographystyle{plain}
\bibliography{../bib/gamesem,../bib/higherorder,../bib/lambdacalculus}


\end{document}
