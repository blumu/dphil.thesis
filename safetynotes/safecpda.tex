\documentclass{article}
\usepackage{amsmath, amsthm, amssymb}
\usepackage[defblank]{paralist}
\usepackage{shadowbox}
\usepackage{a4wide}

\newcommand{\ord}{\mathop{\mathrm{ord}}}
\newcommand{\subseqof}{\sqsubseteq}
\newcommand{\prefixof}{\leqslant}
\newcommand{\suffixof}{\geqslant}
\newcommand\defname[1]{{\bf\em #1}\index{#1}}

% misc
\newcommand\assignar\leftarrow

\newtheorem{theorem}{Theorem}[section]
\newtheorem{corollary}[theorem]{Corollary}
\newtheorem{conjecture}[theorem]{Conjecture}

\newtheorem{lemma}{Lemma}[section]
\newtheorem{proposition}{Proposition}[section]

\theoremstyle{remark}
\newtheorem{remark}{Remark}[section]
\newtheorem{example}{Example}[section]
\newtheorem{property}{Property}[section]

\theoremstyle{definition}
\newtheorem{definition}{Definition}[section]
\newtheorem{algorithm}{Algorithm}[section]

% order-decomposition of a 1-stack
\newcommand\orddec\overline

\newcommand\saferel\rightsquigarrow


\author{William Blum}
\title{Encoding of a safe order-$n$ recursion scheme into a $n$-PDA}

\begin{document}
\maketitle
\begin{abstract}
In \cite{hague-sto07}, Hague, Murawsky, Ong and Serre introduced
higher-order collapsible pushdown automata (CPDA) and showed their
equivalence with higher-order recursion scheme. They proposed an
algorithm that transforms a given order-$n$ recursive  scheme G to
an equivalent order-$n$ CPDA. Here we show that if the recursion
scheme is safe then the generated automaton does not require the
extra power of a CPDA and can in fact be simulated by an order
$n$-PDA ({\it i.e.} push-down automaton). The proof does not make
use of the type-homogeneity constraint (\cite{KNU02}).
\end{abstract}


\section{Preliminaries}

Take an order-$n$ {\bf safe} recursion scheme $G$. Let $Gr(G)$ be the computation graph of $G$.
We consider the $n$-CPDA $CPDA(G)$ as defined in \cite[Definition 5.2]{hague-sto07}.
Recall that the stack-alphabet $\Gamma$ is defined as the set of nodes of the computation graph.
An element of the stack is written $a^{(j,k)}$ where $a\in \Gamma$ and the exponent $(j,k)$
encodes the pointer associated to the stack symbol. We observe that $CPDA(G)$ is such that the link associated to a lambda node $\lambda \overline{\xi}$ is always of the form $(n-\ord{\lambda \overline{\xi}}+1, k)$, therefore the first component of the link is always recoverable from the node itself. For this reason we omit it when representing stack symbols {\it i.e.} we write $\lambda \overline{\xi}^{k}$ to mean $\lambda \overline{\xi}^{(n-\ord{\lambda \overline{\xi}}+1,k)}$.


{\it Remark:} It is possible to redefine $CPDA(G)$ from
\cite{hague-sto07} in a more compact way by merging the subcases of
$(V_1)$ and $(V_0)$. This is done as follows: In case (A), when
pushing the prime child of an application node $@$ on the stack, we
associate a link to it that points to the preceding stack symbol in
the top $1$-stack {\it i.e.} the $@$-node itself.
 This modification permits us to avoid the consideration on $j$ (the child-index of $u$'s binder) in the
 cases ($V_0$) and ($V_1$). The sequences of instruction $pop_1^{p+1}$ can now be replaced by
 $pop_1^p ; collapse$. This gives us the CPDA $CPDA'(G)$ describe in
Figure \ref{fig:cpdaprime}.
\begin{figure}[htbp]
\begin{center}
\makebox{
\begin{shadowbox}[10cm]
Suppose $u$ is a variable which is the $i$-parameter of
its binder and let $p$ be the span of $u$.
\begin{itemize}
\item $(V_1)$ If the variable has order $l\geq 1$, then
$$\delta(u) = push_{n-l+1} ; pop_1^p ; collapse;push_1 E_i(top_1)^{(n-l+1,1)}$$
\item $(V_0)$ If the variable is of ground type then
$$\delta(u) = pop_1^p ; collapse;push_1 E_i(top_1)$$
\end{itemize}
\caption{CPDA'(G): a compact version of CPDA(G).}
\label{fig:cpdaprime}
\end{shadowbox}
}
\end{center}
\end{figure}
This definition is more compact, however it makes it harder to
understand how the CPDA works. We will therefore stick to the
original definition for the purpose of proving our result and we
will make an explicit case analysis on $j$ in the proof.


\subsection{Incremental order-decomposition}

Let $s$ be a 1-stack. Let $\lambda \overline{\eta}_1^{k_1}$ be the
last $\lambda$-node in $s$ with order strictly positive and $\lambda
\overline{\eta}_l^{k_l}$ be the last $\lambda$-node in $s_{\prefixof
\lambda \overline{\eta}_{l-1}^{k_{l-1}}}$ with order strictly
greater than $\ord{\lambda \overline{\eta}_{l-1}^{k_{l-1}}}$, for $1
\leq l \leq r$ where $r$ is defined as the smallest number such that
$s_{\prefixof \lambda \overline{\eta}_{r}^{k_{r}}}$ does not contain
any lambda-node of order strictly greater than $\lambda
\overline{\eta}_{r}^{k_{r}}$. The $1$-stack $s$ can then be written
$$ s = u_{r+1} \cdot \lambda \overline{\eta}_r^{k_r} \cdot u_r \cdot
\ldots \cdot \lambda \overline{\eta}_1^{k_1} \cdot  u_1 $$ for some
sequences $u_i$ of stack symbols for $1 \leq i \leq r+1$ where for
$1\leq l \leq r$, all the lambda-nodes occurring in $u_l$ have order
strictly smaller than $\ord{\lambda \overline{\eta}_l}$ and such
that $\ord{\lambda \overline{\eta}_l^{k_l}}
> \ord{\lambda \overline{\eta}_{l'}^{k_{l'}}}$ whenever $l>l'$.
By convention $r=0$ means that there is no lambda node in $s$ with
positive order.

We call \defname{incremental order-decomposition} (or just
order-decomposition) of $s$ and we write $\orddec{s}$ to denote the
subsequence of $s$ consisting of the lambda-nodes $\lambda
\overline{\eta}_l^{k_l}$ for $1\leq l \leq r$ as defined above. Note
that this sequence is uniquely determined. We also extend this
notion to higher-order stacks: the incremental order-decomposition
of a higher-order stack $s$ is defined as the incremental
order-decomposition of $top_2 s$: $\orddec{s} = \orddec{top_2 s}$.

\begin{lemma}
\label{lem:push1pop1_orderdecompo} Let $s$ be a higher-order stack
with the following order-decomposition $\orddec{s} = \langle \lambda
\overline{\eta}_r^{k_r}, \ldots, \lambda \overline{\eta}_1^{k_1}
\rangle$. Then
\begin{enumerate}[i.]
\item For all lambda node $a \in \Gamma$ and link $(j,k)$ we have
 $$ \orddec{push_1 a^{(j,k)} s} = \left\{
                                       \begin{array}{ll}
                                        \langle a^k \rangle, &  \hbox{if $\ord{a} > \ord{\lambda \overline{\eta}_r^{k_r}}$}; \\
                                         \langle \lambda \overline{\eta}_r^{k_r}, \ldots, \lambda
\overline{\eta}_i^{k_i}, a^k \rangle, & \hbox{otherwise, where $i = \min \{ i \ | \ \ord{a} <
\ord{\lambda \overline{\eta}_i^{k_i}} \}$.}
                                       \end{array}
                                     \right.$$

\item For all non-lambda node $a \in \Gamma$ and link $(j,k)$ we have
$$ \orddec{push_1 a^{(j,k)} s} = \orddec{s} \ .$$

\item If $top_1 s$ is not a lambda-node then
$$ \orddec{pop_1 s} = \orddec{s} \ .$$
\end{enumerate}
\end{lemma}
\begin{proof}
  Trivial from the definition of $\orddec{t}$ and
  $\orddec{s}$.
\end{proof}

\subsection{Safe higher-order stack}

In the following, we use the term \defname{configuration} to refer to an order-$n$ stack where $n$ is the order
of the recursion scheme (which is also the order of CPDA(G)).

\begin{definition}[Safe higher-order stack]
\label{dfn:safestack} Let $s$ be an order-$j$ stack for $j\geq1$, $r
= |\orddec{s}|$ and $1 \leq q \leq r$. We say that $s$ is
\defname{$q$-safe} just if
\begin{itemize}
\item $\orddec{s} = \langle \lambda \overline{\eta}_r^1, \ldots ,
\lambda \overline{\eta}_q^1, \lambda
\overline{\eta}_{q-1}^{k_{q-1}}, \ldots , \lambda
\overline{\eta}_1^{k_1} \rangle$ ({\it i.e.}~{\it i.e.}~the
first $q$ nodes in $\orddec{s}$ have their $k$-component set
to $1$) ;
\item for all $q \leq i \leq r$ such that $n-\ord{\lambda \overline{\eta}_i}+1 <
\ord{s}$, $collapse (s_{\prefixof \lambda \overline{\eta}_i^1})$ is $1$-safe.
\end{itemize}
We say that $s$ is \defname{safe} if it is $1$-safe.
\end{definition}


\begin{lemma}[Stack operations preserving safety]
\label{lem:push1pop1_preserves_safety} Let $s$ be a higher-order
safe stack. Then:
\begin{enumerate}
  \item If $top_1\ s$ is not a lambda-node then $pop\ s$ is safe;
  \item If $a$ is a lambda-node then $push_1\ a^{(j,1)}\ s$ is safe for $1 \leq j \leq n$;
  \item If $a$ is not a lambda-node then $push_1\ a^{(j,k)}\ s$
is safe for $1 \leq j \leq n$, $k \geq 1$.
\end{enumerate}
\end{lemma}
\begin{proof}
This is a direct consequence of Lemma
\ref{lem:push1pop1_orderdecompo}.
\end{proof}

\begin{definition}
For $q\geq 0$, we define the relation $\saferel_q$ on higher-order
stacks as follows.

$s \saferel_q s'$ just if
\begin{itemize}
\item $\orddec{top_2\ s} = \langle \lambda \overline{\eta}_r^1, \ldots , \lambda
\overline{\eta}_q^1, \lambda \overline{\eta}_{q-1}^{k_{q-1}}, \ldots
, \lambda \overline{\eta}_1^{k_1} \rangle$ ;

\item $s' = collapse (s_{\prefixof \lambda \overline{\eta}_i^1})$ for some $i$ in
$\{1..q\}$ such that $n - \ord{\lambda \overline{\eta}_i} + 1 <
\ord{s}$.
\end{itemize}

We abbreviate $\saferel_1$ into $\saferel$ and we write $\saferel^*$ to denote the transitive reflexive closure of $\saferel$.

\end{definition}
The definition of safety can therefore be reformulated as follows
using the relation $\saferel_q$:
\begin{lemma}
\label{lem:qsafety_equivdef}
 Let $s$ be a stack.
\begin{enumerate}[i.]
  \item $s$ is $q$-safe if and only if $top_2\ s$ is $q$-safe and for all $s'$ such that $s \saferel_q s'$, $s'$ is $1$-safe;

  \item Consequently, $s$ is safe if and only if for all $s'$ such that $s
  \saferel^*
s'$, $\orddec{top_2\ s'} = \langle \lambda \overline{\eta}_r^1,
\ldots , \lambda \overline{\eta}_1^{1} \rangle$.
\end{enumerate}
\end{lemma}

Let $s$ be a higher-order stack. We define $s^{\langle j \rangle}$ as the operation that replaces
every link occurring in $s$ of the form $(j,k)$ by $(j,k+1)$. Formally,
\begin{align*}
{a^{(j,k)}}^{\langle j \rangle} &= a^{(j,k)}   \\
{a^{(j,k)}}^{\langle j' \rangle} &= a^{(j,k+1)} &   \mbox{when $j\neq j'$,}\\
[s_1 \ldots s_p]^{\langle j \rangle} &= [s_1^{\langle j \rangle} \ldots s_p^{\langle j \rangle}] \ .
\end{align*}

\begin{lemma}
\label{lem:incrk_qsafe}
 Let $j\geq2$ and $s$ be a higher-order stack
such that $\orddec{s} = \langle \lambda \overline{\eta}_r^{k_r},
\ldots , \lambda \overline{\eta}_1^{k_1} \rangle$ contains some node
with order $>n-j+1$ and let $q$ be the index of the last such node.

If $s$ is safe then $s^{\langle j \rangle}$ is $q$-safe.
\end{lemma}

\begin{lemma}
\label{lem:top_qsafe} If $s$ is safe then $top_{\ord{s}} s$ is safe.
\end{lemma}

\begin{lemma}
\label{lem:cons_qsafety} Let $s$ and $t$ be two stacks such that:
\begin{enumerate}[1.]
\item $s = [s_1 \ldots s_{k+1}]$ is safe and of order $\geq 2$;
\item $t$ is a $q$-safe $(\ord{s}-1)$-stack for some $q\geq1$.
\end{enumerate}
Then $[s_1 \ldots s_{k+1}~ t]$ is $q$-safe.
\end{lemma}
\begin{proof}
We use Lemma \ref{lem:qsafety_equivdef}(i). Firstly, since $t$ is
$q$-safe so is $top_2 [s_1 \ldots s_{k+1} \cdot t]$. Now suppose
that $[s_1 \ldots s_{k+1} \cdot t] \saferel_q s'$, we need to show
that $s'$ is safe. In other words we need to prove that for all
$s''$ such that $s' \saferel s''$, $s''$ is safe. We proceed by
induction on the number of $\ord{s'}-1$-stacks in $s'$.

\hrulefill

 if $t\saferel_q t'$ for some $t'$ then for all lambda node
$\lambda \overline{\xi}$ of order $n-\ord{s}+1$ occurring in
$\orddec{t'}$ we have that $collapse([s_1 \ldots s_{k+1} \cdot
t'_{\prefixof{\lambda \overline{\xi}}}])$ is safe.


\hrulefill

Since $s$ is safe we have:
\begin{equation}
 \orddec{s} = \langle \lambda \overline{\eta}_r^1
, \ldots, \lambda \overline{\eta}_1^1   \rangle \ . \label{eqn:orddec_s}
\end{equation}

Let us first analyze the effect that the operation has on the top
order-$1$ stack. The pointers in $top_2\ s$ are updated as follows:
the $k$ component of each link is increased by one if the order of
the stack symbol is $l$ and is kept unchanged otherwise.

From Eq. \ref{eqn:orddec_s} we get:
\begin{equation}
\orddec{s'} = \orddec{push_{n-l+1}\ s} = \langle
\lambda \overline{\eta}_r^1
, \ldots,  \lambda \overline{\eta}_{b}^1, \lambda \overline{\eta}_{b-1}^{k_{b-1}}, \ldots,
 \lambda \overline{\eta}_1^{k_{1}} \rangle
\ . \label{eqn:orddec_pushj_s}
\end{equation}
Hence $top_2 s'$ is $b$-safe. Thus to show that $s'$ is $b$-safe, it
just remains to prove that $collapse(s'_{\prefixof \lambda
\overline{\eta}_{i}})$ is $1$-safe for all $b \leq i \leq r$.

\hrulefill

The stack symbols that are duplicated by the $push_j$ operation are
the ones lying in the $top$ order-$(j-1)$ stack. More precisely,
suppose $top_{j+1}\ c = [c_1\ \ldots\ c_{l+1}]$ then $top_{j+1}\
(push_j\ c) = [c_1\ \ldots\ c_{l+1}\ c_{l+1}^{\langle j \rangle}]$
and the elements that are created lies in $top_j\ (push_j\ c) =
c_{l+1}^{\langle j \rangle}$.

Take any such element $a^{(l,k)}$. If $l=j$ then the link points to
the same prefix stack as some other stack symbol in $top_j\ c$, thus
the induction hypothesis permits to conclude.

If $l \neq j$ then the link points somewhere in $top_j\ (push_j\ c)$
and in that case $top_2( collapse(c_{\prefixof{a}})) = top_2(
collapse( pop_{j+1} c_{\prefixof{a}}))$ so we can conclude by the
induction hypothesis.


\hrulefill

\end{proof}


\begin{lemma}
\label{lem:pushj_safe_implies_b-safe} Let $c$ be a reachable
configuration of $CPDA(G)$ such that $c$ is \emph{safe} and $top_1\
c$ is a variable $x$ of order $l\geq 1$. Suppose $\orddec{c} =
\langle \lambda \overline{\eta}_r^1 , \ldots, \lambda
\overline{\eta}_1^1   \rangle$. Then
\begin{enumerate}[i.]
\item $\orddec{c}$ contains at least a node with order strictly
greater than $l$. Let $b$ be the index of the last lambda node
in $\orddec{c}$ verifying this condition;

\item $\lambda \overline{\eta}_{b}$ is precisely $x$'s binder;

\item $push_{n-l+1}\ c$ is $b$-safe.\footnote{We reproduce here the definition of the
CPDA operation $push_j$ from \cite{hague-sto07}:
$$ push_j \underbrace{[ s_1 \ldots s_{l+1} ]}_{s} =
\left\{
  \begin{array}{ll}
\    [s_1\ \ldots\ s_{l+1}\ s_{l+1}^{\langle j \rangle}]  &\hbox{if $j = \ord{s}$;}\\
\    [s_1\ \ldots\ s_{l+1}\ push_j\ s_{l+1}]  &\hbox{if $j<\ord{s}$.}
 \end{array}
\right.
$$}

\end{enumerate}
\end{lemma}
\begin{proof}
\begin{enumerate}[i.]
\item By Corollary 8 from \cite{hague-sto07}, the $top_2$
     stack of a configuration contains the P-view of some
    traversal whose last visited node is the $top_1$ symbol, and
    by Proposition 6 from \cite{OngLics2006}, the P-view of a
    traversal is exactly the path from the last visited node to
    the root of the computation tree (obtained by unfolding the
    computation graph). Consequently, since the order of the
    binder of a variable is strictly greater than the order of
    the variable itself, $x$'s binder must occur in $\orddec{c}$
    and therefore conditions (i) is satisfied.

\item $\lambda \overline{\eta}_{b}$ is precisely the binder of $x$.
    Indeed, since the recursion scheme is safe, $x$ is
 incrementally bound (see \cite{blumong:safelambdacalculus})
 which means that its binder is the first $\lambda$-node in the
 path to the root in the computation tree with order strictly
 greater than $x$.

\item We show a preliminary result.
    Let $s=[s_1 \ldots s_{c+1}]$ be any safe higher-order stack of order
    $\geq n-l+1 \geq 2$ (not necessarily a proper configuration). We
    show that $push_{n-l+1} s$ is $q$-safe by induction on the order of
    $s$. There are two subcases, the base case being handled by the
    first one with $\ord{s} = n-l+1 = 2$:
    \begin{enumerate}[a.]
      \item Suppose that $n-l+1 = \ord{s}$ then
    $push_{n-l+1} s = [ s_1 \ldots s_{c+1} s_{c+1}^{\langle n-l+1
    \rangle}]$.

    Since $s$ is safe, by Lemma \ref{lem:top_qsafe} so is $s_{c+1}$
    which in turn implies that $s_{c+1}^{\langle n-l+1\rangle}$ is
    $b$-safe by Lemma \ref{lem:incrk_qsafe}. Finally Lemma
    \ref{lem:cons_qsafety} shows that  $push_{n-l+1} s$ is $b$-safe.

      \item Suppose that $n-l+1 < \ord{s}$ then
    $push_{n-l+1} s = [ s_1 \ldots s_{c+1} push_{n-l+1} s_{c+1}]$.

    $s_{c+1}$ is safe by Lemma \ref{lem:top_qsafe} thus by the
    induction hypothesis $push_{n-l+1} s_{c+1}$ is $b$-safe and so
    is $push_{n-l+1} s$ by Lemma \ref{lem:cons_qsafety}.
        \end{enumerate}
    Now it suffices to apply this result on the configuration $c$, which
    is an order-$n$ safe stack, to obtain that $push_{n-l+1}\ c$ is
    $b$-safe.
\end{enumerate}
\end{proof}





\section{Simulation}

We claim that provided that the input recursion scheme is safe, the
$collapse$ operation in $CPDA(G)$ can be simulated by the operation
$pop_{n-\ord{top_1(s)}+1}$. (If instead $CPDA'(G)$ is used then
$collapse$ is simulated by $pop_{n-\ord{top_1(s)}+1}$ if $top_1(s)$
is a $j$-child with $j\geq 2$ and by $pop_1$ if it is a prime
child).

\section{Proof of correctness}

\begin{lemma}
Let $s$ be a reachable configuration (with respect to the
$\rightarrow$-step relation\footnote{Recall that $s\rightarrow s'$
just if $s' = \delta(top_1 s)(s)$ where $\delta$ is the transition
function defined in \cite[Figure 2]{hague-sto07}.}). Then:
\begin{enumerate}[i.]
\item $s$ is safe;
\item Whenever the $collapse$ operation is executed during a
transition $s \rightarrow s'$, the $top_1$ symbol of stack has a
link of type $k=1$.
\end{enumerate}
\end{lemma}
\begin{proof}
We prove (i) and (ii) simultaneously by induction on the number of
$\rightarrow$-steps. Let $u = top_1\ s$ and $s' =\delta(u)(s)$.
Suppose that $s$ verifies (i) and (ii) we prove that so does $s'$:
\begin{itemize}
\item[Case (A)] Pushing @ on the top $1$-stack preserves safety (Lemma
\ref{lem:push1pop1_preserves_safety}), therefore by the
induction hypothesis, condition (i) also holds for $s'$. Since
the $collapse$ operation is not executed, (ii) holds as well.

\item Case (S)
We have $s' = push_1 a^{(n-\ord{a}+1,1)} s$ where $a$ is a
lambda-node. Lemma \ref{lem:push1pop1_preserves_safety} and the
induction hypothesis implies that conditions (i) also holds for
$s'$. Condition (ii) holds trivially.

\item Case (L) Again, Lemma \ref{lem:push1pop1_preserves_safety}
shows that (i) holds and (ii) trivially holds.

\item Case ($V_1$) Suppose $u$ is labelled by a variable $x$ of order $l\geq 1$.

Since $s$ is safe we have $ \orddec{s} = \langle \lambda
\overline{\eta}_r^1 , \ldots, \lambda \overline{\eta}_1^1
\rangle$, and by Lemma \ref{lem:pushj_safe_implies_b-safe}
$push_{n-l+1}\ s$ is $b$-safe where $b$ is the index of the
lambda node $\lambda \overline{\eta}_{b}$ that binds $x$.

%\begin{itemize}
%\item Suppose that $\lambda \overline{\eta}_{b}$ is a prime child, then $top_1 ((push_{n-l+1};pop_1^{p+1}) s) =
%@$ and therefore by Lemma
%\ref{lem:push1pop1_preserves_safety}
%$(push_{n-l+1};pop_1^p;pop_1) s$ is safe.
%
%By Lemma \ref{lem:push1pop1_preserves_safety} again, $s' =
%\left(push_{n-l+1};pop_1^p;push_1 E_i(top_1)^{(n-l+1,1)}
%\right) s$ is safe thus condition (i) holds. Moreover
%condition (ii) holds trivially.
%
%\item Suppose that $\lambda \overline{\eta}_{b}$ is a $j$-child with $j>0$.

After performing $pop_1^p$, the top stack symbol contains $x$'s
binder which is precisely $\lambda \overline{\eta}_{b}^1$. The
link attached to this node is of type $k=1$, this guarantees
that condition (ii) is verified when the $collapse$ instruction
is executed.

Furthermore, since $push_{n-l+1}\ s$ is $b$-safe, we have that
$t = collapse \left( (push_{n-l+1}\ s)_{\prefixof \lambda
\overline{\eta}_{b}} \right) = (push_{n-l+1};pop_1^p;collapse)
s$ is safe. Since $s' = push_1 E_i(top_1)^{(n-l+1,1)}\ t$, by
Lemma \ref{lem:push1pop1_preserves_safety} $s'$ is also safe
thus condition (i) holds.
%\end{itemize}

\item ($V_0$) This case is treated identically to ($V_1$).
\end{itemize}
\end{proof}

\bibliographystyle{abbrv}
\bibliography{../bib/dphil-all}

\end{document}
