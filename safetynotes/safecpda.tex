\documentclass{article}
\usepackage{amsmath, amsthm, amssymb}
\usepackage[defblank]{paralist}
\usepackage{shadowbox}
\usepackage{a4wide}

\newcommand{\ord}{\mathop{\mathrm{ord}}}
\newcommand{\subseqof}{\sqsubseteq}
\newcommand{\prefixof}{\leqslant}
\newcommand{\suffixof}{\geqslant}
\newcommand\defname[1]{{\bf\em #1}\index{#1}}

% misc
\newcommand\assignar\leftarrow

\newtheorem{theorem}{Theorem}[section]
\newtheorem{corollary}[theorem]{Corollary}
\newtheorem{conjecture}[theorem]{Conjecture}

\newtheorem{lemma}{Lemma}[section]
\newtheorem{proposition}{Proposition}[section]

\theoremstyle{remark}
\newtheorem{remark}{Remark}[section]
\newtheorem{example}{Example}[section]
\newtheorem{property}{Property}[section]

\theoremstyle{definition}
\newtheorem{definition}{Definition}[section]
\newtheorem{algorithm}{Algorithm}[section]

% order-decomposition of a 1-stack
\newcommand\orddec\overline

\author{William Blum}
\title{Encoding of safe order-$n$ recursion scheme into $n$-PDA}

\begin{document}
\maketitle
\begin{abstract}
In \cite{hague-sto07}, Hague, Murawsky, Ong and Serre introduced
higher-order collapsible pushdown automata (CPDA) and showed their
equivalence with higher-order recursion scheme. They proposed an
algorithm that transforms a given order-$n$ recursive  scheme G to
an equivalent order-$n$ CPDA. Here we show that if the recursion
scheme is safe then the generated automaton does not require the
extra power of a CPDA and can in fact be simulated by an order
$n$-PDA ({\it i.e.} push-down automaton). The proof does not make
use of the type-homogeneity constraint (\cite{KNU02}).
\end{abstract}


\section{Preliminaries}

Take an order-$n$ {\bf safe} recursion scheme $G$. Let $Gr(G)$ be the computation graph of $G$.
We consider the $n$-CPDA $CPDA(G)$ as defined in \cite[Definition 5.2]{hague-sto07}.
Recall that the stack-alphabet $\Gamma$ is defined as the set of nodes of the computation graph.
An element of the stack is written $a^{(j,k)}$ where $a\in \Gamma$ and the exponent $(j,k)$
encodes the pointer associated to the stack symbol. We observe that $CPDA(G)$ is such that the link associated to a lambda node $\lambda \overline{\xi}$ is always of the form $(n-\ord{\lambda \overline{\xi}}+1, k)$, therefore the first component of the link is always recoverable from the node itself. For this reason we omit it when representing stack symbols {\it i.e.} we write $\lambda \overline{\xi}^{k}$ to mean $\lambda \overline{\xi}^{(n-\ord{\lambda \overline{\xi}}+1,k)}$.


{\it Remark:} It is possible to define the
CPDA $CPDA(G)$ constructed in \cite{hague-sto07} in a more compact way by
merging the subcases of $(V_1)$ and $(V_0)$. This is done as follows: In case (A), when pushing the prime child of an application node $@$ on the stack, we associate a link to it that points to the preceding stack symbol in the top $1$-stack {\it i.e.} the $@$-node itself.
 This modification permits us to avoid the consideration on $j$ (the child-index of $u$'s binder) in the
 cases ($V_0$) and ($V_1$). The sequences of instruction $pop_1^{p+1}$ can now be replaced by
 $pop_1^p ; collapse$ and the cases become:
 \bigskip
\begin{center}
\makebox{
\begin{shadowbox}[10cm]
Suppose $u$ is a variable which is the $i$-parameter of its binder and let $p$ be the span of $u$.
\begin{itemize}
\item $(V_1)$ If the variable has order $l\geq 1$, then
$$\delta(u) = push_{n-l+1} ; pop_1^p ; collapse;push_1 E_i(top_1)^{(n-l+1,1)}$$
\item $(V_0)$ If the variable is of ground type then
$$\delta(u) = pop_1^p ; collapse;push_1 E_i(top_1)$$
\end{itemize}
\end{shadowbox}
}
\end{center}
This definition is more compact, however it makes it harder to
understand how $CPDA(G)$ works. For that reason, for the purpose of
proving our result, we will use the original definition and we will
make an explicit case analysis on $j$ in the proof.


\begin{lemma}
  \label{lem:links_pointto_reachable}
In CPDA(G), links that are associated to stack symbols always point
to reachable configurations. Moreover if the link is created during
a transition $s\rightarrow s'$ where $s$ is a configuration
reachable in $n$ steps then the configuration pointed to by the link
is reachable in $\leq n$ steps.
\end{lemma}
\begin{proof}
  In the definition of $\delta$, whenever a $push_1$ instruction is used during a transition $s \rightarrow s'$,
the link attached to the stack symbol always point to the stack
configuration $s$.
\end{proof}


\subsection{Incremental order-decomposition}

Let $s$ be a 1-stack. Let $\lambda \overline{\eta}_1^{k_1}$ be the
last $\lambda$-node in $s$ with order strictly greater than $1$ and
$\lambda \overline{\eta}_l^{k_l}$ be the last $\lambda$-node in
$s_{\prefixof \lambda \overline{\eta}_{l-1}^{k_{l-1}}}$ with order
strictly greater than $\ord{\lambda
\overline{\eta}_{l-1}^{k_{l-1}}}$, for $1 \leq l \leq r$ where $r$
is defined as the smallest number such that $s_{\prefixof \lambda
\overline{\eta}_{r}^{k_{r}}}$ does not contain any lambda-node of
order strictly greater than $\lambda \overline{\eta}_{r}^{k_{r}}$.
The $1$-stack $s$ can then be written
$$ s = u_{r+1} \cdot \lambda \overline{\eta}_r^{k_r} \cdot u_r \cdot
\ldots \cdot \lambda \overline{\eta}_1^{k_1} \cdot  u_1 $$ for some
sequences $u_i$ of stack symbols for $1 \leq i \leq r+1$ where for
$1\leq l \leq r$, all the lambda-nodes occurring in $u_l$ have order
strictly smaller than $\ord{\lambda \overline{\eta}_l}$ and such
that $\ord{\lambda \overline{\eta}_l^{k_l}}
> \ord{\lambda \overline{\eta}_{l'}^{k_{l'}}}$ whenever $l>l'$ .

We call \defname{incremental order-decomposition} (or just
order-decomposition) of $s$ and we write $\orddec{s}$ to denote the
subsequence of $s$ consisting of the lambda-nodes $\lambda
\overline{\eta}_l^{k_l}$ for $1\leq l \leq r$ as defined above. Note
that this sequence is uniquely determined.

We also extend this notion to higher-order stacks: the incremental
order-decomposition of a higher-order stack $s$ is defined as the
incremental order-decomposition of $top_2 s$: $\orddec{s} =
\orddec{top_2 s}$.

We say that an order-decomposition $\orddec{s}$ is
\defname{safe} if all the $k$-components are equal
to $1$: $$ \orddec{s} = \langle \lambda \overline{\eta}_r^1, \ldots
, \lambda \overline{\eta}_1^1  \rangle \ .$$



\begin{lemma}
\label{lem:push1pop1_orderdecompo} Let $s$ be a higher-order stack
with the following order-decomposition $\orddec{s} = \langle \lambda
\overline{\eta}_r^{k_r}, \ldots, \lambda \overline{\eta}_1^{k_1}
\rangle$. Then
\begin{enumerate}[i.]
\item For all lambda node $a \in \Gamma$ and link $(j,k)$ we have
 $$ \orddec{push_1 a^{(j,k)} s} = \left\{
                                       \begin{array}{ll}
                                        \langle a^k \rangle, &  \hbox{if $\ord{a} > \ord{\lambda \overline{\eta}_r^{k_r}}$}; \\
                                         \langle \lambda \overline{\eta}_r^{k_r}, \ldots, \lambda
\overline{\eta}_i^{k_i}, a^k \rangle, & \hbox{otherwise, where $i = \min \{ i \ | \ \ord{a} <
\ord{\lambda \overline{\eta}_i^{k_i}} \}$.}
                                       \end{array}
                                     \right.$$

\item For all non-lambda node $a \in \Gamma$ and link $(j,k)$ we have
$$ \orddec{push_1 a^{(j,k)} s} = \orddec{s} \ .$$

\item If $top_1 s$ is not a lambda-node then
$$ \orddec{pop_1 s} = \orddec{s} \ .$$
\end{enumerate}
\end{lemma}
\begin{proof}
  Trivial from the definition of $\orddec{t}$ and
  $\orddec{s}$.
\end{proof}


\begin{lemma}[Stack operations preserving safe decomposition]
\label{lem:push1pop1_preserves_safety} Let $s$ be a higher-order
stack having a safe order-decomposition. Then:
\begin{enumerate}
  \item If $top_1 s$ is not a lambda-node then $\orddec{pop\ s}$ is safe;
  \item If $a$ is a lambda-node then $\orddec{push_1 a^{(j,1)} s}$ is safe for $1 \leq j \leq n$;
  \item If $a$ is not a lambda-node then $\orddec{push_1 a^{(j,k)} s}$
is safe for $1 \leq j \leq n$ and $k \geq 1$.
\end{enumerate}
\end{lemma}
\begin{proof}
This is a direct consequence of Lemma
\ref{lem:push1pop1_orderdecompo}.
\end{proof}

\section{Simulation}

We claim that provided that the input recursion scheme is safe, the $collapse$ operation
in $CPDA(G)$ can be simulated by the operation $pop_{n-\ord{top_1(s)}+1}$.

\section{Proof of correctness}

\begin{lemma}
Let $s$ be reachable configuration (with respect to the
$\rightarrow$-step relation\footnote{Recall that $s\rightarrow s'$
just if $s' = \delta(top_1 s)(s)$ where $\delta$ is the transition
function defined in \cite[Figure 2]{hague-sto07}}). Then:
\begin{enumerate}[i.]
\item The order-decomposition of $s$ is safe;
\item Whenever the $collapse$ operation is executed during a
transition $s \rightarrow s'$, the $top_1$ symbol of stack has a
link of type $k=1$.
\end{enumerate}
\end{lemma}
\begin{proof}
We prove (i) and (ii) simultaneously by induction on the number of $\rightarrow$-steps.

Let $u$ be the top-1 element of the stack $s$ and let $s'$ be the
stack obtained after performing the transition. Suppose that (i) and
(ii) are satisfied for all configurations that are reachable in less
or equal steps than $s$, we prove that it is also so case for $s'$:
\begin{itemize}
\item[Case (A)] Pushing @ on the top $1$-stack preserves the safety of the
order-decomposition (Lemma
\ref{lem:push1pop1_preserves_safety}), therefore by the
induction hypothesis, condition (i) also holds for $s'$. Since
the $collapse$ operation is not executed, (ii) holds as well.

\item Case (S)
We have $s' = push_1 a^{(n-\ord{a}+1,1)} s$ where $a$ is a
lambda-node. Lemma \ref{lem:push1pop1_preserves_safety} and the
induction hypothesis implies that conditions (i) also holds for
$s'$. Condition (ii) holds trivially.

\item Case (L) Again, Lemma \ref{lem:push1pop1_preserves_safety}
shows that (i) holds and (ii) trivially holds.

\item Case ($V_1$) Suppose u is labelled by a variable $x$.

By the induction hypothesis we have:
$$ \orddec{s} = \langle \lambda \overline{\eta}_r^1
, \ldots,  \lambda \overline{\eta}_1^1   \rangle \ .$$


After performing the operation $push_{n-l+1}$, the pointers in
the top $1$-stack are updated as follows: the $k$ component of
each link is increased by one if the order of the stack symbol
is $l$ and is kept unchanged otherwise. Let $b$ be the index of
the last lambda node in the order decomposition of $top_2 s$
with order strictly greater than $l$ (there must be at least one
since the binder of $u$ occurs in $top_2 s$). We have:
$$ \orddec{push_{n-l+1} s} = \langle
\lambda \overline{\eta}_r^1
, \ldots,  \lambda \overline{\eta}_{b}^1, \lambda \overline{\eta}_{b-1}^2, \ldots,
 \lambda \overline{\eta}_1^2 \rangle
\ .$$


Since the recursion scheme is safe, $x$ is incrementally bound
 (see \cite{blumong:safelambdacalculus}) which means that its
 binder is the first $\lambda$-node in the path to the root in
 the computation tree (obtained by unfolding the computation
 graph) with order strictly greater than $x$. But by Corollary 8
 from \cite{hague-sto07}, the $top_2$ stack of a configuration
 contains the P-view of a traversal finishing with the $top_1$
 symbol, and by Proposition 6 from \cite{OngLics2006}, the
 P-view of a traversal is exactly the path from the last visited
 node to the root of the computation
tree. Hence the binder of $x$ is precisely $\lambda
\overline{\eta}_{b}^1$ and therefore:
$$ \orddec{(push_{n-l+1};pop_1^p) s} =
\langle \lambda \overline{\eta}_r^1, \ldots,  \lambda
\overline{\eta}_{b}^1 \rangle$$ which is safe.

\begin{itemize}
\item Suppose that $\lambda \overline{\eta}_{b}$ is a prime child, then $top_1 ((push_{n-l+1};pop_1^{p+1}) s) =
@$ and therefore by Lemma
\ref{lem:push1pop1_preserves_safety}
$\orddec{(push_{n-l+1};pop_1^p;pop_1) s}$ is safe.

By Lemma \ref{lem:push1pop1_preserves_safety} again,
$\orddec{ s'} = \orddec{ \left(push_{n-l+1};pop_1^p;push_1
E_i(top_1)^{(n-l+1,1)} \right) s}$ is safe thus condition
(i) holds. Condition (ii) holds trivially.

\item Suppose that $\lambda \overline{\eta}_{b}$ is a $j$-child with $j>0$.

The next operation is a $collapse$. Since the link attached
to the top stack symbol ($\lambda \overline{\eta}_{b}^1$) is
indeed of type $k=1$, condition (ii) is verified.

Let $t$ be the stack $(push_{n-l+1};pop_1^p;collapse) s$. By
Lemma \ref{lem:links_pointto_reachable} $t$ is a reachable
state (in no more steps than $s$), therefore by the
induction hypothesis (i) we have
$$ \orddec{t} = \langle \lambda \overline{\eta'}_{r'}^1,
\ldots, \lambda \overline{\eta'}_{1}^1 \rangle \ .$$

We have $s' = push_1 E_i(top_1)^{(n-l+1,1)} t$, therefore
since $\orddec{t}$ is safe by Lemma
\ref{lem:push1pop1_preserves_safety} $\orddec{s'}$ is also
safe. Hence condition (i) holds.
\end{itemize}

\item ($V_0$) This case is treated identically to ($V_1$).
\end{itemize}
\end{proof}

\bibliographystyle{abbrv}
\bibliography{../bib/dphil-all}

\end{document}
