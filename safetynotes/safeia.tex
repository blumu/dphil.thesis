
\begin{document}

\section{Safe Idealized Algol}

In this section, we present two different approaches that one can
follow to extend the safety restriction to a language featuring
variable referencing, such as Idealized Algol. This gives rise to
different version of ``Safe Idealized Algol''. In the first version,
all free variables must respect the safety condition whereas in the
second version, only variables that are not abstracted by the
\ianewvar construct (including variables of type $var$) are required
to respect the safety condition.


\subsection{First version: Safe IA}
- Is just IA where the application and abstraction rules are
restricted as in the safe lambda calculus. Equivalently, it is the
safe lambda calculus extended with the constants and constructs of
PCF and IA.

- The rules are formally given by:


- Game semantics of Safe PCF extends immediately to Safe IA.

- Observational equivalence.

There is an important theorem (\cite{AM97a}) about the
game-semantics model of IA which states that two IA terms are
equivalent if and only if the set of complete plays of their
denotations are equal. This was used in \cite{ghicamccusker00} to
show that observational equivalence for the $IA_2$ fragment of IA is
decidable, the set of complete plays being representable by regular
expressions. In \cite{Ong02} it as shown that it is still decidable
 for the $IA_3+Y_0$: the sets of complete plays corresponding exactly to the context-free languages
 and therefore the problem reduces to the DPDA equivalence problem which is itself decidable with an unknown
complexity.

Imposing the safety condition suggest an improvement in complexity.
For instance, in the case of $IA_3$, the complexity lies somewhere
between the complexity of regular language equivalence and
context-free language equivalence. In fact Safe $IA_3$ contains
terms whose denotation is context free such as $\lambda f . f
(\lambda x .x )$, therefore the complexity must be strictly higher
than for equivalence of regular languages.


\subsection{Second version: Safe IA'}

Rules We define new judgment forms where the context is partitioned
in two components $\Gamma$ and $\Gamma^{\ianewvar}$: the component
contains variables that can be abstracted by a $\lambda$, the second
contains variables that are abstracted by a \ianewvar construct.

%%%rules

Safe IA' consists of the term formed with the rules above and such
that the context $\Gamma^{\ianewvar}$ is empty.

\end{document}


\psset{linecolor=darkGreen,linewidth=0.5pt}


\author{William Blum}
\title{Safe Idealized Algol}

\begin{document}
\maketitle

\section{Safe Idealized Algol}

In this section, we present two different approaches that one can
follow to extend the safety restriction to a language featuring
variable referencing, such as Idealized Algol. This gives rise to
different version of ``Safe Idealized Algol''. In the first version,
all free variables must respect the safety condition whereas in the
second version, only variables that are not abstracted by the \ianew
construct (including variables of type $var$) are required to
respect the safety condition.


\subsection{First version: Safe IA}
- Is just IA where the application and abstraction rules are
restricted as in the safe lambda calculus. Equivalently, it is the
safe lambda calculus extended with the constants and constructs of
PCF and IA.

- The rules are formally given by:


- Game semantics of Safe PCF extends immediately to Safe IA.

- Observational equivalence.

There is an important theorem (\cite{AM97a}) about the
game-semantics model of IA which states that two IA terms are
equivalent if and only if the set of complete plays of their
denotations are equal. This was used in \cite{ghicamccusker00} to
show that observational equivalence for the $IA_2$ fragment of IA is
decidable, the set of complete plays being representable by regular
expressions. In \cite{Ong02} it as shown that it is still decidable
 for the $IA_3+Y_0$: the sets of complete plays corresponding exactly to the context-free languages
 and therefore the problem reduces to the DPDA equivalence problem which is itself decidable with an unknown
complexity.

Imposing the safety condition suggest an improvement in complexity.
For instance, in the case of $IA_3$, the complexity lies somewhere
between the complexity of regular language equivalence and
context-free language equivalence. In fact Safe $IA_3$ contains
terms whose denotation is context free such as $\lambda f . f
(\lambda x .x )$, therefore the complexity must be strictly higher
than for equivalence of regular languages.


\subsection{Second version: Safe IA'}

Rules We define new judgment forms where the context is partitioned
in two components $\Gamma$ and $\Gamma^{\ianew}$: the component
contains variables that can be abstracted by a $\lambda$, the second
contains variables that are abstracted by a \ianew construct.

%%%rules

Safe IA' consists of the term formed with the rules above and such
that the context $\Gamma^{\ianew}$ is empty.


\subsection{P-incrementally justified strategies}

Let $A$ be the arena of an IA type. We defined the following subset
of moves of the arena $A$:
$$ IA^{var}_{A} =\{ m \in A \ | \ m \in \{ read \} \union \{ write_i \ | \ i \in \nat \} \mbox{ and } level(m) \leq 1 \} $$


%Consider an IA term $\Gamma \vdash M :A$.

We say that a strategy $\sigma : \sem{\Gamma} \rightarrow \sem{A}$
denoting an IA term $\Gamma \vdash M :A$ is \emph{IA P-incrementally
justified} just if for all P-move $m$ not in
$IA^{var}_{\sem{\Gamma}}$, $m$ points to the last O-move in the
P-view with order $> \ord{m}$.


\bibliographystyle{plain}
\bibliography{../bib/higherorder,../bib/gamesem,../bib/lambdacalculus}

\end{document}
