
\begin{document}

\section{Safe Idealized Algol}

In this section, we present two different approaches that one can
follow to extend the safety restriction to a language featuring
variable referencing, such as Idealized Algol. This gives rise to
different version of ``Safe Idealized Algol''. In the first version,
all free variables must respect the safety condition whereas in the
second version, only variables that are not abstracted by the
\ianewvar construct (including variables of type $var$) are required
to respect the safety condition.


\subsection{First version: Safe IA}
- Is just IA where the application and abstraction rules are
restricted as in the safe lambda calculus. Equivalently, it is the
safe lambda calculus extended with the constants and constructs of
PCF and IA.

- The rules are formally given by:


- Game semantics of Safe PCF extends immediately to Safe IA.

- Observational equivalence.

There is an important theorem (\cite{AM97a}) about the
game-semantics model of IA which states that two IA terms are
equivalent if and only if the set of complete plays of their
denotations are equal. This was used in \cite{ghicamccusker00} to
show that observational equivalence for the $IA_2$ fragment of IA is
decidable, the set of complete plays being representable by regular
expressions. In \cite{Ong02} it as shown that it is still decidable
 for the $IA_3+Y_0$: the sets of complete plays corresponding exactly to the context-free languages
 and therefore the problem reduces to the DPDA equivalence problem which is itself decidable with an unknown
complexity.

Imposing the safety condition suggest an improvement in complexity.
For instance, in the case of $IA_3$, the complexity lies somewhere
between the complexity of regular language equivalence and
context-free language equivalence. In fact Safe $IA_3$ contains
terms whose denotation is context free such as $\lambda f . f
(\lambda x .x )$, therefore the complexity must be strictly higher
than for equivalence of regular languages.


\subsection{Second version: Safe IA'}

Rules We define new judgment forms where the context is partitioned
in two components $\Gamma$ and $\Gamma^{\ianewvar}$: the component
contains variables that can be abstracted by a $\lambda$, the second
contains variables that are abstracted by a \ianewvar construct.

%%%rules

Safe IA' consists of the term formed with the rules above and such
that the context $\Gamma^{\ianewvar}$ is empty.

\end{document}


\psset{linecolor=darkGreen,linewidth=0.5pt}


\author{William Blum}
\title{Safe Idealized Algol}

\begin{document}
\maketitle

\section{Safe Idealized Algol}

In this section, we present two possible approaches to extend the
safety restriction to a language featuring variable referencing,
such as Idealized Algol. This gives rise to two different versions
of ``Safe Idealized Algol''. In the first version, all free
variables must respect the safety condition whereas in the second
version, only variables that are not abstracted by the \ianew\
construct (including variables of type \iavar) are required to
respect the safety condition.


\subsection{First version: Safe IA}

 The obvious way to define safety for IA terms consists in extending the safe lambda calculus by adding rules
 for constants and constructs of PCF and IA.
 Equivalently, this is the system of rules of IA where the application and abstraction rules
 have been restricted as in the safe lambda calculus.
The rules are formally given in Table \ref{tab:safeia_formrules}.


%\newenvironment{FramedTable}%
%{\begin{table}[htbp]
%\begin{Sbox}%
%\setlength{\mylength}{\textwidth}%
%\addtolength{\mylength}{-2\fboxsep}%
%\addtolength{\mylength}{-2\fboxrule}%
%\begin{minipage}{\mylength}}%
%{\end{minipage}\end{Sbox}\shadowbox{\TheSbox}\end{table}}%


\begin{table}
 {\bf Lambda calculus part}

$$ \rulename{var} \ \rulef{}{x : A  \vdash_s x : A}
\qquad \rulename{wk} \ \rulef{\Gamma \vdash_s s :
A}{\Delta \vdash_s s : A} \quad \Gamma \subset
\Delta
$$

%\fbox
{
%\begin{Bcenter}
$$ \rulename{app} \ \rulef{\Gamma  \vdash_s s : (A_1,\ldots,A_n,B) \quad
\Gamma  \vdash_s t_1 : A_1 \quad \ldots \quad \Gamma  \vdash_s t_n : A_n
} {\Gamma  \vdash_s s t_1 \ldots t_n : B} \quad C(\Gamma ; B)$$ \\
$$ \rulename{abs} \ \rulef{\Gamma, x_1 : A_1, \ldots, x_n : A_n
  \vdash_s s : B} {\Gamma \vdash_s \lambda x_1 \ldots x_n . s :
  (A_1, \ldots ,A_n,B)} \quad C(\Gamma ; (A_1, \ldots ,A_n,B) )$$
%\end{Bcenter}
}

where the side-condition $C(\Gamma ; B)$ expresses that $\forall y:Y
\in \Gamma, \forall Y' \in Pr(Y) . \ord Y' \geq \ord B$.


{\bf Arithmetic and recursion}
$$ \rulename{const} \ \rulef{}{\emptyset  \vdash n :\texttt{exp}}
\qquad \rulename{succ} \ \rulef{\Gamma \vdash M:\texttt{exp} }{\Gamma \vdash \texttt{succ}\ M:\texttt{exp}}
\qquad \rulename{pred} \ \rulef{\Gamma \vdash M:\texttt{exp} }{\Gamma \vdash \texttt{pred}\ M:\texttt{exp}}$$

$$
\rulename{cond} \ \rulef{\Gamma\vdash M : \texttt{exp} \qquad \Gamma\vdash N_1 : \texttt{exp} \qquad \Gamma\vdash N_2 : \texttt{exp} }{\Gamma \vdash \texttt{cond}\ M\ N_1\ N_2}
\qquad  \rulename{rec} \ \rulef{\Gamma \vdash M : A\rightarrow A }{ \Gamma \vdash Y_A M : A}$$

{\bf Imperative constructs}
$$ \rulef{\Gamma| \Gamma^{\ianew} \vdash M : \texttt{com} \quad \Gamma \vdash N :A}
    {\Gamma \vdash \texttt{seq}_A \ M\ N\ : A} \quad A \in \{ \texttt{com}, \texttt{exp}\}$$

$$ \rulef{\Gamma \vdash M : \texttt{var} \quad \Gamma \vdash N : \texttt{exp}}
    {\Gamma \vdash \texttt{assign}\ M\ N\ : \texttt{com}}
\qquad
 \rulef{\Gamma \vdash M : \texttt{var}}
    {\Gamma \vdash \texttt{deref}\ M\ : \texttt{exp}}$$

$$ \rulef{\Gamma , x : \texttt{var} \vdash M : A}
    {\Gamma  \vdash \texttt{new } x \texttt{ in } M} \quad A \in \{ \texttt{com}, \texttt{exp}\}$$

$$ \rulef{\Gamma \vdash M_1 : \texttt{exp} \rightarrow \texttt{com} \quad \Gamma \vdash M_2 : \texttt{exp}}
    {\Gamma \vdash \texttt{mkvar } M_1\ M_2\ : \texttt{var}}$$
\caption{Formation rules for Safe IA} \label{tab:safeia_formrules}
\end{table}


\notetoself{ - Game semantics of Safe PCF extends immediately to
Safe IA.

- Observational equivalence.}

\subsubsection{Algorithmic game semantics}

An important theorem (\cite{AM97a}) about the game-semantics model
of IA states that two IA terms are equivalent if and only if the set
of complete plays of their denotations are equal. This result has
been used in \cite{ghicamccusker00} to show that observational
equivalence for the $IA_2$ fragment of IA is decidable -- the set of
complete plays being representable by regular expressions. In
\cite{Ong02} it as shown that it is still decidable
 for the $IA_3+Y_0$: the sets of complete plays corresponding exactly to the context-free languages
 and therefore the problem reduces to the DPDA equivalence problem which is itself decidable with an unknown
complexity.

Imposing the safety condition suggest an improvement in complexity.
For instance, in the case of $IA_3$, the complexity lies somewhere
between the complexity of regular language equivalence and
context-free language equivalence. In fact Safe $IA_3$ contains
terms whose denotation is context free -- for instance $\lambda f .
f (\lambda x .x )$ -- therefore the complexity must be strictly
higher than the complexity of regular language equivalence.


\subsection{Second version: Safe IA'}

We introduce a new judgment of the form $\Gamma | \Gamma^{\ianew}
\vdash_{s'} M : A$ where the context is partitioned into two
components: The first component contains standard lambda calculus
variables -- those that are abstracted by a $\lambda$-abstraction;
the second is called the block-variable context, it contains
variables that are abstracted by an occurrence of the \ianew\
construct. The component $\Gamma^{\ianew}$ contains variables of
type \iavar\ only while the other component can contain variables of
any type including the type \iavar. It is straightforward to
redefine the typing rule of IA in such a way that these two distinct
contexts are maintained appropriately. In particular the abstraction
rule must only be able to abstract variables taken from the first
component of the context; similarly the \ianew\ block construct must
only used to ``abstract'' variables from the block-variable context.
The full system of rules is given in table
\ref{tab:safeia2_formrules}.

\begin{table}[htbp]
{\bf Lambda calculus part}

$$ \rulename{var} \ \rulef{}{x : A | \emptyset \vdash_s x : A} \quad
\rulename{var^{\ianew}} \ \rulef{}{\emptyset | x : \iavar \vdash_s x : \iavar} $$
$$
\rulename{wk} \ \rulef{\Gamma | \Gamma^{\ianew} \vdash_s s : A}{\Delta | \Gamma^{\ianew} \vdash_s s : A} \quad
\Gamma \subset \Delta \qquad
\rulename{wk^{\ianew}} \ \rulef{\Gamma| \Gamma^{\ianew} \vdash_s s : A}{\Gamma| \Delta^{\ianew} \vdash_s s : A} \quad
\Gamma^{\ianew} \subset \Delta^{\ianew}
$$

$$ \rulename{app} \ \rulef{\Gamma | \Gamma^{\ianew} \vdash_s s : (A_1,\ldots,A_n,B) \quad
\Gamma | \Gamma^{\ianew} \vdash_s t_1 : A_1 \quad \ldots \quad \Gamma | \Gamma^{\ianew} \vdash_s t_n : A_n
} {\Gamma | \Gamma^{\ianew} \vdash_s s t_1 \ldots t_n : B} \ C(\Gamma ; B)$$
$$ \rulename{abs} \ \rulef{\Gamma, x_1 : A_1, \ldots, x_n : A_n |
\Gamma^{\ianew}
  \vdash_s s : B} {\Gamma | \Gamma^{\ianew} \vdash_s \lambda x_1 \ldots x_n . s :
  (A_1, \ldots ,A_n,B)} \ C(\Gamma ; (A_1, \ldots ,A_n,B) )$$


where $\Gamma$ and $\Delta$ are finite sets of variables and
$\Gamma^{\ianew}$ and $\Delta^{\ianew}$ are finite sets of variables
of type \iavar\ and where the side-condition $C(\Gamma ; B)$
expresses that $\forall y:Y \in \Gamma, \forall Y' \in Pr(Y) . \ord
Y' \geq \ord B$.


{\bf Arithmetic and recursion}
$$ \rulename{const} \ \rulef{}{\emptyset | \emptyset \vdash n :\texttt{exp}}
\qquad \rulename{succ} \ \rulef{\Gamma | \Gamma^{\ianew}\vdash M:\texttt{exp} }{\Gamma| \Gamma^{\ianew} \vdash \texttt{succ}\ M:\texttt{exp}}
\qquad \rulename{pred} \ \rulef{\Gamma| \Gamma^{\ianew} \vdash M:\texttt{exp} }{\Gamma| \Gamma^{\ianew} \vdash \texttt{pred}\ M:\texttt{exp}}$$

$$
\rulename{cond} \ \rulef{\Gamma| \Gamma^{\ianew} \vdash M : \texttt{exp} \qquad \Gamma| \Gamma^{\ianew} \vdash N_1 : \texttt{exp} \qquad \Gamma| \Gamma^{\ianew} \vdash N_2 : \texttt{exp} }{\Gamma| \Gamma^{\ianew} \vdash \texttt{cond}\ M\ N_1\ N_2}
\qquad  \rulename{rec} \ \rulef{\Gamma| \Gamma^{\ianew} \vdash M : A\rightarrow A }{ \Gamma| \Gamma^{\ianew} \vdash Y_A M : A}$$

{\bf Imperative constructs}
$$ \rulef{\Gamma| \Gamma^{\ianew} \vdash M : \texttt{com} \quad \Gamma| \Gamma^{\ianew} \vdash N :A}
    {\Gamma| \Gamma^{\ianew} \vdash \texttt{seq}_A \ M\ N\ : A} \quad A \in \{ \texttt{com}, \texttt{exp}\}$$

$$ \rulef{\Gamma| \Gamma^{\ianew} \vdash M : \texttt{var} \quad \Gamma| \Gamma^{\ianew} \vdash N : \texttt{exp}}
    {\Gamma| \Gamma^{\ianew} \vdash \texttt{assign}\ M\ N\ : \texttt{com}}
\qquad
 \rulef{\Gamma| \Gamma^{\ianew} \vdash M : \texttt{var}}
    {\Gamma| \Gamma^{\ianew} \vdash \texttt{deref}\ M\ : \texttt{exp}}$$

$$ \rulef{\Gamma | \Gamma^{\ianew}, x : \texttt{var} \vdash M : A}
    {\Gamma | \Gamma^{\ianew} \vdash \texttt{new } x \texttt{ in } M} \quad A \in \{ \texttt{com}, \texttt{exp}\}$$

$$ \rulef{\Gamma| \Gamma^{\ianew} \vdash M_1 : \texttt{exp} \rightarrow \texttt{com} \quad \Gamma| \Gamma^{\ianew} \vdash M_2 : \texttt{exp}}
    {\Gamma| \Gamma^{\ianew} \vdash \texttt{mkvar } M_1\ M_2\ : \texttt{var}}$$


\caption{Formation rules for Safe IA'} \label{tab:safeia2_formrules}
\end{table}

 We defined Safe IA terms to be the subset of terms typable with
the system of rules of table \ref{tab:safeia2_formrules} for which
the block-variable context is empty. For commodity we will denote
these terms by the judgment $\Gamma \vdash_{s'} M : A$, assuming the
following implicit rule:
$$ \rulef{\Gamma | \emptyset \vdash_{s'} M : A} {\Gamma \vdash_{s'} M : A}$$

\subsection{P-incrementally justified strategies}

Let $A$ be the arena of an IA type. We defined the following subset
of moves of the arena $A$:
$$ IA^{var}_{A} =\{ m \in A \ | \ m \in \{ read \} \union \{ write_i \ | \ i \in \nat \} \mbox{ and } level(m) \leq 1 \} $$


%Consider an IA term $\Gamma \vdash M :A$.

We say that a strategy $\sigma : \sem{\Gamma} \rightarrow \sem{A}$
denoting an IA term $\Gamma \vdash M :A$ is \emph{IA P-incrementally
justified} just if for all P-move $m$ not in
$IA^{var}_{\sem{\Gamma}}$, $m$ points to the last O-move in the
P-view with order $> \ord{m}$.


\bibliographystyle{plain}
\bibliography{../bib/higherorder,../bib/gamesem,../bib/lambdacalculus}

\end{document}
