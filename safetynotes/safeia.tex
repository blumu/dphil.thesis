
\begin{document}

\section{Safe Idealized Algol}

In this section, we present two different approaches that one can
follow to extend the safety restriction to a language featuring
variable referencing, such as Idealized Algol. This gives rise to
different version of ``Safe Idealized Algol''. In the first version,
all free variables must respect the safety condition whereas in the
second version, only variables that are not abstracted by the
\ianewvar construct (including variables of type $var$) are required
to respect the safety condition.


\subsection{First version: Safe IA}
- Is just IA where the application and abstraction rules are
restricted as in the safe lambda calculus. Equivalently, it is the
safe lambda calculus extended with the constants and constructs of
PCF and IA.

- The rules are formally given by:


- Game semantics of Safe PCF extends immediately to Safe IA.

- Observational equivalence.

There is an important theorem (\cite{AM97a}) about the
game-semantics model of IA which states that two IA terms are
equivalent if and only if the set of complete plays of their
denotations are equal. This was used in \cite{ghicamccusker00} to
show that observational equivalence for the $IA_2$ fragment of IA is
decidable, the set of complete plays being representable by regular
expressions. In \cite{Ong02} it as shown that it is still decidable
 for the $IA_3+Y_0$: the sets of complete plays corresponding exactly to the context-free languages
 and therefore the problem reduces to the DPDA equivalence problem which is itself decidable with an unknown
complexity.

Imposing the safety condition suggest an improvement in complexity.
For instance, in the case of $IA_3$, the complexity lies somewhere
between the complexity of regular language equivalence and
context-free language equivalence. In fact Safe $IA_3$ contains
terms whose denotation is context free such as $\lambda f . f
(\lambda x .x )$, therefore the complexity must be strictly higher
than for equivalence of regular languages.


\subsection{Second version: Safe IA'}

Rules We define new judgment forms where the context is partitioned
in two components $\Gamma$ and $\Gamma^{\ianewvar}$: the component
contains variables that can be abstracted by a $\lambda$, the second
contains variables that are abstracted by a \ianewvar construct.

%%%rules

Safe IA' consists of the term formed with the rules above and such
that the context $\Gamma^{\ianewvar}$ is empty.

\end{document}


\psset{linecolor=darkGreen,linewidth=0.5pt}


\author{William Blum}
\title{Safe Idealized Algol}

\begin{document}
\maketitle

\section{Safe Idealized Algol}

In this section, we present two possible approaches to accommodate
the safety restriction to a language such as Idealized Algol
featuring block-variable constructs. This gives rise to two
different versions of ``Safe Idealized Algol''. In the first
version, all free variables are required to satisfy the safety
constraint whereas in the second version, only variables that are
not abstracted by the \ianew\ construct (including variables of type
\iavar) are required to satisfy the safety constraint.

We recall the definition of the \defname{safety constraint} $SC(\Gamma ; B)$
for some finite type alphabet $\Gamma$ and type $B$:
$$SC(\Gamma ; B) \quad \mbox{iff} \quad  \forall x:A\; \in \Gamma . \forall A' \in Pr(A) . \ord A' \geq \ord B$$
where $Pr(A)$ denotes the set of prime sub-types of $A$.



\subsection{First version: Safe IA}

 The obvious way to define safety for IA terms consists in extending the safe lambda calculus by adding rules
 for constants and constructs of IA.
Said equivalently, it is the system of rules of IA where the
application and abstraction rules  have been restricted by the
safety constraint as in the safe lambda calculus. The rules are
formally given in Table \ref{tab:safeia_formrules}. The circled
rules are those that differs from their IA counterpart. Note that
the $\rulename{new}$ rule remains the same: it does not have the
side-condition present in the $\rulename{abs}$ rule. Therefore the
calculus allows us to (indirectly) abstract variables of type
\iavar\ without having to meet the safety requirement.


\newlength{\mylength}
\newenvironment{FramedTable}%
{\begin{table}[htbp]
\begin{Sbox}%
\setlength{\mylength}{\textwidth}%
\addtolength{\mylength}{-2\fboxsep}%
\addtolength{\mylength}{-2\fboxrule}%
\begin{minipage}{\mylength}}%
{\end{minipage}\end{Sbox}\shadowbox{\TheSbox}\end{table}}%



\begin{FramedTable}
 {\bf Lambda calculus part}

$$ \rulename{var} \ \rulef{}{x : A  \vdash_s x : A}
\qquad \rulename{wk} \ \rulef{\Gamma \vdash_s s :
A}{\Delta \vdash_s s : A} \quad \Gamma \subset
\Delta
$$

\begin{center}
\ovalbox{%
\begin{Bcenter}
$ \dps \rulename{app} \ \rulef{\Gamma  \vdash_s s : (A_1,\ldots,A_n,B) \quad
\Gamma  \vdash_s t_1 : A_1 \quad \ldots \quad \Gamma  \vdash_s t_n : A_n
} {\dps \Gamma  \vdash_s s t_1 \ldots t_n : B} \quad SC(\Gamma ; B)$ \\[12pt]
$ \dps  \rulename{abs} \ \rulef{\Gamma, x_1 : A_1, \ldots, x_n : A_n
  \vdash_s s : B} {\Gamma \vdash_s \lambda x_1 \ldots x_n . s :
  (A_1, \ldots ,A_n,B)} \quad SC(\Gamma ; (A_1, \ldots ,A_n,B) )$
\end{Bcenter}
}
\end{center}
\smallskip


{\bf Arithmetic and recursion}
$$ \rulename{const} \ \rulef{}{\emptyset  \vdash_s n :\texttt{exp}}
\qquad \rulename{succ} \ \rulef{\Gamma \vdash_s M:\texttt{exp} }{\Gamma \vdash_s \texttt{succ}\ M:\texttt{exp}}
\qquad \rulename{pred} \ \rulef{\Gamma \vdash_s M:\texttt{exp} }{\Gamma \vdash_s \texttt{pred}\ M:\texttt{exp}}$$

$$
\rulename{cond} \ \rulef{\Gamma\vdash_s M : \texttt{exp} \quad \Gamma\vdash_s N_1 : \texttt{exp} \quad \Gamma\vdash_s N_2 : \texttt{exp} }{\Gamma \vdash_s \texttt{cond}\ M\ N_1\ N_2}
\quad  \rulename{rec} \ \rulef{\Gamma \vdash_s M : A\rightarrow A }{ \Gamma \vdash_s Y_A M : A}$$

{\bf Imperative constructs}
$$ \rulename{seq} \ \rulef{\Gamma| \Gamma^{\ianew} \vdash_s M : \texttt{com} \quad \Gamma \vdash_s N :A}
    {\Gamma \vdash_s \texttt{seq}_A \ M\ N\ : A} \quad A \in \{ \texttt{com}, \texttt{exp}\}$$

$$ \rulename{assign} \ \rulef{\Gamma \vdash_s M : \texttt{var} \quad \Gamma \vdash_s N : \texttt{exp}}
    {\Gamma \vdash_s \texttt{assign}\ M\ N\ : \texttt{com}}
\qquad \rulename{deref} \
 \rulef{\Gamma \vdash_s M : \texttt{var}}
    {\Gamma \vdash_s \texttt{deref}\ M\ : \texttt{exp}}$$

$$ \rulename{new} \ \rulef{\Gamma , x : \texttt{var} \vdash_s M : A}
    {\Gamma  \vdash_s \texttt{new } x \texttt{ in } M : A} \quad A \in \{ \texttt{com}, \texttt{exp}\}$$

$$ \rulename{mkvar} \ \rulef{\Gamma \vdash_s M_1 : \texttt{exp} \rightarrow \texttt{com} \quad \Gamma \vdash_s M_2 : \texttt{exp}}
    {\Gamma \vdash_s \texttt{mkvar } M_1\ M_2\ : \texttt{var}}$$

\caption{Formation rules for Safe IA}
\label{tab:safeia_formrules}
\end{FramedTable}


\notetoself{ - Game semantics of Safe PCF extends immediately to
Safe IA.

- Observational equivalence.}

\subsubsection{Algorithmic game semantics}

An important theorem (\cite{AM97a}) about the game-semantics model
of IA states that two IA terms are equivalent if and only if the set
of complete plays of their denotations are equal. This result has
been used in \cite{ghicamccusker00} to show that observational
equivalence for the $IA_2$ fragment of IA is decidable -- the set of
complete plays being representable by regular expressions. In
\cite{Ong02} it as shown that it is still decidable
 for the $IA_3+Y_0$: the sets of complete plays corresponding exactly to the context-free languages
 and therefore the problem reduces to the DPDA equivalence problem which is itself decidable with an unknown
complexity.

Imposing the safety condition suggest an improvement in complexity.
For instance, in the case of $IA_3$, the complexity lies somewhere
between the complexity of regular language equivalence and
context-free language equivalence. In fact Safe $IA_3$ contains
terms whose denotation is context free -- for instance $\lambda f .
f (\lambda x .x )$ -- therefore the complexity must be strictly
higher than the complexity of regular language equivalence.


\subsection{Second version: Safe IA'}

It turns out that the previous definition of Safe IA is needlessly
restrictive. This section presents a language that subsumes the
previous one while preserving its good properties.

We introduce a new judgment of the form $\Gamma | \Gamma^{\ianew}
\vdash_{s'} M : A$ where the context is partitioned into two
components: The first component contains standard lambda calculus
variables -- those that are abstracted by a $\lambda$-abstraction;
the second is called the block-variable context, it contains
variables that are abstracted by an occurrence of the \ianew\
construct. The component $\Gamma^{\ianew}$ contains variables of
type \iavar\ only while the other component can contain variables of
any type including the type \iavar. It is straightforward to
redefine the typing rule of IA in such a way that these two distinct
contexts are maintained appropriately. In particular the abstraction
rule must only be able to abstract variables taken from the first
component of the context; similarly the \ianew\ block construct must
only be used to ``abstract'' variables from the block-variable
context. The full system of rules is given in Table
\ref{tab:safeia2_formrules}. The rules that are circled correspond
to those that differs from their counterpart in Table
\ref{tab:safeia_formrules}; the  differences being that:
\begin{itemize}
\item the rule $\rulename{abs}$ and $\rulename{new}$ can only
abstract variables from their respective context component;
\item the side-condition in $\rulename{abs}$ refers only to
the first context component.
\end{itemize}

\begin{FramedTable}
{\bf Lambda calculus part}

$$ \rulename{var} \ \rulef{}{x : A | \emptyset \vdash_{s'} x : A} \quad
\rulename{var^{\ianew}} \ \rulef{}{\emptyset | x : \iavar \vdash_{s'} x : \iavar} $$
$$
\rulename{wk} \ \rulef{\Gamma | \Gamma^{\ianew} \vdash_{s'} s : A}{\Delta | \Gamma^{\ianew} \vdash_{s'} s : A} \quad
\Gamma \subset \Delta \qquad
\rulename{wk^{\ianew}} \ \rulef{\Gamma| \Gamma^{\ianew} \vdash_{s'} s : A}{\Gamma| \Delta^{\ianew} \vdash_{s'} s : A} \quad
\Gamma^{\ianew} \subset \Delta^{\ianew}
$$

where $\Gamma^{\ianew}$ and $\Delta^{\ianew}$ are finite sets of variables
of type \iavar.

$$ \rulename{app} \ \rulef{\Gamma | \Gamma^{\ianew} \vdash_{s'} s : (A_1,\ldots,A_n,B) \quad
\Gamma | \Gamma^{\ianew} \vdash_{s'} t_1 : A_1 \quad \ldots \quad \Gamma | \Gamma^{\ianew} \vdash_{s'} t_n : A_n
} {\Gamma | \Gamma^{\ianew} \vdash_{s'} s t_1 \ldots t_n : B} \ SC(\Gamma ; B)$$

\begin{center}\ovalbox{$
 \rulename{abs} \ \dps  \rulef{\dps \Gamma, x_1 : A_1, \ldots, x_n : A_n |
\Gamma^{\ianew}
  \vdash_{s'} s : B} {\Gamma | \Gamma^{\ianew} \vdash_{s'} \lambda x_1 \ldots x_n . s :
  (A_1, \ldots ,A_n,B)} \ SC(\Gamma ; (A_1, \ldots ,A_n,B) )$}\end{center}


{\bf Arithmetic and recursion}
$$ \rulename{const} \ \rulef{}{\emptyset | \emptyset \vdash_{s'} n :\texttt{exp}}
\quad \rulename{succ} \ \rulef{\Gamma | \Gamma^{\ianew}\vdash_{s'} M:\texttt{exp} }{\Gamma| \Gamma^{\ianew} \vdash_{s'} \texttt{succ}\ M:\texttt{exp}}
\quad \rulename{pred} \ \rulef{\Gamma| \Gamma^{\ianew} \vdash_{s'} M:\texttt{exp} }{\Gamma| \Gamma^{\ianew} \vdash_{s'} \texttt{pred}\ M:\texttt{exp}}$$

$$
\rulename{cond} \ \rulef{\Gamma| \Gamma^{\ianew} \vdash_{s'} M : \texttt{exp} \quad \Gamma| \Gamma^{\ianew} \vdash_{s'} N_1 : \texttt{exp} \quad \Gamma| \Gamma^{\ianew} \vdash_{s'} N_2 : \texttt{exp} }{\Gamma| \Gamma^{\ianew} \vdash_{s'} \texttt{cond}\ M\ N_1\ N_2}
\quad  \rulename{rec} \ \rulef{\Gamma| \Gamma^{\ianew} \vdash_{s'} M : A\rightarrow A }{ \Gamma| \Gamma^{\ianew} \vdash_{s'} Y_A M : A}$$

{\bf Imperative constructs}
$$ \rulename{seq} \ \rulef{\Gamma| \Gamma^{\ianew} \vdash_{s'} M : \texttt{com} \quad \Gamma| \Gamma^{\ianew} \vdash_{s'} N :A}
    {\Gamma| \Gamma^{\ianew} \vdash_{s'} \texttt{seq}_A \ M\ N\ : A} \quad A \in \{ \texttt{com}, \texttt{exp}\}$$

$$ \rulename{assign} \ \rulef{\Gamma| \Gamma^{\ianew} \vdash_{s'} M : \texttt{var} \quad \Gamma| \Gamma^{\ianew} \vdash_{s'} N : \texttt{exp}}
    {\Gamma| \Gamma^{\ianew} \vdash_{s'} \texttt{assign}\ M\ N\ : \texttt{com}}
\qquad
 \rulename{deref} \ \rulef{\Gamma| \Gamma^{\ianew} \vdash_{s'} M : \texttt{var}}
    {\Gamma| \Gamma^{\ianew} \vdash_{s'} \texttt{deref}\ M\ : \texttt{exp}}$$


\begin{center}\ovalbox{$
\rulename{new} \ \dps \rulef{\Gamma | \Gamma^{\ianew}, x :
\texttt{var} \vdash_{s'} M : A}
    {\Gamma | \Gamma^{\ianew} \vdash_{s'} \texttt{new } x \texttt{ in } M : A} \quad A \in \{ \texttt{com}, \texttt{exp}\} $}\end{center}

$$ \rulename{mkvar} \ \rulef{\Gamma| \Gamma^{\ianew} \vdash_{s'} M_1 : \texttt{exp} \rightarrow \texttt{com} \quad \Gamma| \Gamma^{\ianew} \vdash_{s'} M_2 : \texttt{exp}}
    {\Gamma| \Gamma^{\ianew} \vdash_{s'} \texttt{mkvar } M_1\ M_2\ : \texttt{var}}$$

\caption{Formation rules for Safe IA'}
\label{tab:safeia2_formrules}
\end{FramedTable}

 We define Safe IA' terms to be the subset of terms typable with
the system of rules of Table \ref{tab:safeia2_formrules} for which
the block-variable context is empty.
For convenience we introduce the additional rule
$$ \rulef{\Gamma | \emptyset \vdash_{s'} M : A} {\Gamma \vdash_{s'} M : A}$$
so that Safe IA' terms are exactly those denoted by judgments of the
form $\Gamma \vdash_{s'} M : A$.

\subsection{Examples}

Clearly Safe IA is a subset of Safe IA', the following example
showing that the inclusion is strict:

$$ \vdash_{s'} \lambda f^{(\iaexp \rightarrow \iacom) \rightarrow \iaexp} . \ianewin{i}\ f ( \lambda x . \iaassign\ i\ x ) : \iaexp$$
$$ \mbox{ but } \not \vdash_s \lambda f^{(\iaexp \rightarrow \iacom) \rightarrow \iaexp} . \ianewin{i}\ f ( \lambda x . \iaassign\ i\ x ) : \iaexp \ .$$

\notetoself{Need more examples}

\subsection{Game-semantic denotation}

In this section, our aim is to extend the game-semantic
characterization result of the safe lambda calculus to safe IA:
\begin{proposition}
\label{prop:safeia_closedpij}
  Let $\Gamma \vdash_{s'} M : A$ be as safe IA' term then $\sem{\Gamma \vdash_{s'} M :
  A}$ is closed P-incrementally justified.
\end{proposition}


Let us first introduce some definitions before proving this
proposition.

\begin{definition}
\label{def:pij_modulo} Let $\sigma$ be a strategy on some game $A$
and $\mathfrak{M}$ be as subset of moves of the arena $A$. We say
that $\sigma$ is P-i.j. modulo $\mathfrak{M}$ iff for all $s m \in
\sigma$ where $m \not\in \mathfrak{M}$ the play $s m$ is P-i.j.

Similarly we say that $\sigma$ is \emph{closed} P-i.j. modulo
$\mathfrak{M}$ iff for all $s m \in \sigma$ where $m \not\in
\mathfrak{M}$ the play $s m$ is \emph{closed} P-i.j.
\end{definition}
Hence a strategy is P-i.j. if and only if it is P-i.j. modulo
$\emptyset$. Instead of showing Proposition
\ref{prop:safeia_closedpij} we will prove the following more general
result:
\begin{proposition}
\label{prop:safeia_closedpijmodulo} Let $\Gamma | \Gamma^{\ianew}
\vdash_{s'} M : A $ be a Safe IA' term. Its denotation $\sem{\Gamma
| \Gamma^{\ianew} \vdash_{s'} M : A} = \sem{\Gamma, \Gamma^{\ianew}
\vdash M : A}$ is closed P-i.j. modulo
$\mathfrak{M}_{\Gamma^{\ianew}}$ where
$\mathfrak{M}_{\Gamma^{\ianew}}$ is the set of initial moves in
$\Gamma^{\ianew}$ (theses moves are of type $read$ or $write_i$ for
some $i\in \nat$).
\end{proposition}


\begin{lemma}
Let $\sigma : A \rightarrow B$ and $\mu : B \rightarrow C$.
  Let $\mathfrak{M}$ be any set of initial moves in $A$ of order $0$.
  If $\sigma$ is P-i.j (resp. closed P-i.j)  modulo $\mathfrak{M}$ and $\mu$ is P-i.j
  then $\sigma \fatsemi \mu$ is P-i.j (resp. closed P-i.j) modulo $\mathfrak{M}$.
\end{lemma}
\notetoself{
\begin{proof}
\end{proof}
}

Let $cell : I \rightarrow !\iavar$ denotes the ``storage cell''
strategy (see \cite{abramsky:game-semantics-tutorial}). One can
easily check that it is closed P-incrementally justified.
\begin{lemma}
Let $\sigma : C \otimes !\iavar \rightarrow B$ with $B \in \{\iacom,
\iaexp \}$ and $\mathfrak{M}$ be any set of initial moves in $C
\otimes !\iavar$ of order $0$.
  If $\sigma$ is P-i.j (resp. closed P-i.j) modulo $\mathfrak{M}$
  then $(id_C \otimes cell) \fatsemi \sigma$ is P-i.j (resp. closed P-i.j) modulo $\mathfrak{M} \inter C$.
\end{lemma}
\notetoself{
\begin{proof}
\end{proof}
}


\notetoself{
\begin{proof}[Proof of Prop. \ref{prop:safeia_closedpijmodulo}]
By induction on the structure of the term. Take a Safe IA' term
$\Gamma | \Gamma^\iavar \vdash M :A$.
\end{proof}
}

\bibliographystyle{plain}
\bibliography{../bib/higherorder,../bib/gamesem,../bib/lambdacalculus}

\end{document}
