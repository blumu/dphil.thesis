\chapter{First-Year work}

\section{Coursework}
I have attended the following courses: \emph{Automata Logic and
Games} in Hilary term 2005, \emph{Domain theory} in Michaelmas term
2005 and \emph{Categories Proofs and Programs} in Hilary term 2006.

\section{Teaching}

I was demonstrator for \emph{Network and Operating Systems}
practicals in Hilary term 2005, I tutored two groups of students for
the \emph{Introduction to Specification} classes (Hilary 2006). I
also did the marking for one group.

\section{Meetings and conferences}
\begin{itemize}
\item I attended Bonn spring school on GAMES in March 2005;

\item  I attended BCTCS (British Colloquium in
Theoretical Computer Science) in Nottingham in March 2005. There I
gave a talk about my MSc dissertation ``Termination analysis of a
subset of CoreML'';

\item I attended PAT \emph{Program transformation and Analysis} in Copenhagen, July 2005;

\item Marktoberdorf Summer School;
\item CSL (Computer Science Logic) August 2005:
I helped organizing the conference;
\item I visited the Isaac Newton Institute in Cambridge in February
2006.
\end{itemize}
I have also presented a project during the Computer Laboratory open
days.


\section{Summary of research activities}

\subsection{Game semantics}

During the past months, I have studied a restriction of
lambda-calculus called ``safe lambda-calculus''. \emph{Safety} is a
syntactic property originally defined by Knapik et al. in
\cite{KNU02} for higher-order recursion schemes (grammars). In their
paper they proved that the MSO theory of the term tree generated by
a safe recursion scheme of level $n$ is decidable. More recently,
Ong proved in \cite{OngLics2006} that the safety assumption is in
fact not necessary.

I am interested in the transposition of the safety property from
grammars to lambda terms. A definition of the safe
$\lambda$-calculus was first given in a technical report by Aehlig,
de Miranda and Ong in \cite{safety-mirlong2004}. One interesting
property of safe lambda terms is that performing substitution on
such terms does not involve renaming of the variable.

I have investigated different possible definitions of a safe lambda
calculus and I have proposed a more general notion of safety that do
not assume homogeneity of types while still preserving  the \emph{no
variable renaming} property.

I also tried to relate the safety restriction and the
\emph{size-change termination} property defined in
\cite{jones01,jones04}. Jones conjectured that any simply typed term
is size-change terminating, however Damien Sereni disproved this
conjecture by exhibiting a class of counter-examples
(\cite{serenistypesct05}). The terms of this class are safe simply
typed terms (not all of homogeneous type) and not size-change
terminating. This suggests that there is no real interesting
relation between safety and size-change termination.


Recently, inspired by my reading on game semantics
\citep{abramsky:game-semantics-tutorial} and by the technics
developed by Luke Ong in \citep{OngLics2006}, I have proved a result
on the game semantics of safe terms: the pointers in the game
semantics of safe simply type terms can be recovered uniquely from
the sequence of moves. This result is similar to the standard result
in game semantics which says that pointers of strategies can be
recovered uniquely for arena of order 2 at most. A consequence of
this result is that the semantics of such terms can be described by
(extended) regular expressions \cite{ghicamccusker00}.


\subsection{Verification}

In parallel, I worked on a separate project with Matthew Hagues and
Luke Ong. We developed a SAT-based  model checker for verifying
Linear Temporal Logic formulae (LTL) on programs expressed as finite
state machines. Our approach consists in combining techniques
presented in two papers: \cite{hammer:truly,
DBLP:conf/cav/McMillan03}.

In \cite{DBLP:conf/cav/McMillan03}, McMillan describes an
acceleration technique for the SAT-based Bounded Model Checking
problem based on Craig interpolants that improves the standard
SAT-based model checking methods for positive instances.

In \citep{hammer:truly}, Hammer et al. introduced a new kind of
automata called \emph{Linearly Weak Alternating Automata},
abbreviated LWAA. The set of languages recognized by these automata
are exactly the set of languages definable in LTL. There is a
straightforward translation from LTL formulae to LWAAs. The size of
the resulting automaton is linear in the size of the LTL formula.
Checking emptiness of LWAA then amounts to finding in the
configuration graph a lasso verifying certain conditions.

Our approach can be summarized as follows: we translate the model
checking problem into an emptiness checking of a \emph{Linear Weak
Alternating Automaton}, or LWAA for short: the automata is empty if
and only if the formula is true. The emptiness of the automaton is
expressed in term of reachability problem. As in the traditional
SAT-based bounded-model checking approach (\cite{biere99symbolic}),
we then construct a boolean formula that it is satisfiable if and
only if the desired configuration is reachable in at most $k$ steps
(i.e. there is a counter-example of length $k$ at most).

Then instead of using the traditional SAT-solver technique, that
iterates $k$ until the completeness threshold is reached, we use the
acceleration method described in \cite{DBLP:conf/cav/McMillan03}.
The principle is the following: for every iteration of $k$, if the
formula is not satisfiable then we perform some over-approximation
of the set of initial configuration. If after over-approximating the
initial configuration, the final configuration can be reached in $k$
steps then we are still not sure that the formula has a valid
counter-example since the counter-example obtained may be spuriously
created by the over-approximation: hence we increase $k$ and move on
to the next iteration. However, if after performing several
over-approximation we reach a fixed point and the formula is still
not satisfiable (not counter-example of length $k$ at most) then we
know that there cannot be any counter-example of any length : we
have therefore reached the completeness threshold and we know that
the formula is true.


There are two reasons why we think that our approach may lead to a
gain of performance. First, determining emptiness of a LWAA is more
costly than determining emptiness of a B\"uchi automaton however we
save time during the construction of the automaton since the size of
a LWAA is linear in the length of the formula as opposed to the
standard translation which produces a B\"uchi automaton of size
exponential in the length of the formula. Secondly, in the case when
there is no counter-example, McMillan's acceleration method based on
over-approximation permits to detect quickly if we have reached the
completeness threshold.


%\cite{ckos2005}
We have produced an experimental implementation in OCaml and C. The
program parses a file in the NuSMV format (\cite{CAV02:nusmv})
containing the kripke structure of the model and the set of LTL
properties to verify. Our tools can be interfaced with two SAT
solvers: ZChaff \citep{zChaff} and MiniSat \citep{ES03}. We also use
BDD to perform simplification on the propositional formula and to
generate the CNF representation that the SAT solver takes as its
input.

Compared to the LWAASpin LTL model checker (\cite{hammer:truly}),
our tool performs quite poorly. As soon as a model is taken into
account, our procedure generate increasingly bigger propositional
formulae that the SAT solver struggle to solve. However for pure LTL
emptiness checking, our tool performs quite well.

It seems disappointing that our approach does not give good result
for model checking, however the reason seems to be that the
SAT-solvers we are using produce bad interpolants. In the future we
would like to interface our model checking tool with other SAT
solvers and interpolers.

There are also some optimizations that we have not finished to
implement. For instance, there are different ways of encoding the
bounded model checking problem into a propositional formula. We are
going to do some experiments do determine which encoding gives the
best result.

\chapter{Research plan}

My research plan for the coming year is as follows: first I will
continue to work on the Safe $\lambda$-calculus. My immediate goal
is to extend the result about game semantics of Safe simply-typed
$\lambda$-calculus to the case of Safe Idealized Algol and
investigate applications in algorithmic game semantics. There are
also further questions about Safe $\lambda$-calculus that I need to
look at: what is the categorical interpretation of Safe
$\lambda$-calculus? What kind of proof theory do we obtain by the
Curry-Howard isomorphism? Which complexity class is characterized by
the Safe-$\lambda$ calculus?

In parallel to that line of research, I will continue to work with
Matthew Hagues and Luke Ong on the LTL model checking problem.


%I also want to investigate some application of game semantics to
%program analysis and transformation by trying to extend the work of
%Dimovski et al. (\cite{DBLP:conf/sas/DimovskiGL05}) on
%data-absraction refinement based on game semantics.
