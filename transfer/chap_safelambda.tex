\chapter{Safe $\lambda$-calculus}

We recall the definition of the safe $\lambda$-calculus given in
\cite{Ong2005}.

\input safelmd.tex


\begin{lem}[Basic properties]
\label{lem:safe_basic_prop} Suppose $\Gamma \vdash_s M : B$ is a
valid judgment then

\begin{itemize}
\item $B$ is homogeneous
\item Every free variables of $M$ has order at least $ord(M)$
\end{itemize}
\end{lem}


\subsection{Simultaneous substitution}

The first interesting property that we will prove for the safe
$\lambda$-calculus is that when performing substitution on safe
$\lambda$ term, there is no need to rename bound variables provided
that the substitution is performed \emph{simultaneously} on all free
variables of the same order.

The requirement that the substitution is performed simultaneously is
quite strong: implementing simultaneously substitution requires to
have access to an unbound number of fresh variables. Therefore in
safe lambda calculus the fact that there is no variable capture
during substitution does not really lead to a complete economy of
variable names.


\begin{dfn}[Simultaneous substitution]
Substitution for simply typed lambda term is defined as follow:
\begin{eqnarray*}
x \subst{t}{x} &=& t\\
 y\subst{t}{x} &=& y \quad \mbox{for } x \not \neq y,\\
(M_1 M_2) \subst{t}{y} &=& (M_1 \subst{t}{y}) (M_2 \subst{t}{y})\\
(\lambda x . M) \subst{t}{y} &=& \lambda z . M \subst{z}{x}
\subst{t}{y} \mbox{where $z$ is a fresh variable}
\end{eqnarray*}

Simultaneous substitution is defined as follow:

$$M\subst{N_1 \ldots N_n}{x_1 \ldots x_n} = M \subst{z_2}{x_2} \ldots \subst{z_n}{x_n} \subst{N_1}{x_1} \subst{N_2}{z_2} \ldots \subst{N_n}{z_n} $$
where $z_2 \ldots z_n$ are fresh variables.


In presence of constant symbols, (this is the case in the safe
lambda calculus), we add the following definition:
$$f \subst{t}{x} = f \quad \mbox{ where $f \in \Sigma$ is a first-order constant}$$

\end{dfn}

In fact, we can define the simultaneous substitution inductively
without relying on the definition of the standard substitution. Here
is the definition specialized to the safe lambda calculus case:

\begin{dfn}[Simultaneous substitution in the safe-lambda calculus]
\label{dnf:safe_simsubst}
 We use the notation
$\subst{\overline{N}}{\overline{x}}$ for $\subst{N_1 \ldots N_n}{x_1
\ldots x_n}$:
\begin{eqnarray*}
f \subst{\overline{N}}{\overline{x}} &=& f \quad \mbox{ where $f\in\Sigma$ is a first-order constant} \\
x_i \subst{\overline{N}}{\overline{x}} &=& N_i\\
 y \subst{\overline{N}}{\overline{x}} &=& y \quad \mbox{ if } y \not \neq x_i \mbox{ for all } i,\\
(M N_1 \ldots N_l) \subst{\overline{N}}{\overline{x}} &=& (M \subst{\overline{N}}{\overline{x}}) (N_1 \subst{\overline{N}}{\overline{x}}) \ldots  (N_l \subst{\overline{N}}{\overline{x}})\\
(\lambda \overline{y} : \overline{A}. T)
\subst{\overline{N}}{\overline{x}} &=& \lambda \overline{z} . T
\subst{\overline{z}}{\overline{y}}
\subst{\overline{N}}{\overline{x}} \\
&& \mbox{where $T$ is a safe
term and $\overline{z} = z_1, \ldots z_p$ are all fresh variables}
\end{eqnarray*}
\end{dfn}

This alternative definition permits us to observe the following two
properties:

\begin{property}[Simultaneous substitution on safe terms]
\label{prop:subst_preserve_safety} \
\begin{enumerate}
\item Performing simultaneous substitution on a safe term can be
achieved by inductively applying the simultaneous substitution on
other \emph{safe} sub-terms only.

\item Simultaneous substitution of safe terms preserves safety.
\end{enumerate}
\end{property}

\begin{proof}
\begin{enumerate}
\item By analysing the inductive definition \ref{dnf:safe_simsubst}, we
observe that each substitution is performed on a safe term provided
that the original term is safe. For the abstraction case, we remark
that the substitution $\subst{\overline{z}}{\overline{y}}$ is just a
renaming of variable that preserve safety.

\item Consider the safe terms $\Gamma \vdash_s S : A$ and  $\Gamma \vdash_s N_i : B_i$ for $i = 1..n$.

We prove that $S \subst{\overline{x}}{\overline{N}}$ is safe by
induction on the size of the proof tree of $\Gamma \vdash_s S : A$.
We just give the detail for the abstraction case:

Assume that we proved the property for all term whose proof tree is
smaller than $S$. Suppose $S = \lambda \overline{y} : \overline{A}.
T$ where $T$ is a safe term, then $T
\subst{\overline{z}}{\overline{y}}$ is just the term $T$ with its
variable $\overline{y}$ renamed to fresh names therefore it is safe.
By the induction hypothesis,
$T\subst{\overline{z}}{\overline{y}}\subst{\overline{N}}{\overline{x}}$
is also safe. We can apply the rule (abs) of the safe-lambda
calculus and we get that $\lambda \overline{z} . T
\subst{\overline{z}}{\overline{y}}
\subst{\overline{N}}{\overline{x}}$ is safe.
\end{enumerate}
\end{proof}


\subsection{Simultaneous substitution does not involve renaming}



\begin{lem}[No variable clash lemma]
In the safe $\lambda$-calculus, there is no clash of variable name
when performing substitution:
\[ \qquad M[N_1 / x_1 , \cdots, N_n / x_n] \]
 provided the substitution is performed simultaneously on
\emph{all} free variables of the same order in $M$
i.e.~$\makeset{x_1, \cdots, x_n}$ is the set variables of the same
order as $x_1$ that occur free in $M$.
\end{lem}

\begin{proof}
First we note that if the substitution is not simultaneous ($M
\subst{N_1}{x_1}\ldots \subst{N_n}{x_n}$), then a variable capture
arises if some $N_i$ has a free occurrence of a variable $x_j$ with
$j>i$. However this capture does not happen when performing the
substitutions simultaneously as follow: $M \subst{N_1, \ldots
N_n}{x_1, \ldots x_n}$.



Suppose that a variable capture occurs in the term $M$: $M$ has a
subterm $\lambda y_1 \ldots y_p. T$ such that some $x_i$ appears
freely in $T$ and some $y_k$ appears freely in $N_i$. By Property
\ref{prop:subst_preserve_safety}, we can assume that the subterm
$\lambda y_1 \ldots y_p . T$ is safe.

Since $x_i$ appears freely in the safe term $\lambda y_1 \ldots y_p
. T$, by Lemma \ref{lem:safe_basic_prop} (ii) we get:
$$ ord(x_i) \geq ord(\lambda y_1 \ldots y_p . T) \geq 1+ ord(y_k) > ord(y_k)$$

Since $y_k$ appears freely in the safe term $N_i$, Lemma
\ref{lem:safe_basic_prop} (ii) gives:

$$ ord(y_k) \geq ord(N_i) = ord(x_i)$$

Hence we reach a contradiction.
\end{proof}


\subsection{Simultaneous $\beta$ reduction}

We now define a notion of beta reduction that realizes simultaneous
substitution. Consider a simply-typed term $P$. A simultaneous
$\beta$-redex is a $P$ sub-term of the kind

$$R_1 \equiv (\lambda x_1 x_2 \ldots x_n . M) N_1 N_2 \ldots N_n$$

Reduction is only performed if the simultaneous $\beta$-redex
encompasses as many lambda abstraction of the same order as
possible. Such a redex (which cannot be extended to take into
account one more lambda abstraction of the same order) is called a
$\beta_s$-redex.


Example: consider a term $P$ with a subterm $((\lambda x_1 x_2
\ldots x_n . M) N_1 N_2 \ldots N_n) N_{n+1}$. Suppose that $M$ is
the abstraction $M \equiv \lambda x_{n+1} . U$ where $ord(x_{n+1}) =
x_1$. Then the redex $R_1$ will not be considered since it can be
enlarged as the redex $(\lambda x_1 x_2 \ldots x_n x_{n+1}. M) N_1
N_2 \ldots N_n N_{n+1}$. Now suppose instead that the term is formed
in such a way that there is no $N_{n+1}$ applied on the right of
$R_1$ then the redex $R_1$ will be considered (whether or not $M$ is
an abstraction).

We now give the formal definitions:

The following abbreviation are used $\overline{x} = x_1 \ldots x_n$,
$\overline{N} = N_1 \ldots N_n$, $\overline{x_l} = x_1 \ldots x_l$,
$\overline{x_r} = x_{l+1} \ldots x_n$, $\overline{N_l} = N_1 \ldots
N_l$ and $\lambda \overline{x} : \overline{A} . T = \lambda
x_1^{A_1} \ldots x_1^{A_n} . T$.

\begin{dfn}[$\beta_s$-redex]
A safe simply typed lambda term is a redex if it has one of the
following forms:
\begin{itemize}
\item $(\lambda \overline{x} : \overline{A} . T) \overline{N}$
\quad with $|\overline{x}| = |\overline{N}| = n$, $ord(T) \leq
ord(\overline{x}) = ord(x_1) = \ldots = ord(x_n)$.

\item $(\lambda \overline{x_l} : \overline{A_l} \ \overline{x_r}: \overline{A_r} . T) \overline{N_l}$
\quad with $|\overline{x_l}| = |\overline{N_l}| = l$, $ord(T) \leq
ord(\overline{x}) = ord(x_1) = \ldots = ord(x_n)$.
\end{itemize}

These two cases correspond respectively to the formation rules (App)
and (App+) of the safe lambda calculus.

\end{dfn}

\begin{dfn}[Simultaneous $\beta$-reduction] \
\begin{itemize}
\item The relation $\beta_s$ is defined on the set of $\beta_s$-redex.
\begin{eqnarray*}
\beta_s &=&
 \{  \left( (\lambda \overline{x} : \overline{A} . T) \overline{N}, T\subst{\overline{x}}{\overline{N}} \right) \\
&& \mbox{ where }
     |\overline{x}| = |\overline{N}| = n \mbox{ and } ord(T) \leq ord(\overline{x}) = ord(x_1) = \ldots = ord(x_n)
\} \\
 && \union \quad \\
&& \{
    \left( (\lambda \overline{x_l} : \overline{A_l} \  \overline{x_r}: \overline{A_r} . T) \overline{N_l}, \lambda \overline{x_r}: \overline{A_r} . T\subst{\overline{x_l}}{\overline{N_l}} \right) \\
&& \mbox{ where }
 |\overline{x}| = |\overline{N}| = n \mbox{ and } ord(T) \leq ord(\overline{x}) = ord(x_1) = \ldots = ord(x_n)
\}
\end{eqnarray*}

Note that in the second case, the substitution is done under the
$\lambda \overline{x_r}$. The side condition of the formation rule
(App+) guarantees that there will not be any variable capture.

\item
The simultaneous $\beta$-reduction noted $\betasred$ is the closure
of the relation $\beta_s$ by compatibility with the formation rules
of the safe $\lambda$-calculus.
\end{itemize}
\end{dfn}

Note that $\beta_s$-redex are the only redex that can be reduced by
$\betasred$.




\subsection{Some properties of $\beta_s$ reduction}

We remark that $\betasred \subset \betaredtr$ (i.e. the simultaneous
$\beta$-reduction relation) is included in the transitive closure of
the $\beta$-reduction relation. More precisely, if $M \betasred N$
then $M \betaredtr N$. Simultaneous $\beta$-reduction is a certain
kind of multi-steps $\beta$-reduction.

\begin{lem} In the simply typed $\lambda$-calculus setting:
\begin{enumerate}
\item $\betasred$ is strongly normalizing.
\item $\beta_s$ has the unique normal form property.
\item $\beta_s$ has the Church-Rosser property.
\end{enumerate}
\end{lem}

\begin{proof}
\begin{enumerate}

%Proof of weak normalization:
%We know that $\betared$ is strongly normalizing: any reduction
%strategy leads to a normal form. In particular any strategy that
%performs consecutive $\beta$-redex of the same order consecutively
%will lead to a normal form. Since any such $\beta$ reduction
%strategy is also a $\beta_s$ reduction strategy, we can conclude
%that $\betasred$ is weakly normalizing.

\item This is because $\betasred \subset \betaredtr$ and $\betared$ is strongly normalizing (in the simply typed lambda calculus).

\item
A term has a $\beta_s$-redex iff it has a $\beta$-redex therefore
the set of $\beta_s$ normal form is equal to the set of $\beta_s$
normal form. Hence, the unicity of $\beta$ normal form implies the
unicity of $\beta_s$ normal form.

\item is a consequence of (i) and (ii).
\end{enumerate}
\end{proof}


%%
%%\begin{dfn}[Safe sub-terms]
%%Given a safe term $\Gamma \vdash_s M : A$, we define the set
%%$Safesub(M)$ of safe sub-terms of a $M$:
%%
%%\def\longtype#1#2{(\overline{{#1}_1} \, | \, \cdots \, | \, \overline{{#1}_{#2}} \, | \, o)}
%%
%%\begin{eqnarray*}
%%Safesub(b) &=& b  \qquad \mbox{ ($b \in \Sigma$ a first order constant)} \\
%%Safesub(x) &=& x \qquad \mbox{ ($x$ a variable) } \\
%%Safesub(M N_1 \ldots N_l) &=& \{M N_1 \ldots N_l \} \union Safesub(M) \union \Union_{i=1..l} Safesub(N_i) \\
%%&&\mbox{ where $M : \longtype{B}{n}$ and $ord(N_1) = \ldots =
%%ord(N_l) = ord(\overline{B})$}.\\
%%Safesub(\lambda x_1 \ldots x_n . N) &=& \{ \lambda x_1 \ldots x_n . N \} \union Safesub(R) \\
%%&&\mbox{ where $ord(x_1) = \ldots = ord(x_n)$ and $ord(x_1) \geq ord(N)$}\\
%%&&\mbox{ and $R = N$ if $x_1 \in fv(M)$ and $R = \lambda x_2 \ldots x_n . N$ otherwise.}\\
%%\end{eqnarray*}
%%
%%\end{dfn}
%%
%%It is easy to check that $Safesub(M)$ is indeed the set of all safe
%%sub-terms of $M$.



\begin{lem}
$\beta_s$-reduction preserves safety. (i.e. $M$ safe term and $M
\beta_s N$ implies $N$ safe)
\end{lem}


\begin{proof}
Simultaneous substitution preserves safety (property
\ref{prop:subst_preserve_safety}), therefore we just need to prove
that the relation $\beta_s$ preserves safety and the result will
follow:

 Suppose $s\ \beta_s\ t$ then $s$ is a $\beta_s$-redex. There are two kinds of them
 depending on which rule has been used last to form the redex.

\begin{itemize}
\item Suppose the last rules used is (App), then the redex is
$$s \equiv (\lambda x_1 \ldots x_n . M) N_1 \ldots N_n \qquad \betasred \qquad M[N_1 / x_1 , \cdots, N_n / x_n] \equiv t$$
where $ord(M) \leq ord(x_1) = \ldots = ord(x_n)$

The first premise of the rule (App) tells us that $M$ is safe,
therefore since substitution preserves safety, (property
\ref{prop:subst_preserve_safety}), $t$ is safe.

\item Suppose the last rules used is (App+), then the redex is

 $$
s \equiv  (\lambda \overline{x_l} : \overline{A_l} \
\overline{x_r}: \overline{A_r} . T) \overline{N_l} \qquad \betasred
\qquad \lambda \overline{x_r}: \overline{A_r} .
T\subst{\overline{x_l}}{\overline{N_l}} \equiv  t
$$
where $ord(T) \leq ord(x_1) = \ldots = ord(x_n)$

$T\subst{\overline{x_l}}{\overline{N_l}}$ is safe for the same
reason as in the first case. We can then apply the rule (Abs) and
that prove the safety of $t$.
\end{itemize}
\end{proof}



\begin{rem}
\label{rem:betasred_notpreserv_unsafety} While $\betasred$ preserves
safety it does not however preserves un-safety: given two terms of
the same type, one safe $\Gamma \vdash_s S : A$ and the other unsafe
$\Gamma \vdash U : A$, the term $(\lambda x y . y) U S$ is unsafe
but it $\beta_s$-reduces to $S$ which is safe.
\end{rem}

\subsection{Pointer-less strategies}
\label{subsec:ptrless_strat}

Up to order 2, the semantics of PCF terms is entirely defined by
pointer-less strategies. In other words, the pointers can be
uniquely reconstructed from any non justified sequence of moves
satisfying the visibility and well-bracketing condition.

At level 3 however, pointers cannot be omitted. There is an example
in \cite{abramsky:game-semantics} to illustrate this. Consider the
following two terms of type $((\nat \typar \nat) \typar \nat) \typar
\nat$:

$$M_1 = \lambda f . f (\lambda x . f (\lambda y .y ))$$
$$M_2 = \lambda f . f (\lambda x . f (\lambda y .x ))$$

We assign tags to the types in order to identify in which arena the
questions are asked: $((\nat^1 \typar \nat^2) \typar \nat^3) \typar
\nat^4$. Consider now the following pointer-less sequence of moves
$s = q^4 q^3 q^2 q^3 q^2 q^1$. It is possible to retrieve the
pointers of the first five moves but there is an ambiguity for the
last move: does it point to the first or second occurrence of $q^3$
in the sequence $s$?

Note that the visibility condition does not eliminate the ambiguity,
since the two occurrences of $q^3$ both appear in the P-view at that
point (after recovering the pointers of $s$ up to the second last
move we get $s = \xymatrix @=12pt@M=0pt{ q^4 & q^3 \ar@/_/[l] & q^2
\ar@/_/[l] & q^3 \ar@/_/[ll] & q^2 \ar@/_/[l] & q^1 }$ , therefore
the P-view of $s$ is $s$ itself.)


In fact these two different possibilities correspond to two
different strategies. Suppose that the link goes to the first
occurrence of $q^3$ then it means that the proponent is requesting
the value of the variable $x$ bound in the subterm $\lambda x . f (
\lambda y. ... )$. If P needs to know the value of $x$, this is
because P is in fact following the strategy of the subterm $\lambda
y . x$. And the entire play is part of the strategy $\sem{M_2}$.

Similarly, if the link points to the second occurrence of $q^3$ then
the play belongs to the strategy $\sem{M_1}$.

\subsection{Game semantics of safe $\lambda$ terms}

We would like to find out whether the safety condition defined in
\cite{Ong2005} leads to a pointer economy in the corresponding game
semantics.

The example of section \ref{subsec:ptrless_strat} is a good example
to start with. We observe that for this particular example and in
the safe $\lambda$-calculus setting, the ambiguity that led us to
the addition of pointers to strategies disappear. More precisely,
$M_1$ is a safe term whereas $M_2$ is not. Indeed, there is a free
occurrence of the variable $x$ of type $o$ in the subterm $f
(\lambda y . x)$ which is not abstracted together with $y$ of type
$o$.


\begin{enumerate}
\item
Is it the case that in general, the pointers from the semantics of
safe $\lambda$-terms can be reconstructed uniquely from the moves of
the play?


\item
Is there any unsafe term whose game semantics is a strategy where
pointers can be recovered?

The answer is yes: take the term $T_i = (\lambda x y . y) M_i S$
where $i =1..2$ and $\Gamma \vdash_s S : A$. $T_1$ and $T_2$ both
$\beta$-reduce to the safe term $S$, therefore
$\sem{T_1}=\sem{T_2}=\sem{S}$. But $T_1$ is safe whereas $T_2$ is
unsafe. Since it is possible to recover the pointer from the game
semantics of $S$, it is as well possible to recover the pointer from
the semantics of $T_2$ which is unsafe.

\item
Is there any unsafe $\beta$-normal form whose game semantics is a
strategy where pointers can be recovered?


\end{enumerate}


\subsection{$\eta$-extension}

Let $\eta$-normal form of a term is the term obtained after
hereditarily $\eta$-expanding every subterm.

\def\etanfaux#1{\lceil#1 \rceil}
\def\etanf#1{\eta-nf\left( #1 \right)}

\begin{eqnarray*}
%\etanf{\lambda x . M} = \lambda x . \etanf{M}\\
\end{eqnarray*}

\subsection{Pointers in the game semantics of safe terms are recoverable}

We claim that the pointers in the game semantics of a safe term are
uniquely recoverable.

Consider a term $M$ safe, we can assume that $M$ is in $\eta$ normal
form (provided that safety is preserved by $\eta$-expansion.

The term can be represented by a computation tree: nodes at even
depth (starting at level 0) correspond to $\lambda$ and nodes at odd
length corresponds to either application $@$, variable $x$ or
variable followed by an application $f@$. A $\lambda$ node
represented consecutive abstraction of variables.

There justification pointers going upward from variable occurrences
to their bindings.

In the game semantics of the term $M$, the pointers for O and P
answers can be recovered by using the well-bracketing condition.

For O-question, the justification pointer always points to its
parent node in the computation tree.

For P-question, suppose P ask for the value of variable $x$. Then
there may be several choices for the destination of the pointer but
we claim that in the case of safe terms, it should point to the
closest parent node (in the path from the root to P-question) whose
order is greater than the order of $x$.

\subsection{Safe lambda calculus without homogeneous types}
\section{Safe lambda calculus without homogeneous types}

We use a set of sequents of the form $\Gamma \vdash^{i} M : A$ where
the meaning is ``variables in $\Gamma$ have orders at least
$\ord{A}+i$'' where $i \in \zset$. The following set of rules are
defined for $i \in \zset$:

$$ \mathbf{(seq^i_\delta)} \quad \rulef{\Gamma \vdash^{i} M : A}{\Gamma \vdash^{i-\delta} M : A} \quad i \in \zset, \delta > 0  $$

$$ \mathbf{(var)} \quad  \rulef{}{x : A\vdash^{0} x : A} $$

$$ \mathbf{(wk^i)} \quad  \rulef{\Gamma \vdash^i M : A}{\Gamma , x : B \vdash^i M : A} \quad \ord{B} \geq \ord{A} + i $$

$$ \mathbf{(app^i)} \quad  \rulef{\Gamma \vdash^i M : A\rightarrow B
                                        \qquad \Gamma \vdash^{0} N : A}
                                   {\Gamma  \vdash^{i+\delta} M N : B}
                                    \qquad
                                   \delta = \max\left(0, 1 + \ord{A} - \ord{B}\right)$$

$$ \mathbf{(abs^i)} \quad  \rulef{\Gamma, \overline{x} : \overline{A} \vdash^{i} M : B}
                                   {\Gamma  \vdash^{0} \lambda \overline{x} : \overline{A} . M : (\overline{A},B)} \qquad
                                   \forall y \in \Gamma : \ord{y} \geq \ord{\overline{A},B}$$


Note that:
\begin{itemize}
\item $(\overline{A},B)$ denotes the type $(A_1,A_2, \ldots, A_n, B)$;
\item all the types appearing in the rule are not required to be homogeneous. For instance in the rule $\mathbf{(app^i)}$, the type $A \rightarrow B$ is not necessarily
homogeneous;
\item the environment $\Gamma, \overline{x}$ is not stratified. In particular, variables in $\overline{x}$ do not necessarily have the same order. Also
there may be variable in $\Gamma$ of order smaller than $\ord{x}$
for some variable $x$ in $\overline{x}$.
\item The sequents that we really want to prove are those of type $\Gamma \vdash^0 M$. Those terms are the safe terms.
Other terms are only used as intermediate steps in a proof.
\end{itemize}

\begin{rem}
\label{rem:rulesineg}

This set of rules is equivalent (in term of safe terms that can be
generated) to the same set of rules where $i$ is restricted to be a
negative integer and where the rule $(app^i)$ becomes:

$$ \mathbf{(app^i)} \quad  \rulef{\Gamma \vdash^i M : A\rightarrow B
                                        \qquad \Gamma \vdash^{0} N : A}
                                   {\Gamma  \vdash^{\min(i+\delta,0)} M N : B}
                                    \qquad
                                   \delta = \max\left(0, 1 + \ord{A} - \ord{B}\right) \quad i \leq 0 $$

With this new set of rules, the sequents of the form $\Gamma
\vdash^{k} M$ with $k>0$ cannot be derived anymore, however, the set
of safe terms that can built remain the same. Indeed, suppose that
we derive $\Gamma \vdash^0 M$ using the sequent $\Gamma \vdash^k N$
with $k>0$ somewhere in the proof. Then an easy induction shows that
the sequent $\Gamma \vdash^0 N$ can as well be derived by making use
of the rule $(seq^i_\delta)$ for $i\leq 0$.
\end{rem}

\begin{lem}[Basic properties]
\label{lem:nonhomosafe_basic_prop} Suppose $\Gamma \vdash^0 M : B$
is a valid judgment then every variable in $\Gamma$ has order at
least $ord(M)$.
\end{lem}
\begin{proof}
An easy induction on the proof tree shows that if $\Gamma \vdash^{i}
M : A$ then the variables in $\Gamma$ have orders at least
$\ord{A}+i$. The induction step for the application is: suppose
$\Gamma  \vdash^{i+\delta} M N : B$ where $\Gamma \vdash^i M :
A\rightarrow B$. Then by induction we have $\forall y \in \Gamma :
\ord{y} \geq \ord{A\rightarrow B} + i = \max(1+\ord{A}, \ord{B})+i =
\delta + \ord{B} + i$.
\end{proof}

\begin{lem}[No variable capture lemma]
Provided that substitution is done simultaneously (even for variable
of different order), there is not variable capture when performing
substitution on a safe (non homogeneous) term.
\end{lem}

\begin{proof}
Suppose that a capture occurs during the substitution $M[N/\varphi]$
where $M$ and $N$ are safe. Then the following conditions must hold:
\begin{enumerate}
\item $\varphi:A, \Gamma \vdash^0 M$,
\item $\Gamma \vdash^0 N$,
\item there is a subterm $\lambda \overline{x} . L$ in $M$ (where the abstraction is taken as wide as possible) such that:
\item $\varphi \in fv(\lambda \overline{x} . L)$ (and therefore $\varphi \in fv(L)$),
\item $x \in fv(N)$ for some $x \in \overline{x}$.
\end{enumerate}

By lemma \ref{lem:nonhomosafe_basic_prop} and (v) we have:

\begin{equation}
\ord{x} \geq \ord{N} = \ord{\varphi} \label{eq:xigeqphi}
\end{equation}

$\lambda \overline{x} . L$ is a subterm of $M$, therefore (since the
abstraction $\lambda \overline{x}.L$ is taken as large as possible)
there is a node $\Sigma \vdash^u \lambda \overline{x} . L$ in the
proof tree for some $u$.

There are only three kind of rules that can derive an abstraction:
$\mathbf{(abs^i)}$, $\mathbf{(seq^i_\delta)}$ and $\mathbf{(wk^i)}$.
The only rule that can introduce the abstraction is
$\mathbf{(abs^i)}$. Therefore the proof tree has the following form:

$$ \rulef{
    \rulef{
        \rulef{
            \rulef  {\ldots}
                   {\Sigma' \vdash^0 \lambda \overline{x} . L} \mathbf{(abs^i)}
        }
        {\ldots} r_1
    }
    {\vdots} r_2
    }
    { \Sigma \vdash^u \lambda \overline{x} . L } r_l
    \qquad \mbox{where } r_j \in \{ \mathbf{(seq^i_\delta)},\ \mathbf{(wk^i)}\ |\ i \in \zset, \delta > 0 \},
            \quad j\in 1..l.
$$


Since $\varphi \in fv (L)$ we must have $\varphi \in \Sigma'$ and
since $\Sigma' \vdash^0 \lambda \overline{x} . L$, by lemma
\ref{lem:nonhomosafe_basic_prop} we have:

$$\ord{\varphi} \geq \ord{\lambda \overline{x} . L} \geq 1 + \ord{x} > \ord{x}$$

which contradicts equation (\ref{eq:xigeqphi}).
\end{proof}


\section{Particular case of homogeneously-safe lambda terms}

We look at a particular sub-class of lambda terms. The types of
these terms respect a property call homogeneity as defined in
section \ref{sec:homotypes}. A type $(A_1, A_2, \ldots A_n, o)$ is
said to be homogeneous whenever $\order{A_1} \geq \order{A_2} \geq
\ldots \geq  \order{A_n}$ and each of the $A_i$ are homogeneous. A
term is homogeneous if its type is homogeneous.


In their definition of safety (\cite{KNU02}), Knapik et al. require
that all the recursion equations of a safe recursion scheme have a
homogeneous type.

In the rules defining safety for the non-homogeneous case, the only
rule that can potentially introduce a non-homogeneous term from a
homogeneous one is the abstraction rule. But such a term (a lambda
abstraction) corresponds exactly to a recursion equation in the
recursion scheme setting of Knapik et al. Therefore requiring that
recursions equation have homogeneous type is the same as requiring
that all sequents appearing in the proof tree of a safe term are of
homogeneous type.

We say that a term is homogeneously-safe if its type is homogeneous
and there is a proof tree showing its safety where all the sequents
of the proof tree are of homogenous type!

\begin{lem}
\label{lem:homog_judg_zero_minusone} If a term is homogeneously-safe
then there is valid proof tree showing that it is safe containing
only judgments of the form $\Gamma \vdash^{k} M : T$ with $k\in
\{-1,0\}$.
\end{lem}

\begin{proof}
Assume that $\Gamma \vdash^{0} S : T_S$ with $T_S$ homogeneous.


Because of remark \ref{rem:rulesineg} we just need to show that
there is a proof tree where there is no sequent of the form $\Gamma
\vdash^{k} M$ with $k<-1$.

Suppose that the proof tree of $\Gamma \vdash^{0} S : T_S$ contains
$\Gamma \vdash^{-k} M : T$ for $k>0$ and $T$ a homogeneous type.

The term $M$ is unsafe but we hope that eventually we will form a
safe term with it. Since $M$ is unsafe, its order must be strictly
greater than $1$: we assume that $T = \overline{A} | B$. The
homogeneity of $\overline{A} | B$ implies $ord(M) = 1 +
ord(\overline{A})$.

We observe that the only two possible ways to make a safe term is to
use the rule $(app^i)$ or $(abs^i)$ for some $i$ (they are the only
rules which can decrease $k$):

\begin{itemize}
\item
 Suppose that we want to form a homogeneously-safe term by abstracting a variable. Respecting type homogeneity
 requires $ord(x) \geq ord(A)$.

Then it is easy to see that the sequent $\Gamma \vdash^{-k} M : A
\rightarrow B$ was too strong and that we could have derived the
sequent $\Gamma \vdash^0 M : A \rightarrow B$ instead!

\item
Suppose that we want to form a safe term by applying another term
safe term $\Gamma \vdash^0 N : A$ to $\Gamma \vdash^{-k} M : A
\rightarrow B$ (that way the unsafe term $M$ does not appear at an
operand position).

Using the application rules once may not be enough to get a safe
term, it may be necessary to perform several consecutive
applications until the order of the term becomes low enough. We now
consider the very last such application, the one that turns the non
safe term into a safe one. This consideration allows us to assume
that in the type $A \rightarrow B$, $A$ is the last type of its
partition, i.e. $\ord{A} \geq \ord{B}$ and $\ord{M} = 1 + \ord{A}$.

We observe that in the rule $(app^{-i})$, the environments  of the
two premises ($\Gamma$) are the same. The second premise is $\Gamma
\vdash^{0} N : A$ therefore by lemma
\ref{lem:nonhomosafe_basic_prop} we have:

\begin{equation}
\forall x \in \Gamma : \ord{x} \geq \ord{N} = \ord{A} = \ord{M} - 1
\end{equation}

Again the sequent $\Gamma \vdash^{-k} M : A \rightarrow B$ was too
strong and we could have derived the sequent $\Gamma \vdash^{-1} M :
A \rightarrow B$ instead!

\end{itemize}
\end{proof}

From this lemma we can derive rules for the homogeneously-safe
lambda calculus.

\subsection{The example of the application rule}

We are now about to derive the application rules specialized for the
case of homogeneous types. We recall the rule $\mathbf{(app^i)}$:
$$ \mathbf{(app^i)} \quad  \rulef{\Gamma \vdash^i M : A\rightarrow B
                                        \qquad \Gamma \vdash^{0} N : A}
                                   {\Gamma  \vdash^{u} M N : B}
                                    \qquad
                                   u = \min(i+\max\left(0, 1 + \ord{A} - \ord{B}\right),0) \quad i \in \{ -1, 0 \} $$

Type homogeneity implies that $\ord{A} \geq \ord{B} - 1$.

\begin{itemize}
\item Suppose that $\ord{A} \geq \ord{B}$ then the condition $i \in \{-1, 0\}$ implies $u=0$ and we obtain the following rule:
$$ \mathbf{(app_1^i)} \quad  \rulef{\Gamma \vdash^i M : A\rightarrow B
                                        \qquad \Gamma \vdash^{0} N : A }
                                   {\Gamma  \vdash^{0} M N : B}
                                    \qquad \ord{A} \geq \ord{B} ,\quad i \in \{ -1, 0 \} $$

\item Suppose that $\ord{A} = \ord{B} - 1$ then
$ u = \min(i,0) = i$  (since $i \in \{-1,0\}$).
 We obtain the following rule:
$$ \mathbf{(app_2^i)} \quad  \rulef{\Gamma \vdash^i M : A\rightarrow B
                                        \qquad \Gamma \vdash^{0} N : A,
                                   }
                                   {\Gamma  \vdash^i M N : B}
                                    \qquad \ord{A} = \ord{B} - 1, \quad i \in \{ -1, 0 \} $$
\end{itemize}

In fact $\mathbf{(app_1^0)}$ is redundant since we can derive it
from $\mathbf{(app_1^{-1})}$ and $\mathbf{(seq^0_1)}$. The rules
$\mathbf{(app_1^i)}$ and $\mathbf{(app_2^i)}$ can be restated as
follow:
$$ \mathbf{(app^0)} \quad  \rulef{\Gamma \vdash^0 M : A\rightarrow B
                                        \qquad \Gamma \vdash^{0} N : A }
                                {\Gamma  \vdash^{0} M N : B}$$

$$ \mathbf{(app^{-1})} \quad  \rulef{\Gamma \vdash^{-1} M : A\rightarrow B
                                        \qquad \Gamma \vdash^0 N : A}
                                   {\Gamma  \vdash^0 M N : B}
                                    \qquad \ord{A} \geq \ord{B}$$

$$ \mathbf{(app'^{-1})} \quad  \rulef{\Gamma \vdash^{-1} M : A\rightarrow B
                                        \qquad \Gamma \vdash^0 N : A}
                                   {\Gamma  \vdash^{-1} M N : B}
                                    \qquad \ord{A} = \ord{B} - 1$$


\subsection{The abstraction rule}

Let us derive the abstraction rule specialized for the case of
homogeneous types. We recall the rule $\mathbf{(abs)}$:
$$ \mathbf{(abs^i)} \quad  \rulef{\Gamma, \overline{x} : \overline{A} \vdash^{i} M : B}
                                   {\Gamma  \vdash^{0} \lambda \overline{x} : \overline{A} . M : (\overline{A},B)} \qquad
                                   \forall y \in \Gamma : \ord{y} \geq \ord{\overline{A},B}$$

We now partitionned the context $\Gamma$ according to the order of
the variables. The partition are written in decreasing order of type
order. The notation $\Gamma | \overline{x}:\overline{A}$ means that
$\overline{x}:\overline{A}$ is the lowest partition of the context.

We also use the notation $(\overline{A}|B)$ to denote the
homogeneous type $(A_1, A_2, \ldots A_n, B)$ where $\ord{A_1} =
\ord{A_2} =  \ldots \ord{A_n} \geq \ord{B} -1$.


Suppose that we abstract the single variable $\overline{x} = x$,
then in order to respect the side condition, we need to abstract all
variables of order lower or equal to $\ord{x}$. In particular we
need to abstract the partition of the order of $x$.

Moreover to respect type homogeneity, we need to abstract variables
of the lowest order first.

Hence we can change the abstraction rule so that it only allows
abstraction of the lowest variable partition. The rule can then be
used repeatedely if further partitions need to be abstracted. We
obtained the following rule where the side-condition has
disappeared:

$$ \mathbf{(abs^i)} \quad  \rulef{\Gamma| \overline{x} : \overline{A} \vdash^{i} M : B}
                                   {\Gamma  \vdash^{0} \lambda \overline{x} : \overline{A} . M : (\overline{A}|B)}$$


\subsection{The entire set of rules}

Table \ref{tab:homosafelmd_rules} gives the entire set of rules.


\begin{table}[htbp]
$$ \mathbf{(seq)} \quad \rulef{\Gamma \vdash^{0} M : A}{\Gamma \vdash^{-1} M : A} $$

$$ \mathbf{(var)} \quad  \rulef{}{x : A\vdash^{0} x : A} $$

$$ \mathbf{(wk^{0})} \quad  \rulef{\Gamma \vdash^{0} M : A}{\Gamma , x : B \vdash^{0} M : A} \quad \ord{B} \geq \ord{A} $$

$$ \mathbf{(wk^{-1})} \quad  \rulef{\Gamma \vdash^{-1} M : A}{\Gamma , x : B \vdash^{-1} M : A} \quad \ord{B} \geq \ord{A} -1$$


$$ \mathbf{(app^{-1})} \quad  \rulef{\Gamma \vdash^{-1} M : A\rightarrow B
                                        \qquad \Gamma \vdash^0 N : A,
                                   }
                                   {\Gamma  \vdash^0 M N : B}
                                    \qquad \ord{A} \geq \ord{B}$$

$$ \mathbf{(app'^{-1})} \quad  \rulef{\Gamma \vdash^{-1} M : A\rightarrow B
                                        \qquad \Gamma \vdash^0 N : A,
                                   }
                                   {\Gamma  \vdash^{-1} M N : B}
                                    \qquad \ord{A} = \ord{B} - 1$$

$$ \mathbf{(app^0)} \quad  \rulef{\Gamma \vdash^0 M : A\rightarrow B
                                        \qquad \Gamma \vdash^{0} N : A,
                                   }
                                   {\Gamma  \vdash^{0} M N : B}$$

$$ \mathbf{(abs^i)} \quad  \rulef{\Gamma| \overline{x} : \overline{A} \vdash^{i} M : B}
                                   {\Gamma  \vdash^{0} \lambda \overline{x} : \overline{A} . M : (\overline{A}|B)}$$
\caption{Rules of the homogeneous safe lambda calculus}
\label{tab:homosafelmd_rules}
\end{table}


If we rename the sequents $\vdash^{0}$ and $\vdash^{-1}$ into
$\vdash^+$ and $\vdash^{-}$ respectively we observe that the rules
are similar to the ones given in \cite{Ong2005} except that the rule
$\mathbf{(app'^{-1})}$ is missing in \cite{Ong2005}.

