\chapter{Computation trees, traversals and game semantics}

The aim of this chapter is to develop tools that will be used in the
next chapter to give a characterisation of the game semantics of the
Safe $\lambda$-Calculus. Establishing such a characterisation is
complicated by the fact that Safety is a syntactic restriction
whereas Game Semantics is by nature a syntax-independent semantics.
We therefore need to make an explicit correspondence between the
game denotation of a term and its syntax.

Our approach follows ideas recently introduced in
\cite{OngLics2006}, mainly the notion of computation tree of a
simply-typed $\lambda$-term and traversals over the computation
tree. A computation tree can be regarded as an abstract syntax tree
(AST) of the $\eta$-long normal form of a term. A traversal is a
justified sequence of nodes of the computation tree respecting some
formation rules. Traversals are used to describe computations. An
interesting property is that the \emph{P-view} of a traversal
(computed in the same way as P-view of plays in Game Semantics) is a
path in the computation tree.

The main result that we will prove in this chapter is called the
\emph{Correspondence Theorem} (theorem \ref{thm:correspondence}). It
states that traversals over the computation tree are just
representations of the uncovering of plays in the
strategy-denotation of the term. Hence there is an isomorphism
between the strategy denotation of a term and its revealed game
denotation (i.e. its strategy denotation where internal moves are
not hidden after composition). This theorem permits us to explore
the effect that a given syntactic restriction has on the strategy
denotating a term.

To really make use of the Correspondence Theorem, it will be
necessary to restate it in the standard game-semantic framework in
which internal moves are hidden. For that purpose, we will define a
\emph{reduction} operation on traversals responsible of eliminating
the ``internal nodes'' of the computation. This leads to a
correspondence between the standard game denotation of a term and
the set of reductions of traversals over its computation tree.
Fortunately, the reduction process preserves the good properties of
traversals. This is guaranteed by the facts that the P-view of the
reduction of a traversal is equal to the reduction of the P-view of
the traversal, and the O-view of a traversal is the same as the
O-view of its reduction (lemma \ref{lem:redtrav_trav}). \vspace{8pt}

\emph{Related works}: Traversals of a computation tree provide a way
to perform \emph{local computation} of $\beta$-reductions as opposed
to a global approach where the $\beta$-reduction is implemented by
performing substitutions. A notion of local computation of
$\beta$-reduction has been investigated in
\cite{DanosRegnier-Localandasynchronou} through the use of special
graphs called ``virtual nets'' that embed the lambda-calculus.

In \cite{DBLP:conf/lics/AspertiDLR94}, a notion of graph based on
Lamping's graphs \citep{lamping} is introduced to represent
$\lambda$-terms. The authors unify different notions of paths
(regular, legal, consistent and persistent paths) that have appeared
in the literature as ways to implement graph-based reduction of
lambda-expressions. We can regard a traversal as an alternative
notion of path adapted to the graph representation of
$\lambda$-expressions given by computation trees.



%Is there any unsafe term whose game semantics is a strategy where
%pointers can be recovered?
%
%The answer is yes: take the term $T_i = (\lambda x y . y) M_i S$
%where $i =1..2$ and $\Gamma \vdash_s S : A$. $T_1$ and $T_2$ both
%$\beta$-reduce to the safe term $S$, therefore
%$\sem{T_1}=\sem{T_2}=\sem{S}$. But $T_1$ is safe whereas $T_2$ is
%unsafe. Since it is possible to recover the pointer from the game
%semantics of $S$, it is as well possible to recover the pointer from
%the semantics of $T_2$ which is unsafe.

\section{Computation tree}
We work in the general setting of the simply-typed
$\lambda$-calculus extended with a fixed set $\Sigma$ of
higher-order constants.

\subsection{$\eta$-long normal form and computation tree}

The $\eta$-long normal form appeared in
\citep{DBLP:journals/tcs/JensenP76} and
\citep{DBLP:journals/tcs/Huet75} under the names \emph{long reduced
form} and \emph{$\eta$-normal form} respectively. It was then
investigated in \citep{huet76} under the name \emph{extensional
form}.

The $\eta$-expansion of $M: A\typear B$ is defined to be the term
$\lambda x . M x : A\typear B$ where $x:A$ is a fresh variable. A
term $M : (A_1,\ldots,A_n,o)$ can be expanded in several steps into
$\lambda \varphi_1 \ldots \varphi_l . M \varphi_1 \ldots \varphi_l$
where the $\varphi_i:A_i$ are fresh variables. The $\eta$-normal
form of a term is obtained by hereditarily $\eta$-expanding every
subterm occurring at an operand position.

\begin{dfn}[$\eta$-long normal form]
A simply-typed term is either an abstraction or it can be written uniquely as
$s_0 s_1 \ldots s_m$ where $m\geq0$ and $s_0$ is a variable, a $\Sigma$-constant or an abstraction.
The $\eta$-long normal form of a term $M$, written $\aux{M}$ or sometimes $\etanf{M}$,
is defined as follows:
\begin{align*}
\aux{\alpha s_1 \ldots s_m : (A_1,\ldots,A_n,o)} &= \lambda \overline{\varphi} . \alpha \aux{s_1}\ldots \aux{s_m} \aux{\varphi_1} \ldots \aux{\varphi_n}
& \mbox{with $m,n\geq0$}\\
%\aux{(\lambda x . s_0) s_1 \ldots s_m } &=& (\lambda x . \aux{s_0}) \aux{s_1} \aux{s_2} \ldots \aux{s_m}
\aux{\lambda x . s } &= \lambda x . \aux{s} \\
\aux{(\lambda x . s_0) s_1 \ldots s_m : (A_1,\ldots,A_n,o) } &= \lambda \overline{\varphi} . (\lambda x . \aux{s_0}) \aux{s_1} \ldots \aux{s_m} \aux{\varphi_1} \ldots \aux{\varphi_n}
& \mbox{with $m\geq 1,n\geq0$}
\end{align*}
where $x$ and each $\varphi_i : A_i$ are variables and $\alpha$ is
either a variable or a constant.
\end{dfn}

For $n=0$, the first clause in the definition becomes:
$$\aux{x s_1 \ldots s_m : o} = \lambda . x \aux{s_1} \aux{s_2} \ldots \aux{s_m},$$
and we deliberately keep the \textsl{dummy} lambda in the right-hand
side of the equation because it will play an important role in the
correspondence with game semantics.



Note that our version of the $\eta$-long normal form is defined not only for $\beta$-normal terms but also for any simply-typed term.
Moreover it is defined in such a way that $\beta$-normality is preserved:
\begin{lem}
The $\eta$-long normal form of a term in $\beta$-normal form is also in $\beta$-normal form.
\end{lem}
\begin{proof}
By induction on the structure of the term and the order of its type.
\emph{Base case}:
If $M=x:0$ then $\aux{x} = \lambda . x$ is also in $\beta$-nf.
\emph{Step case}:
The case $M = \aux{(\lambda x . s_0) s_1 \ldots s_m : (A_1,\ldots,A_n,o)}$ with $m>0$ is not possible since $M$ is in
$\beta$-normal form.
Suppose $M = \lambda x . s$ then $s$ is in $\beta$-nf. By the induction hypothesis $\aux{s}$ is also in $\beta$-nf and therefore
so is $\aux{M} = \lambda x . \aux{s}$.

Suppose $M= \alpha s_1 \ldots s_m : (A_1,\ldots,A_n,o)$. Let $i,j$
range over $1..n$ and $1..m$ respectively. The $s_j$ are in
$\beta$-nf and the $\varphi_i$ are variables of order smaller than
$M$, therefore by the induction hypothesis the $\aux{\varphi_i}$ and
the $\aux{s_j}$ are in $\beta$-nf. Hence $\aux{M}$ is also in
$\beta$-nf.
\end{proof}


The computation tree of term is a certain tree representation of its
$\eta$-long normal form. It is defined as follows:
\begin{dfn}[Computation tree]
For any term $M$ in $\eta$-normal form we define the tree $\tau(M)$ by induction
on the structure of the term.
Since $M$ is in $\eta$-normal form, there are only two cases:
$M$ is either an abstraction or it is of ground type and can be written uniquely as
$s_0 s_1 \ldots s_m : 0$ where $m\geq0$,  $s_0$ is a variable, a
constant or an abstraction and each of the $s_j$ for $j\in 1..m$ is in $\eta$-normal form:
\begin{itemize}
\item the tree for $\lambda x_1 \ldots x_n. s$ where $n\geq0$ and $s$ is not an abstraction is:
$$ \tau(\lambda x_1 \ldots x_n . s) =
      \pstree[levelsep=4ex]
        { \TR{\lambda x_1 \ldots x_n} }
        { \SubTree{\tau(s)^{-}} }
$$
where $\tau(s)^{-}$ denotes the tree obtained after deleting the
root of $\tau(s)$.


\item the tree for $\alpha s_1 \ldots s_m : o$ where $m\geq0$ and $\alpha$ is a variable or constant is:
$$ \tau( \alpha s_1 \ldots s_m) =
    \tree{\lambda}
    {
        \pstree[levelsep=4ex]
            { \TR{\alpha} }
            { \SubTree{\tau(s_1)} \SubTree[linestyle=none]{\ldots} \SubTree{\tau(s_m)}
            }
    }
$$


\item the tree for $(\lambda x.s_0) s_1 \ldots s_n : o$ where $n \geq 1$ is:
$$ \tau((\lambda x.s_0) s_1 \ldots s_n) =
    \tree{\lambda}
    {
        \pstree[levelsep=4ex]
            { \TR{@} }
            {
            \SubTree{\tau(\lambda x.s_0)}    \SubTree{\tau(s_1)} \SubTree[linestyle=none]{\ldots} \SubTree{\tau(s_n)}
            }
    }
$$
\end{itemize}

The \emph{computation tree} of a simply-typed term $M$ (whether or not in $\eta$-normal form) is written $\tau(M)$
and defined to be $\tau(M) = \tau(\etanf{M})$.
\end{dfn}

The nodes (and leaves) of the tree are of three kinds:
\begin{itemize}
\item $\lambda$-nodes labelled $\lambda \overline{x}$ (note that a $\lambda$-node represents several consecutive variable abstractions),
\item application nodes labelled @,
\item variable or constant nodes labelled $\alpha$ for some constant or variable $\alpha$.
\end{itemize}
We write $N$ for the set of nodes of $\tau(M)$, $N_\Sigma$ for the set of $\Sigma$-labelled nodes,
$N_@$ for the set of @-labelled nodes, $N_{var}$ for the set of variable nodes and
$N_{fv}$ for the subset of $N_{var}$ constituted of free-variable nodes.


Let $\mathcal{T}$ denote the set of $\lambda$-terms.
Each subtree of the computation tree $\tau(M)$ represents a subterm of $\aux{M}$.
We define the function $\kappa : N \rightarrow \mathcal{T}$ that maps a node $n \in N$ to the subterm of $\aux{M}$
represented by the subtree of $\tau(M)$ rooted at $n$.
In particular if $r$ is the root of $\tau(M)$ then $\kappa(r) = \aux{M}$.

\begin{dfn}[Node order]
\label{def:nodeorder}
The node-order function $\textsf{ord}$ is defined on nodes as follows:
\begin{eqnarray*}
\ord{n} = \left\{
  \begin{array}{ll}
    \ord{T}, & \hbox{if $n$ is a variable or constant of type $T$;} \\
    1 + \max_{z\in \overline{\xi}\union fv(M)} \ord{z}, & \hbox{if $n$ is labelled $\lambda \overline{\xi}$ and is the root of $\tau(M)$;} \\
    1 + \max_{z\in \overline{\xi}} \ord{z}, & \hbox{if $n$ is labelled $\lambda \overline{\xi}$ and is not the root;} \\
    0, & \hbox{if $n$ is labelled @.}
  \end{array}
\right.
\end{eqnarray*}
\end{dfn}

\noindent Some remarks:
\begin{itemize}
\item In a computation tree, nodes at even level are $\lambda$-nodes and nodes at odd level are either application nodes,
variable or constant nodes;

\item for any ground type variable or constant $\alpha$,
$\tau(\alpha) = \tau(\lambda . \alpha) =  \pstree[levelsep=3ex]
    { \TR{\lambda } }
    { \TR{\alpha}
    }$;

\item for any higher-order variable or constant $\alpha : (A_1,\ldots,A_p,o)$, the computation tree $\tau(\alpha)$ has the following form:
$ \pstree[levelsep=3ex]{\TR{\lambda}}
        {\pstree[levelsep=3ex]
                { \TR{\alpha} }
                { \tree{\lambda \overline{\xi_1}}{\TR{\ldots}} \TR{\ldots} \tree{\lambda \overline{\xi_p}}{\TR{\ldots}}
                }
        }
$;

\item for any tree of the form
        $ \pstree[levelsep=4ex]
            { \TR{\lambda \overline{\varphi}} }
            { \pstree[levelsep=3ex]
                {\TR{n}}
                {\TR{\lambda \overline{\xi_1}} \TR{\ldots} \TR{\lambda \overline{\xi_p}}}
            }
        $,
    we have $\ord{\kappa(n)}=0$.

\end{itemize}



\subsection{Pointers and justified sequence of nodes}

\begin{dfn}[Binder]
Let $n$ be a variable node of the computation tree labelled $x$. We
say that a node $n$ is bound by the node $m$, and $m$ is called the
binder of $n$, if $m$ is the closest node in the path from $n$ to
the root of the tree such that $m$ is labelled $\lambda
\overline{\xi}$ with $x\in \overline{\xi}$.
\end{dfn}

\begin{dfn}[Enabling]
The enabling relation $\vdash$ is defined on the set of nodes of the
computation tree. We write $m \vdash n$ and we say that $m$ enables
$n$ if and only if
\begin{itemize}
\item $n$ is a bound variable node and $m$ is the binder of $n$,
\item or $n$ is a free variable node and $m$ is the root of the computation tree,
\item or $n$ is a $\lambda$-node and $m$ is the parent node of $n$.
\end{itemize}
\end{dfn}

We call \emph{input-variable} a variable that is hereditarily justified by the root of the computation tree.
Free variables and variables bound by the root are example of input-variables.

\begin{dfn}[Justified sequence of nodes]
A \emph{justified sequence of nodes} is a sequence of
nodes of the computation tree $\tau(M)$ with pointers attached to the nodes. A node $n$ in the sequence
that is either a variable node or a lambda-node different from the root of the computation tree
has a pointer to a node $m$ occurring before $n$ in the sequence such that $m \vdash n$.
If $n$ points to $m$ then we say that $m$ \emph{justifies} $n$ and we represent the pointer in the sequence as follows:
$$\rnode{m}{m} \cdot \ldots \cdot \rnode{n}{n} \bkptr[nodesep=1pt]{40}{n}{m}$$
\end{dfn}
Note that justified sequences are also defined for open terms:
occurrences of nodes in $N_{fv}$ must point to an occurrence of the
root of the computation tree.


A pointer is sometime labeled with an index $i$: if $m$ is a
$\lambda$-node then it indicates that $n$ is labelled with the $i$th
variable abstracted in $m$; otherwise it indicates that $n$ is the
$i$th child of $m$. A pointer in a justified sequence of nodes has
therefore one of the following forms: \vspace{2pt}
$$
\rnode{m}{r} \cdot \ldots \cdot \rnode{n}{z} \bkptr[nodesep=1pt]{40}{n}{m}
\hspace{1.5cm}
\rnode{m}{\lambda \overline{\xi}} \cdot \ldots \cdot \rnode{n}{\xi_i} \bkptr[nodesep=1pt]{40}{n}{m} \bklabel{i}
\hspace{1.5cm}
\rnode{m}{@ } \cdot \ldots \cdot \rnode{n}{\lambda \overline{\eta}} \bkptr[nodesep=1pt]{40}{n}{m} \bklabel{j}
\hspace{1.5cm}
\rnode{m}{\alpha } \cdot \ldots \cdot \rnode{n}{\lambda \overline{\eta}} \bkptr[nodesep=1pt]{40}{n}{m} \bklabel{k}
$$
where $r$ denotes the root of $\tau(M)$, $z \in N_{fv}$, $\xi_1,
\ldots \xi_n$ are bound variables, $\alpha \in N_{\Sigma} \union
N_{var}$, $i \in 1..n$, $j$ ranges from $0$ to the number of
children nodes of @ minus 1 and $k \in 1 ..arity(\alpha)$.

The following numbering conventions are used:
\begin{itemize}
\item the first child of a @-node is numbered $0$,
\item the first child of a variable or constant node is numbered $1$,
\item variables in $\overline{\xi}$ are numbered from $1$ onward ($\overline{\xi} = \xi_1 \ldots \xi_n$).
\end{itemize}
We use the notation $n.i$ to denote the $i$th child of node $n$.


We write $s = t$ to denote that the justified sequences $t$ and $s$
have same nodes \emph{and} pointers. Justified sequence of nodes can
be ordered using the prefix ordering: $t \sqsubseteq t'$ if and only
if $t=t'$ or the sequence of nodes $t$ is a finite prefix of $t'$
(and the pointers of $t$ are the same as the pointers of the
corresponding prefix of $t'$). Note that with this definition,
infinite justified sequences can also be compared. This ordering
gives rise to a complete partial order.

We say that a node $n_0$ of a justified sequence is hereditarily justified by $n_p$ if there are nodes $n_1, n_2, \ldots n_{p-1}$ in
the sequence such that for all $i\in 0..p-1$, $n_i$ points to $n_{i+1}$.

If $N$ is a set of nodes and $s$ a justified sequence of nodes then
we write $s \upharpoonright N$ to denote the subsequence of $s$
obtained by keeping only the nodes that are hereditarily
justified by nodes in $N$. This subsequence is also a justified
sequence of nodes. If $n$ denotes a node of $\tau(M)$ we
abbreviate $s \upharpoonright \{ n \}$ into $ s\upharpoonright n$.

\begin{lem}
\label{lem:filtercontinous}
For any set of node $N$, the filtering function $\_ \upharpoonright N$ defined on the cpo of justified sequences ordered by the prefix ordering
is continuous.
\end{lem}
\begin{proof}
Clearly $\_ \upharpoonright N$ is monotonous.
Suppose that $(t_i)_{i\in\omega}$ is a chain of justified sequence of nodes. Let $u$ be a finite prefix of $(\bigvee t_i) \upharpoonright r$.
Then $u = s \upharpoonright r$ for some finite prefix $s$ of $\bigvee t_i$. Since $s$ is finite we must have $s \sqsubseteq t_j$ for some $j\in\omega$.
Therefore $u \sqsubseteq t_j \upharpoonright r \sqsubseteq \bigvee (t_j \upharpoonright r)$.
This is valid for any finite prefix $u$ therefore $(\bigvee t_i) \upharpoonright r \sqsubseteq \bigvee (t_j \upharpoonright r)$.
\end{proof}


\begin{dfn}[P-view of justified sequence of nodes]
The P-view of a justified sequence of nodes $t$ of $\tau(M)$, written $\pview{t}$, is defined as follows:
\begin{eqnarray*}
 \pview{\epsilon} &=&  \epsilon \\
 \pview{s \cdot n }  &=&  \pview{s} \cdot n \\
 \pview{s \cdot \rnode{m}{m} \cdot \ldots \cdot \rnode{lmd}{\lambda \overline{\xi}}} &=& \pview{s} \cdot \rnode{m2}{m} \cdot \rnode{lmd2}{\lambda \overline{\xi}}
   \bkptr[nodesep=1pt]{30}{lmd}{m}
   \bkptr[nodesep=1pt]{60}{lmd2}{m2} \\
 \pview{s \cdot r }  &=&  r
\end{eqnarray*}
where $r$ is the root of the tree $\tau(M)$ and $n$ ranges over
non-lambda nodes (i.e. $N_\Sigma \union N_@ \union N_{var}$).

In the second clause, the pointer associated to $n$ is preserved
from the left-hand side to the right-hand side i.e. if in the
left-hand side, $n$ points to some node in $s$ that is also present
in $\pview{s}$ then in the right-hand side, $n$ points to this
occurrence of the node in $\pview{s}$.

Similarly, in the third clause the pointer associated to $m$ is preserved.
\end{dfn}

We also define O-view, the dual notion of P-view:
\begin{dfn}[O-view of justified sequence of nodes]
The O-view of a justified sequence of nodes $t$ of $\tau(M)$, written $\oview{t}$, is defined as follows:
\begin{eqnarray*}
 \oview{\epsilon} &=&  \epsilon \\
 \oview{s \cdot \lambda \overline{\xi} }  &=&  \oview{s} \cdot \lambda \overline{\xi} \\
 \oview{s \cdot \rnode{m}{m} \cdot \ldots \cdot \rnode{x}{x}} &=& \oview{s} \cdot \rnode{m2}{m} \cdot \rnode{n2}{x} \\
   \bkptr[nodesep=1pt]{30}{x}{m}
   \bkptr[nodesep=1pt]{60}{n2}{m2}
 \oview{s \cdot n }  &=&  n
\end{eqnarray*}
where $x$ ranges over variable nodes and  $n$ ranges over non-lambda
nodes without pointer (i.e. $N_@ \union N_\Sigma$).

The pointer associated to $\lambda \overline{\xi}$ in the second
equality and the pointer associated to $m$ in the third equality are
preserved from the left-hand side to the right-hand side of the
equalities.
\end{dfn}

\begin{dfn}[Alternation and Visibility] \ \\
A justified sequence of nodes $s$ satisfies:
\begin{itemize}
\item \emph{Alternation} if for any two consecutive nodes in $s$, one is a $\lambda$-node
and the other is not;

\item \emph{P-visibility} if every variable node in $s$ points to a node occurring in the P-view a that point;

\item  \emph{O-visibility} if every lambda node in $s$ points to a node occurring in the O-view a that point.
\end{itemize}
\end{dfn}

\begin{property}
\label{proper:pview_visibility}
The P-view (resp. O-view) of a justified sequence verifying P-visibility (resp. O-visibility)
is a well-formed justified sequence verifying P-visibility (resp. P-visibility).
\end{property}
This is proved by an easy induction.

\subsection{Adding value-leaves to the computation tree}
\label{sec:adding_value_leaves}

We now add leaves to the computation tree that has been defined in the previous section.
These leaves, called \emph{value-leaves}, are attached to the nodes of the computation tree. Each
value-leaf corresponds to a possible value of the base type $o$.
We write $\mathcal{D}$ to denote the set of values of the base type
$o$. The values leaves are added as follows: every  %$\lambda$-node or variable
node $n \in \tau(M)$ has a child leaf denoted by $v_n$ for each possible value $v \in \mathcal{D}$.

%@-nodes and $\Sigma$-nodes do not have child leaves.

%If $n$ is a $\lambda$-node then its value-leaves are numbered from $1$ onwards.
%If $n$ is a variable or constant node then its children nodes are numbered from $1$ to $arity(n)$ and
%its value-leaves are numbered from $arity(n)+1$ onwards.
%If $n$ is an application node then its value-leaves are numbered from $1$ onwards.

Everything that we have defined for computation tree can be lifted
to this new version of computation tree. The node order of a
value-leaf is defined to be $0$. The enabling relation $\vdash$ is
extended so that every leaf is enabled by its parent node. The
definition of justified sequence does not change.
When representing a link in a justified sequence going from a value-leaf $v_n$ to a node $n$,
we label the link with $v$:
$$
\rnode{n}{n} \cdot \ldots \cdot \rnode{vn}{v_n} \bkptr[nodesep=1pt]{40}{vn}{n} \bklabel{v}
$$


For the definition
of P-view, O-view and visibility, value-leaves are treated as
$\lambda$-nodes if they are at odd level in the computation tree and
as variable nodes if there at a even level.

From now the term ``computation tree'' refers to this extended
definition.
\vspace{10pt}

Let $n$ be a node of a justified sequence of nodes.
% that is either a $\lambda$-node or a variable node.
If there is an occurrence of a value-leaf $v_n$ in the sequence that points to $n$ we say that
$n$ is \emph{matched} by $v_n$. If there is no value-leaf in the sequence that points to $n$ we
say that $n$ is an \emph{unmatched node}.
The last unmatched node is called the \emph{pending node}.
A justified sequence of nodes is \emph{well-bracketed} if
each value-leaf in the traversal points to the pending node at that point.

If $t$ is a traversal then we write $?(t)$ to denote the subsequence
of $t$ consisting only of unmatched nodes.

\subsection{Traversal of the computation tree}
\label{subsec:traversal} We first define traversals for computation
tree of simply-typed $\lambda$-terms with no interpreted constants.
We will then we show how to extend the definition to the general
setting of $\lambda$-calculus augmented with interpreted constants.

\subsubsection{Traversals for simply-typed $\lambda$-terms}
Intuitively, a \emph{traversal} is a justified sequence of nodes of the computation tree where each node
indicates a step that is taken during the evaluation of the term.

\begin{dfn}[Traversals for pure simply-typed $\lambda$-terms]
\label{def:traversal}
In the simply-typed $\lambda$-calculus with no constants,
a traversal over a computation tree $\tau(M)$
is a justified sequence of nodes defined by induction on the rules
given below. A \emph{maximal-traversal} is a traversal that cannot be
extended by any rule. If $T$ denotes a computation tree then we write $\travset(T)$
to denote the set of traversals of $T$. We also use the abbreviation $\travset(M)$ for $\travset(\tau(M))$.

\emph{Initialization rules}
\begin{itemize}
\item ($\epsilon$) The empty sequence of node $\epsilon$ is a traversal.
\item (Root) The length 1 sequence $r$, where $r$ is denotes the root of $\tau(M)$, is a traversal.
\end{itemize}

\emph{Structural rules}
\begin{itemize}
\item (Lam) Suppose that $t \cdot \lambda \overline{\xi}$ is a traversal and $n$ is the only child node of $\lambda \overline{\xi}$ in
the computation tree then
$$t \cdot \lambda \overline{\xi} \cdot n$$
is also a traversal
where $n$ points to the (only) occurrence of its enabler in $\pview{t \cdot \lambda \overline{\xi}}$.
In particular, if $n$ is a free variable node then $n$ points to the first node of $t$.

\item (App) If $t \cdot @$ is a traversal then so is
$$t \cdot \rnode{m}{@} \cdot
\rnode{n}{n} \bkptr[nodesep=1pt]{60}{n}{m} \bklabelc{0}
$$

i.e. the next visited node is the $0$th child node of @ : the
node corresponding to the operator of the application.
\end{itemize}

\emph{Input-variable rules}
\begin{itemize}
\item (InputVar$^0$) If $t = t_1 \cdot x \cdot t_2$ is a traversal where
$x$ is the pending node in $t$ (i.e. $?(t_2)=\epsilon$)
and $x$ is a ground-type input-variable then for any $v \in \mathcal{D}$ the following is a traversal
$$t_1 \cdot \rnode{x}{x} \cdot t_2 \cdot \rnode{xv}{v_x}
\bkptr[nodesep=1pt]{40}{xv}{x} \bklabelc{v}$$


\item (InputVar$^{\geq 1}$)
If $t = t_1 \cdot x \cdot t_2$ is a traversal where
$x$ is the pending node in $t$ (i.e. $?(t_2)=\epsilon$)
and $x$ is a higher-order input-variable then the following is a traversal:
$$t_1 \cdot \rnode{m}{x} \cdot t_2 \cdot
\rnode{n}{n} \bkptr[nodesep=1pt]{40}{n}{m} \bklabelc{i} \qquad
\mbox{ for } 1 \leq i \leq arity(x).$$
Moreover for any $v\in \mathcal{D}$ the sequence $t_1 \cdot \rnode{x}{x} \cdot t_2 \cdot
\rnode{xv}{v_x} \bkptr[nodesep=1pt]{40}{xv}{x} \bklabelc{v}$ is also a traversal.
\end{itemize}

\emph{Copy-cat rules}
\begin{itemize}
  \item (CCAnswer-@)
%  If $t \cdot \lambda \overline{\xi} \cdot \rnode{app}{@} \cdot \rnode{lz}{\lambda \overline{z}} \cdot \ldots \cdot  \rnode{lzv}{v_{\lambda \overline{z}}}
%              \bkptr[nodesep=1pt]{30}{lzv}{lz} \bklabelc{v}
%              \bkptr[nodesep=1pt]{40}{lz}{app} \bklabelc{0}$
%              is a traversal then so is:
%              $t \cdot \rnode{lmd}{\lambda \overline{\xi}} \cdot \rnode{app}{@} \cdot \rnode{lz}{\lambda \overline{z}} \cdot \ldots \cdot \rnode{lzv}{v_{\lambda \overline{z}}} \cdot
%              \rnode{lmdv}{v_{\lambda \overline{\xi}}}
%              \bkptr[nodesep=1pt]{30}{lzv}{lz} \bklabelc{v}
%              \bkptr[nodesep=1pt]{40}{lz}{app} \bklabelc{0}
%                \bkptr[nodesep=1pt]{30}{lmdv}{lmd} \bklabelc{v}$.
  If $t \cdot \rnode{app}{@} \cdot \rnode{lz}{\lambda \overline{z}} \cdot \ldots \cdot \rnode{lzv}{v_{\lambda \overline{z}}}
              \bkptr[nodesep=1pt]{30}{lzv}{lz} \bklabelc{v}
              \bkptr[nodesep=1pt]{40}{lz}{app} \bklabelc{0}$
              is a traversal then so is:
              $t \cdot \rnode{app}{@} \cdot \rnode{lz}{\lambda \overline{z}} \cdot \ldots \cdot \rnode{lzv}{v_{\lambda \overline{z}}} \cdot \rnode{appv}{v_@}
              \bkptr[nodesep=1pt]{30}{lzv}{lz} \bklabelc{v}
              \bkptr[nodesep=1pt]{40}{lz}{app} \bklabelc{0}
              \bkptr[nodesep=1pt]{30}{appv}{app} \bklabelc{v}$.


  \item (CCAnswer-$\lambda$) If $t \cdot \lambda \overline{\xi} \cdot \rnode{x}{x} \cdot \ldots \cdot  \rnode{xv}{v_x}
              \bkptr[nodesep=1pt]{30}{xv}{x} \bklabelc{v}$
              is a traversal then so is:
              $t \cdot \rnode{lmd}{\lambda \overline{\xi}} \cdot \rnode{x}{x} \cdot \ldots \cdot \rnode{xv}{v_x} \cdot
              \rnode{lmdv}{v_{\lambda \overline{\xi}}}
              \bkptr[nodesep=1pt]{20}{xv}{x} \bklabelc{v}
                \bkptr[nodesep=1pt]{20}{lmdv}{lmd} \bklabelc{v}$.

     \item (CCAnswer-var) If $t \cdot y \cdot \rnode{lmd}{\lambda \overline{\xi}}
                   \cdot \ldots
                   \cdot \rnode{lmdv}{v_{\lambda \overline{\xi}}} \bkptr[nodesep=1pt]{30}{lmdv}{lmd} \bklabelc{v}$ is a traversal,
                   where $y$ is a non input-variable node, then the following is also a traversal:
        $$t \cdot \rnode{y}{y}
            \cdot \rnode{lmd}{\lambda \overline{\xi}}
            \cdot \ldots
            \cdot \rnode{lmdv}{v_{\lambda \overline{\xi}}}
            \cdot \rnode{yv}{v_y}
                \bkptr[nodesep=3pt]{35}{yv}{y} \bklabelc{v}
                \bkptr[nodesep=1pt]{30}{lmdv}{lmd} \bklabelc{v}.$$


\item (Var)
If $t \cdot x_i$ is a traversal where $x_i$ is not an input-variable,
then the rule (Var) permits to visit the node corresponding to the subterm that would be substituted
for $x_i$ if all the $\beta$-redexes occurring in $M$ were reduced.

The binding node $\lambda \overline{x}$ necessarily occur previously
in the traversal. Since $x$ is not hereditarily justified by the
root, $\lambda \overline{x}$ is not the root of the tree and
therefore its justifier $p$ - which is also its parent node - occurs
immediately before itself it in the traversal. We do a case analysis
on $p$:

    \begin{itemize}
    \item Suppose $p$ is an @-node then $\lambda \overline{x}$ is necessarily the first child node of $p$
    and $p$ has exactly $|\overline{x}| + 1$ children nodes:
    $$\pstree[levelsep=7ex]{\TR{\stackrel{\vdots}{@^{[p]}}}}
    {   \pstree[linestyle=dotted,levelsep=4ex]{\TR{\lambda \overline{x}}\treelabel{0}}
            {\TR{x_i }}
        \tree{\lambda \overline{\eta_1}}{\vdots}\treelabel{1}
        \TR[edge=\dotedge]{}
        \tree{\lambda \overline{\eta_i}}{\vdots}\treelabel{i}
        \TR[edge=\dotedge]{}
        \tree{\lambda \overline{\eta_{|x|}}}{\vdots}\treelabel{|x|}
    }
    $$
    In that case, the next step of the traversal is a jump to $\lambda \overline{\eta_i}$ -- the $i$th child of
    @ -- which corresponds to the subterm that would be substituted for $x_i$ if the $\beta$-reduction was
    performed:
    \vspace{0.3cm}
    $$t' \cdot \rnode{n}{@^{[p]}} \cdot
    \rnode{lx}{\lambda \overline{x}} \cdot \ldots \cdot
    \rnode{x}{x_i} \cdot
    \rnode{mi}{\lambda \overline{\eta_i}} \cdot \ldots
    \bkptr[ncurv=0.45]{45}{mi}{n} \bklabel{i}
    \bkptr[ncurv=0.6]{50}{x}{lx} \bklabel{i} \in \travset(M)
    $$

    \item Suppose $p$ is variable node $y$, then
    necessarily the node $\lambda \overline{x}$ has been added to the traversal $t_{\leq y}$ using the (Var) rule
    (this is proved in proposition \ref{prop:pviewtrav_is_path}(i)).
    Therefore $y$ is substituted by the term $\kappa(\lambda \overline{x})$ during the evaluation of the term
    and we have $\ord{y} = \ord{\lambda \overline{x}}$.

    Consequently, during reduction, the variable $x_i$ is substituted by the subterm represented by
    $\lambda \overline{\eta_i}$ -- the $i$th child node of $y$.
    Hence the following justified sequence is also a traversal:
    \vspace{0.2cm}
    $$t' \cdot \rnode{n}{y^{[n]}} \cdot
    \rnode{lx}{\lambda \overline{x}} \cdot \ldots \cdot
    \rnode{x}{x_i} \cdot
    \rnode{mi}{\lambda \overline{\eta_i}} \cdot \ldots
    \bkptr[ncurv=0.6]{50}{x}{lx} \bklabel{i}
    \bkptr[ncurv=0.5]{50}{mi}{n} \bklabel{i}$$
    \end{itemize}
\end{itemize}
Note that a traversal always starts with the root of the tree.
\end{dfn}

\begin{rem}
Our notions of computation tree and traversal differ slightly from
\cite{OngLics2006}.

Firstly, our computation tree do not have nodes labelled with
(uninterpreted) first-order constants. On the other hand, there are
nodes which are labelled by free variables of any order. Since
uninterpreted constants can be regarded as free variables, we do not
lose any expressivity. The traversal rules (InputVar$^0$) and
(InputVar$^\geq 1$) provide a more general version of the (Sig) rule
of \cite{OngLics2006}.

Secondly we have introduced copy-cat rules that permit to visit the
value-leaves of the computation tree. The presence of value-leaves
is necessary to model free variables as well as the interpreted
constants present in extensions of the $\lambda$-calculus such as
\pcf\ or \ialgol.
\end{rem}

\begin{exmp}
Consider the following computation tree:
$$\tree{\lambda}
{
    \tree{@}
    {
        \pstree[levelsep=8ex,linestyle=dotted]{\TR{\lambda y}\treelabel{0} }
        {
            \pstree[levelsep=8ex]{\TR{y}}
            {
                \tree{\lambda \overline{\eta_1}}{\vdots} \treelabel{1}
                \TR[edge=\dotedge]{}
                \tree{\lambda \overline{\eta_i}}{\vdots}\treelabel{i}
                \TR[edge=\dotedge]{}
                \tree{\lambda \overline{\eta_n}}{\vdots}\treelabel{n}
            }
        }
        \pstree[levelsep=6ex,linestyle=dotted]{\TR{\lambda \overline{x}}\treelabel{1}}{ \tree{x_i}{\TR{} \TR{} } }
    }
}
$$
An example of traversal of this tree is:
\vspace{0.3cm}
$$ \lambda \cdot
\rnode{app}{@}  \cdot
\rnode{ly}{\lambda y} \cdot \ldots \cdot
\rnode{y}{y} \cdot
\rnode{lx}{\lambda \overline{x}} \cdot \ldots \cdot
\rnode{x}{x_i} \cdot
\rnode{leta}{\lambda \overline{\eta_i} } \cdot \ldots
\bkptr[ncurv=0.6,nodesep=0]{40}{x}{lx}  \bklabel{i}
\bkptr[ncurv=0.5]{50}{leta}{y}  \bklabel{i}
\bkptr[ncurv=0.6,nodesep=0]{40}{y}{ly}  \bklabel{1}
\bkptr[ncurv=0.5]{50}{lx}{app}  \bklabel{1}$$
\end{exmp}

\subsubsection{Traversals for interpreted constants}

\begin{dfn}[Well-behaved traversal rule]
\label{def:wellbehaved_traversal}
A traversal rule is \emph{well-behaved} if it can be stated under the following form:
$$\rulef{t = t_1\cdot n \cdot t_2 \in \travset \quad ?(t_2) = \epsilon \quad P(t)}
  { \stackrel{  \rule{0pt}{3pt} }{t' = t_1\cdot \rnode{n}{n} \cdot t_2 \cdot \rnode{m}{m} \in \travset} }
   \bkptr[nodesep=1pt]{35}{m}{n}
    \ m\in S(t)
   $$
such that:
\begin{enumerate}
  \item $n$ is a variable or a constant node;
  \item $P$ expresses some condition on $t$;
  \item $S(t)$ is some subset of $E(n)$, the set of children $\lambda$-nodes and value-leaves of $n$.
  If $S(t)$ has more than one element then the rule is non-deterministic.
\end{enumerate}
\end{dfn}
Note that if $t$ is well-bracketed then $t'$ is also well-bracketed
and if $?(t)$ satisfies alternation (resp. visibility) then so does $?(t')$.


The rules (InputVar$^0$) and (InputVar$^{\geq1}$) are two examples of
non-deterministic well-behaved traversal rules for which
$S(t)$ is exactly the set of all children-nodes and value-leaves of $n$:
$S(t) = \{ n.i \ |\ i \in 1..arity(n) \} \union  \{ v_n \ | \ v \in \mathcal{D} \} $.


In the presence of higher-order interpreted constants, additional rules must be specified to indicate how
the constant nodes should be traversed in the computation tree. These rules
are specific to the language that is being studied.
In the last section of this chapter we will define such traversals for the interpreted constants of
\pcf\ and \ialgol.

From now on, we consider a simply-typed $\lambda$-calculus language extended with
higher-order interpreted constants for which some constant traversal rules have been defined
and we take the following condition as a prerequisite:
\begin{center}
  \textbf{Condition (WB) :} the constant traversal rules are well-behaved.
\end{center}


\subsubsection{Some properties of traversals}

\begin{prop}
\label{prop:pviewtrav_is_path}
Let $t$ be a traversal. Then:
\begin{itemize}
\item[(i)] $t$ is a well-defined and well-bracketed justified sequence;
\item[(ii)] $?(t)$ is a well-defined justified sequence verifying alternation, P-visibility and O-visibility;
\item[(iii)] $\pview{?(t)}$ is a path in the computation tree going from the root to the last node in $?(t)$.
\end{itemize}
\end{prop}
This is the counterpart of proposition 6 from
\cite{OngHoMchecking2006} which is proved by induction on the
traversal rules. This proof can be easily adapted to take into
account the constant rules (using the assumption that constants
rules are well-behaved) and the presence of value-leaves in the
traversal.
\begin{proof}
We just give a partial proof of (i). We prove that in the second case of the (Var) rule, where $p$ is a variable node $y$,
the node $\lambda \overline{x}$ has necessarily been added to the traversal $t_{\leq y}$ using the (Var) rule.

Suppose that an (InputVar) rule is used to produce $t_{<y} \cdot y
\cdot \lambda \overline{x}$, then $y$ is an input-variable node and
also the parent node of $\lambda \overline{x}$. But $x$ is not an
input-variable therefore it cannot be the child node of an
input-variable. Hence (Var) is the only rule which can be used to
produce $t_{<y} \cdot y \cdot \lambda \overline{x}$.
\end{proof}

%In particular to prove that the copy-cat rules are well-defined, one needs to ensure that
%if the last two unmatched nodes are $y$ and $\lambda \overline{\xi}$ in that order, for some non input-variable node $y$ then necessary
%      $y$ and $\lambda \overline{\xi}$ are consecutive nodes in the traversal.
%    This is because in a traversal, a non input-variable $y$ is always followed by a lambda node and whenever this lambda node is answered
%    there is only one way to extend the traversal : by using the copy cat rule to answer the $y$ node.


\begin{dfn}[Traversal reduction]
Let $r$ be the root of the computation tree. We say that the
justified sequence of nodes \emph{$s$ is a reduction of the
traversal $t$} just when $s = t \upharpoonright r$.
\end{dfn}

Since @-nodes and $\Sigma$-constants do not have pointer, the
reduction of traversal contains only nodes in $N_\lambda \union
N_{var}$.


\begin{lem}
\label{lem:var_followedby_child} Let $M$ be a term in $\beta$-normal
form and $t$ be a traversal of $\tau(M)$. If $?(t) = u_1 \cdot
\rnode{m}{m} \cdot u_2 \cdot \rnode{n}{n}
\bkptr[nodesep=1pt]{20}{n}{m}$ where $m$ is a not a $\lambda$-node
then $u_2 = \epsilon$.
\end{lem}
\begin{proof}
By induction on the traversal rules. The only relevant rules are (Var), (CCAnswer-var), (InputVar$^0$), (InputVar$^{\geq 1}$)
and the constant rules.
Since the term is in $\beta$-normal form, there is no @-node in $\tau(M)$ and therefore (Var) cannot be used.
For the rules (CCAnswer-var), (InputVar$^0$) and (InputVar$^{\geq 1})$ we just use the well-bracketedness of traversals.
For the constant rules, the result is a consequence of condition (WB) stating that constant rules are the well-behaved.
\end{proof}

\begin{lem}[View of a traversal reduction]
\label{lem:redtrav_trav} Let $M$ be a term in $\beta$-normal form,
$r$ be the root of $\tau(M)$ and $t$ be a traversal of $\tau(M)$. We
have
\begin{itemize}
\item[(i)] $ \pview{?(t) \upharpoonright  r } = \pview{?(t)} \upharpoonright r$;
\item[(ii)] if the last node in $t$ is hereditarily justified by $r$ then $ \oview{?(t) \upharpoonright r } = \oview{?(t)}$.
\end{itemize}
\end{lem}

\begin{proof}
(i) By induction. Base case: it is trivially true for the empty
traversal $t = \epsilon$. Step case: consider a traversal $t$ and
suppose that the property (i) is verified for all traversal smaller
than $t$. There are three cases:
\begin{itemize}
\item Suppose $?(t) = t' \cdot r$ then:
    \begin{align*}
    \pview{?(t)} \upharpoonright  r
        &=  \pview{t' \cdot r } \upharpoonright  r       & (\mbox{definition of } ?(t))\\
        &=  r \upharpoonright  r                         & (\mbox{def. P-view})\\
        &=  r                                                & (\mbox{def. operator $\upharpoonright$})\\
        &=  \pview{(t' \upharpoonright  r ) \cdot r }    & (\mbox{def. P-view})\\
        &=  \pview{(t' \cdot r)  \upharpoonright  r }    & (\mbox{def. operator $\upharpoonright$})\\
        &= \pview{?(t) \upharpoonright  r }                & (\mbox{definition of } ?(t))
    \end{align*}

\item Suppose $?(t) = t' \cdot n$ where $n$ is a non-lambda
move. We have:
    \begin{equation}
    \pview{?(t)} = \pview{t' \cdot n} = \pview{t'} \cdot n  \label{eq_tprime}
    \end{equation}
    \begin{itemize}
    \item If $n$ is not hereditarily justified by $r$ then:
    \begin{align*}
    \pview{?(t)} \upharpoonright  r
        &= (\pview{t'} \cdot n) \upharpoonright  r  & (\mbox{equation \ref{eq_tprime}}) \\
        &= \pview{t'} \upharpoonright  r            & (n \mbox{ is not hereditarily justified by } r) \\
        &= \pview{t' \upharpoonright  r }           & (\mbox{induction hypothesis}) \\
        &= \pview{(t' \cdot n) \upharpoonright  r } & (n \mbox{ is not hereditarily justified by } r) \\
        &= \pview{?(t) \upharpoonright  r  }           & (\mbox{definition of } ?(t))
    \end{align*}

    \item If $n$ is hereditarily justified by $r$ then:
    \begin{align*}
    \pview{?(t)} \upharpoonright  r
    &= (\pview{t'} \cdot n) \upharpoonright  r      & (\mbox{equation \ref{eq_tprime}}) \\
    &= (\pview{t'} \upharpoonright  r  ) \cdot n    & (n \mbox{ is hereditarily justified by } r)\\
    &= \pview{t' \upharpoonright  r } \cdot n       & (\mbox{induction hypothesis}) \\
    &= \pview{(t' \upharpoonright  r ) \cdot n }    & (\mbox{P-view computation}) \\
    &= \pview{(t' \cdot n) \upharpoonright  r  }    & (n \mbox{ is hereditarily justified by } r) \\
    &= \pview{?(t) \upharpoonright  r  }               & (\mbox{definition of } ?(t))
    \end{align*}
    \end{itemize}


\item Suppose that $?(t) =  t' \cdot \rnode{m}{m} \cdot  u \cdot \rnode{lmd}{n}
    \bkptr[nodesep=1pt]{30}{lmd}{m}$ where $n$ is a $\lambda$-node then by lemma
    \ref{lem:var_followedby_child} we have $u = \epsilon$ and therefore:
        \begin{align*}
        \pview{?(t)} \upharpoonright  r
        &= \pview{t' \cdot \rnode{m}{m} \cdot \rnode{n}{n}} \upharpoonright  r
               \bkptr[nodesep=1pt]{60}{n}{m}                   & (u=\epsilon)\\
        &= (\pview{t'} \cdot \rnode{m}{m} \cdot \rnode{lmd}{n} ) \upharpoonright  r
               \bkptr[nodesep=1pt]{60}{lmd}{m}                 & (\mbox{P-view computation}) \\
        &= \pview{t'} \upharpoonright  r                & (m, n \mbox{ are not hereditarily justified by } r) \\
        &= \pview{t' \upharpoonright  r }               & \mbox{(induction hypothesis)} \\
        &= \pview{ (t' \cdot \rnode{m}{m} \cdot \rnode{lmd}{n}) \upharpoonright r }
                        \bkptr[nodesep=1pt,ncurv=0.7]{40}{lmd}{m}
                                                            & (\mbox{def. operator $\upharpoonright$ and } m, \lambda \mbox{ are not her. just. by } r) \\
        &= \pview{ ?(t) \upharpoonright r }                & \mbox{(def. of $?(t)$)}
        \end{align*}
\end{itemize}
(ii) By a straightforward induction similar to (i).
\end{proof}

\begin{lem}[Traversal of $\beta$-normal terms]
\label{lem:betaeta_trav}
Let $M$ be a $\beta$-normal term, $r$ be the root of the tree $\tau(M)$ and
$t$ be a traversal of $\tau(M)$.
For any node $n$ occurring in $t$:
\begin{eqnarray*}
r \mbox{ does not hereditarily justify } n  \  \iff \   n \mbox{ is
hereditarily justified by some node in } N_\Sigma.
\end{eqnarray*}
%\begin{itemize}
%\item[(i)]
%for any node $n$ occurring in $t$:
%\begin{eqnarray*}
%r \mbox{ does not hereditarily justify } n  \  \iff \   n \mbox{
%is hereditarily justified by some node in } N_\Sigma;
%\end{eqnarray*}
%\item[(ii)] For any $\lambda$-node $n$ occurring in $t$, $t_{\geq n} \in \travset(\kappa(n))$,
%
% where $t_{\geq n}$ denotes the justified sequence of nodes obtained by taking the suffix of $t$ starting at $n$ and
% such that any dangling link going from a variable node to a node preceding $n$ is ``fixed'' into a pointer going to $n$.
% \end{itemize}
\end{lem}
\begin{proof}
%(i)
 In a computation tree, the only nodes that do not have justification pointer are:
the root $r$, @-nodes and $\Sigma$-constant nodes. But since $M$ is
in $\beta$-normal form, there is no @-node in the computation tree.
Hence nodes are either hereditarily justified by $r$ or hereditarily
justified by a node in $N_\Sigma$. Moreover $r$ is not in $N_\Sigma$
therefore the ``or'' is exclusive : a node cannot be hereditarily
justified at the same time by $r$ and by some node in $N_\Sigma$.

%(ii) Since $M$ is in $\beta$-normal, the rules (App) and (Var) cannot be used. Therefore the traversals
%follow an inductive exploration of the computation tree without making any ``jump''.
%Consequently, by taking the prefix of $t$ starting at a $\lambda$-node, we obtain
%a traversal of a sub-computation tree of $\tau(M)$. However by taking the prefix we obtain some dangling pointers.
%The ``fix'' applied to the dangling pointers correspond to the
%The formal proof is by an easy induction on the traversal rules. For the constant rules, we appeal to well-behaviour of the rules.

\end{proof}


\section{Game semantics of simply-typed $\lambda$-calculus with $\Sigma$-constants}
\label{sec:assumptions}

We are working in the general setting of an applied simply-typed $\lambda$-calculus with a given set of higher-order constants $\Sigma$.
The operational semantics of these constants is given by certain reduction rules.
We assume that a fully abstract model of the calculus is provided by mean of a category of well-bracketed games.
For instance, if $\Sigma$ is the set of \pcf\ constants then we work in the category $\mathcal{C}_{b}$
of well-bracketed defined in section \ref{subsec:pcfgamemodel} of the first chapter.

We will use the alternative representation of strategy defined in remark \ref{rem:atlern_strategy}: a
strategy is given by a prefix-closed set instead of an ``even length
prefix''-closed set. In practice this means that we replace the set
of plays $\sigma$ by $\sigma \union \textsf{dom}(\sigma)$. This
permits to avoid considerations on the parity of the length of
traversals when we show the correspondence between traversals and
game semantics. We write $\sem{\Gamma \vdash M : A}$ for the strategy denoting the simply-typed term
$\Gamma \vdash M : A$ and $\prefset(S)$ to denote the
prefix-closure of the set $S$.


\subsection{Relationship between computation trees and arenas}

\subsubsection{Example}
Consider the following term $M \equiv \lambda f z . (\lambda g x . f (f x)) (\lambda y. y) z$ of type $(o \typear o) \typear o \typear o$.
Its $\eta$-long normal form is $\lambda f z . (\lambda g x . f (f x)) (\lambda y. y) (\lambda .z)$.
The computation tree is:

$$
\tree{\lambda f z}
{ \tree{@}
    {
        \tree{\lambda g x}
            { \tree{f}{   \tree{\lambda}{ \tree{f}{  \tree{\lambda}{\TR{x}}} }  }
            }
        \tree{\lambda y}{\TR{y}}
        \tree{\lambda}{\TR{z}}
    }
}
$$

The arena for the type $(o \typear o) \typear o \typear o$ is:
$$\tree{q^1}
{
    \tree{q^3}
        {  \tree{q^4}
                {\TR{a^4_1} \TR{\ldots}}
            \TR{a^3_1} \TR{\ldots} }
    \tree{q^2}
    { \TR{a^2_1} \TR{a^2_2}\TR{\ldots} }
    \TR{a_1} \TR{a_2}\TR{\ldots}
}
$$

\newlength{\yNull}
\def\bow{\quad\psarc{->}(0,\yNull){1.5ex}{90}{270}}

The figure below represents the computation tree (left) and the
arena (right). The dashed line defines a partial function $\varphi$
from the set of nodes in the computation tree to the set of moves.
For simplicity, we now omit answers moves when representing arenas.
$$
\tree{ \Rnode{root} {\lambda f z}^{[1]} }
     {  \tree{@^{[2]}}
        {   \tree{\lambda g x ^{[3]}}
                { \tree{\Rnode{f}{f^{[6]}}}{  \tree{\Rnode{lmd}\lambda^{[7]}}{ \tree{\Rnode{f2}{f^{[8]}}} {\tree{\Rnode{lmd2}\lambda^{[9]}}{\TR{x^{[10]}}}}}  }
                }
            \tree{\lambda y ^{[4]}}{\TR{y}}
            \tree{\lambda ^{[5]}}{\TR{\Rnode{z}z}}
        }
    }
\hspace{3cm}
  \tree[levelsep=12ex]{ \Rnode{q1}q^1 }
    {   \pstree[levelsep=4ex]{\TR{\Rnode{q3}q^3}}{\TR{\Rnode{q4}q^4}}
        \TR{\Rnode{q2}q^2}
        \TR{\Rnode{q5}q^5}
    }
\psset{nodesep=1pt,arrows=->,arcangle=-20,arrowsize=2pt 1,linestyle=dashed,linewidth=0.3pt}
\ncline{->}{root}{q1} \aput*{:U}{\varphi}
\ncarc{->}{z}{q2}
\ncline{->}{f}{q3}
\ncline{->}{lmd}{q4}
\ncline{->}{f2}{q3}
\ncline{->}{lmd2}{q4}
$$

Consider the justified sequence of moves $s \in \sem{M}$:
\vspace{0.2cm}
 $$s =
\rnode{q1}{q}^1\ \rnode{q3}{q}^3\ \rnode{q4}{q}^4\ \rnode{q3b}{q}^3\ \rnode{q4b}{q}^4\ \rnode{q2}{q}^2
\bkptr[offset=-3pt]{60}{q3}{q1}
\bkptr[offset=-3pt,ncurv=0.5]{60}{q3b}{q1}
\bkptr[offset=-3pt]{60}{q4}{q3}
\bkptr[offset=-3pt]{60}{q4b}{q3b}
\bkptr[offset=-3pt,ncurv=0.5]{60}{q2}{q1}
\in \sem{M}$$

There is a corresponding justified sequence of nodes in the computation tree:
\vspace{0.5cm}
$$r =
\rnode{q1}{\lambda f z} \cdot
\rnode{q3}{f}^{[6]} \cdot
\rnode{q4}{\lambda^{[7]}} \cdot
\rnode{q3b}{f}^{[8]} \cdot
\rnode{q4b}{\lambda^{[9]}} \cdot
\rnode{q2}{z}
\bkptr[ncurv=1]{60}{q3}{q1}
\bkptr[ncurv=1]{60}{q4}{q3}
\bkptr[ncurv=0.5]{75}{q3b}{q1}
\bkptr[ncurv=1]{50}{q4b}{q3b}
\bkptr[ncurv=0.4]{80}{q2}{q1}$$
such that $s_i = \varphi(r_i)$ for all $i < |s|$.

The sequence $r$ is in fact the reduction of the following
traversal: \vspace*{1cm}
$$t = \rnode{q1}{\lambda f
z} \cdot \rnode{n2}{@^{[2]}} \cdot \rnode{n3}{\lambda g x^{[3]}}
\cdot \rnode{q3}{f}^{[6]} \cdot \rnode{q4}{\lambda^{[7]}} \cdot
\rnode{q3b}{f}^{[8]} \cdot \rnode{q4b}{\lambda^{[9]}} \cdot
\rnode{n8}{x^{[10]}} \cdot \rnode{n9}{\lambda^{[5]}} \cdot
\rnode{q2}{z} \bkptr[ncurv=0.6]{60}{q3}{q1}
\bkptr[ncurv=1]{60}{q4}{q3} \bkptr[ncurv=0.4]{75}{q3b}{q1}
\bkptr[ncurv=0.8]{70}{q4b}{q3b} \bkptr[ncurv=0.4]{80}{q2}{q1}
\bkptr[ncurv=0.4]{60}{n3}{n2} \bkptr[ncurv=0.4]{60}{n8}{n3}
\bkptr[ncurv=0.4]{60}{n9}{n2}.
$$

By representing side-by-side the computation tree and the type arena of a term in $\eta$-normal form we have observed
that some nodes of the computation tree can be mapped to question moves of the arena.
In the next section, we show how to define this mapping in a systematic manner.

\subsubsection{Formal definition}

Let us establish precisely the relationship between arenas of the
game semantics and the computation trees. Let $\Gamma \vdash M : A$
be a term in $\eta$-long normal form. The computation tree $\tau(M)$
is represented by a pair $(V,E)$ where $V$ is the set of vertices of
the trees and $E$ is the edges relation. $V = N \union L$ where $N$
is the set of nodes and $L$ is the set of value-leaves.

The relation $E \subseteq V \times V$ gives the parent-child relation on the vertices of the tree.
We write $V_\$$ for $N_\$ \union (E(N_\$) \inter L)$ where $\$$ ranges over $\{@, var, \Sigma, fv \}$.


Let $\mathcal{D}$ be the set of values of the base type $o$. If $n$
is a node in $N$ then the value-leaves in
$E_l(n)$ attached to the node $n$ are written $v_n$ where $v$ ranges in $\mathcal{D}$.
Similarly, if $q$ is a question in $\sem{A}$ then the answer moves
enabled by $q$ are written $v_q$ where $v$ ranges in $\mathcal{D}$.

If $A$ is an arena and $q$ is a move in $A$ then we write $A_q$ to
denote the subarena of $A$ rooted at $q$.

\begin{dfn}[Relation between moves of the arena and nodes of the computation tree]
\label{def:phi_procedure}
We consider the computation tree of a simply-typed-term.
For any arena $A$, we define a function $f_A(n,q)$ taking two parameters:
a node $n$ of the computation tree and a question move $q$ of the arena $A$
such that $q$ and $n$ have the same type.
$f_A(n,q)$ returns a partial function from $V$ to $A$. It is defined as follows:
\noindent
\begin{itemize}
\item[case 1] If $n$ is labelled labelled $\lambda$ or is a ground type variable node then
        $$f_A(n,q) = \{ n \mapsto q \} \quad \union \quad  \{ v_n \mapsto v_q \ | \ v \in \mathcal{D} \}$$

\item[case 2]  If $n$ is a $\lambda$-node labelled $\lambda \overline{\xi} = \lambda \xi_1 \ldots \xi_p$ with $p\geq 1$ and with a child node labelled $\alpha$ and $\vdash( q ) = \{ q^1, \ldots, q^m \} \union \{  v_q : v \in \mathcal{D} \} $ then the computation tree and the arena $A_q$ have the following form
    (value-leaves and answer moves are not represented for simplicity):
    $$ \tree{ \Rnode{r}\lambda \overline{\xi}  ^{[n]}}
        {
            \tree[levelsep=6ex]{\alpha}
            {   \TR{\ldots} \TR{\ldots} \TR{\ldots}
            }
        }
    \hspace{3cm}
    \tree{ \Rnode{q0}q }
        {
            \tree[linestyle=dotted]{q^1}{\TR{} \TR{} }
            \tree[linestyle=dotted]{q^2}{\TR{} \TR{} }
            \TR{\ldots}
            \tree[linestyle=dotted]{q^p}{\TR{} \TR{} }
        }
    \psset{nodesep=1pt,arrows=->,arcangle=-20,arrowsize=2pt 1,linestyle=dashed,linewidth=0.3pt}
    \ncline{->}{r}{q0}
    \ncarc{->}{q2}{z}
    \ncline{->}{q3}{f}
    \ncline{->}{q4}{lmd}
    \ncline{->}{q3}{f2}
    \ncline{->}{q4}{lmd2}
    $$

    For each of the abstracted variable $\xi_i$ there exists a corresponding question move $q^i$ of the same order
    in the arena.  $f_A(n,q)$ maps each free occurrence of a variable $\xi_i$ to the corresponding move $q^i$:
    $$
    f_A(n,q) =  \{ n \mapsto q \} \quad  \union \quad  \{ v_n \mapsto v_q \ | \ v \in \mathcal{D} \}
                      \quad \union \quad  \Union_{\stackrel{\displaystyle m \in N | n \vdash m}{\displaystyle m \mbox{ labelled } \xi_i}} f_A( m, q^i)
    $$

\item[case 3] If $n$ is a variable node $x$ of higher-order type $(A_1,\ldots,A_m,o)$
with children nodes $\lambda \overline{\eta}_1$, \ldots, $\lambda \overline{\eta}_m$ and
$\vdash( q ) = \{ q^1, \ldots, q^m \} \union \{  v_q : v \in \mathcal{D} \} $ then the computation tree and the arena $A_q$ have the following form:
    $$\tree{\Rnode{r}{x^{[n]}}}
        {   \tree{\TR{\lambda \overline{\eta}_1}}{\vdots} \TR{\ldots}
        \tree{\TR{\lambda \overline{\eta}_m }}{\vdots}
        }
    \hspace{3cm}
    \tree{ \Rnode{q0}q }
        {
            \tree[linestyle=dotted]{\Rnode{q1}{q^1}}{\TR{} \TR{} }
            \tree[linestyle=dotted]{\Rnode{q2}{q^2}}{\TR{} \TR{} }
            \TR{\ldots}
            \tree[linestyle=dotted]{\Rnode{qm}{q^m}}{\TR{} \TR{} }
        }
    \psset{nodesep=1pt,arrows=->,arcangle=-20,arrowsize=2pt 1,linestyle=dashed,linewidth=0.3pt}
    \ncline{->}{r}{q0}
    \ncarc{->}{q2}{z}
    \ncline{->}{q3}{f}
    \ncline{->}{q4}{lmd}
    \ncline{->}{q3}{f2}
    \ncline{->}{q4}{lmd2}
    $$

    $f_A(n,q)$ maps each child node of $n$ to the corresponding question move $q^i$ of the same type
    in the arena $A_q$:
    $$f_A(n,q) =
         \{ n \mapsto q \} \quad \union\quad \{ v_n \mapsto v_q \ | \ v \in \mathcal{D}   \} \quad\union\quad     \Union_{i=1..m} f_A( \lambda \overline{\eta}_i, q^i)
    $$
\end{itemize}

Note that $f_A(n,q)$ is only a partial function from $V$ to $A$ since it is defined only
on nodes that are hereditarily justified by the root \emph{and} not hereditarily justified by a free variable node.
In other words, $f_A(n,q)$ is undefined on nodes that are hereditarily justified by $N_{fv} \union N_@ \union N_\Sigma$.
\end{dfn}

We write $\mathcal{M}_M$ to denote the following disjoint union of arenas:
$$\mathcal{M}_M = \sem{\Gamma \rightarrow T} \quad \uplus \quad  \biguplus_{n \in N \inter E\relimg{N_@ \union N_\Sigma} } \sem{type(\kappa(n))}.$$

Moves in $\mathcal{M}_M$ are implicitly tagged so it is possible to recover the arena in which they belong.


\begin{dfn}[Total mapping from nodes to moves]
Let $\Gamma \vdash M : T$ be a simply-typed term
with $\Gamma = x_1:X_1 \ldots x_p : X_p$.
We write $q_{\sem{\Gamma}}^1$, \ldots, $q_{\sem{\Gamma}}^p$ to denote the initial question moves of the
component $\Gamma$ of the arena $\sem{\Gamma \rightarrow T}$ and $q^0_A$ to denote the single initial question of any arena $A$
(arenas involved in the game semantics of pure simply-typed $\lambda$-calculus have only one root).
$r$ denotes the root of the computation tree.

We define the total function $\varphi_M : V_\lambda \union V_{var} \rightarrow \mathcal{M}_M$ as follows:
\begin{align*}
\varphi_M =
        f_{\sem{\Gamma \rightarrow T}}(r, q^0_{\sem{\Gamma \rightarrow T}}) \quad
    & \union \quad
    \Union_{n \in N_{fv} | n \mbox{ {\small labelled} } x_i }  f_{\sem{\Gamma \rightarrow T}}(n, q^i_{\sem{\Gamma}} ) \\
    & \union \quad
        \Union_{n \in N \inter E \relimg{N_@ \union N_\Sigma}}  f_{\sem{type(\kappa(n))}}(n, q^0_{\sem{type(\kappa(n))}} )
\end{align*}
When there is no ambiguity we just write $\varphi$ instead of $\varphi_M$.
\end{dfn}

Nodes of $\tau(M)$ are either hereditarily justified by the root, by
a @-node or by a $\Sigma$-node, therefore $\varphi_M$ is totally
defined on $V_\lambda \union V_{var} = V\setminus (V_@ \union
V_\Sigma)$.

\begin{exmp}
Consider the term $\lambda x . (\lambda g . g x) (\lambda y . y)$ with $x,y:o$ and $g:(o,o)$.
The diagram below represents the computation tree (middle), the arenas
$\sem{(o,o)\rightarrow o}$ (left), $\sem{o \rightarrow o}$ (right), $\sem{o\rightarrow o}$ (rightmost)
and the function $\varphi = f(\lambda x, q_{\lambda x}) \union f(\lambda g, q_{\lambda g}) \union f(\lambda y, q_{\lambda y})$
(dashed-lines).
$$
\psset{levelsep=4ex}
\pstree{\TR[name=root]{\lambda x}}
{
    \pstree{\TR[name=App]{@}}
    {
            \pstree{\TR[name=lg]{\lambda g}}
                { \pstree{\TR[name=lgg]{g}}{
                        \pstree{\TR[name=lgg1]{\lambda}}
                        { \TR[name=lgg1x]{x}  } } }
            \pstree{\TR[name=ly]{\lambda y}}
                    {\TR[name=lyy]{y}}
    }
}
\rput(5cm,-1cm){
  \pstree{\TR[name=A1lx]{q_{\lambda x}}}
        { \TR[name=A1x]{q_x} }
}
\rput(-6cm,-1.5cm){
    \pstree{\TR[name=A2lg]{q_{\lambda g}}}
    {
        \pstree{\TR[name=A2g]{q_g}}
        {  \TR[name=A2g1]{q_{g_1}}   }
    }}
\rput(2.5cm,-1.5cm){
    \pstree{\TR[name=A3ly]{q_{\lambda y}}}
        { \TR[name=A3y]{q_y}
        }
}
\psset{nodesep=1pt,arrows=->,arcangle=-20,arrowsize=2pt 1,linestyle=dashed,linewidth=0.3pt}
\ncline{->}{root}{A1lx} \mput*{f(\lambda x, q_{\lambda x})}
\ncarc{->}{lgg1x}{A1x}
\ncline{->}{lg}{A2lg} \mput*{f(\lambda g, q_{\lambda g})}
\ncline{->}{lgg}{A2g}
\ncline{->}{lgg1}{A2g1}
\ncline{->}{ly}{A3ly} \mput*{f(\lambda y, q_{\lambda y})}
\ncline{->}{lyy}{A3y}
$$
\end{exmp}

The following properties are immediate consequences of the definition of the procedure $f$:
\begin{property} \
\label{proper:phi_conserve_order}
\begin{itemize}
\item[(i)] $\varphi$ maps $\lambda$-nodes to O-questions, variable nodes to
P-questions, value-leaves of $\lambda$-nodes to P-answers and
value-leaves of variable nodes to O-answers;
\item[(ii)] $\varphi$ maps nodes of a given order to moves of the same order.
\end{itemize}
\end{property}
Remark: we recall that in definition \ref{def:nodeorder}, the
node-order is defined differently for the root $\lambda$-node and
other $\lambda$-nodes. This convention was chosen to guarantee that
property (ii) holds.

By extension, the function $\varphi$ is also defined on justified
sequences of nodes: if $t = t_0 t_1 \ldots$ is a justified sequence
of nodes in $V_\lambda \union V_{var}$ then $\varphi(t)$ is defined
to be the following sequence of moves:
$$\varphi(t) = \varphi(t_0)\ \varphi(t_1)\  \varphi(t_2) \ldots$$
where the pointers of $\varphi(t)$ are defined to be exactly those
of $t$. This definition implies that $\varphi : (V_\lambda \union
V_{var})^* \rightarrow \mathcal{M}^*$ regarded as a function from
pointer-less sequences of nodes to pointer-less sequences of moves
is a monoid homomorphism.

\begin{property}
\label{proper:phi_pview} Let $t$ be a justified sequence of nodes. The following properties hold:
\begin{itemize}
\item[(i)] $\varphi(t)$ and $t$ have the same pointers;
\item[(ii)] the P-view of $\varphi(t)$ and the P-view of $t$ are computed
identically: the set of indices of elements that must be removed
from both sequences in order to obtain their P-view is the same;
\item[(iii)] the O-view of $\varphi(t)$ and the O-view of $t$ are computed identically;
\item[(iv)] if $t$ is a justified sequence of nodes in $V_\lambda \union V_{var}$ then $?(\varphi(t)) =
\varphi(?(t))$,
\end{itemize}
where $?(\varphi(t))$ denotes the subsequence of $\varphi(t)$ consisting of the unanswered questions 
and $?(t)$ denotes the subsequence of $t$ consisting of the unmatched nodes (see the
definition in section \ref{sec:adding_value_leaves}).
\end{property}
\begin{proof}
(i): By definition of $\varphi$, $t$ and $\varphi(t)$ have the same
pointers;

(ii) and (iii): $\varphi$ maps lambda nodes to O-question,
non-lambda nodes to P-question, value-leaves of lambda nodes to P-answers and
value-leaves of non-lambda to O-answers. Therefore since $t$ and $\varphi(t)$ have the
same pointers, the computations of the P-view (resp. O-view) of the
sequence of moves and the P-view (resp. O-view) of the sequence of
nodes follow the same steps;

(iv) is a consequence of (i).

\end{proof}


\subsection{Category of interaction games}
\label{sec:interaction_semantics}

In game semantics, strategy composition is achieved by performing a
CSP-like ``composition + hiding''. It is possible to define an
alternative semantics where the internal moves are not hidden when
performing composition. This semantics is named \emph{interaction}
semantics in \cite{DBLP:conf/sas/DimovskiGL05} and \emph{revealed
semantics} in \cite{willgreenlandthesis}.

In addition to the moves of the standard semantics, the interaction semantics contains certain
internal moves of the computation.
Consequently, the interaction semantics depends on the syntactical structure of the term and therefore cannot
lead to a full abstraction result. However this semantics will prove to be useful to identify
a correspondence between the game semantics
of a term and the traversals of its computation tree.

We will be interested in the interaction semantics computed from the
$\eta$-normal form of a term. However we do not want to keep all the internal moves. We will only keep the internal
moves that are produced when composing two subterms of the computation tree that are joined by an @-node.
This means that when computing the strategy of
$y N_1 \ldots N_p$ where $y$ is a variable, we keep the internal moves of $N_1$, \ldots, $N_p$, but
we omit the internal moves produced by the copy-cat projection strategy denoting $y$.

\begin{dfn}[Type-tree]
We call \emph{type decomposition tree} or \emph{type-tree}, a tree whose leaves are labelled with linear simple types
and nodes are labelled with symbol in $\{ ;, \times, \otimes, \dagger, \Lambda \}$.

Nodes labelled $;$, $\times$ or $\otimes$ are binary nodes and nodes labelled $\dagger$ or $\Lambda$ are unary nodes.

Every node or leaf of the tree has a linear type, this type is determined by the
structure of the tree as follows:
\begin{itemize}
\item a leaf has the type of its label;

\item a $\dagger$-node with the child node of type $!A \multimap B$ has type $!A \multimap !B$;

\item a $\Lambda$-node with the child node of type $A \otimes B \multimap C$ has type $A \multimap (B \multimap C)$;

\item a $\times$-node with two children nodes of type $A$
and $B$ has type $A \times B$;

\item a $\otimes$-node with two children nodes of type $A$
and $B$ has type $A\otimes B$;

\item a $;$-node with two children nodes of type $A\multimap B$
and $B \multimap C$ has type $A \multimap C$.
\end{itemize}

For a type-tree to be well-defined, the type of the children nodes
must be compatible with the meaning of the node, for instance the
two children nodes of a ;-node must be of type
$A\multimap B$ and $B\multimap C$.

We write $type(T)$ to denote the type represented by the root of the tree $T$. An we say that $T$ is a \emph{valid
tree decomposition} of $type(T)$.

If $T_1$ and $T_2$ are type-tree we write $T_1 \times T_2$ to denote the tree obtained by attaching $T_1$ and $T_2$ to a $\times$-node.
Similarly we use the notations $T_1 \otimes T_2$, $T_1 ; T_2$, $\Lambda(T_1)$ and  $T_1^\dagger$.
\end{dfn}


Let $T$ be a type-tree. Each leaf or node of type $A$ in $T$ can be mapped to the
(standard) arena $\sem{A}$. By taking the image of $T$ across this mapping we obtain a tree whose leaves and nodes are labelled by arenas.
This tree, written $\intersem{T}$, is called the \emph{interaction arena} of type $T$.
We write $root(\intersem{T})$ to denote the arena located at the root of the interaction arena $\intersem{T}$.

A \emph{revealed strategy} $\Sigma$ on the interaction arena
$\intersem{T}$ is a composition of several standard strategies where
certain internal moves are not hidden. Formally this can be defined
as follows:
\begin{dfn}[Revealed strategy]
A revealed strategy $\Sigma$ on a game $\intersem{T}$, written
$\Sigma: \intersem{T}$, is a tree type $T$ where
\begin{itemize}
\item each leaf $\sem{A}$ of
$\intersem{T}$ is annotated with a (standard) strategy $\sigma$ on the
game $\sem{A}$;
\item each $;$-node is annotated with a set of indices $U \subseteq \nat$.
\end{itemize}
\end{dfn}
A $;$-node with children of type $A\multimap B$ and $B\multimap C$ is annotated with a set of indices $U$ indicating
which components of $B$ should be uncovered when performing composition.
\begin{exmp}
The diagrams below represent a type-tree $T$ (left) the
corresponding interaction arena $\intersem{T}$ (middle) and an
revealed strategy $\Sigma$ (right):
$$
\pstree[levelsep=6ex]{\TR{;}}
        {
            \pstree[levelsep=6ex]{\TR{;}}
            { \TR{A\multimap B}
              \TR{B\multimap C}
            }
            \TR{C\multimap D}
        }
\hspace{1cm}
\pstree[levelsep=6ex]{\TR{;}}
        {
            \pstree[levelsep=6ex]{\TR{;}}
            { \TR{\sem{A\multimap B}}
              \TR{\sem{B\multimap C}}
            }
            \TR{\sem{C\multimap D}}
        }
\hspace{1cm}
\pstree[levelsep=6ex]{\TR{;^{\{0\}}}}
        {
            \pstree[levelsep=6ex]{\TR{;^{\{0\}}}}
            { \TR{A\multimap B^{\sigma_1}}
              \TR{B\multimap C^{\sigma_2}}
            }
            \TR{C\multimap D^{\sigma_3}}
        }
$$
\end{exmp}
A revealed strategy can also be written as an expression, for
instance the strategy represented above is given by the expression
$\Sigma = (\sigma_1 ;^{\{0\}} \sigma_2) ;^{\{0\}} \sigma_3$. We will
use the abbreviation $\Sigma_1 \fatsemi^U \Sigma_2$ for
$\Sigma_1^\dagger ; ^U \Sigma_2$.

\begin{dfn}[Composition of revealed strategies]
Suppose $\Sigma_1 : \intersem{T_1}$ and $\Sigma_2 : \intersem{T_2}$
are revealed strategies where $type(T_1) = A \multimap B$ and
$type(T_2) = B \multimap C$ then the \emph{interaction composition}
of $\Sigma_1$ and $\Sigma_2$ written $\Sigma_1 ; \Sigma_2$ is the
revealed strategy on $\intersem{T_1 ; T_2}$ obtained by copying the
annotation of the leaves and nodes from $\Sigma_1$ and $\Sigma_2$ to
the corresponding leaves and nodes of the type-tree $T_1 ; T_2$ and
by annotating the root node with $\emptyset$.
\end{dfn}

A play of the interaction semantics, called an \emph{uncovered
play}, is a play containing internal moves.
The moves are implicitly tagged so that it is possible to retrieve in which component
of which node or leaf-arenas the move belongs to. Note that a same move can belong to different node/leaf-arenas.
The internal moves of an interaction play on the game $\intersem{T}$ are those which do not
belong to the arena $root(\intersem{T})$.

For any uncovered play $s$ and any interaction arena $\intersem{T}$
we can define the filtering operator $s\upharpoonright \intersem{T}$ to be the
sequence of moves obtained from $s$ by keeping only the moves
belonging to a node or leaf-arena of $\intersem{T}$.


Revealed strategies can alternatively be represented by mean of sets
of uncovered plays instead of annotated type-trees. This set is
defined inductively on the structure of the annotated type-tree
$\Sigma$ as follows:
\begin{itemize}
\item for a leaf $\sem{A}$ of $\Sigma$ annotated by $\sigma :\sem{A}$, it is just the set of plays of the standard strategy $\sigma$;
\item for a $\otimes$-node with two children strategies $\Sigma_1$ and $\Sigma_2$, it is the tensor product written $\Sigma_1 \otimes \Sigma_2$;
\item for a $\times$-node, it is the pairing written $\langle \Sigma_1, \Sigma_2 \rangle$;
\item for a $\dagger$-node with a child strategy $\Sigma$, it is the promotion written $\Sigma^\dagger$;
\item for a $\Lambda$-node with a child strategy $\Sigma$, it is the same set of plays with the moves retagged appropriately;

\item for a $;^U$-node, it is the ``uncovered-composition'' of $\Sigma_1 : \intersem{T_1}$ and $\Sigma_2 :\intersem{T_2}$ which is written $\Sigma_1
;^U \Sigma_1$ and defined as follows: suppose that $type(T_1) = A
\multimap B_0 \times \ldots \times B_l$ and $type(T_2) = B_0 \times
\ldots \times B_l \multimap C$ then $\Sigma_1 ;^U \Sigma_1$ is the
set of uncovered plays obtained by performing the usual composition
while ignoring and copying the internal moves from arenas in
$\intersem{T_1}$ or $\intersem{T_2}$ and preserving any internal
move produced by the composition in some component $B_k$ for $k \in
U$. Formally:
$$ \Sigma_1 \| \Sigma_2 = \{ u \in int(\intersem{T}) \ | \ u \upharpoonright \intersem{T_1} \in \Sigma_1 \mbox{ and } u \upharpoonright \intersem{T_2} \in \Sigma_2 \}$$
$$ \Sigma_1 ;^{\{i_0, \ldots i_l\}} \Sigma_2 = \{ u \upharpoonright A, B_{i_0}, \ldots, B_{i_l}, C \ | \ u \in \Sigma_1 \| \Sigma_2 \}$$
where $int(\intersem{T})$ denotes the set of sequences of moves in (some arena of) $\intersem{T}$;
\end{itemize}
where the tensor product, pairing and promotion are defined similarly as in the standard game semantics.


We can now define the category $\mathcal{I}$ of interaction games:
\begin{dfn}[Category of interaction games]
The category of interaction games is denoted by $\mathcal{I}$. The
objects of $\mathcal{I}$ are those of $\mathcal{C}$ i.e. the arenas
$\sem{A}$ for some linear type $A$. The morphisms of the category
are the revealed strategies: a morphism from $A$ to $B$ is an
revealed strategy $\Sigma$ on some interaction arena $\intersem{T}$
such that $root(\intersem{T}) = \sem{!A\multimap B}$.

The composition of two morphisms $\Sigma_1$ and $\Sigma_2$ is given
by $\Sigma_1 \fatsemi \Sigma_2 = \Sigma_1^\dagger ; \Sigma_2$ where
$;$ denotes the revealed strategy composition. The identity on $A$
is the revealed strategy given by the single annotated leaf $\sem{!A
\multimap A}^{der_A}$.
\end{dfn}

It can be checked that this indeed defines a category. The constructions of the category $\mathcal{C}$ can be transposed to $\mathcal{I}$
making $\mathcal{I}$ a cartesian closed category.


\begin{dfn}[Valid strategy]
Consider a term $\Gamma \vdash M : A$ and an revealed strategy
$\Sigma : \intersem{T}$. We say that $\Sigma$ is a valid revealed
strategy for $M$ if $root(\intersem{T}) = \sem{\Gamma \rightarrow
A}$ or equivalently if $type(T) = \Gamma \rightarrow A$.
\end{dfn}


\subsubsection{Modeling the $\lambda$-calculus in $\mathcal{I}$}

We would like to use the category $\mathcal{I}$ to model terms of
the simply-typed lambda calculus.
Depending on the internal moves that we wish to hide, we obtain different possible interaction strategies for a given term.
We now fix a unique strategy denotation which is computed from the $\eta$-normal form of the term.

\begin{dfn}[Revealed denotation of a term]
\label{dfn:interactionstrategy_ofterms}
The \emph{revealed game denotation} or \emph{revealed
strategy} of $M$ written $\intersem{\Gamma \vdash M : A}$ is defined by structural induction on the $\eta$-long normal form of $M$ as follows:

Let $\overline{\xi} = \xi_1 : Y_1, \ldots \xi_n : Y_n$
and $z$ be a variable ranging in $\Gamma \union \overline{\xi}$. If $z\in \Gamma$ then $\pi_{z}$ denotes
the $i^{th}$ projection copycat strategy $\pi_i : \sem{\Gamma \union \overline{\xi}} \rightarrow \sem{X_i}$. If $z = \xi_j$ then
$\pi_{z}$ denotes the $(n+j)^{th}$ projection $\pi_{n+j} : \sem{\Gamma \union \overline{\xi}} \rightarrow \sem{Y_j}$.
\begin{eqnarray*}
\intersem{\Gamma \vdash \lambda \overline{\xi} . z } &=& \Lambda^n(\pi_{z})  \\
\intersem{\Gamma \vdash \lambda \overline{\xi} . z N_1 \ldots N_p} &=& \Lambda^n(\langle \pi_z, \intersem{\Gamma \vdash N_1 : A_1}, \ldots, \intersem{\Gamma \vdash N_p : A_p}  \rangle \fatsemi ^{\emptyset} ev^p) \\
\intersem{\Gamma \vdash \lambda \overline{\xi}. f N_1 \ldots N_p} &=& \langle \intersem{\Gamma \vdash N_1}, \ldots, \intersem{\Gamma \vdash N_p} \rangle \fatsemi^{0..p-1} \sem{f} \\
\intersem{\Gamma \vdash \lambda \overline{\xi} . N_0 \ldots N_p} &=& \Lambda^n(\langle \intersem{\Gamma \vdash N_0 : A_0}, \ldots, \intersem{\Gamma \vdash N_p : A_p}  \rangle \fatsemi^{0..p} ev^p)
\end{eqnarray*}
where $\Gamma \vdash N_0 : (A_1,\ldots,A_p,B)$, $\Gamma \vdash z : (A_1,\ldots,A_p,B)$, $\Gamma \vdash N_k : A_k$ for $k\in 1..p$,
$f : (A_1,\ldots,A_p,B) \in \Sigma$ and $ev^p$ denotes the evaluation strategy with $p$ parameters.

We write $\intersem{\Gamma \rightarrow A}_M$ to denote the
interaction arena of the revealed strategy $\intersem{\Gamma \vdash
M : A}$.
\end{dfn}
Note that when computing $\intersem{z N_1 \ldots N_p}$, for some variable $z$, the internal moves of $N_1$, \ldots, $N_p$ are preserved but
we omit the internal moves produced by the copy-cat projection strategy denoting $z$.



%\begin{dfn}[Revealed denotation of a term]
%\label{dfn:interactionstrategy_ofterms} Let $\Gamma \vdash M : A$ be
%a term with $\Gamma = x_1:X_1, \ldots, x_k:X_k$. Let $\pi_i :
%\sem{\Gamma \rightarrow X_i}$ denote the $i$th projection copycat
%strategy and $ev^p$ denote the evaluation strategy with $p$
%parameters.
%
%The \emph{revealed game denotation of $M$} or \emph{revealed
%strategy of $M$} written $\intersem{\Gamma \vdash M : A}$ is the
%revealed strategy defined by structural induction on the computation
%tree $\tau(M)$ as follows:
%
%\begin{tabularx}{14cm}{cX}
%$\tree[levelsep=6ex]{\lambda \xi_1\ldots \xi_n}{\TR{x_i}}$ &
%       $\intersem{M} = \Lambda^n(\pi_i)$ \\ \hline
%$ \tree[levelsep=6ex]{\lambda \xi_1\ldots \xi_n}
%        { \tree[levelsep=6ex]{x_i}
%            {   \TR{\tau(N_1)} \TR{\ldots} \TR{\tau(N_p)}}}
%    $
%&    where $\Gamma \vdash x_i : (A_1,\ldots,A_p,B)$ and $\Gamma \vdash N_j : A_j$ for $j\in 1..p$
%    $$\intersem{M} = \Lambda^n(\langle \pi_i, \intersem{\Gamma \vdash N_1 : A_1}, \ldots, \intersem{\Gamma \vdash N_p : A_p}  \rangle
%    \fatsemi ^{1..p} ev^p)$$
%\\ \hline
%$ \tree[levelsep=6ex]{\lambda \xi_1\ldots \xi_n}
%        { \tree[levelsep=6ex]{@}
%            {   \TR{\tau(N_0)} \TR{\ldots} \TR{\tau(N_p)}}}
%    $ &
%    where $\Gamma \vdash N_0 : (A_1,\ldots,A_p,B)$ and $\Gamma \vdash N_j : A_j$ for $j\in 1..p$
%    $$\intersem{M} = \Lambda^n(\langle \intersem{\Gamma \vdash N_0 : A_0}, \ldots, \intersem{\Gamma \vdash N_p : A_p}  \rangle
%    \fatsemi^{0..p} ev^p)$$
%\end{tabularx}
%\vspace{10pt}
%
%We write $\intersem{\Gamma \rightarrow A}_M$ to denote the
%interaction arena of the revealed strategy $\intersem{\Gamma \vdash
%M : A}$.
%\end{dfn}
%


\begin{exmp}
Consider the term $\lambda x . (\lambda f . f x) (\lambda y . y)$.
Its computation tree is:
$$
\tree{\lambda x} {
    \pstree[levelsep=4ex]{\TR{@}}
    {       \pstree[levelsep=4ex]{\TR{\lambda f}}
                { \tree{f}{  \tree{\lambda}{ \TR{x}  } } }
            \pstree[levelsep=4ex]{\TR{\lambda y}}
                    {\TR{y}}
    } }
$$
and its revealed strategy is $\langle \sem{ x:X \vdash \lambda f . f
x} , \sem{ x:X \vdash \lambda y . y} \rangle \fatsemi^{\{0,1\}}
ev_2$.
\end{exmp}


\subsubsection{From interaction semantics to standard semantics and vice-versa}

In the standard semantics, given two strategies $\sigma : A \rightarrow B$, $\tau : B \rightarrow C$ and
a sequence $s \in \sigma \fatsemi \tau$, it is possible to (uniquely) recover the internal moves. The uncovered sequence is written
${\bf u}(s, \sigma, \tau)$. The algorithm to obtain this unique uncovering is given in part II of \cite{hylandong_pcf}.

Given a term $M$, we can completely uncover the internal moves of a
sequence $s\in\sem{M}$ by performing the uncovering recursively at
every @-node of the computation tree. This operation is called
\emph{full-uncovering with respect to $M$}.

Conversely, the standard semantics can be recovered from the
interaction semantics by filtering the moves, keeping only those
played in the root arena:
\begin{eqnarray}
 \sem{\Gamma \vdash M : A} = \intersem{\Gamma \vdash M : A} \upharpoonright \sem{\Gamma \rightarrow T} \label{eqn:int_std_gamsem}
\end{eqnarray}


\subsubsection{Full abstraction}

Let $\mathcal{I'}$ denote lluf sub-category of $\mathcal{I}$
consisting only of strategies $\Sigma$ with a single annotated leaf
and no nodes. We have the following lemma:
\begin{lem}[$\mathcal{I'}$ is isomorphic to $\mathcal{C}$]
$\mathcal{I'} \cong \mathcal{C}$
\end{lem}
\begin{proof}
We define the functor $F:\mathcal{I'} \rightarrow \mathcal{C}$
by $F(A) = A$ for any object $A\in \mathcal{I'}$ and for $\Sigma \in \mathcal{I'}(A,B)$,
$F(\Sigma)$ is defined to be the annotation $\sigma$ of the only leaf in $\Sigma$.
The functor $G:\mathcal{C} \rightarrow \mathcal{I'}$ is defined by
$G(A) = A$ for any object $A\in \mathcal{C}$ and for $\sigma \in \mathcal{C}(A,B)$,
$G(\sigma)$ is the tree formed with the single annotated leaf $\sem{A}^\sigma$.
Then $F;G =id_{\mathcal{I'}}$ and $G;F =id_{\mathcal{C}}$.
\end{proof}

Consequently the lluf sub-category $\mathcal{I'}$ is fully abstract for the simply-typed lambda calculus.
Note that this is a major difference with $\mathcal{I}$ which is not fully-abstract since there may be several maps denoting a given
term.





\subsection{The correspondence theorem for the pure simply-typed $\lambda$-calculus}
In this section, we establish a
connection between the interaction semantics of a simply-typed term without constants ($\Sigma = \emptyset$)
and the traversals of its computation tree.

\subsubsection{Removing @-nodes from traversals}

When defining computation trees, it was necessary to introduce
application nodes (labelled @) in order to connect the operator and
the operand of an application. The presence of @-nodes has also
another advantage: it ensures that the lambda-nodes are all at even
level in the computation tree. Consequently a traversal respects
Alternation.

Application nodes are however redundant in the sense that they do
not play any role in the computation of the term. In other words,
the @-nodes occurring in traversals are superfluous. In fact it is
necessary to filter them out if we want to establish the
correspondence with the interaction game semantics.

\begin{dfn}[Filtering @-nodes in traversals]
\label{dfn:appnode_filter}
Let $t$ be a traversal of $\tau(M)$.
We write $t-@$ for the sequence of nodes with pointers obtained by
\begin{itemize}
\item removing from $t$ all @-nodes and value-leaves of a @-node;
\item replacing any link pointing to an @-node by a link pointing to the immediate predecessor of @ in $t$.
\end{itemize}

Suppose $u = t-@$ is a sequence of nodes obtained by applying the
previously defined transformation on the traversal $t$, then $t$ can
be partially recovered from $u$ by reinserting the @-nodes as
follows. For each @-node @ in the computation tree with parent node
denoted by $p$, we perform the following operations:
\begin{enumerate}
\item replace every occurrence of the pattern $p \cdot n$, where $n$ is a $\lambda$-nodes,
by $p \cdot @ \cdot n$;
\item replace any link in $u$ starting from a $\lambda$-node and pointing to $p$ by a link pointing to the inserted @-node;
\item if there is an occurrence in $u$ of a value-leaf $v_p$ pointing to $p$ then insert a value-leaf $v_@$
immediately before $v_p$ and make it points to the node immediately
following $p$ (which is also the $@$-node that we inserted in 1).
\end{enumerate}
We write $u+@$ for this second transformation.
\end{dfn}
These transformations are well-defined because in a traversal, an @-node
always occurs in-between two nodes $n_1$ and $n_2$ such that  $n_1$ is the parent node of @
and $n_2$ is the first child node of @ in the computation tree:
$$      \pstree[levelsep=4ex]{\TR{n_1}\treelabel{0} }
        {
            \pstree[levelsep=3ex]{\TR{@}}
            {
                \tree{n_2}{\vdots}
                \TR[edge=\dedge]{}
                \TR[edge=\dedge]{}
            }
        }
$$
Remark: $t-@$ is not a proper justified sequence
since after removing a @-node, any $\lambda$-node justified by @ will become
justified by the parent of @ which is also a $\lambda$-node.

The following lemma follows directly from the definition:
\begin{lem}
\label{lem:minus_at_plus_at}
For any traversal $t$ we have $(t-@)+@ \sqsubseteq t$ and if $t$ does not end with an @-node then
$(t-@)+@ = t$.
\end{lem}

Let $M$ be a term and $r$ be the root of $\tau(M)$. We introduce the following notations:
\begin{eqnarray*}
\travset(M)^{-@} &=& \{ t - @ \ | \  t \in \travset(M) \} \\
\travset(M)^{\upharpoonright r} &=& \{ t  \upharpoonright r \ | \  t  \in \travset(M) \} .
\end{eqnarray*}

\begin{lem}
Let $M$ be a pure simply-typed term and $r$ be the root of $\tau(M)$.
If $M$ is in $\beta$-normal form then $t = t \upharpoonright r = t - @$ for any $t \in \travset(M)$.
Consequently,
$$\travset(M)^{-@} \cong \travset(M) \cong  \travset(M)^{\upharpoonright r }.$$
\end{lem}
\begin{proof}
This is because the computation tree of a term in $\beta$-normal
does not contain any @-node and therefore all the nodes are
hereditarily justified by the root.
\end{proof}



\begin{lem}[Filtering lemma] Let $\Gamma \vdash M :T$ be a term and $r$ be the root of $\tau(M)$.
\label{lem:varphi_filter}
For any traversal $t$ of the computation tree we have
$\varphi(t-@) \upharpoonright \sem{\Gamma \rightarrow T} = \varphi(t\upharpoonright r)$.
Consequently:
$$ \varphi(\travset^{-@}(M)) \upharpoonright \sem{\Gamma \rightarrow T} = \varphi(\travset^{\upharpoonright r}(M)).$$
\end{lem}
\begin{proof}
    From the definition of $\varphi$, the nodes of the computation tree that are mapped by $\varphi$
    to moves of the arena $\sem{\Gamma \rightarrow T}$ are exactly the nodes that are hereditarily justified by $r$.
    The result follows from the fact that @-nodes are not hereditarily justified by the root.
\end{proof}

The function $\varphi$ regarded as a function from the set of vertices $V_\lambda \union V_{var}$ of the computation tree to moves in arenas is not injective.
For instance the two occurrences of $x$ in the computation tree of the term $\lambda f x. f x x$ are mapped to the same question. However
the function $\varphi$ regarded as a function from sequences of nodes to sequences of moves is injective:
\begin{lem}[$\varphi$ is injective]
\label{lem:varphiinjective}
$\varphi$ regarded as a function defined on the set of
sequences of nodes is injective in the sense that for any two traversals $t_1$ and $t_2$:
\begin{itemize}
\item[(i)] if $\varphi (t_1 - @ ) = \varphi (t_2 - @ )$ then $t_1-@ =t_2 -@$;
\item[(ii)] if $\varphi (t_1 \upharpoonright r ) = \varphi (t_2 \upharpoonright r )$ then $t_1\upharpoonright r = t_2\upharpoonright r$.
\end{itemize}
\end{lem}
\begin{proof}
(i) The set of traversals of a computation tree verifies the following property:
\begin{equation}
t \cdot n_1, t \cdot n_2 \in \travset \mbox{ where } n_1 \neq n_2 \mbox{ and $n_1, n_2$ are not @-node implies } \varphi(n_1) \neq \varphi(n_2). \label{lem:varphiinjective:eq1}
\end{equation}
Indeed, the only possible case where $\varphi$ maps two different
nodes to the same move is when $n_1$ and $n_2$ are two nodes
labelled with the same variable $x$. Hence the two traversals $t
\cdot n_1$ and $t \cdot n_2$ must have been formed using either rule
(Lam) or (App). But these two rules are deterministic and their
domain of definition is disjoint. This contradict the fact that $n_1
\neq n_2$.

Now suppose that $t_1-@\neq t_2-@$ then necessarily $t_1 \neq t_2$. Therefore
 $t_1 = t' \cdot n_1 \cdot u_1$ and $t_2 = t' \cdot n_2 \cdot u_2$ for some sequences $t'$, $u_1$, $u_2$
and some nodes $n_1\neq n_2$. By property \ref{lem:varphiinjective:eq1} we have $\varphi(n_1) \neq \varphi(n_2)$.
If we regard sequences of nodes and moves as \emph{pointer-less} sequences then we are allowed to write the following:
$$ (t' \cdot n_1 \cdot u_1) - @ = (t' - @) \cdot n_1 \cdot (u_1 -@),$$
and since $\varphi_M$ is a monoid homomorphism (provided that we ignore the justification pointers) we have:
$$ \varphi(t_1-@) = \varphi(t'-@) \cdot \varphi(n_1) \cdot \varphi(u_1) \neq \varphi(t'-@) \cdot \varphi(n_2) \cdot \varphi(u_2) = \varphi(t_2-@).$$

(ii) Again, suppose that $t \upharpoonright r \neq t' \upharpoonright r$ then
 $t_1 = t'_1 \cdot n_1 \cdot u_1$ and $t_2 = t_2' \cdot n_2 \cdot u_2$ for some sequences $t_1'$, $t_2'$, $u_1$, $u_2$
 such that $t'_1 \upharpoonright r = t'_2 \upharpoonright r $
and some nodes $n_1 \neq n_2$ both hereditarily justified by the root.
For the same reason as in (i), we must have $\varphi(n_1) \neq \varphi(n_2)$. Hence:
$$ \varphi(t_1\upharpoonright r) =
        \varphi(t'_1\upharpoonright r) \cdot \varphi(n_1) \cdot \varphi(u_1 \upharpoonright r)
    \neq \varphi(t'_1\upharpoonright r) \cdot \varphi(n_2) \cdot \varphi(u_2 \upharpoonright r)
         = \varphi(t_2\upharpoonright r).$$
\end{proof}

\begin{cor} \
\label{cor:varphi_bij}
\begin{itemize}
\item[(i)] $\varphi$ defines a bijection from $\travset(M)^{-@}$
to $\varphi(\travset(M)^{-@})$;
\item[(ii)] $\varphi$ defines a bijection from $\travset(M)^{\upharpoonright r}$ to
$\varphi(\travset(M)^{\upharpoonright r})$.
\end{itemize}
\end{cor}

\subsubsection{The correspondence theorem}
We are now going to state and prove the correspondence theorem
for the pure simply-typed $\lambda$-calculus without constants ($\Sigma = \emptyset$).
The result extends immediately to the simply-typed $\lambda$-calculus with \emph{uninterpreted} constants by
considering constants as being free variables.
We use the cartesian closed category of games $\mathcal{C}$ (defined in section \ref{subsec:pcfgamemodel} of the first chapter) as
a model of the simply-typed $\lambda$-calculus. We write $\sem{\Gamma \vdash M : A}$ for the strategy denoting the simply-typed term
$\Gamma \vdash M : A$.

\begin{prop}
\label{prop:rel_gamesem_trav} Let $\Gamma \vdash M : T$ be a term of
the pure simply-typed $\lambda$-calculus and $r$ be the root of
$\tau(M)$. We have:
\begin{itemize}
\item[(i)]  $\varphi_M(\travset(M)^{-@}) = \intersem{M}$
\item[(ii)] $\varphi_M(\travset(M)^{\upharpoonright r}) = \sem{M}$.
\end{itemize}
\end{prop}


\begin{rem} The proof that follows is quite tedious but the idea is simple. Let us give the intuition.
    We start by reducing the problem to the case of closed terms only. Then the proof proceeds by induction on the structure of the computation tree.
    It is straightforward to prove the result for term that are abstraction of a single variable.
    Now consider an application $M$ with the following computation tree $\tau(M)$:
    $$ \tree[levelsep=4ex]{\lambda \overline{\xi}}
        { \tree[levelsep=4ex]{@}
            {   \TR{\tau(N_0)} \TR{\ldots} \TR{\tau(N_p)}}}
    $$

    A traversal of $\tau(M)$ proceeds as follows: it starts at the root $\lambda \overline{\xi}$ of the tree $\tau(M)$ (rule
    (Root)), it then passes the node @ (rule (Lam)).
    After this initialization part, it proceeds by traversing the term $N_0$ (rule (App)).
    At some point, while traversing $N_0$, some variable $y_i$ bound by the root of $N_0$ is visited. The traversal
    of $N_0$ is interrupted and there is a jump (rule (Var)) to the root of $\tau(N_i)$. The process goes on by traversing $\tau(N_i)$.
    When traversing $N_i$, if the traversal encounters a variable bound by the root of $\tau(N_i)$ then the traversal of $N_i$ is interrupted and
    the traversal of $N_0$ resumes.  This schema is repeated until the traversal of $\tau(N_0)$ is completed\footnote{Since we are considering
    simply-typed terms, the traversal does indeed terminate. However this will not be true anymore in the \pcf\ case.}.

    The traversal of $M$ is therefore made of an initialization part followed by an interleaving of a traversal of $N_0$ and
    several traversals of $N_i$ for $i=1..p$. This schema is reminiscent of the way the evaluation copycat map $ev$ works in game semantics.

    The key idea is that every time the traversal pauses the traversal of a subterm and switches to another one,
    the jump is permitted by one of the four copycat rules (Var), (CCAnswer-@), (CCAnswer-$\lambda$) or (CCAnswer-var).
    We show by (a second) induction that these copycat rules defines exactly what the copycat strategy $ev$ performs on sets of moves.

%    In the game semantics, the evaluation map (a copy-cat strategy) copies this opening move to an initial move $m_0$ in the game
%    $B_0$ and the game continues in $B_0$. We reflect this in the traversal : we make $t$ follow
%    the ``script'' given by the traversal $t^0_{m_0}$.
%    The rule (App) allow us to initiate this simulation  by visiting the  first move in $t^0_{m_0}$: the root of $\tau(N_0)$.
%
%    This simulation continues until it reaches a node $\alpha_0$ which is hereditarily justified by the root
%    $\tau(N_0)$: $\alpha_0$ is present in the reduction of traversal of $t^0_{m_0}$ therefore $\varphi_{N_0}(\alpha_0)$ is an un-hidden move played in $A_0$.
%
%    In the game semantics this corresponds to a move played in a component $A_k$ for some $k\in 1..p$ of
%    of the game $B_0$ in which case the evaluation map copies the move to an initial move $m_1$ in the corresponding component $B_k$.
%
%    To reflect this the traversal now opens up a new thread and simulates the traversal $t^k_{m_1}$.  Again, this simulation stops when we reach a node
%    $\alpha_1$ in $t^k_{m_1}$ which is hereditarily justified by the root of $\tau(N_k)$: $\alpha_1$ must be present in the reduction of traversal
%    of $t^k_{m_1}$ therefore $\varphi_{N_k}(\alpha_1)$ is an un-hidden move played in $A_k$.
%    In the game semantics, this move $\alpha$ is copied back to the component $B_k$ of the game $B_0$.
%
%    The traversal now resumes the simulation of $t^0_{m_0}$. And the process goes continuously.
\end{rem}

Let us fix some notation: we write $s\upharpoonright A,B$ for the
sequence obtained from $s$ by keeping only the moves that are in $A$ or $B$ and by removing any link pointing to a move that
has been removed.
If $m$ is an initial move, we write $s \upharpoonright m$ to
denote the thread of $s$ initiated by $m$, i.e. the sequence obtained from $s$ by keeping all the moves
hereditarily justified by $m$.
We also write $s \upharpoonright A,B,m$ where $m$ is an initial move
for the sequence obtained from $s \upharpoonright A,B$ by keeping
all moves hereditarily justified by $m$.



\begin{proof}
(i) Suppose $\Gamma = \xi_1:X_1,\ldots \xi_n:X_n$. Then we have:
\begin{eqnarray*}
\intersem{\Gamma \vdash M:T} &=& \Lambda^n( \intersem{\emptyset \vdash \lambda \xi_1\ldots \xi_n . M: (X_1,\ldots,X_n,T) } ) \\
        &\simeq& \intersem{\emptyset \vdash \lambda \xi_1\ldots \xi_n . M: (X_1,\ldots,X_n,T) }.
\end{eqnarray*}
Similarly the computation tree $\tau(M)$ is isomorphic to
$\tau(\lambda \xi_1\ldots \xi_n . M)$ (up to a renaming of the root
of the computation tree) therefore $\travset(M)$ is also isomorphic
to $\travset(\lambda \xi_1\ldots \xi_n . M)$. Hence we can make the
assumption that $M$ is a closed term. If we prove that the property
is true for all closed terms of a given height then it will be
automatically true for any open term of the same height.


Let us assume that $M$ is already in $\eta$-long normal form. We
proceed by induction on the height of the tree $\tau(M)$ and by
case analysis on the structure of the computation tree:
\begin{itemize}
  \item (abstraction of a variable): $M \equiv \lambda \overline{\xi} .
  x$.  Since $M$ is in $\eta$-long normal form, $x$ must be of ground type and since $M$ is
      closed we have $x = \xi_i \in \overline{\xi}$ for some $i$.
      Hence $\tau(M)$ has the following shape:
        $$ \tree[levelsep=6ex]{ \lambda \overline{\xi}^{[0]} }{\TR{\xi_i^{[1]}}}$$
        The arena is of the following form (only question moves are represented):
        $$ \tree{ q_0 }
        {   \tree[linestyle=dotted]{q^1}{\TR{} \TR{} }
            \tree[linestyle=dotted]{q^2}{\TR{} \TR{} }
            \TR{\ldots}
            \tree[linestyle=dotted]{q^n}{\TR{} \TR{} }
        }$$

        Let $\pi_i$ denote the $i$th projection of the interaction game
        semantics. We have:
        \begin{align*}
        \intersem{M} &= \intersem{\emptyset \vdash \lambda \overline{\xi} . \xi_i} \\
                     &= \Lambda^n(\intersem{\overline{\xi} \vdash  \xi_i}) \\
                     &= \Lambda^n(\pi_i) \\
                     &\cong \pi_i \\
                     &= \prefset(\{ q_0 \cdot q^i \cdot v_{q^i} \cdot v_{q_0} \ | \ v\in \mathcal{D}
                     \}).
        \end{align*}

        Since $M$ is in $\beta$-normal we have $\travset(M)^{-@} = \travset(M)$.
        It is easy to see that the set of traversals of $M$ is the set of prefix of
        the traversal $\lambda \overline{\xi} \cdot \xi_i \cdot v_{\xi_i} \cdot v_{\lambda \overline{\xi}}$:
        $$ \travset^{-@}(M) = \travset(M) = \prefset( \lambda \overline{\xi} \cdot \xi_i \cdot v_{\xi_i} \cdot v_{\lambda \overline{\xi}})
        $$

        The pointers of the traversal $\lambda \overline{\xi} \cdot \xi_i \cdot v_{\xi_i} \cdot
        v_{\lambda \overline{\xi}}$ are the same as the play $q_0 \cdot q^i \cdot v_{q^i} \cdot
        v_{q_0}$, therefore since $\varphi_M(\lambda \overline{\xi}) = q_0$ and $\varphi_M(\xi_i) =
        q^i$ we have:
        $$ \varphi_M(\travset^{-@}(M)) = \intersem{M}.$$


    \item (abstraction of an application): we have $M = \lambda \overline{\xi} . N_0 N_1 \ldots N_p$. Let $\Gamma$ be the context
    $\Gamma = \overline{\xi} : \overline{X}$. Then we have the following sequents:
    $\emptyset \vdash M : (X_1,\ldots,X_n,o)$,
    $\Gamma \vdash N_0 N_1 \ldots N_p : o$,
    $\Gamma \vdash N_i : B_i$ for $i\in 0..p$ with $B_0 = (B_1,\ldots,B_p,o)$ and $p\geq 1$.

    There are two subcases, either $N_0 \equiv \xi_i$ where $\alpha$ is a variable in $\overline{\xi}$ and the tree has the following form:
    $$ \tree[levelsep=6ex]{\lambda \overline{\xi}^{[0]}}
        { \tree[levelsep=6ex]{\xi_i^{[1]}}
            {   \TR{\tau(N_1)} \TR{\ldots} \TR{\tau(N_p)}}}
    $$
    or $N_0$ is not a variable and the tree $\tau(M)$ has the following form:
    $$ \tree[levelsep=6ex]{\lambda \overline{\xi}^{[0]}}
        { \tree[levelsep=6ex]{@^{[1]}}
            {
            \tree[levelsep=6ex]{\lambda y_1 \ldots y_p}{\ldots}
            \TR{\tau(N_1)} \TR{\ldots} \TR{\tau(N_p)}}}
    $$

    We only consider the second case since the first one can be treated
    similarly. Moreover we make the assumption that $p=1$. It is
    straightforward to generalize to any $p\geq1$.
    We write $\lambda \overline{z}$ to denote the root of the tree $\tau(N_1)$.


    We have:
    \begin{align*}
    \intersem{M}
        &=  \Lambda^n( \intersem{\Gamma \vdash N_0 N_1 : o} )
            & \mbox{(game semantics for abstraction)}\\
        &\cong  \intersem{\Gamma \vdash N_0 N_1 : o}
            & \mbox{(up to moves retagging)}\\
        &=  \langle \intersem{\Gamma \vdash N_0}, \intersem{\Gamma \vdash N_1} \rangle \fatsemi^{0..1} ev
            & \mbox{(game semantics for application)}\\
        &=  \langle \varphi_{N_0} (\travset^{-@}(N_0)), \varphi_{N_1}(\travset^{-@}(N_1) \rangle \fatsemi^{0..1} ev
            & \mbox{(induction hypothesis)}\\
        &=  \langle \varphi_{M} (\travset^{-@}(N_0)), \varphi_{M}(\travset^{-@}(N_1)) \rangle \fatsemi^{0..1} ev
            & \mbox{($\varphi_M = f(0,q_0) \union \varphi_{N_0} \union \varphi_{N_1}$)} \\
        &=  \underbrace{\langle \varphi_{M} (\travset^{-@}(N_0)), \varphi_{M}(\travset^{-@}(N_1)) \rangle}_{\sigma} \parallel ev
            & \mbox{($\fatsemi^{0..1}$ and $\parallel$ are the same operator)}
    \end{align*}


    The strategies $\sigma$ and $ev$ are defined on the arena $!A \multimap B$ and $!B \multimap C$ respectively where:
    \begin{eqnarray*}
        A &=& \intersem{\Gamma} = \intersem{X_1} \times \ldots \times \intersem{X_n}\\
        B &=& \intersem{B_0} \times \intersem{B_1} = \intersem{B_1' \rightarrow o'} \times \intersem{B_1} \\
        C &=& \intersem{o}
    \end{eqnarray*}

    We have $u \in \intersem{M} \cong \sigma^{\dag} \parallel ev$ if and only if
    \begin{eqnarray*}
      &&      \left\{
            \begin{array}{ll}
                u \in int(!A,!B,C)\\
                u \upharpoonright !A,!B  \in \sigma^\dagger \\
                u \upharpoonright !B,C  \in  ev
            \end{array}
            \right. \\
    & \mbox{or equivalently} & \left\{
    \begin{array}{ll}
        u \in int(!A,!B,C) \\
        \hbox{for any initial $m$ in $u \upharpoonright !A,!B$ there is $j \in 0..p$ such that } \\
        \left\{\begin{array}{ll}
            u \upharpoonright !A,B_j, m \in \varphi_{M} (\travset^{-@}(N_j)) \label{eq:def_z} \\
            u \upharpoonright !A, B_k,m = \epsilon \quad \mbox{ for every } k\neq j \label{eq:b}
        \end{array}
        \right.
    \end{array}
    \right.
    \end{eqnarray*}


    We first prove that $\intersem{M} \subseteq \varphi_{M}( \travset^{-@}(M)
    )$.


    Suppose $u \in \intersem{M}$. We give a constructive proof that
    there exists a sequence of nodes $t$ in $N$ such that $\varphi_M(t-@) = u$ by induction on the length of $u$.
    Let $q_o$ be the initial question of the arena $\sem{M}$ and $q_1$ the initial question of $\sem{N_0}$.

    Base cases:
    \begin{itemize}
    \item $u=\epsilon$ then $\varphi(\epsilon) = u$ where the traversal $\epsilon$ is formed with the rule ($\epsilon$).
    \item If $|u|=1$ then $u=q_0$ is the initial move in $C$ and $\varphi(\lambda \overline{\xi}) = u$. The traversal
    $\lambda \overline{\xi}$ is formed with the rule (Root).
    \end{itemize}

    Step cases: Suppose that $u' = \varphi_M(t'-@)$ and $u = u' \cdot m \in \intersem{M}$ with $|u|>1$ for some traversal $t'$ of $\tau(M)$.
    Let us write $m^1$ for the last move in $u'$.

    \begin{enumerate}
    \item Suppose $m \in C$. In $C$ there are no internal moves, the only moves of $C$ are therefore $q_0$ and
    $v_{q_0}$ for some $v\in\mathcal{D}$. But $q_0$ can occur only once in $u$, therefore since $|u|>1$ we must have $m = v_{q_0}$
    for some $v\in \mathcal{D}$.  Since $m$ is an answer move to the initial question, it must be
    the duplication  (performed by the copy-cat evaluation strategy) of the move $m^1$ played in $o'$.
    Hence $m^1=v_{q_1}$. By the induction hypothesis, $n'$ -- the last move in $t'$ -- is equal to
    $\varphi(m^1) = v_{\lambda y_1}$.

    By property \ref{proper:phi_pview}(iv), $?(u') = \varphi(?(t'-@))$ and
    since $q_0$ is the pending question in $u'$, the first node of $t'$ is also the pending node in $t'$.
    This permits us to use the rule (CCAnswer-$\lambda$) to produce the traversal $t = t' \cdot v_{\lambda \overline{\xi}}$
    where $v_{\lambda \overline{\xi}}$ points to the first node in $t'$. Clearly, $\varphi(t-@) = u$.



    \item Suppose that $m,m^1 \in A \union B_0$.
    The strategy $ev$ is responsible for switching thread in $B_0$ therefore, in the interaction semantics,
    there must be a copycat move in-between two moves belonging to two different threads.
    Since $m$ and $m^1$ are consecutive moves in the sequence $u$, they must belong to the same thread i.e. there are
    hereditarily justified  by the same initial $m_0$ in $B_0$.


    We then have $(u \upharpoonright !A, !B)\upharpoonright m_0 = \varphi_{N_0}(t_0-@)$ for some traversal $t_0$ of $N_0$.
    Consequently  $\varphi_{N_0}(n^1) = m^1$ and $\varphi_{N_0}(n) = m$
    where $n^1 \cdot n$ are the last two moves in $t_0-@$.

    $n$ points to some node in $t_0$ that also occurs in $t'$. Let us call $n^2$ this node.
    Since $(u \upharpoonright !A, !B)\upharpoonright m_0 = \varphi_{N_0}(t_0-@)$,
    $n_2$ must have the same position in $t'$ as the node justifying $m$ in $u'$.
    Hence we just need to take $t = t' \cdot n$ where $n$ points to $n^2$ in $t'$.

    The sequence $t$ is indeed a valid traversal of $\tau(M)$
    because the rule used by the traversal $t_0$
    of $\tau(N_0)$ to visit the node $n$ after $n^1$ can also be used by the traversal $t'$ of $\tau(M)$
    to visit $n$ after $n^1$.
    This can be checked formally by inspecting all the traversal rules. The key reason is that
    all the nodes in $t_0-@$ are present in $t'$ with the same pointers but with some nodes interleaved in between.
    However these interleaved nodes are inserted in a way that still permits to use the traversal rule.

    \item Suppose that $m,m^1 \in A \union B_1$.
    The proof is similar to the previous case.

    \item Suppose that $m \in A \union B_0$ and $m^1 \in A \union B_1$.

    $t$ is obtained from $t-@$ using the transformation $+@$. We apply the same transformation to $u$ in order
    to make $O$-questions and $P$-questions in $u$ match with $\lambda$-nodes and variable nodes in $t'$ respectively.
    We write this sequence $u+@$.
    The $+@$ operation inserts nodes in the sequence but not at the end,
    therefore $m^1$, the last move in $u'$, is also the last move in $u'+@$.
    Let us note $n^1$ for the last move in $t'$.

        \begin{enumerate}
        \item If $n^1$ is the application node @ then it must be the parent of the node $\lambda y_1$ since it
        is the only non-internal @-node present in $t'$.
        Therefore $t'=\lambda \overline{\xi} \cdot @$ and $u= q_0 \cdot m$.
        But $m$ is the copy of $q_0$ replicated by $ev$ in $o'$ therefore $m=q_1$.
        Applying the (App) rule on $t'$ produces the traversal $\lambda \overline{\xi} \cdot @ \cdot \lambda y_1$
        with $\varphi((\lambda \overline{\xi} \cdot @ \cdot \lambda y_1)-@ ) = q_0 \cdot q_1 = u$.

        \item If $n^1$ is a variable node then $m^1$ is a P-move and $m$ is an O-move
            and therefore $m$ is the copy of $m^1$ duplicated in $B_1$ by the evaluation strategy.
            Consequently, $m^1$ points to some $m^2$ and $m$ points to the node preceding $m^2$ denoted by $m^3$.
            The diagram below shows an example of such sequence:
                $$
                \begin{array}{cccccccc}
                & (B_1' &\rightarrow & o') & \times & B_1 & \rightarrow & o' \\
                O & &&&&&& \rnode{q0}{q_0 (\lambda \overline{\xi})} \\
                P & &&&&& \\
                O & && \rnode{q1}{q_1 (\lambda \overline{y})} \\
                P & \rnode{m3}{m^3 (y_1)} \\
                O & &&&& \rnode{m2}{m^2 (\lambda \overline{z})} \\
                P & &&&& \rnode{m1}{m^1 (z_i)} \\
                O & \rnode{m}{m} \\
                \end{array}
                \ncline[nodesep=3pt]{->}{q1}{q0} \mput*{@}
                \nccurve[nodesep=3pt,ncurv=2,angleA=180,angleB=180]{->}{m1}{m2}
                \ncarc[nodesep=3pt,ncurv=1,angleA=90,angleB=180]{->}{m3}{q1}
                \ncarc[nodesep=3pt,ncurv=1,angleA=90,angleB=180]{->}{m}{m3}
                \ncline[nodesep=3pt]{->}{m2}{q0}
                $$

        $t'$  and $u+@$ have the following forms:
        \begin{eqnarray*}
                t'&=& \ldots \cdot n^3 \cdot \rnode{n2}{n^2} \cdot \ldots \cdot \rnode{n1}{n^1} \\ \\
                u+@ &=& \ldots \cdot \rnode{m3}{m^3} \cdot \rnode{m2}{m^2} \cdot \ldots \cdot \rnode{m1}{m^1} \cdot \rnode{m}{m}
            \bkptr{30}{m1}{m2} \bkptr{30}{m}{m3}
            \bkptr{30}{n1}{n2}
        \end{eqnarray*}

        Since $n^1$ is a variable node, $n^2$ must be a $\lambda$-node.
        $n^3$ could be either a variable node or an @-node. In fact $n^3$ is necessarily a variable node. Indeed,
        $n^3$ is mapped to $m^3$ by $\varphi_{N_0}$ and $m^3$ belongs to $\sem{B_i'}$ (i.e. it is not
        an internal move of $\intersem{B_i'}$). The function $\varphi_{N_0}$ is defined in such a way that
        only nodes which are hereditarily justified by the root of $\tau(N_0)$ are mapped to nodes in $\sem{B_1'}$.
        Hence $n^3$ is hereditarily justified by the root and consequently it cannot be an @-node.

        Hence $n^1$ is a variable node, $n^2$ is a $\lambda$-node and $n^3$ is a variable node. We
        can therefore apply the (Var) rule to $t'$ and we obtain a traversal of the following form:

        \begin{eqnarray*}
            t&=& \ldots \cdot \rnode{n3}{n^3} \cdot \rnode{n2}{n^2} \cdot \ldots \cdot \rnode{n1}{n^1} \cdot \rnode{n}{n}
            \bkptr{30}{n1}{n2} \bkptr{30}{n}{n3}
        \end{eqnarray*}

        We have $\varphi(t'-@) = u'$ by the induction hypothesis and $\varphi(n) = m$ by definition of $\varphi$.
        Therefore since $m$ and $n$ point to the same position we have $\varphi(t-@) = u$.

        \item If $n^1$ is the value-leaf of a variable node then we proceed the same way as in the previous case:
        $n^1$ is a value-leaf of the variable node $n^2$ and we can use the
        (CCAnswer-$\lambda$) rule to extend the traversal $t'$.

        \item Suppose that $n^1$ is a lambda node, in which case $m^1$ is an O-move, then
        necessarily, $m^1$ is a move copied by the evaluation strategy
         from $B_1'$ to $B_1$. The move following $m^1$ should also be played in $B_1$ before being copied
         back to $B_1'$ by the evaluation strategy. But since $m \in B_0$, this case does not happen.


        \item If $n^1$ is a value-leaf of a lambda node then $n^2$ is a lambda node and $n^3$ is a variable node.
        We can therefore use the rule (CCAnswer-var) or (CCAnswer-@) to extend the traversal $t'$.
        \end{enumerate}

    \item Suppose $m \in A \union B_1$ and $m^1 \in A \union B_0$ then
    the proof is similar to the previous case.
    \end{enumerate}


  For the converse, $\varphi_{M}( \travset^{-@}(M) ) \subseteq \intersem{M}$, it is an easy induction
  on the traversal rules. We omit the details here.
\end{itemize}

(ii) is an immediate consequence of (i):
\begin{align*}
\sem{M} &= \intersem{M} \upharpoonright \sem{\Gamma \rightarrow T} & \mbox{(eq. \ref{eqn:int_std_gamsem})} \\
        &= \varphi_M(\travset^{-@}(M)) \upharpoonright \sem{\Gamma \rightarrow T} & \mbox{(by (i))}\\
        &= \varphi_M(\travset^{\upharpoonright r}(M)) & \mbox{(lemma \ref{lem:varphi_filter})}
\end{align*}
\end{proof}


Putting corollary \ref{cor:varphi_bij} and proposition
\ref{prop:rel_gamesem_trav} together we obtain the following theorem
which establish a correspondence between the game-denotation of a
term and the set of traversals of its computation tree:

\begin{thm}[The Correspondence Theorem]
\label{thm:correspondence}
 For any pure simply-typed term $\Gamma \vdash M$,
$\varphi_M$ defines a bijection from $\travset(M)^{\upharpoonright
r}$ to $\sem{M}$ and a bijection from $\travset(M)^{-@}$ to
$\intersem{M}$:
\begin{eqnarray*}
 \varphi_M  &:& \travset(\Gamma \vdash M)^{\upharpoonright r} \stackrel{\cong}{\longrightarrow} \sem{\Gamma \vdash M} \\
 \varphi_M  &:& \travset(\Gamma \vdash M)^{-@} \stackrel{\cong}{\longrightarrow} \intersem{\Gamma \vdash M}
\end{eqnarray*}

Moreover if $M$ is in $\beta$-normal form and $s$ is a
\emph{maximal} play then  $t$ is a \emph{maximal} traversal.
\end{thm}

\begin{proof}
The first part is an immediate consequence of corollary
\ref{cor:varphi_bij} and proposition
\ref{prop:rel_gamesem_trav}.

Finally, if $M$ is in $\beta$-normal form then
$\travset(M)^{\upharpoonright r} = \travset(M)$
therefore $\varphi$ is a bijection from $\travset(M)$ to
$\sem{M}$. Suppose $s$ is a maximal play and suppose $t' \sqsubseteq
t$ then since $\varphi$ is monotonous we have $s = \varphi(t) \sqsubseteq
\varphi(t')$. But $s$ is maximal therefore $s = \varphi(t') =
\varphi(t)$ and because $\varphi$ is injective we have $t'=t$.
\end{proof}

The following diagram recapitulates the main results of this section:
$$
\xymatrix @C=6pc{
                                           & \travset(M)^{-@} \ar@/_/[dl]_{+@}  \ar[r]^{\varphi_M}_\cong & \intersem{M} \ar@/_/[dd]_{\_ \upharpoonright \sem{\Gamma\rightarrow T}} \\
\travset(M) \ar@/_/[ur]_{-@}^{} \ar[dr]^{\_ \upharpoonright r}  \\
                                           & \travset(M)^{\upharpoonright r} \ar[r]^{\varphi_M}_\cong & \sem{M} \ar@/_/[uu]^{\cong}_{\mbox{full uncovering}}
}
$$
