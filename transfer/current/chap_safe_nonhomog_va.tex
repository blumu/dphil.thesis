g\clearpage
\section{Non-homogeneous Safe $\lambda$-Calculus - VERSION A}
\todobox{ PROBLEM WITH THIS VERSION OF THE RULES: reduction does not preserve safety although substitution does! }

In section \ref{sec:safe_alt}, we have presented a safe lambda calculus in the setting of homogeneous types.
In this section, we try to give a general notion of safety for the simply typed $\lambda$-calculus.
The rules we give here do not assume homogeneity of the types.

We will call safe terms the simply typed lambda terms that are typable within the following system of formation rules:

\subsection{Rules}

 We use a set of sequents of the form $\Gamma \vdash^{i} M :
A$ where the meaning is ``variables in $\Gamma$ have orders at least
$\ord{A}+i$'' where $i \leq 0$. The following set of rules are
defined for $i \leq 0$:

$$ \mathbf{(seq^i_\delta)} \quad \rulef{\Gamma \vdash^{i} M : A}{\Gamma \vdash^{i-\delta} M : A} \quad i \leq 0, \delta > 0  $$

$$ \mathbf{(var)} \quad  \rulef{}{x : A\vdash^{0} x : A} $$

$$ \mathbf{(wk^i)} \quad  \rulef{\Gamma \vdash^i M : A}{\Gamma , x : B \vdash^i M : A} \quad \ord{B} \geq \ord{A} + i $$

$$ \mathbf{(app^i)} \quad  \rulef{\Gamma \vdash^i M : A\rightarrow B
                                        \qquad \Gamma \vdash^{0} N : A}
                                   {\Gamma  \vdash^{\min(i+\delta,0)} M N : B}
                                    \qquad
                                   \delta = \max\left(0, 1 + \ord{A} - \ord{B}\right)$$

$$ \mathbf{(abs^i)} \quad  \rulef{\Gamma, \overline{x} : \overline{A} \vdash^{i} M : B}
                                   {\Gamma  \vdash^{0} \lambda \overline{x} : \overline{A} . M : (\overline{A},B)} \qquad
%                                   \left\{
%                                     \begin{array}{ll}
                                       \forall y \in \Gamma : \ord{y} \geq \ord{\overline{A},B}\\
%                                       \forall y \in fv(M) : \ord{y} \geq \ord{B}
%                                     \end{array}
%                                   \right.
                                   $$


Note that:
\begin{itemize}
\item $(\overline{A},B)$ denotes the type $(A_1,A_2, \ldots, A_n, B)$;
\item all the types appearing in the rule are not required to be homogeneous. For instance in the rule $\mathbf{(app^i)}$ for the type $A \rightarrow B$
it is not necessary that $\ord{A} \geq \ord{B}$;
\item the environment $\Gamma, \overline{x}$ is not stratified. In particular, variables in $\overline{x}$ do not necessarily have the same
order;
\item In the abstraction rule, the side-condition imposes that at least all the variable of the lowest order
in the context are abstracted. However other variables can also be
abstracted together with the lowest order variables. Moreover there
is not constraint on the order on which the variables are abstracted
(contrary to what happens in the homogeneous case);
\item The sequents $\Gamma \vdash^0 M$ are the \emph{safe terms} that we want to generate.
Other terms are only used as intermediate sequents in a proof tree.
\end{itemize}

\begin{exmp}
Suppose $x:o$, $f:o\rightarrow o$ and $\varphi:(o\rightarrow
o)\rightarrow o$ then the term $$\vdash^0 \lambda x f \varphi .
\varphi : o \rightarrow (o\rightarrow o) \rightarrow ((o\rightarrow
o)\rightarrow o) \rightarrow (o\rightarrow o)\rightarrow o$$ is
valid although its type is not homogeneous
\end{exmp}


\begin{lem}[Basic properties]
\label{va_lem:nonhomosafe_basic_prop} Suppose $\Gamma \vdash^i M : B$
is a valid judgment then every variable in $\Gamma$ has order at
least $ord(M)+i$.
\end{lem}
\begin{proof}
An easy induction. The step case for the application is: suppose
$\Gamma \vdash^{i+\delta} M N : B$ where $\Gamma \vdash^i M :
A\rightarrow B$. Then by induction we have $\forall y \in \Gamma :
\ord{y} \geq \ord{A\rightarrow B} + i = \max(1+\ord{A}, \ord{B})+i =
\delta + \ord{B} + i \geq \min(i+\delta,0) + \ord{B}$.
\end{proof}

\subsection{Substitution in the safe lambda calculus}

The traditional notion of substitution, on which the $\lambda$-calculus is based on, is the following one:
\begin{dfn}[Substitution]
\label{va_dfn:subst}
\begin{eqnarray*}
c \subst{t}{x} &=& c \quad \mbox{where $c$ is a $\Sigma$-constant}\\
x \subst{t}{x} &=& t\\
 y\subst{t}{x} &=& y \quad \mbox{for } x \not \neq y,\\
(M_1 M_2) \subst{t}{x} &=& (M_1 \subst{t}{x}) (M_2 \subst{t}{x})\\
(\lambda x . M) \subst{t}{x} &=& \lambda x . M\\
(\lambda y . M) \subst{t}{x} &=& \lambda z . M \subst{z}{y}
\subst{t}{x} \mbox{where $z$ is a fresh variable and $x\not = y$}
\end{eqnarray*}
\end{dfn}

In the setting of the safe lambda calculus, the notion of substitution can be simplified.
Indeed, we remark that for safe $\lambda$-terms there is no need to rename variables
when performing substitution:

\begin{lem}[No variable capture lemma]
\label{va_lem:noclash}
There is no variable capture when performing substitution on a safe term.
\end{lem}
\begin{proof}
Suppose that a capture occurs during the substitution $M[N/\varphi]$
where $M$ and $N$ are safe. Then the following conditions must hold:
\begin{enumerate}
\item $\varphi:A, \Gamma \vdash^0 M$,
\item $\Gamma' \vdash^0 N$,
\item there is a subterm $\lambda \overline{x} . L$ in $M$ (where the abstraction is taken as wide as possible) such that:
\item $\varphi \in fv(\lambda \overline{x} . L)$ (and therefore $\varphi \in fv(L)$),
\item $x \in fv(N)$ for some $x \in \overline{x}$.
\end{enumerate}

By lemma \ref{va_lem:nonhomosafe_basic_prop} and (v) we have:

\begin{equation}
\ord{x} \geq \ord{N} = \ord{\varphi} \label{va_eq:xigeqphi}
\end{equation}

The abstraction $\lambda \overline{x} . L$ (taken as large as possible)
is a subterm of $M$, therefore there is a node $\Sigma \vdash^u \lambda \overline{x} . L$  for some $u$ in the
proof tree of $\varphi:A, \Gamma \vdash^0 M$.

There are only three kinds of rules that can produce an abstraction:
$\mathbf{(abs^i)}$, $\mathbf{(seq^i_\delta)}$ and $\mathbf{(wk^i)}$.
The only one that can introduce the abstraction is
$\mathbf{(abs^i)}$. Therefore the proof tree has the following form:
$$ \rulef{
    \rulef{
        \rulef{
            \rulef  {\ldots}
                   {\Sigma' \vdash^0 \lambda \overline{x} . L} \mathbf{(abs^i)}
        }
        {\ldots} r_1
    }
    {\vdots} r_2
    }
    { \Sigma \vdash^u \lambda \overline{x} . L } r_l
    \qquad \mbox{where for } j\in 1..l: r_j \in \{ \mathbf{(seq^i_\delta)},\ \mathbf{(wk^i)}\ |\ i \in \zset, \delta > 0 \}.
$$


Since $\varphi \in fv (L)$ we must have $\varphi \in \Sigma'$ and
since $\Sigma' \vdash^0 \lambda \overline{x} . L$, by lemma
\ref{va_lem:nonhomosafe_basic_prop} we have:
$$\ord{\varphi} \geq \ord{\lambda \overline{x} . L} \geq \max(1+ \ord{x}, \ord{L}) > \ord{x}$$

which contradicts equation (\ref{va_eq:xigeqphi}).
\end{proof}

Hence, in the safe lambda calculus setting, we can omit to
to rename variable when performing substitution. The equation
$$(\lambda x . M) \subst{t}{y} = \lambda z . M \subst{z}{x}
\subst{t}{y} \mbox{where $z$ is a fresh variable}$$
becomes
$$(\lambda x . M) \subst{t}{y} = \lambda x . M \subst{t}{y}$$



Unfortunately, this notion of substitution is still not adequate for
the purpose of the safe simply-typed lambda calculus. The problem is
that performing a single $\beta$-reduction on a safe term will not
necessarily produce another safe term.

To fix this problem, we need to be able to reduce several
consecutive $\beta$-redex at the same time until we obtain a safe
term. Consequently, we need a means of performing several
substitutions at the same time. To achieve this, we introduce the
\emph{simultaneous substitution},
 a generalization of the standard substitution given in definition \ref{va_dfn:subst}.

\begin{dfn}[Simultaneous substitution]
 We note $\subst{\overline{N}}{\overline{x}}$ for $\subst{N_1 \ldots N_n}{x_1
\ldots x_n}$:
\begin{eqnarray*}
c \subst{\overline{N}}{\overline{x}} &=& c \quad \mbox{where $c$ is a $\Sigma$-constant}\\
x_i \subst{\overline{N}}{\overline{x}} &=& N_i\\
 y \subst{\overline{N}}{\overline{x}} &=& y \quad \mbox{ if } y \not \neq x_i \mbox{ for all } i,\\
(M N) \subst{\overline{N}}{\overline{x}} &=& (M \subst{\overline{N}}{\overline{x}}) (N \subst{\overline{N}}{\overline{x}}) \\
(\lambda x_i . M) \subst{\overline{N}}{\overline{x}} &=& \lambda x_i . M
\subst{N_1 \ldots N_{i-1} N_{i+1}\ldots N_n}{x_1 \ldots x_{i-1} x_{i+1}\ldots x_n} \\
(\lambda y . M)
\subst{\overline{N}}{\overline{x}} &=& \lambda z . M \subst{z}{y} \subst{\overline{N}}{\overline{x}} \\
&& \mbox{where $z$ is a fresh variables and } y \neq x_i \mbox{ for all } i
\end{eqnarray*}
\end{dfn}

In general, variable captures should be avoided, this explains why the definition
of simultaneous substitution uses auxiliary fresh variables.
However in the current setting, lemma \ref{va_lem:noclash} can clearly be transposed to
the simultaneous substitution therefore there is no need to rename variable.

The notion of substitution that we need is therefore
the \emph{capture permitting simultaneous substitution} defined as follows:

\begin{dfn}[Capture permitting simultaneous substitution]
 We use the notation
$\subst{\overline{N}}{\overline{x}}$ for $\subst{N_1 \ldots N_n}{x_1
\ldots x_n}$:
\begin{eqnarray*}
c \subst{\overline{N}}{\overline{x}} &=& c \quad \mbox{where $c$ is a $\Sigma$-constant}\\
 x_i \subst{\overline{N}}{\overline{x}} &=& N_i\\
 y \subst{\overline{N}}{\overline{x}} &=& y \quad \mbox{where } x \not \neq y_i \mbox{ for all } i,\\
(M_1 M_2) \subst{\overline{N}}{\overline{x}} &=& (M_1 \subst{\overline{N}}{\overline{x}}) (M_2 \subst{\overline{N}}{\overline{x}})\\
(\lambda x_i . M) \subst{\overline{N}}{\overline{x}} &=& \lambda x_i . M
\subst{N_1 \ldots N_{i-1} N_{i+1}\ldots N_n}{x_1 \ldots x_{i-1} x_{i+1}\ldots x_n} \\
(\lambda y . M) \subst{\overline{N}}{\overline{x}} &=& \lambda y . M \subst{\overline{N}}{\overline{x}} \mbox{where $y \not = x_i$ for all $i$}
\qquad \mathbf{(\star)}
\end{eqnarray*}
The symbol $\mathbf{(\star)}$ identifies the equation that changed compared to the previous definition.
\end{dfn}

\begin{lem}
\label{va_lem:subst_preserve_i}
$$ \Gamma,\overline{x} : \overline{A}\vdash^i M : T
\quad \mbox{and} \quad \Gamma \vdash^0 N_k : B_k \mbox{, } k \in
1..n \qquad \mbox{ implies } \qquad \Gamma \vdash^i
M[\overline{N}/\overline{x}] : T$$
\end{lem}

\begin{proof}
Suppose that $\Gamma,\overline{x}: \overline{A} \vdash^i M :T$ and
$\Gamma \vdash^0 N_k : B_k$ for $k \in 1..n$.

We prove $\Gamma \vdash^i M[\overline{N}/\overline{x}]$ by induction
on the size of the proof tree of $\Gamma,\overline{x}:\overline{A}
\vdash^i M : T$ and by case analysis on the last rule used. We just
give the detail for the abstraction case. Suppose that the property
is verified for terms whose proof tree is smaller than $M$. Suppose
$\Gamma,\overline{x}:\overline{A} \vdash^0 \lambda \overline{y} :
\overline{C}. P : (\overline{C}|D)$ where $\Gamma,
\overline{x}:\overline{A}, \overline{y}:\overline{C} \vdash^i P :
D$, then by the induction hypothesis $\Gamma,
\overline{y}:\overline{C} \vdash^i
P\subst{\overline{N}}{\overline{x}} : D$. Applying the rule
$\rulename{abs^i}$ gives $\Gamma \vdash^0 \lambda
\overline{y}:\overline{C} . P \subst{\overline{N}}{\overline{x}}$.
\end{proof}


\begin{cor}[Simultaneous substitution preserves safety]
If $M$ is safe and $N_k$ is safe for $k \in 1..n$ then  $M[\overline{N}/\overline{x}]$ is safe
\end{cor}

\subsection{Safe-redex}

In the simply-typed lambda calculus a redex is a term of the form
$(\lambda x . M) N$. We generalize this definition to the safe
lambda calculus:

\begin{dfn}[Safe redex]
We call safe redex a term of the form $(\lambda \overline{x} . M)
N_1 \ldots N_l$ such that:
\begin{itemize}
\item $ \Gamma \vdash^0 (\lambda \overline{x} . M) N_1 \ldots N_l $
\item the variable $\overline{x}=x_1\ldots x_n$ are abstracted altogether by one occurrence of the rule $\rulename{abs}$ in the proof tree.
%\item The terms $(\lambda \overline{x} . M)$, $N_1$, $N_l$ are applied together at once using the $\rulename{app}$ rule :
%$$   \rulef{
%            \Sigma \vdash^{-1} \lambda \overline{x} . M
%            \quad
%            \Sigma \vdash^0 N_1         \quad \ldots \quad \Sigma \vdash^0 N_l
%    }
%    {
%       \Sigma \vdash^0 (\lambda \overline{x} . L) N_1 \ldots N_l
%    } (\mathbf{app})
%$$
%Consequently each $N_i$ is safe.

\item $l\leq n$
\end{itemize}

\end{dfn}

Consequently, not all multi-$\beta$-redex of the form $(\lambda
\overline{x} . M) N_1 \ldots N_l$ is a safe-redex. It is important
to require that the abstraction $\lambda \overline{x}$ can be done
at once, otherwise we would not be able to prove that reducing a
safe-redex produces a safe term.


\todobox{Define the safe reduction: Consider the safe-redex
$(\lambda \overline{x} . M) N_1 \ldots N_l$, it reduces to $\lambda
x_l \ldots x_n . M \subst{N_1 \ldots N_l}{x_1 \ldots x_l}$. The
relation $\beta_s$ is defined on safe-redex: $(s\mapsto t) \in
\beta_s$ iff $s \equiv (\lambda \overline{x} . M) N_1 \ldots N_l$ is
a safe redex and $t \equiv \lambda x_l \ldots x_n . M \subst{N_1
\ldots N_l}{x_1 \ldots x_l}$ }

\todobox{Show that $\betasred \subseteq \betared^*$.}


Using the previous lemma, we will now prove that reducing a
safe-redex produces a safe term:

\begin{lem}
\label{va_lem:safereduction} A safe redex $(\lambda \overline{x} . M )
\overline{N}$ where the $N_i$ are safe reduces to the safe term
$M\subst{\overline{N}}{\overline{x}}$.
\end{lem}

\begin{proof}
\todobox{NEED TO ADAPT THE FOLLOWING PROOF FROM VERSION B TO VERSION A OF THE RULES.}

We note $\overline{A}$ for $A_1, \ldots , A_n$, $\overline{x}'$ for
$x_1 \ldots x_l$ and $\overline{x}''$ for $x_{l+1} \ldots x_n$.

A safe-redex has a proof tree of the following form:
$$
   \rulef{
        \rulef{
            \rulef{
                \rulef{
                    \rulef
                        { \rulef
                            {\vdots}
                            {\Sigma',\overline{x}:\overline{A}\vdash^i L:C  }
                        }
                        {\Sigma' \vdash^0 \lambda \overline{x} . L : \overline{A}|C} \rulename{abs^i}
                }
                {\vdots} r_1
            }
            {\vdots} r_2
            }
            { \Sigma \vdash^{-1} \lambda \overline{x} . L : A_1, \ldots , A_l|B} r_q
            \quad
            \Sigma \vdash^0 N_1 : A_1
            \quad \ldots \quad \Sigma \vdash^0 N_l : A_l
    }
    {
       \Sigma \vdash^0 (\lambda \overline{x} . L) N_1 \ldots N_l : B
    } (\mathbf{app})
$$
with the following conditions:
\begin{enumerate}
\item for $j\in 1..q$, $r_j \in \{ \rulename{seq}, \rulename{wk^0}, \rulename{wk^{-1}} \}$ therefore
$\Sigma = \Sigma' \union \Delta$ where $\Delta$ contains the
variables introduced by the rules $r_1 \ldots r_q$.

\item $A_1, \ldots , A_l|B = A_1, \ldots , A_n|C$ and $l\leq n$. Therefore
$\ord{B} \geq \ord{C}$.
\item The side condition of the rule $\rulename{abs}$ gives: $\forall z \in \Sigma : \ord{z} \geq \ord{B}$
\end{enumerate}


The conditions 2 and 3 ensure that $\forall z \in \Delta : \ord{z}
\geq \ord{C}$ therefore we can use the weakening rule to introduce
all the variable of $\Delta$ in the context of the sequent
$\Sigma',\overline{x}:\overline{A}\vdash^i L:C$:

$$\rulef{\rulef{ \Sigma',\overline{x}:\overline{A}\vdash^i L:C  }
        {\vdots} (wk^i_0)}
        {\Sigma,\overline{x}:\overline{A}\vdash^i L:C} (wk^i_0)
$$

By lemma \ref{va_lem:subst_preserve_i} we obtain:
$$ \Sigma, \overline{x}'':\overline{A}'' \vdash^i L\subst{N_1 \ldots N_l}{\overline{x}'}$$
Finally using the abstraction rule:
$$ \Sigma \vdash^0 \lambda \overline{x}'':\overline{A}'' . L\subst{N_1 \ldots N_l}{\overline{x}'}$$
\end{proof}




\subsection{Particular case of homogeneously-safe lambda terms}

In this section, we will refine the rules of the non-homogenous safe lambda calculus to the homogeneous case.

We recall the definition of type homogeneity given in section \ref{sec:homotypes}:
a type $(A_1, A_2, \ldots A_n, o)$ is said to be homogeneous whenever $\ord{A_1} \geq \ord{A_2} \geq
\ldots \geq  \ord{A_n}$ and each of the $A_i$ are homogeneous. A term is said homogeneous if its type is homogeneous.

Safety in the sense of Knapik et al. (\cite{KNU02}) is defined for homogeneous recursion scheme.
A recursion scheme is homogeneous if
\begin{itemize}
    \item[(i)] the $\Sigma$ constants are of order 1 at most and therefore have homogeneous types;
    \item[(ii)] the variables in the alphabet $\Delta$ have homogeneous types,
    \item[(iii)] and every recursion equations of a safe recursion scheme has a homogeneous type.
\end{itemize}

We now impose the same restrictions on the terms generated by our system rules.
The first and second restriction imposes that every applicative terms in $\mathcal{T}(\Delta)$ must be homogeneous.

Abstractions in simply-typed terms correspond to recursion equations of recursion schemes,
therefore by imposing restriction (iii) to our calculus, all the sequents generated by the rules must be of homogeneous type.

We say that a term is \emph{homogeneously-safe} if its type is homogeneous
and there is a proof tree showing its safety where all the sequents
of the proof tree are of homogenous type.
\vspace{0.5cm}

We are now ready to look at each rule individually and try to specialize them to the homogeneous case.

\subsubsection{The application rule}
We recall the rule $\mathbf{(app^i)}$:
$$
 \mathbf{(app^i)} \quad  \rulef{\Gamma \vdash^i M : A\rightarrow B
                                        \qquad \Gamma \vdash^{0} N : A}
                                   {\Gamma  \vdash^{u} M N : B}
\quad \mbox{where } u = \min(i+\max\left(0, 1 + \ord{A} -
\ord{B}\right),0)
$$

Because of type homogeneity we have $\ord{A\typear B } = 1 +
\ord{A}$. The second premise gives $\forall z \in \Gamma : \ord{z}
\geq \ord{A} = 1 + \ord{A} - 1$. Hence the exponent $i$ in the first
premise can be replaced by $-1$.


Moreover type homogeneity implies $\ord{A} \geq \ord{B}-1$ therefore
$1 + \ord{A} - \ord{B} \geq 0$ and
$$ u = \min(i+1 + \ord{A} - \ord{B},0) = \min(\ord{A} - \ord{B},0)$$

\begin{itemize}
\item Suppose that $\ord{A} \geq \ord{B}$ then $u=0$ and we obtain the following rule:
$$ \mathbf{(app_1)} \quad  \rulef{\Gamma \vdash^{-1} M : A\rightarrow B
                                        \qquad \Gamma \vdash^{0} N : A }
                                   {\Gamma  \vdash^{0} M N : B}
                                    \qquad \ord{A} \geq \ord{B}$$

\item Suppose that $\ord{A} = \ord{B} - 1$ then
$ u = -1$ and we obtain the following rule:
$$ \mathbf{(app_2)} \quad  \rulef{\Gamma \vdash^{-1} M : A\rightarrow B
                                        \qquad \Gamma \vdash^{0} N : A
                                   }
                                   {\Gamma  \vdash^{-1} M N : B}
                                    \qquad \ord{A} = \ord{B} - 1$$
\end{itemize}



\subsubsection{The abstraction rule}

Let us derive the abstraction rule specialized for the case of
homogeneous types. We recall the rule $\rulename{abs}$:
$$ \rulename{abs^i} \quad  \rulef{\Gamma, \overline{x} : \overline{A} \vdash^{i} M : B}
                                   {\Gamma  \vdash^{0} \lambda \overline{x} : \overline{A} . M : (\overline{A},B)} \qquad
                                   \forall y \in \Gamma : \ord{y} \geq \ord{\overline{A},B}$$

We also use the notation $(\overline{A}|B)$ to denote the
homogeneous type $(A_1, A_2, \ldots A_n, B)$ where $\ord{A_1} =
\ord{A_2} =  \ldots \ord{A_n} \geq \ord{B} -1$.


Suppose that we abstract the single variable $\overline{x} = x$,
then in order to respect the side condition, we need to abstract all
variables of order lower or equal to $\ord{x}$. In particular we
need to abstract the partition of the order of $x$. Moreover to
respect type homogeneity, we need to abstract variables of the
lowest order first.

Hence we can change the abstraction rule so that it only allows
abstraction of the lowest variable partition. The rule can then be
used repeatedly if further partitions need to be abstracted.

The context $\Gamma$ is partitioned according to the order of the
variables. The partitions are written in decreasing order of type
order. The notation $\Gamma | \overline{x}:\overline{A}$ means that
$\overline{x}:\overline{A}$ is the lowest partition of the context.
We obtained the following rule:
$$ \rulename{abs^i} \quad  \rulef{\Gamma| \overline{x} : \overline{A} \vdash^{-1} M : B}
                                   {\Gamma  \vdash^{0} \lambda \overline{x} : \overline{A} . M : (\overline{A}|B)}$$

We note that the side-condition has disappeared.

\subsubsection{Remaining rules}

$\rulename{app}$ and $\rulename{abs}$ are the only syntactic rules (rules performing a structural modification on the term).
The judgement sequents appearing in these rules are all of the form
$\Gamma \vdash^0 M : T$ and $\Gamma \vdash^{-1} M : T$. Hence we can constraint the
other rules by forcing the sequents appearing in their premises and conclusions to be of the form
$\Gamma \vdash^0 M : T$ or $\Gamma \vdash^{-1} M : T$. This does not alter the set of terms that can be generated by the entire set of rules.

The rules $\rulename{wk^i}$,
$\rulename{var}$, $\rulename{const}$, $\rulename{seq}$ and
$\rulename{perm}$ are given in table \ref{va_tab:homosafelmd_rules}.


\subsubsection{The entire system of rules}

Table \ref{va_tab:homosafelmd_rules} recapitulates
the entire set of rules including the simplified .

\begin{table}[htbp]
$$  \rulename{perm}
    \rulef
      { \Gamma \vdash^0 M:B \qquad \sigma(\Gamma)  \hbox{ homogeneous} }
      { \sigma(\Gamma) \vdash^0 M : B }
\qquad \rulename{seq} \quad \rulef{\Gamma \vdash^{0} M : A}{\Gamma
\vdash^{-1} M : A}
$$

$$
 \rulename{const}
    \rulef{} {\vdash^0 b : o^r \rightarrow o} \quad b : o^r \rightarrow o \in \Sigma
\qquad
 \rulename{var} \quad  \rulef{}{x : A\vdash^{0} x : A} $$

$$ \rulename{wk^{0}} \quad  \rulef{\Gamma \vdash^{0} M : A}{\Gamma , x : B \vdash^{0} M : A} \quad \ord{B} \geq \ord{A} $$

$$ \rulename{wk^{-1}} \quad  \rulef{\Gamma \vdash^{-1} M : A}{\Gamma , x : B \vdash^{-1} M : A} \quad \ord{B} \geq \ord{A} -1$$


$$ \mathbf{(app_1)} \quad  \rulef{\Gamma \vdash^{-1} M : A\rightarrow B
                                        \qquad \Gamma \vdash^{0} N : A }
                                   {\Gamma  \vdash^{0} M N : B}
                                    \qquad \ord{A} \geq \ord{B}$$

$$ \mathbf{(app_2)} \quad  \rulef{\Gamma \vdash^{-1} M : A\rightarrow B
                                        \qquad \Gamma \vdash^{0} N : A
                                   }
                                   {\Gamma  \vdash^{-1} M N : B}
                                    \qquad \ord{A} = \ord{B} - 1$$

$$ \rulename{abs} \quad  \rulef{\Gamma| \overline{x} : \overline{A} \vdash^{-1} M : B}
                                   {\Gamma  \vdash^{0} \lambda \overline{x} : \overline{A} . M : (\overline{A}|B)}$$
\caption{Rules of the homogeneously-safe lambda calculus}
\label{va_tab:homosafelmd_rules}
\end{table}

\subsubsection{Comparison with the rules of table \ref{tab:homosafelmd_rules_refined}}

The application rules differ from the application rules of table \ref{tab:homosafelmd_rules_refined}.

Here is a counter-example. Consider the following alphabet:
\begin{eqnarray*}
 x&:&o\\
 \varphi, \theta &:& (o \typear o) \typear o \\
 f &:& \tau = (o \typear o) \typear (o \typear o) \typear o
\end{eqnarray*}
and the following terms:
\begin{eqnarray*}
\emptyset &\vdash^0& M \equiv (\lambda \varphi \theta . \varphi (\lambda x . x)) : ((o \typear o) \typear o) \typear ((o \typear o) \typear o) \typear o \\
f:\tau &\vdash^0& N \equiv f (\lambda x . x) : (o \typear o) \typear o
\end{eqnarray*}
Then we have $ f:\tau \vdash^{-1} M $ which by
$\rulename{app_2}$ gives:
$$ f : \tau \vdash^{-1} M N : ((o \typear o) \typear o) \typear o$$

However this judgement is not valid with the system of rules given in table \ref{tab:homosafelmd_rules_refined}. This is because
in this system whenever $\Gamma \vdash^i M$, any variable $x$ \emph{occurring free} in $M$ must verify $\ord{x}\geq\ord{M}$. But
here $f$ occurs freely in $M N$ and $\ord{f} = 2 < 3 = \ord{M N}$.
