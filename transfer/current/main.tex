\documentclass[nocenter,sfbold]{thesis}

\input preambule.tex

%\includeonly{chap_previouswork,chap_gamesem,safe_background,safe_homog,safe_nonhomog,safe_gamesem2,chap_further}
%\includeonly{safe_gamesem2}


\author{William Blum}
\title{Transfer thesis}
\institution{Oxford University Computing Laboratory}

\begin{document}
\maketitle \tableofcontents

\chapapp{}
\part{Academic activities} %Previous work and plan of proposed work}
%\chapapp{Chapter}
\chapapp{}

\chapter{First-Year work}
%\addcontentsline{toc}{chapter}{First-Year work}

\section{Coursework}
I have attended the following courses: \emph{Automata Logic and
Games} in Hilary term 2005, \emph{Domain theory} in Michaelmas term
2005 and \emph{Categories Proofs and Programs} in Hilary term 2006.

\section{Teaching}

I was the demonstrator for \emph{Network and Operating Systems}
practicals in Hilary term 2005, I tutored two groups of students for
the \emph{Introduction to Specification} classes (Hilary 2006) and I
was the marker for one group.

\section{Meetings and conferences}
\begin{itemize}
\item I attended Bonn spring school on GAMES in March 2005;

\item  I attended BCTCS (British Colloquium in
Theoretical Computer Science) in Nottingham in March 2005 where I
gave a presentation based on my MSc dissertation ``Termination
analysis of a subset of CoreML'';

\item I attended PAT \emph{Program transformation and Analysis} in Copenhagen, July 2005;

\item Marktoberdorf Summer School;
\item CSL (Computer Science Logic) August 2005:
I helped to organise the conference;
\item I visited the Isaac Newton Institute in Cambridge in February
2006.
\end{itemize}
I have also done a presentation during the Computer Laboratory open
days.


\section{Research}

\subsection{Game semantics}

During the past months, I have studied a restriction of
lambda-calculus called ``safe lambda-calculus''. \emph{Safety} is a
syntactic property originally defined in \cite{KNU02} for
higher-order recursion schemes (grammars). In their paper they
proved that the MSO theory of the term tree generated by a safe
recursion scheme of level $n$ is decidable. More recently, Ong
proved in \cite{OngLics2006} that the safety assumption is in fact
not necessary for the decidability of MSO theories.

I am interested in the transposition of the safety property from
grammars to lambda terms. A definition of the safe
$\lambda$-calculus was first given in a technical report by Aehlig,
de Miranda and Ong in \cite{safety-mirlong2004}. One interesting
property is that performing substitution on safe terms does not
require a renaming of the variable.

I have investigated different possible definitions of a safe lambda
calculus and have proposed a more general notion of safety that does
not assume homogeneity of types while still preserving  the ``no
variable renaming'' property.

I also tried to relate the safety restriction and the
\emph{size-change termination} property defined in
\cite{jones01,jones04}. Jones conjectured that any simply-typed term
is size-change terminating, however Damien Sereni disproved this
conjecture by exhibiting a class of counter-examples
(\cite{serenistypesct05}). It turns out that the simply-typed terms
of this class are all safe (but not necessary of homogeneous type)
and not size-change terminating. This suggests that there is no real
interesting relation between safety and size-change termination.


Recently, inspired by my reading on game semantics
\citep{abramsky:game-semantics-tutorial} and by the techniques
developed by Luke Ong in \citep{OngLics2006}, I have proved a result
on the game semantics of safe terms: the pointers in the game
semantics of safe simply-typed terms can be recovered uniquely from
the sequence of moves. This result is similar to the standard result
in game semantics which says that pointers of strategies can be
recovered uniquely for arena of order 2 at most.


\subsection{Verification}

In parallel, I worked on a separate project with Matthew Hagues and
Luke Ong. We developed a SAT-based  model checker for verifying
Linear Temporal Logic (LTL) formulae on programs expressed as finite
state machines. Our approach combines techniques presented in two
papers: \cite{hammer:truly, DBLP:conf/cav/McMillan03}.

In \cite{DBLP:conf/cav/McMillan03}, McMillan describes an
acceleration technique for the SAT-based Bounded Model Checking
problem based on Craig interpolants. His algorithm significantly
improves the performance of the standard SAT-based model checking
method in the case of positive instances.

In \citep{hammer:truly}, Hammer \emph{et al.} introduced a new kind
of automata called \emph{Linearly Weak Alternating Automata},
abbreviated LWAA. The set of languages recognized by these automata
are exactly the set of languages definable in LTL. There is a
straightforward translation from LTL formulae to LWAAs. The size of
the resulting automaton is linear in the size of the LTL formula.
Checking emptiness of LWAA then amounts to searching the
configuration graph for a lasso verifying certain conditions.

Our approach can be summarized as follows: we translate the model
checking problem into an emptiness checking of LWAA. The automata is
empty if and only if the formula is true. The emptiness of the
automaton is expressed in term of a reachability problem. As in the
traditional SAT-based bounded-model checking approach
(\cite{biere99symbolic}), we construct a boolean formula which is
satisfiable if and only if the desired configuration is reachable in
at most $k$ steps (i.e. there is a counter-example of length $k$ at
most).

Furthermore, instead of using the traditional SAT-solver technique,
which iterates $k$ until the completeness threshold is reached, we
use the acceleration method described in
\cite{DBLP:conf/cav/McMillan03}. The principle is the following: for
every iteration of $k$, if the formula is not satisfiable then we
perform some over-approximation of the set of initial configuration.

Suppose that the final configuration becomes reachable in $k$ steps
from the over-approximated initial configuration then we are still
uncertain whether the formula has a valid counter-example because
the counter-example obtained may be spuriously created by the
over-approximation. We therefore increase $k$ and move on to the
next iteration. However, if after performing several
over-approximations we reach a fixed point and the formula is still
not satisfiable (not counter-example of length $k$ at most) then we
know that there cannot be any counter-example of any length. We have
therefore reached the completeness threshold and we know that the
formula is true.


There are two reasons why we think that our approach may lead to a
gain of performance. Firstly, although determining emptiness of a
LWAA is more costly than determining emptiness of a B\"uchi
automaton, we save time during the construction of the automaton
because the size of a LWAA is linear in the length of the formula as
opposed to the standard translation which produces a B\"uchi
automaton of size exponential in the length of the formula.
Secondly, in the case where there is no counter-example, McMillan's
acceleration method based on over-approximation permits quick
detection of attaintment of the completeness threshold.


%\cite{ckos2005}
We have produced an experimental implementation in OCaml and C. The
program parses a file in the NuSMV format (\cite{CAV02:nusmv})
containing the kripke structure of the model and the set of LTL
properties to verify. Our tools can be interfaced with two SAT
solvers: ZChaff \citep{zChaff} and MiniSat \citep{ES03}. We also use
BDD to perform simplification on the propositional formula and to
generate the CNF representation that the SAT solver takes as input.

Compared to the LWAASpin LTL model checker (\cite{hammer:truly}),
our tool performs quite poorly. As soon as a model is taken into
account, our procedure generates increasingly bigger propositional
formulae that the SAT solver struggles to solve. However, for pure
LTL emptiness checking, our tool performs quite well.

It seems disappointing that our approach does not give good results
for model checking, however the reason seems to be that the
SAT-solvers we are using produce bad interpolants. In the future, we
would like to interface our model checking tool with other SAT
solvers and interpolers.

Furthermore, there are optimizations that we have not finished to
implement. These include the optimization of the encoding of the
bounded model checking problem into a propositional formula. We
propose to do some experimental tests do discover the encoding
giving the best performance.

\chapter{Research plan}
%\addcontentsline{toc}{chapter}{Research plan}

My research plan for the coming year is as follows: first I will
continue to work on the Safe $\lambda$-calculus. My immediate goal
is to extend the result I obtained about the unique recoverability
of pointers in the game semantics of Safe simply-typed
$\lambda$-calculus to the case of other languages like Safe
Idealized Algol. I also wish to investigate applications in
algorithmic game semantics. There are also further questions about
Safe $\lambda$-calculus that have to be addressed: what is the
categorical interpretation of Safe $\lambda$-calculus? What kind of
proof theory do we obtain by the Curry-Howard isomorphism? Which
complexity class is characterized by the Safe-$\lambda$ calculus?

In parallel to that line of research, I will continue to work with
Matthew Hagues and Luke Ong on the LTL model checking problem.


%I also want to investigate some application of game semantics to
%program analysis and transformation by trying to extend the work of
%Dimovski \emph{et al.} (\cite{DBLP:conf/sas/DimovskiGL05}) on
%data-absraction refinement based on game semantics.


\setcounter{chapter}{0}
\chapapp{Chapter}
\part{Summary of work so far}

The first chapter of this part is devoted to the presentation of the
basics and main results of game semantics. The categorical
interpretation of game semantics is presented as well as the full
abstraction result for \pcf. We also give a brief summary of the
main results in algorithmic game semantics. There is no personal
contribution in this chapter.

In the second chapter we present the \emph{Safe $\lambda$-Calculus}.
Originally, \emph{safety} has been introduced as a syntactical
restriction on higher-order grammars in order to show a decidability
result about MSO theory of infinite trees \citep{KNU02}. In
\cite{safety-mirlong2004}, Aehlig, de Miranda and Ong  proposed an
adaptation of the safety restriction to the $\lambda$-calculus. This
restriction gives rise to the Safe $\lambda$-Calculus. We first
present this calculus and then give a more general definition which
does not make any assumption on the types of the terms.

In the third chapter, following ideas described in
\cite{OngLics2006}, we introduce the notions of computation tree of
a simply-typed term and traversal over a computation tree. We prove
a theorem showing a correspondence between traversals of the
computation tree and the game semantics of a term. Based on that
correspondence, we give a characterisation of the game semantics of
safe terms by a property called ``incremental justification''. In
incrementally-justified strategies, pointers are superfluous (i.e.
they can be recovered uniquely from the underlying sequence of
moves). This simplification of the game semantics suggests some
potential applications in algorithmic game semantics. We finish the
chapter by extending the result to Safe \pcf\ and by giving the key
elements for an extension to full Safe Idealized Algol.


% first chapter
\chapter{Game semantics}

The aim of this chapter is to introduce game semantics. It starts
with a history of game semantics and a presentation of the full
abstraction problem for PCF which has been solved using game
semantics. It then goes on by introducing the basic notions of game
semantics and by giving a categorical interpretation of games.
Finally we show how games are used to define a syntax-independent
model of programming languages like PCF and Idealized Algol (IA).

This chapter is largely based on the tutorial by Samson Abramsky
tutorial on Game Semantics \cite{abramsky:game-semantics-tutorial}.
Many details and proofs will be omitted and we refer the reader to
\cite{hylandong_pcf, abramsky94full} for a complete description of
game semantics.

\section{History}

\subsection{Game semantics}

In the 1950s, Paul Lorenzen invented Game semantics as a new
approach to study semantics of intuitionistic logic \citep{lor61}.
In this setting, the notion of logical truth is modeled using game
theoretic concepts such as the existence of winning strategy.

Four decade later, game semantics is used to prove the full
completeness of Multiplicative Linear Logic (MLL)
\citep{abramsky92games,HO93a}. Shortly after, a connection between
games and linear logic has been established. Game semantics has then
been used as a new paradigm to study formal models of programming
languages. The idea is to model the execution of a program as a game
played by two protagonists: the Opponent representing the
environment and the Proponent representing the system. The meaning
of the program is then modeled by a strategy for the Proponent.


Subsequently, these game-based model have been used to give a
solution to the long-standing problem of ``Full abstraction of PCF''
\citep{abramsky94full, hylandong_pcf,Nickau:lfcs94}.

Based on that major result, and in a more applied direction, games
have been used as a new tool for software verification
\cite{ghicamccusker00}. This open-up a new field called Algorithmic
Game Semantics \citep{Abr02}.




\subsection{Model of programming languages}

Before the 1980s, there were many approaches to define models for
programming languages. Among the successful ones, there were the
axiomatic, operational and denotational semantics:
\begin{itemize}
\item Operational semantics gives a meaning to a program by describing the
behaviour of a machine executing the program. It is defined formally
by giving a state transition system.
\item Axiomatic semantics defined the behaviour of the program
with axioms and is used to prove program correctness by static
analysis of the code of the program.
\item The denotational semantics approach consists in mapping a program to a mathematical structure
having good properties such as compositionality. This mapping is
achieved by structural induction on the syntax of the program.
\end{itemize}

In the 1990s, three different independent research groups: Samson
Abramsky, Radhakrishnan Jagadeesan and Pasquale Malacaria
\citep{abramsky94full}, Martin Hyland and Luke Ong
\citep{hylandong_pcf} and Nickau \citep{Nickau:lfcs94} have
introduced game semantics, a new kind of semantics, in order to
solve a long standing problem in the semanticists community :
finding a fully abstract model for PCF.

\subsection{The problem of full abstraction for PCF}

PCF is a simple programming language introduced in a classical paper
by Plotkin ``LCF considered as a programming language''
(\cite{DBLP:journals/tcs/Plotkin77}). PCF is based on LCF, the Logic
of Computable Functions devised by Dana Scott in \cite{scott_lcf}.
It is a simply typed lambda calculus extended with arithmetic
operators, conditional and recursion.

The problem of the Full Abstraction for PCF goes back to the 1970s.
In \citep{scott93}, Scott gave a model for PCF based on domain
theory. This model gives a sound interpretation of observational
equivalence: if two terms have the same domain theoretic
interpretation then they are observationally equivalent. However the
converse is not true: there exist two PCF terms which are
observationally equivalent but have different domain theoretic
denotation. We say that the model is not fully abstract.

The key reason why the domain theoretic model of PCF is not fully
abstract is that the parallel-or operator defined by the following
truth table
\begin{center}
\begin{tabular}{l|lll}
p-or  & $\bot$ & tt & ff \\ \hline
$\bot$ & $\bot$ & tt & $\bot$\\
tt & tt & tt & tt\\
ff & $\bot$ & tt & ff\\
\end{tabular}
\end{center}
is not definable as a PCF term! It is possible to create two
different PCF terms that always behave the same except when they are
apply to a term computing p-or. Since p-or is not definable in PCF,
these two terms will have the same denotation. This implies that the
model is not fully abstract.


It is possible to patch PCF by adding the operator $p-or$, the
resulting language ``PCF+p-or'' becomes fully-abstracted by Scott
domain theoretic model \citep{DBLP:journals/tcs/Plotkin77}. However
the language we are now dealing with is strictly more powerful than
PCF, it allows parallel execution of commands whereas PCF only
permits sequential execution.

Another approach consists in eliminating  the undefinable elements
(like p-or) by strengthening the conditions on the function used in
the model. This approach has been followed by Berry in
\cite{berry-stable,gberry-thesis} where he gives a model based on
stable functions, a class of function smaller than the class of
strict and continuous function. Unfortunately this approach did not
succeed.

The only successful approaches to obtain a fully abstract model for
PCF were the ones taken by Ambramsky, Jagadeesan and Malacaria
\citep{abramsky94full}, Hyland and Ong \citep{hylandong_pcf} and
Nickau \citep{Nickau:lfcs94}, all based on game semantics.

This result has then been adapted to other varieties of programming
paradigm including languages with stores (Idealized Algol),
call-by-value \citep{honda99gametheoretic, abramsky98callbyvalue}
and call-by-name, general referencees
\citep{DBLP:conf/lics/AbramskyHM98}, polymorphism
\citep{DBLP:journals/apal/AbramskyJ05}, control features
(continuation and exception), non determinism, concurrency. In all
these cases, the game semantics model led to a syntax-independent
fully abstract model of the corresponding language.

\section{Games}
\label{sec:catgames}

We now introduce formally the notion of game that will be used in
the following section to give a model of the programming languages
PCF and Idealized Algol. The definitions are taken from
\cite{abramsky:game-semantics-tutorial, hylandong_pcf,
abramsky94full}.


\subsection{Arenas and Games}

The games we are interested in are two-players games. The players are named O for Opponent and P for Proponent.

The game played by O and P is constraint by something called
\emph{arena}. The arena defines the possible moves of the game. By
analogy with real board games, the arena represents the board
together with the rules that tell how players can make their moves
on the board. In fact the analogy with board game stops here. Our
games can be thought as dialog games: one person O interviews
another person P, P tries to answer the initial O-question by
possibly asking O some precisions about its initial question.
Moreover, the notion of winner and winning strategy will not be
relevant in our setting.


More formally, the arena can be seen as a forest of trees whose nodes are possible questions and leaves are possible answers.
The arena is partitioned into two kinds of moves: the moves that can be played by P and the ones that can be played by O.
A move is either a question to the other player or an answer to a question previously asked by the other player.

Each move of the game must be justified by another move that has already been played by the other player. This justification relation
is induced by the edges of the forest arena. Moreover, an answer must always be justified by the question that it answers and a question
is always justified by another question.

\begin{dfn}[Arena]
An arena is a structure $\langle M, \lambda, \vdash \rangle$ where:
\begin{itemize}
\item $M$ is the set of possible moves;
\item $(M,\vdash)$ is a forest of trees;

\item $\lambda : M \rightarrow \{ O, P\} \times \{Q, A\}$ is a labeling functions indicating whether a given move
    is a question or an answer and whether it can be played by O or by P.

    $\lambda = [\lambda^{OP},\lambda^{QA}]$ where $\lambda^{OP} : M \rightarrow  \{ O, P\}$
    and $\lambda^{QA} : M \rightarrow  \{ Q, A\}$.

    \begin{itemize}
    \item If $\lambda^{OP} (m) = O$, we call $m$ and O-move otherwise $m$ is a P-move.
    $\lambda^{QA} (m) = Q$ indicates that $m$ is a question otherwise $m$ is an answer.

    \item For any leaf $l$ of the tree $(M,\vdash)$, $\lambda^{QA} (l) = A$ and for any node
    $n \in (M,\vdash)$, $\lambda^{QA} (n) = Q$.
    \end{itemize}

\item The forest of tree $(M,\vdash)$ respect the following condition:
    \begin{itemize}
    \item[(e1)] The roots are O-moves: for any root $r$ of $(M,\vdash)$, $\lambda^{OP} (r) = O$.
    \item[(e2)] Answers are enabled by questions: $m \vdash n  \zand \lambda^{QA}(n) = A \imp \lambda^{QA}(m) = Q$.
    % Or more succinctly, if we write $\dashv$ the relation $\vdash^-1$: $\lambda^{QA} \left( \dashv( (\lambda^{QA})^{-1}(\{A\}) ) \right) = \{ O \}$
    \item[(e3)] A player move must be justified by a move played by the other player:
         $m\vdash n \imp \lambda^{OP}(m) \neq \lambda^{OP}(n)$.
    \end{itemize}
\end{itemize}
\end{dfn}

For commodity we write the set $\{O,P\} \times \{Q,A\}$ as $\{OQ,OA,PQ,PA\}$.
$\overline{\lambda}$ denotes the labeling function $\lambda$ with the question and answer swapped. For instance:
$$\overline{\lambda(m)} = OQ \iff \lambda(m) = PQ$$

The roots of the forest of tree $(M,\vdash)$ are the \emph{initial moves}.

For example, the simplest possible arena is written $\mathbf{1}$ and
denotes the arena which set of moves $M$ is empty.

\begin{exmp}[The flat arena]
\label{exmp:flatarena}

 Let $A$ be any countable set then the flat arena over $A$
is defined to be the arena $\langle M, \lambda, \vdash \rangle$ such
that $M$ has one move $q$ with $\lambda(q) = OQ$ and for each
element in $A$, there is a corresponding move $a_i$ in $M$ with
$\lambda(a_i) = PA$ for some $i \in \nat$. The enabling relation
$\vdash$ is defined to be $\{ q \vdash a_i \ | i \in \nat \}$.

This arena is represented by the following tree:
\begin{center}
  \pstree[levelsep=6ex]
    { \TR{$q$} }
    {    \TR{$a_1$} \TR{$a_2$} \TR{\ldots} }
\end{center}
The vertices represent the moves and the edges represent the
enabling relation.

The flat arena over $\nat$ and $\mathbb{B}$ is written
$\mathbf{int}$ and  $\mathbf{bool}$ respectively.

\end{exmp}

Once the arena has been defined, the bases of the game are set and the players have something to play with.
We now need to describe the state of the game, for that purpose
we introduced \emph{justified sequences of moves}. Sequence of moves are used to record the history of all the moves that have been
played.

\begin{dfn}[Justified sequence of moves]
A justified sequence is a sequence of moves $s$ together with an associated sequence of pointers. Any
move $m$ in the sequence that is not initial has as pointer that points to a previous move $n$ that justifies it (i.e. $n \vdash m$).
\end{dfn}

The pointers of a justified sequences are represented with arrows.
This is an example of justified sequence of moves:
$$\rnode{q4}{q}^4
\rnode{q3}{q}^3 \rnode{q2}{q}^2 \rnode{q3b}{q}^3 \rnode{q2b}{q}^2
\rnode{q1}{q}^1 \bkptrc{q3}{q4} \bkptrc{q2}{q3}
\bkptrc[ncurv=0.6]{q3b}{q4} \bkptrc{q2b}{q3b}$$

The first move of a justified sequence must be an O-move since
initial moves are all O-moves.

Notation: we write $s t$ or sometimes $s \cdot t$ do denote the
sequences obtain by concatenating $s$ and $t$. The empty sequence is
written $\epsilon$.

 A justified sequence has two particular subsequences which
will be of particular interest later on when we introduce
strategies. These subsequences are called the P-view and the O-view
of the sequence. The idea is that a view describes the local context
of the game. Here is the formal definition:

\begin{dfn}[View]
Given a justified sequence of moves $s$. We define the proponent view (P-view) noted $\pview{s}$ by induction:
\begin{align*}
\pview{\epsilon} &= \epsilon \\
\pview{s \cdot m} &= \pview{s} \cdot \ m && \mbox{ if $m$ is a P-move} \\
\pview{s \cdot m} &= m && \mbox{ if $m$ is initial (O-move) } \\
\pview{ s \cdot \rnode{m}{m} \cdot t \cdot \rnode{n}{n} \bkptra{50}{n}{m} } &=
 \pview{s} \cdot \rnode{mm}{m} \cdot \rnode{nn}{n} \bkptra{70}{nn}{mm} && \mbox{ if $n$ is a non initial O-move }
\end{align*}
The O-view $\oview{s}$ is defined similarly:
\begin{align*}
\oview{\epsilon} &= \epsilon \\
\oview{s \cdot m} &= \oview{s} \cdot \ m && \mbox{ if $m$ is a O-move} \\
\oview{ s \cdot \rnode{m}{m} \cdot t \cdot \rnode{n}{n} \bkptra{50}{n}{m} } &=
 \pview{s} \cdot \rnode{mm}{m} \cdot \rnode{nn}{n} \bkptra{70}{nn}{mm} && \mbox{ if $n$ is a P-move }
\end{align*}
\end{dfn}

In fact not all justified sequences will be of interest for the
games that we will use. We call \emph{legal position} any justified
sequence verifying two additional conditions: alternation and
visibility. Alternation says that players O and P plays
alternatively. Visibility expresses that each non-initial move is
justified by a move situated in the local context at that point.
Intuitively, the visibility condition gives some coherence to the
justification pointers of the sequence.

\begin{dfn}[Legal position]
A legal position is a justified sequence of move $s$ respecting the following constraint:
\begin{itemize}
\item Alternation: For any subsequence $m \cdot n$ of $s$, $\lambda^{OP}(m) \neq \lambda^{OP}(n)$.
\item Visibility: For any subsequence $t m$ of $s$ where $m$ is not initial, if $m$ is a P-move then $m$ points to a move in $\pview{s}$
and if $m$ is a O-move then $m$ points to a move in $\oview{s}$.
\end{itemize}

The set of legal position of an arena $A$ is noted $L_A$.
\end{dfn}

We say that a move $n$ is hereditarily justified by a move $m$ if there is a sequence of move
$m_1, \ldots, m_q$ such that:
$$ m \vdash m_1 \vdash m_2 \vdash \ldots m_q \vdash n$$
If a move has no justification pointer, we says that it is an
\emph{initial move} (in that case it must be a root of the forest
arena).

Suppose that $n$ is an occurrence of a move in the sequence $s$ then
$s \upharpoonright n$ denotes the subsequence of $s$ containing all the moves hereditarily justified by $n$.
Similarly, $s \upharpoonright I$ denotes the
subsequence of $s$ containing all the moves hereditarily justified by the moves in $I$.

\begin{dfn}[Game]
A game is a structure $\langle M, \lambda, \vdash, P \rangle$ such that
\begin{itemize}
\item $ \langle M, \lambda, \vdash \rangle$ is an arena.
\item $P$ is called the set of valid positions, it is:
    \begin{itemize}
    \item a non-empty prefix closed subset of the set of legal position
    \item closed by initial hereditary filtering: if $s$ is a valid position then for any set $I$ of occurrences of initial moves
    in $s$, $s\upharpoonright I$ is also a valid position.
    \end{itemize}
\end{itemize}
\end{dfn}

\begin{exmp}  Consider the flat arena  $\mathbf{int}$.
The set of valid position $P = \{ \epsilon, q \} \union \{ q \cdot
a_i \ | i \in \nat \}$ defines a game on the arena $\mathbf{int}$.
\end{exmp}

\subsection{Constructions on games}
\label{sec:gameconstruction}

We now define game constructors that will be useful later on.

Consider the two functions $f : A \rightarrow C$ and $g : B
\rightarrow C$, we write $[f,g]$ to denote the pairing of $f$ and
$g$ defined on the direct sum $A + B$. Given a game $A$ with a set
of moves $M_A$, we use the filtering operator $s \upharpoonright A$
do denote the subsequence of $s$ consisting of all moves in $M_A$.
Although this notation conflicts with the hereditarily filtering
operator, it should not cause any confusion.

\subsubsection{Tensor product}
Given two games $A$ and $B$ we define the tensor product constructor
$A \otimes B$ as follows:
\begin{eqnarray*}
  M_{A \otimes B} &=& M_A + M_B \\
  \lambda_{A\otimes B} &=& [\lambda_A,\lambda_B] \\
  \vdash_{A\otimes B} & = & \vdash_{A}\ \union\ \vdash_{B} \\
  P_{A\otimes B} & = & \{ s \in L_{A\otimes B} | s \upharpoonright A \in P_A \wedge s \ \upharpoonright B \in P_B  \}.
\end{eqnarray*}

In particular,  $n$ is initial in $A\otimes B$ if and only if $n$ is
initial in A or B. And $m \vdash_{A\otimes B} n$  holds if and only if $m
\vdash_{A} n$ or $m \vdash_{B} n$ holds.

\subsubsection{Function space}
The game $A \otimes B$ is defined as follows:
\begin{eqnarray*}
  M_{A \multimap B} &=& M_A + M_B \\
  \lambda_{A\multimap B} &=& [\overline{\lambda_A},\lambda_B] \\
  \vdash_{A\multimap B} & = & \vdash_{A}\ \union\ \vdash_{B}\ \union\  \{ (m,n) \ |\ m \mbox{ initial in } B \wedge n \mbox{ initial in } A \} \\
  P_{A\otimes B} & = & \{ s \in L_{A\otimes B} | s \upharpoonright A \in P_A \wedge s \ \upharpoonright B \in P_B  \}.
\end{eqnarray*}

Graphically if we draw a triangle to represent an arena $A$ then the
arena for $A \multimap B$ is represented as follows:
\begin{center}
\psset{xunit=.5pt,yunit=.5pt,runit=.5pt}
\begin{pspicture}(150,80)
\rput[tr](150,80){ \pnode(27,40){b} \pstribox{B} } \rput[bl](0,0){
\pnode(27,40){a} \pstribox{A} } \ncline{->}{a}{b}
\end{pspicture}
\end{center}

\subsubsection{Cartesian product}
The game $A \& B$ is defined as follows:
\begin{eqnarray*}
  M_{A \& B} &=& M_A + M_B \\
  \lambda_{A\& B} &=& [\lambda_A,\lambda_B] \\
  \vdash_{A\& B} & = & \vdash_{A}\ \union\ \vdash_{B} \\
  P_{A\& B} & = & \{ s \in L_{A\otimes B} | s \upharpoonright A \in P_A \wedge s \ \upharpoonright B = \epsilon  \} \\
        &&   \union \{ s \in L_{A\otimes B} | s \upharpoonright A \in P_B \wedge s \ \upharpoonright A = \epsilon  \}.
\end{eqnarray*}

A play of the game $A \& B$ is either a play of $A$ or a play of $B$ whether a play
of the game $A \otimes B$ may be an interleaving of plays on $A$ and plays on $B$.

\subsection{Representation of plays}

Plays of the game are usually represented in a table diagram. The
columns of the table correspond to the different components of the
arena and each row corresponds to one move in the play. The first
row always represents an O-move, this is because O is the only
player who can open a game (since roots of the arena are O-moves).

As an example the play
$$\rnode{q1}{q}\
 \rnode{q2}{q}
 \ \rnode{a2}{8}
\  \rnode{a1}{12}
  \bkptrc{a1}{q1}
\bkptrc{a2}{q2} $$
on the
game $\textbf{int} \multimap \textbf{int} $ can be represented by
the following diagram:

\begin{center}
\begin{tabular}{cccc}
\textbf{int} & $\imp$ & \textbf{int} & \\
&& q & O\\
q  &&& P\\
8  &&& O\\
&& 12 & P
\end{tabular}
\end{center}

When it is necessary, the justification pointers of the play can also
be shown on the diagram.


\subsection{Strategy}

\subsubsection{Definition}

During a game, the player who has to play may have several choices
for his next move. The move that he makes is chosen according to a
given strategy.

A strategy is a rule telling the player which move to make when the
game is in a given position. More abstractly, a strategy is a
partial function mapping legal position where Proponent has to move
to P-moves.

\begin{dfn}[Strategy]
A strategy for player P on a given game $\langle M, \lambda, \vdash, P \rangle$ is a
non-empty set of even-length positions from $P$ such that:
\begin{enumerate}
\item (\emph{no unreachable position}) $sab \in \sigma \imp s \in \sigma$
\item (\emph{determinacy}) $sab, sac \in \sigma \quad \imp \quad  b = c$  and $b$ has the same justifier as
$c$.
\end{enumerate}
\end{dfn}

The idea is that the presence of the even-length sequence $s a b$ in
$\sigma$ tells the player P that whenever the game is in position
$s$ and player O plays the move $a$ then it must respond by playing
the move $b$.

The first condition ensures that the strategy $\sigma$ only
considers positions that the strategy itself could have led to in a
previous move. The second condition in the definition requires that
this choice of move is deterministic (i.e. there is a function $f$
from the set of odd length position to the set of moves $M$ such
that $f(s a) = b$).


For any game $A$, the smallest possible strategy is the strategy
that never respond given by $\{ \epsilon \}$. It is called the
\emph{empty strategy} and denoted $\bot$.

\subsubsection{Copy-cat strategy}

For any arena $A$ there is a strategy on the game $A \multimap A$
called the \emph{copy-cat strategy}. We write $A_1$ and $A_2$ to
denote the first and second copy of the arena $A$ in the game $A
\multimap A$. If $A$ is the arena $A_1$ then $A^\perp$ denotes the
arena $A_2$ and reciprocally.

Let $A$ be one of the arena $A_1$ or $A_2$. The copy-cat strategy
operates as follows: whenever P has to respond to an O-move played
in $A$, it replicates the move played by O in the arena $A^{\perp}$
after that $O$ has to respond in $A^{\perp}$ and $P$ replicates this
response in $(A^\perp)^\perp = A$ and so on and so forth.


More formally, the copy-cat strategy is defined by:
$$ \textsf{id}_A = \{ s \in P^{\textsf{even}}_{A \multimap A} \ | \ \forall t \sqsubseteq^{\textsf{even}} s\ .\ t \upharpoonright A_1 = t \upharpoonright A_2 \}$$
where $P^{\textsf{even}}_A$ denotes the valid position of even
length in the game $A$ and $t \sqsubseteq^{\textsf{even}} s$ denotes
that $t$ is an even length prefix of $s$.

The copy-cat strategy is also called \emph{identity strategy} since
it is the identity for strategy composition as we will see in the
next paragraph.

\begin{exmp} The copy-cat strategy on $\textbf{int}$ is:
$$\begin{array}{ccc}
\textbf{int} & \imp & \textbf{int} \\
&& q\\
q \\
n \\
&& n
\end{array}
$$
Note that we introduced this type of diagram to represent plays of
games but, as we can see here, the same diagrams can be used to
represent strategies when the play represented is general enough.

The copy-cat strategy on $\textbf{int} \typar \textbf{int}$ is given
by the following diagram:
$$\begin{array}{ccccccc}
(\textbf{int} & \imp & \textbf{int}) & \imp & (\textbf{int} & \imp & \textbf{int}) \\
&&&& && q\\
&& q\\
q \\
&&&& q \\
&&&& m \\
m\\
&& n \\
&&&& && n
\end{array}$$
\end{exmp}

\subsubsection{Composition}

It is well-known that any model of the simply typed lambda-calculus
is a cartesian closed category \citep{CroleRL:catt}. Games are used
to give a fully-abstract model of PCF, an extended simply typed
lambda calculus, therefore the game model should fit into a
cartesian closed category. This category will have games as objects
and strategies as morphisms. In a category, morphisms should be able
to compose together, therefore there should be an appropriate notion
of strategy composition.

Composition of strategies is an essential feature of game semantics.
As we will see in the following section, in the game model of PCF,
strategies represent programs. Therefore, strategy composition will
prove to be very useful : obtaining the model of a composed program
boils down to composing the strategies of the composing programs.

The way composition is defined for strategies is similar to
``parallel composition plus hiding'' in the trace semantics of CSP
\citep{hoare_csp}. Consider two strategies $\sigma : A \multimap B$
and $\tau : B \multimap C$ that we wish to compose.

For any sequence of moves $u$ on three arenas $A$, $B$, $C$, we call
projection of $s$ on the game $A \multimap B$ and we note $u
\upharpoonright A,B$ the subsequence of $s$ obtained by removing
from $u$ the moves in $C$ and pointers to moves in $C$. The
projection on $B \multimap C$ is defined similarly.

The definition of the projection on $A \multimap B$ differs
slightly: $u \upharpoonright A,C$ is the subsequence of $u$
consisting of the moves from $A$ and $C$ with some additional
pointers: we add a pointer from $a \in A$ to $c\in C$ whenever $a$
points to some move $b \in B$ itself pointing to $c$. All the
pointers to moves in $B$ are removed.


First we remark that for a given legal position $s$ in the game $A
\multimap C$, there is what is called an \emph{uncovering} of $s$.
The uncovering of $s$ is the maximal justified sequence of moves $u$
from the games $A$, $B$ and $C$ such that:
\begin{itemize}
\item The sequence $s$, considered as a pointer-less sequence, is a subsequence of
$u$;
\item the projection of $u$ on the game $A \multimap B$ lies in the
strategy $\sigma$;
\item the projection of $u$ on the game $B \multimap C$
lies in the strategy $\tau$;
\item and the projection of $u$ on the game $A \multimap C$ is a subsequence of $s$ (here the term ``subsequence'' refers to the sequence of nodes together with the auxiliary sequence of pointers).
\end{itemize}
This uncovering, noted $uncover(s, \sigma, \tau)$, is
defined uniquely for given strategies $\sigma$, $\tau$ and legal
position $s$ (this is proved in part II of \cite{hylandong_pcf}).

We define $\sigma \| \tau $ to be the set of uncovering of legal
positions in $A \multimap C$:
$$ \sigma \| \tau = \{ uncover(s, \sigma, \tau) \ | \ s \mbox{ is a legal position in } A \multimap C \}$$

The composition of $\sigma$, $\tau$ is defined to be the set of
projections of uncovering of legal positions in $A \multimap C$:

\begin{dfn}[Strategy composition]
Consider $\sigma : A \multimap B$ and  $\tau : B \multimap C$ two
strategies. We define $\sigma ; \tau$ to be:
$$ \sigma ; \tau = \{ u \upharpoonright A,C \ | \ u \in \sigma \|
\tau \}$$
\end{dfn}

It can be verified that composition is well-defined and associative
\citep{hylandong_pcf} and that the copy-cat strategy $\textsf{id}_A$ is the identity for composition.

\subsubsection{Constraint on strategies}

Different classes of strategies will be considered depending on the
features of the language that we want to model. Here is a list of
common restrictions that we will consider:
\begin{itemize}
\item \emph{Well-bracketing:} In a well-bracketed strategies the players always answer the last unanswered question (called the pending question) first.
If we represent Opponent's question as ``['', Proponent's answer as
``]'', Proponent's question as ``('' and Opponent's answers as ``)''
then requiring that the last pending question is answered first is
the same as requiring that the string representing the play is a
prefix of a well-bracketed sequence.

\item \emph{History-free strategies:} A strategy is history-free if the Proponent's move at any position of the game where he has to play
is determined by the last move of the Opponent. In other words, the
history prior to the last move is ignored by the Proponent when
deciding how to respond.

\item \emph{History-sensitive strategies:} The Proponent follows a history-sensitive strategy if he needs to have access to the full
history of the moves in order to decide which move to make.

\item \emph{Innocence:} a strategy is innocent if it determines Proponent's moves based on a restricted view of the history of the play, mainly the P-view
at that point. Such strategies can be specified by a partial
function mapping P-views to P-moves called the \emph{view function}. However not every partial
function from P-views to P-moves gives rise to an innocent strategy
(a sufficient condition is given in \cite{hylandong_pcf}).
\end{itemize}

The formal definition of innocence follows:
\begin{dfn}[Innocence]
Given positions $sab, ta \in L_A$ where $sab$ has even length and
$\pview{sa} = \pview{ta}$, there is a unique extension of $ta$ by
the move $b$ together with a justification pointer such that
$\pview{sab} = \pview{sa}$. We write this extension
$\textsf{match}(sab,ta)$.

The strategy $\sigma:A$ is \emph{innocent} if and only if:
$$ \left(
     \begin{array}{c}
       \pview{sa} = \pview{ta} \\
       sab \in \sigma \\
       t\in \sigma \wedge ta \in P_A \\
     \end{array}
   \right)
\quad \imp\quad  \textsf{match}(sab,ta) \in \sigma$$

\end{dfn}


\subsection{Categorical interpretation}

In this section we recall some results about the categorical representation of Games.
These results with complete details and proofs can be found in \cite{McC96b,hylandong_pcf,abramsky94full}.
We refer the reader to \cite{CroleRL:catt} for more information about category theory.

We consider the category $\mathcal{G}$ whose objects are games and morphisms are
strategies. A morphism from $A$ to $B$ is a strategy on the game $A \multimap B$.

Three other sub-categories of $\mathcal{G}$ are considered: each of them correspond to some restriction on strategies:
$\mathcal{G}_i$ is the sub-category
of $\mathcal{G}$ whose morphisms are the innocent strategies,
$\mathcal{G}_b$ has only the well-bracketed strategies and $\mathcal{G}_{ib}$ has the innocent and well-bracketed strategies.

\begin{prop}
$\mathcal{G}$, $\mathcal{G}_i$, $\mathcal{G}_b$ and $\mathcal{G}_{ib}$ are categories.
\end{prop}

Proving this requires to prove that composition of strategies is well-defined, associative, has a unit (the copy-cat strategy), preserves innocence and
well-bracketedness. See \cite{hylandong_pcf,abramsky94full} for a proof.


\subsubsection{Monoidal structure}

We have already defined the tensor product on games in section \ref{sec:gameconstruction}.
We now define the corresponding transformation on morphisms:
given two strategies $\sigma : A \multimap B$ and $\tau : C \multimap D$ the strategy
$\sigma \otimes \tau : (A \otimes C) \multimap (B\otimes D)$ is defined by:
$$ \sigma \otimes \tau = \{ s \in L_{A \otimes C \multimap B\otimes D} \ s \upharpoonright A,B \in \sigma
\wedge s \upharpoonright C,D \in \tau \}$$

It can be shown that the tensor product is associative, commutative and has
$I = \langle \emptyset, \emptyset,\emptyset, \{ \epsilon \} \rangle $ as identity.
Hence the game categories $\mathcal{G}$ is a symmetric monoidal categories. Moreover
$\mathcal{G}_i$ and  $\mathcal{G}_b$ are sub-symmetric monoidal categories of $\mathcal{G}$,
and $\mathcal{G}_{ib}$ is a sub-symmetric monoidal category of $\mathcal{G}_i$, $\mathcal{G}_b$ and
$\mathcal{G}$.

\subsubsection{Closed structure}

Given the games $A$, $B$ and $C$, we can transform strategies on $A\otimes B \multimap C$ to
strategies on $A \multimap (B \multimap C)$ by retagging the moves to the appropriate arenas. This transformation
defines an isomorphism noted $\Lambda_B$ and called currying. Therefore the hom-set $\mathcal{G}(A\otimes B, C)$ is isomorphic to the hom-set
$\mathcal{G}(A,B\multimap C)$ which makes $\mathcal{G}$ an autonomous (i.e. symmetric monoidal closed) category.

We write $ev_{A,B} : (A \multimap B) \otimes A \rightarrow B$ to denote the \emph{evaluation strategy} obtained by uncurrying the
identity map on $A \rightarrow B$. $ev_{A,B}$ is in fact the copycat strategy for the game
$(A \multimap B) \otimes A \rightarrow B$.

$\mathcal{G}_i$ and  $\mathcal{G}_b$ are sub-autonomous categories of $\mathcal{G}$,
and $\mathcal{G}_{ib}$ is a sub-autonomous category of $\mathcal{G}_i$, $\mathcal{G}_b$ and
$\mathcal{G}$.

\subsubsection{Cartesian product}
The cartesian product defined in section \ref{sec:gameconstruction} is indeed a cartesian product in the category
$\mathcal{G}$, $\mathcal{G}_i$, $\mathcal{G}_b$ and $\mathcal{G}_{ib}$.

The projections $\pi_1:A \& B \rightarrow A$ and $\pi_1:A \& B \rightarrow B$ are given by the obvious copy-cat strategies.
Given two category morphisms $\sigma :C \rightarrow A$ and $\tau : C \rightarrow B$ the pairing function
$\langle \sigma, \tau \rangle : C \rightarrow A \& B$ is given by:
\begin{eqnarray*}
\langle \sigma, \tau \rangle &=& \{ s \in L_{C\multimap A\&B} \ | \ s \upharpoonright C,A \in \sigma \wedge s \upharpoonright B = \epsilon  \} \\
&\union& \{ s \in L_{C\multimap A\&B} \ | \ s \upharpoonright C,A \in \sigma \wedge s \upharpoonright B = \epsilon  \}
\end{eqnarray*}

\subsubsection{Cartesian closed structure}
Having defined the cartesian product is not enough to turn $\mathcal{G}$ into a cartesian closed category :
we also need to define a terminal object $I$ and the exponential construct $A \imp B$ for any two games $A$ and $B$.
In fact, this cannot be done in the current categories $\mathcal{G}$ and we have to move on to another category
of games noted $\mathcal{C}$ whose objects and morphisms are certain sub-classes of games and strategies.

Before introducing the category $\mathcal{C}$ we need some new definitions:


For any game $A$ we define the exponential game noted $!A$.
The game $!A$ corresponds to a repeated version of the game $A$. Plays of $!A$ are interleaving of plays of
$A$. It is defined as follows:
\begin{eqnarray*}
  M_{!A} &=& M_A \\
  \lambda_{!A} &=& \lambda_A \\
  \vdash_{!A} & = & \vdash_{A} \\
  P_{!A} & = & \{ s \in L_{!A} | \mbox{ for each initial move $m$, } s \upharpoonright m \in P_A \}
\end{eqnarray*}
The following equalities hold:
\begin{eqnarray*}
  !(A \& B) &=& !A \otimes !B\\
  I &=& !I
\end{eqnarray*}

\begin{dfn}[Well-opened games]
A game $A$ is well-opened if for any position $s \in P_A$ the only initial move is the first
one.
\end{dfn}

Well-opened games have single thread of dialog. Then can be turned into games with multiple-thread of dialog
using the promotion operator:

\begin{dfn}[Promotion]
Consider a well-opened game $B$.
Given a strategy on ${!A} \multimap B$, we define it promotion $\sigma^\dagger : {!A} \multimap {!B}$ to be the
strategy which plays several copies of $\sigma$. It is formally defined by:
$$ \sigma^\dagger = \{ s \in L_{{!A} \multimap !B} \ | \ \mbox{ for all initial $m$, } s \upharpoonright m \in \sigma  \}.$$
\end{dfn}

It can be shown that promotion is well-defined (it is indeed a strategy) and that it preserves innocence and
well-bracketedness.


We now introduce the category of well-opened games:
\begin{dfn}[Category of well-opened games]
The category $\mathcal{C}$ of well-opened games is defined as follow:
\begin{enumerate}
\item The objects are the well-opened games,
\item a morphism $\sigma : A \rightarrow B$ is a strategy for the game $!A \multimap B$,
\item the identity map for $A$ is the copy-cat strategy on $!A \multimap A$ (which is well-defined for well-opened games).
It is called dereliction, noted
$\textsf{der}_A$ and defined formally by:
$$ \textsf{der}_A = \{ s \in P^{\textsf{even}}_{{!A} \multimap A} \ | \ \forall t \sqsubseteq^{\textsf{even}} s \ . \ t \upharpoonright {!A} = t \upharpoonright A \},$$
\item composition of morphisms $\sigma : {!A} \multimap B$ and $\tau : {!B} \multimap C$
noted $\sigma \fatsemi \tau : {!A} \multimap C$ is defined as $\sigma^\dagger;\tau$.
\end{enumerate}
\end{dfn}
$\mathcal{C}$ is a well-defined category and the three sub-categories
$\mathcal{C}_i$, $\mathcal{C}_b$, $\mathcal{C}_{ib}$ corresponding to sub-category
with innocent strategies, well-bracketed strategies and innocent and well-bracketed strategies respectively.


The category $\mathcal{C}$ has a terminal object $I$, for any two games $A$ and $B$ a product $A \& B$ and
an exponential $A \imp B$ defined to be $!A \multimap B$. The hom-sets $\mathcal{C}(A \& B,C)$ and
$\mathcal{C}(A,!B \multimap C)$ are isomorphic. Indeed:
\begin{eqnarray*}
\mathcal{C}(A\& B,C) &=& \mathcal{G}(!(A\& B),C) \\
&=& \mathcal{G}({!A}\otimes {!B},C) \\
&\cong& \mathcal{G}({!A}, {!B} \multimap C) \qquad  \mbox{($\mathcal{G}$ is a closed monoidal category)}\\
&=& \mathcal{C}(A, {!B} \multimap C)
\end{eqnarray*}
Hence $\mathcal{C}$ is a cartesian closed category. Moreover $\mathcal{C}_i$ and $\mathcal{C}_b$
are sub-cartesian closed caterogies of $\mathcal{C}$ and $\mathcal{C}_{ib}$ is as sub-cartesian closed category
of each of $\mathcal{C}$, $\mathcal{C}_i$ and $\mathcal{C}_b$.





\subsubsection{Order enrichment}

Strategies can be ordered using the inclusion ordering. Under this
ordering, the set of strategies on a given game $A$ is a pointed
directed complete partial order : the least upper bounds is the
union of two strategies and the least element is the empty strategy
$\{ \epsilon \}$.

Moreover all the operators on strategies that we have defined so far
(composition, tensor product, ...) are continuous. Hence the
category $\mathcal{C}$ and $\mathcal{G}$ are cpo-enriched.

This significant characteristic will prove to be extremely useful
when it comes to model programming languages with recursion such as
PCF.


\subsubsection{Intrinsic preoder}

We now define a pre-ordering on strategies. We assume that we are working in one of the categories
$\mathcal{C}$, $\mathcal{C}_i$, $\mathcal{C}_b$, $\mathcal{C}_{ib}$.

Let $\Sigma$ be the game with a single question $q$ and single answer $a$. There are only two strategies on $\Sigma$:
$\bot = \{ \epsilon \}$ and $\top = \{ \epsilon, q a \}$ which are both innocent and well-bracketed. These strategies are used
to test strategies: for any strategy $\sigma : {\bf 1} \rightarrow A$ and for any test strategy $\alpha : A \rightarrow \Sigma$ we say that $\sigma$
passes the test $\alpha$ if $\sigma \fatsemi \alpha = \top$.

The intrinsic preorder noted $\lesssim$ is then defined as follows:
for any strategy $\sigma,\tau$ on the game $A$, $\sigma \lesssim \tau$ if $\tau$ passes all the test passed by $\sigma$. Formally:
$$ \sigma \lesssim \tau \quad \iff \quad \forall \alpha : A \rightarrow \Sigma. \sigma \fatsemi \tau = \top \imp \tau \fatsemi \alpha = \top$$

One can check that the relation $\lesssim$ is indeed a preorder on the set of strategies of the considered category.
This preorder defines classes of equivalence: two strategies are in the same equivalence class if no test can distinguish them.
The quotiented category is written $\bf C/\lesssim$ where $\bf C$ ranges over $\{ \mathcal{C}_i, \mathcal{C}_i, \mathcal{C}_b, \mathcal{C}_{ib} \}$.

Later on we will state the full abstraction of the game semantics model of PCF. This result will
be proved in the quotiented category.

\subsection{Special case of arenas of order 2 at most}
In this section, we consider a restricted class of arena and prove a
property on the games played on these arenas.

The height of the arena is the length of the longest sequence of moves
$m_1 \ldots m_h$ in $M$ such that $m_1 \vdash m_2 \vdash \ldots \vdash m_h$.

The order of an arena $\langle M, \lambda, \vdash \rangle$ is defined to be
$h-2$ where $h$ is the height of the forest of trees $(M, \vdash)$.


\begin{lem}[Pointers are superfluous up to order 2]
Let $A$ be the arena of order at most 2. Let $s$ be a justified sequence of moves in the arena $A$ satisfying
 alternation, visibility, well-openedness and well-bracketing then
the pointers of the sequence $s$ can be reconstructed uniquely.
\end{lem}



\begin{proof}
We represent an arena graphically as a forest of trees. We choose to display the sub-trees of a given node
from left to right by decreasing order of the sub-arena order. This reordering is harmless since reordering children nodes
produces isomorphic arenas.

Let $A$ be an arena of order 2.
The justified sequence that we consider are well-opened therefore there is only one initial move in the sequence (the first move). Consequently
if $A$ is a forest arena (i.e. with multiple roots), the problem can be reduced to the case of single root arena just by replacing
the arena $A$ by the tree of the forest $A$ whose root is the first move of the justified sequence.
Therefore we assume that $A$ has a single root. $A$ has the following shape:
\begin{center}
\
  \pstree[levelsep=6ex]
    { \TR{$q$} }
    {
\SubTree{$T_1$} \SubTree[linestyle=none]{$\ldots$} \SubTree{$T_n$}
    \TR{$a_1$} \TR{$a_2$} \TR{\ldots} }
\end{center}

where each triangle $T_i$ represents an arena of order 0 or 1.


We write $I_k$, for $k=0$ or $1$, the set of indices $i$ such that the arena $T_i$ has order $k$:
$$I_k = \{ i \in 1.. n\ |\ \order{T_i} = k \}$$

Here is a graphic representation of the arenas $T_i$ for $i \in I_0$ and $T_j$ for $j \in I_1$:
\begin{center}
\
  \pstree[levelsep=6ex]
    {\TR{$q^i$}}
    { \TR{$a_1^i$} \TR{$a_2^i$} \TR{\ldots} }
\hspace{2cm}
  \pstree[levelsep=6ex]
    { \TR{$p^j$} }
    {
      \pstree[levelsep=6ex]
        { \TR{$q^j$} }
        { \TR{$a_1^j$} \TR{$a_2^j$} \TR{\ldots} }
      \TR{$b_1^j$} \TR{$b_2^j$} \TR{\ldots}
    }
\end{center}



For any justified sequence of moves $u$, we write $?(u)$ for the
subsequence of $u$ consisting of the questions in the sequence $u$
that are still pending at the end of the sequence.

Let $L$ be the following language $L = \{\ p^i q^i\ | \ i \in I_1
\}$. We consider the following cases:

\begin{center}
\begin{tabular}{c|c|l|l}
Case & $\lambda_{OP}(m)$ & $?(u) \in$ & condition \\ \hline
0 & O & $\{ \epsilon \}$ \\
A & P & $q$ \\
B & O & $q \cdot L^* \cdot p^i$     & $i \in I_1$ \\
C & P & $q \cdot L^* \cdot p^i q^i$ & $i \in I_1$ \\
D & O & $q \cdot L^* \cdot q^i$      & $i \in I_0$ \\
\end{tabular}
\end{center}

We use the notation $\hat{s}$ to denote a legal and well-bracketed
\emph{justified} sequence of moves and $s$ to denote the same
sequence of moves with pointers removed.

Note that the well-bracketing condition already tells us how to
uniquely recover the pointers for P answer moves: a P-answers points
to the last pending question having the same tag. However for O
answers, we will see that the visibility condition already ensures
the unique recoverability of the pointer and that the
well-bracketing condition is not needed.


We prove by induction on the sequence of moves $u$ that $?(u)$
corresponds to either case 0, A, B, C or D and that the pointers in
$u$ can be recovered uniquely.

\textbf{Base cases:}

If $u$ is the empty sequence $\epsilon$ then there is no pointer to
recover and it corresponds to case 0.

If $u$ is a singleton then it must be the initial question $q$ and
there is not pointer to recover. This corresponds to case A.

\textbf{Step case:}

Consider a legal well-bracketed justified sequence $\hat{s}$ where
$s = u \cdot m$ and $m \in M_A$. The induction hypothesis tells us
that the pointers of $u$ can be recovered (and therefore the P-view
or O-view at that point can be computed) and that $u$ corresponds to
one of the cases 0,A,B,C or D.

We proceed by case analysis on $u$:

\begin{description}

\item[case 0] This case cannot happen because $?(u) = \epsilon$ ($u$ is a complete play) implies that there cannot be any further move $m$.

Indeed the visibility condition implies that $m$ must point to a
P-question in the O-view at that point. But since $u$ is a complete
play, the O-view is $\oview{\hat{u}} = q a$ which does not contain
any P-question. Hence the move $m$ cannot be justified and is not
valid.


\item[case A] $?(u) = q$ and the last move $m$ is played by P.
    There are several cases:
    \begin{itemize}
    \item $m$ is an answer $a_k$ (to the initial question
    $q$) for some $k$, then $m$ points to $q$:

    $\hat{s} = \justseq{ q & \ldots & m \pointto{ll}}$

    and $?(s) = \epsilon$ therefore $s$ correspond to the case 0 (complete play).

    \item $m = q^i$ where $q^i$ is an order 0 question ($i \in I_0$).
    Then $q^i$ points to the initial question $q$ and $s$ falls into category D.

    \item $m = p^i$, a first order question, then $p^i$ points to $q$,

    $?(s)= q p^i$ and it is O's turn after $s$ therefore $s$ falls into category B.

    \end{itemize}


\item[case B] $?(u) \in q \cdot L^* \cdot p^i$ where $i \in I_1$ and O plays the move $m$.

We now analyse the different possible O-moves:
\begin{itemize}
\item Suppose that O gives the (tagged) answer $b^j$ for some $j \in I_1$ then
the visibility condition constraints it to point to a question in
the O-view at that point.

We remark that the last move in $\hat{u}$ must be $p^i$. Indeed,
suppose that there is a move $x \in M_A$ such that $\hat{u} =
\justseq{q & \ldots & p^i\ x \pointto{ll}}$ then by visibility, the
O-move $x$ should points to a move in the O-view a that point. The
O-view is $q p^i$, therefore $x$ can only points to $p^i$. But then,
$p^i$ is not a pending question in $s$ which is a contradiction.


Therefore $\oview{\hat{u}} = \oview{ \justseq{ q & \ldots & p^i
\pointto{ll}} } = q p^i$.

Hence $b^j$ can only point to $p^i$ (and therefore $i=j$).

We then have $?(s) = ?(u \cdot b^i) \in  q \cdot L^*$ which is
covered by case A and C.

\item The only other possible O-move is $q^i$ which, again by the visibility condition, points necessarily
to the previous move $p^i$. We then have $?(s) = ?(u \cdot q^i) \in
q \cdot L^* \cdot p^i q^i$. This falls into category C.

\end{itemize}

\item[case C] $?(u) \in q \cdot L^* \cdot p^i q^i$ where $i \in I_1$ and the move $m$ is played by $P$.

Suppose $m$ is an answer, then the well-bracketing condition imposes
to answer to $q^i$ first. The move $m$ is therefore an integer $a^i$
pointing to $q^i$. We then have $?(s) = ?(u \cdot a^i) \in  q \cdot
L^* \cdot p^i$. This correspond to case B.


Suppose $m$ is a question then there are two cases:
\begin{itemize}
\item $m = q^j$ with $j \in I_0$, the pointer goes to the initial question $q$ and $s$ falls into category D.
\item $m = p^j$ with $j \in I_1$, the pointer goes to the initial question $q$ and $s$ falls into category B.
\end{itemize}

\item[case D] $?(u) \in q \cdot L^* \cdot q^i$ where $i \in I_0$ and the move $m$ is played by $O$.

    The same argument as in case B holds. However there is now another possible move:
    the answer $m = a^i_k$ for some $k$.  This moves can only points to
    $q^i$ (this is the only pending question tagged by $i \in I_0$).

    Then $?(\hat{s}) = ?(\hat{u}\cdot a^i_k) = ?(\justseq{ q & \ldots & q^i \pointto{ll} & \ldots & a^i_k \pointto{ll}}) \in q \cdot L^* $ therefore $s$ falls either into category A or C.

\end{description}

This completes the induction.
\end{proof}


\subsection{Pointers are necessary}
\label{subsec:pointer_necessary}

Up to order 2, the semantics of PCF terms is entirely defined by
pointer-less strategies. In other words, the pointers can be
uniquely reconstructed from any non justified sequence of moves
satisfying the visibility and well-bracketing condition.

At level 3 however, pointers cannot be omitted in general. Here is
an example taken from \cite{abramsky:game-semantics-tutorial}
illustrating this. Consider the following two terms, called the
Kierstead terms, of type $((\nat \typar \nat) \typar \nat) \typar
\nat$:

$$M_1 = \lambda f . f (\lambda x . f (\lambda y .y ))$$
$$M_2 = \lambda f . f (\lambda x . f (\lambda y .x ))$$

We assign tags to the types in order to identify in which arena the
questions are asked: $((\nat^1 \typar \nat^2) \typar \nat^3) \typar
\nat^4$. Consider now the following pointer-less sequence of moves
$s = q^4 q^3 q^2 q^3 q^2 q^1$. It is possible to retrieve the
pointers of the first five moves but there is an ambiguity for the
last move: does it point to the first or second occurrence of $q^2$
in the sequence $s$?

Note that the visibility condition does not eliminate the ambiguity,
since the two occurrences of $q^2$ both appear in the P-view at that
point (after recovering the pointers of $s$ up to the second last
move we get:
$$s = \rnode{q4}{q}^4
\rnode{q3}{q}^3
\rnode{q2}{q}^2
\rnode{q3b}{q}^3
\rnode{q2b}{q}^2
\rnode{q1}{q}^1
\bkptrc{q3}{q4}
\bkptrc{q2}{q3}
\bkptrc[ncurv=0.6]{q3b}{q4}
\bkptrc{q2b}{q3b}$$

 therefore the P-view of $s$ is $s$ itself.)

In fact these two different possibilities correspond to two
different strategies. Suppose that the link goes to the first
occurrence of $q^2$ then it means that the proponent is requesting
the value of the variable $x$ bound in the subterm $\lambda x . f (
\lambda y. ... )$. If P needs to know the value of $x$, this is
because P is in fact following the strategy of the subterm $\lambda
y . x$. And the entire play is part of the strategy $\sem{M_2}$.

Similarly, if the link points to the second occurrence of $q^2$ then
the play belongs to the strategy $\sem{M_1}$.

\section{The fully abstract game model for PCF}

In this section we introduce the functional languages PCF. We then
describe the game model introduced in \cite{abramsky94full} and
finally we will state the full abstraction result.

\subsection{The syntax of PCF}
PCF is a simply-type $\lambda$-calculus with the following
additions: integer constants  (of ground type), first-order
arithmetic operators, if-then-else branching, and the recursion
combinator $Y_A : (A\rightarrow A)\rightarrow A$ for any type $A$.

The types of PCF are given by the following grammar:
$$ T ::= \texttt{exp}\ |\ T \rightarrow T$$

and the structure of terms is given by:
\begin{eqnarray*}
 M ::= x\ |\ \lambda x :A . M \ |\ M M \ |\ \\
\ |\ n \ |\ \texttt{succ } M \ |\  \texttt{pred } M \\
\ |\ \texttt{cond } M M M \ |\ \texttt{Y}_A\ M
\end{eqnarray*}

where $x$ ranges over a set of countably many variables and $n$
ranges over the set of natural numbers.

Terms are generated according to the formation rules given in table
\ref{tab:pcf_formrules} where the judgement is of the form $ \Gamma  \vdash M : A$.

\begin{table}[htbp]
$$ (var) \rulef{}{x_1:A_1, x_2:A_2, \ldots x_n : A_n  \vdash x_i : A_i}\ i \in 1..n$$
$$ (app) \rulef{\Gamma \vdash M : A\rightarrow B \qquad \Gamma \vdash N:A}{\Gamma \vdash M\ N : B}
\qquad (abs) \rulef{\Gamma, x:A \vdash M : B}{\Gamma \vdash \lambda x :A . M : A\rightarrow B}$$

$$ (const) \rulef{}{\Gamma \vdash n :\texttt{exp}}
\qquad (succ) \rulef{\Gamma \vdash M:\texttt{exp} }{\Gamma \vdash \texttt{succ}\ M:\texttt{exp}}
\qquad (pred) \rulef{\Gamma \vdash M:\texttt{exp} }{\Gamma \vdash \texttt{pred}\ M:\texttt{exp}}$$

$$
(cond) \rulef{\Gamma \vdash M : exp \qquad \Gamma \vdash N_1 : exp \qquad \Gamma \vdash N_2 : exp }{\Gamma \vdash \texttt{cond}\ M\ N_1\ N_2}
\qquad  (rec) \rulef{\Gamma \vdash M : A\rightarrow A }{ \Gamma \vdash Y_A M : A}$$

\caption{Formation rules for PCF terms}
\label{tab:pcf_formrules}
\end{table}

\subsection{Operational semantics of PCF}

We give the big-step operational semantics of PCF. The notation $M \eval V$ means
that the closed term $M$ evaluates to the canonical form $V$. The canonical forms are given by the following
grammar:
$$V ::= n\ |\ \lambda x. M$$
In other word, a canonical form is either a number or a function.

The full operational semantics is given in table
\ref{tab:bigstep_pcf}. The evaluation rules are defined for closed
terms only therefore the context $\Gamma$ is not present in the
rules. We write $M \eval$ if $M \eval V$ for some value $V$.

\begin{table}[htbp]
$$\rulef{}{V \eval V} \quad \mbox{ provided that $V$ is in canonical form.} $$

$$ \rulef{M \eval \lambda x. M' \quad M'\subst{x}{N} \eval V}{M N \eval V}$$

$$\rulef{M \eval n}{\texttt{succ}\ M \eval n+1}
\qquad \rulef{M \eval n+1}{\texttt{pred}\ M \eval n}
\qquad \rulef{M \eval 0}{\texttt{pred}\ M \eval 0}$$

$$\rulef{M \eval 0 \quad N_1 \eval V}{\texttt{cond}\ M N_1 N_2  \eval V}
\qquad
 \rulef{M \eval n+1 \quad N_2 \eval V}{\texttt{cond}\ M N_1 N_2  \eval V}$$

$$\rulef{M (\mathrm{Y} M) \eval V }{\texttt{Y} M \eval V}$$
\label{tab:bigstep_pcf}
\caption{Big-step operational semantics of PCF}
\end{table}



\subsection{Game model of PCF}
\label{subsec:pcfgamemodel}

As we have seen in section \ref{sec:catgames}, games and strategies
form a cartesian closed category, therefore games can model the
simply-typed $\lambda$-calculus. We are now about to make this
connection concrete by explicitly giving the strategy corresponding
to a given $\lambda$-term. We will then extend the game model to PCF
and IA.

\subsubsection{Simply-typed $\lambda$-calculus fragment}

In the games that we are considering, the Opponent represents the
environment and the Proponent represents the lambda term. Opponent
opens the game by asking a question such as ``What is the output of
the function?'', the proponent then may then ask further information
such that ``What is the input of the function?'' O can then provide
$P$ with an answer (the value of the input) or can pursue with
another question. The dialog goes on until O gets the answer to his
initial question.

O represents the environment, he is responsible for proving input
values while P plays from the term's point of view: he is
responsible for performing the computation and returning the output
to O. P plays according to the strategy that is associated to the
$\lambda$-term being modeled.

We recall that in the cartesian closed category $\mathcal{C}$, the
objects are the arenas and the morphisms are the strategies. Given a
simple type $A$, we will model it as an arena $\sem{A}$. A context
$\Gamma = x_1 :A_1, \ldots x_n:A_n$ will be mapped to the arena
$\sem{\Gamma} = \sem{A_1} \times \ldots \times \sem{A_n}$ and a term
$\Gamma \vdash M : A$ will be modeled by a strategy on the arena
$\sem{\Gamma} \rightarrow \sem{A}$. Since $\mathcal{C}$ is cartesian
closed, there is is a terminal object $\textbf{1}$ (the empty arena)
that models the empty context ($\sem{\Gamma} = \textbf{1}$).


Let $\omega$ denotes the set of natural numbers. Consider the
following flat arena over $\omega$:
$$  \pstree[levelsep=6ex]
    {\TR[name=R]{q}}
    { \TR{1} \TR{2} \TR{\ldots}
    }
$$
Then the base type \texttt{exp} is interpreted by the flat game
$\nat$ over the previous arena where the set of valid position is:
$$P_N = \{ \epsilon, q \} \union \{ qn \ | \ n \in \omega \}$$


In this arena, there is only one question: the initial O-question, P
can then answer by playing a natural number $i \in \omega$. There
are only two kinds strategy on this arena:
\begin{itemize}
\item the empty strategy where P never answer the initial question. This corresponds to a non terminating computation;
\item the strategies where P answers by playing a number $n$. This models the numerical constants of the language.
\end{itemize}

Given the interpretation of base types, we define the interpretation
of $A\rightarrow B$ by induction:
$$\sem{A \rightarrow B} = \sem{A} \Rightarrow \sem{B}$$

where the operator $\Rightarrow$ denotes the arena construction $!A
\multimap B$, the exponential object of the cartesian closed
category $\mathcal{C}$.



Variables are interpreted by projection:
$$\sem{x_1 : A_1, \ldots, x_n:A_n \vdash x_i : A_i} = \pi_i : \sem{A_i} \times \ldots \times \sem{A_i} \times \ldots \times \sem{A_n} \rightarrow  \sem{A_i}$$

The abstraction $\Gamma \vdash \lambda x :A.M : A \rightarrow B$ is
modeled by a strategy on the arena $\sem{\Gamma} \rightarrow
(\sem{A}\Rightarrow\sem{B})$. This strategy is obtain by using the
currying operator of the cartesian closed category:
$$\sem{\Gamma \vdash \lambda x :A.M : A \rightarrow B} = \Lambda( \sem{\Gamma, x :A \vdash M : B})$$

The application $\Gamma \vdash M N$ is modeled using the evaluation
map $ev_{A,B} : (A\Rightarrow B)\times A \rightarrow B$:

$$\sem{\Gamma \vdash M N} = \langle \sem{\Gamma \vdash M, \Gamma \vdash N} \rangle \fatsemi ev_{A,B}$$


\subsubsection{PCF fragment}

We now show how to model PCF constructs in the game semantics
setting. In the following, the sub-arena of a game are tagged in
order to distinguish identical arenas present in different
components of the game. Moves are also tagged in the exponent in
order to identify the sub-arena in which moves are played. We will
omit the pointers in the play when there is no ambiguity.

The successor arithmetic operator is modeled by the following
strategy on the arena $\nat^1 \Rightarrow \nat^0$:
$$\sem{\texttt{succ}} = \{q^0 \cdot q^1 \cdot n^1 \cdot (n+1)^0\ |\ n \in \nat \}$$

The predecessor arithmetic operator is denoted by the strategy
$$\sem{\texttt{pred}} = \{q^0 \cdot q^1 \cdot n^1 \cdot (n-1)^0\ |\ n >0 \} \union \{ q^0 \cdot q^1 \cdot 0^1 \cdot 0^0 \} $$

Then given a term $\Gamma \vdash \texttt{succ }M : \texttt{exp}$ we
define:
$$\sem{\Gamma \vdash \texttt{succ } M : \texttt{exp}} = \sem{\Gamma \vdash M} \fatsemi \sem{\texttt{succ}} $$
$$\sem{\Gamma \vdash \texttt{pred } M : \texttt{exp}} = \sem{\Gamma \vdash M} \fatsemi \sem{\texttt{pred}} $$


The conditional operator is denoted by the following strategy on the
arena $\nat^3 \times \nat^2 \times \nat ^1 \Rightarrow \nat^0$:
$$\sem{\texttt{cond}} =
    \{ q^0 \cdot q^3 \cdot 0 \cdot q^2 \cdot n^2 \cdot n^0 \ | \ n \in \nat \}
    \union
    \{ q^0 \cdot q^3 \cdot m \cdot q^2 \cdot n^2 \cdot n^0 \ | \ m >0, n \in \nat \}
    $$


Given a term $\Gamma \vdash \texttt{cond}\ M\ N_1\ N_2$ we define:
$$\sem{\Gamma \vdash \texttt{cond}\ M\ N_1\ N_2} =
\langle \sem{\Gamma \vdash M}, \sem{\Gamma \vdash N_1}, \sem{\Gamma
\vdash N_2} \rangle \fatsemi \sem{\texttt{cond}}$$


The interpretation of the \texttt{Y} combinator is a bit more
complicated.

Consider the term $\Gamma \vdash M : A \rightarrow A$, its semantics
$f$ is a strategy on $\sem{\Gamma} \times \sem{A} \rightarrow
\sem{A}$. We define the chain $g_n$ of strategies on the arena
$\sem{\Gamma} \rightarrow \sem{A}$ as follows:
\begin{eqnarray*}
g_0 &=& \perp \\
g_{n+1} &=&  F(g_n) = \langle id_{\sem{\Gamma}}, g_n\rangle \fatsemi f
\end{eqnarray*}

where $\perp$ denotes the empty strategy $\{ \epsilon \}$.

It is easy to see that the $g_n$ forms a chain. We define
$\sem{\texttt{Y } M}$ to be the least upper bound of the chain $g_n$
(i.e. the  least fixed point of $F$). Its existence is guaranteed by
the fact that the category of games is cpo-enriched.

Since all the strategies that we have given are innocent and
well-bracketed, the game model of PCF can be interpreted in any of
the four categories $\mathcal{C}$, $\mathcal{C}_i$, $\mathcal{C}_b$,
$\mathcal{C}_{ib}$.



\subsection{Full-abstraction of PCF}
In this section we state the full abstraction result proved in
\cite{abramsky94full} and \cite{hylandong_pcf}.


\subsubsection{Observational preorder}

A context noted $C[-]$ is a term containing a hole denoted by $-$.
If $C[-]$ is a context then $C[A]$ denotes the term obtained after
replacing the hole by the term $A$.

If $M$ is a PCF term then we write $C[M]$ to denote the term
obtained after replacing the hole by the term $M$. $C[M]$ is
well-formed provided that $M$ has the appropriate type. Remark: this
capture permitting substitution must be distinguished from the
capture-free substitution which is noted $M[N/x]$ for any two terms
$M$ and $N$.


\begin{dfn}[Observational preorder]
We define the relation on terms $\obspre$ as follows: suppose $M$
and $N$ are two closed terms of the same type then:
\begin{eqnarray*}
M \obspre N &\iff& \parbox{10cm}{for all context $C[-]$ such that
                $C[M]$ and $C[N]$ are well-formed closed term of type \texttt{exp},
                    $C[M] \eval$ implies $C[N] \eval$}
\end{eqnarray*}
Observational equivalence is defined as the reflexive closure of
$\obspre$ noted $\obseq$.
\end{dfn}

Said informally, two programs are observationally equivalent if then
can be safely interchanged in any program context.

\subsubsection{Soundness and adequacy}
A model of a programming language is said to be \emph{sound} or
\emph{inequationally sound} if whenever the denotation of two
programs are equal then the two programs are observationally
equivalent, or more formally if for any closed terms $M$ and $N$ of
the same type:
$$ \sem{M} \subseteq \sem{N} \imp M \obspre N.$$

In a way, soundness is the minimum one can require for a model of
programming language: it guarantees that we can reason about the
program by manipulating the object of the denotational model.

It can be shown that the game model of PCF is sound for evaluation
and computationally adequate. These two properties imply the
soundness of the game model:

We said that the evaluation relation $\eval$ is sound if the
denotation is preserved by evaluation:
\begin{lem}[Soundness of evaluation]
\label{lem:evalsoundness}
 Let $M$ be a PCF term then
$$M \eval V \quad \imp \quad \sem{M} = \sem{V}.$$
\end{lem}

\begin{dfn}[Computable terms] \
\begin{itemize}
\item A closed term $\vdash M$ of base type is computable if $\sem{M} \neq \bot$
implies $M \eval$.
\item A higher-order closed term $\vdash M : A\rightarrow B$ is computable if $M N$ is computable for any computable closed term $\vdash  N:A$.
\item An open term $x_1 : A_1, \ldots, x_n : A_n \vdash M : A\rightarrow B$ is computable if $\vdash M [N_1/x_1, \ldots N_n/x_n]$ is computable
for all computable closed terms $N_1:A_1, \ldots, N_n:A_n$.
\end{itemize}
\end{dfn}

A model is \emph{computationally adequate} if all
terms are computable.
\begin{lem}[Computational adequacy]
\label{lem:computadequacy}
The game model of PCF is
computationally adequate.
\end{lem}
We refer the reader to \cite{abramsky:game-semantics-tutorial} for
the proofs.

Inequational soundness follows from the last two lemmas:
\begin{prop}[Inequational soundness]
\label{prop:ineqsoundness} Let $M$ and $N$ be two closed terms then
$$\sem{M} \subseteq \sem{N} \implies  M \obspre N $$
\end{prop}
\begin{proof}
  Suppose that $\sem{M} \subseteq \sem{N}$ and $C[M] \eval$ for some context $C[-]$. Then by compositionality of game semantics we also have
  $C[\sem{M}] \subseteq C[\sem{N}]$.
  Lemma \ref{lem:evalsoundness} gives $\sem{C[M]} \neq \bot$, therefore $\sem{C[N]} \neq \bot$.
  Lemma \ref{lem:computadequacy} then implies that $C[N] \eval$.
  Hence $M \obspre N$.
\end{proof}

\subsubsection{Definability}

We will now consider only strategies that are innocent and
well-bracketed (i.e. we work in the category $\mathcal{C}_{ib}$).

The definability result says that every compact element of the model
is the denotation of some term.
The compact morphisms of the category $\mathcal{C}_{ib}$ are those
with finite view-function.

The economical syntax of PCF prevents us from stating this
result directly: we need to consider an extension of PCF with some additional
constants. Indeed, there are strategies that are not the denotation of any term
in PCF, for instance the ternary conditional strategy : this
strategy denotes the computation that tests the value of its first
parameter, if its equal to zero or one then it returns the value of
the second or third parameter respectively, otherwise it returns the
value of the fourth parameter. This strategy is illustrated by the
following diagram:
$$
\begin{array}{ccccccccc}
!\bf N & \otimes & !\bf N & \otimes & !\bf N & \otimes & !\bf N & \multimap & !\bf N \\
&&&&&&&&q \\
q \\
0 \\
&& q \\
&& n \\
&&&&&&&&n \\
\hline
&&&&&&&&q \\
q \\
1 \\
&&&& q \\
&&&& n \\
&&&&&&&&n \\
\hline
&&&&&&&&q \\
q \\
m>1 \\
&&&&&& q \\
&&&&&& n \\
&&&&&&&&n \\
\end{array}
$$

It is possible to simulate this computation in PCF using the conditional operator, for instance the following term is a potential candidate:
$$ T = \texttt{cond}\ M\  N_1 (\texttt{cond}\  (\texttt{pred } M)\  N_2\  N_3)$$

Unfortunately the game semantics of $T$ is not given by the strategy that we have just defined, it is instead the following one:
$$
\begin{array}{ccccccccc}
!\bf N & \otimes & !\bf N & \otimes & !\bf N & \otimes & !\bf N & \multimap & !\bf N \\
&&&&&&&&q \\
q \\
0 \\
&& q \\
&& n \\
&&&&&&&&n \\
\hline
&&&&&&&&q \\
q \\
1 \\
q \\
0 \\
&&&& q \\
&&&& n \\
&&&&&&&&n \\
\hline
&&&&&&&&q \\
q \\
m>1 \\
q \\
m-1>0 \\
&&&&&& q \\
&&&&&& n \\
&&&&&&&&n \\
\end{array}
$$

To make up for this deficiency we add a family of terms to PCF: the $k$-ary conditionals:
$$ \texttt{case}_k\ N\ N_1\ N_2\ \ldots\ N_k$$
with the desired operational semantics:
$$ \rulef{M \eval i \quad N_{i+1} \eval V}{\texttt{case}_k\ N\ N_1\ N_2\ \ldots\ N_k\ \eval V}\ i \in \{0, \ldots,k-1\}.$$
The denotation of this term is given by the first strategy illustrated above.
The extended language is called PCF'.

We can now prove the definability result:
\begin{prop}[Definability]
\label{prop:definability} Let $A$ be a PCF type and $\sigma$ be a compact innocent and well-bracketed
strategy on $A$. There exists a PCF' term $M$ such that $\sem{M} = \sigma$.
\end{prop}

Note that definability is proved for PCF' and not for PCF.
Nevertheless, PCF' is a conservative extension of PCF: if $M$ and $N$ are terms such that for any PCF-context $C[-]$,
$C[M] \eval \imp C[N] \eval$ then the same is true for any PCF'-context. This is because $\texttt{case}_k$ constructs can be ``simulated''
in PCF, for instance $\texttt{case}_3$ can be replaced by the PCF term $T$ which shares the same operational semantics.

This observation will allow us to use definability in PCF' to
prove the full-abstraction of PCF.


\subsubsection{Full abstraction}

Full abstraction of PCF cannot be stated directly in the category $\mathcal{C}_{ib}$. Instead we need to consider the quotiented category
$\mathcal{C}_{ib}/\lesssim_{ib}$.

First we need to show that $\mathcal{C}_{ib}/\lesssim_{ib}$ is a model of PCF.
$\mathcal{C}_{ib}/\lesssim_{ib}$ is a posset-enriched cartesian closed category. The game semantics of the basic types and constants of PCF
can be transposed from $\mathcal{C}_{ib}$ to $\mathcal{C}_{ib}/\lesssim_{ib}$. Unfortunately it is not know whether $\mathcal{C}_{ib}/\lesssim_{ib}$
is enriched over the category of CPOs. However it can be proved that it is a rational category \citep{abramsky94full}
and this suffices to ensure that $\mathcal{C}_{ib}/\lesssim_{ib}$ is indeed a model of PCF.

The full abstraction of the game model then follows from
proposition \ref{prop:ineqsoundness} and \ref{prop:definability}:
\begin{thm}[Full abstraction]
Let $M$ and $N$ be two closed IA-terms.
$$\sem{M} \precsim_{ib} \sem{N} \ \iff \ M \obspre N$$
where $\precsim_{ib}$ denotes the intrinsic preorder of the category $\mathcal{C}_{ib}$.
\end{thm}

\section{The fully abstract game model for Idealized Algol (\ialgol)}

We now extend the work of the previous section to the language
\ialgol, an imperative extension of PCF. We start by giving the
syntax and operational semantics of the language, we then describe
the game model which was introduced in \cite{abramsky99full}.
Finally we will state the full abstraction result for the game
model.

\subsection{The syntax of \ialgol}
IA is an extension of PCF introduced by J.C. Reynold in
\cite{Reynolds81}. It adds imperative features such as local
variables and sequential composition.

On top of \texttt{exp}, PCF has the following two new types:
\texttt{com} for commands and \texttt{var} for variables. There is a
constant \texttt{skip} of type \texttt{com} which corresponds to the
command that do nothing.

Commands can be composed using the
sequential composition operator $\texttt{seq}_A$: suppose that $M$ and $N$ are of type
\texttt{com} and $A$ respectively then they can be composed to form the term
$S = \texttt{seq}_A M N : \texttt{com}$. $S$ denotes program that executes $M$ until it terminates and then
behave like $N:A$. If $A = \texttt{exp}$ then the expression is allowed to have a side-effect and $S$ returns the expression computed by $N$, if
$A = \texttt{com}$ then the command $N$ is executed after $M$.
We say that the language has \emph{active expressions} to indicate the presence of the
sequential operator $\texttt{seq}_{exp}$ in the language.


Local variable are
declared using the \texttt{new} operator, variable content is altered
using \texttt{assign} and retrieved using \texttt{deref}.

In addition IA has the constant \texttt{mkvar} that can be used to
create a particular kind of variables. The \texttt{mkvar} operator
works in an object oriented fashion: it takes two arguments, the
first one is a function (called the acceptor) that affects a value
to the variable and the second argument is an expression that
returns a value from the variable. This mechanism is very much like
the ``set/get'' object programming paradigm used by C++ programmers.

Variables created with \texttt{mkvar} are less constraint than the
variables created with \texttt{new}. Indeed, variables created with
\texttt{new} act like memory cells, they obey the following rule: the value read
from the variable is always the last value that has been assigned to
it. This rule does not apply to variables created with
\texttt{mkvar}. For instance the variable:
$$\texttt{mkvar}\ (\lambda v.\texttt{skip})\ 0$$
will always return $0$ even if another number has been assigned it.


One may think that this addition to the language is artificial,
however the full abstraction result of the game model of IA relies
upon this addition. At present, it is still an open problem to find
a fully abstract model of IA deprived of \texttt{mkvar}.

Judgement are of the form $\Gamma \vdash M : A$.
If the judgement $\Gamma = \emptyset$ we say that $M$ is a closed term.
The set of additional formations rules completing those of PCF are
given in table \ref{tab:ia_formrules}.

\begin{table}[htbp]
$$ \rulef{\Gamma \vdash M : \texttt{com} \quad \Gamma \vdash N :A}
    {\Gamma \vdash \texttt{seq}_A \ M\ N\ : A} \quad A \in \{ \texttt{com}, \texttt{exp}\}$$

$$ \rulef{\Gamma \vdash M : \texttt{var} \quad \Gamma \vdash N : \texttt{exp}}
    {\Gamma \vdash \texttt{assign}\ M\ N\ : \texttt{com}}
\qquad
 \rulef{\Gamma \vdash M : \texttt{var}}
    {\Gamma \vdash \texttt{deref}\ M\ : \texttt{exp}}$$

$$ \rulef{\Gamma, x : \texttt{var} \vdash M : A}
    {\Gamma \vdash \texttt{new } x \texttt{ in } M} \quad A \in \{ \texttt{com}, \texttt{exp}\}$$

$$ \rulef{\Gamma \vdash M_1 : \texttt{exp} \rightarrow \texttt{com} \quad \Gamma \vdash M_2 : \texttt{exp}}
    {\Gamma \vdash \texttt{mkvar } M_1\ M_2\ : \texttt{var}}$$

\caption{Formation rules for IA terms}
\label{tab:ia_formrules}
\end{table}


\subsection{Operational semantics of \ialgol}

The operational semantics of IA is given in a slightly different form compared to PCF.
Instead of giving the semantics for closed terms we consider terms
whose free variables are all of type \texttt{var}. Terms are
``closed'' by mean of stores. A store is a function mapping free
variables of type \texttt{var} to natural numbers. Suppose $\Gamma$
is a context containing only variables of type \texttt{var}, then we
say that $\Gamma$ is a \texttt{var}-context. A store with domain
$\Gamma$ is called a $\Gamma$-store. The notation $s\ |\ x \mapsto
n$ refers to the store that maps $x$ to $n$ and otherwise maps
variables according to the store $s$.

%%%% The following is poorly written:
%
%In PCF, the evaluation rules were given for closed terms only.
%Suppose that we proceed the same way for IA and consider the
%evaluation rule for the $\texttt{new}$ construct: the conclusion is
%$\texttt{new } x:=0 \texttt{ in } M$ and the premise is an
%evaluation for a certain term constructed from $M$, more precisely
%the term $M$ where \emph{some} occurrences of $x$ are replaced by
%the value $0$. Because of the presence of the \texttt{assign}
%operator, we cannot simply replace all the occurrences of $x$ in $M$
%(the required substitution is  more complicated than the
%substitution used for beta-reduction).


The canonical forms for IA are given by the grammar:
$$ V ::= \texttt{skip}\ |\ n\ |\ \lambda x. M\ |\ x\ |\  \texttt{mkvar}\ M\ N$$

where $n \in \nat$ and $x: \texttt{var}$.


In \ialgol, a program is a term together with a $\Gamma$-store such
that $\Gamma \vdash M : A$. The evaluation semantics is expressed by
the judgment form:
$$s,M \eval s', V$$
where $s$ and $s'$ are $\Gamma$-stores, $V$ is a canonical form and $\Gamma \vdash V : A$.

The operational semantics for IA is given by the rule of PCF (table \ref{tab:bigstep_pcf})
together with the rules of table \ref{tab:bigstep_ia} where the following abbreviation is used:
$$ \rulef{M_1 \eval V_1 \quad M_2 \eval V_2}{M \eval V} \qquad \mbox{for} \qquad
  \rulef{s,M_1 \eval s',V_1 \quad s', M_2 \eval s'',V_2 }{s,M \eval s'',V}
$$


\begin{table}[htbp]
$$\mbox{\textbf{Sequencing }}
    \rulef{M \eval \iaskip \quad N \eval V}{\texttt{seq } M\ N \eval V}
$$

$$\mbox{\textbf{Variables }}
    \rulef{s,N \eval s',n \quad s',M \eval s'',x}{s, \iaassign\ M\ N \eval (s''\ |\ x \mapsto n),\iaskip}
\qquad
    \rulef{s,M \eval s',x }{s, \iaderef\ M \eval s',s'(x)}$$

$$\mbox{\texttt{\textbf{mkvar}}}
    \rulef{N \eval n \quad M \eval \texttt{mkvar}\ M_1\ M_2 \quad M_1\ n \eval \iaskip}
    {\iaassign\ M\ N \eval \iaskip}
\qquad
    \rulef{N \eval \texttt{mkvar } M_1\ M_2 \quad M_2\ \eval n}
    {\iaderef\ M \eval n}
$$

$$\mbox{\textbf{Block}}
    \rulef{(s\ |\ x \mapsto 0),M \eval (s'\ |\ x \mapsto n),V }
    {s, \texttt{new } x \texttt{ in } M \eval s',V}
$$

\label{tab:bigstep_ia}
\caption{Big-step operational semantics of IA}
\end{table}


\subsection{Game model of \ialgol}

All the strategies used to model PCF are well-bracketed and
innocent. On the other hand, to obtain a model of IA we need to
introduce strategies that are not innocent.
This is necessary to model memory cell variable created with the \texttt{new} operator.
The intuition is that a cell needs to
remember what was the last value written in it in order to be able
to return it when it is read, and this can only be done by looking
at the whole history of moves, not only those present in the P-view.

Hence we now consider the categories $\mathcal{C}$ and $\mathcal{C}_b$.

\subsubsection{Base types}

The type \texttt{com} is modeled by the flat game with a single initial question \texttt{run} and a single answer
\texttt{done}. The idea is that O can request the execution of a command by playing \texttt{run}, P then execute the command
and if it terminates acknowledge it by playing \texttt{done}.

The variable type \texttt{var} is modeled by the game $\mathtt{com^{\bf N} \times exp}$ illustrated below:
\begin{center}
\begin{pspicture}(10cm,1.7cm)
$\rput[b]{0}(3cm,0){
\pstree[treemode=U,levelsep=8ex,nodesep=2pt]
    {\TR[name=R]{\mathtt{ok}}}
    { \TR{\mathtt{write}_0} \TR{\mathtt{write}_1} \TR{\mathtt{write}_2} \TR{\ldots}
    }
}
\rput[b]{0}(8cm,0){
\pstree[levelsep=8ex,nodesep=2pt]
    { \TR[name=R]{\mathtt{read}} }
    { \TR{0} \TR{1} \TR{2} \TR{\ldots} }
    }$
\end{pspicture}
\end{center}

\subsubsection{Constants}

\texttt{skip} is interpreted by the strategy $\{ \epsilon, \iarun \cdot \iadone \}$.
The sequential composition $\mathtt{seq_{exp}}$ is interpreted by the following strategy:
$$
\begin{array}{ccccc}
!\mathtt{com} & \otimes & ! \mathtt{exp} & \multimap & \mathtt{exp}\\
&&&&q\\
\iarun\\
\iadone\\
&&q\\
&&n\\
&&&&n
\end{array}
$$

Assignment \iaassign and dereferencing \iaderef are denoted  by the
following strategies (left and right respectively):
$$
\begin{array}{ccccc}
!\mathtt{var} & \otimes & ! \mathtt{exp} & \multimap & \mathtt{com}\\
&&&&q\\
&&q\\
&&n\\
\iawrite_n\\
\iaok\\
&&&&\iadone
\end{array}
\hspace{3cm}
\begin{array}{ccccc}
!\mathtt{var} & \multimap & \mathtt{exp}\\
&&q\\
\iaread\\
n\\
&&n
\end{array}
$$

\iamkvar is modeled by the paired strategy $\langle \iamkvar_{acc} , \iamkvar_{exp}
\rangle$ where $\iamkvar_{acc}$ and $\iamkvar_{exp}$ are the following strategies:
$$
\begin{array}{ccccccc}
(\mathtt{!exp} & \multimap & \mathtt{com}) & \otimes & !\mathtt{exp} & \multimap & \mathtt{com}^\omega\\
&&&&&&\iawrite_n\\
&&\iarun\\
q\\
n\\
&&\iadone \\
&&&&&&\iaok
\end{array}
\hspace{2cm}
\begin{array}{ccccccc}
(\mathtt{!exp} & \multimap & \mathtt{com}) & \otimes & !\mathtt{exp} & \multimap & \mathtt{exp}\\
&&&&&&\iaread\\
&&&&\iaread\\
&&&&n\\
&&&&&&n
\end{array}
$$


The strategies used until now are all innocent. In order to model the \ianew operator, we need to introduce non-innocent strategies, sometimes called
\emph{knowing strategies}. We define the knowing well-bracketed strategy $cell : I \multimap !\mathtt{var}$ that models a storage cell: it responds to \iawrite\
with \iaok\ and responds
to \iaread\ with the last value written or $0$ if no value has yet been written.

Consider the term $\Gamma,x:\mathtt{var} \vdash M : A$ modeled by $\sem{M}$ then the term
 $\Gamma \vdash \ianew\ x \texttt{ in } M : A$  will be modeled by the strategy $(id_{\sem{\Gamma}} \otimes cell) \fatsemi\fatsemi \sem{M}$ on the game
 $!\Gamma \multimap \iacom$.

\subsection{Full abstraction of \ialgol}

We now state the full abstraction result. All the details are omitted, the reader is refered
to \cite{abramsky:game-semantics-tutorial,AM97a} for the proofs.

\subsubsection{Inequational soundness}

The inequational soundness result can be also proved for \ialgol.
Proving soundness of the evaluation requires a bit more work than in the PCF case because
the store needs to be made explicit. Also, an appropriate notion of \emph{computable term} must be defined
that takes into account the presence of stores in the evaluation semantics.
Again it is possible to prove that the model is computational adequate.
The inequational soundness then follows from evaluation soundness and computational adequacy:

%\begin{lem}[Soundness for IA terms] Let $\Gamma \vdash M : A$ be an IA term and a $\Gamma$ store $s$.
%If $s,M \eval s',V$ then the plays of $\sem{s,M} : I \multimap A
%\otimes !\Gamma$ which begin with a move of $A$ are identical to
%those of $\sem{s',V}$.
%\end{lem}

\begin{prop}[Inequational soundness]
\label{prop:ia_ineqsoundness} Let $M$ and $N$ be two \ialgol\ closed terms then
$$\sem{M} \subseteq \sem{N} \implies  M \obspre N $$
\end{prop}

\subsubsection{Definability}

The proof of definability is based on a factoring argument: strategies in
$\mathcal{G}_b$ can all be obtained by composing the non-innocent strategy $cell$ with an innocent strategy.
The strategy $cell$ can therefore be viewed as a generic non-innocent strategy. Using this factorization argument,
it is possible to prove the definability result:
\begin{prop}[Definability]
\label{prop:ia_definability} Let $\sigma$ be a compact well-bracketed
strategy on a game $A$ denoting a IA type. There is an IA-term $M$ such
that $\sem{M} = \sigma$.
\end{prop}

\subsubsection{Full abstraction}

Full abstraction for the model $\mathcal{C}_b$ is a consequence of proposition
\ref{prop:ia_ineqsoundness} and \ref{prop:ia_definability}:
\begin{thm}[Full abstraction]
Let $M$ and $N$ be two closed \ialgol-terms.
$$\sem{M} \precsim_b \sem{N} \ \iff \ M \obspre N$$
where $\precsim_b$ denotes the intrinsic preorder of the category
$\mathcal{C}_b$.
\end{thm}


\section{Algorithmic game semantics}

After the resolution of the ``Full Abstraction of PCF'' problem,
game semantics has become a very successful paradigm in fundamental
computer science. It has permitted to give full abstract semantics
for a variety of programming languages. More recently, game
semantics has emerged as a new approach to program verification and
program analysis. In particular in the paper \cite{ghicamccusker00},
the authors considered a fragment of Idealized Algol for which the
game semantics of programs can be expressed simply using regular
expressions. In this setting, observational equivalence of programs
becomes decidable. Consequently, numbers of interesting verification
problem become solvable. This development opened up a new direction
of research called \emph{Algorithmic game semantics}.

\subsection{Characterization of observational equivalence}

In \citep{AM97a} it is shown that observational equivalence of IA is
characterized by complete plays.

A play of a game is \emph{complete} if it is maximal and all
question have been answered. A game is \emph{simple} if the complete
plays are exactly those in which the initial question has been
answered. It can be shown that for any IA type $T$, $\sem{T}$ is a
simple game. The following characterization theorem holds for simple
games:
\begin{thm}[Characterization Theorem for Simple Game (Abramsky, McCusker 1997)]
Let $\sigma$ and $\tau$ be strategies on a simple game $A$ then:
$$\sigma \leq \tau \iff \textsf{comp}(\sigma) = \textsf{comp}(\tau)$$
\end{thm}
Therefore terms in IA are fully described by the complete plays of
the corresponding strategies.

\subsection{Finitary fragments of Idealized algol}
We introduce
some fragments of the language \ialgol. Firstly, \emph{Finitary
Idealized Algol} denotes the recursion-free sub-fragment of \ialgol\
over finite ground types. A term $\Gamma \vdash M:T$ of finitary
Idealized algol is an $i^{th}$-order term if $T$ is of order $i$ at
most and the variables in $\Gamma$ are of order strictly less than
$i$. $\ialgol_i$ denotes the fragment of finitary Idealized Algol
consisting of terms of $i^{th}$-order terms. $\ialgol_i +
\textsf{while}$ denotes the fragment $\ialgol_i$ augmented with
primitive recursion. Finally $\ialgol_i + \textsf{Y}_j$ where $j
\leq i$ denotes the fragment $\ialgol_i$ augmented with a set of
fixed-point iterators $\textsf{Y}_A : (A\rightarrow A ) \rightarrow
A$ for any type $A$ of order $j$ at most.

We recall the observational equivalence decision problem: given two
$\beta$-normal forms $M$ and $N$ in a given fragment of \ialgol,
does $M \approx N$ hold?

This problem has been investigated and decidability results have
been obtained for a complete class of fragments of Idealized Algol.
These results help us to understand the limits of Algorithmic Game
Semantics. We now present briefly those results.

\subsubsection{$\ialgol_2$ fragment}
In \cite{ghicamccusker00}, Dan R. Ghica and Guy McCusker considered the $\ialgol_2$ fragment.
They show that in $\ialgol_2$ the set of complete plays are
representable by extended regular languages.

\begin{lem}[Ghica and McCusker 2000]
For any $\ialgol_2$-term $\Gamma \vdash M : T$, the set of complete
plays of $\sem{\Gamma \vdash M : T}$ is regular.
\end{lem}
Since equivalence of regular expression is decidable, this shows
decidability of observational equivalence of $\ialgol_2$-terms. In
the same paper they show that the same result holds for the
$\ialgol_2 +\textsf{while}$ fragment.

In \cite{Ong02}, it is shown that observational equivalence is
undecidable for $\ialgol_2 + \textsf{Y}_1$.


\subsubsection{Other fragments of IA}

Observational equivalence is decidable for $\ialgol_3$. This is
proved in \cite{Ong02} by reduction to the \emph{Deterministic
Push-down Automata Equivalence} problem. Unfortunately, this result
does not extend beyond order $3$: Murawski showed in
\cite{murawski03program} that the problem is undecidable for
$\ialgol_i$ with $i\geq4$.

However in $\ialgol_3 + \textsf{while}$ the problem becomes
decidable: it is shown in \cite{C:MW05} that the problem is EXPTIME
in $\ialgol_2 + \textsf{while}$ and $\ialgol_3 + \textsf{while}$.

Moreover in \cite{C:MOW05} it is shown that $\ialgol_i +Y_0$, for $i
= 1, 2, 3$ is as difficult as the DPDA equivalence problem. This
problem is decidable \citep{DBLP:journals/tcs/Senizergues01} but no
complexity result is known about it. We only know that it is
primitive recursive \citep{stirling02}.

\subsubsection{The complete classification}
\begin{center}
\begin{tabular}{rcccc}
Fragment  & pure & +while & +Y0 & +Y1 \\ \hline \hline
$\ialgol_0$ & PTIME & ??? & ??? & $\times$  \\
$\ialgol_1$ & coNP & PSPACE & DPDA EQUIV & ??? \\
$\ialgol_2$ & PSPACE & PSPACE & DPDA EQUIV & undecidable \\
$\ialgol_3$ &EXPTIME & EXPTIME & DPDA EQUIV & undecidable \\
$\ialgol_i, i \geq 4$  & undecidable & undecidable & undecidable
& undecidable
\end{tabular}
\end{center}

The $\times$ symbol denotes undefined \ialgol\ fragments.

The coNP and PSPACE results are due to Murawski \citep{Mur04b}.

%\input dataref


% second chapter
\chapter{Safe $\lambda$-Calculus}
In \cite{KNU02}, the authors introduced a restriction on
higher-order grammars called \emph{safety} in order to study the
infinite hierarchy of trees recognized by a higher-order pushdown
automaton. They proved that trees recognized by pushdown automata of
level $n$ coincide with trees generated by safe higher-order
grammars of level $n$. This characterisation permitted them to prove
the decidability of the monadic second-order theory of infinite
trees recognized by a higher-order pushdown automaton of any level.

Safety has also appeared in a different form in \cite{Dam82} under
the name \emph{restriction of derived types}. The forthcoming thesis
of Jolie de Miranda \citep{demirandathesis} contains a comparison of
safety and the restriction of derived types.

More recently, Ong proved in \cite{OngLics2006} that the safety
assumption of \cite{KNU02} is in fact not necessary. More precisely,
the paper shows that the MSO theory of trees generated by order-$n$
recursion schemes is $n$-EXPTIME complete.

For this particular problem, \emph{safety} happens to be an
artificial restriction. However when the \emph{safety} condition is
transposed to the simply-typed $\lambda$-calculus, it gives some
interesting properties. In particular, for safe terms, it becomes
unnecessary to rename variables when performing substitution.

This chapter starts with a presentation of the original version of
the safe $\lambda$-calculus where types are required to satisfy a
condition called homogeneity. We then give a more general definition
which does not require type homogeneity.

\section{Homogeneous Safe $\lambda$-Calculus}
\label{sec:safe_homog}

\subsection{Type homogeneity}
Let $Types$ be the set of simple types generated by the grammar $A
\, ::= \, o \; | \; A \typear A$. Any type different from the base
type $o$ can be written $(A_1, \cdots, A_n, o)$ for some $n \geq 1$,
which is a shorthand for $A_1 \typear \cdots \typear A_n \typear o$ (by
convention, $\rightarrow$ associates to the right). If $T=(A_1,
\cdots, A_n, o)$ then the arity of $T$, written $arity(T)$, is
defined to be $n$.

Suppose that a ranking function ${\sf rank} :
Types \funto (L, \leq)$ is given where $(L, \leq)$ is any linearly ordered
set. Possible candidates for the ranking function are:
\begin{itemize}
\item ${\sf ord} : Types \funto (\nat,\leq)$ with $\ord{o} = 0$
and $\ord{A \typear B} = \max(\ord{A}+1, \ord{B})$;
\item ${\sf height} : Types \funto (\nat,\leq)$ with
$\slheight{A \typear B} = 1 + \max(\slheight{A}, \slheight{B})$ and
$\slheight{o} = 0$ ;
\item ${\sf nparam} : Types \funto (\nat,\leq)$ with $\nparam{o} = 0$
and $\nparam{A_1, \cdots, A_n} = n$;
\item ${\sf ordernp} : Types \funto (\nat \times \nat,\leq)$ with $ {\sf ordernp} (t)  = \langle \order{t}, \nparam{t} \rangle$ for $t \in Types$.
\end{itemize}
Following \cite{KNU02}, we say that a type is {\sf rank}-homogeneous
if it is $o$ or if it is $(A_1, \cdots, A_n, o)$ with the condition
that $\rank{A_1} \geq \rank{A_2}\geq \cdots \geq \rank{A_n}$ and
each $A_1$, \ldots, $A_n$ is {\sf rank}-homogeneous.



Suppose that $\overline{A_1}$, $\overline{A_2}$, \ldots,
$\overline{A_n}$ are $n$ lists of types, where $A_{ij}$ denotes the
$j$th type of list $\overline{A_i}$ and $l_i$ the size of
$\overline{A_i}$, then the notation $A \; = \; (\overline{A_1} \, |
\, \cdots \, | \, \overline{A_r} \, | \, o)$ means that
\begin{itemize}
  \item $A$ is the type $(A_{11},A_{12},\cdots, A_{1l_1}, A_{21}, \cdots,A_{2l_2}, \cdots A_{n1},\cdots, A_{nl_n},o)$
  \item $\forall i: \forall u,v \in A_i : \rank u = \rank v $
  \item $\forall i,j . \forall u \in A_i . \forall v \in A_j . i<j \implies \rank u >
   \rank v $
\end{itemize}
and therefore $A$ is {\sf rank}-homogenous. This notation organises
the $A_{ij}$s into partitions according to their ranks. Suppose $B =
(\overline{B_1} \, | \, \cdots \, | \, \overline{B_m} \, | \, o)$,
we write $(\overline{A_1} \, | \, \cdots \, | \, \overline{A_n} \, |
\, {B})$ to mean
\[(\overline{A_1} \, | \, \cdots \, | \, \overline{A_n} \, | \,
\overline{B_1} \, | \, \cdots \, | \, \overline{B_m} \, | \, o).\]

From now on, we only consider the rank function {\sf ord}. We will
use the term ``homogeneous'' to refer to {\sf ord}-homogeneity.


\subsection{Safe Higher-Order Grammars}
We now present the original notion of safety introduced in \cite{KNU02} as a restriction for higher-order grammars. We present briefly the notion of higher-order grammar. The reader is referred to \cite{KNU02,demirandathesis,safety-mirlong2004}
for more details.

Suppose that $\Gamma$ is a set of typed symbols then the set of \emph{applicative terms} written $\mathcal{T}(\Gamma)$ is the
closure of $\Gamma$ under the application rule i.e. if $s: A\rightarrow B$ and $t:A$ are in $\mathcal{T}(\Gamma)$ then so is $st :B$.

\begin{dfn}[Higher-order grammar]
A \emph{higher-order grammar} is a tuple $\langle \Sigma,
\mathcal{N}, V, \mathcal{R}, S \rangle$, where
\begin{itemize}
\item $\Sigma$ is a ranked alphabet of terminals of order at most 1,
\item $V$ is a finite set of typed variables,
\item $\mathcal{N}$ is a finite set of homogeneously-typed non-terminals,
\item $S$ a distinguished symbol of $\mathcal{N}$ of ground type, called the start symbol,
\item $\mathcal{R}$ is a finite set of production rules, one for each $F : (A_1, \ldots, A_n, o) \in \mathcal{N}$, of the form
    $$ F z_1 \ldots z_m \rightarrow e$$
where $z_i$ is a variable of type $A_i$ and $e$ is an applicative
term of type $o$ in $\mathcal{T}(\Sigma \union \mathcal{N} \union
\{z_1 \ldots z_m \} )$. The $z_i$s are called the \emph{parameters}
of the rule.
\end{itemize}
\end{dfn}
A higher-order recursion scheme is a \emph{deterministic} higher-order grammar i.e. for each non-terminal $F \in \mathcal{N}$ there is exactly one production rule with $F$ on the left hand side.
Higher-order recursion schemes are used as generators
of infinite trees.

The order of a rewrite rule is the order of the non-terminal symbol
appearing on the left hand side of the rule. The order of a grammar
is the highest order of its non-terminals.

Safety is a syntactic restriction on higher-order grammars. It can be formulated as
follows:
\begin{dfn}[Safe higher-order grammar]
  Let $G$ be a higher-order grammar $G$ of order $n$
    whose non-terminals are of homogeneous type.
    $G$ is \emph{unsafe} if and only if there is a rewrite rule $F z_1 \ldots z_m \rightarrow e$ where
   $e$ contains a subterm $t$ such that:
  \begin{enumerate}
    \item $t$ occurs in an operand position in $e$,
    \item $t$ is of order $k>0$,
    \item $t$ contains a parameter of order strictly less than $k$.
  \end{enumerate}
  $G$ is \emph{safe} if it is not unsafe.
\end{dfn}

Let us illustrate the definition with an example taken from \cite{KNU02}:
\begin{exmp} Let $f:(o,o,o)$, $g,h:(o,o)$ and $a,b:o$ be $\Sigma$ constants.
 The grammar of level 3 with non-terminals $S:o$ and $F: ((o,o),o,o,o)$ and production rules:
\begin{eqnarray*}
    S &\rightarrow&  F g a b \\
    F \varphi x y &\rightarrow& f ( F ( F \varphi x ) y (h y)) (f (\varphi x) y)
\end{eqnarray*}
is not safe because the term $F \varphi x : (o,o)$ containing a variable of order $0$
occurs at an operand position in the right-hand side expression of the second rule.

On the other hand, the grammar with the following production rules is safe:
\begin{eqnarray*}
    S &\rightarrow&  G (g a) b \\
    G z y &\rightarrow& f ( G ( G z y) (h y)) (f z y)
\end{eqnarray*}
Moreover it can be shown that these two grammars are equivalent in the sense that they generate the same
infinite tree.
\end{exmp}


\subsection{Rules of the Safe $\lambda$-Calculus}

There is a correspondence between higher-order grammars and the simply-typed $\lambda$-calculus. The non-terminals of a recursion scheme can be interpreted as $\lambda$-abstractions in the
simply-typed $\lambda$-calculus. The $\Sigma$-constants are
interpreted as ``constructors'' constants (in the sense of
constructor used in functional programming languages to represent
abstract data-types such as trees). The notions of variable and
application are directly transposed to the equivalent notions in the
simply-typed $\lambda$-calculus. Using this analogy it is possible
to derive a version of the safety restriction for the
$\lambda$-calculus.

The safe $\lambda$-calculus has been first proposed in
\cite{DBLP:conf/fossacs/AehligMO05}, a corrected definition appeared
in \cite{Ong2005}. The definition that we give here is slightly more
general in the sense that we allow the use of $\Sigma$-constants of
any higher-order type whereas the original definition only allows
first-order constants.


The \textbf{safe $\lambda$-calculus} is a sub-system of the
simply-typed $\lambda$-calculus. Typing judgements (or
terms-in-context) are of the form:
\begin{equation}
\nonumber \seq{\overline{x_1}:\overline{A_1} \, | \, \cdots \, | \,
\overline{x_n} :  \overline{A_n}}{M : B}
\end{equation}
which is shorthand for $\seq{x_{11} : A_{11}, \cdots, x_{1r}:
A_{1r}, A_{21},\ldots }{M : B}$ such that the context variables are listed in decreasing type order and
 with the condition that $\ord{x_{ik}} < \ord{x_{jl}}$ for any $k, l$ and $i<j$.

\emph{Valid typing judgements} of the system are defined by
induction over the following rules, where $\Delta$ is a given
homogeneously-typed alphabet i.e.\ a countable set of symbols such that each
symbol has an homogeneous type, and $\Sigma$ is a set of homogeneously-typed
constants:

$$ \rulename{wk}
    {   \rulef{ \seq{\Gamma}{M:B} \qquad {\Gamma \subset \Delta} }
             { \seq{\Delta }{M : B}}
   }
\qquad
    \rulename{perm}
    {
      \rulef { \seq{\Gamma}{M:B} \qquad \sigma(\Gamma) \hbox{ homogeneous} }
            { \seq{\sigma(\Gamma)}{M : B} }
    }
$$

$$ \rulename{\Sigma\mbox{\textbf{-const}}}  \rulef{}{\seq{}{b : A}}\ b:A \in \Sigma
\qquad
 \rulename{var} \rulef{}{\seq{x_{ij} : A_{ij}\, }{x_{ij} : A_{ij}}}
$$

$$\rulename{abs}
\rulef{
 {\seq{\overline{x_1} : \overline{A_1}\, | \, \cdots\, | \, \overline{x_{n+1}} : \overline{A_{n+1}}}{M : B}}            \qquad \ord{\overline{A_{n+1}}} > \ord{B}
 }
 { \seq{\overline{x_1} : \overline{A_1}\, | \, \cdots\, | \, \overline{x_{n}} : \overline{A_{n}}}
     { \lambda \overline{x_{n+1}} : \overline{A_{n+1}} .M : (\overline{A_{n+1}} \, | \, B)  }
 }
$$

$$ \rulename{app} \rulef{{\seq{\Gamma}{M : (\overline{B_1} \, | \, \cdots \, | \, \overline{B_m} \, | \, o)} \qquad
\seq{\Gamma}{N_1 : B_{11}} \quad \cdots \quad \seq{\Gamma}{N_{l} :
B_{1l}} \qquad l = |\overline{B_1}| }}
    { \seq{\Gamma}{M N_1
\cdots N_{l} : (\overline{B_2} \, | \, \cdots \, | \,
\overline{B_m} \, | \, o)}} $$

$$ \rulename{app^+} \rulef
    {\seq{\Gamma}{M : (\overline{B_1} \, | \, \cdots \, | \, \overline{B_m} \, | \, o)} \qquad
    \seq{\Gamma}{N_1 : B_{11}} \quad \cdots \quad \seq{\Gamma}{N_{l} :
    B_{1l}} \qquad l < |\overline{B_1}| }
    { \seq{\Gamma}{M N_1
    \cdots N_{l} : (\overline{B} \, | \, \overline{B_2} \, | \ \cdots \, | \,
    \overline{B_m} \, | \, o)}} $$

where $\overline{B_1} = B_{11}, \ldots, B_{1l},\overline{B}$ with
the condition that every variable in $\Gamma$ has an order strictly greater
than $\ord{\overline{B_1}}$.






\begin{property}[Basic properties]
\label{proper:safe_basic_prop} Suppose $\Gamma \vdash M : B$ is a
valid judgment then
\begin{itemize}
\item[(i)] $B$ is homogeneous;
\item[(ii)] every free variable of $M$ has order at least $\ord{B}$;
\item[(iii)] $fv(M) \vdash M : B$,
\end{itemize}
where $fv(M) \subseteq \Gamma$ denotes the context constituted of the variables
in $\Gamma$ occurring free in $M$.
\end{property}
\begin{proof}
(i) and (ii) are proved by an easy structural induction. (iii) is
due to the fact that the weakening rule is the only rule which can
introduce a variable not occurring freely in $M$ in the context of a
typing judgement.
\end{proof}

We now define a special kind of substitution that performs
simultaneous substitution and permits variable capture i.e. that
does not rename variables when the substitution is performed on an
abstraction.

\begin{dfn}[Capture-permitting simultaneous substitution (for homogeneous safe terms)]
\label{dnf:safe_simsubst} We use the notation
$\subst{\overline{N}}{\overline{x}}$ for $\subst{N_1 \ldots N_n}{x_1
\ldots x_n}$ and $\overline{y}:\overline{A}$ for $y_1:A_1, \ldots,
y_p:A_p$. A safe term has necessarily one of the forms occurring on
the left-hand side of the following equations, where $M$, $N_1,
\ldots N_l$ are safe terms. The capture-permitting simultaneous
substitution is then defined by:
\begin{eqnarray*}
c \subst{\overline{N}}{\overline{x}} &=& c \quad \mbox{ where $c$ is a $\Sigma$-constant}\\
x_i \subst{\overline{N}}{\overline{x}} &=& N_i\\
 y \subst{\overline{N}}{\overline{x}} &=& y \quad \mbox{ if } y \not \neq x_i \mbox{ for all } i,\\
(M N_1 \ldots N_l) \subst{\overline{N}}{\overline{x}} &=& (M \subst{\overline{N}}{\overline{x}}) (N_1 \subst{\overline{N}}{\overline{x}}) \ldots  (N_l \subst{\overline{N}}{\overline{x}})\\
(\lambda \overline{y} : \overline{A}. M)
\subst{\overline{N}}{\overline{x}} &=& \lambda \overline{y} . M
\subst{\overline{N} \upharpoonright I}{\overline{x} \upharpoonright I} \\
&& \mbox{where } I  = \{ i \in 1..n \ | \ x_i \not \in \overline{y} \}
\end{eqnarray*}

where $ \upharpoonright$ is the index filtering operator: if $s$ is
a sequence and $I$ a set of indices then $s \upharpoonright I$ is
the subsequence of $s$ obtained by keeping only the element in $s$
at positions in $I$.
\end{dfn}

This substitution is well-defined for safe terms in the sense that safety is preserved by substitution:

\begin{lem}[Capture-permitting simultaneous substitution preserves safety]
\label{lem:subst_preserve_safety} Let $\Gamma \union \overline{x}
\vdash M$ be a safe term where $\overline{x}$ denotes a list of
variables (which do not necessarily belong to the same partition).

For any safe terms $\Gamma \vdash N_1, \cdots, \Gamma \vdash N_n$,
the capture-permitting simultaneous substitution $M[N_1 / x_1 ,
\cdots, N_n / x_n]$ is safe. In other words, the following judgment
is valid:
$$ \Gamma \vdash M[N_1 / x_1 , \cdots, N_n / x_n] $$
\end{lem}
\begin{proof}
An easy proof by an induction on the structure of the safe term.
\end{proof}



With the traditional substitution, it is necessary to rename
variables when performing substitution on an abstraction in order to
avoid possible variable capture. As a consequence, in order to
implement substitution one needs to have access to an unbound number
of variable names. An interesting property of the homogeneous Safe
$\lambda$-Calculus is that variable capture never occurs when
performing substitution. In other words, the traditional
substitution can be safely replaced by the capture-permitting
substitution:

\begin{lem}[No variable capture lemma]
\label{lem:homog_nocapture} In the safe $\lambda$-calculus, there is
no variable capture when performing the following capture-permitting
simultaneous substitution:
$$ M[N_1 / x_1 , \cdots, N_n / x_n] $$
provided that $\Gamma \union \overline{x} \vdash M$, $\Gamma \vdash  N_1, \cdots ,\Gamma \vdash  N_n$ are valid judgments.
\end{lem}

\begin{proof}
We prove the result by induction. The variable, constant and
application cases are trivial. For the abstraction case, suppose $M
= \lambda \overline{y} : \overline{A}. P$ where $\overline{y} = y_1
\ldots y_p$. The capture-permitting simultaneous substitution gives:
$$M \subst{\overline{N}}{\overline{x}} = \lambda \overline{y} . P
\subst{\overline{N} \upharpoonright I}{\overline{x} \upharpoonright
I} \mbox{ where } I  = \{ i \in 1..n \ | \ x_i \not \in \overline{y}
\}. $$


By the induction hypothesis there is no variable capture in $P
\subst{\overline{N} \upharpoonright I}{\overline{x} \upharpoonright
I}$. Hence variable capture can only happen when the variable $y_j$
occurs freely in $N_i$ and $x_i$ occurs freely in $P$ for some $i
\in I$ and $j \in 1..p$. In that case, property
\ref{proper:safe_basic_prop} (ii) gives:
$$ \ord{y_j} \geq \ord{N_i} = \ord{x_i}$$

Moreover $i\in I$ therefore $x_i \not \in \overline{y}$ and since $x_i$ occurs freely in $P$, $x_i$ must also occur freely in the safe term
$\lambda \overline{y}. P$. Thus, property \ref{proper:safe_basic_prop} (ii) gives:
$$ \ord{x_i} \geq \ord{\lambda y_1 \ldots y_p . T} \geq 1+ \ord{y_j} > \ord{y_j}$$

which, together with the previous equation, gives a contradiction.
\end{proof}




\subsection{Safe $\beta$-reduction}

We now introduce the notion of safe $\beta$-redex and show how to
reduce them using the capture-permitting simultaneous substitution.
We will then show that a safe $\beta$-reduction reduces to a safe
term.


In the simply-typed lambda calculus a redex is a term of the form $(\lambda x . M) N$.
We generalize this notion to the safe lambda calculus. We call multi-redex a term of the form
$(\lambda x_1 \ldots x_n . M) N_1 \ldots N_l$ (it is not required to have $n=l$).


We say that a multi-redex is safe if it respects the formation rules
of the safe $\lambda$-calculus: the multi-redex $(\lambda x_1 \ldots
x_n . M) N_1 \ldots N_l$ is a safe redex if the variable
$x_1,\ldots,x_n$ are abstracted altogether at once using the
abstraction rule and if the terms $N_1 \ldots N_l$ are applied to
the term $\lambda x_1 \ldots x_n . M$ at once using either the rule
$\rulename{app^+}$ or $\rulename{app}$. The formal definition
follows:

\begin{dfn}[Safe redex]
A safe redex is a term of the form:
$$(\lambda \overline{x} . M) N_1 \ldots N_l$$
such that
\begin{itemize}
  \item variables $\overline{x}=x_1\ldots x_n$ are abstracted
altogether by one occurrence of the rule $\rulename{abs}$ in the
proof tree (possibly followed by the weakening rule). This implies
that:
$$\ord{M} -1 \leq \ord{\overline{x}} = \ord{x_1} = \ldots = \ord{x_n};$$
\item the terms $(\lambda \overline{x} . M)$, $N_1$,
$N_l$ are applied together at once using either:
\begin{itemize}
    \item the rule $\rulename{app}$:
        $$   \rulef{
                    \Sigma \vdash \lambda \overline{x} . M : (\overline{B_1}|\ldots|\overline{B_m}|o)
                    \quad
                    \Sigma \vdash N_1         \quad \ldots \quad \Sigma \vdash N_l
                    \quad l = |\overline{B_1}|
            }
            {
            \Sigma \vdash (\lambda \overline{x} . M) N_1 \ldots N_l
            } (\mathbf{app}),
        $$
        in which case  $n\leq |\overline{B_1}| = l$;

\item or the rule $\rulename{app^+}$:
        $$   \rulef{
                    \Sigma \vdash \lambda \overline{x} . M : (\overline{B_1}|\ldots|\overline{B_m}|o)
                    \quad
                    \Sigma \vdash N_1         \quad \ldots \quad \Sigma \vdash N_l
                    \quad l < |\overline{B_1}|
            }
            {
            \Sigma \vdash (\lambda \overline{x} . L) N_1 \ldots N_l
            } (\mathbf{app^+}),
        $$
      in which case $n \leq |\overline{B_1}|$ and no relation holds between $n$ and $l$.
\end{itemize}
\end{itemize}
It is not required to have $n = |\overline{B_1}|$.
\end{dfn}

Note that there are safe terms of the form $(\lambda x_1 \ldots x_n
. M) N_1 \ldots N_l$ with $l>n$. For instance the term $ (\lambda f
. ((\lambda g h . h) a) ) a a$ of type $o \rightarrow o$ for some
constant $a:o \rightarrow o$ and variables $x : o$ and $f,g,h:o
\rightarrow o$, can be formed using the $\rulename{app}$ rule as
follows:
$$ \rulef{
    \emptyset \vdash (\lambda f . ((\lambda g h . h) a) ) : (o,o),(o,o),o,o
        \quad \emptyset \vdash a : o,o
        \quad \emptyset \vdash a : o,o
    }
    {
       \emptyset \vdash (\lambda f . ((\lambda g h . h) a) ) a a : o,o
    } \rulename{app}
$$


\begin{dfn}[Safe reduction $\beta_s$] \
\label{dfn:safereduction} For the sake of concision, the following
abbreviations are used $\overline{x} = x_1 \ldots x_n$,
$\overline{N} = N_1 \ldots N_l$, and when $n\geq l$, $\overline{x_L}
= x_1 \ldots x_l$, $\overline{x_R} = x_{l+1} \ldots x_n$.
\begin{itemize}
\item The relation $\beta_s$ is defined on the set of safe redex as follows:
\begin{eqnarray*}
\beta_s &=&
\{  \ (\lambda \overline{x} : \overline{A} . T) N_1 \ldots N_l \mapsto \lambda \overline{x_R}. T\subst{\overline{N}}{\overline{x_L}}  \\
&& \mbox{ where $(\lambda \overline{x} : \overline{A} . T) N_1 \ldots N_l$ is a safe redex such that $n> l$}
\} \\
&\union&
\{ \ (\lambda \overline{x} : \overline{A} . T) N_1 \ldots N_l \mapsto T\subst{\overline{N}}{\overline{x}} N_{n+1} \ldots N_l  \\
&& \mbox{ where $(\lambda \overline{x} : \overline{A} . T) N_1 \ldots N_l$ is a safe redex such that $n\leq l$}
\}
\end{eqnarray*}
where the notation $\subst{\overline{N}}{\overline{x}}$ denotes the capture-permitting simultaneous substitution.

\item
The safe $\beta$-reduction, written $\betasred$, is the closure of
the relation $\beta_s$ by compatibility with the formation rules of
the safe $\lambda$-calculus.
\end{itemize}
\end{dfn}



We observe that safe $\beta$-reduction is a certain kind of multi-steps $\beta$-reduction.
\begin{property}
$\betasred \subset \betaredtr$, i.e. the safe
$\beta$-reduction relation is included in the transitive closure of the $\beta$-reduction relation.
\end{property}
\begin{proof}
Suppose that $(M\mapsto N) \in \beta_s$. We show that $M \betared^* N$.
\begin{itemize}
\item Suppose that the safe-redex is
$M \equiv (\lambda \overline{x} : \overline{A} . T) N_1 \ldots N_l$ such that $n\leq l$ then:
\begin{eqnarray*}
 M &=_\alpha& (\lambda z_1 \ldots z_n .T [z_1,\ldots z_n /x_1,\ldots x_n] ) \ N_1  N_2 \ldots N_l
            \\
&& \mbox{where the $z_i$ are fresh variables}  \\
     &\betared& (\lambda z_2 \ldots z_n .T [z_1,\ldots z_n /x_1,\ldots x_n] \subst{N_1}{z_1} ) \ N_2 \ldots N_l \\
&& \mbox{ (because the $z_i$s do not occur freely in $N_1$) }\\
%%    &=_\alpha& (\lambda z_2 \ldots z_n .T [z_2,\ldots z_n /x_2,\ldots x_n] \subst{N_1}{x_1})\  N_2 \ldots N_l  \qquad \mbox{where the $z_i$ are fresh variables}  \\
    &\betared& \ldots \\
    &\betared& (T [z_1,\ldots z_n /x_1,\ldots x_n] \subst{N_1}{z_1}  \ldots \subst{N_n}{z_n})\  N_{n+1} \ldots N_l \\
    &\betared& (T [N_1\ldots N_l/x_1,\ldots x_l])\ N_{n+1} \ldots
    N_l,
\end{eqnarray*}
and since $T$ is safe, the substitution $T [N_1\ldots N_l/x_1,\ldots
x_l]$ in the last equation can be performed using the
capture-permitting substitution. Hence $M \betared^* N$.

\item
 Suppose that $M \equiv (\lambda \overline{x} : \overline{A} . T) N_1 \ldots N_l$ such that $n> l$, then necessarily
the redex must be formed using the $\rulename{app^+}$ rule. The
side-condition of this rules says that the free variables of the
terms $N_1, \ldots N_l$ have all order strictly greater than
$\ord{\overline{x}}$, hence the $x_i$s do not occur freely in $N_1,
\ldots N_l$. Therefore:
\begin{eqnarray*}
 M &=& (\lambda x_1 \ldots x_n .T) \ N_1  N_2 \ldots N_l  \\
     &\betared& (\lambda x_2 \ldots x_n .T \subst{N_1}{x_1} ) \ N_2 \ldots N_l \\
            && \mbox{(for $i \in 2..n$, $x_i$ does not occur freely in $N_1$)}\\
    &\betared& \ldots \\
    &\betared& \lambda x_{l+1} \ldots x_n . T \subst{N_1}{x_1}  \ldots \subst{N_l}{x_l} \\
        && \mbox{(for $i \in (l+1)..n$,  $x_i$ does not occur freely in $N_l$)}\\
    &\betared& \lambda x_{l+1} \ldots x_n . T [N_1\ldots, N_l /  \ x_1,\ldots, x_l] \\
        && \mbox{(the $x_i$ do not occur freely in $N_1, \ldots
        N_l$)},
\end{eqnarray*}
and since $T$ is safe, the substitution $T [N_1\ldots N_l/x_1,\ldots
x_l]$ in the last equation can be performed using the
capture-permitting substitution. Hence $M \betared^* N$.
\end{itemize}
\end{proof}

\begin{property} In the simply-typed $\lambda$-calculus:
\begin{enumerate}
\item $\betasred$ is strongly normalizing.
\item $\beta_s$ has the unique normal form property.
\item $\beta_s$ has the Church-Rosser property.
\end{enumerate}
\end{property}

\begin{proof}
1. This is because $\betasred \subset \betaredtr$ and, $\betared$ is
strongly normalizing in the simply-typed $\lambda$-calculus. 2. A
term has a safe redex iff it has a $\beta$-redex therefore the set
of $\beta_s$ normal forms is equal to the set of $\beta_s$ normal
forms. Hence, the unicity of $\beta$-normal form implies the unicity
of $\beta_s$-normal form. 3. is a consequence of 1 and 2.
\end{proof}


Capture-permitting simultaneous substitution preserves safety (lemma
\ref{lem:subst_preserve_safety}), consequently any safe redex
reduces to a safe term:

\begin{lem}[The safe reduction $\beta_s$ preserves safety]
\label{lem:homoh_safered_preserve_safety}
If $M$ is safe and $M \betasred N$ then $N$ is safe.
\end{lem}

\begin{proof}
It suffices to show that the relation $\beta_s$ preserves safety.
Consider the safe-redex $(s\mapsto t) \in \beta_s$ where $ s \equiv (\lambda x_1 \ldots x_n . M) N_1 \ldots N_l $ .
We proceed by case analysis on the last rule used to form the redex.
\begin{itemize}
\item Suppose the last rule used is $\rulename{app}$, then necessarily $n\leq l$ and the reduction is
$$(\lambda x_1 \ldots x_n . M) N_1 \ldots N_l \qquad \mapsto  \qquad t \equiv M[N_1 / x_1 , \cdots, N_n / x_n]\ N_{n+1} \ldots N_l.$$
The first premise of the rule $\rulename{app}$ tells us that $M$ is safe therefore using lemma \ref{lem:subst_preserve_safety} and
the application rule we obtain that $t$ is safe.

\item Suppose the last rule used is $\rulename{app^+}$ and $n> l$ then the reduction is
$$ (\lambda \overline{x_L} : \overline{A_L} \
\overline{x_R}: \overline{A_R} . T) \overline{N_L} \qquad \mapsto
\qquad t \equiv \lambda \overline{x_R}: \overline{A_R} .
T\subst{\overline{x_L}}{\overline{N_L}}.
$$
By lemma \ref{lem:subst_preserve_safety}, $T\subst{\overline{x_L}}{\overline{N_L}}$ is a safe term.
Using the rule $\rulename{abs}$ we derive that $t$ is safe.

\item Suppose the last rule used is $\rulename{app^+}$ and $n\leq l$ then the reduction is
$$(\lambda x_1 \ldots x_n . M) N_1 \ldots N_l \qquad \mapsto \qquad t \equiv M[N_1 / x_1 , \cdots, N_n / x_n]\ N_{n+1} \ldots N_l$$
We conclude that $t$ is safe similarly to case $\rulename{app}$.

\item Rule $\rulename{wk}$ $\rulename{seq}$: these cases reduce to one of the previous cases.
\end{itemize}
\end{proof}


\begin{rem}
\label{rem:betasred_notpreserv_unsafety} $\betasred$ \emph{does not}
preserves un-safety: given two terms $S$ safe and $U$ unsafe of the
same type, the term $(\lambda x y . y) U S$ is also unsafe but it
$\beta_s$-reduces to $S$ which is safe.
\end{rem}


\subsection{An alternative system of rules}


In this section, we will refine the formation rules
given in the previous section. We say that $\Gamma \vdash M : A$ satisfies $P_i$ for $i \in \zset$ if the
variables in $\Gamma$ all have orders at least $\ord{A}+i$. We introduce the notation $\Gamma \vdash^{i} M : A$ for $i \in
\zset$ to mean that $\Gamma \vdash M : A$ is a valid judgment satisfying $P_i$.


We remark that if $\Gamma \vdash M : A$ then the variables in $\Gamma$ with order
strictly smaller than $M$ cannot occur freely in $M$ and therefore it is possible to restrict
the context to a smaller number of variables:

\begin{lem}[Context reduction]
\label{lem:restriction}

If $\Gamma \vdash^i M : A$ then $\Gamma' \vdash^{0} M : A$
where $$\Gamma' = \{ z \in \Gamma \ |
\ \ord{z} \geq \ord{M} \} = \Gamma \setminus \{ z \in \Gamma \ | \ \ord{M} + i \leq \ord{z} < \ord{M} \}$$
\end{lem}
\begin{proof}
If $i\geq 0$ then the result is trivial. Suppose $i<0$. We proceed
by structural induction and case analysis. We only give the details
for the application cases $\rulename{app}$ and $\rulename{app^+}$:
\begin{itemize}
\item Case of the rule $\rulename{app}$:

    \[ (\mathbf{app}) \
    \rulef
        {\seq{\Gamma}{M : (\overline{B_1} \, | \, \cdots \, | \, \overline{B_m} \, | \, o)} \qquad
            \seq{\Gamma}{N_1 : B_{11}} \quad \cdots \quad \seq{\Gamma}{N_{l} :
            B_{1l}} \qquad l = |\overline{B_1}| }
        { \seq{\Gamma}{M N_1
            \cdots N_{l} : (\overline{B_2} \, | \, \cdots \, | \,
            \overline{B_m} \, | \, o)}}
    \]

    If the conclusion satisfies $P_i$ then, for all $z \in \Gamma$:
    \begin{eqnarray*}
    \ord{z} \geq 1 + \ord{\overline{B_2}} + i
    &=& 1 + \ord{\overline{B_1}} + \ord{\overline{B_2}} - \ord{\overline{B_1}} + i \\
    &=& \ord{M} + (\ord{\overline{B_2}} - \ord{\overline{B_1}} + i)
    \end{eqnarray*}
    Therefore the first premise satisfies $P_j$ where $j={\ord{\overline{B_2}} - \ord{\overline{B_1}} + i}$.
    Hence by the induction hypothesis,
    $$\Gamma' \vdash^{0} M : (\overline{B_1} \, | \, \cdots \, | \, \overline{B_m} \, | \, o)$$
    where $\Gamma' = \Gamma \setminus \{ z \in \Gamma \ | \ \ord{M} + j \leq \ord{z} < \ord{M} \}$.


    Similarly, for all $z \in \Sigma$:
    \begin{eqnarray*}
    \ord{z} \geq 1 + \ord{\overline{B_2}} + i
    &=& \ord{\overline{B_1}} + (1+\ord{\overline{B_2}} - \ord{\overline{B_1}} + i) \\
    &=& \ord{\overline{B_1}} + j+1
    \end{eqnarray*}
    Hence by the induction hypothesis:
    $$\Gamma'' \vdash^0 N_k : B_{1k} \mbox{ for } k \in 1..l$$
    where $\Gamma'' = \Gamma \setminus \{ z \in \Gamma \ | \ \ord{M} + j+1 \leq \ord{z} < \ord{M} \}$.

    Furthermore, $\Gamma'' = \Gamma' \union \{ z \in \Gamma \ | \ \ord{M} + j = \ord{z}\}$ therefore
    the weakening rule gives:
    $$\Gamma'' \vdash^{-1} M : (\overline{B_1} \, | \, \cdots \, | \, \overline{B_m} \, | \, o)$$

    Finally the $\rulename{app}$ rule gives:
    $$\rulef{\Gamma'' \vdash^{-1} M : (\overline{B_1} \, | \, \cdots \, | \, \overline{B_m} \, | \, o)
    \quad \Gamma'' \vdash^0 N_1 : B_{11} \quad \ldots \quad \Gamma'' \vdash^0 N_1 : B_{1l}
    }
        { \Gamma'' \vdash M N_1 \ldots N_l : (\overline{B_2} \, | \, \cdots \, | \,
            \overline{B_m} \, | \, o)}
    $$
    such that for all $z\in \Gamma''$:
    \begin{eqnarray*}
    \ord{z} \geq \ord{\overline{B_1}}
    &\geq& 1 + \ord{\overline{B_2}} = \ord{M N_1 \ldots N_l}
    \end{eqnarray*}

    Therefore:
    $$\Gamma'' \vdash^0 M N_1 \ldots N_l : (\overline{B_2} \, | \, \cdots \, | \,
            \overline{B_m} \, | \, o)$$

\item $\rulename{app^+}$  The side-condition of the rule $\rulename{app^+}$ ensures that the first premise
 satisfies $P_0$. The conclusion of the rule has the same order as the first premise
 therefore the conclusion also satisfies $P_0$.
\end{itemize}
\end{proof}


\begin{lem}
\label{lem:prooftree01only} If $\Gamma \vdash^{0} M : T$ or $\Gamma
\vdash^{-1} M : T$ then there is a valid proof tree showing $\Gamma
\vdash M : T$ such that all the judgments appearing in the proof
tree satisfy either $P_0$ or $P_{-1}$.
\end{lem}


\begin{proof}
Since $P_{-1}$ implies $P_0$, w.l.o.g. we can assume that the
judgment $\Gamma \vdash M : T$ satisfies $P_{-1}$. We show that
there is a proof tree for $\Gamma \vdash M : T$ where all the nodes
of the tree satisfy $P_0$ or $P_{-1}$. We proceed by structural
induction and case analysis on the last rule used to show $\Gamma
\vdash M : T$:
\begin{itemize}
\item Axiom $\rulename{\Sigma\mbox{\textbf{-const}}}$: the context is empty therefore the sequent satisfies $P_{-1}$.

\item Axiom $\rulename{var}$: the context contains only the variable itself therefore the sequent satisfies $P_0$.

\item Rule $\rulename{wk}$: The premise is $\Delta \vdash M : T$ with $\Delta \subset \Gamma$. Since
$\Gamma \vdash M : T$ satisfies $P_{-1}$ and $\Delta \subset \Gamma$ the premise must also satisfy $P_{-1}$. We can conclude using the
induction hypothesis.

\item Rule $\rulename{perm}$: By the induction hypothesis.


\item Rule $\rulename{abs}$: the second premise of the rule guarantees that the first
premise satisfies $P_{-1}$.

\item Rule $\rulename{app^+}$: The first premise has the same order as the
conclusion of the rule therefore the first premise satisfies
$P_0$. The side-condition of the rule $\rulename{app^+}$ ensures that all the other premises satisfy $P_0$.

\item Rule $\rulename{app}$:

$$ \rulename{app} \
    \rulef{
        { \Gamma \vdash M : (\overline{A} \, | B)
        \qquad
        \Gamma \vdash N_1 : A_1 \quad \cdots \quad \Gamma \vdash N_{l} : A_l \qquad l = |\overline{A}|
        }
    }
    {
        \Gamma \vdash^0 M N_1 \cdots N_{l} : B
    }
$$

Applying lemma \ref{lem:restriction} to the first premise we obtain:
\begin{equation}
 \Sigma \vdash^0 M : (\overline{A} \, | B) \label{eq:seq1}
\end{equation}
where $\Sigma = \{ z \in \Gamma \ | \ \ord{z} \geq \ord{(\overline{A} \, | B)} \} = \{ z \in \Gamma \ | \ \ord{z} \geq 1 + \ord{\overline{A}} \}.$

Applying lemma \ref{lem:restriction} to each of the remaining
premises gives  :
$$ \Sigma' \vdash^0 N_i : A_i \quad \mbox{ for all } i \in 1..p$$
where $\Sigma' = \{ z \in \Gamma \ | \ \ord{z} \geq \ord{A_i} =
\ord{\overline{A}} \} \supseteq \Sigma.$

If the inclusion $\Sigma \subseteq \Sigma'$ is strict then we apply the weakening rule to sequent (\ref{eq:seq1}):
$$ \rulef{\Sigma \vdash^0 M : (\overline{A} \, | B)}{\Sigma' \vdash^{-1} M : (\overline{A} \, | B)} \rulename{wk} $$

Finally, we obtain the following proof tree:
$$  \rulef{
        \rulef{
            { \Sigma' \vdash^{-1} M : (\overline{A} \, | B)
            \qquad
            \Sigma' \vdash^0 N_1 : A_1 \quad \cdots \quad \Sigma' \vdash^0 N_{l} : A_l \qquad l = |\overline{A}|
            }
        }
        {
            \Sigma' \vdash^0 M N_1 \cdots N_{l} : B
        } \rulename{app}
    }
    {
         \Gamma \vdash^0 M N_1 \cdots N_{l} : B
    } \rulename{wk}
$$

where the last weakening rules is applied only if the inclusion $\Sigma' \subseteq \Gamma$ is strict.

We can now conclude by applying the induction hypothesis on the
sequents $\Sigma' \vdash^{-1} M$, $\Sigma' \vdash^0 N_1$, \ldots,
$\Sigma' \vdash^0 N_l$ .
\end{itemize}
\end{proof}

\subsubsection{An Alternative Definition of the Homogeneous Safe $\lambda$-Calculus}

Using the observations that we have just made, we will now derive
new rules for the safe $\lambda$-calculus with homogeneous type. We
want a system of rules generating sequents that satisfy $P_0$. Also,
it must be able to generate intermediate sequents that do not
necessarily satisfy $P_0$ provided that they can be used to produce
\emph{in fine} terms satisfying $P_0$.

Because of the lemma \ref{lem:prooftree01only}, we know that the
only necessary intermediate sequents are those that either satisfy
$P_0$ or $P_{-1}$. Hence, we can assume by default that premises of
the rules all satisfy $P_{-1}$ at least.

First we define an additional rule expressing the fact that $P_0$
implies $P_{-1}$:
$$ \rulename{seq} \  \rulef{\Gamma \vdash^{0} M : A}{\Gamma \vdash^{-1} M : A} $$

The weakening rule can be rewritten as follows:
$$ \rulename{wk^{0}} \   \rulef{\Gamma \vdash^{0} M : A}{\Gamma , x : B \vdash^{0} M : A} \quad \ord{B} \geq \ord{A} $$
$$ \rulename{wk^{-1}} \   \rulef{\Gamma \vdash^{-1} M : A}{\Gamma , x : B \vdash^{-1} M : A} \quad \ord{B} \geq \ord{A} -1$$

Because of the context reduction lemma, any sequent satisfying $P_{-1}$ can be obtained
by applying the weakening rule $\rulename{wk^{-1}}$ or the rule $\rulename{seq}$ to another sequent
satisfying $P_0$. Therefore, with the exception of these two rules, we only need to use rules whose conclusion sequents satisfy $P_0$:
\begin{itemize}
\item For the rules $\rulename{perm}$, $\rulename{const}$ and $\rulename{var}$, only the tagging of the sequents
changes:
$$ \rulename{var} \   \rulef{}{x : A\vdash^{0} x : A}
\qquad
\rulename{\Sigma\mbox{\textbf{-const}}}  \  \rulef{}{\vdash^0 b : A} \ b:A \in \Sigma
$$

$$
  \rulename{perm} \  \rulef{
      { \Gamma \vdash^0 M:B \qquad \sigma(\Gamma)  } \hbox{ homogeneous}
    }
      { \sigma(\Gamma) \vdash^0 M : B }
$$

\item $\rulename{abs}$ The abstraction rule has a side condition
expressing the fact that the premise satisfies $P_0$ or $P_{-1}$. Since this is always true for sequents
generated by our new system of rules, we can drop the side condition:
$$ \rulename{abs} \   \rulef{\Gamma | \overline{x} : \overline{A} \vdash^{-1} M : B}
                                   {\Gamma  \vdash^{0} \lambda \overline{x} : \overline{A} . M : (\overline{A},B)}$$


\item $\rulename{app}$ The application rule has the following form:
$$ \rulename{app} \
    \rulef{
        { \Gamma \vdash^{-1} M : (\overline{A} \, | B)
        \qquad
        \Gamma \vdash^{-1} N_1 : A_1 \quad \cdots \quad \Gamma \vdash^{-1} N_{l} : A_l \qquad l = |\overline{A}|
        }
    }
    {
        \Gamma \vdash^0 M N_1 \cdots N_{l} : B
    }
$$

Since the first premise satisfies $P_{-1}$, by property \ref{proper:safe_basic_prop}(ii) we have:
$$\forall z \in \Gamma : \ord{z} \geq 1 + \ord{\overline{A}} -1 = \ord{\overline{A}} = \ord{\overline{N}}$$
Hence, all the sequents of the premises but the first must satisfy $P_0$. The rule (app) is therefore given by:
$$ \rulename{app} \
    \rulef{
        { \Gamma \vdash^{-1} M : (\overline{A} \, | B)
        \qquad
        \Gamma \vdash^0 N_1 : A_1 \quad \cdots \quad \Gamma \vdash^0 N_{l} : A_l \qquad l = |\overline{A}|
        }
    }{
        \Gamma \vdash^0 M N_1 \cdots N_{l} : B
      }
$$

\item For the application rule $\rulename{app^+}$, the type of the sequent in the first premise has the same order
as the type of the conclusion premise, and since the conclusion
satisfies $P_0$, the first premise must also satisfy $P_0$. The
side-condition implies that all the other sequents in the premise
satisfy $P_0$. Moreover since the first premise satisfies $P_0$, the
side-condition must hold. Hence the rule becomes:
$$ \rulename{app^+} \
    \rulef{
        \Gamma \vdash^0 M : (\overline{B_1} \, | \, \cdots \, | \, \overline{B_m} \, | \, o) \qquad
        \Gamma \vdash^0 N_1 : B_{11} \quad \cdots \quad \Gamma \vdash^0 N_{l} : B_{1l} \qquad l < |\overline{B_1}|
    }
    {
        \Gamma \vdash^0 M N_1 \cdots N_{l} : (\overline{B} \, | \, \cdots \, | \, \overline{B_m} \, | \, o)
    }
$$
where $\overline{B_1} = B_{11}, \ldots, B_{1l},\overline{B}$.
Clearly, this rule can be equivalently stated as:
$$ \rulef{\Gamma \vdash^0 M : A\rightarrow B
                                        \qquad \Gamma \vdash^{0} N : A
                                   }
                                   {\Gamma  \vdash^{0} M N : B}$$
\end{itemize}

The full set of rules is given in table \ref{tab:homosafelmd_rules_refined}.

\begin{table}[htbp]
$$  \rulename{perm} \
    \rulef{
      { \Gamma \vdash^0 M:B \qquad \sigma(\Gamma)  } \hbox{ homogeneous}
    }
    { \sigma(\Gamma) \vdash^0 M : B
    }
\qquad
\rulename{seq} \  \rulef{\Gamma \vdash^{0} M : A}{\Gamma \vdash^{-1} M : A}
$$

$$
\rulename{\Sigma\mbox{\textbf{-const}}} \  \rulef{}{\vdash^0 b : A}\ b:A \in \Sigma
\qquad
 \rulename{var} \   \rulef{}{x : A\vdash^{0} x : A} $$

$$ \rulename{wk^{0}} \   \rulef{\Gamma \vdash^{0} M : A}{\Gamma , x : B \vdash^{0} M : A} \quad \ord{B} \geq \ord{A} $$

$$ \rulename{wk^{-1}} \   \rulef{\Gamma \vdash^{-1} M : A}{\Gamma , x : B \vdash^{-1} M : A} \quad \ord{B} \geq \ord{A} -1$$


$$ \rulename{app} \
    \rulef
        {   \Gamma \vdash^{-1} M : (\overline{A} \, | B)
            \qquad
            \Gamma \vdash^0 N_1 : A_1 \quad \cdots \quad \Gamma \vdash^0 N_{l} : A_l \qquad l = |\overline{A}|
        }
        {
            \Gamma \vdash^0 M N_1 \cdots N_{l} : B
        }
$$

$$ \rulename{app^+} \   \rulef{\Gamma \vdash^0 M : A\rightarrow B
                                        \qquad \Gamma \vdash^{0} N : A
                                   }
                                   {\Gamma  \vdash^{0} M N : B}$$

$$ \rulename{abs} \   \rulef{\Gamma| \overline{x} : \overline{A} \vdash^{-1} M : B}
                                   {\Gamma  \vdash^{0} \lambda \overline{x} : \overline{A} . M : (\overline{A}|B)}$$


where $\Gamma| \overline{x} : \overline{A}$ means that the lowest type-partition of the context is
$\overline{x} : \overline{A}$.
\caption{Alternative rules for the homogeneous safe lambda calculus}
\label{tab:homosafelmd_rules_refined}
\end{table}
%%%

\clearpage

\section{Safe $\lambda$-Calculus without the Homogeneity Constraint}
\label{sec:safe_nonhomog}


In section \ref{sec:safe_homog}, we have presented a version of the
safe lambda calculus where types are required to be homogeneous. We
now give a more general version of the safe simply-typed
$\lambda$-calculus where type homogeneity is not required.

\subsection{Rules}

We use a set of sequents of the form $\Gamma \vdash M : A$ where
$\Gamma$ is the context of the term and $A$ is its type. Let
$\Sigma$ be a set of higher-order constants. We call safe terms any
simply-typed lambda term that is typable within the following system
of formation rules:
$$ \rulename{var} \   \rulef{}{x : A\vdash x : A}
\qquad  \rulename{const} \   \rulef{}{\vdash f : A} \quad f \in \Sigma
\qquad  \rulename{wk} \   \rulef{\Gamma \vdash M : A}{\Delta \vdash M : A} \quad \Gamma \subset \Delta$$

$$ \rulename{app} \  \rulef{\Gamma \vdash M : (A,\ldots,A_l,B)
                                        \qquad \Gamma \vdash N_1 : A_1
                                        \quad \ldots \quad \Gamma \vdash N_l : A_l  }
                                   {\Gamma  \vdash M N_1 \ldots N_l : B}
                                    \quad
                                   \forall y \in \Gamma : \ord{y} \geq \ord{B}$$

$$ \rulename{abs} \   \rulef{\Gamma \union \overline{x} : \overline{A} \vdash M : B}
                                   {\Gamma  \vdash \lambda \overline{x} : \overline{A} . M : (\overline{A},B)} \qquad
                                   \forall y \in \Gamma : \ord{y} \geq \ord{\overline{A},B}$$


Remark:
\begin{itemize}
\item $(\overline{A},B)$ denotes the type $(A_1,A_2, \ldots, A_n, B)$;
\item all the types appearing in the rule are not required to be homogeneous (for instance
it is possible to have $\ord{A_l} < \ord{B}$ in rule $\rulename{app}$) ;
\item the environment $\Gamma \union \overline{x}:\overline{A}$ is not stratified, in particular, variables in $\overline{x}$ do not necessarily have the same order;
\item in the abstraction rule, the side-condition imposes that at least all variables of the lowest order
in the context are abstracted. Variables of greater order can also be
abstracted together with the lowest order variables and, in contrast to
the homogeneous safe lambda calculus, there is no constraint on the
order in which these variables are abstracted;
\end{itemize}

\begin{exmp}
For $x:o$, $f:(o,o)$ and $\varphi:((o,o),o)$ the term $$\vdash \lambda x f \varphi .
\varphi : (o , (o, o) , ((o,o),o) , (o,o),o)$$ is
a valid safe term that is not homogeneously typed.
\end{exmp}

\begin{exmp}
For $x:o$, $g:(o,(o,o),o)$, the term $\vdash \lambda g x . g x$ is unsafe and not homogeneously typed
and the term $\lambda g x . g x (\lambda x . x)$ is safe and not homogeneously typed.
\end{exmp}

Side-remark: safety is preserved by full $\eta$-expansion. Indeed,
consider the safe term $\Gamma \vdash M:(A_1,\ldots,A_l,o)$ where
$(A_1,\ldots,A_l,o)$ is not necessarily homogeneous. Its full $\eta$-expansion
is $\lambda x_1 .. x_l . M x_1 \dots x_l$ for some variables
$x_1:A_1, \ldots, x_l:A_l$ fresh in $M$. For all $i \in 1..l$ we
have $\Gamma, \Sigma \vdash x_i :A_i$ where $\Sigma = \{ x_1:A_1,
\cdots x_l :A_l \}$. Applying $\rulename{app}$ we obtain $\Gamma,
\Sigma \vdash M x_1 \ldots x_l$ and by the (abs) rule we get
$$\Gamma \vdash \lambda x_1:A_1 \ldots x_l:A_l .M x_1 \ldots x_l.$$

\begin{lem}[Context reduction]
\label{lem:nonhomosafe_basic_prop}
If $\Gamma \vdash M : B$ is a valid judgment then
\begin{enumerate}
\item $fv(M) \vdash M : B$
\item every variable in $\Gamma$ \emph{occurring free in $M$} has order at
least $ord(M)$.
\end{enumerate}
where $fv(M)$ denotes the context constituted of the free variables occurring in $M$.
\end{lem}
\begin{proof}
(i) Suppose that some variable $x$ in $\Gamma$ does not occur free
in $M$, then necessarily $x$ has been introduced in the context
using the weakening rule. Hence $\Gamma\setminus \{ x \} \vdash M$
must also be typable. (ii) An easy structural induction.
\end{proof}

The converse of this lemma is not true: consider the simply-typed
term $\lambda y z. (\lambda x . y ) z$ with $x,y,z:o$. This term is
closed therefore it satisfies property (i) and (ii) of lemma
\ref{lem:nonhomosafe_basic_prop}. However it is not typable by the
rules of the safe lambda-calculus since the subterm $\lambda x .y$
is not safe.

\subsection{Substitution in the safe lambda calculus}

The traditional notion of substitution, on which the
$\lambda$-calculus is based, is defined as follows:
\begin{dfn}[Substitution]
\label{dfn:subst}
\begin{eqnarray*}
c \subst{t}{x} &=& c \quad \mbox{where $c$ is a $\Sigma$-constant},\\
x \subst{t}{x} &=& t\\
 y\subst{t}{x} &=& y \quad \mbox{for } x \not \neq y,\\
(M_1 M_2) \subst{t}{x} &=& (M_1 \subst{t}{x}) (M_2 \subst{t}{x})\\
(\lambda x . M) \subst{t}{x} &=& \lambda x . M\\
(\lambda y . M) \subst{t}{x} &=& \lambda z . M \subst{z}{y}
\subst{t}{x} \mbox{where $z$ is a fresh variable and $x\not = y$}.
\end{eqnarray*}
\end{dfn}

In the setting of the safe lambda calculus, the notion of
substitution can be simplified. Indeed, similarly to what we observe
in the homogeneous safe $\lambda$-calculus, we remark that for safe
$\lambda$-terms there is no need to rename variables when performing
substitution:

\begin{lem}[No variable capture lemma]
\label{lem:noclash} There is no variable capture when performing
substitution on a safe term.
\end{lem}

This is the counterpart of lemma \ref{lem:homog_nocapture}. The
proof (which does not rely on homogeneity) is the same.
Consequently, in the safe lambda calculus setting, we can omit to
rename variable when performing substitution. The equation
$$(\lambda x . M) \subst{t}{y} = \lambda z . M \subst{z}{x}
\subst{t}{y} \mbox{where $z$ is a fresh variable}$$ becomes
$$(\lambda x . M) \subst{t}{y} = \lambda x . M \subst{t}{y}.$$

Unfortunately, this notion of substitution is still not adequate for
the purpose of the safe simply-typed lambda calculus. The problem is
that performing a single $\beta$-reduction on a safe term will not
necessarily produce another safe term.

The solution consists in reducing several consecutive $\beta$-redex
at the same time until we obtain a safe term. To achieve this, we
introduce the \emph{simultaneous substitution}, a generalization of
the standard substitution given in definition \ref{dfn:subst}.

\begin{dfn}[Simultaneous substitution]
\label{dnf:simsubst}
 The expression $\subst{\overline{N}}{\overline{x}}$ is an abbreviation for $\subst{N_1 \ldots N_n}{x_1
\ldots x_n}$:
\begin{eqnarray*}
c \subst{\overline{N}}{\overline{x}} &=& c \quad \mbox{where $c$ is a $\Sigma$-constant},\\
x_i \subst{\overline{N}}{\overline{x}} &=& N_i\\
 y \subst{\overline{N}}{\overline{x}} &=& y \quad \mbox{ if } y \not \neq x_i \mbox{ for all } i,\\
(M N) \subst{\overline{N}}{\overline{x}} &=& (M \subst{\overline{N}}{\overline{x}}) (N \subst{\overline{N}}{\overline{x}}) \\
(\lambda x_i . M) \subst{\overline{N}}{\overline{x}} &=& \lambda x_i
. M
\subst{N_1 \ldots N_{i-1} N_{i+1}\ldots N_n}{x_1 \ldots x_{i-1} x_{i+1}\ldots x_n} \\
(\lambda y . M)
\subst{\overline{N}}{\overline{x}} &=& \lambda z . M \subst{z}{y} \subst{\overline{N}}{\overline{x}} \\
&& \mbox{where $z$ is a fresh variables and } y \neq x_i \mbox{ for
all } i.
\end{eqnarray*}
\end{dfn}

In general, variable capture should be avoided, this explains why
the definition of simultaneous substitution uses auxiliary fresh
variables. However in the current setting, lemma \ref{lem:noclash}
can clearly be transposed to the simultaneous substitution,
therefore there is no need to rename variables.

The notion of substitution that we need is therefore the
\emph{capture-permitting simultaneous substitution} defined as
follows:

\begin{dfn}[Capture-permitting simultaneous substitution]
 We use the notation
$\subst{\overline{N}}{\overline{x}}$ for $\subst{N_1 \ldots N_n}{x_1
\ldots x_n}$:
\begin{eqnarray*}
c \subst{\overline{N}}{\overline{x}} &=& c \quad \mbox{where $c$ is a $\Sigma$-constant},\\
 x_i \subst{\overline{N}}{\overline{x}} &=& N_i\\
 y \subst{\overline{N}}{\overline{x}} &=& y \quad \mbox{where } x \not \neq y_i \mbox{ for all } i,\\
(M_1 M_2) \subst{\overline{N}}{\overline{x}} &=& (M_1 \subst{\overline{N}}{\overline{x}}) (M_2 \subst{\overline{N}}{\overline{x}})\\
(\lambda x_i . M) \subst{\overline{N}}{\overline{x}} &=& \lambda x_i
. M
\subst{N_1 \ldots N_{i-1} N_{i+1}\ldots N_n}{x_1 \ldots x_{i-1} x_{i+1}\ldots x_n} \\
(\lambda y . M) \subst{\overline{N}}{\overline{x}} &=& \lambda y . M
\subst{\overline{N}}{\overline{x}} \mbox{where $y \not = x_i$ for
all $i$}. \qquad \mathbf{(\star)}
\end{eqnarray*}
The symbol $\mathbf{(\star)}$ identifies the equation which has
changed compared to the previous definition.
\end{dfn}

\begin{lem}[Substitution preserves safety]
\label{lem:subst_preserve_i}
$$ \Gamma\union \overline{x} : \overline{A}\vdash M : T
\quad \mbox{and} \quad \Gamma \vdash N_k : B_k \mbox{, } k \in
1..n \qquad \mbox{ implies } \qquad \Gamma \vdash
M[\overline{N}/\overline{x}] : T$$
\end{lem}

\begin{proof}
Suppose that $\Gamma \union \overline{x}: \overline{A} \vdash M :T$ and
$\Gamma \vdash N_k : B_k$ for $k \in 1..n$.

We prove $\Gamma \vdash M[\overline{N}/\overline{x}]$ by induction
on the size of the proof tree of $\Gamma\union
\overline{x}:\overline{A} \vdash M : T$ and by case analysis on the
last rule used. We only give the proof for the abstraction case. If
$\Gamma \union \overline{x}:\overline{A} \vdash \lambda \overline{y}
: \overline{C}. P : (\overline{C}|D)$ where $\Gamma\union
\overline{x}:\overline{A}\union \overline{y}:\overline{C} \vdash P :
D$, then by the induction hypothesis $\Gamma\union
\overline{y}:\overline{C} \vdash P\subst{\overline{N}}{\overline{x}}
: D$. Applying the rule $\rulename{abs}$ gives $\Gamma \vdash
\lambda \overline{y}:\overline{C} . P
\subst{\overline{N}}{\overline{x}}$.
\end{proof}

\subsection{Safe-redex}
In the simply-typed lambda calculus a redex is a term of the form
$(\lambda x . M) N$. We generalize this definition to the safe
lambda calculus:
\begin{dfn}[Safe redex]
We call safe redex a term of the form $(\lambda \overline{x} . M)
N_1 \ldots N_l$ such that:
\begin{itemize}
\item $ \Gamma \vdash (\lambda \overline{x} . M) N_1 \ldots N_l $;
\item the variable $\overline{x}=x_1\ldots x_n$ are abstracted altogether by one occurrence of the rule $\rulename{abs}$ in the proof
tree;
\item the terms $(\lambda \overline{x} . M)$, $N_1$, $N_l$ are applied together at once using the $\rulename{app}$ rule :
$$   \rulef{
            \Sigma \vdash \lambda \overline{x} . M
            \quad
            \Sigma \vdash N_1         \quad \ldots \quad \Sigma \vdash N_l
    }
    {
       \Sigma \vdash (\lambda \overline{x} . L) N_1 \ldots N_l
    } (\mathbf{app})
$$
and consequently each $N_i$ is safe;
\end{itemize}
\end{dfn}

The relation $\beta_s$ is defined exactly the same way as in the homogeneous safe $\lambda$-calculus. The safe $\beta$-reduction $\betasred$ is defined as the closure of $\beta_s$ by
compatibility with the formation rules of the safe
$\lambda$-calculus.  It is straightforward to show, as we did for the homogeneous safe $\lambda$-calculus, that $\betasred \subset \betaredtr$.


\begin{lem}
\label{lem:safereduction} A safe redex reduces to a safe term.
\end{lem}

This lemma, which is a consequence of lemma
\ref{lem:subst_preserve_i}, is the counterpart of lemma
\ref{lem:homoh_safered_preserve_safety} in the homogeneous safe
lambda calculus. Their proofs are identical.


\subsection{Particular case of homogeneously-safe lambda terms}

In this section, we derive a new set of rules by adding the type-homogeneity restriction to the non-homogenous safe lambda calculus.

We recall the definition of type-homogeneity from section
\ref{sec:safe_homog}: a type $(A_1, A_2, \ldots A_n, o)$ is said to
be homogeneous whenever $\ord{A_1} \geq \ord{A_2} \geq \ldots \geq
\ord{A_n}$ and each of the $A_i$ is homogeneous. A term is said to
be homogeneous if its type is homogeneous.

We now impose type-homogeneity to all the sequents present in the
rules of the safe $\lambda$-calculus: we say that a term is
\emph{homogeneously-safe} if there is a proof tree showing its
safety in which all sequents are of homogenous type. Consequently a
homogeneously-safe term is safe and has an homogenous type.

We say that $\Gamma \vdash M : A$ verifies $P_i$ for $i \in \zset$
if all the variables in $\Gamma$ have order at least $\ord{A}+i$.
Lemma \ref{lem:nonhomosafe_basic_prop} can then be restated as
follows:
\begin{lem}[Context reduction]
\label{lem:context_reduction} If $\Gamma \vdash M : A$ then the sequent $fv(M) \vdash M : A$ is valid and satisfies $P_0$.
\end{lem}


We now prove that if we impose the homogeneity of types, the set of
rules of the non-homogenous safe $\lambda$-calculus and the rules of
table \ref{tab:homosafelmd_rules_refined} are equivalent.  We recall
that in the system of rules of table
\ref{tab:homosafelmd_rules_refined}, if the sequent $\Gamma
\vdash^{i} M : A$ is valid for some $i \in \zset$ then all the
variables in $\Gamma$ have orders at least $\ord{A}+i$.

\begin{prop}[Homogeneity restriction]
\label{prop:nonhomogsafe_homog_restriction}
Let $k \in \{ 0, -1 \}$. The sequent $\Gamma \vdash M : A$ is valid, homogeneously-safe and satisfies $P_k$
if and only if the sequent $\Gamma \vdash^k M : A$ is valid in the system of rules of table \ref{tab:homosafelmd_rules_refined}.
\end{prop}

\begin{proof}
\emph{If}: An easy induction by case analysis on the last rule used to derive $\Gamma \vdash^0 M : A$.

\emph{Only if}:
Consider an homogeneously-safe term $\Gamma \vdash S : T$ satisfying $P_0$.
We proceed by induction and case analysis on the last rule used to derive $\Gamma \vdash S : T$.
We only give the details for the application and abstraction
case:
\begin{itemize}
\item \textbf{Abstraction.} We recall the abstraction rule:
$$ \rulename{abs} \quad  \rulef{\Gamma \union \overline{x} : \overline{A} \vdash M : B}
                                   {\Gamma  \vdash \lambda \overline{x} : \overline{A} . M : (\overline{A},B)} \qquad
                                   \forall y \in \Gamma : \ord{y} \geq \ord{\overline{A},B}$$

Type homogeneity requires that for all $i$: $\ord{x_i} = \ord{A_i} \geq
\ord{B} -1$. Therefore the premise of the rule verifies $P_{-1}$. Using the induction hypothesis we have:
\begin{equation}
\Gamma \union \overline{x} : \overline{A} \vdash^{-1} M : B. \label{eq:prop:nonhomogsafe_homog_restriction:abs1}
\end{equation}

We now partition the context $\Gamma$ according to the order of
the variables. The partitions are written in decreasing order of
type order. The notation $\Gamma | \overline{x}:\overline{A}$ means
that $\overline{x}:\overline{A}$ is the lowest partition of the
context.
We also use the notation $(\overline{A}|B)$ to denote the
homogeneous type $(A_1, A_2, \ldots A_n, B)$ where $\ord{A_1} =
\ord{A_2} =  \ldots \ord{A_n} \geq \ord{B} -1$.


Suppose that we abstract a single variable $x$, then in order to
respect the side condition, we need to abstract all variables of
order less or equal to $\ord{x}$. In particular we need to abstract
the partition of the order of $x$. Moreover to respect type
homogeneity, we need to abstract variables of the lowest order
first.

Hence $\overline{x}$ must contain at least the lowest variable
partition (all the variables of the lowest order). If $\overline{x}$
contains variables of different order, then the instance of the
abstraction rule can be replaced by consecutive instances of the
abstraction rule, one for each of the different variable order in
$\overline{x}$. Therefore, without loss of generality, we can assume
that $\overline{x}$ only contains the lowest partition, that is to
say, $\overline{x}$ \emph{is} the lowest partition.

The sequent \ref{eq:prop:nonhomogsafe_homog_restriction:abs1} therefore becomes:
$$\Gamma | \overline{x} : \overline{A} \vdash^{-1} M : B.$$

We conclude by applying the abstraction rule of table
\ref{tab:homosafelmd_rules_refined}:
$$ \rulename{abs} \quad  \rulef{\Gamma| \overline{x} : \overline{A} \vdash^{-1} M : B}
                                   {\Gamma  \vdash^{0} \lambda \overline{x} : \overline{A} . M : (\overline{A}|B)}$$



\item \textbf{Application.} We recall the application rule:
$$ \rulename{app} \  \rulef{\Gamma \vdash M : (A,\ldots,A_l,B)
                                        \qquad \Gamma \vdash N_1 : A_1
                                        \quad \ldots \quad \Gamma \vdash N_l : A_l  }
                                   {\Gamma  \vdash M N_1 \ldots N_l : B}
                                    \quad
                                   \forall y \in \Gamma : \ord{y} \geq \ord{B}$$

The term in the conclusion is homogeneously safe therefore the term in the first premise must be of homogeneous \
type. This implies that $\ord{A_1} \geq \ldots \geq \ord{A_l}
\geq \ord{B} - 1$.
Furthermore, we can make the assumption that $\ord{A_1} = \ldots = \ord{A_l} = \ord{\overline{A}}$
(it is always possible to replace an instance of the application rule
by several consecutive instances of this kind).

By lemma \ref{lem:context_reduction}, we have for all $i \in 1..l$:
$$fv(N_i) \vdash N_i : A_i \mbox{ is valid and satisfies } P_0.$$

Let $\Sigma = \Union_{i=1..p} fv(N_i)$. Since $\ord{A_1} = \ldots = \ord{A_l}$, by applying the weakening rule we get for all $i\in 1..p$:
$$\Sigma \vdash N_i : A_i \mbox{ is valid and satisfies } P_0.$$


Applying lemma \ref{lem:context_reduction} to the term $M$ we have:
$$fv(M) \vdash M : (A_1,\ldots,A_l,B) \mbox{ is valid and satisfies } P_0.$$

The weakening rule $\rulename{wk}$ then gives:
$fv(M) \union \Sigma \vdash M : (A_1,\ldots,A_l,B)$.
Since $\Sigma \vdash N_i : A_i$ satisfies $P_0$, for any
$z \in \Sigma$ we have $\ord{z} \geq \ord{A_i} = \ord{(A_1,\ldots,A_l,B)} - 1$.
Hence:
\begin{equation}
fv(M) \union \Sigma \vdash M : (A_1,\ldots,A_l,B) \mbox{ is valid
and satisfies } P_{-1}
\label{eq:prop:nonhomogsafe_homog_restriction:m}.
\end{equation}

Similarly, for all $i \in 1..p$, the weakening rule gives $fv(M) \union \Sigma \vdash N_i : A_i$.
Since $fv(M) \vdash M : (A_1,\ldots,A_l,B)$ satisfies $P_0$,
for any $z \in fv(M)$ we have $\ord{z} \geq \ord{M} \geq \ord{A_i}$. Hence:
\begin{equation}
fv(M) \union \Sigma \vdash N_i : A_i \mbox{ is valid and satisfies }
P_0 \label{eq:prop:nonhomogsafe_homog_restriction:ni}.
\end{equation}

Let us define the context $\Sigma' = fv(M) \union \Sigma$. Using the induction hypothesis on equation
\ref{eq:prop:nonhomogsafe_homog_restriction:m} and \ref{eq:prop:nonhomogsafe_homog_restriction:ni} we have:
$$
\Sigma' \vdash^{-1} M : (A_1,\ldots,A_l,B) \qquad \mbox{and} \qquad
\Sigma' \vdash^0 N_i : A_i \mbox{ for all } i \in 1..l.
$$


We consider the following two sub-cases:
\begin{itemize}
\item If $A_1, \ldots, A_l$ forms a type partition then we can apply
rule $\rulename{app}$ of table \ref{tab:homosafelmd_rules_refined}:

$$ \rulef{\Sigma' \vdash^{-1} M : \overline{A} | B
                                        \qquad \Sigma' \vdash^{0} N_1 :
                                        A_1
                                        \quad \ldots \quad \Sigma' \vdash^{0} N_l :
                                        A_l
                                        \quad l = |\overline{A}|
                                        }
                                   {\Sigma'  \vdash^{0} M N_1 \ldots N_l : B} \quad  \rulename{app}
$$
where $\overline{A} = A_1, \ldots, A_l$.

\item  Suppose that $A_1, \ldots, A_l$ does not form a type partition, then we
have $$\ord{A_1} = \ldots = \ord{A_l} = \ord{B} - 1.$$

The side condition in the original instance of the application rule
says that for any variable $y$ in $\Gamma$ we have
$$\ord{y} \geq \ord{B} = 1 + \ord{A_l} = \ord{(A_1,\ldots, A_l,B)} = \ord{M}.$$

In particular the variables in $\Sigma' \subseteq \Gamma$ are of order greater than $\ord{M}$ and consequently
the sequent $\Sigma' \vdash M : (A,\ldots,A_l,B)$ verifies $P_0$. The induction hypothesis then gives:
$$\Sigma' \vdash^0 M : (A,\ldots,A_l,B)$$

By using $l$ consecutive instances of the rules $\rulename{app^+}$ from table \ref{tab:homosafelmd_rules_refined} we get:
$$  \rulef{ \rulef{ \rulef{ \Sigma' \vdash^0 M : (A_1,\ldots, A_l,B)
                    \qquad \Sigma'\vdash^{0} N_1 : A_1
                    }{ \Sigma' \vdash^0 M N_1 : (A_2,\ldots, A_l,B)} \quad \rulename{app^+}
          }
          { \vdots
          }
          \quad \rulename{app^+}
       }
       { \Sigma'  \vdash^{0} M N_1 \ldots N_l : B } \quad \rulename{app^+}
$$
\end{itemize}

In both cases we have proved that $\Sigma'  \vdash^{0} M N_1 \ldots N_l : B$ is a valid sequent.

Clearly $\Sigma' \subseteq \Gamma$ since $fv(M) \subseteq \Gamma$ and $\Sigma' = \Union_{i\in1..l} fv(N_i) \subseteq \Gamma$.
Suppose that $\Sigma' = \Gamma$ then the proof is done.
Suppose that $\Sigma' \subset \Gamma$, then the side condition in the original instance of the application rule says that all
the variables in $\Gamma$ have order
greater or equal to $\ord{B}$, we can therefore apply the weakening rule $\rulename{wk^0}$
of table \ref{tab:homosafelmd_rules_refined} exactly $|\Gamma\setminus \Sigma'|$ times and get:
$$ \rulef{\Sigma'  \vdash^{0} M N_1 \ldots N_l : B}
                                   {\Gamma  \vdash^{0} M N_1 \ldots N_l : B} \quad
                                   \rulename{wk^0}.
$$


\end{itemize}
\end{proof}


\subsection{Examples}
\subsubsection{Example 1}
Let $f,g:o\rightarrow o$, $x,y:o\rightarrow o$, $\Gamma =
g:o\rightarrow o$ and $\Gamma' = g:o\rightarrow o, y:o$. The term
$(\lambda f x . x) g y $ is safe. One possible proof tree is:
$$ \rulef{
        \rulef{
            \rulef{
                \rulef{\vdots}{\Gamma \vdash \lambda f x. x}      \qquad \axiomf{\Gamma \vdash g} }
            {\Gamma \vdash (\lambda f x. x) g} \rulename{app}
        }
        { \Gamma' \vdash (\lambda f x. x) g } \rulename{wk}
        \qquad \axiomf{\Gamma' \vdash y}
    }
    { \Gamma' \vdash (\lambda f x. x) g y } \rulename{app}
$$
Here is another proof for the same judgment:
$$ \rulef{  \rulef{ \rulef{\vdots}{\Gamma \vdash \lambda f x. x} }{\Gamma' \vdash \lambda f x. x} \rulename{wk}    \qquad \rulef{}{\Gamma' \vdash g} \qquad \rulef{}{\Gamma' \vdash y}}
    {\Gamma' \vdash (\lambda f x. x) g y } \rulename{app}$$

We see on this particular example that there may exist different
proof trees deriving the same judgment.

\subsubsection{Example 2 - Damien Sereni's SCT counter-example}
In \cite{serenistypesct05}, the following counter-example is given
to show that not all simply-typed terms are size-change terminating
(see \cite{jones01} for a definition of size-change termination):

$$ E =  (\lambda a . a (\lambda b . a (\lambda c d .d))) (\lambda e . e (\lambda f .f))$$
where:
\begin{eqnarray*}
a &:& \sigma \typear \mu \typear \mu \\
b &:& \tau \typear \tau \\
c &:& \tau \typear \tau \\
d &:& \mu \\
e &:& \sigma = (\tau \typear \tau) \typear \mu \typear \mu \\
f &:& \tau
\end{eqnarray*}
and $\tau$, $\mu$ and $\sigma$ are type variables.

This example shows that the rules of the safe $\lambda$-calculus
without the homogeneity restriction generates a class of terms that
strictly contains the class of terms generated by the rules of the
homogeneous safe $\lambda$-calculus of section \ref{sec:safe_homog}.

Indeed, for $E$ to be an homogeneous safe lambda term, in the sense
of the rules of section \ref{sec:safe_homog}, $\tau$ and $\mu$ must
be homogeneous types and the variables $a,b,c,d,e,f$ must be
homogeneously typed. This implies that $ \ord{\tau} \geq
\ord{\mu}-1$. Conversely, if this condition is met then $\vdash E :
\mu \typear \mu$ is a valid judgement of the \emph{homogeneous} safe $\lambda$-calculus.

In the safe $\lambda$-calculus \emph{without} the homogeneity
constraint, however, the judgement $\vdash E : \mu \typear \mu$ is
always valid whatever the types $\mu$ and $\tau$ are.

% \clearpage
\section{Non-homogeneous safe $\lambda$-calculus - VERSION A}

In section \ref{sec:safe_alt}, we have presented a safe lambda calculus in the setting of homogeneous types.
In this section, we try to give a general notion of safety for the simply typed $\lambda$-calculus.
The rules we give here do not assume homogeneity of the types.

We will call safe terms the simply typed lambda terms that are typable within the following system of formation rules:

\subsection{Rules}

 We use a set of sequents of the form $\Gamma \vdash^{i} M :
A$ where the meaning is ``variables in $\Gamma$ have orders at least
$\ord{A}+i$'' where $i \leq 0$. The following set of rules are
defined for $i \leq 0$:

$$ \mathbf{(seq^i_\delta)} \quad \rulef{\Gamma \vdash^{i} M : A}{\Gamma \vdash^{i-\delta} M : A} \quad i \leq 0, \delta > 0  $$

$$ \mathbf{(var)} \quad  \rulef{}{x : A\vdash^{0} x : A} $$

$$ \mathbf{(wk^i)} \quad  \rulef{\Gamma \vdash^i M : A}{\Gamma , x : B \vdash^i M : A} \quad \ord{B} \geq \ord{A} + i $$

$$ \mathbf{(app^i)} \quad  \rulef{\Gamma \vdash^i M : A\rightarrow B
                                        \qquad \Gamma \vdash^{0} N : A}
                                   {\Gamma  \vdash^{\min(i+\delta,0)} M N : B}
                                    \qquad
                                   \delta = \max\left(0, 1 + \ord{A} - \ord{B}\right)$$

$$ \mathbf{(abs^i)} \quad  \rulef{\Gamma, \overline{x} : \overline{A} \vdash^{i} M : B}
                                   {\Gamma  \vdash^{0} \lambda \overline{x} : \overline{A} . M : (\overline{A},B)} \qquad
%                                   \left\{
%                                     \begin{array}{ll}
                                       \forall y \in \Gamma : \ord{y} \geq \ord{\overline{A},B}\\
%                                       \forall y \in fv(M) : \ord{y} \geq \ord{B}
%                                     \end{array}
%                                   \right.
                                   $$


Note that:
\begin{itemize}
\item $(\overline{A},B)$ denotes the type $(A_1,A_2, \ldots, A_n, B)$;
\item all the types appearing in the rule are not required to be homogeneous. For instance in the rule $\mathbf{(app^i)}$ for the type $A \rightarrow B$
it is not necessary that $\ord{A} \geq \ord{B}$;
\item the environment $\Gamma, \overline{x}$ is not stratified. In particular, variables in $\overline{x}$ do not necessarily have the same
order;
\item In the abstraction rule, the side-condition imposes that at least all the variable of the lowest order
in the context are abstracted. However other variables can also be
abstracted together with the lowest order variables. Moreover there
is not constraint on the order on which the variables are abstracted
(contrary to what happens in the homogeneous case);
\item The sequents $\Gamma \vdash^0 M$ are the \emph{safe terms} that we want to generate.
Other terms are only used as intermediate sequents in a proof tree.
\end{itemize}

\begin{exmp}
Suppose $x:o$, $f:o\rightarrow o$ and $\varphi:(o\rightarrow
o)\rightarrow o$ then the term $$\vdash^0 \lambda x f \varphi .
\varphi : o \rightarrow (o\rightarrow o) \rightarrow ((o\rightarrow
o)\rightarrow o) \rightarrow (o\rightarrow o)\rightarrow o$$ is
valid although its type is not homogeneous
\end{exmp}


\begin{lem}[Basic properties]
\label{va_lem:nonhomosafe_basic_prop} Suppose $\Gamma \vdash^i M : B$
is a valid judgment then every variable in $\Gamma$ has order at
least $ord(M)+i$.
\end{lem}
\begin{proof}
An easy induction. The step case for the application is: suppose
$\Gamma \vdash^{i+\delta} M N : B$ where $\Gamma \vdash^i M :
A\rightarrow B$. Then by induction we have $\forall y \in \Gamma :
\ord{y} \geq \ord{A\rightarrow B} + i = \max(1+\ord{A}, \ord{B})+i =
\delta + \ord{B} + i \geq \min(i+\delta,0) + \ord{B}$.
\end{proof}

\subsection{Substitution in the safe lambda calculus}

The traditional notion of substitution, on which the $\lambda$-calculus is based on, is the following one:
\begin{dfn}[Substitution]
\label{va_dfn:subst}
\begin{eqnarray*}
c \subst{t}{x} &=& c \quad \mbox{where $c$ is a $\Sigma$-constant}\\
x \subst{t}{x} &=& t\\
 y\subst{t}{x} &=& y \quad \mbox{for } x \not \neq y,\\
(M_1 M_2) \subst{t}{x} &=& (M_1 \subst{t}{x}) (M_2 \subst{t}{x})\\
(\lambda x . M) \subst{t}{x} &=& \lambda x . M\\
(\lambda y . M) \subst{t}{x} &=& \lambda z . M \subst{z}{y}
\subst{t}{x} \mbox{where $z$ is a fresh variable and $x\not = y$}
\end{eqnarray*}
\end{dfn}

In the setting of the safe lambda calculus, the notion of substitution can be simplified.
Indeed, we remark that for safe $\lambda$-terms there is no need to rename variables
when performing substitution:

\begin{lem}[No variable capture lemma]
\label{va_lem:noclash}
There is no variable capture when performing substitution on a safe term.
\end{lem}
\begin{proof}
Suppose that a capture occurs during the substitution $M[N/\varphi]$
where $M$ and $N$ are safe. Then the following conditions must hold:
\begin{enumerate}
\item $\varphi:A, \Gamma \vdash^0 M$,
\item $\Gamma' \vdash^0 N$,
\item there is a subterm $\lambda \overline{x} . L$ in $M$ (where the abstraction is taken as wide as possible) such that:
\item $\varphi \in fv(\lambda \overline{x} . L)$ (and therefore $\varphi \in fv(L)$),
\item $x \in fv(N)$ for some $x \in \overline{x}$.
\end{enumerate}

By lemma \ref{va_lem:nonhomosafe_basic_prop} and (v) we have:

\begin{equation}
\ord{x} \geq \ord{N} = \ord{\varphi} \label{va_eq:xigeqphi}
\end{equation}

The abstraction $\lambda \overline{x} . L$ (taken as large as possible)
is a subterm of $M$, therefore there is a node $\Sigma \vdash^u \lambda \overline{x} . L$  for some $u$ in the
proof tree of $\varphi:A, \Gamma \vdash^0 M$.

There are only three kinds of rules that can produce an abstraction:
$\mathbf{(abs^i)}$, $\mathbf{(seq^i_\delta)}$ and $\mathbf{(wk^i)}$.
The only one that can introduce the abstraction is
$\mathbf{(abs^i)}$. Therefore the proof tree has the following form:
$$ \rulef{
    \rulef{
        \rulef{
            \rulef  {\ldots}
                   {\Sigma' \vdash^0 \lambda \overline{x} . L} \mathbf{(abs^i)}
        }
        {\ldots} r_1
    }
    {\vdots} r_2
    }
    { \Sigma \vdash^u \lambda \overline{x} . L } r_l
    \qquad \mbox{where for } j\in 1..l: r_j \in \{ \mathbf{(seq^i_\delta)},\ \mathbf{(wk^i)}\ |\ i \in \zset, \delta > 0 \}.
$$


Since $\varphi \in fv (L)$ we must have $\varphi \in \Sigma'$ and
since $\Sigma' \vdash^0 \lambda \overline{x} . L$, by lemma
\ref{va_lem:nonhomosafe_basic_prop} we have:
$$\ord{\varphi} \geq \ord{\lambda \overline{x} . L} \geq \max(1+ \ord{x}, \ord{L}) > \ord{x}$$

which contradicts equation (\ref{va_eq:xigeqphi}).
\end{proof}

Hence, in the safe lambda calculus setting, we can omit to
to rename variable when performing substitution. The equation
$$(\lambda x . M) \subst{t}{y} = \lambda z . M \subst{z}{x}
\subst{t}{y} \mbox{where $z$ is a fresh variable}$$
becomes
$$(\lambda x . M) \subst{t}{y} = \lambda x . M \subst{t}{y}$$



Unfortunately, this notion of substitution is still not adequate for
the purpose of the safe simply-typed lambda calculus. The problem is
that performing a single $\beta$-reduction on a safe term will not
necessarily produce another safe term.

To fix this problem, we need to be able to reduce several
consecutive $\beta$-redex at the same time until we obtain a safe
term. Consequently, we need a mean of performing several
substitutions at the same time. To achieve this, we introduce the
\emph{simultaneous substitution},
 a generalization of the standard substitution given in definition \ref{va_dfn:subst}.

\begin{dfn}[Simultaneous substitution]
 We use the notation
$\subst{\overline{N}}{\overline{x}}$ for $\subst{N_1 \ldots N_n}{x_1
\ldots x_n}$:
\begin{eqnarray*}
c \subst{\overline{N}}{\overline{x}} &=& c \quad \mbox{where $c$ is a $\Sigma$-constant}\\
x_i \subst{\overline{N}}{\overline{x}} &=& N_i\\
 y \subst{\overline{N}}{\overline{x}} &=& y \quad \mbox{ if } y \not \neq x_i \mbox{ for all } i,\\
(M N) \subst{\overline{N}}{\overline{x}} &=& (M \subst{\overline{N}}{\overline{x}}) (N \subst{\overline{N}}{\overline{x}}) \\
(\lambda x_i . M) \subst{\overline{N}}{\overline{x}} &=& \lambda x_i . M
\subst{N_1 \ldots N_{i-1} N_{i+1}\ldots N_n}{x_1 \ldots x_{i-1} x_{i+1}\ldots x_n} \\
(\lambda y . M)
\subst{\overline{N}}{\overline{x}} &=& \lambda z . M \subst{z}{y} \subst{\overline{N}}{\overline{x}} \\
&& \mbox{where $z$ is a fresh variables and } y \neq x_i \mbox{ for all } i
\end{eqnarray*}
\end{dfn}

In general, variable captures should be avoided, this explains why the definition
of simultaneous substitution uses auxiliary fresh variables.
However in the current setting, lemma \ref{va_lem:noclash} can clearly be transposed to
the simultaneous substitution therefore there is no need to rename variable.

The notion of substitution that we need is therefore
the \emph{capture permitting simultaneous substitution} defined as follows:

\begin{dfn}[Capture permitting simultaneous substitution]
 We use the notation
$\subst{\overline{N}}{\overline{x}}$ for $\subst{N_1 \ldots N_n}{x_1
\ldots x_n}$:
\begin{eqnarray*}
c \subst{\overline{N}}{\overline{x}} &=& c \quad \mbox{where $c$ is a $\Sigma$-constant}\\
 x_i \subst{\overline{N}}{\overline{x}} &=& N_i\\
 y \subst{\overline{N}}{\overline{x}} &=& y \quad \mbox{where } x \not \neq y_i \mbox{ for all } i,\\
(M_1 M_2) \subst{\overline{N}}{\overline{x}} &=& (M_1 \subst{\overline{N}}{\overline{x}}) (M_2 \subst{\overline{N}}{\overline{x}})\\
(\lambda x_i . M) \subst{\overline{N}}{\overline{x}} &=& \lambda x_i . M
\subst{N_1 \ldots N_{i-1} N_{i+1}\ldots N_n}{x_1 \ldots x_{i-1} x_{i+1}\ldots x_n} \\
(\lambda y . M) \subst{\overline{N}}{\overline{x}} &=& \lambda y . M \subst{\overline{N}}{\overline{x}} \mbox{where $y \not = x_i$ for all $i$}
\qquad \mathbf{(\star)}
\end{eqnarray*}
The symbol $\mathbf{(\star)}$ identifies the equation that changed compared to the previous definition.
\end{dfn}

\begin{lem}
\label{va_lem:subst_preserve_i}
$$ \Gamma,\overline{x} : \overline{A}\vdash^i M : T
\quad \mbox{and} \quad \Gamma \vdash^0 N_k : B_k \mbox{, } k \in
1..n \qquad \mbox{ implies } \qquad \Gamma \vdash^i
M[\overline{N}/\overline{x}] : T$$
\end{lem}

\begin{proof}
Suppose that $\Gamma,\overline{x}: \overline{A} \vdash^i M :T$ and
$\Gamma \vdash^0 N_k : B_k$ for $k \in 1..n$.

We prove $\Gamma \vdash^i M[\overline{N}/\overline{x}]$ by induction
on the size of the proof tree of $\Gamma,\overline{x}:\overline{A}
\vdash^i M : T$ and by case analysis on the last rule used. We just
give the detail for the abstraction case. Suppose that the property
is verified for terms whose proof tree is smaller than $M$. Suppose
$\Gamma,\overline{x}:\overline{A} \vdash^0 \lambda \overline{y} :
\overline{C}. P : (\overline{C}|D)$ where $\Gamma,
\overline{x}:\overline{A}, \overline{y}:\overline{C} \vdash^i P :
D$, then by the induction hypothesis $\Gamma,
\overline{y}:\overline{C} \vdash^i
P\subst{\overline{N}}{\overline{x}} : D$. Applying the rule
$\rulename{abs^i}$ gives $\Gamma \vdash^0 \lambda
\overline{y}:\overline{C} . P \subst{\overline{N}}{\overline{x}}$.
\end{proof}


\begin{cor}[Simultaneous substitution preserves safety]
If $M$ is safe and $N_k$ is safe for $k \in 1..n$ then  $M[\overline{N}/\overline{x}]$ is safe
\end{cor}

\subsection{Safe-redex}

In the simply-typed lambda calculus a redex is a term of the form
$(\lambda x . M) N$. We generalize this definition to the safe
lambda calculus:

\begin{dfn}[Safe redex]
We call safe redex a term of the form $(\lambda \overline{x} . M)
N_1 \ldots N_l$ such that:
\begin{itemize}
\item $ \Gamma \vdash^0 (\lambda \overline{x} . M) N_1 \ldots N_l $
\item the variable $\overline{x}=x_1\ldots x_n$ are abstracted altogether by one occurrence of the rule $\rulename{abs}$ in the proof tree.
%\item The terms $(\lambda \overline{x} . M)$, $N_1$, $N_l$ are applied together at once using the $\rulename{app}$ rule :
%$$   \rulef{
%            \Sigma \vdash^{-1} \lambda \overline{x} . M
%            \quad
%            \Sigma \vdash^0 N_1         \quad \ldots \quad \Sigma \vdash^0 N_l
%    }
%    {
%       \Sigma \vdash^0 (\lambda \overline{x} . L) N_1 \ldots N_l
%    } (\mathbf{app})
%$$
%Consequently each $N_i$ is safe.

\item $l\leq n$
\end{itemize}

\end{dfn}

Consequently, not all multi-$\beta$-redex of the form $(\lambda
\overline{x} . M) N_1 \ldots N_l$ is a safe-redex. It is important
to require that the abstraction $\lambda \overline{x}$ can be done
at once, otherwise we would not be able to prove that reducing a
safe-redex produces a safe term.


\todobox{Define the safe reduction: Consider the safe-redex
$(\lambda \overline{x} . M) N_1 \ldots N_l$, it reduces to $\lambda
x_l \ldots x_n . M \subst{N_1 \ldots N_l}{x_1 \ldots x_l}$. The
relation $\beta_s$ is defined on safe-redex: $(s\mapsto t) \in
\beta_s$ iff $s \equiv (\lambda \overline{x} . M) N_1 \ldots N_l$ is
a safe redex and $t \equiv \lambda x_l \ldots x_n . M \subst{N_1
\ldots N_l}{x_1 \ldots x_l}$ }

\todobox{Show that $\betasred \subseteq \betared^*$.}


Using the previous lemma, we will now prove that reducing a
safe-redex produces a safe term:

\begin{lem}
\label{va_lem:safereduction} A safe redex $(\lambda \overline{x} . M )
\overline{N}$ where the $N_i$ are safe reduces to the safe term
$M\subst{\overline{N}}{\overline{x}}$.
\end{lem}

\begin{proof}
%% NEED TO ADAPT THE FOLLOWING TO VERSION A OF THE RULES:
%%We note $\overline{A}$ for $A_1, \ldots , A_n$, $\overline{x}'$ for
%%$x_1 \ldots x_l$ and $\overline{x}''$ for $x_{l+1} \ldots x_n$.
%%
%%A safe-redex has a proof tree of the following form:
%%$$
%%   \rulef{
%%        \rulef{
%%            \rulef{
%%                \rulef{
%%                    \rulef
%%                        { \rulef
%%                            {\vdots}
%%                            {\Sigma',\overline{x}:\overline{A}\vdash^i L:C  }
%%                        }
%%                        {\Sigma' \vdash^0 \lambda \overline{x} . L : \overline{A}|C} \rulename{abs^i}
%%                }
%%                {\vdots} r_1
%%            }
%%            {\vdots} r_2
%%            }
%%            { \Sigma \vdash^{-1} \lambda \overline{x} . L : A_1, \ldots , A_l|B} r_q
%%            \quad
%%            \Sigma \vdash^0 N_1 : A_1
%%            \quad \ldots \quad \Sigma \vdash^0 N_l : A_l
%%    }
%%    {
%%       \Sigma \vdash^0 (\lambda \overline{x} . L) N_1 \ldots N_l : B
%%    } (\mathbf{app})
%%$$
%%with the following conditions:
%%\begin{enumerate}
%%\item for $j\in 1..q$, $r_j \in \{ \rulename{seq}, \rulename{wk^0}, \rulename{wk^{-1}} \}$ therefore
%%$\Sigma = \Sigma' \union \Delta$ where $\Delta$ contains the
%%variables introduced by the rules $r_1 \ldots r_q$.
%%
%%\item $A_1, \ldots , A_l|B = A_1, \ldots , A_n|C$ and $l\leq n$. Therefore
%%$\ord{B} \geq \ord{C}$.
%%\item The side condition of the rule $\rulename{abs}$ gives: $\forall z \in \Sigma : \ord{z} \geq \ord{B}$
%%\end{enumerate}
%%
%%
%%The conditions 2 and 3 ensure that $\forall z \in \Delta : \ord{z}
%%\geq \ord{C}$ therefore we can use the weakening rule to introduce
%%all the variable of $\Delta$ in the context of the sequent
%%$\Sigma',\overline{x}:\overline{A}\vdash^i L:C$:
%%
%%$$\rulef{\rulef{ \Sigma',\overline{x}:\overline{A}\vdash^i L:C  }
%%        {\vdots} (wk^i_0)}
%%        {\Sigma,\overline{x}:\overline{A}\vdash^i L:C} (wk^i_0)
%%$$
%%
%%By lemma \ref{va_lem:subst_preserve_i} we obtain:
%%$$ \Sigma, \overline{x}'':\overline{A}'' \vdash^i L\subst{N_1 \ldots N_l}{\overline{x}'}$$
%%Finally using the abstraction rule:
%%$$ \Sigma \vdash^0 \lambda \overline{x}'':\overline{A}'' . L\subst{N_1 \ldots N_l}{\overline{x}'}$$
\end{proof}




\subsection{Examples}

\subsubsection{Example 1 - Damien Sereni SCT counter-example}
In \cite{serenistypesct05}, the following counter-example is given
to show that not all simply-typed terms are size-change terminating
(see \cite{jones01} for a definition of size-change termination):

$$ E =  (\lambda a . a (\lambda b . a (\lambda c d .d))) (\lambda e . e (\lambda f .f))$$
where:
\begin{eqnarray*}
a &:& ((\tau \typear \tau) \typear \mu \typear \mu) \typear \mu \typear \mu \\
b &:& \tau \typear \tau \\
c &:& \tau \typear \tau \\
d &:& \mu \\
e &:& (\tau \typear \tau) \typear \mu \typear \mu \\
f &:& \tau
\end{eqnarray*}




\subsection{Particular case of homogeneously-safe lambda terms}

We look at a particular class of lambda terms: those having a
homogeneous type (as defined in section \ref{sec:homotypes}). We
recall that a type $(A_1, A_2, \ldots A_n, o)$ is said to be
homogeneous whenever $\order{A_1} \geq \order{A_2} \geq \ldots \geq
\order{A_n}$ and each of the $A_i$ are homogeneous. A term is
homogeneous if its type is homogeneous.


In their definition of safety (\cite{KNU02}), Knapik et al. require
that all the recursion equations of a safe recursion scheme have a
homogeneous type.

In the rules defining safety for the non-homogeneous case, the only
rule that can potentially introduce a non-homogeneous term from a
homogeneous one is the abstraction rule. But such a term (lambda
abstraction) corresponds exactly to a recursion equation in the
recursion scheme setting of Knapik et al. Therefore requiring that
recursions equation have homogeneous type is the same as requiring
that all sequents appearing in the proof tree of a safe term are of
homogeneous type.

We say that a term is homogeneously-safe if its type is homogeneous
and there is a proof tree showing its safety where all the sequents
of the proof tree are of homogenous type.

We are now going to specialize the rules of the safe
$\lambda$-calculus to obtain a system of rules for
homogeneously-safe terms.

\subsubsection{The application rule}
We recall the rule $\mathbf{(app^i)}$:
$$
 \mathbf{(app^i)} \quad  \rulef{\Gamma \vdash^i M : A\rightarrow B
                                        \qquad \Gamma \vdash^{0} N : A}
                                   {\Gamma  \vdash^{u} M N : B}
\quad \mbox{where } u = \min(i+\max\left(0, 1 + \ord{A} -
\ord{B}\right),0)
$$

Because of type homogeneity we have $\ord{A\typear B } = 1 +
\ord{A}$. The second premise gives $\forall z \in \Gamma : \ord{z}
\geq \ord{A} = 1 + \ord{A} - 1$. Hence the exponent $i$ in the first
premise can be replaced by $-1$.


Moreover type homogeneity implies $\ord{A} \geq \ord{B}-1$ therefore
$1 + \ord{A} - \ord{B} \geq 0$ and
$$ u = \min(i+1 + \ord{A} - \ord{B},0) = \min(\ord{A} - \ord{B},0)$$

\begin{itemize}
\item Suppose that $\ord{A} \geq \ord{B}$ then $u=0$ and we obtain the following rule:
$$ \mathbf{(app_1)} \quad  \rulef{\Gamma \vdash^{-1} M : A\rightarrow B
                                        \qquad \Gamma \vdash^{0} N : A }
                                   {\Gamma  \vdash^{0} M N : B}
                                    \qquad \ord{A} \geq \ord{B}$$

\item Suppose that $\ord{A} = \ord{B} - 1$ then
$ u = -1$ and we obtain the following rule:
$$ \mathbf{(app_2)} \quad  \rulef{\Gamma \vdash^{-1} M : A\rightarrow B
                                        \qquad \Gamma \vdash^{0} N : A
                                   }
                                   {\Gamma  \vdash^{-1} M N : B}
                                    \qquad \ord{A} = \ord{B} - 1$$
\end{itemize}



\subsubsection{The abstraction rule}

Let us derive the abstraction rule specialized for the case of
homogeneous types. We recall the rule $\rulename{abs}$:
$$ \rulename{abs^i} \quad  \rulef{\Gamma, \overline{x} : \overline{A} \vdash^{i} M : B}
                                   {\Gamma  \vdash^{0} \lambda \overline{x} : \overline{A} . M : (\overline{A},B)} \qquad
                                   \forall y \in \Gamma : \ord{y} \geq \ord{\overline{A},B}$$

We also use the notation $(\overline{A}|B)$ to denote the
homogeneous type $(A_1, A_2, \ldots A_n, B)$ where $\ord{A_1} =
\ord{A_2} =  \ldots \ord{A_n} \geq \ord{B} -1$.


Suppose that we abstract the single variable $\overline{x} = x$,
then in order to respect the side condition, we need to abstract all
variables of order lower or equal to $\ord{x}$. In particular we
need to abstract the partition of the order of $x$. Moreover to
respect type homogeneity, we need to abstract variables of the
lowest order first.

Hence we can change the abstraction rule so that it only allows
abstraction of the lowest variable partition. The rule can then be
used repeatedly if further partitions need to be abstracted.

The context $\Gamma$ is partitioned according to the order of the
variables. The partitions are written in decreasing order of type
order. The notation $\Gamma | \overline{x}:\overline{A}$ means that
$\overline{x}:\overline{A}$ is the lowest partition of the context.
We obtained the following rule:
$$ \rulename{abs^i} \quad  \rulef{\Gamma| \overline{x} : \overline{A} \vdash^{-1} M : B}
                                   {\Gamma  \vdash^{0} \lambda \overline{x} : \overline{A} . M : (\overline{A}|B)}$$

Note that the side-condition has disappeared.

\subsubsection{The other rules}

\begin{lem}
If a term is homogeneously-safe then there is valid proof tree
showing that it is safe containing only judgments of the form
$\Gamma \vdash^{k} M : T$ with $k\in \{-1,0\}$.
\end{lem}

\begin{proof}
This is a direct consequence from the fact that the sequents
appearing in the rules $\rulename{app}$ and $\rulename{abs}$ are all
of the form $\Gamma \vdash^{k} M : T$ with $k\in \{-1,0\}$. The
result can be proved formally with an easy structural induction.
\end{proof}

This lemma permits us to simplify the rules $\rulename{wk^i}$,
$\rulename{var}$, $\rulename{const}$, $\rulename{seq}$ and
$\rulename{perm}$. Table \ref{va_tab:homosafelmd_rules} recapitulates
the entire set of rules.

\begin{table}[htbp]
$$  \rulename{perm} {
      { \Gamma \vdash^0 M:B \qquad \sigma(\Gamma)  } \hbox{ homogeneous}
    \over
      { \sigma(\Gamma) \vdash^0 M : B }
    }
\qquad \rulename{seq} \quad \rulef{\Gamma \vdash^{0} M : A}{\Gamma
\vdash^{-1} M : A}
$$

$$
 \rulename{const}
    { \over { \vdash^0 b : o^r \rightarrow o}} \quad b : o^r \rightarrow o \in \Sigma
\qquad
 \rulename{var} \quad  \rulef{}{x : A\vdash^{0} x : A} $$

$$ \rulename{wk^{0}} \quad  \rulef{\Gamma \vdash^{0} M : A}{\Gamma , x : B \vdash^{0} M : A} \quad \ord{B} \geq \ord{A} $$

$$ \rulename{wk^{-1}} \quad  \rulef{\Gamma \vdash^{-1} M : A}{\Gamma , x : B \vdash^{-1} M : A} \quad \ord{B} \geq \ord{A} -1$$


$$ \mathbf{(app_1)} \quad  \rulef{\Gamma \vdash^{-1} M : A\rightarrow B
                                        \qquad \Gamma \vdash^{0} N : A }
                                   {\Gamma  \vdash^{0} M N : B}
                                    \qquad \ord{A} \geq \ord{B}$$

$$ \mathbf{(app_2)} \quad  \rulef{\Gamma \vdash^{-1} M : A\rightarrow B
                                        \qquad \Gamma \vdash^{0} N : A
                                   }
                                   {\Gamma  \vdash^{-1} M N : B}
                                    \qquad \ord{A} = \ord{B} - 1$$

$$ \rulename{abs} \quad  \rulef{\Gamma| \overline{x} : \overline{A} \vdash^{-1} M : B}
                                   {\Gamma  \vdash^{0} \lambda \overline{x} : \overline{A} . M : (\overline{A}|B)}$$
\caption{Rules of the homogeneously-safe lambda calculus}
\label{va_tab:homosafelmd_rules}
\end{table}

\subsubsection{Comparison with the rules of table \ref{tab:homosafelmd_rules_refined}}

The application rules are the only rules that do not match the
definition of table \ref{tab:homosafelmd_rules_refined}.

Here is a counter-example. Suppose:
\begin{eqnarray*}
 x&:&o\\
 \varphi, \theta &:& (o \typear o) \typear o \\
 f &:& \tau = (o \typear o) \typear (o \typear o) \typear o \\
\emptyset &\vdash^0& M \equiv (\lambda \varphi \theta . \varphi (\lambda x . x)) \\
f:\tau &\vdash^0& N \equiv f (\lambda x . x)
\end{eqnarray*}
Then we have $ f:\tau \vdash^{-1} M $ and using the
$\rulename{app_2}$ we get:
$$ f : \tau \vdash^{-1} M N$$

This term is not valid for the system of rules given in table \ref{tab:homosafelmd_rules_refined} simply because
for this system of rules if $\Gamma \vdash^i M$ then for all variable $x$ \emph{occurring free} in $M$, $\ord{x}\geq\ord{M}$. However
here $f$ occurs freely in $M N$ and $\ord{f} = 2 < 3 = \ord{M N}$.

%We observe that these rules correspond exactly to the rules given in
%the previous section in table \ref{tab:homosafelmd_rules_refined}.


% third chapter
\chapter{Computation trees, traversals and game semantics}

The aim of this chapter is to develop tools that will be used in the
next chapter to give a characterisation of the game semantics of the
Safe $\lambda$-Calculus. Establishing such a characterisation is
complicated by the fact that Safety is a syntactic restriction
whereas Game Semantics is by nature a syntax-independent semantics.
We therefore need to make an explicit correspondence between the
game denotation of a term and its syntax.

Our approach follows ideas recently introduced in
\cite{OngLics2006}, mainly the notion of computation tree of a
simply-typed $\lambda$-term and traversals over the computation
tree. A computation tree can be regarded as an abstract syntax tree
(AST) of the $\eta$-long normal form of a term. A traversal is a
justified sequence of nodes of the computation tree respecting some
formation rules. Traversals are used to describe computations. An
interesting property is that the \emph{P-view} of a traversal
(computed in the same way as P-view of plays in Game Semantics) is a
path in the computation tree.

The main result that we will prove in this chapter is called the
\emph{Correspondence Theorem} (theorem \ref{thm:correspondence}). It
states that traversals over the computation tree are just
representations of the uncovering of plays in the
strategy-denotation of the term. Hence there is an isomorphism
between the strategy denotation of a term and its revealed game
denotation (i.e. its strategy denotation where internal moves are
not hidden after composition). This theorem permits us to explore
the effect that a given syntactic restriction has on the strategy
denotating a term.

To really make use of the Correspondence Theorem, it will be
necessary to restate it in the standard game-semantic framework in
which internal moves are hidden. For that purpose, we will define a
\emph{reduction} operation on traversals responsible of eliminating
the ``internal nodes'' of the computation. This leads to a
correspondence between the standard game denotation of a term and
the set of reductions of traversals over its computation tree.
Fortunately, the reduction process preserves the good properties of
traversals. This is guaranteed by the facts that the P-view of the
reduction of a traversal is equal to the reduction of the P-view of
the traversal, and the O-view of a traversal is the same as the
O-view of its reduction (lemma \ref{lem:redtrav_trav}). \vspace{8pt}

\emph{Related works}: Traversals of a computation tree provide a way
to perform \emph{local computation} of $\beta$-reductions as opposed
to a global approach where the $\beta$-reduction is implemented by
performing substitutions. A notion of local computation of
$\beta$-reduction has been investigated in
\cite{DanosRegnier-Localandasynchronou} through the use of special
graphs called ``virtual nets'' that embed the lambda-calculus.

In \cite{DBLP:conf/lics/AspertiDLR94}, a notion of graph based on
Lamping's graphs \citep{lamping} is introduced to represent
$\lambda$-terms. The authors unify different notions of paths
(regular, legal, consistent and persistent paths) that have appeared
in the literature as ways to implement graph-based reduction of
lambda-expressions. We can regard a traversal as an alternative
notion of path adapted to the graph representation of
$\lambda$-expressions given by computation trees.



%Is there any unsafe term whose game semantics is a strategy where
%pointers can be recovered?
%
%The answer is yes: take the term $T_i = (\lambda x y . y) M_i S$
%where $i =1..2$ and $\Gamma \vdash_s S : A$. $T_1$ and $T_2$ both
%$\beta$-reduce to the safe term $S$, therefore
%$\sem{T_1}=\sem{T_2}=\sem{S}$. But $T_1$ is safe whereas $T_2$ is
%unsafe. Since it is possible to recover the pointer from the game
%semantics of $S$, it is as well possible to recover the pointer from
%the semantics of $T_2$ which is unsafe.

\section{Computation tree}
We work in the general setting of the simply-typed
$\lambda$-calculus extended with a fixed set $\Sigma$ of
higher-order constants.

\subsection{$\eta$-long normal form}

The $\eta$-long normal form appeared in
\citep{DBLP:journals/tcs/JensenP76} and
\citep{DBLP:journals/tcs/Huet75} under the names \emph{long reduced
form} and \emph{$\eta$-normal form} respectively. It was then
investigated in \citep{huet76} under the name \emph{extensional
form}.

The $\eta$-expansion of $M: A\typear B$ is defined to be the term
$\lambda x . M x : A\typear B$ where $x:A$ is a fresh variable. A
term $M : (A_1,\ldots,A_n,o)$ can be expanded in several steps into
$\lambda \varphi_1 \ldots \varphi_l . M \varphi_1 \ldots \varphi_l$
where the $\varphi_i:A_i$ are fresh variables. The $\eta$-normal
form of a term is obtained by hereditarily $\eta$-expanding every
subterm occurring at an operand position.

\begin{dfn}[$\eta$-long normal form]
A simply-typed term is either an abstraction or it can be written uniquely as
$s_0 s_1 \ldots s_m$ where $m\geq0$ and $s_0$ is a variable, a $\Sigma$-constant or an abstraction.
The $\eta$-long normal form of a term $t$, written $\elnf{t}$ or sometimes $\etanf{t}$,
is defined as follows:
\begin{align*}
\elnf{\lambda x . s } &= \lambda x . \elnf{s} \\
\elnf{\alpha s_1 \ldots s_m : (A_1,\ldots,A_n,o)} &= \lambda \overline{\varphi} . \alpha \elnf{s_1}\ldots \elnf{s_m} \elnf{\varphi_1} \ldots \elnf{\varphi_n}
& \mbox{with $m,n\geq0$}\\
%\elnf{(\lambda x . s_0) s_1 \ldots s_m } &=& (\lambda x . \elnf{s_0}) \elnf{s_1} \elnf{s_2} \ldots \elnf{s_m}
\elnf{(\lambda x . s) s_1 \ldots s_p : (A_1,\ldots,A_n,o) } &= \lambda \overline{\varphi} . (\lambda x . \elnf{s}) \elnf{s_1} \ldots \elnf{s_p} \elnf{\varphi_1} \ldots \elnf{\varphi_n}
& \mbox{with $p\geq 1,n\geq 0$}
\end{align*}
where $x$ and each $\varphi_i : A_i$ are variables and $\alpha$ is
either a variable or a constant.
\end{dfn}

For $n=0$, the first clause in the definition becomes:
$$\elnf{x s_1 \ldots s_m : o} = \lambda . x \elnf{s_1} \elnf{s_2} \ldots \elnf{s_m},$$
and we deliberately keep the \textsl{dummy} lambda in the right-hand
side of the equation because it will play an important role in the
correspondence with game semantics.



Note that our version of the $\eta$-long normal form is defined not only for $\beta$-normal terms but also for any simply-typed term.
Moreover it is defined in such a way that $\beta$-normality is preserved:
\begin{lem}
The $\eta$-long normal form of a term in $\beta$-normal form is also in $\beta$-normal form.
\end{lem}
\begin{proof}
By induction on the structure of the term and the order of its type.
\emph{Base case}:
If $M=x:0$ then $\elnf{x} = \lambda . x$ is also in $\beta$-nf.
\emph{Step case}:
The case $M = (\lambda x . s) s_1 \ldots s_m : (A_1,\ldots,A_n,o)$ with $m>0$ is not possible since $M$ is in
$\beta$-normal form.
Suppose $M = \lambda x . s$ then $s$ is in $\beta$-nf. By the induction hypothesis $\elnf{s}$ is also in $\beta$-nf and therefore
so is $\elnf{M} = \lambda x . \elnf{s}$.

Suppose $M= \alpha s_1 \ldots s_m : (A_1,\ldots,A_n,o)$. Let $i,j$
range over $1..n$ and $1..m$ respectively. The $s_j$ are in
$\beta$-nf and the $\varphi_i$ are variables of order smaller than
$M$, therefore by the induction hypothesis the $\elnf{\varphi_i}$ and
the $\elnf{s_j}$ are in $\beta$-nf. Hence $\elnf{M}$ is also in
$\beta$-nf.
\end{proof}

\begin{lem}[$\eta$-long normalisation preserves safety]
If $\Gamma \vdash s$ then $\Gamma \vdash \elnf{s}$.
\end{lem}
\begin{proof}

First we observe that for any variable or constant $x$ we have $x \vdash \elnf{x}$. The proof is by induction on $\ord{x}$. Base case: $x$ is of ground type and we have $x \vdash x = \elnf{x}$. Step case:
$x:(A_1, \ldots, A_n,o)$ with $n>0$. Let $\varphi_i:A_i$ be fresh variables for $1\leq i\leq n$. The (var) rules gives $\varphi_i  \vdash \varphi_i$ and since $\ord{A_i} < \ord{x}$ the induction hypothesis gives $\varphi_i \vdash \elnf{\varphi_i}$. Using (wk) we obtain $x, \overline{\varphi} \vdash \elnf{\varphi_i}$.
The application rule gives $x, \overline{\varphi} \vdash x \elnf{\varphi_1} \ldots \elnf{\varphi_n} : o$ and the abstraction rule gives $ x \vdash \lambda \overline{\varphi} . x \elnf{\varphi_1} \ldots \elnf{\varphi_n} = \elnf{x}$.


We now prove the lemma by induction on the structure of $s$.
The base case (where $s$ is some variable $x$) is covered by the previous observation.
\emph{Step case:} 
\begin{itemize}
\item $s = x s_1 \ldots s_m$ with $x: (B_1, \ldots, B_m, A_1, \ldots, A_n, o)$ with $m\geq 0$, $n>0$ and $s_i : B_i$ for $1 \leq i \leq m$. 

Let $\varphi_i: A_i$ be fresh variables for $1\leq i \leq n$. By the previous observation we have $\varphi_i \vdash \elnf{\varphi_i}$ which in turn gives $\Gamma , \overline{\varphi} \vdash \elnf{\varphi_i}$ using the weakening rule.

The judgement $\Gamma \vdash x s_1 \ldots s_m$ is formed using the (app) rule therefore each $s_j$ is safe for $1\leq j \leq m$. By the induction hypothesis we have $\Gamma \vdash \elnf{s_j}$ and by weakening we get $\Gamma, \overline{\varphi} \vdash \elnf{s_j}$.

The application rule gives $\Gamma, \overline{\varphi} \vdash 
x \elnf{s_1} \ldots \elnf{s_m} \elnf{\varphi_1} \ldots \elnf{\varphi_n} : o$. Finally the (abs) rule gives $\Gamma \vdash \lambda \overline{\varphi} . x \elnf{s_1} \ldots \elnf{s_m}  \elnf{\varphi_1} \ldots \elnf{\varphi_n} = \elnf{s}$, the side-condition of (abs) being met since $\ord{\elnf{s}} = \ord{s}$.


\item $s = t s_0 \ldots s_m$ where $t$ is an abstraction. Again, using the induction hypothesis it is easy to show that $\Gamma \vdash \elnf{s} = \elnf{t} \elnf{s_0} \ldots \elnf{s_m} \elnf{\varphi_1} \ldots \elnf{\varphi_n}$ holds for some fresh variables $\varphi_1$, \ldots, $\varphi_n$.

\item $s = \lambda \overline{\eta} . t$ where $t$ is not an abstraction. By the induction hypothesis we have $\Gamma, \overline{\eta} \vdash \elnf{t}$ and by the abstraction rule we have $\Gamma \vdash \lambda \overline{\eta} . \elnf{t} = \elnf{s}$.
\end{itemize}
\end{proof}

Note that in general the converse does not hold, for instance $\lambda x^o . f^{o,(o,o),o} x^o$ is unsafe although $\elnf{\lambda x . f x} = \lambda x^o \varphi^{o,o} . f x \varphi$ is safe (and not homogeneous). For terms with homogeneous types however, the converse does hold:
\begin{lem}
If $\Gamma \vdash \elnf{s}$ is homogeneously safe (i.e. it is a safe judgement of the safe $\lambda$-calculus and each sequent occurring at the nodes of the proof tree is homogeneously typed) then
$\Gamma \vdash s$ is homogeneously safe.
\end{lem}


\subsection{Computation tree}
The computation tree of a term is a certain tree representation of its
$\eta$-long normal form. It is defined as follows:
\begin{dfn}[Computation tree]
For any term $M$ in $\eta$-normal form we define the tree $\tau(M)$ by induction
on the structure of the term.
Since $M$ is in $\eta$-normal form, there are only two cases:
$M$ is either an abstraction or it is of ground type and can be written uniquely as
$s_0 s_1 \ldots s_m : 0$ where $m\geq0$,  $s_0$ is a variable, a
constant or an abstraction and each of the $s_j$ for $j\in 1..m$ is in $\eta$-normal form:
\begin{itemize}
\item the tree for $\lambda x_1 \ldots x_n. s$ where $n\geq0$ and $s$ is not an abstraction is:
$$ \tau(\lambda x_1 \ldots x_n . s) =
      \pstree[levelsep=3ex]
        { \TR{\lambda x_1 \ldots x_n} }
        { \SubTree{\tau(s)^{-}} }
$$
where $\tau(s)^{-}$ denotes the tree obtained after deleting the root of $\tau(s)$.


\item the tree for $\alpha s_1 \ldots s_m : o$ where $m\geq0$ and $\alpha$ is a variable or constant is:
$$ \tau( \alpha s_1 \ldots s_m) =
    \tree{\lambda}
    {
        \pstree[levelsep=3ex]
            { \TR{\alpha} }
            { \SubTree{\tau(s_1)} \SubTree[linestyle=none]{\ldots} \SubTree{\tau(s_m)}
            }
    }
$$


\item the tree for $(\lambda x.s) s_1 \ldots s_n : o$ where $n \geq 1$ is:
$$ \tau((\lambda x.s) s_1 \ldots s_n) =
    \tree{\lambda}
    {
        \pstree[levelsep=3ex]
            { \TR{@} }
            {
            \SubTree{\tau(\lambda x.s)}    \SubTree{\tau(s_1)} \SubTree[linestyle=none]{\ldots} \SubTree{\tau(s_n)}
            }
    }
$$
\end{itemize}

The \emph{computation tree} of a simply-typed term $M$ (whether or not in $\eta$-normal form) is written $\tau(M)$
and defined to be $\tau(M) = \tau(\etanf{M})$.
\end{dfn}

The nodes (and leaves) of the tree are of three kinds:
\begin{itemize}
\item $\lambda$-nodes labelled $\lambda \overline{x}$ (note that a $\lambda$-node represents several consecutive variable abstractions),
\item application nodes labelled @,
\item variable or constant nodes labelled $\alpha$ for some constant or variable $\alpha$.
\end{itemize}
We write $N$ for the set of nodes of $\tau(M)$, $N_\Sigma$ for the set of $\Sigma$-labelled nodes,
$N_@$ for the set of @-labelled nodes, $N_{var}$ for the set of variable nodes,
$N_{fv}$ for the subset of $N_{var}$ constituted of free-variable nodes and $r$ for the root of $\tau(M)$.


Let $\mathcal{T}$ denote the set of $\lambda$-terms.
Each subtree of the computation tree $\tau(M)$ represents a subterm of $\elnf{M}$.
We define the function $\kappa : N \rightarrow \mathcal{T}$ that maps a node $n \in N$ to the subterm of $\elnf{M}$
represented by the subtree of $\tau(M)$ rooted at $n$.
In particular if $r$ is the root of $\tau(M)$ then $\kappa(r) = \elnf{M}$.

\begin{dfn}[Type and order of a node]
\label{def:nodeorder}
Suppose $\Gamma \vdash M : T$. 
Each node $n$ of $\tau(M)$ is assigned a type $type(n)$ defined as follows:
\begin{eqnarray*}
type(r) &=& \Gamma \rightarrow T \\
type(\alpha:A) &=& A, \mbox{ where $\alpha$ is a variable or constant} \\
type(n) &=& B, \hbox{ where 
$\kappa(n) : B$ for $n \in (N_\lambda \union N_@) \setminus \{r \}$\ .}
\end{eqnarray*}
The order of a node $n$ written $\ord{n}$ is defined to be $\ord{type(n)}$.
\end{dfn}

In particular, $\ord{@} = 0$, $\ord{\lambda \overline{\xi}} = 1+ \max_{z\in \overline{\xi}} \ord{z}$ for $r \neq \lambda \overline{\xi}$
and if $r=\lambda \overline{\xi}$ then $\ord{r} = 1 + \max_{z\in \overline{\xi}\union \Gamma} \ord{z}$ with the convention that $\max \emptyset = -1$.

\noindent Some remarks:
\begin{itemize}
\item In a computation tree, nodes at even level are $\lambda$-nodes and nodes at odd level are either application nodes,
variable or constant nodes;

\item for any ground type variable or constant $\alpha$,
$\tau(\alpha) = \tau(\lambda . \alpha) =  \pstree[levelsep=3ex]
    { \TR{\lambda } }
    { \TR{\alpha}
    }$;

\item for any higher-order variable or constant $\alpha : (A_1,\ldots,A_p,o)$, the computation tree $\tau(\alpha)$ has the following form:
$ \pstree[levelsep=3ex]{\TR{\lambda}}
        {\pstree[levelsep=3ex]
                { \TR{\alpha} }
                { \tree{\lambda \overline{\xi_1}}{\TR{\ldots}} \TR{\ldots} \tree{\lambda \overline{\xi_p}}{\TR{\ldots}}
                }
        }
$;

\item for any tree of the form
        $ \pstree[levelsep=4ex]
            { \TR{\lambda \overline{\varphi}} }
            { \pstree[levelsep=3ex]
                {\TR{n}}
                {\TR{\lambda \overline{\xi_1}} \TR{\ldots} \TR{\lambda \overline{\xi_p}}}
            }
        $,
    we have $\ord{\kappa(n)}=0$.

\end{itemize}



\subsection{Pointers and justified sequence of nodes}

\begin{dfn}[Binder]
Let $n$ be a variable node of the computation tree labelled $x$. We
say that a node $n$ is bound by the node $m$, and $m$ is called the
binder of $n$, if $m$ is the closest node in the path from $n$ to
the root of the tree such that $m$ is labelled $\lambda
\overline{\xi}$ with $x\in \overline{\xi}$.
\end{dfn}

\begin{dfn}[Enabling]
The enabling relation $\vdash$ is defined on the set of nodes of the
computation tree. We write $m \vdash n$ and we say that $m$ enables
$n$ if and only if
\begin{itemize}
\item $n$ is a bound variable node and $m$ is the binder of $n$,
\item or $n$ is a free variable node and $m$ is the root of the computation tree,
\item or $n$ is a $\lambda$-node and $m$ is the parent node of $n$.
\end{itemize}
\end{dfn}

For any set of nodes $S$ we write $S^{\upharpoonright r}$ for $\{ n \in S \ | \ r  \vdash^* n \}$ -- the subset of $S$ constituted of
nodes hereditarily enabled by $r$.
We call \defname{input-variables nodes} the elements of $N_{var}^{\upharpoonright r}$ i.e.\ 
variables that are hereditarily enabled by the root. $N_{var}^{\upharpoonright r}$ is also the set of nodes that are hereditarily enabled by a free variable or by a variable bound by the root.

\begin{dfn}[Justified sequence of nodes]
A \emph{justified sequence of nodes} is a sequence of
nodes of the computation tree $\tau(M)$ with pointers attached to the nodes. A node $n$ in the sequence
that is either a variable node or a lambda-node different from the root of the computation tree
has a pointer to a node $m$ occurring before $n$ in the sequence such that $m \vdash n$.
If $n$ points to $m$ then we say that $m$ \emph{justifies} $n$ and we represent the pointer in the sequence as follows:
$$\rnode{m}{m} \cdot \ldots \cdot \rnode{n}{n} \link[nodesep=1pt]{40}{n}{m}$$
Such pointer is sometimes labelled with an index $i$: 
indicating that $n$ is labelled with the $i$th
variable abstracted in $m$ if $m$ is a $\lambda$-node, or that $n$ is the $i$th child of $m$
otherwise. 
\end{dfn}
Note that justified sequences are also defined for open terms:
occurrences of nodes in $N_{fv}$ must point to an occurrence of the
root of the computation tree.

A pointer in a justified sequence of nodes has
one of the following forms: \vspace{2pt}
$$
\rnode{m}{r} \cdot \ldots \cdot \rnode{n}{z} \link[nodesep=1pt]{40}{n}{m}
\hspace{1.5cm}
\rnode{m}{\lambda \overline{\xi}} \cdot \ldots \cdot \rnode{n}{\xi_i} \link[nodesep=1pt]{40}{n}{m} \lnklabel{i}
\hspace{1.5cm}
\rnode{m}{@ } \cdot \ldots \cdot \rnode{n}{\lambda \overline{\eta}} \link[nodesep=1pt]{40}{n}{m} \lnklabel{j}
\hspace{1.5cm}
\rnode{m}{\alpha } \cdot \ldots \cdot \rnode{n}{\lambda \overline{\eta}} \link[nodesep=1pt]{40}{n}{m} \lnklabel{k}
$$
where $r$ denotes the root of $\tau(M)$, $z \in N_{fv}$, $\xi_1,
\ldots \xi_n$ are bound variables, $\alpha \in N_{\Sigma} \union
N_{var}$, $i \in 1..n$, $j$ ranges from $0$ to the number of
children nodes of @ minus 1 and $k \in 1 ..arity(\alpha)$.

The following numbering conventions are used:
\begin{itemize}
\item the first child of a @-node is numbered $0$,
\item the first child of a variable or constant node is numbered $1$,
\item variables in $\overline{\xi}$ are numbered from $1$ onward ($\overline{\xi} = \xi_1 \ldots \xi_n$).
\end{itemize}
We use the notation $n.i$ to denote the $i$th child of node $n$.


We write $s = t$ to denote that the justified sequences $t$ and $s$
have same nodes \emph{and} pointers. Justified sequence of nodes can
be ordered using the prefix ordering: $t \sqsubseteq t'$ if and only
if $t=t'$ or the sequence of nodes $t$ is a finite prefix of $t'$
(and the pointers of $t$ are the same as the pointers of the
corresponding prefix of $t'$). Note that with this definition,
infinite justified sequences can also be compared. This ordering
gives rise to a complete partial order.

We say that a node $n_0$ of a justified sequence is hereditarily justified by $n_p$ if there are nodes $n_1, n_2, \ldots n_{p-1}$ in
the sequence such that for all $i\in 0..p-1$, $n_i$ points to $n_{i+1}$.

If $H$ is a set of nodes and $s$ a justified sequence of nodes then we write $s \upharpoonright H$ to denote the subsequence of $s$ obtained by keeping only the nodes that are hereditarily justified by nodes in $H$. This subsequence is also a justified
sequence of nodes. If $n$ denotes a node of $\tau(M)$ we abbreviate $s \upharpoonright \{ n \}$ into $ s\upharpoonright n$.

\begin{lem}
\label{lem:filtercontinous}
For any set of node $N$, the filtering function $\_ \upharpoonright N$ defined on the cpo of justified sequences ordered by the prefix ordering
is continuous.
\end{lem}
\begin{proof}
Clearly $\_ \upharpoonright N$ is monotonous.
Suppose that $(t_i)_{i\in\omega}$ is a chain of justified sequence of nodes. Let $u$ be a finite prefix of $(\bigvee t_i) \upharpoonright r$.
Then $u = s \upharpoonright r$ for some finite prefix $s$ of $\bigvee t_i$. Since $s$ is finite we must have $s \sqsubseteq t_j$ for some $j\in\omega$.
Therefore $u \sqsubseteq t_j \upharpoonright r \sqsubseteq \bigvee (t_j \upharpoonright r)$.
This is valid for any finite prefix $u$ therefore $(\bigvee t_i) \upharpoonright r \sqsubseteq \bigvee (t_j \upharpoonright r)$.
\end{proof}


\begin{dfn}[P-view of justified sequence of nodes]
The P-view of a justified sequence of nodes $t$ of $\tau(M)$, written $\pview{t}$, is defined as follows:
\begin{eqnarray*}
 \pview{\epsilon} &=&  \epsilon \\
 \pview{s \cdot n }  &=&  \pview{s} \cdot n \\
 \pview{s \cdot \rnode{m}{m} \cdot \ldots \cdot \rnode{lmd}{\lambda \overline{\xi}}} &=& \pview{s} \cdot \rnode{m2}{m} \cdot \rnode{lmd2}{\lambda \overline{\xi}}
   \link[nodesep=1pt]{30}{lmd}{m}
   \link[nodesep=1pt]{60}{lmd2}{m2} \\
 \pview{s \cdot r }  &=&  r
\end{eqnarray*}
where $r$ is the root of the tree $\tau(M)$ and $n$ ranges over
non-lambda nodes (i.e. $N_\Sigma \union N_@ \union N_{var}$).

In the second clause, the pointer associated to $n$ is preserved
from the left-hand side to the right-hand side i.e. if in the
left-hand side, $n$ points to some node in $s$ that is also present
in $\pview{s}$ then in the right-hand side, $n$ points to this
occurrence of the node in $\pview{s}$.

Similarly, in the third clause the pointer associated to $m$ is preserved.
\end{dfn}

We also define O-view, the dual notion of P-view:
\begin{dfn}[O-view of justified sequence of nodes]
The O-view of a justified sequence of nodes $t$ of $\tau(M)$, written $\oview{t}$, is defined as follows:
\begin{eqnarray*}
 \oview{\epsilon} &=&  \epsilon \\
 \oview{s \cdot \lambda \overline{\xi} }  &=&  \oview{s} \cdot \lambda \overline{\xi} \\
 \oview{s \cdot \rnode{m}{m} \cdot \ldots \cdot \rnode{x}{x}} &=& \oview{s} \cdot \rnode{m2}{m} \cdot \rnode{n2}{x} \\
   \link[nodesep=1pt]{30}{x}{m}
   \link[nodesep=1pt]{60}{n2}{m2}
 \oview{s \cdot n }  &=&  n
\end{eqnarray*}
where $x$ ranges over variable nodes and  $n$ ranges over non-lambda
nodes without pointer (i.e. $N_@ \union N_\Sigma$).

The pointer associated to $\lambda \overline{\xi}$ in the second
equality and the pointer associated to $m$ in the third equality are
preserved from the left-hand side to the right-hand side of the
equalities.
\end{dfn}

\begin{dfn}[Alternation and Visibility] \ \\
A justified sequence of nodes $s$ satisfies:
\begin{itemize}
\item \emph{Alternation} if for any two consecutive nodes in $s$, one is a $\lambda$-node
and the other is not;

\item \emph{P-visibility} if every variable node in $s$ points to a node occurring in the P-view a that point;

\item  \emph{O-visibility} if every lambda node in $s$ points to a node occurring in the O-view a that point.
\end{itemize}
\end{dfn}

\begin{property}
\label{proper:pview_visibility}
The P-view (resp. O-view) of a justified sequence verifying P-visibility (resp. O-visibility)
is a well-formed justified sequence verifying P-visibility (resp. P-visibility).
\end{property}
This is proved by an easy induction.

\subsection{Adding value-leaves to the computation tree}
\label{sec:adding_value_leaves}

We now add leaves to the computation tree that has been defined in the previous section.
These leaves, called \emph{value-leaves}, are attached to the nodes of the computation tree. Each
value-leaf corresponds to a possible value of the base type $o$.
We write $\mathcal{D}$ to denote the set of values of the base type
$o$. The values leaves are added as follows: every  %$\lambda$-node or variable
node $n \in \tau(M)$ has a child leaf denoted by $v_n$ for each possible value $v \in \mathcal{D}$.

%@-nodes and $\Sigma$-nodes do not have child leaves.

%If $n$ is a $\lambda$-node then its value-leaves are numbered from $1$ onwards.
%If $n$ is a variable or constant node then its children nodes are numbered from $1$ to $arity(n)$ and
%its value-leaves are numbered from $arity(n)+1$ onwards.
%If $n$ is an application node then its value-leaves are numbered from $1$ onwards.

Everything that we have defined for computation tree can be lifted
to this new version of computation tree. The node order of a
value-leaf is defined to be $0$. The enabling relation $\vdash$ is
extended so that every leaf is enabled by its parent node. The
definition of justified sequence does not change.
When representing a link in a justified sequence going from a value-leaf $v_n$ to a node $n$,
we label the link with $v$:
$$
\rnode{n}{n} \cdot \ldots \cdot \rnode{vn}{v_n} \link[nodesep=1pt]{40}{vn}{n} \lnklabel{v}
$$


For the definition
of P-view, O-view and visibility, value-leaves are treated as
$\lambda$-nodes if they are at odd level in the computation tree and
as variable nodes if there at a even level.

From now the term ``computation tree'' refers to this extended
definition.
\vspace{10pt}

Let $n$ be a node of a justified sequence of nodes.
% that is either a $\lambda$-node or a variable node.
If there is an occurrence of a value-leaf $v_n$ in the sequence that points to $n$ we say that
$n$ is \emph{matched} by $v_n$. If there is no value-leaf in the sequence that points to $n$ we
say that $n$ is an \emph{unmatched node}.
The last unmatched node is called the \emph{pending node}.
A justified sequence of nodes is \emph{well-bracketed} if
each value-leaf in the traversal points to the pending node at that point.

If $t$ is a traversal then we write $?(t)$ to denote the subsequence
of $t$ consisting only of unmatched nodes.

\subsection{Traversal of the computation tree}
\label{subsec:traversal} 
Intuitively, a \emph{traversal} is a justified sequence of nodes of the computation tree where each node indicates a step that is taken during the evaluation of the term.

\subsubsection{Traversals for simply-typed $\lambda$-terms}
We first define traversals for computation trees of simply typed $\lambda$-terms with no interpreted constants.
We will then we show how to extend the definition to the general setting of $\lambda$-calculus augmented with interpreted constants.

\begin{dfn}[Traversals for simply-typed $\lambda$-terms]
\label{def:traversal}
In the simply-typed $\lambda$-calculus without interpreted constants, a traversal over a computation tree $\tau(M)$ is a justified sequence of nodes defined by induction on the rules
given below. A \emph{maximal-traversal} is a traversal that cannot be
extended by any rule. If $T$ denotes a computation tree then we write $\travset(T)$
to denote the set of traversals of $T$. We also use the abbreviation $\travset(M)$ for $\travset(\tau(M))$.

\emph{Initialization rules}
\begin{itemize}
\item ($\epsilon$) The empty sequence of node $\epsilon$ is a traversal.
\item (Root) The length 1 sequence $r$, where $r$ is denotes the root of $\tau(M)$, is a traversal.
\end{itemize}

\emph{Structural rules}
\begin{itemize}
\item (Lam) Suppose that $t \cdot \lambda \overline{\xi}$ is a traversal and $n$ is the only child node of $\lambda \overline{\xi}$ in
the computation tree then
$$t \cdot \lambda \overline{\xi} \cdot n$$
is also a traversal
where $n$ points to the \emph{only} occurrence of its enabler in $\pview{t \cdot \lambda \overline{\xi}}$.
(Prop. \ref{prop:pviewtrav_is_path} shows that traversals are well-defined and that indeed $n$'s enabler occurs only once in the P-view at that point).

In particular, if $n$ is a free variable node then $n$ points to the first node of $t$.

\item (App) If $t \cdot @$ is a traversal then so is
$$t \cdot \rnode{m}{@} \cdot
\rnode{n}{n} \link[nodesep=1pt]{60}{n}{m} \lnklabelc{0}
$$

i.e. the next visited node is the $0$th child node of @ : the
node corresponding to the operator of the application.
\end{itemize}

\emph{Input-variable rules}
\begin{itemize}
\item (InputVar$^{val}$) If $t = t_1 \cdot x \cdot t_2$ is a traversal where $x \in N^{\upharpoonright r}_{var}$ and $x$ is the pending node in $t$ (i.e. $?(t)=?(t_1) \cdot x$) then for any $v \in \mathcal{D}$,
\raisebox{0cm}[0.4cm]{$t_1 \cdot \rnode{x}{x} \cdot t_2 \cdot \rnode{xv}{v_x}
\link[nodesep=0.5pt]{20}{xv}{x} \lnklabelc{v}$} is a traversal.


\item (InputVar)
If $t = t_1 \cdot x \cdot t_2$ is a traversal where
$x \in N_{var}^{\upharpoonright r}$ and $x$ is the pending node in $t$ (i.e. $?(t)=?(t_1) \cdot x$) then so is
\raisebox{0cm}[0.4cm]{$t_1 \cdot \rnode{m}{x} \cdot t_2 \cdot \rnode{n}{n} \link[nodesep=0.5pt]{25}{n}{m} \lnklabelc{i}$}
for all $\lambda$-node $n$ whose parent occurs in $\oview{t_1 \cdot x}$, $n$ pointing to some occurrence of its parent node in $\oview{t_1 \cdot x}$.
\end{itemize}

\emph{Copy-cat rules}
\begin{itemize}
  \item (CCAnswer-@)
%  If $t \cdot \lambda \overline{\xi} \cdot \rnode{app}{@} \cdot \rnode{lz}{\lambda \overline{z}} \cdot \ldots \cdot  \rnode{lzv}{v_{\lambda \overline{z}}}
%              \link[nodesep=1pt]{30}{lzv}{lz} \lnklabelc{v}
%              \link[nodesep=1pt]{40}{lz}{app} \lnklabelc{0}$
%              is a traversal then so is:
%              $t \cdot \rnode{lmd}{\lambda \overline{\xi}} \cdot \rnode{app}{@} \cdot \rnode{lz}{\lambda \overline{z}} \cdot \ldots \cdot \rnode{lzv}{v_{\lambda \overline{z}}} \cdot
%              \rnode{lmdv}{v_{\lambda \overline{\xi}}}
%              \link[nodesep=1pt]{30}{lzv}{lz} \lnklabelc{v}
%              \link[nodesep=1pt]{40}{lz}{app} \lnklabelc{0}
%                \link[nodesep=1pt]{30}{lmdv}{lmd} \lnklabelc{v}$.
  If $t \cdot \rnode{app}{@} \cdot \rnode{lz}{\lambda \overline{z}} \cdot \ldots \cdot \rnode{lzv}{v_{\lambda \overline{z}}}
              \link[nodesep=1pt]{30}{lzv}{lz} \lnklabelc{v}
              \link[nodesep=1pt]{40}{lz}{app} \lnklabelc{0}$
              is a traversal then so is:
              $t \cdot \rnode{app}{@} \cdot \rnode{lz}{\lambda \overline{z}} \cdot \ldots \cdot \rnode{lzv}{v_{\lambda \overline{z}}} \cdot \rnode{appv}{v_@}
              \link[nodesep=1pt]{30}{lzv}{lz} \lnklabelc{v}
              \link[nodesep=1pt]{40}{lz}{app} \lnklabelc{0}
              \link[nodesep=1pt]{30}{appv}{app} \lnklabelc{v}$.


  \item (CCAnswer-$\lambda$) If $t \cdot \lambda \overline{\xi} \cdot \rnode{x}{x} \cdot \ldots \cdot  \rnode{xv}{v_x}
              \link[nodesep=1pt]{30}{xv}{x} \lnklabelc{v}$
              is a traversal then so is:
              $t \cdot \rnode{lmd}{\lambda \overline{\xi}} \cdot \rnode{x}{x} \cdot \ldots \cdot \rnode{xv}{v_x} \cdot
              \rnode{lmdv}{v_{\lambda \overline{\xi}}}
              \link[nodesep=1pt]{20}{xv}{x} \lnklabelc{v}
                \link[nodesep=1pt]{20}{lmdv}{lmd} \lnklabelc{v}$.

     \item (CCAnswer-var) If $t \cdot y \cdot \rnode{lmd}{\lambda \overline{\xi}}
                   \cdot \ldots
                   \cdot \rnode{lmdv}{v_{\lambda \overline{\xi}}} \link[nodesep=1pt]{30}{lmdv}{lmd} \lnklabelc{v}$ is a traversal,
                   where $y$ is a non input-variable node, then the following is also a traversal:
        $$t \cdot \rnode{y}{y}
            \cdot \rnode{lmd}{\lambda \overline{\xi}}
            \cdot \ldots
            \cdot \rnode{lmdv}{v_{\lambda \overline{\xi}}}
            \cdot \rnode{yv}{v_y}
                \link[nodesep=3pt]{35}{yv}{y} \lnklabelc{v}
                \link[nodesep=1pt]{30}{lmdv}{lmd} \lnklabelc{v}.$$


\item (Var)
If $t \cdot x_i$ is a traversal where $x_i$ is not an input-variable,
then the rule (Var) permits to visit the node corresponding to the subterm that would be substituted
for $x_i$ if all the $\beta$-redexes occurring in $M$ were reduced.

The binding node $\lambda \overline{x}$ necessarily occurs previously
in the traversal. Since $x$ is not hereditarily justified by the
root, $\lambda \overline{x}$ is not the root of the tree and
therefore its justifier $p$ - which is also its parent node - occurs
immediately before itself it in the traversal. We do a case analysis
on $p$:

    \begin{itemize}
    \item Suppose $p$ is an @-node then $\lambda \overline{x}$ is necessarily the first child node of $p$
    and $p$ has exactly $|\overline{x}| + 1$ children nodes:
    $$\pstree[levelsep=7ex]{\TR{\stackrel{\vdots}{@^{[p]}}}}
    {   \pstree[linestyle=dotted,levelsep=4ex]{\TR{\lambda \overline{x}}\treelabel{0}}
            {\TR{x_i }}
        \tree{\lambda \overline{\eta_1}}{\vdots}\treelabel{1}
        \TR[edge=\dotedge]{}
        \tree{\lambda \overline{\eta_i}}{\vdots}\treelabel{i}
        \TR[edge=\dotedge]{}
        \tree{\lambda \overline{\eta_{|x|}}}{\vdots}\treelabel{|x|}
    }
    $$
    In that case, the next step of the traversal is a jump to $\lambda \overline{\eta_i}$ -- the $i$th child of
    @ -- which corresponds to the subterm that would be substituted for $x_i$ if the $\beta$-reduction was
    performed:
    \vspace{0.3cm}
    $$t' \cdot \rnode{n}{@^{[p]}} \cdot
    \rnode{lx}{\lambda \overline{x}} \cdot \ldots \cdot
    \rnode{x}{x_i} \cdot
    \rnode{mi}{\lambda \overline{\eta_i}} \cdot \ldots
    \link[ncurv=0.45]{45}{mi}{n} \lnklabel{i}
    \link[ncurv=0.6]{50}{x}{lx} \lnklabel{i} \in \travset(M)
    $$

    \item Suppose $p$ is variable node $y$, then
    necessarily the node $\lambda \overline{x}$ has been added to the traversal $t_{\leq y}$ using the (Var) rule
    (this is proved in proposition \ref{prop:pviewtrav_is_path}(i)).
    Therefore $y$ is substituted by the term $\kappa(\lambda \overline{x})$ during the evaluation of the term
    and we have $\ord{y} = \ord{\lambda \overline{x}}$.

    Consequently, during reduction, the variable $x_i$ is substituted by the subterm represented by
    $\lambda \overline{\eta_i}$ -- the $i$th child node of $y$.
    Hence the following justified sequence is also a traversal:
    \vspace{0.2cm}
    $$t' \cdot \rnode{n}{y^{[n]}} \cdot
    \rnode{lx}{\lambda \overline{x}} \cdot \ldots \cdot
    \rnode{x}{x_i} \cdot
    \rnode{mi}{\lambda \overline{\eta_i}} \cdot \ldots
    \link[ncurv=0.6]{50}{x}{lx} \lnklabel{i}
    \link[ncurv=0.5]{50}{mi}{n} \lnklabel{i}$$
    \end{itemize}
\end{itemize}
Note that a traversal always starts with the root of the tree.
\end{dfn}

\begin{rem}
Our notions of computation tree and traversal differ slightly from
\cite{OngLics2006}.

Firstly, our computation trees do not have nodes labelled with (uninterpreted) first-order constants. On the other hand, there are nodes which are labelled by free variables of any order. Since uninterpreted constants can be regarded as free variables, we do not lose any expressivity. Also, the traversal rules (InputVar$^{val}$) and (InputVar) provide a more general version of the (Sig) rule of \cite{OngLics2006} that models free variables and not just ``constructor'' constants.

Secondly we have introduced copy-cat rules that permit to visit the
value-leaves of the computation tree. The presence of value-leaves
is necessary to model free variables as well as the interpreted
constants present in extensions of the $\lambda$-calculus such as
\pcf\ or \ialgol.
\end{rem}

\begin{exmp}
Consider the following computation tree:
$$\tree{\lambda}
{
    \tree{@}
    {
        \pstree[levelsep=8ex,linestyle=dotted]{\TR{\lambda y}\treelabel{0} }
        {
            \pstree[levelsep=8ex]{\TR{y}}
            {
                \tree{\lambda \overline{\eta_1}}{\vdots} \treelabel{1}
                \TR[edge=\dotedge]{}
                \tree{\lambda \overline{\eta_i}}{\vdots}\treelabel{i}
                \TR[edge=\dotedge]{}
                \tree{\lambda \overline{\eta_n}}{\vdots}\treelabel{n}
            }
        }
        \pstree[levelsep=6ex,linestyle=dotted]{\TR{\lambda \overline{x}}\treelabel{1}}{ \tree{x_i}{\TR{} \TR{} } }
    }
}
$$
An example of traversal of this tree is:
\vspace{0.3cm}
$$ \lambda \cdot
\rnode{app}{@}  \cdot
\rnode{ly}{\lambda y} \cdot \ldots \cdot
\rnode{y}{y} \cdot
\rnode{lx}{\lambda \overline{x}} \cdot \ldots \cdot
\rnode{x}{x_i} \cdot
\rnode{leta}{\lambda \overline{\eta_i} } \cdot \ldots
\link[ncurv=0.6,nodesep=0]{40}{x}{lx}  \lnklabel{i}
\link[ncurv=0.5]{50}{leta}{y}  \lnklabel{i}
\link[ncurv=0.6,nodesep=0]{40}{y}{ly}  \lnklabel{1}
\link[ncurv=0.5]{50}{lx}{app}  \lnklabel{1}$$
\end{exmp}

\subsubsection{Traversals for interpreted constants}

\begin{dfn}[Well-behaved traversal rule]
\label{def:wellbehaved_traversal}
A traversal rule is \emph{well-behaved} if it can be stated under the following form:
$$\rulef{t = t_1\cdot n \cdot t_2 \in \travset \quad ?(t) = ?(t_1) \cdot n \quad P(t)}
  { \stackrel{  \rule{0pt}{3pt} }{t' = t_1\cdot \rnode{n}{n} \cdot t_2 \cdot \rnode{m}{m} \in \travset} }
   \link[nodesep=1pt]{35}{m}{n}
    \ m\in S(t)
   $$
such that:
\begin{enumerate}
  \item $n$ is a variable or a constant node;
  \item $P$ expresses some condition on $t$;
  \item $S(t)$ is some subset of $E(n)$, the set of children $\lambda$-nodes and value-leaves of $n$.
  If $S(t)$ has more than one element then the rule is non-deterministic.
\end{enumerate}
\end{dfn}
Note that if $t$ is well-bracketed then $t'$ is also well-bracketed
and if $?(t)$ satisfies alternation (resp. visibility) then so does $?(t')$.


\begin{exmp} The rule (InputVar$^{val}$) is an example of non-deterministic well-behaved traversal rule for which $S(t)$ is exactly the set of all children value-leaves of $n$:
$S(t) = \{ v_n \ | \ v \in \mathcal{D} \} $.
However (InputVar) is not well-behaved since it can jump to any node in the O-view at that point and not necessarily to a children node of the last pending node.
\end{exmp}

In the presence of higher-order interpreted constants, additional rules must be specified to indicate how
the constant nodes should be traversed in the computation tree. These rules
are specific to the language that is being studied.
In the last section of this chapter we will define such traversals for the interpreted constants of
\pcf\ and \ialgol.

From now on, we consider a simply-typed $\lambda$-calculus language extended with
higher-order interpreted constants for which some constant traversal rules have been defined
and we take the following condition as a prerequisite:
\begin{center}
  \textbf{(Condition WB)} The constant traversal rules are well-behaved.
\end{center}


\subsubsection{Some properties of traversals}

\begin{prop}
\label{prop:pviewtrav_is_path}
Let $t$ be a traversal. Then:
\begin{itemize}
\item[(i)] $t$ is a well-defined and well-bracketed justified sequence;
\item[(ii)] $?(t)$ is a well-defined justified sequence verifying alternation, P-visibility and O-visibility;
\item[(iii)] $\pview{?(t)}$ is the path in the computation tree going from the root to the last node in $?(t)$.
\end{itemize}
\end{prop}
This is the counterpart of proposition 6 from
\cite{OngHoMchecking2006} which is proved by induction on the
traversal rules. This proof can be easily adapted to take into
account the constant rules (using the assumption that constants
rules are well-behaved) and the presence of value-leaves in the
traversal.
\begin{proof}
The proof of (i), (ii) and (iii) is done simultaneously by induction on the traversal rules. We consider the rules (Var) and (Lam) only. 

Rule (Var): we just give a partial proof of (i). See proposition 6 from \cite{OngHoMchecking2006} for the details of (i), (ii) and (iii). We have to show that in the second case of the (Var) rule, where $p$ is a variable node $y$, the node $\lambda \overline{x}$ has necessarily been added to the traversal $t_{\leq y}$ using the (Var) rule. This is immediate since if the rule (InputVar) was used to produce $t_{<y} \cdot y \cdot \lambda \overline{x}$ this would imply that $\lambda \overline{x}$ is hereditarily justified by the root which in turn implies that $x_i$ is an input-variable. Hence reaching a contradiction.

Rule (Lam): we need to show that $n$'s enabler occurs only once in the P-view at that point. By the induction hypothesis we have (by (iii)) that $\pview{?(t \cdot \lambda \overline{\xi})}$ is a path in the computation tree from the root to $\lambda \overline{\xi}$. $n$'s enabler occurs only once in this path: it is precisely it's binding node. Therefore the traversal $t \cdot \lambda \overline{\xi} \cdot n$ is well-defined and $?(t \cdot \lambda \overline{\xi} \cdot n)$ satisfies P-visibility i.e.\ we have proved (i) and (ii). Since $n$ is a child of $\lambda \overline{\xi}$ we also have (iii).
\end{proof}

%In particular to prove that the copy-cat rules are well-defined, one needs to ensure that
%if the last two unmatched nodes are $y$ and $\lambda \overline{\xi}$ in that order, for some non input-variable node $y$ then necessary
%      $y$ and $\lambda \overline{\xi}$ are consecutive nodes in the traversal.
%    This is because in a traversal, a non input-variable $y$ is always followed by a lambda node and whenever this lambda node is answered
%    there is only one way to extend the traversal : by using the copy cat rule to answer the $y$ node.


\begin{dfn}[Traversal reduction]
Let $r$ be the root of the computation tree. We say that the
justified sequence of nodes \emph{$s$ is a reduction of the
traversal $t$} just when $s = t \upharpoonright r$.
\end{dfn}

Since @-nodes and $\Sigma$-constants do not have pointers, the
reduction of traversal contains only nodes in $N_\lambda \union
N_{var}$.


\begin{lem}
\label{lem:var_followedby_child} Let $M$ be a term in $\beta$-normal
form and $t$ be a traversal of $\tau(M)$. If $?(t) = u_1 \cdot
\rnode{m}{m} \cdot u_2 \cdot \rnode{n}{n}
\link[nodesep=1pt]{20}{n}{m}$ where $m \in (N_{var} \union N_{\Sigma}) \setminus (N^{\upharpoonright r}_{var} \union N^{\upharpoonright r}_{\Sigma})$ 
then $u_2 = \epsilon$.
\end{lem}
\begin{proof}
By induction on the traversal rules. The only relevant rules are (Var), (CCAnswer-var), (InputVar$^{val}$), (InputVar)
and the constant rules.
Since the term is in $\beta$-normal form, there is no @-node in $\tau(M)$ and therefore (Var) cannot be used. 
Since $m$ is not hereditarily justified by the root, it is not an input-variable and therefore the rules 
(InputVar$^{val}$) and (InputVar) cannot be used.
For the rule (CCAnswer-var) the result follows from the well-bracketedness of traversals.
For constant rules, the result follows from the well-behaviour of constant rules (condition WB).
\end{proof}

\begin{lem}[View of a traversal reduction]
\label{lem:redtrav_trav} Suppose that $M$ is a $\beta$-normal term and $t$ is a traversal of $\tau(M)$ then
\begin{itemize}
\item[(i)] $ \pview{?(t) \upharpoonright  r } = \pview{?(t)} \upharpoonright r$\ ;
\item[(ii)] if the last node in $t$ is hereditarily justified by $r$ then $ \oview{?(t) \upharpoonright r } = \oview{?(t)}$\ .
\end{itemize}
\end{lem}

\begin{proof}
(i) By induction. It is trivially true for the empty
traversal. Step case: consider a traversal $t$ and
suppose that the property (i) is verified for all traversal smaller
than $t$. There are three cases:
\begin{itemize}
\item If $?(t) = t' \cdot r$ then we have:
    \begin{align*}
    \pview{?(t)} \upharpoonright  r
        &=  \pview{t' \cdot r } \upharpoonright  r       & (\mbox{definition of } ?(t))\\
        &=  r \upharpoonright  r                         & (\mbox{def. P-view})\\
        &=  r                                                & (\mbox{def. operator $\upharpoonright$})\\
        &=  \pview{(t' \upharpoonright  r ) \cdot r }    & (\mbox{def. P-view})\\
        &=  \pview{(t' \cdot r)  \upharpoonright  r }    & (\mbox{def. operator $\upharpoonright$})\\
        &= \pview{?(t) \upharpoonright  r }                & (\mbox{definition of } ?(t))
    \end{align*}

\item If $?(t) = t' \cdot n$ where $n \in N_{var} \union N_{\Sigma}$ then we have:
%    \begin{equation}
%    \pview{?(t)} = \pview{t' \cdot n} = \pview{t'} \cdot n  \label{eq_tprime}
%    \end{equation}
%    \begin{itemize}
%    \item If $n \not \in N^{\upharpoonright r}_{var} \union N^{\upharpoonright r}_{\Sigma}$ then:
%    \begin{align*}
%    \pview{?(t)} \upharpoonright  r
%        &= (\pview{t'} \cdot n) \upharpoonright  r  & (\mbox{equation \ref{eq_tprime}}) \\
%        &= \pview{t'} \upharpoonright  r            & (n \mbox{ is not hereditarily justified by } r) \\
%        &= \pview{t' \upharpoonright  r }           & (\mbox{induction hypothesis}) \\
%        &= \pview{(t' \cdot n) \upharpoonright  r } & (n \mbox{ is not hereditarily justified by } r) \\
%        &= \pview{?(t) \upharpoonright  r  }           & (\mbox{definition of } ?(t))
%    \end{align*}
%
%    \item If $n \in N^{\upharpoonright r}_{var} \union N^{\upharpoonright r}_{\Sigma}$ then:
%    \begin{align*}
%    \pview{?(t)} \upharpoonright  r
%    &= (\pview{t'} \cdot n) \upharpoonright  r      & (\mbox{equation \ref{eq_tprime}}) \\
%    &= (\pview{t'} \upharpoonright  r  ) \cdot n    & (n \mbox{ is hereditarily justified by } r)\\
%    &= \pview{t' \upharpoonright  r } \cdot n       & (\mbox{induction hypothesis}) \\
%    &= \pview{(t' \upharpoonright  r ) \cdot n }    & (\mbox{P-view computation}) \\
%    &= \pview{(t' \cdot n) \upharpoonright  r  }    & (n \mbox{ is hereditarily justified by } r) \\
%    &= \pview{?(t) \upharpoonright  r  }               & (\mbox{definition of } ?(t))
%    \end{align*}
%    \end{itemize}
    \begin{align*}
    \pview{?(t)} \upharpoonright  r
&= \pview{t' \cdot n} \upharpoonright  r & (\mbox{definition of } ?(t))\\ 
        &= (\pview{t'} \cdot n) \upharpoonright  r  & (\mbox{P-view computation}) \\
        &= \pview{t'} \upharpoonright  r  \cdot (n \upharpoonright  r)            & (\mbox{definition of filtering $\upharpoonright$}) \\
        &= \pview{t' \upharpoonright  r } \cdot (n \upharpoonright  r)           & (\mbox{induction hypothesis}) \\
        &= \pview{t' \upharpoonright  r \cdot (n \upharpoonright  r) } & (\mbox{P-view computation, $n \in N_{var}$}) \\
        &= \pview{(t' \cdot n ) \upharpoonright  r  }           & (\mbox{definition of filtering $\upharpoonright$}) \\
        &= \pview{?(t) \upharpoonright  r  }
 & (\mbox{definition of } ?(t), u = \epsilon)
    \end{align*}


\item If $?(t) =  t' \cdot \rnode{m}{m} \cdot  u \cdot \rnode{lmd}{n}
    \link[nodesep=1pt]{30}{lmd}{m}$ where $m\in N_\lambda \setminus N^{\upharpoonright r}_\lambda$ then $u = \epsilon$ by lemma
    \ref{lem:var_followedby_child} and:
        \begin{align*}
        \pview{?(t)} \upharpoonright  r
        &= \pview{t' \cdot \rnode{m}{m} \cdot \rnode{n}{n}} \upharpoonright  r
               \link[nodesep=1pt]{60}{n}{m}                   & (u=\epsilon)\\
        &= (\pview{t'} \cdot \rnode{m}{m} \cdot \rnode{lmd}{n} ) \upharpoonright  r
               \link[nodesep=1pt]{60}{lmd}{m}                 & (\mbox{P-view computation}) \\
        &= \pview{t'} \upharpoonright  r                & (m, n \not\in N^{\upharpoonright r}) \\
        &= \pview{t' \upharpoonright  r }               & \mbox{(induction hypothesis)} \\
        &= \pview{ (t' \cdot \rnode{m}{m} \cdot \rnode{lmd}{n}) \upharpoonright r }
\link[nodesep=1pt,ncurv=0.7]{40}{lmd}{m}                                                          & (m, n \not\in N^{\upharpoonright r}) \\
        &= \pview{ ?(t) \upharpoonright r }                & \mbox{(def. of $?(t)$)}
        \end{align*}

\item If $?(t) =  t' \cdot \rnode{m}{m} \cdot u \cdot \rnode{lmd}{n}
    \link[nodesep=1pt]{30}{lmd}{m}$ where $m\in N^{\upharpoonright r}_\lambda$ then we have:
        \begin{align*}
        \pview{?(t)} \upharpoonright  r
        &= \pview{t' \cdot \rnode{m}{m} \cdot u \cdot \rnode{n}{n}} \upharpoonright  r
               \link[nodesep=1pt]{40}{n}{m}                   & (\mbox{definition of } ?(t))\\
        &= (\pview{t'} \cdot \rnode{m}{m} \cdot  \rnode{lmd}{n} ) \upharpoonright  r
               \link[nodesep=1pt]{60}{lmd}{m}                 & (\mbox{P-view computation}) \\
        &= \pview{t'} \upharpoonright  r \cdot \rnode{m}{m} \cdot  \rnode{lmd}{n} 
               \link[nodesep=1pt]{60}{lmd}{m}                 & (m, n \in N^{\upharpoonright r}) \\
        &= \pview{t'\upharpoonright r}  \cdot \rnode{m}{m} \cdot  \rnode{lmd}{n} 
               \link[nodesep=1pt]{60}{lmd}{m}                 & \mbox{(induction hypothesis)} \\
        &= \pview{ t' \upharpoonright r \cdot \rnode{m}{m} \cdot (u \upharpoonright r) \cdot \rnode{lmd}{n}}
\link[nodesep=1pt,ncurv=0.6]{35}{lmd}{m}                                                          & (\mbox{P-view computation}) \\
        &= \pview{ (t' \cdot \rnode{m}{m} \cdot u \cdot \rnode{lmd}{n}) \upharpoonright r }
\link[nodesep=1pt,ncurv=0.7]{35}{lmd}{m}                                                          & (m, n \in N^{\upharpoonright r}) \\
        &= \pview{ ?(t) \upharpoonright r }                & \mbox{(def. of $?(t)$)}
        \end{align*}
\end{itemize}
(ii) By a straightforward induction similar to (i).
\end{proof}

\begin{lem}[Traversal of $\beta$-normal terms]
\label{lem:betaeta_trav}
Let $M$ be a $\beta$-normal term, $r$ be the root of the tree $\tau(M)$ and
$t$ be a traversal of $\tau(M)$.
For any node $n$ occurring in $t$:
\begin{eqnarray*}
r \mbox{ does not hereditarily justify } n  \  \iff \   n \mbox{ is
hereditarily justified by some node in } N_\Sigma.
\end{eqnarray*}
%\begin{itemize}
%\item[(i)]
%for any node $n$ occurring in $t$:
%\begin{eqnarray*}
%r \mbox{ does not hereditarily justify } n  \  \iff \   n \mbox{
%is hereditarily justified by some node in } N_\Sigma;
%\end{eqnarray*}
%\item[(ii)] For any $\lambda$-node $n$ occurring in $t$, $t_{\geq n} \in \travset(\kappa(n))$,
%
% where $t_{\geq n}$ denotes the justified sequence of nodes obtained by taking the suffix of $t$ starting at $n$ and
% such that any dangling link going from a variable node to a node preceding $n$ is ``fixed'' into a pointer going to $n$.
% \end{itemize}
\end{lem}
\begin{proof}
%(i)
 In a computation tree, the only nodes that do not have justification pointer are:
the root $r$, @-nodes and $\Sigma$-constant nodes. But since $M$ is
in $\beta$-normal form, there is no @-node in the computation tree.
Hence nodes are either hereditarily justified by $r$ or hereditarily
justified by a node in $N_\Sigma$. Moreover $r$ is not in $N_\Sigma$
therefore the ``or'' is exclusive : a node cannot be hereditarily
justified at the same time by $r$ and by some node in $N_\Sigma$.

%(ii) Since $M$ is in $\beta$-normal, the rules (App) and (Var) cannot be used. Therefore the traversals
%follow an inductive exploration of the computation tree without making any ``jump''.
%Consequently, by taking the prefix of $t$ starting at a $\lambda$-node, we obtain
%a traversal of a sub-computation tree of $\tau(M)$. However by taking the prefix we obtain some dangling pointers.
%The ``fix'' applied to the dangling pointers correspond to the
%The formal proof is by an easy induction on the traversal rules. For the constant rules, we appeal to well-behaviour of the rules.

\end{proof}


\section{Game semantics of simply-typed $\lambda$-calculus with $\Sigma$-constants}
\label{sec:assumptions}

We are working in the general setting of an applied simply-typed
$\lambda$-calculus with a given set of higher-order constants
$\Sigma$. The operational semantics of these constants is given by
certain reduction rules. We assume that a fully abstract model of
the calculus is provided by means of a category of well-bracketed
games. For instance, if $\Sigma$ is the set of \pcf\ constants then
we work in the category $\mathcal{C}_{b}$ of well-bracketed defined
in section \ref{subsec:pcfgamemodel} of the first chapter.

We will use the alternative representation of strategy defined in remark \ref{rem:atlern_strategy}: a
strategy is given by a prefix-closed set instead of an ``even length
prefix''-closed set. In practice this means that we replace the set
of plays $\sigma$ by $\sigma \union \textsf{dom}(\sigma)$. This
permits to avoid considerations on the parity of the length of
traversals when we show the correspondence between traversals and
game semantics. We write $\sem{\Gamma \vdash M : A}$ for the strategy denoting the simply-typed term
$\Gamma \vdash M : A$ and $\prefset(S)$ to denote the
prefix-closure of the set $S$.


\subsection{Relationship between computation trees and arenas}

\subsubsection{Example}
Consider the following term $M \equiv \lambda f z . (\lambda g x . f (f x)) (\lambda y. y) z$ of type $(o \typear o) \typear o \typear o$.
Its $\eta$-long normal form is $\lambda f z . (\lambda g x . f (f x)) (\lambda y. y) (\lambda .z)$.
The computation tree is:

$$
\tree{\lambda f z}
{ \tree{@}
    {
        \tree{\lambda g x}
            { \tree{f}{   \tree{\lambda}{ \tree{f}{  \tree{\lambda}{\TR{x}}} }  }
            }
        \tree{\lambda y}{\TR{y}}
        \tree{\lambda}{\TR{z}}
    }
}
$$

The arena for the type $(o \typear o) \typear o \typear o$ is:
$$\tree{q^1}
{
    \tree{q^3}
        {  \tree{q^4}
                {\TR{a^4_1} \TR{\ldots}}
            \TR{a^3_1} \TR{\ldots} }
    \tree{q^2}
    { \TR{a^2_1} \TR{a^2_2}\TR{\ldots} }
    \TR{a_1} \TR{a_2}\TR{\ldots}
}
$$

\newlength{\yNull}
\def\bow{\quad\psarc{->}(0,\yNull){1.5ex}{90}{270}}

The figure below represents the computation tree (left) and the
arena (right). The dashed line defines a partial function $\varphi$
from the set of nodes in the computation tree to the set of moves.
For simplicity, we now omit answers moves when representing arenas.
$$
\tree{ \Rnode{root} {\lambda f z}^{[1]} }
     {  \tree{@^{[2]}}
        {   \tree{\lambda g x ^{[3]}}
                { \tree{\Rnode{f}{f^{[6]}}}{  \tree{\Rnode{lmd}\lambda^{[7]}}{ \tree{\Rnode{f2}{f^{[8]}}} {\tree{\Rnode{lmd2}\lambda^{[9]}}{\TR{x^{[10]}}}}}  }
                }
            \tree{\lambda y ^{[4]}}{\TR{y}}
            \tree{\lambda ^{[5]}}{\TR{\Rnode{z}z}}
        }
    }
\hspace{3cm}
  \tree[levelsep=12ex]{ \Rnode{q1}q^1 }
    {   \pstree[levelsep=4ex]{\TR{\Rnode{q3}q^3}}{\TR{\Rnode{q4}q^4}}
        \TR{\Rnode{q2}q^2}
        \TR{\Rnode{q5}q^5}
    }
\psset{nodesep=1pt,arrows=->,arcangle=-20,arrowsize=2pt 1,linestyle=dashed,linewidth=0.3pt}
\ncline{->}{root}{q1} \aput*{:U}{\varphi}
\ncarc{->}{z}{q2}
\ncline{->}{f}{q3}
\ncline{->}{lmd}{q4}
\ncline{->}{f2}{q3}
\ncline{->}{lmd2}{q4}
$$

Consider the justified sequence of moves $s \in \sem{M}$:
\vspace{0.2cm}
 $$s =
\rnode{q1}{q}^1\ \rnode{q3}{q}^3\ \rnode{q4}{q}^4\ \rnode{q3b}{q}^3\ \rnode{q4b}{q}^4\ \rnode{q2}{q}^2
\link[offset=-3pt]{60}{q3}{q1}
\link[offset=-3pt,ncurv=0.5]{60}{q3b}{q1}
\link[offset=-3pt]{60}{q4}{q3}
\link[offset=-3pt]{60}{q4b}{q3b}
\link[offset=-3pt,ncurv=0.5]{60}{q2}{q1}
\in \sem{M}$$

There is a corresponding justified sequence of nodes in the computation tree:
\vspace{0.5cm}
$$r =
\rnode{q1}{\lambda f z} \cdot
\rnode{q3}{f}^{[6]} \cdot
\rnode{q4}{\lambda^{[7]}} \cdot
\rnode{q3b}{f}^{[8]} \cdot
\rnode{q4b}{\lambda^{[9]}} \cdot
\rnode{q2}{z}
\link[ncurv=1]{60}{q3}{q1}
\link[ncurv=1]{60}{q4}{q3}
\link[ncurv=0.5]{75}{q3b}{q1}
\link[ncurv=1]{50}{q4b}{q3b}
\link[ncurv=0.4]{80}{q2}{q1}$$
such that $s_i = \varphi(r_i)$ for all $i < |s|$.

The sequence $r$ is in fact the reduction of the following
traversal: \vspace*{1cm}
$$t = \rnode{q1}{\lambda f
z} \cdot \rnode{n2}{@^{[2]}} \cdot \rnode{n3}{\lambda g x^{[3]}}
\cdot \rnode{q3}{f}^{[6]} \cdot \rnode{q4}{\lambda^{[7]}} \cdot
\rnode{q3b}{f}^{[8]} \cdot \rnode{q4b}{\lambda^{[9]}} \cdot
\rnode{n8}{x^{[10]}} \cdot \rnode{n9}{\lambda^{[5]}} \cdot
\rnode{q2}{z} \link[ncurv=0.6]{60}{q3}{q1}
\link[ncurv=1]{60}{q4}{q3} \link[ncurv=0.4]{75}{q3b}{q1}
\link[ncurv=0.8]{70}{q4b}{q3b} \link[ncurv=0.4]{80}{q2}{q1}
\link[ncurv=0.4]{60}{n3}{n2} \link[ncurv=0.4]{60}{n8}{n3}
\link[ncurv=0.4]{60}{n9}{n2}.
$$

By representing side-by-side the computation tree and the type arena of a term in $\eta$-normal form we have observed
that some nodes of the computation tree can be mapped to question moves of the arena.
In the next section, we show how to define this mapping in a systematic manner.

\subsubsection{Formal definition}

Let us establish precisely the relationship between arenas of the
game semantics and the computation trees. Let $\Gamma \vdash M : A$
be a term in $\eta$-long normal form. The computation tree $\tau(M)$
is represented by a pair $(V,E)$ where $V$ is the set of vertices of
the trees and $E$ is the edges relation. $V = N \union L$ where $N$
is the set of nodes and $L$ is the set of value-leaves.
The relation $E \subseteq V \times V$ gives the parent-child relation on the vertices of the tree.
We write $V_\$$ for $N_\$ \union (E(N_\$) \inter L)$ where $\$$ ranges over $\{@, var, \Sigma, fv \}$.


Let $\mathcal{D}$ be the set of values of the base type $o$. If $n$ is a node in $N$ then the value-leaves attached to the node $n$ are written $v_n$ where $v$ ranges in $\mathcal{D}$.
Similarly, if $q$ is a question in $\sem{A}$ then the answer moves enabled by $q$ are written $v_q$ where $v$ ranges in $\mathcal{D}$.

If $A$ is an arena and $q$ is a move in $A$ then we write $A_q$ to
denote the subarena of $A$ rooted at $q$.

\begin{dfn}[Mapping from nodes to moves]
\label{def:phi_procedure}
For any term $\Gamma \vdash M :T$, we define a function $\psi_M^{n,q}$ where the parameter $n$ is a node of the computation tree $\tau(M)$ and the parameter $q$ is a question move of the arena $\sem{T}$ such that $q$ and $n$ are of type $(A_1. \ldots A_p, o)$ for some $p\geq 0$.
Let $\vdash( q ) = \{ q^1, \ldots, q^p \} \union \{  v_q : v \in \mathcal{D} \}$.

The function $\psi_N^{n,q}$ from $V^{\upharpoonright n}$ to $\sem{T}$ is defined as follows:
\noindent
\begin{itemize}
\item[case 1] If $p=0$ ($n$ is labelled with $\lambda$ or with a ground type variable) then
        $$\psi_M^{n,q} = \{ n \mapsto q \} \quad \union \quad  \{ v_n \mapsto v_q \ | \ v \in \mathcal{D} \}$$

\item[case 2]  If $p\geq 1$ and $n \in N_{\lambda}$ with $n$ labelled $\lambda \overline{\xi} = \lambda \xi_1 \ldots \xi_p$ and with a child node labelled $\alpha$ then the computation tree and the arena $\sem{T}$ have the following forms (value-leaves and answer moves are not represented for simplicity):
    $$ \tree{ \Rnode{r}\lambda \overline{\xi}  ^{[n]}}
        {
            \tree[levelsep=6ex]{\alpha}
            {   \TR{\ldots} \TR{\ldots} \TR{\ldots}
            }
        }
    \hspace{3cm}
    \tree{ \Rnode{q0}q }
        {
            \tree[linestyle=dotted]{q^1}{\TR{} \TR{} }
            \tree[linestyle=dotted]{q^2}{\TR{} \TR{} }
            \TR{\ldots}
            \tree[linestyle=dotted]{q^p}{\TR{} \TR{} }
        }
    \psset{nodesep=1pt,arrows=->,arcangle=-20,arrowsize=2pt 1,linestyle=dashed,linewidth=0.3pt}
    \ncline{->}{r}{q0}
    \ncarc{->}{q2}{z}
    \ncline{->}{q3}{f}
    \ncline{->}{q4}{lmd}
    \ncline{->}{q3}{f2}
    \ncline{->}{q4}{lmd2}
    $$

    For each abstracted variable $\xi_i$ there exists a corresponding question move $q^i$ of the same order in the arena. $\psi_M^{n,q}$ maps each free occurrence of a variable $\xi_i$ to the corresponding move $q^i$:
    $$
    \psi_M^{n,q} =  \{ n \mapsto q \} \quad  \union \quad  \{ v_n \mapsto v_q \ | \ v \in \mathcal{D} \}
                      \quad \union \quad  \Union_{\displaystyle m \in N | n \vdash_i m} \psi_M^{m, q^i}$$

\item[case 3] If $p\geq 1$ and $n\in N_{var}$ then $n$ is labelled with a variable $x:(A_1,\ldots,A_p,o)$
with children nodes $\lambda \overline{\eta}_1$, \ldots, $\lambda \overline{\eta}_p$. The computation tree $\tau(M)$ and the arena $\sem{T}$ have the following forms:
    $$\tree{\Rnode{r}{x^{[n]}}}
        {   \tree{\TR{\lambda \overline{\eta}_1}}{\vdots} \TR{\ldots}
        \tree{\TR{\lambda \overline{\eta}_p }}{\vdots}
        }
    \hspace{3cm}
    \tree{ \Rnode{q0}q }
        {
\tree[linestyle=dotted]{\Rnode{q1}{q^1}}{\TR{} \TR{} }
            \tree[linestyle=dotted]{\Rnode{q2}{q^2}}{\TR{} \TR{} }
            \TR{\ldots}
            \tree[linestyle=dotted]{\Rnode{qp}{q^p}}{\TR{} \TR{} }
        }
    \psset{nodesep=1pt,arrows=->,arcangle=-20,arrowsize=2pt 1,linestyle=dashed,linewidth=0.3pt}
    \ncline{->}{r}{q0}
    \ncarc{->}{q2}{z}
    \ncline{->}{q3}{f}
    \ncline{->}{q4}{lmd}
    \ncline{->}{q3}{f2}
    \ncline{->}{q4}{lmd2}
    $$

    $\psi_M^{n,q}$ maps each child node of $n$ to the corresponding question move $q^i$ of the same type
in the arena $\sem{T}$:
    $$\psi_M^{n,q} =
         \{ n \mapsto q \} \quad \union\quad \{ v_n \mapsto v_q \ | \ v \in \mathcal{D}   \} \quad\union\quad     \Union_{i=1..m} \psi_M^{ \lambda \overline{\eta}_i, q^i}
    $$
\end{itemize}

Note that $\psi_M^{n,q}$ is only a partial function from $V$ to $A$ since it is defined only
on nodes that are hereditarily justified by the root but not hereditarily justified by a free variable node. In other words, $\psi_M^{n,q}$ is undefined on nodes that are hereditarily justified by $N_{fv} \union N_@ \union N_\Sigma$.
\end{dfn}

We write $\mathcal{M}_M$ to denote the following disjoint union of arenas:
$$\mathcal{M}_M = \sem{\Gamma \rightarrow T} \quad \uplus \quad  \biguplus_{n \in N \inter E\relimg{N_@ \union N_\Sigma} } \sem{type(\kappa(n))}.$$

Moves in $\mathcal{M}_M$ are implicitly tagged so it is possible to recover the arena in which they belong.


\begin{dfn}[Total mapping from nodes to moves]
For a closed simply-typed term $\vdash M : T$
we define the total function $\varphi_M : V_\lambda \union V_{var} \rightarrow \mathcal{M}_M$ as:
\begin{align*}
\varphi_M =
        \psi_{\sem{\Gamma \rightarrow T}}^{r, q^0_{\sem{\Gamma \rightarrow T}}} \quad
    \union \quad
        \Union_{n \in N \inter E \relimg{N_@ \union N_\Sigma}}  \psi_{\sem{type(\kappa(n))}}^{n, q^0_{\sem{type(\kappa(n))}}} )
\end{align*}
where $q^0_A$ denotes the only initial question of the arena $A$ (arenas involved in the game semantics of simply-typed $\lambda$-calculus have a single  root).

For an open term $\Gamma \vdash M : T$ 
with $\Gamma = x_1:X_1 \ldots x_p : X_p$ we define
$\varphi_M$ as $\varphi_{\lambda x_1 \ldots x_n . M}$.

When there is no ambiguity we omit the subscript in $\varphi_M$.
\end{dfn}

Nodes of $\tau(M)$ are either hereditarily justified by the root, by
a @-node or by a $\Sigma$-node, therefore $\varphi_M$ is totally
defined on $V_\lambda \union V_{var} = V\setminus (V_@ \union
V_\Sigma)$.

\begin{exmp}
Consider the term $\lambda x . (\lambda g . g x) (\lambda y . y)$ with $x,y:o$ and $g:(o,o)$.
The diagram below represents the computation tree (middle), the arenas
$\sem{(o,o)\rightarrow o}$ (left), $\sem{o \rightarrow o}$ (right), $\sem{o\rightarrow o}$ (rightmost)
and the function $\varphi = f(\lambda x, q_{\lambda x}) \union f(\lambda g, q_{\lambda g}) \union f(\lambda y, q_{\lambda y})$
(dashed-lines).
$$
\psset{levelsep=4ex}
\pstree{\TR[name=root]{\lambda x}}
{
    \pstree{\TR[name=App]{@}}
    {
            \pstree{\TR[name=lg]{\lambda g}}
                { \pstree{\TR[name=lgg]{g}}{
                        \pstree{\TR[name=lgg1]{\lambda}}
                        { \TR[name=lgg1x]{x}  } } }
            \pstree{\TR[name=ly]{\lambda y}}
                    {\TR[name=lyy]{y}}
    }
}
\rput(5cm,-1cm){
  \pstree{\TR[name=A1lx]{q_{\lambda x}}}
        { \TR[name=A1x]{q_x} }
}
\rput(-6cm,-1.5cm){
    \pstree{\TR[name=A2lg]{q_{\lambda g}}}
    {
        \pstree{\TR[name=A2g]{q_g}}
        {  \TR[name=A2g1]{q_{g_1}}   }
    }}
\rput(2.5cm,-1.5cm){
    \pstree{\TR[name=A3ly]{q_{\lambda y}}}
        { \TR[name=A3y]{q_y}
        }
}
\psset{nodesep=1pt,arrows=->,arcangle=-20,arrowsize=2pt 1,linestyle=dashed,linewidth=0.3pt}
\ncline{->}{root}{A1lx} \mput*{f(\lambda x, q_{\lambda x})}
\ncarc{->}{lgg1x}{A1x}
\ncline{->}{lg}{A2lg} \mput*{f(\lambda g, q_{\lambda g})}
\ncline{->}{lgg}{A2g}
\ncline{->}{lgg1}{A2g1}
\ncline{->}{ly}{A3ly} \mput*{f(\lambda y, q_{\lambda y})}
\ncline{->}{lyy}{A3y}
$$
\end{exmp}

The following properties are immediate consequences of the definition of the procedure $f$:
\begin{property} \
\label{proper:phi_conserve_order}
\begin{itemize}
\item[(i)] $\varphi$ maps $\lambda$-nodes to O-questions, variable nodes to
P-questions, value-leaves of $\lambda$-nodes to P-answers and
value-leaves of variable nodes to O-answers;
\item[(ii)] $\varphi$ maps nodes of a given order to moves of the same order.
\end{itemize}
\end{property}
Remark: we recall that in definition \ref{def:nodeorder}, the
node-order is defined differently for the root $\lambda$-node and
other $\lambda$-nodes. This convention was chosen to guarantee that
property (ii) holds.

By extension, the function $\varphi$ is also defined on justified
sequences of nodes: if $t = t_0 t_1 \ldots$ is a justified sequence
of nodes in $V_\lambda \union V_{var}$ then $\varphi(t)$ is defined
to be the following sequence of moves:
$$\varphi(t) = \varphi(t_0)\ \varphi(t_1)\  \varphi(t_2) \ldots$$
where the pointers of $\varphi(t)$ are defined to be exactly those
of $t$. This definition implies that $\varphi : (V_\lambda \union
V_{var})^* \rightarrow \mathcal{M}^*$ regarded as a function from
pointer-less sequences of nodes to pointer-less sequences of moves
is a monoid homomorphism.

\begin{property}
\label{proper:phi_pview} Let $t$ be a justified sequence of nodes. The following properties hold:
\begin{itemize}
\item[(i)] $\varphi(t)$ and $t$ have the same pointers;
\item[(ii)] the P-view of $\varphi(t)$ and the P-view of $t$ are computed
identically: the set of indices of elements that must be removed
from both sequences in order to obtain their P-view is the same;
\item[(iii)] the O-view of $\varphi(t)$ and the O-view of $t$ are computed identically;
\item[(iv)] if $t$ is a justified sequence of nodes in $V_\lambda \union V_{var}$ then $?(\varphi(t)) =
\varphi(?(t))$,
\end{itemize}
where $?(\varphi(t))$ denotes the subsequence of $\varphi(t)$ consisting of the unanswered questions
and $?(t)$ denotes the subsequence of $t$ consisting of the unmatched nodes (see the
definition in section \ref{sec:adding_value_leaves}).
\end{property}
\begin{proof}
(i): By definition of $\varphi$, $t$ and $\varphi(t)$ have the same
pointers;

(ii) and (iii): $\varphi$ maps lambda nodes to O-question,
non-lambda nodes to P-question, value-leaves of lambda nodes to P-answers and
value-leaves of non-lambda to O-answers. Therefore since $t$ and $\varphi(t)$ have the
same pointers, the computations of the P-view (resp. O-view) of the
sequence of moves and the P-view (resp. O-view) of the sequence of
nodes follow the same steps;

(iv) is a consequence of (i).

\end{proof}


\subsection{Category of interaction games}
\label{sec:interaction_semantics}

In game semantics, strategy composition is achieved by performing a
CSP-like ``composition + hiding''. It is possible to define an
alternative semantics where the internal moves are not hidden when
performing composition. This semantics is named \emph{revealed semantics} in \cite{willgreenlandthesis} and \emph{interaction}
semantics in \cite{DBLP:conf/sas/DimovskiGL05}.

In addition to the moves of the standard semantics, the interaction semantics contains certain
internal moves of the computation.
Consequently, the interaction semantics depends on the syntactical structure of the term and therefore cannot
lead to a full abstraction result. However this semantics will prove to be useful to identify
a correspondence between the game semantics
of a term and the traversals of its computation tree.

We will be interested in the interaction semantics computed from the
$\eta$-normal form of a term. However we do not want to keep all the internal moves. We will only keep the internal
moves that are produced when composing two subterms of the computation tree that are joined by an @-node.
This means that when computing the strategy of
$y N_1 \ldots N_p$ where $y$ is a variable, we keep the internal moves of $N_1$, \ldots, $N_p$, but
we omit the internal moves produced by the copy-cat projection strategy denoting $y$.

\begin{dfn}[Type-tree]
We call \emph{type decomposition tree} or \emph{type-tree}, a tree whose leaves are labelled with linear simple types
and nodes are labelled with symbol in $\{ ;, \times, \otimes, \dagger, \Lambda \}$.

Nodes labelled $;$, $\times$ or $\otimes$ are binary nodes and nodes labelled $\dagger$ or $\Lambda$ are unary nodes.

Every node or leaf of the tree has a linear type, this type is determined by the
structure of the tree as follows:
\begin{itemize}
\item a leaf has the type of its label;

\item a $\dagger$-node with the child node of type $!A \multimap B$ has type $!A \multimap !B$;

\item a $\Lambda$-node with the child node of type $A \otimes B \multimap C$ has type $A \multimap (B \multimap C)$;

\item a $\times$-node with two children nodes of type $A$
and $B$ has type $A \times B$;

\item a $\otimes$-node with two children nodes of type $A$
and $B$ has type $A\otimes B$;

\item a $;$-node with two children nodes of type $A\multimap B$
and $B \multimap C$ has type $A \multimap C$.
\end{itemize}

For a type-tree to be well-defined, the type of the children nodes
must be compatible with the meaning of the node, for instance the
two children nodes of a ;-node must be of type
$A\multimap B$ and $B\multimap C$.

We write $type(T)$ to denote the type represented by the root of the tree $T$ and we say that $T$ is a \emph{valid tree decomposition} of $type(T)$.

If $T_1$ and $T_2$ are type-trees we write $T_1 \times T_2$ to denote the tree obtained by attaching $T_1$ and $T_2$ to a $\times$-node.
Similarly we use the notations $T_1 \otimes T_2$, $T_1 ; T_2$, $\Lambda(T_1)$ and  $T_1^\dagger$.
\end{dfn}


Let $T$ be a type-tree. Each leaf or node of type $A$ in $T$ can be mapped to the
(standard) arena $\sem{A}$. By taking the image of $T$ across this mapping we obtain a tree whose leaves and nodes are labelled by arenas.
This tree, written $\intersem{T}$, is called the \emph{interaction arena} of type $T$.
We write $root(\intersem{T})$ to denote the arena located at the root of the interaction arena $\intersem{T}$.

A \emph{revealed strategy} $\Sigma$ on the interaction arena $\intersem{T}$ is a composition of several standard strategies where certain internal moves are not hidden. Formally this can be defined
as follows:
\begin{dfn}[Revealed strategy]
A revealed strategy $\Sigma$ on a game $\intersem{T}$, written
$\Sigma: \intersem{T}$, is a type-tree $T$ where
\begin{itemize}
\item each leaf $\sem{A}$ of
$\intersem{T}$ is annotated with a (standard) strategy $\sigma$ on the
game $\sem{A}$;
\item each $;$-node is annotated with a set of indices $U \subseteq \nat$.
\end{itemize}
\end{dfn}

A $;$-node with children of type $A\multimap B$ and $B\multimap C$ is annotated with a set of indices $U$ indicating which components of $B$ should be uncovered when performing composition.
More precisely, if $B = B_0 \times \ldots \times B_l$ then the revealed strategy built by connecting two revealed strategies $\Sigma_1 : \intersem{A\multimap B}$ and $\Sigma_2 : \intersem{B\multimap C}$ 
using a $;$-node annotated with $U$ represents the
set of uncovered plays obtained
by performing the usual composition while ignoring and copying the internal moves already in $\Sigma_1$ 
and $\Sigma_2$ and preserving any internal
move produced by the composition in some component $B_k$ for $k \in U$. 

\begin{exmp}
The diagrams below represent a type-tree $T$ (left) the corresponding interaction arena $\intersem{T}$ (middle) and a revealed strategy $\Sigma$ (right):
$$
\pstree[levelsep=6ex]{\TR{;}}
        {
            \pstree[levelsep=6ex]{\TR{;}}
            { \TR{A\multimap B}
              \TR{B\multimap C}
            }
            \TR{C\multimap D}
        }
\hspace{1cm}
\pstree[levelsep=6ex]{\TR{;}}
        {
            \pstree[levelsep=6ex]{\TR{;}}
            { \TR{\sem{A\multimap B}}
              \TR{\sem{B\multimap C}}
            }
            \TR{\sem{C\multimap D}}
        }
\hspace{1cm}
\pstree[levelsep=6ex]{\TR{;^{\{0\}}}}
        {
            \pstree[levelsep=6ex]{\TR{;^{\{0\}}}}
            { \TR{A\multimap B^{\sigma_1}}
              \TR{B\multimap C^{\sigma_2}}
            }
            \TR{C\multimap D^{\sigma_3}}
        }
$$
\end{exmp}
A revealed strategy can also be written as an expression, for
instance the strategy represented above is given by the expression
$\Sigma = (\sigma_1 ;^{\{0\}} \sigma_2) ;^{\{0\}} \sigma_3$. We will
use the abbreviation $\Sigma_1 \fatsemi^U \Sigma_2$ for
$\Sigma_1^\dagger ; ^U \Sigma_2$.

\begin{dfn}[Composition of revealed strategies]
Suppose $\Sigma_1 : \intersem{T_1}$ and $\Sigma_2 : \intersem{T_2}$
are revealed strategies where $type(T_1) = A \multimap B$ and
$type(T_2) = B \multimap C$ then the \emph{interaction composition}
of $\Sigma_1$ and $\Sigma_2$ written $\Sigma_1 ; \Sigma_2$ is the
revealed strategy on $\intersem{T_1 ; T_2}$ obtained by copying the
annotation of the leaves and nodes from $\Sigma_1$ and $\Sigma_2$ to
the corresponding leaves and nodes of the type-tree $T_1 ; T_2$ and
by annotating the root node with $\emptyset$.
\end{dfn}

A play of the interaction semantics, called an \emph{uncovered
play}, is a play containing internal moves.
The moves are implicitly tagged so that it is possible to retrieve in which component
of which node or leaf-arenas the move belongs to. Note that a same move can belong to different node/leaf-arenas.
The internal moves of an interaction play on the game $\intersem{T}$ are those which do not
belong to the arena $root(\intersem{T})$.

For any uncovered play $s$ and any interaction arena $\intersem{T}$
we can define the filtering operator $s\upharpoonright \intersem{T}$ to be the
sequence of moves obtained from $s$ by keeping only the moves
belonging to a node or leaf-arena of $\intersem{T}$.


Revealed strategies can alternatively be represented by means of
sets of uncovered plays instead of annotated type-trees. This set is
defined inductively on the structure of the annotated type-tree
$\Sigma$ as follows:
\begin{itemize}
\item for a leaf $\sem{A}$ of $\Sigma$ annotated by $\sigma :\sem{A}$, it is just the set of plays of the standard strategy $\sigma$;
\item for a $\otimes$-node with two children strategies $\Sigma_1$ and $\Sigma_2$, it is the tensor product written $\Sigma_1 \otimes \Sigma_2$;
\item for a $\times$-node, it is the pairing written $\langle \Sigma_1, \Sigma_2 \rangle$;
\item for a $\dagger$-node with a child strategy $\Sigma$, it is the promotion written $\Sigma^\dagger$;
\item for a $\Lambda$-node with a child strategy $\Sigma$, it is the same set of plays with the moves retagged appropriately;

\item for a $;^U$-node, it is the ``uncovered-composition'' of $\Sigma_1 : \intersem{T_1}$ and $\Sigma_2 :\intersem{T_2}$ which is written $\Sigma_1
;^U \Sigma_1$ and defined as follows: suppose that $type(T_1) = A
\multimap B_0 \times \ldots \times B_l$ and $type(T_2) = B_0 \times
\ldots \times B_l \multimap C$ then $\Sigma_1 ;^U \Sigma_1$ is the
set of uncovered plays obtained by performing the usual composition
while ignoring and copying the internal moves from arenas in
$\intersem{T_1}$ or $\intersem{T_2}$ and preserving any internal
move produced by the composition in some component $B_k$ for $k \in
U$. Formally:
$$ \Sigma_1 \| \Sigma_2 = \{ u \in int(\intersem{T}) \ | \ u \upharpoonright \intersem{T_1} \in \Sigma_1 \mbox{ and } u \upharpoonright \intersem{T_2} \in \Sigma_2 \}$$
$$ \Sigma_1 ;^U \Sigma_2 = \{ cover(u,U) \ | \ u \in \Sigma_1 \| \Sigma_2 \}$$
where $int(\intersem{T})$ denotes the set of sequences of moves in (some arena of) $\intersem{T}$ 
and $cover(u,U)$ denotes the subsequence of $u$ obtained by removing nodes that are in $B_j$ for some $j \in ( 0..l ) \setminus U$;
\end{itemize}
where the tensor product, pairing and promotion are defined similarly as in the standard game semantics.

In chapter \ref{chap:gamesem} we defined the category of games whose objects are the arenas $\sem{A}$ for some linear type $A$ and morphisms are the strategies. We now define the category $\mathcal{I}$ of interaction games:
\begin{dfn}[Category of interaction games]
We write $\mathcal{I}$ for the category of interaction games whose objects are those of $\mathcal{C}$ i.e.\ the arenas
$\sem{A}$ for some linear type $A$, and morphisms are the revealed strategies: a morphism from $A$ to $B$ is a revealed strategy $\Sigma$ on some interaction arena $\intersem{T}$
such that $root(\intersem{T}) = \sem{!A\multimap B}$.

The composition of two morphisms $\Sigma_1$ and $\Sigma_2$ is given
by $\Sigma_1 \fatsemi \Sigma_2 = \Sigma_1^\dagger ; \Sigma_2$ where
$;$ denotes the revealed strategy composition. The identity on $A$
is the revealed strategy given by the single annotated leaf $\sem{!A
\multimap A}^{der_A}$.
\end{dfn}

It can be checked that this indeed defines a category. The constructions of the category $\mathcal{C}$ can be transposed to $\mathcal{I}$
making $\mathcal{I}$ a cartesian closed category.


\begin{dfn}[Valid strategy]
Consider a term $\Gamma \vdash M : A$ and a revealed strategy
$\Sigma : \intersem{T}$. We say that $\Sigma$ is a valid revealed
strategy for $M$ if $root(\intersem{T}) = \sem{\Gamma \rightarrow
A}$ or equivalently if $type(T) = \Gamma \rightarrow A$.
\end{dfn}


\subsubsection{Modeling the $\lambda$-calculus in $\mathcal{I}$}

We would like to use the category $\mathcal{I}$ to model terms of
the simply-typed lambda calculus.
Depending on the internal moves that we wish to hide, we obtain different possible interaction strategies for a given term.
We now fix a unique strategy denotation which is computed from the $\eta$-normal form of the term.

\begin{dfn}[Revealed denotation of a term]
\label{dfn:interactionstrategy_ofterms}
The \emph{revealed game denotation} or \emph{revealed
strategy} of $M$ written $\intersem{\Gamma \vdash M : A}$ is defined by structural induction on the $\eta$-long normal form of $M$ as follows:

Let $\overline{\xi} = \xi_1 : Y_1, \ldots \xi_n : Y_n$
and $z$ be a variable ranging in $\Gamma \union \overline{\xi}$. If $z\in \Gamma$ then $\pi_{z}$ denotes
the $i^{th}$ projection copycat strategy $\pi_i : \sem{\Gamma \union \overline{\xi}} \rightarrow \sem{X_i}$ for some $1 \leq i \leq |\Gamma|$. If $z = \xi_j$ then
$\pi_{z}$ denotes the $(n+j)^{th}$ projection $\pi_{n+j} : \sem{\Gamma \union \overline{\xi}} \rightarrow \sem{Y_j}$.
\begin{eqnarray*}
\intersem{\Gamma \vdash \lambda \overline{\xi} . z } &=& \Lambda^n(\pi_{z})  \\
\intersem{\Gamma \vdash \lambda \overline{\xi} . z N_1 \ldots N_p} &=& \Lambda^n(\langle \pi_z, \intersem{\Gamma \vdash N_1 : A_1}, \ldots, \intersem{\Gamma \vdash N_p : A_p}  \rangle \fatsemi ^{1..p} ev^p) \\
\intersem{\Gamma \vdash \lambda \overline{\xi}. f N_1 \ldots N_p} &=& \langle \intersem{\Gamma \vdash N_1}, \ldots, \intersem{\Gamma \vdash N_p} \rangle \fatsemi^{0..p-1} \sem{f} \\
\intersem{\Gamma \vdash \lambda \overline{\xi} . N_0 \ldots N_p} &=& \Lambda^n(\langle \intersem{\Gamma \vdash N_0 : A_0}, \ldots, \intersem{\Gamma \vdash N_p : A_p}  \rangle \fatsemi^{\{0..p\}} ev^p)
\end{eqnarray*}
where $\Gamma \vdash N_0 : (A_1,\ldots,A_p,B)$, $\Gamma \vdash z : (A_1,\ldots,A_p,B)$, $\Gamma \vdash N_k : A_k$ for $k\in 1..p$,
$f : (A_1,\ldots,A_p,B) \in \Sigma$ and $ev^p$ denotes the evaluation strategy with $p$ parameters.

We write $\intersem{\Gamma \rightarrow A}_M$ to denote the
interaction arena of the revealed strategy $\intersem{\Gamma \vdash
M : A}$.
\end{dfn}
Note that when computing $\intersem{z N_1 \ldots N_p}$, for some variable $z$, the internal moves of $N_1$, \ldots, $N_p$ are preserved but
we omit the internal moves produced by the copy-cat projection strategy denoting $z$.



%\begin{dfn}[Revealed denotation of a term]
%\label{dfn:interactionstrategy_ofterms} Let $\Gamma \vdash M : A$ be
%a term with $\Gamma = x_1:X_1, \ldots, x_k:X_k$. Let $\pi_i :
%\sem{\Gamma \rightarrow X_i}$ denote the $i$th projection copycat
%strategy and $ev^p$ denote the evaluation strategy with $p$
%parameters.
%
%The \emph{revealed game denotation of $M$} or \emph{revealed
%strategy of $M$} written $\intersem{\Gamma \vdash M : A}$ is the
%revealed strategy defined by structural induction on the computation
%tree $\tau(M)$ as follows:
%
%\begin{tabularx}{14cm}{cX}
%$\tree[levelsep=6ex]{\lambda \xi_1\ldots \xi_n}{\TR{x_i}}$ &
%       $\intersem{M} = \Lambda^n(\pi_i)$ \\ \hline
%$ \tree[levelsep=6ex]{\lambda \xi_1\ldots \xi_n}
%        { \tree[levelsep=6ex]{x_i}
%            {   \TR{\tau(N_1)} \TR{\ldots} \TR{\tau(N_p)}}}
%    $
%&    where $\Gamma \vdash x_i : (A_1,\ldots,A_p,B)$ and $\Gamma \vdash N_j : A_j$ for $j\in 1..p$
%    $$\intersem{M} = \Lambda^n(\langle \pi_i, \intersem{\Gamma \vdash N_1 : A_1}, \ldots, \intersem{\Gamma \vdash N_p : A_p}  \rangle
%    \fatsemi ^{1..p} ev^p)$$
%\\ \hline
%$ \tree[levelsep=6ex]{\lambda \xi_1\ldots \xi_n}
%        { \tree[levelsep=6ex]{@}
%            {   \TR{\tau(N_0)} \TR{\ldots} \TR{\tau(N_p)}}}
%    $ &
%    where $\Gamma \vdash N_0 : (A_1,\ldots,A_p,B)$ and $\Gamma \vdash N_j : A_j$ for $j\in 1..p$
%    $$\intersem{M} = \Lambda^n(\langle \intersem{\Gamma \vdash N_0 : A_0}, \ldots, \intersem{\Gamma \vdash N_p : A_p}  \rangle
%    \fatsemi^{0..p} ev^p)$$
%\end{tabularx}
%\vspace{10pt}
%
%We write $\intersem{\Gamma \rightarrow A}_M$ to denote the
%interaction arena of the revealed strategy $\intersem{\Gamma \vdash
%M : A}$.
%\end{dfn}
%


\begin{exmp}
Consider the term $\lambda x . (\lambda f . f x) (\lambda y . y)$.
Its computation tree is:
$$
\tree{\lambda x} {
    \pstree[levelsep=4ex]{\TR{@}}
    {       \pstree[levelsep=4ex]{\TR{\lambda f}}
                { \tree{f}{  \tree{\lambda}{ \TR{x}  } } }
            \pstree[levelsep=4ex]{\TR{\lambda y}}
                    {\TR{y}}
    } }
$$
and its revealed strategy is $\langle \sem{ x:X \vdash \lambda f . f
x} , \sem{ x:X \vdash \lambda y . y} \rangle \fatsemi^{\{0,1\}}
ev_2$.
\end{exmp}


\subsubsection{From interaction semantics to standard semantics and vice-versa}

In the standard semantics, given two strategies $\sigma : A \rightarrow B$, $\tau : B \rightarrow C$ and
a sequence $s \in \sigma \fatsemi \tau$, it is possible to (uniquely) recover the internal moves. The uncovered sequence is written
${\bf u}(s, \sigma, \tau)$. The algorithm to obtain this unique uncovering is given in part II of \cite{hylandong_pcf}.

Given a term $M$, we can completely uncover the internal moves of a
sequence $s\in\sem{M}$ by performing the uncovering recursively at
every @-node of the computation tree. This operation is called
\emph{full-uncovering with respect to $M$}.

Conversely, the standard semantics can be recovered from the
interaction semantics by filtering the moves, keeping only those
played in the root arena:
\begin{eqnarray}
 \sem{\Gamma \vdash M : A} = \intersem{\Gamma \vdash M : A} \upharpoonright \sem{\Gamma \rightarrow T} \label{eqn:int_std_gamsem}
\end{eqnarray}


\subsubsection{Full abstraction}

Let $\mathcal{I'}$ denote lluf sub-category of $\mathcal{I}$
consisting only of strategies $\Sigma$ with a single annotated leaf
and no nodes. We have the following lemma:
\begin{lem}[$\mathcal{I'}$ is isomorphic to $\mathcal{C}$]
$\mathcal{I'} \cong \mathcal{C}$
\end{lem}
\begin{proof}
We define the functor $F:\mathcal{I'} \rightarrow \mathcal{C}$
by $F(A) = A$ for any object $A\in \mathcal{I'}$ and for $\Sigma \in \mathcal{I'}(A,B)$,
$F(\Sigma)$ is defined to be the annotation $\sigma$ of the only leaf in $\Sigma$.
The functor $G:\mathcal{C} \rightarrow \mathcal{I'}$ is defined by
$G(A) = A$ for any object $A\in \mathcal{C}$ and for $\sigma \in \mathcal{C}(A,B)$,
$G(\sigma)$ is the tree formed with the single annotated leaf $\sem{A}^\sigma$.
Then $F;G =id_{\mathcal{I'}}$ and $G;F =id_{\mathcal{C}}$.
\end{proof}

Consequently the lluf sub-category $\mathcal{I'}$ is fully abstract for the simply-typed lambda calculus.
Note that this is a major difference with $\mathcal{I}$ which is not fully-abstract since there may be several maps denoting a given
term.





\subsection{The correspondence theorem for the simply-typed $\lambda$-calculus without interpreted constants}
In this section, we establish a
connection between the interaction semantics of a simply-typed term without constants ($\Sigma = \emptyset$)
and the traversals of its computation tree. In the following, we fix a term $\Gamma \vdash M : T$.

\subsubsection{Removing @-nodes from traversals}

When defining computation trees, it was necessary to introduce
application nodes (labelled @) in order to connect the operator and
the operand of an application. The presence of @-nodes has also
another advantage: it ensures that the lambda-nodes are all at even
level in the computation tree. Consequently a traversal respects
Alternation.

Application nodes are however redundant in the sense that they do
not play any role in the computation of the term. In other words,
the @-nodes occurring in traversals are superfluous. In fact it is
necessary to filter them out if we want to establish the
correspondence with the interaction game semantics.

\begin{dfn}[Filtering @-nodes in traversals]
\label{dfn:appnode_filter}
Let $t$ be a traversal of $\tau(M)$.
We write $t-@$ for the sequence of nodes with pointers obtained by
\begin{itemize}
\item removing from $t$ all @-nodes and value-leaves of a @-node;
\item replacing any link pointing to an @-node by a link pointing to the immediate predecessor of @ in $t$.
\end{itemize}

Suppose $u = t-@$ is a sequence of nodes obtained by applying the
previously defined transformation on the traversal $t$, then $t$ can
be partially recovered from $u$ by reinserting the @-nodes as
follows. For each @-node @ in the computation tree with parent node
denoted by $p$, we perform the following operations:
\begin{enumerate}
\item replace every occurrence of the pattern $p \cdot n$, where $n$ is a $\lambda$-nodes,
by $p \cdot @ \cdot n$;
\item replace any link in $u$ starting from a $\lambda$-node and pointing to $p$ by a link pointing to the inserted @-node;
\item if there is an occurrence in $u$ of a value-leaf $v_p$ pointing to $p$ then insert a value-leaf $v_@$
immediately before $v_p$ and make it point to the node immediately
following $p$ (which is also the $@$-node that we inserted in 1).
\end{enumerate}
We write $u+@$ for this second transformation.
\end{dfn}
These transformations are well-defined because in a traversal, an @-node
always occurs in-between two nodes $n_1$ and $n_2$ such that  $n_1$ is the parent node of @
and $n_2$ is the first child node of @ in the computation tree:
$$      \pstree[levelsep=4ex]{\TR{n_1}\treelabel{0} }
        {
            \pstree[levelsep=3ex]{\TR{@}}
            {
                \tree{n_2}{\vdots}
                \TR[edge=\dedge]{}
                \TR[edge=\dedge]{}
            }
        }
$$
Remark: $t-@$ is not a proper justified sequence
since after removing a @-node, any $\lambda$-node justified by @ will become
justified by the parent of @ which is also a $\lambda$-node.

The following lemma follows directly from the definition:
\begin{lem}
\label{lem:minus_at_plus_at}
For any traversal $t$ we have $(t-@)+@ \sqsubseteq t$ and if $t$ does not end with an @-node then
$(t-@)+@ = t$.
\end{lem}

Let $r$ denote the root of $\tau(M)$. We introduce the following notations:
\begin{eqnarray*}
\travset(M)^{-@} &=& \{ t - @ \ | \  t \in \travset(M) \} \\
\travset(M)^{\upharpoonright r} &=& \{ t  \upharpoonright r \ | \  t  \in \travset(M) \} \ .
\end{eqnarray*}

\begin{lem}
If $M$ is in $\beta$-normal form then $t = t \upharpoonright r = t - @$ for any $t \in \travset(M)$.
Consequently $\travset(M)^{-@} = \travset(M) =  \travset(M)^{\upharpoonright r }$\ .
\end{lem}
\begin{proof}
The computation tree of a  $\beta$-normal term does not contain any @-node therefore all the nodes are hereditarily justified by the root.
\end{proof}



\begin{lem}[Filtering lemma] 
\label{lem:varphi_filter}
For any traversal $t$ we have
$\varphi(t-@) \upharpoonright \sem{\Gamma \rightarrow T} = \varphi(t\upharpoonright r)$.
Consequently:
$$ \varphi(\travset^{-@}(M)) \upharpoonright \sem{\Gamma \rightarrow T} = \varphi(\travset^{\upharpoonright r}(M))\ .$$
\end{lem}
\begin{proof}
    From the definition of $\varphi$, the nodes of the computation tree that are mapped by $\varphi$
    to moves of the arena $\sem{\Gamma \rightarrow T}$ are exactly the nodes that are hereditarily justified by $r$.
    The result follows from the fact that @-nodes are not hereditarily justified by the root.
\end{proof}

The function $\varphi$ regarded as a function from the set of vertices $V_\lambda \union V_{var}$ of the computation tree to moves in arenas is not injective.
For instance the two occurrences of $x$ in the computation tree of the term $\lambda f x. f x x$ are mapped to the same question. However
the function $\varphi$ regarded as a function from sequences of nodes to sequences of moves is injective:
\begin{lem}[$\varphi$ is injective]
\label{lem:varphiinjective}
$\varphi$ regarded as a function defined on the set of
sequences of nodes is injective in the sense that for any two traversals $t_1$ and $t_2$:
\begin{itemize}
\item[(i)] if $\varphi (t_1 - @ ) = \varphi (t_2 - @ )$ then $t_1-@ =t_2 -@$\ ;
\item[(ii)] if $\varphi (t_1 \upharpoonright r ) = \varphi (t_2 \upharpoonright r )$ then $t_1\upharpoonright r = t_2\upharpoonright r$\ .
\end{itemize}
\end{lem}
\begin{proof}
For any node $n$ of a traversal $t$ we write $ptr(n)$ to denote the distance between $n$ and its justifier node in $t$. If $n$ has not link then we set $ptr(n)=0$. We also use the same notation for sequences of moves.

\begin{lem}[Preleminary lemma]
\label{lem:varphiinjective:prelem}
\begin{equation}
\left(
  \begin{array}{ll}
    t \cdot n_1, t \cdot n_2 \in \travset \\
    \zand\ n_1 \neq n_2
  \end{array}
\right) 
 \mbox{ implies } n_1,n_2 \in N^{\upharpoonright r}_{\lambda} \zand ( \varphi(n_1) \neq \varphi(n_2) \zor ptr(n_1) \neq ptr(n_2) ) \ . \end{equation}
\end{lem}
\begin{proof}
Let $t \cdot n_1, t \cdot n_2 \in \travset$.
First we remark that the traversal rules have a weak form of determinism which ensures that $n_1$ and $n_2$ belong to the same category of node i.e.\ they must be both in $N_{var}$, $N_@$ or $N_\lambda$.

Suppose that $n_1, n_2 \in N_@$ then $t \cdot n_1$ and $t \cdot n_2$ were formed using the (App) rule. Since this rule is deterministic we must have $n_1=n_2$ which violates the second hypothesis.


Suppose that $n_1,n_2\in N_{var}$. The traversals $t \cdot n_1$ and $t \cdot n_2$ must have been formed using either rule (Lam) or (App). But these two rules are deterministic and their domains of definition are disjoint. Hence again the second hypothesis is violated.

Suppose that $n_1,n_2\in N_\lambda$ then
the traversals $t \cdot n_1$ and $t \cdot n_2$ must have been formed using either rule (Root), (App), (Var) or (InputVar). Since all these rules have disjoint domains of definition, the same rule must have been use to form $t \cdot n_1$ and $t \cdot n_2$. Supposed that one of the rules (Root), (App) and (Var) has been used then since they are all deterministic we have $n_1=n_2$ which violates the second hypothesis. Consequently, the rule (InputVar) must have been used and therefore $n_1,n_2 \in N_\lambda^{\upharpoonright r}$. By definition of (InputVar), in order to have $n_1\neq n_2$ and $\varphi(n_1) = \varphi(n_2)$, the parent node of the last node in $t$ must occurs at more than one position in $\oview{t}$ and $n_1,n_2$ correspond to the child node of two different occurrences of that parent node in $\oview{t}$. But then the links associated to $n_1$ and $n_2$ will point to their respective occurrence of that parent node in $\oview{t}$ hence $ptr(n_1) \neq ptr(n_2)$.
\end{proof}

(i) Suppose that $t_1-@\neq t_2-@$ then necessarily $t_1 \neq t_2$. Therefore there are some sequences $t'$, $u_1$, $u_2$ and some nodes $n_1,n_2$ such that
 $t_1 = t' \cdot n_1 \cdot u_1$, $t_2 = t' \cdot n_2 \cdot u_2$ and either $n_1\neq n_2$ or $ptr(n_1) \neq ptr(n_2)$.

If $n_1 = n_2$ then $ptr(n_1) \neq ptr(n_2)$ therefore $n_1,n_2 \not\in N_@$ (otherwise $ptr(n_1) = 0 = ptr(n_2)$). Since $ptr(\varphi(n_1)) = ptr(n_1)$ and  $ptr(\varphi(n_2)) = ptr(n_2)$ we must have $\varphi(t' \cdot n_1) \neq \varphi(t' \cdot n_2)$. Since $n_1,n_2 \not\in N_@$ we also have $\varphi((t' \cdot n_1)-@) \neq \varphi((t' \cdot n_2)-@)$. Hence $\varphi(t_1-@) \neq \varphi(t_2-@)$.

If $n_1 \neq n_2$ then by Lemma \ref{lem:varphiinjective:prelem} we have $n_1,n_2 \not\in N_@$ and $\varphi(n_1) \neq \varphi(n_2)$ or $ptr(n_1) \neq ptr(n_2)$. 
Again this implies that $\varphi(t_1-@) \neq \varphi(t_2-@)$.


(ii) Suppose that $t \upharpoonright r \neq t' \upharpoonright r$ then necessarily $t \neq t'$ which in turn implies that for some sequences $t_1'$, $t_2'$, $u_1$, $u_2$ and some nodes $n_1 \neq n_2$ 
we have $t_1 = t' \cdot n_1 \cdot u_1$, $t_2 = t' \cdot n_2 \cdot u_2$ and either $n_1\neq n_2$ or $ptr(n_1) \neq ptr(n_2)$.

If $n_1 = n_2$ then $ptr(n_1) \neq ptr(n_2)$. An   analysis of the traversal rules shows that the rule (InputVar) is the only rule which can visit the same node with two different pointers. Hence $n_1,n_2 \in N_\lambda^{\upharpoonright r}$.
Therefore $\varphi( (t'\cdot n_1) \upharpoonright r ) = \varphi( (t'\upharpoonright r) \cdot n_1 )  \neq \varphi( (t'\upharpoonright r) \cdot n_2 )$. Hence    $\varphi( t_1\upharpoonright r ) \neq \varphi( t_2\upharpoonright r )$.

If $n_1 \neq n_2$ then we can use Lemma \ref{lem:varphiinjective:prelem}
to obtain $\varphi( t_1\upharpoonright r ) \neq \varphi( t_2\upharpoonright r )$.
\end{proof}

\begin{cor} \
\label{cor:varphi_bij}
\begin{itemize}
\item[(i)] $\varphi$ defines a bijection from $\travset(M)^{-@}$
to $\varphi(\travset(M)^{-@})$\ ;
\item[(ii)] $\varphi$ defines a bijection from $\travset(M)^{\upharpoonright r}$ to
$\varphi(\travset(M)^{\upharpoonright r})$\ .
\end{itemize}
\end{cor}

\subsubsection{The correspondence theorem}
We are now going to state and prove the correspondence theorem
for the simply-typed $\lambda$-calculus without interpreted constants ($\Sigma = \emptyset$).
The result extends immediately to the simply-typed $\lambda$-calculus with \emph{uninterpreted} constants by considering constants as being free variables.
We use the cartesian closed category of games $\mathcal{C}$ (defined in section \ref{subsec:pcfgamemodel} of the first chapter) as
a model of the simply-typed $\lambda$-calculus. We write $\sem{\Gamma \vdash M : A}$ for the strategy denoting the simply-typed term
$\Gamma \vdash M : A$.

\begin{prop}
\label{prop:rel_gamesem_trav} Let $\Gamma \vdash M : T$ be a term of
the simply-typed $\lambda$-calculus and $r$ be the root of
$\tau(M)$. We have:
\begin{itemize}
\item[(i)]  $\varphi_M(\travset(M)^{-@}) = \intersem{M}$ \ ;
\item[(ii)] $\varphi_M(\travset(M)^{\upharpoonright r}) = \sem{M}$ \ .
\end{itemize}
\end{prop}


\begin{rem} The proof that follows is quite tedious but the idea is simple. Let us give the intuition.
    We start by reducing the problem to the case of closed terms only. Then the proof proceeds by induction on the structure of the computation tree.
The base case is straightforward. Now consider an application $M$ with the following computation tree $\tau(M)$:
    $$ \tree[levelsep=4ex]{\lambda \overline{\xi}}
        { \tree[levelsep=4ex]{@}
            {   \TR{\tau(N_0)} \TR{\ldots} \TR{\tau(N_p)}}}
    $$

    A traversal of $\tau(M)$ proceeds as follows: it starts at the root $\lambda \overline{\xi}$ of the tree $\tau(M)$ (rule
    (Root)), it then passes the node @ (rule (Lam)).
    After this initialization part, it proceeds by traversing the term $N_0$ (rule (App)).
    At some point, while traversing $N_0$, some variable $y_i$ bound by the root of $N_0$ is visited. The traversal
    of $N_0$ is interrupted and there is a jump (rule (Var)) to the root of $\tau(N_i)$. The process goes on by traversing $\tau(N_i)$.
    When traversing $N_i$, if the traversal encounters a variable bound by the root of $\tau(N_i)$ then the traversal of $N_i$ is interrupted and
    the traversal of $N_0$ resumes.  This schema is repeated until the traversal of $\tau(N_0)$ is completed\footnote{Since we are considering
    simply-typed terms, the traversal does indeed terminate. However this will not be true anymore in the \pcf\ case.}.

    The traversal of $M$ is therefore made of an initialization part followed by an interleaving of a traversal of $N_0$ and
    several traversals of $N_i$ for $i=1..p$. This schema is reminiscent of the way the evaluation copycat map $ev$ works in game semantics.

    The key idea is that every time the traversal pauses the traversal of a subterm and switches to another one,
    the jump is permitted by one of the four copycat rules (Var), (CCAnswer-@), (CCAnswer-$\lambda$) or (CCAnswer-var).
    We show by (a second) induction that these copycat rules defines exactly what the copycat strategy $ev$ performs on sets of moves.

%    In the game semantics, the evaluation map (a copy-cat strategy) copies this opening move to an initial move $m_0$ in the game
%    $B_0$ and the game continues in $B_0$. We reflect this in the traversal : we make $t$ follow
%    the ``script'' given by the traversal $t^0_{m_0}$.
%    The rule (App) allow us to initiate this simulation  by visiting the  first move in $t^0_{m_0}$: the root of $\tau(N_0)$.
%
%    This simulation continues until it reaches a node $\alpha_0$ which is hereditarily justified by the root
%    $\tau(N_0)$: $\alpha_0$ is present in the reduction of traversal of $t^0_{m_0}$ therefore $\varphi_{N_0}(\alpha_0)$ is an un-hidden move played in $A_0$.
%
%    In the game semantics this corresponds to a move played in a component $A_k$ for some $k\in 1..p$ of
%    of the game $B_0$ in which case the evaluation map copies the move to an initial move $m_1$ in the corresponding component $B_k$.
%
%    To reflect this the traversal now opens up a new thread and simulates the traversal $t^k_{m_1}$.  Again, this simulation stops when we reach a node
%    $\alpha_1$ in $t^k_{m_1}$ which is hereditarily justified by the root of $\tau(N_k)$: $\alpha_1$ must be present in the reduction of traversal
%    of $t^k_{m_1}$ therefore $\varphi_{N_k}(\alpha_1)$ is an un-hidden move played in $A_k$.
%    In the game semantics, this move $\alpha$ is copied back to the component $B_k$ of the game $B_0$.
%
%    The traversal now resumes the simulation of $t^0_{m_0}$. And the process goes continuously.
\end{rem}

Let us fix some notation: we write $s\upharpoonright A,B$ for the
sequence obtained from $s$ by keeping only the moves that are in $A$ or $B$ and by removing any link pointing to a move that
has been removed.
If $m$ is an initial move, we write $s \upharpoonright m$ to
denote the thread of $s$ initiated by $m$, i.e. the sequence obtained from $s$ by keeping all the moves
hereditarily justified by $m$.
We also write $s \upharpoonright A,B,m$ where $m$ is an initial move
for the sequence obtained from $s \upharpoonright A,B$ by keeping
all moves hereditarily justified by $m$.



\begin{proof}
(i) Suppose $\Gamma = \xi_1:X_1,\ldots \xi_n:X_n$. Then we have:
\begin{eqnarray*}
\intersem{\Gamma \vdash M:T} &=& \Lambda^n( \intersem{\emptyset \vdash \lambda \xi_1\ldots \xi_n . M: (X_1,\ldots,X_n,T) } ) \\
        &\simeq& \intersem{\emptyset \vdash \lambda \xi_1\ldots \xi_n . M: (X_1,\ldots,X_n,T) }\ .
\end{eqnarray*}
Similarly the computation tree $\tau(M)$ is isomorphic to
$\tau(\lambda \xi_1\ldots \xi_n . M)$ (up to a renaming of the root
of the computation tree) therefore $\travset(M)$ is also isomorphic
to $\travset(\lambda \xi_1\ldots \xi_n . M)$. Hence we can make the
assumption that $M$ is a closed term. If we prove that the property
is true for all closed terms of a given height then it will be
automatically true for any open term of the same height.


Let us assume that $M$ is already in $\eta$-long normal form. We
proceed by induction on the height of the tree $\tau(M)$ and by
case analysis on the structure of the computation tree:
\begin{itemize}
  \item (abstraction of a variable): $M \equiv \lambda \overline{\xi} .
  x$.  Since $M$ is in $\eta$-long normal form, $x$ must be of ground type and since $M$ is
      closed we have $x = \xi_i \in \overline{\xi}$ for some $i$.
      Hence $\tau(M)$ has the following shape:
        $$ \tree[levelsep=6ex]{ \lambda \overline{\xi}^{[0]} }{\TR{\xi_i^{[1]}}}$$
        The arena is of the following form (only question moves are represented):
        $$ \tree{ q_0 }
        {   \tree[linestyle=dotted]{q^1}{\TR{} \TR{} }
            \tree[linestyle=dotted]{q^2}{\TR{} \TR{} }
            \TR{\ldots}
            \tree[linestyle=dotted]{q^n}{\TR{} \TR{} }
        }$$

        Let $\pi_i$ denote the $i$th projection of the interaction game
        semantics. We have:
        \begin{align*}
        \intersem{M} &= \intersem{\emptyset \vdash \lambda \overline{\xi} . \xi_i} \\
                     &= \Lambda^n(\intersem{\overline{\xi} \vdash  \xi_i}) \\
                     &= \Lambda^n(\pi_i) \\
                     &\cong \pi_i \\
                     &= \prefset(\{ q_0 \cdot q^i \cdot v_{q^i} \cdot v_{q_0} \ | \ v\in \mathcal{D}
                     \})\ .
        \end{align*}

        Since $M$ is in $\beta$-normal we have $\travset(M)^{-@} = \travset(M)$.
        It is easy to see that the set of traversals of $M$ is the set of prefix of
        the traversal $\lambda \overline{\xi} \cdot \xi_i \cdot v_{\xi_i} \cdot v_{\lambda \overline{\xi}}$:
        $$ \travset^{-@}(M) = \travset(M) = \prefset( \lambda \overline{\xi} \cdot \xi_i \cdot v_{\xi_i} \cdot v_{\lambda \overline{\xi}}) \ .
        $$

        The pointers of the traversal $\lambda \overline{\xi} \cdot \xi_i \cdot v_{\xi_i} \cdot
        v_{\lambda \overline{\xi}}$ are the same as the play $q_0 \cdot q^i \cdot v_{q^i} \cdot
        v_{q_0}$, therefore since $\varphi_M(\lambda \overline{\xi}) = q_0$ and $\varphi_M(\xi_i) =
        q^i$ we have:
        $$ \varphi_M(\travset^{-@}(M)) = \intersem{M}\ .$$


    \item (abstraction of an application): we have $M = \lambda \overline{\xi} . N_0 N_1 \ldots N_p$. Let $\Gamma$ be the context
    $\Gamma = \overline{\xi} : \overline{X}$. Then we have the following sequents:
    $\emptyset \vdash M : (X_1,\ldots,X_n,o)$,
    $\Gamma \vdash N_0 N_1 \ldots N_p : o$,
    $\Gamma \vdash N_i : B_i$ for $i\in 0..p$ with $B_0 = (B_1,\ldots,B_p,o)$ and $p\geq 1$.

    There are two subcases, either $N_0 \equiv \xi_i$ where $\alpha$ is a variable in $\overline{\xi}$ and the tree has the following form:
    $$ \tree[levelsep=6ex]{\lambda \overline{\xi}^{[0]}}
        { \tree[levelsep=6ex]{\xi_i^{[1]}}
            {   \TR{\tau(N_1)} \TR{\ldots} \TR{\tau(N_p)}}}
    $$
    or $N_0$ is not a variable and the tree $\tau(M)$ has the following form:
    $$ \tree[levelsep=6ex]{\lambda \overline{\xi}^{[0]}}
        { \tree[levelsep=6ex]{@^{[1]}}
            {
            \tree[levelsep=6ex]{\lambda y_1 \ldots y_p}{\ldots}
            \TR{\tau(N_1)} \TR{\ldots} \TR{\tau(N_p)}}}
    $$

    We only consider the second case since the first one can be treated
    similarly. Moreover we make the assumption that $p=1$. It is
    straightforward to generalize to any $p\geq1$.
    We write $\lambda \overline{z}$ to denote the root of the tree $\tau(N_1)$.


    We have:
    \begin{align*}
    \intersem{M}
        &=  \Lambda^n( \intersem{\Gamma \vdash N_0 N_1 : o} )
            & \mbox{(game semantics for abstraction)}\\
        &\cong  \intersem{\Gamma \vdash N_0 N_1 : o}
            & \mbox{(up to moves retagging)}\\
        &=  \langle \intersem{\Gamma \vdash N_0}, \intersem{\Gamma \vdash N_1} \rangle \fatsemi^{0..1} ev
            & \mbox{(game semantics for application)}\\
        &=  \langle \varphi_{N_0} (\travset^{-@}(N_0)), \varphi_{N_1}(\travset^{-@}(N_1) \rangle \fatsemi^{0..1} ev
            & \mbox{(induction hypothesis)}\\
        &=  \langle \varphi_{M} (\travset^{-@}(N_0)), \varphi_{M}(\travset^{-@}(N_1)) \rangle \fatsemi^{0..1} ev
            & \mbox{($\varphi_M = f(0,q_0) \union \varphi_{N_0} \union \varphi_{N_1}$)} \\
        &=  \underbrace{\langle \varphi_{M} (\travset^{-@}(N_0)), \varphi_{M}(\travset^{-@}(N_1)) \rangle}_{\sigma} \parallel ev
            & \mbox{($\fatsemi^{0..1}$ and $\parallel$ are the same operator)}
    \end{align*}


    The strategies $\sigma$ and $ev$ are defined on the arena $!A \multimap B$ and $!B \multimap C$ respectively where:
    \begin{eqnarray*}
        A &=& \intersem{\Gamma} = \intersem{X_1} \times \ldots \times \intersem{X_n}\\
        B &=& \intersem{B_0} \times \intersem{B_1} = \intersem{B_1' \rightarrow o'} \times \intersem{B_1} \\
        C &=& \intersem{o}
    \end{eqnarray*}

    We have $u \in \intersem{M} \cong \sigma^{\dag} \parallel ev$ if and only if
    \begin{eqnarray*}
      &&      \left\{
            \begin{array}{ll}
                u \in int(!A,!B,C)\\
                u \upharpoonright !A,!B  \in \sigma^\dagger \\
                u \upharpoonright !B,C  \in  ev
            \end{array}
            \right. \\
    & \mbox{or equivalently} & \left\{
    \begin{array}{ll}
        u \in int(!A,!B,C) \\
        \hbox{for any initial $m$ in $u \upharpoonright !A,!B$ there is $j \in 0..p$ such that } \\
        \left\{\begin{array}{ll}
            u \upharpoonright !A,B_j, m \in \varphi_{M} (\travset^{-@}(N_j)) \label{eq:def_z} \\
            u \upharpoonright !A, B_k,m = \epsilon \quad \mbox{ for every } k\neq j \label{eq:b}
        \end{array}
        \right.
    \end{array}
    \right.
    \end{eqnarray*}


    We first prove that $\intersem{M} \subseteq \varphi_{M}( \travset^{-@}(M)
    )$.


    Suppose $u \in \intersem{M}$. We give a constructive proof that
    there exists a sequence of nodes $t$ in $N$ such that $\varphi_M(t-@) = u$ by induction on the length of $u$.
    Let $q_o$ be the initial question of the arena $\sem{M}$ and $q_1$ the initial question of $\sem{N_0}$.

    Base cases:
    \begin{itemize}
    \item $u=\epsilon$ then $\varphi(\epsilon) = u$ where the traversal $\epsilon$ is formed with the rule ($\epsilon$).
    \item If $|u|=1$ then $u=q_0$ is the initial move in $C$ and $\varphi(\lambda \overline{\xi}) = u$. The traversal
    $\lambda \overline{\xi}$ is formed with the rule (Root).
    \end{itemize}

    Step cases: Suppose that $u' = \varphi_M(t'-@)$ and $u = u' \cdot m \in \intersem{M}$ with $|u|>1$ for some traversal $t'$ of $\tau(M)$.
    Let us write $m^1$ for the last move in $u'$.

    \begin{enumerate}
    \item Suppose $m \in C$. In $C$ there are no internal moves, the only moves of $C$ are therefore $q_0$ and
    $v_{q_0}$ for some $v\in\mathcal{D}$. But $q_0$ can occur only once in $u$, therefore since $|u|>1$ we must have $m = v_{q_0}$
    for some $v\in \mathcal{D}$.  Since $m$ is an answer move to the initial question, it must be
    the duplication  (performed by the copy-cat evaluation strategy) of the move $m^1$ played in $o'$.
    Hence $m^1=v_{q_1}$. By the induction hypothesis, $n'$ -- the last move in $t'$ -- is equal to
    $\varphi(m^1) = v_{\lambda y_1}$.

    By property \ref{proper:phi_pview}(iv), $?(u') = \varphi(?(t'-@))$ and
    since $q_0$ is the pending question in $u'$, the first node of $t'$ is also the pending node in $t'$.
    This permits us to use the rule (CCAnswer-$\lambda$) to produce the traversal $t = t' \cdot v_{\lambda \overline{\xi}}$
    where $v_{\lambda \overline{\xi}}$ points to the first node in $t'$. Clearly, $\varphi(t-@) = u$.



    \item Suppose that $m,m^1 \in A \union B_0$.
    The strategy $ev$ is responsible for switching thread in $B_0$ therefore, in the interaction semantics,
    there must be a copycat move in-between two moves belonging to two different threads.
    Since $m$ and $m^1$ are consecutive moves in the sequence $u$, they must belong to the same thread i.e. there are
    hereditarily justified  by the same initial $m_0$ in $B_0$.


    We then have $(u \upharpoonright !A, !B)\upharpoonright m_0 = \varphi_{N_0}(t_0-@)$ for some traversal $t_0$ of $N_0$.
    Consequently  $\varphi_{N_0}(n^1) = m^1$ and $\varphi_{N_0}(n) = m$
    where $n^1 \cdot n$ are the last two moves in $t_0-@$.

    $n$ points to some node in $t_0$ that also occurs in $t'$. Let us call $n^2$ this node.
    Since $(u \upharpoonright !A, !B)\upharpoonright m_0 = \varphi_{N_0}(t_0-@)$,
    $n_2$ must have the same position in $t'$ as the node justifying $m$ in $u'$.
    Hence we just need to take $t = t' \cdot n$ where $n$ points to $n^2$ in $t'$.

    The sequence $t$ is indeed a valid traversal of $\tau(M)$
    because the rule used by the traversal $t_0$
    of $\tau(N_0)$ to visit the node $n$ after $n^1$ can also be used by the traversal $t'$ of $\tau(M)$
    to visit $n$ after $n^1$.
    This can be checked formally by inspecting all the traversal rules. The key reason is that
    all the nodes in $t_0-@$ are present in $t'$ with the same pointers but with some nodes interleaved in between.
    However these interleaved nodes are inserted in a way that still permits to use the traversal rule.

    \item Suppose that $m,m^1 \in A \union B_1$.
    The proof is similar to the previous case.

    \item Suppose that $m \in A \union B_0$ and $m^1 \in A \union B_1$.

    $t$ is obtained from $t-@$ using the transformation $+@$. We apply the same transformation to $u$ in order
    to make $O$-questions and $P$-questions in $u$ match with $\lambda$-nodes and variable nodes in $t'$ respectively.
    We write this sequence $u+@$.
    The $+@$ operation inserts nodes in the sequence but not at the end,
    therefore $m^1$, the last move in $u'$, is also the last move in $u'+@$.
    Let us note $n^1$ for the last move in $t'$.

        \begin{enumerate}
        \item If $n^1$ is the application node @ then it must be the parent of the node $\lambda y_1$ since it
        is the only non-internal @-node present in $t'$.
        Therefore $t'=\lambda \overline{\xi} \cdot @$ and $u= q_0 \cdot m$.
        But $m$ is the copy of $q_0$ replicated by $ev$ in $o'$ therefore $m=q_1$.
        Applying the (App) rule on $t'$ produces the traversal $\lambda \overline{\xi} \cdot @ \cdot \lambda y_1$
        with $\varphi((\lambda \overline{\xi} \cdot @ \cdot \lambda y_1)-@ ) = q_0 \cdot q_1 = u$.

        \item If $n^1$ is a variable node then $m^1$ is a P-move and $m$ is an O-move
            and therefore $m$ is the copy of $m^1$ duplicated in $B_1$ by the evaluation strategy.
            Consequently, $m^1$ points to some $m^2$ and $m$ points to the node preceding $m^2$ denoted by $m^3$.
            The diagram below shows an example of such sequence:
                $$
                \begin{array}{cccccccc}
                & (B_1' &\rightarrow & o') & \times & B_1 & \rightarrow & o' \\
                O & &&&&&& \rnode{q0}{q_0 (\lambda \overline{\xi})} \\
                P & &&&&& \\
                O & && \rnode{q1}{q_1 (\lambda \overline{y})} \\
                P & \rnode{m3}{m^3 (y_1)} \\
                O & &&&& \rnode{m2}{m^2 (\lambda \overline{z})} \\
                P & &&&& \rnode{m1}{m^1 (z_i)} \\
                O & \rnode{m}{m} \\
                \end{array}
                \ncline[nodesep=3pt]{->}{q1}{q0} \mput*{@}
                \nccurve[nodesep=3pt,ncurv=2,angleA=180,angleB=180]{->}{m1}{m2}
                \ncarc[nodesep=3pt,ncurv=1,angleA=90,angleB=180]{->}{m3}{q1}
                \ncarc[nodesep=3pt,ncurv=1,angleA=90,angleB=180]{->}{m}{m3}
                \ncline[nodesep=3pt]{->}{m2}{q0}
                $$

        $t'$  and $u+@$ have the following forms:
        \begin{eqnarray*}
                t'&=& \ldots \cdot n^3 \cdot \rnode{n2}{n^2} \cdot \ldots \cdot \rnode{n1}{n^1} \\ \\
                u+@ &=& \ldots \cdot \rnode{m3}{m^3} \cdot \rnode{m2}{m^2} \cdot \ldots \cdot \rnode{m1}{m^1} \cdot \rnode{m}{m}
            \link{30}{m1}{m2} \link{30}{m}{m3}
            \link{30}{n1}{n2}
        \end{eqnarray*}

        Since $n^1$ is a variable node, $n^2$ must be a $\lambda$-node.
        $n^3$ could be either a variable node or an @-node. In fact $n^3$ is necessarily a variable node. Indeed,
        $n^3$ is mapped to $m^3$ by $\varphi_{N_0}$ and $m^3$ belongs to $\sem{B_i'}$ (i.e. it is not
        an internal move of $\intersem{B_i'}$). The function $\varphi_{N_0}$ is defined in such a way that
        only nodes which are hereditarily justified by the root of $\tau(N_0)$ are mapped to nodes in $\sem{B_1'}$.
        Hence $n^3$ is hereditarily justified by the root and consequently it cannot be an @-node.

        Hence $n^1$ is a variable node, $n^2$ is a $\lambda$-node and $n^3$ is a variable node. We
        can therefore apply the (Var) rule to $t'$ and we obtain a traversal of the following form:

        \begin{eqnarray*}
            t&=& \ldots \cdot \rnode{n3}{n^3} \cdot \rnode{n2}{n^2} \cdot \ldots \cdot \rnode{n1}{n^1} \cdot \rnode{n}{n}
            \link{30}{n1}{n2} \link{30}{n}{n3}
        \end{eqnarray*}

        We have $\varphi(t'-@) = u'$ by the induction hypothesis and $\varphi(n) = m$ by definition of $\varphi$.
        Therefore since $m$ and $n$ point to the same position we have $\varphi(t-@) = u$.

        \item If $n^1$ is the value-leaf of a variable node then we proceed the same way as in the previous case:
        $n^1$ is a value-leaf of the variable node $n^2$ and we can use the
        (CCAnswer-$\lambda$) rule to extend the traversal $t'$.

        \item Suppose that $n^1$ is a lambda node, in which case $m^1$ is an O-move, then
        necessarily, $m^1$ is a move copied by the evaluation strategy
         from $B_1'$ to $B_1$. The move following $m^1$ should also be played in $B_1$ before being copied
         back to $B_1'$ by the evaluation strategy. But since $m \in B_0$, this case does not happen.


        \item If $n^1$ is a value-leaf of a lambda node then $n^2$ is a lambda node and $n^3$ is a variable node.
        We can therefore use the rule (CCAnswer-var) or (CCAnswer-@) to extend the traversal $t'$.
        \end{enumerate}

    \item Suppose $m \in A \union B_1$ and $m^1 \in A \union B_0$ then
    the proof is similar to the previous case.
    \end{enumerate}


  For the converse, $\varphi_{M}( \travset^{-@}(M) ) \subseteq \intersem{M}$, it is an easy induction
  on the traversal rules. We omit the details here.
\end{itemize}

(ii) is an immediate consequence of (i):
\begin{align*}
\sem{M} &= \intersem{M} \upharpoonright \sem{\Gamma \rightarrow T} & \mbox{(eq. \ref{eqn:int_std_gamsem})} \\
        &= \varphi_M(\travset^{-@}(M)) \upharpoonright \sem{\Gamma \rightarrow T} & \mbox{(by (i))}\\
        &= \varphi_M(\travset^{\upharpoonright r}(M)) & \mbox{(lemma \ref{lem:varphi_filter})}
\end{align*}
\end{proof}


Putting corollary \ref{cor:varphi_bij} and proposition
\ref{prop:rel_gamesem_trav} together we obtain the following theorem
which establishes a correspondence between the game-denotation of a
term and the set of traversals of its computation tree:

\begin{thm}[The Correspondence Theorem]
\label{thm:correspondence}
 For any simply-typed term $\Gamma \vdash M$,
$\varphi_M$ defines a bijection from $\travset(M)^{\upharpoonright
r}$ to $\sem{M}$ and a bijection from $\travset(M)^{-@}$ to
$\intersem{M}$:
\begin{eqnarray*}
 \varphi_M  &:& \travset(\Gamma \vdash M)^{\upharpoonright r} \stackrel{\cong}{\longrightarrow} \sem{\Gamma \vdash M} \\
 \varphi_M  &:& \travset(\Gamma \vdash M)^{-@} \stackrel{\cong}{\longrightarrow} \intersem{\Gamma \vdash M}
\end{eqnarray*}

Moreover when $M$ is in $\beta$-normal form, for any traversal $t$, if $\varphi_M(t)$ is a \emph{maximal} play then $t$ is a \emph{maximal} traversal.
\end{thm}

\begin{proof}
The first part is an immediate consequence of corollary
\ref{cor:varphi_bij} and proposition
\ref{prop:rel_gamesem_trav}.

Finally, if $M$ is in $\beta$-normal form then
$\travset(M)^{\upharpoonright r} = \travset(M)$
therefore $\varphi$ is defined on $\travset(M)$. Let $t$ be a traversal such that $\varphi(t)$ is a maximal play. Let $t'$ be a traversal such that $t \sqsubseteq t'$. By monotonicity of $\varphi$ this implies $\varphi(t) \sqsubseteq \varphi(t')$ and since $\varphi(t)$ is maximal we have $\varphi(t) = \varphi(t')$. $\varphi$ being injective we conclude $t'=t$.
\end{proof}

The following diagram recapitulates the main results of this section:
$$
\xymatrix @C=6pc{
                                           & \travset(M)^{-@} \ar@/_/[dl]_{+@}  \ar[r]^{\varphi_M}_\cong & \intersem{M} \ar@/_/[dd]_{\_ \upharpoonright \sem{\Gamma\rightarrow T}} \\
\travset(M) \ar@/_/[ur]_{-@}^{} \ar[dr]^{\_ \upharpoonright r}  \\
                                           & \travset(M)^{\upharpoonright r} \ar[r]^{\varphi_M}_\cong & \sem{M} \ar@/_/[uu]^{\cong}_{\mbox{full uncovering}}
}
$$


% fourth chapter
\chapter{Game-semantic characterisation of safety}

Safety has been defined as a syntactical constraint. Since Game
Semantics is by essence syntax-independent, it seems difficult at
first sight to characterise Safety in a game-semantic manner.
However, with the help of the tools developed in the previous
chapter and using the Correspondence Theorem, we can interpret plays
of a strategy as sequences of nodes of some AST of the term.
Therefore it is now possible to investigate the impact of the Safety
restriction on Game Semantics.


The main theorem of this chapter (theorem
\ref{thm:safe_ptr_recoverable}) states that pointers in a play of
the strategy denotation of a safe term can be uniquely recovered
from O-questions' pointers and from the underlying sequence of moves. The proof is in several steps. We start by introducing the notion of
\emph{P-incrementally-justified strategies} and prove that for plays
of such strategies, pointers emanating from P-moves can be reconstructed uniquely from the underlying sequences of moves and from O-moves' pointers. We then introduce the notion of \emph{incrementally-bound computation trees} and prove that incremental-binding coincides with P-incremental-justification (proposition \ref{prop:incrbound_imp_incrjustified}). Finally, we show that safe simply-typed terms in $\beta$-normal form have incrementally-bound computation trees, consequently their game denotation is P-incrementally-justified.


The first section of this chapter is concerned only with the safe $\lambda$-calculus without interpreted constants. In the next
section we extend the result by taking into account the interpreted
constants of \pcf\ and \ialgol. We define the language safe \ialgol\
(resp. safe \pcf) to be the fragment of \ialgol\ (resp. \pcf) where
the application and abstraction rules are constrained the same way
as in the safe $\lambda$-calculus. We show that safe \pcf\ terms are
denoted by P-incrementally-justified strategies and we give the key
elements for a possible extension of the result to Safe Idealized
Algol.

\section{Safe $\lambda$-Calculus}
Let us consider the safe $\lambda$-calculus without interpreted
constants. Our aim is to prove that pointers in the game semantics
of safe terms can be uniquely recovered.

The example of section \ref{subsec:pointer_necessary} gives a good
intuition: in order to distinguish the terms $M_1 = \lambda f . f
(\lambda x . f (\lambda y .y ))$ and $M_2 = \lambda f . f (\lambda x
. f (\lambda y .x ))$ we have to keep the pointers in the plays of
strategies. However, if we limit ourselves to the safe
$\lambda$-Calculus then the ambiguity disappears because $M_1$ is
safe whereas $M_2$ is not (in the subterm $f (\lambda y . x)$, the
free variable $x$ has the same order as $y$ but $x$ is not
abstracted together with $y$).

\begin{dfn}[P-incremental-justification]
A strategy $\sigma$ on a game $A$ is
\emph{P-incrementally\-justified} if and only if for any sequence of
moves $s q \in P_A$ we have:
\begin{eqnarray*}
%s q \in \sigma \wedge q \mbox{ is a O-question } &\implies& \parbox[t]{9cm}{$q$ points to the last P-move in $\oview{?(s)}$ with order strictly greater than $\ord{q}$;} \\
s q \in \sigma \wedge q \mbox{ is a P-question } &\implies&
\parbox[t]{9cm}{$q$  points to the last O-move in $\pview{?(s)}$
with order strictly greater than $\ord{q}$.}
\end{eqnarray*}
\end{dfn}

\begin{lem}
\label{lem:incrjustified_pointers_uniqu_recover} Pointers emanating from P-moves are
superfluous for P-incrementally-justified strategies.
\end{lem}
\begin{proof}
Suppose $\sigma$ is a P-incrementally-justified strategy. We prove that pointers attached to P-moves in a play $s\in \sigma$ are uniquely recoverable by induction on the length of $s$. \noindent \emph{Base case}: if $|s| \leq 1$ then there is no pointer to recover.
\noindent \emph{Step case}: suppose $s m \in \sigma$. If $m$ is an answer move then by the well-bracketing condition $m$ points
to the last unanswered question in $s$. If $m$ is a P-question then by  P-incremental-justification of $\sigma$, $m$ points to the last O-move in
$\pview{?(s)}$ with order strictly greater than $\ord{q}$. Since we have access to O-moves' pointers, we can compute the P-view $\pview{?(s)}$.
Hence $m$'s pointer is uniquely recoverable.
\end{proof}

\begin{exmp}
\label{examp:evnotincrjust}
The denotation of the evaluation map $ev$ is not
P-incrementally-justified. Indeed consider the play $s = q_0 q_1 q_2
q_3 \in \sem{ev}$ shown on the diagram below:
$$\begin{array}{cccccccc}
(A & \implies & B) & \times  & A & \stackrel{ev}{\longrightarrow} & B \\
&&&&&& \rnode{q0}{q_0} \\
&& \rnode{q1}{q_1} \\
 \rnode{q2}{q_2} \\
 &&&&\rnode{q3}{q_3}
\end{array}
\ncline[nodesep=3pt,linewidth=0.5pt]{->}{q3}{q0}
\ncline[nodesep=3pt,linewidth=0.5pt]{->}{q1}{q0}
\ncline[nodesep=3pt,linewidth=0.5pt]{->}{q2}{q1}
$$
The order of the moves are as follows:  $\ord{q_3} = \ord{A}$,
$\ord{q_2} = \ord{A}$, $\ord{q_1} = \max( 1+\ord{A}, \ord{B})$ and
$\ord{q_0} = 1 + \ord{q_1}$. The last O-move in $?(\pview{s})= s$
with order strictly greater than $\ord{q_3}$ is $q_1$.
 But since $q_3$ points to $q_0$, $\sem{ev}$ is not P-incrementally-justified.
\end{exmp}


In a computation tree, a binder node always occurs in the path from the bound node to the root. We now introduce a class of computation trees in which binder nodes can be uniquely recovered from the order
of the nodes. We write $[n_1,n_2]$ to denote the path from node
$n_1$ to node $n_2$ if it exists and $]n_1,n_2]$ for the sequence of
nodes obtained by removing $n_1$ from $[n_1,n_2]$.

\begin{dfn}[Incrementally-bound computation tree]
A variable node $x$ of a computation tree is said to be
\emph{incrementally-bound} if either:
\begin{enumerate}
\item $x$ is \emph{bound} by the first $\lambda$-node in the path to the root that has
order strictly greater than $\ord{x}$. Formally:
$$ x \mbox{ bound by } n \quad \imp \quad n \in [r,x] \wedge \ord{n} > \ord{x} \wedge \forall \lambda\mbox{-node } n' \in ]n,x] . \ord{n'} \leq \ord{x},$$

\item $x$ is a \emph{free variable} and all the $\lambda$-nodes in the path to the root except the root have order
smaller or equal to $\ord{x}$. Formally:
$$ x \mbox{ free } \quad \imp \quad  \forall \lambda\mbox{-node } n' \in ]r,x] . \ord{n'} \leq \ord{x}$$
\end{enumerate}
where $r$ denotes the root of the computation tree.

A computation tree is said to be \emph{incrementally-bound} if all
the variable nodes are incrementally-bound.
\end{dfn}

\begin{prop}[Incremental-binding coincides with P-incremental-justification] \
\label{prop:incrbound_imp_incrjustified} Let $M$ be a $\beta$-normal
term.
\begin{enumerate}
\item[(i)] If $\tau(M)$ is incrementally-bound then $\sem{M}$ is P-incrementally-justified.
\item[(ii)] In the $\lambda$-calculus without interpreted constants, conversely, if $\sem{M}$ is P-incrementally-justified then
$\tau(M)$ is incrementally-bound.
\end{enumerate}
\end{prop}

\begin{proof}
Let $\Gamma \vdash M : A$ be a simply-typed term in $\beta$-normal
form and $r$ denotes the root of $\tau(M)$.

\noindent (i) Suppose that $\tau(M)$ is incrementally-bound.
Consider a justified sequence of move $s \in \sem{\Gamma \vdash M}$
ending with a P-question move $q$ (note that $q$ is also the last
question in $?(s)$). By proposition \ref{prop:rel_gamesem_trav},
there is a traversal $t$ of $\tau(M)$ such that $\varphi_{M}(t
\upharpoonright r) = s$. We assume that the last node $n$ of $t$ is
hereditarily justified by $r$ (otherwise we replace $t$ by its
longest prefix verifying this condition). Then $n$ is also the last
node in $?(t \upharpoonright r)$ and $t \upharpoonright r$.
\begin{itemize}
\item First case: $n$ is a variable node $x$ bound by a node $m$ occurring in $t$.

Since $\tau(M)$ is incrementally-bound,
$m$ is the last $\lambda$-node in $[r,n]$ of order strictly greater
than $\ord{n}$. By visibility, $m$ occurs in $\pview{?(t)}$ and
since $m$ is hereditarily justified by $r$ (because $n$ is) 
$m$ must occur in $\pview{?(t)} \upharpoonright r$ which in turn
is equal to $\pview{?(t) \upharpoonright r}$ (by lemma \ref{lem:redtrav_trav}(i), since $M$ is in $\beta$-normal form).

But $\pview{?(t) \upharpoonright r} = \pview{?(t \upharpoonright r)}$ is a subsequence of
$\pview{?(t)}$ which is equal to $[r,n]$ (by proposition \ref{prop:pviewtrav_is_path}), therefore $m$ is also the last $\lambda$-node in
$\pview{?(t \upharpoonright  r)}$ that has order strictly greater
than $\ord{n}$.

By property \ref{proper:phi_pview} (ii), the P-view of $?(s)$ and
the P-view of $?(t \upharpoonright r)$ are computed similarly and
have the same pointers. This means that node $n$ and  move $q$ both
point to the same position in the justified sequence
$\pview{?(t\upharpoonright r)}$ and $\pview{?(s)}$ respectively.

Finally, since $\varphi$ maps nodes of a given order to moves of the
same order (property \ref{proper:phi_conserve_order}), $q$ must
point to the last O-move in $\pview{?(s)}$ whose order is strictly
greater than $\ord{q}$.


\item  Second case: $n$ is a free variable node $x$. Then $n$ is enabled by the root which is the first node in $t$.
By definition of $\varphi$, $\varphi(n) = x$ must be a move enabled
by the initial move $q_0 = \varphi(r)$ in the arena $\sem{\Gamma
\rightarrow A}$. Therefore $\ord{q_0} > \ord{x}$. Since the
computation tree is incrementally-bound, all the $\lambda$-nodes in
$]r,n]$ have order smaller than $\ord{n}$. Therefore by the
correspondence theorem, all the O-moves in $\pview{?(s)}$ have order
smaller than $\ord{x}$.
\end{itemize}



%\item If $|s|$ is odd then $q$ is an O-move:
%
%$M$ is in $\beta$-normal form and $t$ is a traversal of $\tau(M)$
%whose last node $n$ is hereditarily justified by $r$. Therefore by
%lemma \ref{lem:redtrav_trav} (ii), $ \oview{?(t \upharpoonright r)}
%= \oview{?(t)}$.
%
%A lambda-node always points to its parent node in the computation
%tree. For terms in $\beta$-normal form, this parent node must be a
%variable node of order strictly greater than $\ord{n}$.
%
%By inspecting the formation rules for traversals (definition
%\ref{def:traversal}) we remark that a lambda-node occurring in a
%traversal always points to the last node with order strictly greater
%that $\ord{n}$ in the O-view of the sequence of unmatched nodes at
%that point (there are just two cases, $n$ points either to the
%preceding node or to the third previous node in $\oview{?(t)}$).
%
%Similarly, as in the P-move case, we conclude that $q$ points to the
%last question move in $\oview{?(s)}$ of order strictly greater than
%$\ord{q}$.
%\end{itemize}

\noindent (ii) Suppose that $M$ is $\beta$-normal and the strategy
$\sem{M}$ is P-incrementally-justified. Let $x$ be a variable node of
$\tau(M)$. Since $M$ is $\beta$-normal, by lemma
\ref{lem:betaeta_trav}, $x$ is either hereditarily justified by the root $r$ or by a constant in $N_\Sigma$. Since we are working in the simply-typed
$\lambda$-calculus without constants we have $N_\Sigma = \emptyset$ therefore $x$ is
hereditarily justified by $r$.


We remark that for terms in $\beta$-normal form, every variable node
occurring in the computation tree can be visited by some traversal
i.e. there exists a traversal of the form $t \cdot x$ in
$\travset(M)$. The correspondence theorem gives $\varphi((t \cdot x)
\upharpoonright r) = \varphi((t \upharpoonright r) \cdot x) \in
\sem{M}$. Since $\sem{M}$ is P-incrementally-justified, $\varphi(x)$
must point to the last O-move in $\pview{?(\varphi(t \upharpoonright
r))}$ with order strictly greater than $\ord{\varphi(x)}$.
Consequently $x$ points to the last $\lambda$-node in $\pview{?(t
\upharpoonright r)}$ with order strictly greater than $\ord{x}$. Moreover we
have:
\begin{align*}
\pview{?(t \upharpoonright r)} &= \pview{?(t) \upharpoonright r} = \pview{?(t)} \upharpoonright r & (\mbox{by lemma \ref{lem:redtrav_trav}}) \\
& = [r,x[\  \upharpoonright r & (\mbox{by proposition \ref{prop:pviewtrav_is_path}}) \\
& = [r,x[  & (\mbox{$M$ is in $\beta$-nf and $N_\Sigma = \emptyset$}). 
\end{align*}
Therefore if $x$ is a bound variable node then it is bound by the
last $\lambda$-node in $[r,x[$ with order strictly greater than
$\ord{x}$ and if $x$ is a free variable then it points to $r$ and
therefore all the $\lambda$-node in $]r,x[$ have order smaller than
$\ord{x}$. Hence $\tau(M)$ is incrementally-bound.
\end{proof}


\parpic[r]{
    \psset{levelsep=4ex}
    \pstree{\TR{$\lambda x^3$}}{\pstree{\TR{$f^2$}}{ \pstree{\TR{$\lambda y^1$}}{ \TR{$x^0$} }}}
}

\noindent \emph{Examples:} Consider the $\beta$-normal term $\lambda
x . f (\lambda y .x)$ where $x,y:o$ and $f:(o,o),o$. The figure on
the right represents the computation tree with the order of each
node in the exponent part. Since node $x$ of order $0$ is not bound
by the order 1 node $\lambda y$, $\tau(M)$ is not
incrementally-bound and by proposition
\ref{prop:incrbound_imp_incrjustified} $\sem{\lambda x . f (\lambda
y .x)}$ is not P-incrementally-justified. Similarly we can check that
$\sem{f (\lambda y .x)}$ is not P-incrementally-justified
whereas $\sem{\lambda y. x}$ is.
Also for any higher-order variable $x:A$, the computation tree
$\tau(x)$ is incrementally-bound, therefore the projection
strategies $\pi_i$ are P-incrementally-justified. From these examples
we observe that application does not preserve
P-incremental-justification ($\sem{f}$ and $\sem{\lambda y. x}$ are
P-incrementally-justified whereas $\sem{f (\lambda y .x)}$ is not).

These examples suggest that P-incremental-justification is not a compositional property. Since the
 evaluation map $ev$ is not P-incrementally-justified (see example \ref{examp:evnotincrjust}),
application cannot conserve P-incremental-justification in the general case. One interesting problem would be to find some condition under which  the composition of two P-incrementally-justified strategy remains
P-incrementally-justified.
Unfortunately we have not provided an answer to that question yet.


\begin{lem}[Safe terms have incrementally-bound computation trees]
\label{lem:safe_imp_incrbound} Let $\Gamma \vdash M$ be a
simply-typed term.
\begin{itemize}
\item[(i)] If $M$ is a safe term then $\tau(M)$ is incrementally-bound ;
\item[(ii)] conversely, if $M$ is \emph{closed} and $\tau(M)$ is incrementally-bound then the $\eta$-normal form of $M$ is safe.
\end{itemize}
\end{lem}
\begin{proof}
(i) Suppose that $M$ is safe. The safety property is preserved after
taking the $\eta$-long normal form, therefore $\tau(M)$ is the tree representation of a safe term.

In the safe $\lambda$-calculus, the variables in the context with the the lowest order must be all abstracted
at once when using the abstraction rule. Since the computation
tree merges consecutive abstractions into a single node,
any variable $x$ occurring free in the subtree rooted at a $\lambda$-node $\lambda \overline{\xi}$ different from the root
must have order greater or equal to $\ord{\lambda \overline{\xi}}$. Reciprocally, if a lambda node
$\lambda \overline{\xi}$ binds a variable node $x$ then
$\ord{\lambda \overline{\xi}} = 1+\max_{z\in\overline{\xi}} \ord{z} > \ord{x}$.

Let $x$ be a bound variable node. Its binder occurs in the path from $x$
to the root, therefore, according to the previous observation, $x$ must be bound
by the first $\lambda$-node occurring in $[r,x]$ with order strictly
greater than $\ord{x}$. Let $x$ be a free variable node then $x$ is not bound
by any of the $\lambda$-nodes occurring in $[r,x]$. Once again, by the previous observation, all
these $\lambda$-nodes except $r$ have order smaller than $\ord{x}$. Hence
$\tau$ is incrementally-bound.

(ii) Let $M$ be a closed term such that $\tau(M)$ is incrementally-bound.
We assume that $M$ is already in $\eta$-normal form.
We prove that $M$ is safe by induction on its structure. The base case $M =
\lambda \overline{\xi} . \alpha$ for some variable or constant
$\alpha$ is trivial.
\emph{Step case:} If $M = \lambda \overline{\xi} . N_1 \ldots N_p$.
Let $i$ range over $1..p$. $N_i$ can be written $\lambda
\overline{\eta_i} . N'_i$ where $N'_i$ is not an abstraction. By the
induction hypothesis, $\lambda \overline{\xi} . N_i = \lambda
\overline{\xi} \overline{\eta_i} . N'_i$ is safe.
Hence $\vdash \lambda \overline{\xi} \overline{\eta_i} . N'_i$
is a valid judgment of safe $\lambda$-calculus.
But this judgment can only be derived using the (abs) rule on the term $N'_i$. Hence
$N'_i$ is necessarily safe. Let $z$ be a variable occurring free in
$N'_i$. Since $M$ is closed, $z$ is either bound by $\lambda
\overline{\eta_1}$ or $\lambda \overline{\xi}$. If it is bound by
$\lambda \overline{\xi}$ then because $\tau(M)$ is
incrementally-bound we have $\ord{z} \geq \ord{\lambda
\overline{\eta_1}} = \ord{N_i}$. Hence in both case we can abstract the variables
$\overline{\eta_1}$ using the (abs) rule which shows that $N_i$ is safe.

Each of the $N_i$s are safe and $N_1 \ldots N_p$ is of type $o$ therefore
by the (app) rule we have $\overline{\xi} \vdash N_1 \ldots N_p$. Finally, using the (abs) rule we conclude
with the judgement $\vdash M = \lambda \overline{\xi} . N_1 \ldots N_p$.
\end{proof}

Note that the hypothesis that $M$ is closed in (ii) is necessary.
For instance, the two terms $\lambda x y .x$ and $\lambda y . x$,
where $x,y:o$, have (isomorphic) incrementally-bound computation
trees. However $\lambda x y .x$ is safe whereas $\lambda y . x$ is
not.



Putting proposition \ref{prop:incrbound_imp_incrjustified} and lemma
\ref{lem:safe_imp_incrbound} together we obtain a game-semantic
characterisation of safe terms:
\begin{cor}[P-incrementally-justified strategies characterise safe closed $\eta\beta$-normal terms]
Let $M$ be a simply-typed term without interpreted constants. We have:
$$ \sem{M} \mbox{ is P-incrementally-justified if and only if $\etabetanf{M}$ is safe,} $$
where $\etabetanf{M}$ denotes the $\eta$-normal form of the
$\beta$-normal form of $M$.
\end{cor}



\begin{thm}[P's pointers are superfluous for safe terms]
\label{thm:safe_ptr_recoverable} Pointers emanating from P-moves in the game semantics of
safe terms are uniquely recoverable.
\end{thm}
\begin{proof}
Let $M$ be a safe simply-typed term. The $\beta$-normal form of $M$
denoted by $M'$ is also safe. By lemma \ref{lem:safe_imp_incrbound}
(i), $\tau(M')$ is incrementally-bound and by proposition
\ref{prop:incrbound_imp_incrjustified}, $\sem{M'}$ is a
P-incrementally-justified strategy. By lemma
\ref{lem:incrjustified_pointers_uniqu_recover}, P's pointers in
$\sem{M'}$ are uniquely recoverable. Finally, the soundness of the
game model gives $\sem{M} = \sem{M'}$.
\end{proof}


\section{Safe PCF and Safe Idealized Algol}

Safe Idealized Algol, or safe \ialgol\ for short, is Idealized Algol
where the application and abstraction rules are restricted the same
way as in the safe $\lambda$-calculus (see rules of section
\ref{sec:safe_nonhomog}).

The properties of the safe $\lambda$-calculus can be transposed
straightforwardly to safe \ialgol. In particular, it can be shown
that safety is preserved by $\beta$-reduction and that no variable
capture occurs when performing substitution on a safe term.

A natural question to ask is whether we can extend the result about
game semantics of safe $\lambda$-terms to safe \ialgol-terms. In
this section we lay out the key elements permitting to prove that
the pointers in the game semantics of safe IA terms can be recovered
uniquely.

Such result has potential applications in algorithmic game semantics.
For instance, by following the framework of \cite{ghicamccusker00},
it may be possible to give a characterisation of the game semantics
of some higher-order fragments of safe \ialgol\ using extended
regular expressions. Subsequently, this would lead to the
decidability of program equivalence for the considered fragment.


\subsection{Formation rules of Safe \ialgol}
We call safe \ialgol\ term any term that is typable within the
following system of formation rules:
$$ \rulename{var} \   \rulef{}{x : A\vdash x : A}
%\qquad  \rulename{const} \   \rulef{}{\vdash f : A} \quad f \in \Sigma
\qquad  \rulename{wk} \   \rulef{\Gamma \vdash M : A}{\Delta \vdash
M : A} \quad  \Gamma \subset \Delta$$

$$ \rulename{app} \  \rulef{\Gamma \vdash M : (A,\ldots,A_l,B)
                                        \qquad \Gamma \vdash N_1 : A_1
                                        \quad \ldots \quad \Gamma \vdash N_l : A_l  }
                                   {\Gamma  \vdash M N_1 \ldots N_l : B}
                                    \quad
\mbox{\fbox{$\forall y \in \Gamma : \ord{y} \geq \ord{B}$}}$$

$$ \rulename{abs} \   \rulef{\Gamma \union \overline{x} : \overline{A} \vdash M : B}
                                   {\Gamma  \vdash \lambda \overline{x} : \overline{A} . M : (\overline{A},B)} \quad
\mbox{\fbox{$\forall y \in \Gamma : \ord{y} \geq \ord{\overline{A},B}$}}$$

$$ \rulename{num} \rulef{}{\Gamma \vdash n :\texttt{exp}}
\qquad \rulename{succ} \rulef{\Gamma \vdash M:\texttt{exp} }{\Gamma
\vdash \texttt{succ}\ M:\texttt{exp}} \qquad \rulename{pred}
\rulef{\Gamma \vdash M:\texttt{exp} }{\Gamma \vdash \texttt{pred}\
M:\texttt{exp}}$$

$$
\rulename{cond} \rulef{\Gamma \vdash M : \texttt{exp} \qquad \Gamma
\vdash N_1 : \texttt{exp} \qquad \Gamma \vdash N_2 : \texttt{exp}
}{\Gamma \vdash \texttt{cond}\ M\ N_1\ N_2} \qquad  \rulename{rec}
\rulef{\Gamma \vdash M : A\rightarrow A }{ \Gamma \vdash Y_A M :
A}$$

$$ \rulename{seq} \rulef{\Gamma \vdash M : \texttt{com} \quad \Gamma \vdash N :A}
    {\Gamma \vdash \texttt{seq}_A \ M\ N\ : A} \quad A \in \{ \texttt{com}, \texttt{exp}\}$$

$$ \rulename{assign} \rulef{\Gamma \vdash M : \texttt{var} \quad \Gamma \vdash N : \texttt{exp}}
    {\Gamma \vdash \texttt{assign}\ M\ N\ : \texttt{com}}
\qquad
 \rulename{deref} \rulef{\Gamma \vdash M : \texttt{var}}
    {\Gamma \vdash \texttt{deref}\ M\ : \texttt{exp}}$$

$$ \rulename{new} \rulef{\Gamma, x : \texttt{var} \vdash M : A}
    {\Gamma \vdash \texttt{new } x \texttt{ in } M} \quad A \in \{ \texttt{com}, \texttt{exp}\}$$

$$ \rulename{mkvar} \rulef{\Gamma \vdash M_1 : \texttt{exp} \rightarrow \texttt{com} \quad \Gamma \vdash M_2 : \texttt{exp}}
    {\Gamma \vdash \texttt{mkvar } M_1\ M_2\ : \texttt{var}}$$

\subsection{Small-step semantics of Safe \ialgol}
In the first chapter we defined the operational semantics of
\ialgol\ using a big step semantics. The operational semantics of
\ialgol\ can be defined equivalently using a small-step semantics.
The reduction rules of the small-step semantics are of the form $s,e
\rightarrow s',e'$ where $s$ and $s'$ denotes the stores and $e$ and
$e'$ denotes \ialgol\ expressions.

Let us give the rules that tell how to reduce redexes:
\begin{itemize}
\item the reduction of safe-redex (relation $\beta_s$ from definition \ref{dfn:safereduction});
\item reduction rules for \pcf\ constants:
\begin{eqnarray*}
\pcfsucc\ n &\rightarrow& n+1 \\
\pcfpred\ n+1 &\rightarrow& n \\
\pcfpred\ 0 &\rightarrow& 0 \\
\pcfcond\ 0\ N_1 N_2 &\rightarrow& N_1 \\
\pcfcond\ n+1\ N_1 N_2 &\rightarrow& N_2 \\
Y\ M &\rightarrow& M (Y M)
\end{eqnarray*}
\item reduction rules for \ialgol\ constants:
\begin{eqnarray*}
\iaseq\ \iaskip\  M &\rightarrow& M \\
s, \ianewin{x}\ M &\rightarrow& (s|x\mapsto 0), M \\
s, \iaassign\ x\ n &\rightarrow& (s|x\mapsto n), \iaskip \\
s, \iaderef\ x &\rightarrow& s, s(x) \\
\iaassign\ (\iamkvar M N)\ n &\rightarrow& M n \\
\iaderef\ (\iamkvar M N) &\rightarrow& N
\end{eqnarray*}
\end{itemize}

Redex can also be reduced when they occur as subexpressions within a
larger expression. We make use of evaluation contexts to indicate
when such reduction can happen. Evaluation contexts are given by the
following grammar:
\begin{eqnarray*}
E[-] &::=& - |\ E N\ |\ \pcfsucc\ E\ |\ \pcfpred\ E\ |\ \pcfcond\ E\ N_1\ N_2\ |\ \\
&&    \iaseq\ E\ N\ |\ \iaderef\ E\ |\ \iaassign\ E\ n\ |\ \iaassign\ M\ E \ |\ \\
&&    \iamkvar\ M\ E\ |\ \iamkvar\ E\ M\ |\ \ianewin{x}\ E  .
\end{eqnarray*}

The small-step semantics is completed with following rule:
$$ \rulef{M \rightarrow N}{E[M] \rightarrow E[N]} $$

\begin{lem}[Reduction preserves safety]
\label{lem:ia_safety_preserved} Let $M$ be a safe IA term. If
$M \rightarrow N$ then $N$ is also a safe term.
\end{lem}
This can be proved easily by induction on the structure of M.


\subsection{Safe \pcf\ fragment}
In this section, we show how to extend the results obtained for the
safe $\lambda$-calculus to the \pcf\ fragment of safe \ialgol.

The $Y$ combinator needs a special treatment. In order to deal with
it, we follow the idea of \cite{abramsky:game-semantics-tutorial}:
we consider the sublanguage $\pcf_1$ of \pcf\ in which the only
allowed use of the $Y$ combinator is in terms of the form $Y(
\lambda x:A .x )$ for some type $A$. We will write $\Omega_A$ to
denote the non-terminating term $Y(\lambda x:A .x)$ for a given type
$A$.

We introduce the \emph{syntactic approximants} to $Y_A M$:
\begin{eqnarray*}
Y^0_A M &=& \Gamma \vdash \Omega_A : A\\
Y^{n+1}_A M &=& M( Y^n M )
\end{eqnarray*}
For any \pcf\ term $M$ and natural number $n$, we define $M_n$ to be
the $\pcf_1$ term obtained from $M$ by replacing each subterm of the
form $Y N$ with $Y^n N_n$. We have $\sem{M} = \Union_{n\in\omega}
\sem{M_n}$ (\cite{abramsky:game-semantics-tutorial}, lemma 16).


\subsubsection{Computation tree}

We would like to define a unique computation tree for terms that use
the $Y$ combinator.

Let us first define the computation tree for $\pcf_1$ terms. We
introduce a special $\Sigma$-constant $\bot$ representing the
non-terminating computation of ground type $\Omega_o$. Given any
type $A = (A_1, \ldots, A_n, o)$, the computation tree
$\tau(\Omega_A)$ is defined to be the tree representation of
$\lambda x_1:A_1 \ldots x_n:A_n . \bot$. The computation tree of a
$\pcf_1$ term is then computed inductively in the standard way.

We now introduce a partial order on the set of computation trees.

A \emph{tree} $t$ is a labelling function $t:T\rightarrow L$ where
$T$, called the domain of $t$ and written $dom(t)$, is a non-empty
prefix-closed subset of some free monoid $X^*$ and $L$ denotes the
set of possible labels. Intuitively, $T$ represents the structure of
the tree (the set of all paths) and $t$ is the labelling function
mapping paths to labels. Trees can be ordered using the
\emph{approximation ordering} defined in \cite{KNU02}, section 1: we
write $t' \sqsubseteq t$ if the tree $t'$ is obtained from $t$ by
replacing some of its subtrees by $\bot$. Formally:
$$t' \sqsubseteq t \quad \iff dom(t') \subseteq dom(t) \wedge \forall  w \in dom(t'). (t'(w) = t(w) \vee t'(w) = \bot).$$
The set of all trees together with the approximation ordering is a
complete partial order.

We now consider a strict subset of the set of all trees: the set of
computation trees. A computation tree is a tree which represents the
$\eta$-normal form of some (potentially infinite) \pcf\ term. In
other words a tree is a computation tree if it can be written
$\tau(M)$ for some infinite \pcf\ term $M$. The set $L$ of labels is
constituted of the $\Sigma$-constants, @, the special constant
$\bot$, variables and abstractions of any sequence of variables. We
will write $(CT, \sqsubseteq)$ to denote the set of computation
trees ordered by the approximation ordering $\sqsubseteq$ defined
above. $(CT, \sqsubseteq)$ is also a complete partial order.

It is easy to check that the sequence of computation trees
$(\tau(M_n))_{n\in\omega}$ is a chain. We can therefore define the
computation tree of a \pcf\ term $M$ to be the least upper-bound of
the chain of computation trees of its approximants:
$$\tau(M) = \Union_{n\in\omega}(\tau(M_n))_{n\in\omega}.$$

In other words, we construct the computation tree by expanding
infinitely any subterm of the form $Y M$. For instance consider the
term $M = Y (\lambda f x. f x)$ where $f:(o,o)$ and $x:o$. Its
computation tree $\tau(M)$, represented below, is a tree
representation of the $\eta$-normal form of the infinite term
$(\lambda f x. f x) ((\lambda f x. f x) ((\lambda f x. f x)  (
\ldots$.
$$\tau(M) = \tree{\lambda y}{
                \tree{@}{
                        \tree{\lambda f x} { \tree{f}{\tree{\lambda}{\TR{x}} }}
                        \TR{\tau(M)}
                        \tree{\lambda}{\TR{y}}
                }
            }
$$

The remaining operators of \ialgol\ are treated as standard
constants and the corresponding computation tree is constructed from
the $\eta$-normal form of the term in the standard way. For instance
the diagram below shows the computation tree for $\pcfcond\ b\ x\ y$
(left) and $\lambda x . 5$ (right):
$$
\tree{\lambda b x y}
     {  \tree{\pcfcond}
        {   \tree{\lambda} {\TR{b}}
            \tree{\lambda} {\TR{x}}
            \tree{\lambda} {\TR{y}}
        }
    }
\hspace{2cm} \tree{\lambda x}{  \TR{5} }
$$
The node labelled $5$ has, like any other node, children
value-leaves which are not represented on the diagram above for
simplicity.

\subsubsection{Traversal}

New traversal rules accompany the additional constants of \ialgol.
There is one additional rule for natural number constants:
\begin{itemize}
\item (Nat) If $t \cdot n$ is a traversal where $n$ denotes a node labelled with some numeral constant $i\in \nat$ then
            $t \cdot \rnode{n}{n} \cdot \rnode{in}{i_n} \link[nodesep=0pt]{40}{in}{n}$
            is also a traversal where $i_n$ denotes the value-leaf of $m$ corresponding to the value $i\in \nat$.
\end{itemize}

\noindent The traversals rules for \pcfpred\ and \pcfsucc\ are
defined similarly. For instance, the rules for \pcfsucc\ are:
\begin{itemize}
\item (Succ) If $t \cdot \pcfsucc$ is a traversal and $\lambda$ denotes the only child node of \pcfsucc\ then
$t \cdot \rnode{succ}{\pcfsucc} \cdot \rnode{l}{\lambda}
\link[nodesep=1pt]{60}{l}{succ} \lnklabel{1}$ is also a traversal.

\item (Succ') If
$t_1 \cdot \rnode{succ}{\pcfsucc} \cdot \rnode{l}{\lambda} \cdot t_2
\cdot \rnode{lv}{i_{\lambda}} \link[nodesep=1pt]{60}{l}{succ}
\lnklabel{1} \link[nodesep=1pt]{40}{lv}{l}$ is a traversal for some
$i \in \nat$ then $t_1 \cdot \rnode{succ}{\pcfsucc} \cdot
\rnode{l}{\lambda} \cdot t_2 \cdot \rnode{lv}{i_{\lambda}} \cdot
\rnode{succv}{(i+1)_{\pcfsucc}} \link[nodesep=1pt]{60}{l}{succ}
\lnklabel{1} \link[nodesep=1pt]{25}{succv}{succ}
\link[nodesep=1pt]{40}{lv}{l} $ is also a traversal.
\end{itemize}

\noindent In the computation tree, nodes labelled with \pcfcond\
have three children nodes numbered from $1$ to $3$ corresponding to
the three parameters of the operator \pcfcond. The traversal rules
are:
\begin{itemize}
\item (Cond-If) If $t_1 \cdot \pcfcond$ is a traversal and $\lambda$ denotes the first child of \pcfcond\ then
$t_1 \cdot \rnode{cond}{\pcfcond} \cdot \rnode{l}{\lambda}
\link[nodesep=1pt]{60}{l}{cond} \lnklabel{1}$ is also a traversal.

\item (Cond-ThenElse) If
$t_1 \cdot \rnode{cond}{\pcfcond} \cdot \rnode{l}{\lambda} \cdot t_2
\cdot \rnode{lv}{i_{\lambda}} \link[nodesep=1pt]{60}{l}{cond}
\lnklabel{1} \link[nodesep=1pt]{40}{lv}{l}$ then $t_1 \cdot
\rnode{cond}{\pcfcond} \cdot \rnode{l}{\lambda} \cdot t_2 \cdot
\rnode{lv}{i_{\lambda}} \cdot \rnode{condthenelse}{\lambda}
\link[nodesep=1pt]{60}{l}{cond} \lnklabel{1}
\link[nodesep=1pt]{40}{lv}{l}
\link[nodesep=1pt]{35}{condthenelse}{cond} \lnklabelc{2+[i>0]} $ is
also a traversal.



\item (Cond') If
$t_1 \cdot \rnode{cond}{\pcfcond} \cdot t_2 \cdot \rnode{l}{\lambda}
\cdot t_3 \cdot \rnode{lv}{i_{\lambda}}
\link[nodesep=1pt]{40}{l}{cond} \lnklabel{k}
\link[nodesep=1pt]{40}{lv}{l}$ for $k=2$ or $k=3$ then $t_1 \cdot
\rnode{cond}{\pcfcond} \cdot t_2 \cdot \rnode{l}{\lambda} \cdot t_3
\cdot \rnode{lv}{i_{\lambda}} \cdot \rnode{condv}{i_{\pcfcond}}
\link[nodesep=1pt]{40}{l}{cond} \lnklabel{k}
\link[nodesep=1pt]{40}{lv}{l} \link[nodesep=1pt]{20}{condv}{cond}
$ is also a traversal.
\end{itemize}
It is easy to verify that these traversal rules are all well-behaved
and therefore condition (WB) of section \ref{subsec:traversal} is
met. This completes the definition of traversal for the \pcf\ subset
of \ialgol.

\subsubsection{Interaction semantics}
We recall that the interaction semantics defined in section
\ref{sec:interaction_semantics} takes into account the constants
of the language. For any higher-order constant $f : (A_1,\ldots,A_p,B) \in \Sigma$, definition \ref{dfn:interactionstrategy_ofterms} gives the  revealed strategy of a term of the form $\lambda \overline{\xi}. f N_1 \ldots
N_p$ as follows:
$$ \intersem{\lambda \overline{\xi}. f N_1 \ldots N_p} = \langle \intersem{N_1}, \ldots, \intersem{N_p} \rangle \fatsemi^{0..p-1} \sem{f}.$$
where $\sem{f}$ is the standard strategy denotation of the constant $f$.


\subsubsection{Removing $\Sigma$-nodes from the traversals}

To establish the correspondence with the interaction semantics, we
need to remove the superfluous nodes from the traversals. These
nodes are the @-nodes and the constant nodes. We will use the
operation $-@$ (definition \ref{dfn:appnode_filter}) to filter out
the @-nodes and we introduce a similar operation $-\Sigma$ to
eliminate the $\Sigma$-nodes.

\begin{dfn}[Hiding $\Sigma$-constants in the traversals]
Let $t$ be a traversal of $\tau(M)$. We write $t-\Sigma$ for the
sequence of nodes with pointers obtained by
\begin{itemize}
\item removing from $t$ all nodes labelled with a $\Sigma$-constant or value-leaf justified by a $\Sigma$-constant,
\item replacing any link pointing to a $\Sigma$-constant $f$
by a link pointing to the predecessor of $f$ in $t$.
\end{itemize}

Suppose $u = t-\Sigma$ is a sequence of nodes obtained by applying
the previously defined transformation on the traversal $t$, then $t$
can be partially recovered from $u$ by reinserting the
$\Sigma$-nodes as follows. For each $\Sigma$-node $f$, where $p$
denotes the parent node of $f$, do the following:
    \begin{enumerate}
    \item replace every occurrence of the pattern $p \cdot n$ in $u$ where
    $n$ is a $\lambda$-node by $p \cdot f \cdot n$;

    \item replace any link in $u$ starting from a $\lambda$-node and pointing to $p$ by a link pointing to the inserted node $f$;

    \item for each occurrence in $u$ of a value-leaf $v_p$ pointing to $p$, add the value-leaf $v_f$
    immediately before $v_p$. The links of $v_f$ points to the node immediately following $p$.
    \end{enumerate}
We write $u+\Sigma$ for this second transformation.
\end{dfn}
These transformations are well-defined since in a traversal, a
$\Sigma$-node $f$ always follows immediately its parent
$\lambda$-node $p$, and an occurrence of a value-node $v_p$ always
follows immediately a value-node $v_f$. In other words, if $f$
occurs in $t$ then $t$ must be a prefix of a traversal of the
following form for some $v \in \mathcal{D}$:
$$ \ldots \cdot \rnode{p}{p} \cdot \rnode{f}{f} \cdot \ldots \cdot \rnode{vf}{v_f} \cdot \rnode{vp}{v_p} \cdot \ldots
\link[offset=-4pt]{20}{vf}{f} \link[offset=-4pt]{20}{vp}{p}
$$

Remark: $t-\Sigma$ is not a proper traversal since it does not
satisfy alternation. It is not a proper justified sequence either
since after removing a $\Sigma$-node $f$, any $\lambda$-node
justified by $f$ will become justified by the parent of $f$ which is
also a $\lambda$-node.

The following lemma follows directly from the definition:
\begin{lem}
\label{lem:minus_sig_plus_sig} For any traversal $t$ we have
$(t-\Sigma)+\Sigma \sqsubseteq t$ and if $t$ does not end with an
$\Sigma$-node or a value-leaf of a $\Sigma$-node then
$(t-\Sigma)+\Sigma = t$.
\end{lem}

The operations $-@$ and $-\Sigma$ are commutative: $(t-@)-\Sigma =
(t-\Sigma)-@$. We write $t^*$ to denote $(t-@)-\Sigma$ i.e. the
sequence obtained from $t$ by removing all the @-nodes as well as
the constant nodes together with their associated value-leaves. We
introduce the notation $\travset(M)^{*} = \{ t^* \ | \  t \in
\travset(M) \}$.

\begin{lem}[Filtering lemma]
\label{lem:SIGMACONST:varphi_filter} Let $\Gamma \vdash M :T$ be a
term and $r$ be the root of $\tau(M)$. For any traversal $t$ of the
computation tree we have $ \varphi(\travset^*(M)) \upharpoonright
\sem{\Gamma \rightarrow T} = \varphi(\travset^{\upharpoonright
r}(M)) $.
 Consequently,
$$\varphi(t^*) \upharpoonright \sem{\Gamma \rightarrow T} = \varphi(t\upharpoonright r).$$
\end{lem}
\begin{proof}
    From the definition of $\varphi$, the nodes of the computation tree that $\varphi$ maps
    to moves in the arena $\sem{\Gamma \rightarrow T}$ are exactly the nodes that are hereditarily justified by $r$.
    The result follows from the fact that @-nodes, constant nodes and value-leaves of constant nodes
    are not hereditarily justified by the root.
\end{proof}

The following lemma is the counterpart of lemma
\ref{lem:varphiinjective} and it is proved identically.
\begin{lem}[$\varphi$ is injective]
\label{lem:SIGMACONST:varphiinjective} $\varphi$ regarded as a
function defined on the set of sequences of nodes is injective in
the sense that for any two traversals $t_1$ and $t_2$:
\begin{itemize}
\item[(i)] if $\varphi (t_1^* ) = \varphi (t_2^* )$ then $t_1^* =t_2^*$;
\item[(ii)] if $\varphi (t_1 \upharpoonright r ) = \varphi (t_2 \upharpoonright r )$ then $t_1\upharpoonright r = t_2\upharpoonright r$.
\end{itemize}
\end{lem}

\begin{cor} \
\label{cor:SIGMACONST:varphi_bij}
\begin{itemize}
\item[(i)] $\varphi$ defines a bijection from $\travset(M)^*$
to $\varphi(\travset(M)^*)$;
\item[(ii)] $\varphi$ defines a bijection from $\travset(M)^{\upharpoonright r}$ to
$\varphi(\travset(M)^{\upharpoonright r})$.
\end{itemize}
\end{cor}


\subsubsection{Correspondence theorem}
We would like to prove the counterpart of proposition
\ref{prop:rel_gamesem_trav} in the context of the simply-typed
$\lambda$-calculus \emph{with interpreted PCF constants}. The game
model of the language \pcf\ is given by the category $\mathcal{C}_b$
of well-bracketed strategies. Hence the well-bracketing assumption
stated in section \ref{sec:assumptions} is satisfied.

We first prove that $\travset^{\upharpoonright r}$ is continuous.
\begin{lem}
\label{lem:travred_continuous} Let $(S,\subseteq)$ denote the set of
sets of justified sequences of nodes ordered by subset inclusion.
The function $\travset^{\upharpoonright r} : (CT,\sqsubseteq)
\rightarrow (S,\subseteq)$ is continuous.
\end{lem}
\begin{proof} \
    \begin{description}
    \item[Monotonicity:] Let $T$ and $T'$ be two computation trees such that $T \sqsubseteq T'$
    and let $t$ be some traversal of $T$.
    Traversals ending with a node labelled $\bot$ are maximal therefore $\bot$ can only occur
    at the last position in a traversal. Let us prove the following two properties:
        \begin{itemize}
            \item[(i)]  If $t = t \cdot n$ with $n\neq \bot$ then $t$ is a traversal of $T'$;
            \item[(ii)] if $t= t_1 \cdot \bot$ then $t_1\in \travset(T')$.
        \end{itemize}

        (i) By induction on the length of $t$. It is trivial for the empty traversal.
            Suppose that $t = t_1 \cdot n$ is a traversal with $n \neq \bot$.
            By the induction hypothesis, $t_1$ is a traversal of $T'$.

            We observe that for all traversal rules, the traversal produced is of the form $t_1 \cdot n$ where
            $n$ is defined to be a child node or value-leaf of some node $m$ occurring in $t_1$.
            Moreover, the choice of the node $n$ only depends on the traversal $t_1$
            (for the constant rules, this is guaranteed by assumption (WB)).

            Since $T \sqsubseteq T'$, any node $m$ occurring in $t_1$ belongs
            to $T'$ and the children nodes and leaves of $m$ in $T$ also belong to the tree $T'$.
            Hence $n$ is also present in $T'$ and the rule used to produce the traversal $t$ of $T$
            can be used to produce the traversal $t$ of $T'$.

        (ii) $\bot$ can only occur at the last position in a traversal
        therefore $t_1$ does not end with $\bot$ and by (i) we have $t_1\in \travset(T')$.
\vspace{6pt}

        Hence we have:
        \begin{align*}
        \travset(T)^{\upharpoonright r} &= \{ t \upharpoonright r \ | \ t \in \travset(T)     \} \\
        & = \{ (t\cdot n) \upharpoonright r \ | \ t\cdot n \in \travset(T) \wedge n \neq \bot \}
            \union \{ (t \cdot \bot ) \upharpoonright r \ | \ t \cdot \bot \in \travset(T)  \} \\
\mbox{(by (i) and (ii))} \quad        & \subseteq  \{ (t\cdot n)
\upharpoonright r \ | \ t\cdot n \in \travset(T') \wedge n \neq \bot
\}
            \union \{ t \upharpoonright r \ | \ t \in \travset(T')  \} \\
        & = \travset(T')^{\upharpoonright r}
        \end{align*}

        \item[Continuity:] Let $t \in \travset \left( \Union_{n\in\omega} T_n \right)$.
        We write $t_i$ for the finite prefix of $t$ of length $i$.
        The set of traversals is prefix-closed therefore $t_i \in \travset \left( \Union_{n\in\omega} T_n \right)$ for any $i$.
        Since $t_i$ has finite length we have $t_i \in \travset(T_{j_i})$ for some $j_i \in \omega$.
        Therefore we have:
        \begin{align*}
          t \upharpoonright r &= (\bigvee_{i\in\omega} t_i ) \upharpoonright r   & (\mbox{the sequence $(t_i)_{i\in\omega}$ converges to $t$}) \\
          &= \Union_{i\in\omega} ( t_i \upharpoonright r )   & (\_ \upharpoonright r \mbox{ is continuous, lemma \ref{lem:filtercontinous}}) \\
          &\in \Union_{i\in\omega} \travset^{\upharpoonright r}(T_{j_i})   & (t_i \in \travset(T_{j_i})) \\
          &\subseteq \Union_{i\in\omega} \travset^{\upharpoonright r}(T_i)   & (\mbox{since } \{ j_i \sthat i \in \omega \} \subseteq \omega)
        \end{align*}

        Hence $\travset^{\upharpoonright r} (\Union_{n\in\omega} T_n ) \subseteq \Union_{n\in\omega} \travset^{\upharpoonright r}(T_n).$

    \end{description}
\end{proof}

\begin{prop}
Let $\Gamma \vdash M : T$ be a PCF term and $r$ be the root of
$\tau(M)$. Then:
\begin{align*}
(i)  \quad\varphi_M(\travset(M)^*) = \intersem{M},  \\
(ii) \quad \varphi_M(\travset(M)^{\upharpoonright r}) = \sem{M}.
\end{align*}
\end{prop}
\begin{proof}
We first prove the result for $\pcf_1$: (i) The proof is an
induction identical to the proof of proposition
\ref{prop:rel_gamesem_trav}. However we need to complete the case
analysis with the $\Sigma$-constant cases:
\begin{itemize}
\item The cases \pcfsucc, \pcfpred, \pcfcond\ and numeral constants are straightforward.

\item Suppose $M = \Omega_o$ then $\travset(\Omega_o) = \prefset ( \{ \lambda \cdot \bot \} )$ therefore
$\travset(\Omega_o)^{\upharpoonright r} = \prefset( \{ \lambda \} )$
and $\sem{\Omega_o} = \prefset( \{ q \})$ with $\varphi(\lambda) =
q$. Hence $\sem{\Omega_o} = \varphi
(\travset(\Omega_o)^{\upharpoonright r})$.
\end{itemize}
(ii) is a direct consequence of (i) and the filtering lemma (lemma
\ref{lem:SIGMACONST:varphi_filter}). \vspace{10pt}

\noindent We now extend the result to \pcf. Let $M$ be a \pcf\ term,
we have:
\begin{align*}
\sem{M} &= \Union_{n\in\omega} \sem{M_n} & (\mbox{\cite{abramsky:game-semantics-tutorial}, lemma 16})\\
&= \Union_{n\in\omega} \travset^{\upharpoonright r}(\tau(M_n)) & (M_n \mbox{ is a $\pcf_1$ term}) \\
&= \travset^{\upharpoonright r}(\Union_{n\in\omega} \tau(M_n) ) & (\mbox{by continuity of $\travset^{\upharpoonright r}$, lemma \ref{lem:travred_continuous}}) \\
&= \travset^{\upharpoonright r}(\tau(M)) & (\mbox{by definition of } \tau(M)) \\
&= \travset^{\upharpoonright r}(M) & (\mbox{abbreviation}).
\end{align*}
\end{proof}

Hence by corollary \ref{cor:SIGMACONST:varphi_bij}, $\varphi$
defines a bijection from $\travset(M)^{\upharpoonright r}$ to
$\sem{M}$:
$$\varphi : \travset(M)^{\upharpoonright r} \stackrel{\cong}{\longrightarrow} \sem{M}.$$

\subsubsection{Example: \pcfsucc}

Consider the term $M = \pcfsucc\ 5$ whose computation tree is
represented below. The value-leaves are also represented on the
diagram, they are the vertices attached to their parent node with a
dashed line.
$$
\psmatrix[colsep=3ex,rowsep=2ex]
\lambda^0 \\
\pcfsucc & 0 & 1 & \ldots \\
\lambda^1 & 0 & 1 & \ldots \\
5 & 0 & 1 & \ldots \\
  & 0 & 1 & \ldots
\endpsmatrix
\ncline{1,1}{2,1} \ncline{2,1}{3,1} \ncline{3,1}{4,1}
\valueedge{1,1}{2,2} \valueedge{1,1}{2,3} \valueedge{1,1}{2,4}
\valueedge{2,1}{3,2} \valueedge{2,1}{3,3} \valueedge{2,1}{3,4}
\valueedge{3,1}{4,2} \valueedge{3,1}{4,3} \valueedge{3,1}{4,4}
\valueedge{4,1}{5,2} \valueedge{4,1}{5,3} \valueedge{4,1}{5,4}
$$

The following sequence of nodes is a traversal of $\tau(M)$:
\vspace{18pt}
$$ t = \rnode{l0}{\lambda^0} \cdot \rnode{succ}{\pcfsucc} \cdot \rnode{l1}{\lambda^1} \cdot \rnode{c5}{5} \cdot \rnode{55}{5_5} \cdot \rnode{5l1}{5_{\lambda^1}} \cdot \rnode{6succ}{6_\pcfsucc} \cdot \rnode{6l0}{6_{\lambda^0}}.
\link[offset=-4pt]{20}{6l0}{l0} \link[offset=-4pt]{20}{5l1}{l1}
\link[offset=-4pt]{20}{55}{c5} \link[offset=-4pt]{20}{6succ}{succ}
$$
The subsequences $t^*$ and $t \upharpoonright r$ are given by:
$$
t^* = \rnode{l0}{\lambda^0} \cdot \rnode{l1}{\lambda^1} \cdot
\rnode{5l1}{5_{\lambda^1}} \cdot \rnode{6l0}{6_{\lambda^0}}.
\link[offset=-4pt]{20}{6l0}{l0} \link[offset=-4pt]{20}{5l1}{l1}
\link[offset=-4pt]{20}{l1}{l0} \qquad  \mbox{ and } \qquad t
\upharpoonright r = \rnode{l0}{\lambda^0} \cdot
\rnode{6l0}{6_{\lambda^0}}. \link[offset=-4pt]{20}{6l0}{l0}
$$
We have $\varphi(t^*) = q_0 \cdot q_5 \cdot 5_{q_5} \cdot 5_{q_0}$
and $\varphi(t\upharpoonright r) = q_0 \cdot 5_{q_0}$ where $q_0$
and $q_5$ denote the roots of two flat arenas over $\nat$. These two
sequences of moves correspond to some play of the interaction
semantics and the standard semantics respectively. The interaction
play is represented below:
$$\begin{array}{ccccc}
  \textbf{1} & \stackrel{5}{\multimap} & !\nat & \stackrel{\pcfsucc}{\multimap} & \nat \\
&&&&  \rnode{q0}{q_0} \\
&&  \rnode{q5}{q_5} \\
&&  \rnode{a5}{5_{q_5}} \\
&&&&  \rnode{a6}{6_{q_0}}
\end{array}
\nccurve[nodesep=2pt,ncurv=0.9,angleA=180,angleB=180]{->}{a5}{q5}
\nccurve[nodesep=2pt,ncurv=0.9,angleA=180,angleB=210]{->}{a6}{q0}
\ncarc[nodesep=2pt,ncurv=0.9,angleA=180,angleB=180]{->}{q5}{q0}
$$

\subsubsection{Another example : \pcfcond}

Consider the term $M = \lambda x y . \pcfcond\ 1\ x\ y$. Its
computation tree is represented below (without the value-leaves):
    $$ \tree{\lambda x y}
       {
          \tree{\pcfcond}
          {
            \tree{\lambda^1}{ \TR{1} }
            \tree{\lambda^2}{ \TR{x} }
            \tree{\lambda^3}{ \TR{y} }
          }
      }
    $$
For any value $v \in\mathcal{D}$ the following sequence of nodes is
a traversal of $\tau(M)$: \vspace{18pt}
$$ t = \rnode{lxy}{\lambda x y} \cdot \rnode{cond}{\pcfcond} \cdot \rnode{l1}{\lambda^1} \cdot \rnode{1}{1} \cdot \rnode{11}{1_1}
    \cdot \rnode{l3}{\lambda^3} \cdot \rnode{y}{y} \cdot \rnode{vy}{v_y}  \cdot \rnode{vl3}{v_{\lambda^3}} \cdot \rnode{vcond}{v_{\pcfcond}}
    \cdot \rnode{vlxy}{v_{\lambda x y}}.
\link[offset=-4pt]{20}{vlxy}{lxy}
\link[offset=-4pt]{20}{vcond}{cond}
\link[offset=-4pt]{20}{vl3}{l3} \link[offset=-4pt]{20}{vy}{y}
\link[offset=-4pt]{20}{y}{vxy} \link[offset=-4pt]{20}{l2}{cond}
\link[offset=-4pt]{20}{11}{1} \link[offset=-4pt]{20}{l1}{cond}
$$
The subsequences $t^*$ and $t \upharpoonright r$ are given by:
\vspace{13pt}
$$
t^* =  t = \rnode{lxy}{\lambda x y} \cdot
        \rnode{l1}{\lambda^1} \cdot
        \rnode{l3}{\lambda^3} \cdot
        \rnode{y}{y} \cdot
        \rnode{vy}{v_y}  \cdot
        \rnode{vl3}{v_{\lambda^3}} \cdot
        \rnode{vlxy}{v_{\lambda x y}}
\link[offset=-4pt]{20}{vlxy}{lxy} \link[offset=-4pt]{20}{vl3}{l3}
\link[offset=-4pt]{20}{vy}{y} \link[offset=-4pt]{20}{y}{vxy}
\link[offset=-4pt]{20}{l3}{lxy} \link[offset=-4pt]{20}{l1}{lxy}
\qquad  \mbox{ and } \qquad t \upharpoonright r =
\rnode{lxy}{\lambda x y} \cdot \rnode{y}{y} \cdot \rnode{vy}{v_y}
\cdot \rnode{vlxy}{v_{\lambda x y}}.
\link[offset=-4pt]{20}{vlxy}{lxy} \link[offset=-4pt]{20}{vy}{y}
\link[offset=-4pt]{20}{y}{vxy}
$$
The sequence of moves $\varphi(t^*)$ corresponds to some play of the
interaction semantics and the sequence $\varphi(t\upharpoonright r)$
is a play of the standard semantics obtained by hiding the internal
moves of $\varphi(t^*)$. The interaction play $\varphi(t^*)$ is
represented below:
$$\begin{array}{ccccccccccc}
!\nat & \otimes & !\nat & \stackrel{ \langle \sem{1}, \pi_1,
\pi_2\rangle }{\multimap} & !\nat & \otimes & !\nat & \otimes &
!\nat
& \stackrel{ \pcfcond}{\multimap} & \nat \\
&&&&&&&&&&  \rnode{q0}{q_0^{(\lambda x y)}} \\
&&&&  \rnode{qa}{q_a^{(\lambda^1)}} \\
&&&&  \rnode{1}{1} \\
&&&&&&  \rnode{qb}{q_b^{(\lambda^2)}} \\
&&  \rnode{qy}{q_y^{(y)}} \\
&&  \rnode{vqy}{v_{q_y}} \\
&&&&&&  \rnode{vqb}{v_{q_b}} \\
&&&&&&&&&& \rnode{vq0}{v_{q_0}}
\end{array}
\ncarc[nodesep=2pt,ncurv=0.9,angleA=180,angleB=180]{->}{vq0}{q0}
\ncarc[nodesep=2pt,ncurv=0.9,angleA=180,angleB=180]{->}{vqb}{qb}
\nccurve[nodesep=2pt,ncurv=0.9,angleA=180,angleB=180]{->}{vqy}{qy}
\ncarc[nodesep=2pt,ncurv=0.9,angleA=180,angleB=180]{->}{qy}{qb}
\ncarc[nodesep=2pt,ncurv=0.9,angleA=90,angleB=180]{->}{qb}{q0}
\nccurve[nodesepB=2pt,nodesepA=6pt,ncurv=0.9,angleA=180,angleB=180]{->}{1}{qa}
\ncarc[nodesep=2pt,ncurv=0.9,angleA=90,angleB=180]{->}{qa}{q0}
$$


\subsubsection{Game characterisation of safe terms}

A difficulty arises because of the presence of the Y combinator :
computation trees of \pcf\ terms are potentially infinite. Despite
this particularity, lemma \ref{lem:safe_imp_incrbound} still holds
in the \pcf\ setting:
\begin{lem} \label{lem:pcf_safe_imp_incrbound} If $M$ is a safe
PCF term then $\tau(M)$ is incrementally-bound.
\end{lem}
\begin{proof}
Let $i$ denote the number of occurrences of the Y combinator in $M$.
We first prove by induction on $i$ that $M_k$ is safe for any $k\in
\omega$. \emph{Base case:} $i=0$ then $M_k = M$. \emph{Step case:}
$i>0$. Let $Y_A N$ be a subterm of $M$. Since $M$ is safe, $N$ is
also safe. The number of occurrences of the Y combinator in $N$ is
smaller than $i$ therefore by the induction hypothesis $N_k$ is
safe. Consequently the term $Y_A^k N_k = \underbrace{N_k ( \ldots (
N_k}_{k \mbox{ times}} \Omega ) \ldots )$ is also safe and by
compositionality so is $M_k$.

Clearly, lemma \ref{lem:safe_imp_incrbound}(i) is remains valid for infinite 
$\pcf_1$ terms (the subterms of the form $\Omega$ are just represented by
the constant $\bot$ in the computation tree), thus since $M_k$
is a safe $\pcf_1$ term, $\tau(M_k)$ is incrementally-bound.
Now let $z$ be a variable node in $\tau(M) =
\Union_{k\in\omega} \tau(M_k)$. There exists $k\in \omega$ such
that $z$ belongs to $\tau(M_k) \sqsubseteq \tau(M)$. 
If we write $r_k$ to denote the root of the tree $\tau(M_k)$ then the path $[r_k,z]$ in $\tau(M_k)$ is equal to the path $[r,z]$ in $\tau(M)$.
Hence, since the node $z$ is incrementally-bound in $\tau(M_k)$,
it is also incrementally-bound in $\tau(M)$.
\end{proof}


\begin{thm}
Safe PCF terms are denoted by P-incrementally-justified strategies.
\end{thm}
\begin{proof}
Let $M^{\infty}$ be the $\beta$-normal form of $M$ (i.e. the possibly infinite term obtained by reducing all the redexes in $M$). By lemma \ref{lem:ia_safety_preserved}, safety is preserved by small-step reduction therefore, by lemma \ref{lem:pcf_safe_imp_incrbound}, if $M$ is a \pcf\ term then $\tau(M^{\infty})$ is also 
incrementally-bound.

Since condition (WB) is verified ({\it i.e.} \pcf\ constant rules are well-behaved), lemma \ref{lem:redtrav_trav} holds in the safe \pcf\ setting.
Thus proposition \ref{prop:incrbound_imp_incrjustified}(i) remains valid
in \pcf\ for infinite computation trees: infinite terms in $\beta$-nf
with an incrementally-bound computation tree are denoted
by P-incrementally-justified strategies. Consequently, $\sem{M^{\infty}}$
is P-incrementally-justified.
By soundness of the game denotation, $\sem{M^{\infty}} = \sem{M}$, thus $\sem{M}$ is P-incrementally-justified.
\end{proof}

Consequently, P-pointers are superfluous in the game denotation of safe \pcf\ terms {\it i.e.} pointers emanating from P-moves are uniquely recoverable.

\subsection{Safe \ialgol}

We are now in a position to consider the full safe Idealized Algol
language. The general idea is the same as for safe \pcf, however
there are some difficulties caused by the presence of the two new
base types \iavar\ and \iacom. We just give indications on how to
adapt our framework to the particular case of safe \ialgol\ without
giving the complete proofs. However we believe that enough
indications are given to convince the reader that the argument used
in the \pcf\ case can be easily adapted to \ialgol.

\subsubsection{Computation DAG}
In \pcf, arenas have a single initial move, therefore they can be
regarded as trees. In \ialgol, on the other hand, the base type
\iavar\ is represented by the infinite product of games
$\iacom^{\nat} \times \iaexp$ which has an infinite number of
initial moves. In order to preserve the relationship established
between arenas and computation trees, we need to accommodate the
definition of computation tree to reflect this property. The
consequence is that in \ialgol, ``computation trees'' become
``computation directed acyclic graphs (DAG)'': a computation DAG may
have (possibly infinitely) many roots and two nodes of a given level
can share children at the next level.


We use the notations $\mathcal{D}_{\iaexp} = \nat$ and
$\mathcal{D}_{\iacom} = \{ \iadone \}$ to denote the set of value
leaves of type \iaexp\ and \iacom\ respectively. There are two types
of value-leaves in the computation DAG: the value-leaf \iadone\ of
type \iacom\ and the value-leaves labelled in $\mathcal{D}_{\iaexp}$
of type \iaexp.

Let $n$ be a node. If $\kappa(n)$ is of type $(A_1,\ldots A_n,B)$,
we call $B$ the \emph{return type of $n$}. The set of value-leaves
of a node $n$ is given by $\mathcal{D}_{\iaexp}$ if the return type
of $n$ is \iaexp, by $\mathcal{D}_{\iacom}$ if its return type is
\iacom, and by $\mathcal{D}_{\iaexp} \union \{ \iadone \}$ if its
return type is \iavar.


Table \ref{tab:ia_computationdag} shows the computation DAG for each
construct of \ialgol. The value-leaves are represented in the DAGs
using the following abbreviations:
$$ \tree{n}{ \TRV{\mathcal{D}_\iaexp} }  \quad \mbox{ for }\quad
 \tree{n}{ \TRV{0} \TRV{1} \TRV{2} \TRV{\ldots} }
 \qquad \mbox{ and } \qquad
 \tree{n}{ \TRV{\mathcal{D}_\iadone} }  \quad \mbox{ for }\quad
 \tree{n}{ \TRV{\iadone }}.
$$

A term of type \iavar\ has a computation DAG with an infinite number
of root $\lambda$-nodes. Suppose that $M$ is a term of type \iavar,
then the computation DAG for $\lambda \overline{\xi} . M$ is
obtained by relabelling the root $\lambda$-nodes $\lambda^r$,
$\lambda^{w_0}$, $\lambda^{w_1}$, $\lambda^{w_2}$, \ldots into
$\lambda^r \overline{\xi}$, $\lambda^{w_0} \overline{\xi}$,
$\lambda^{w_1} \overline{\xi}$, $\lambda^{w_2} \overline{\xi}$,
\ldots. For a term $M$  of type \iaexp\ or \iacom, the computation
DAG for $\lambda \overline{\xi} . M$ is computed in the same way as
in the safe $\lambda$-calculus.

\begin{table}
\begin{center}
\begin{tabular}{cc}
$M$ & $\tau(M)$ \\ \hline \hline \\
x $: A \in \{ \iacom, \iaexp \}$ &
    $\psmatrix[colsep=3ex,rowsep=2ex] \lambda \\ x & \mathcal{D}_A \\  & \mathcal{D}_A \endpsmatrix
    \ncline{1,1}{2,1} \valueedge{1,1}{2,2} \valueedge{2,1}{3,2} $
\\ \\
x : \iavar &
    $\psmatrix[colsep=3ex,rowsep=3ex]
    \lambda^r & \lambda^{w_0} & \lambda^{w_1}  & \lambda^{w_2} & \lambda^{w_{\ldots}} \\
    \mathcal{D}_\iaexp &  & x & & \iadone \\
    &  &  & \mathcal{D}_\iaexp & \iadone
    \endpsmatrix
    \ncline{1,1}{2,3} \ncline{1,2}{2,3} \ncline{1,3}{2,3} \ncline{1,4}{2,3} \ncline{1,5}{2,3}
    \valueedge{2,3}{3,4} \valueedge{2,3}{3,5}
    \valueedge{1,1}{2,1}
    \valueedge{1,5}{2,5} \valueedge{1,4}{2,5} \valueedge{1,3}{2,5} \valueedge{1,2}{2,5}
    $
\\ \\
\iaskip : \iacom &
    $\psmatrix[colsep=3ex,rowsep=3ex] \lambda \\ \iaskip & \iadone \\  & \iadone \endpsmatrix
    \ncline{1,1}{2,1} \valueedge{1,1}{2,2} \valueedge{2,1}{3,2} $
\\ \\
$\iaassign\ L\ N :\iacom$ &
    $\psmatrix[colsep=3ex,rowsep=3ex] & \lambda \\ & \iaassign & \iadone \\ \tau(N:\iaexp)  & \tau(L:\iavar) & \iadone \endpsmatrix
    \ncline{1,2}{2,2} \ncline{2,2}{3,2} \ncline{2,2}{3,1}
    \valueedge{1,2}{2,3} \valueedge{2,2}{3,3} $
\\ \\
$\iaderef\ L :\iaexp$ &
    $\psmatrix[colsep=3ex,rowsep=3ex] \lambda \\ \iaderef & \iadone \\ \tau(L:\iavar) & \iadone \endpsmatrix
    \ncline{1,1}{2,1} \ncline{2,1}{3,1} \valueedge{1,1}{2,2} \valueedge{2,1}{3,2} $
\\ \\
$\iaseq_{\iaexp}\ N_1\ N_2 :\iacom$ &
    $\psmatrix[colsep=3ex,rowsep=3ex] & \lambda \\ & \iaseq_{\iaexp} & \mathcal{D}_\iaexp \\ \tau(N_1:\iacom)  & \tau(N_2:\iaexp) & \iadone \endpsmatrix
    \ncline{1,2}{2,2} \ncline{2,2}{3,2} \ncline{2,2}{3,1}
    \valueedge{1,2}{2,3} \valueedge{2,2}{3,3} $
\\ \\
$\iamkvar\ N_w\ N_r :\iavar$ &
    $\psmatrix[colsep=3ex,rowsep=3ex]
    \lambda^r & \lambda^{w_0} & \lambda^{w_1}  & \lambda^{w_2} & \lambda^{w_{\ldots}} \\
    \mathcal{D}_\iaexp &  & \iamkvar & & \iadone \\
    & \tau(N_r) & \tau(N_w) & \mathcal{D}_\iaexp & \iadone
    \endpsmatrix
    \ncline{1,1}{2,3} \ncline{1,2}{2,3} \ncline{1,3}{2,3} \ncline{1,4}{2,3} \ncline{1,5}{2,3}
    \ncline{2,3}{3,2} \ncline{2,3}{3,3}
    \valueedge{2,3}{3,4} \valueedge{2,3}{3,5}
    \valueedge{1,1}{2,1}
    \valueedge{1,5}{2,5} \valueedge{1,4}{2,5} \valueedge{1,3}{2,5} \valueedge{1,2}{2,5}
    $
\\ \\
$\ianewin{x}\ N : A \in \{ \iacom, \iaexp \} $ &
   $\psmatrix[colsep=3ex,rowsep=3ex] \lambda \\ \ianewin{x} & \mathcal{D}_A \\ \tau(N:A) & \mathcal{D}_A \endpsmatrix
    \ncline{1,1}{2,1} \ncline{2,1}{3,1} \valueedge{1,1}{2,2} \valueedge{2,1}{3,2} $
\end{tabular}
\end{center}
  \caption{Computation DAGs for the constructs of \ialgol.}
  \label{tab:ia_computationdag}
\end{table}


\subsubsection{Traversals}
Let $p$ be a node and suppose that its $i$th child $n$ has the
return type \iavar. Then $n$ is in fact constituted of several
$\lambda$-nodes : $\lambda^r \overline{\xi}$, $\lambda^{w_0}
\overline{\xi}$, \ldots. From $p$'s point of view, these nodes are
referenced as follows: $i.r$ refers to $\lambda^r \overline{\xi}$
and  $i.w_k$ refers to $\lambda^{w_k} \overline{\xi}$ for $k \in
\omega$.

\begin{itemize}
\item \emph{The application rule}

There are two rules (app$_{\iaexp}$) and (app$_{\iacom}$)
corresponding to traversals ending with an @-node of return type
\iaexp\ and \iacom\ respectively. These rules are identical to the
rule \iaexp\ of section \ref{subsec:traversal}.

The application rule for $@$-nodes with return type \iavar\ is:
$$(\mbox{app}_{\iavar})
\rulefex[5pt]{t \cdot \rnode{l-}{\lambda^k \overline{\xi}} \cdot
\rnode{app-}{@} \in \travset} {t \cdot \rnode{l}{\lambda^k
\overline{\xi}} \cdot \rnode{app}{@} \cdot \rnode{l2}{\lambda^k
\overline{\eta}} \in \travset }
 \ k \in \{ r, w_0, w_1, \ldots \}
\link{40}{app-}{l-} \lnklabelc{0} \link{40}{app}{l} \lnklabelc{0}
\link{40}{l2}{app} \lnklabelc{0.k}
$$


\item \emph{Input-variable rules}

There are two rules (InputVar$^{\iaexp}$) and (InputVar$^{\iacom}$)
which are the counterparts of rule (InputVar$^0$) of section
\ref{subsec:traversal} and are defined identically.

Let $x$ be an input-variable of type \iavar:
$$ (\mbox{InputVar}^{\iavar})
\rulef{t \cdot \lambda^r \overline{\xi} \cdot x \in \travset}
    {t \cdot \lambda^r \overline{\xi} \cdot \rnode{x}{x} \cdot v_x \in \travset }
\hspace{2cm} (\mbox{InputVar}^{' \iavar}) \rulef{t \cdot
\lambda^{w_i} \overline{\xi} \cdot x \in \travset}
    {t \cdot \lambda^{w_i} \overline{\xi} \cdot \rnode{x}{x} \cdot \iadone_x \in \travset }
$$

\item \emph{IA constants rules}

The rules for \ianew\ are purely structural, they are defined the
same way as the rules (app$_{\iaexp}$), (app$_{\iacom}$) and
(app$_{\iadone}$).

The rules for \iaderef\ are:
$$(\mbox{deref}) \rulefex[6pt]{t \cdot \iaderef \in \travset}{t \cdot \rnode{d}{\iaderef} \cdot \rnode{n}{n} \in \travset }
\link{30}{n}{d} \lnklabelc{1.r} \hspace{1.6cm} (\mbox{deref'})
\rulef{t \cdot \iaderef \cdot n \cdot t_2 \cdot v_n \in \travset} {t
\cdot \iaderef \cdot n \cdot t_2 \cdot v_n \cdot v_{\iaderef}\in
\travset }
$$

The rules for \iaassign\ are:
$$(\mbox{assign}) \rulefex[7pt]{t \cdot \iaassign \in \travset}{t \cdot \rnode{ass}{\iaassign} \cdot \rnode{n}{n} \in \travset }
\link{30}{n}{ass} \lnklabelc{1} \hspace{1.6cm} (\mbox{assign'})
\rulefex[10pt]{t \cdot \iaassign \cdot n \cdot t_2 \cdot v_n \in
\travset} {t \cdot \rnode{ass}{\iaassign} \cdot \rnode{n}{n} \cdot
t_2 \cdot v_n \cdot \rnode{m}{m} \in \travset } \link{13}{m}{ass}
\lnklabelc{2.w_n}
$$
$$(\mbox{assign''})  \rulef{t \cdot \rnode{ass-}{\iaassign} \cdot t_2 \cdot \rnode{m-}{m} \cdot t_3 \cdot \iadone_m \in \travset}
{t \cdot \iaassign \cdot t_2 \cdot m \cdot t_3 \cdot \iadone_m \cdot
\iadone_{\iaassign} \in \travset } \link{20}{m-}{ass-}
\lnklabelc{2.w_k}
$$

The rules for $\iaseq_{\iaexp}$ are:
$$(\mbox{seq}) \rulefex[5pt]{t \cdot \iaseq \in \travset}{t \cdot \rnode{seq}{\iaseq} \cdot \rnode{n}{n} \in \travset }
\link{30}{n}{seq} \lnklabelc{1} \hspace{1.6cm} (\mbox{seq'})
\rulefex[5pt]{t \cdot \iaseq \cdot n \cdot t_2 \cdot v_n \in
\travset} {t \cdot \rnode{seq}{\iaseq} \cdot \rnode{n}{n} \cdot t_2
\cdot v_n \cdot \rnode{m}{m} \in \travset } \link{13}{m}{seq}
\lnklabelc{2}
$$
$$(\mbox{seq''})  \rulef{t \cdot \rnode{seq-}{\iaseq} \cdot t_2 \cdot \rnode{m-}{m} \cdot t_3 \cdot v_m \in \travset}
{t \cdot \iaseq \cdot t_2 \cdot m \cdot t_3 \cdot v_m \cdot
v_{\iaseq} \in \travset } \link{20}{m-}{seq-} \lnklabelc{2}
$$




The rules for \iamkvar\ are:
$$(\mbox{mkvar}_r) \rulefex[5pt]{t \cdot \lambda^r \overline{\xi} \cdot \iamkvar \in \travset}{t \cdot \lambda^r \overline{\xi} \cdot \rnode{d}{\iamkvar} \cdot \rnode{n}{n} \in \travset }
\link{30}{n}{d} \lnklabelc{1} \hspace{1cm} (\mbox{mkvar}_r')
\rulef{t \cdot \iamkvar \cdot n \cdot t_2 \cdot v_n \in \travset} {t
\cdot \iamkvar \cdot n \cdot t_2 \cdot v_n \cdot v_{\iamkvar}\in
\travset } $$
$$(\mbox{mkvar}_w) \rulefex[5pt]{t \cdot \lambda^{w_k} \overline{\xi} \cdot \iamkvar \in \travset}{t \cdot \lambda^{w_k} \overline{\xi} \cdot \rnode{mk}{\iamkvar} \cdot \rnode{n}{n} \in \travset }
\link{30}{n}{mk} \lnklabelc{2} $$
$$ (\mbox{mkvar}_w'')  \rulef{t \cdot \lambda^{w_k} \overline{\xi} \cdot \iamkvar \cdot n \cdot t_2 \cdot \iadone_n \in \travset}
{t \cdot \lambda^{w_k} \overline{\xi} \cdot \iamkvar \cdot n \cdot
t_2 \cdot \iadone_n \cdot \iadone_{\iamkvar} \in \travset }
$$
These four rules are not sufficient to model the constant \iamkvar.
Indeed, consider the term $\iaassign\ (\iamkvar\ (\lambda x . M) N)
7$. The rule (\mbox{mkvar}$_w''$) permits to traverse the node
\iamkvar\ and to go on by traversing the computation tree of
$\lambda x . M$. The problem is that when traversing $\tau(M)$, if
we reach a variable $x$, we are not able to relate $x$ to the value
$7$ that is assigned to the variable.

To overcome this problem, we need to define traversal rules for
variable in such a way that a variable node bound by the second
child of a $\iamkvar$-node is treated differently from other
variables.

\item \emph{Variable rules}
Let $x$ be a non input-variable node and let $b$ be the binder of
$x$. $b$ is either a ``$\ianewin{x}$''-node or a $\lambda$-node.

\begin{itemize}
\item Consider the case where $b$ is a $\lambda$-node. Take $b = \lambda \overline{x}$.
In \ialgol, the only constant nodes of order greater than 1 is
\iamkvar, therefore there are two cases: $\lambda \overline{x}$ is
either the child of a node in $N_@ \union N_{var}$ or it is the
second child of a \iamkvar-node.

To handle the first case, we define a rule similar to the (Var) rule
of section \ref{subsec:traversal} with some modification to take
into account variables $x$ of type \iavar (in which case $x$ has
multiple parent $\lambda$-nodes). We do not give the details here
but it is easy to see how to redefine this rule.

To handle the case where $\lambda \overline{x}$ is the child of a
\iamkvar-node, we define the following rule:
$$ (\mbox{Var}_{\iamkvar})  \rulef{t \cdot \lambda^{w_k} \overline{\xi} \cdot \iamkvar \cdot \lambda \overline{x} \cdot t_2 \cdot x \in \travset}
{t \cdot \lambda^{w_k} \overline{\xi} \cdot \iamkvar \cdot \lambda
\overline{x} \cdot t_2 \cdot x \cdot k_{x} \in \travset }
$$

\item The case $b = \ianewin{x}$ is handle by the following rules.

We call \emph{overwrite of $x$ relatively to an occurrence of a ``\ianewin{x}'' node}, any sequence of nodes of the form
\raisebox{0cm}[0.5cm]{$\rnode{decl}{\ianewin{x}}\cdot \ldots \cdot \lambda^{w_k}\overline{\xi} \cdot \rnode{x}{x}$} \link[nodesep=1pt]{17}{x}{decl} for some $k\in \mathcal{D}_{\iaexp}$ and node $\lambda^{w_k}\overline{\xi}$ parent
of $x$.
$$(\mbox{Var}_w)
\rulef{t \cdot \lambda^{w_k} \overline{\xi} \cdot x \in \travset}
{t \cdot \lambda^{w_k} \overline{\xi} \cdot x \cdot \iadone_x \in \travset },$$

$$(\mbox{Var}_r) \rulef{\raisebox{0cm}[0.5cm]{$
t_1 \cdot \rnode{decl}{\ianewin{x}} \cdot t_2 \cdot \lambda^r \overline{\xi} \cdot \rnode{x}{x} \in
 \travset$ \link[nodesep=2pt]{20}{x}{decl}
}}
{t_1 \cdot \ianewin{x} \cdot t_2 \cdot \lambda^r \overline{\xi} \cdot x \cdot 0_x \in \travset
}
\mbox{ if $t_2$ contains no overwrite of $x$},
$$

$$(\mbox{Var'}_r) \rulef{\raisebox{0cm}[0.5cm]{$t_1 \cdot \rnode{decl}{\ianewin{x}} \cdot t_2 \cdot \lambda^r \overline{\xi} \cdot \rnode{x}{x} \in \travset \link[nodesep=2pt]{20}{x}{decl}$}}
{t_1 \cdot \ianewin{x} \cdot t_2 \cdot \lambda^r \overline{\xi} \cdot x \cdot k_x \in \travset }
\mbox{ if $\lambda^{w_k} \cdot x$ is the last overwrite of $x$ in }
t_2. $$
\end{itemize}
\end{itemize}

\subsubsection{Game semantics correspondence}
The properties that we proved for computation trees and traversals
of the safe $\lambda$-calculus with constants can easily be lifted
to computation DAGs of \ialgol. In particular:
\begin{itemize}
\item constant traversal rules are well-behaved;
\item P-view of traversals are paths in the computation DAG;
\item the P-view of the reduction of a traversal is the reduction of the P-view,
and the O-view of a traversal is the O-view of its reduction (lemma
\ref{lem:redtrav_trav});
\item there is a mapping from vertices of the computation DAG to moves in the interaction game semantics;
\item there is a correspondence between traversals of the computation tree and plays in interaction game semantics;
\item consequently, there is a correspondence between the standard game semantics and
the set of justified sequences of nodes $\travset^{\upharpoonright
r}$.
\end{itemize}

\subsubsection{Game-semantic characterisation of safe terms}
Clearly, the computation DAG of a safe term is incrementally-bound.
By using the correspondence between traversals and plays, it is easy
to prove that incrementally-bound computation trees are denoted by
P-incrementally-justified strategies. Consequently, by lemma
\ref{lem:incrjustified_pointers_uniqu_recover}, P's pointers are superfluous in the
game semantics of safe \ialgol\ terms.

Since the game denotation of an \ialgol\ term is fully determined by
the set of complete plays, this pointer economy suggests that the
game denotation of a safe \ialgol\ can be represented in a compact
way. This raises the question of the decidability of observational
equivalence for safe \ialgol.


% fifth chapter
\chapter{Further possible developments}

In the previous chapter, we have given an account of the game
semantics of Safe $\lambda$-Calculus. However the nature of this
calculus is still not well known. We propose the following possible
roadmap for further research:
\begin{enumerate}
\item prove or disprove that observational equivalence is decidable for Safe \ialgol;
\item find a categorical interpretation of the Safe $\lambda$-Calculus;
\item study the proof theory obtained by the Curry-Howard isomorphism and determine whether it has nice properties that can be helpful in theorem proving;
\item determine which complexity class is characterized by the Safe-$\lambda$ calculus.
\end{enumerate}


In a more general direction of research, we would like to study the
class of languages for which pointers are uniquely recoverable. We
name this class PUR for ``Pointer Uniquely Recoverable''.

We proved that Safe $\lambda$-Calculus is a PUR-language. Another
example is the Serially Re-entrant Idealized Algol (SRIA) proposed
by Abramsky  in \cite{abramsky:mchecking_ia}. This language allows
multiple occurrences or uses of arguments, as long as they do not
overlap in time. In the game semantics denotation of a SRIA term
there is at most one pending occurrence of a question at any time.
Each move has therefore a unique justifier and consequently
justification pointers may be ignored. Safe \ialgol\ is not a
sublanguage of SRIA. One reason for this is that none of the two
Kierstead terms $\lambda f . f (\lambda x . f (\lambda y .y ))$ and
$\lambda f . f (\lambda x . f (\lambda y .x ))$ are Serially
Re-entrant whereas the first one is safe. Conversely, SRIA is not a
sublanguage of Safe \ialgol\ since the term $\lambda f g. f (\lambda
x . g (\lambda y .x ))$ where $f,g:((o,o),o)$ belongs to SRIA but
not to Safe \ialgol. SRIA and Safe \ialgol\ are therefore two
different examples of languages with pointer-less game semantics.

Finitary $\ialgol_2$ is also an example of PUR-language for which
observational equivalence is decidable. As we indicated in the first
chapter, decidability of observational equivalence is a very
appealing property which has immediate applications in the domain of
program verification. Intuitively, PUR-languages seem to be good
candidates of languages for which observational equivalence is
decidable. It would be interesting to discover classes of PUR
languages having this appealing property.

Another possible way to generate PUR-languages might be to constrain
the types of an existing language. In \cite{DBLP:conf/tlca/Joly01},
a notion of ``complexity'' is defined for $\lambda$-terms. It is
proved that a type $T$ can be generated from a finite set of
combinators if and only if there is a constant bounding the
complexity of every closed normal $\lambda$-term of type $T$;
consequently, the only inhabited finitely generated types are the
type of rank $\leq 2$ and the types $(A_1, A_2, \ldots, A_n, o)$
such that for all $i = 1..n$: $A_i = o$ , $A_i = o \rightarrow o$ or
$A_i = o^k \rightarrow o \rightarrow o$.

We know that imposing the first of these two type restrictions to
Finitary \ialgol\ leads to a PUR language. Is is also the case when
imposing the second type restriction?


\bibliographystyle{plainnat}
\bibliography{../../bib/gamesem,../../bib/modelchecking,../bib/proganalys,../../bib/higherorder}

%adds the bibliography to the table of contents
\addcontentsline{toc}{chapter}
         {\protect\numberline{Bibliography\hspace{-96pt}}}
\end{document}
