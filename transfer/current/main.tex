 \ifx\incrcompilation\undefined \input preamble.tex \fi
% 
%\includeonly{chap_previouswork,chap_gamesem,safe_background,safe_homog,safe_nonhomog,safe_gamesem2,chap_further}
%\includeonly{safe_gamesem2} 
                          
 
\author{William Blum}
\title{Transfer thesis}
\institution{Oxford University Computing Laboratory}

\begin{document}
\maketitle \tableofcontents
          
\chapapp{}
\part{Academic activities} %Previous work and plan of proposed work}
\chapter{First-Year work}

\section{Coursework}
I have attended the following courses: \emph{Automata Logic and
Games} in Hilary term 2005, \emph{Domain theory} in Michaelmas term
2005 and \emph{Categories Proofs and Programs} in Hilary term 2006.

\section{Teaching}

I was demonstrator for \emph{Network and Operating Systems}
practicals in Hilary term 2005, I tutored two groups of students for
the \emph{Introduction to Specification} classes (Hilary 2006). I
also did the marking for one group.

\section{Meetings and conferences}
\begin{itemize}
\item I attended Bonn spring school on GAMES in March 2005;

\item  I attended BCTCS (British Colloquium in
Theoretical Computer Science) in Nottingham in March 2005. There I
gave a talk about my MSc dissertation ``Termination analysis of a
subset of CoreML'';

\item I attended PAT \emph{Program transformation and Analysis} in Copenhagen, July 2005;

\item Marktoberdorf Summer School;
\item CSL (Computer Science Logic) August 2005:
I helped organizing the conference;
\item I visited the Isaac Newton Institute in Cambridge in February
2006.
\end{itemize}
I have also presented a project during the Computer Laboratory open
days.


\section{Summary of research activities}

\subsection{Game semantics}

During the past months, I have studied a restriction of
lambda-calculus called ``safe lambda-calculus''. \emph{Safety} is a
syntactic property originally defined by Knapik et al. in
\cite{KNU02} for higher-order recursion schemes (grammars). In their
paper they proved that the MSO theory of the term tree generated by
a safe recursion scheme of level $n$ is decidable. More recently,
Ong proved in \cite{OngLics2006} that the safety assumption is in
fact not necessary.

I am interested in the transposition of the safety property from
grammars to lambda terms. A definition of the safe
$\lambda$-calculus was first given in a technical report by Aehlig,
de Miranda and Ong in \cite{safety-mirlong2004}. One interesting
property of safe lambda terms is that performing substitution on
such terms does not involve renaming of the variable.

I have investigated different possible definitions of a safe lambda
calculus and I have proposed a more general notion of safety that do
not assume homogeneity of types while still preserving  the \emph{no
variable renaming} property.

I also tried to relate the safety restriction and the
\emph{size-change termination} property defined in
\cite{jones01,jones04}. Jones conjectured that any simply typed term
is size-change terminating, however Damien Sereni disproved this
conjecture by exhibiting a class of counter-examples
(\cite{serenistypesct05}). The terms of this class are safe simply
typed terms (not all of homogeneous type) and not size-change
terminating. This suggests that there is no real interesting
relation between safety and size-change termination.


Recently, inspired by my reading on game semantics
\citep{abramsky:game-semantics} and by the technics developed by
Luke Ong in \citep{OngLics2006}, I have proved a result on the game
semantics of safe terms: the pointers in the game semantics of safe
simply type terms can be recovered uniquely from the sequence of
moves. This result is similar to the standard result in game
semantics which says that pointers of strategies can be recovered
uniquely for arena of order 2 at most. A consequence of this result
is that the semantics of such terms can be described by (extended)
regular expressions \cite{ghicamccusker00}.


\subsection{Verification}

In parallel, I worked on a separate project with Matthew Hagues and
Luke Ong. We developed a SAT-based  model checker for verifying
Linear Temporal Logic formulae (LTL) on programs expressed as finite
state machines. Our approach consists in combining techniques
presented in two papers: \cite{hammer:truly,
DBLP:conf/cav/McMillan03}.

In \cite{DBLP:conf/cav/McMillan03}, McMillan describes an
acceleration technique for the SAT-based Bounded Model Checking
problem based on Craig interpolants that improves the standard
SAT-based model checking methods for positive instances.

In \citep{hammer:truly}, Hammer et al. introduced a new kind of
automata called \emph{Linearly Weak Alternating Automata},
abbreviated LWAA. The set of languages recognized by these automata
are exactly the set of languages definable in LTL. There is a
straightforward translation from LTL formulae to LWAAs. The size of
the resulting automaton is linear in the size of the LTL formula.
Checking emptiness of LWAA then amounts to finding in the
configuration graph a lasso verifying certain conditions.

Our approach can be summarized as follows: we translate the model
checking problem into an emptiness checking of a \emph{Linear Weak
Alternating Automaton}, or LWAA for short: the automata is empty if
and only if the formula is true. The emptiness of the automaton is
expressed in term of reachability problem. As in the traditional
SAT-based bounded-model checking approach (\cite{biere99symbolic}),
we then construct a boolean formula that it is satisfiable if and
only if the desired configuration is reachable in at most $k$ steps
(i.e. there is a counter-example of length $k$ at most).

Then instead of using the traditional SAT-solver technique, that
iterates $k$ until the completeness threshold is reached, we use the
acceleration method described in \cite{DBLP:conf/cav/McMillan03}.
The principle is the following: for every iteration of $k$, if the
formula is not satisfiable then we perform some over-approximation
of the set of initial configuration. If after over-approximating the
initial configuration, the final configuration can be reached in $k$
steps then we are still not sure that the formula has a valid
counter-example since the counter-example obtained may be spuriously
created by the over-approximation: hence we increase $k$ and move on
to the next iteration. However, if after performing several
over-approximation we reach a fixed point and the formula is still
not satisfiable (not counter-example of length $k$ at most) then we
know that there cannot be any counter-example of any length : we
have therefore reached the completeness threshold and we know that
the formula is true.


There are two reasons why we think that our approach may lead to a
gain of performance. First, determining emptiness of a LWAA is more
costly than determining emptiness of a B\"uchi automaton however we
save time during the construction of the automaton since the size of
a LWAA is linear in the length of the formula as opposed to the
standard translation which produces a B\"uchi automaton of size
exponential in the length of the formula. Secondly, in the case when
there is no counter-example, McMillan's acceleration method based on
over-approximation permits to detect quickly if we have reached the
completeness threshold.


%\cite{ckos2005}
We have produced an experimental implementation in OCaml and C. The
program parses a file in the NuSMV format (\cite{CAV02:nusmv})
containing the kripke structure of the model and the set of LTL
properties to verify. Our tools can be interfaced with two SAT
solvers: ZChaff \citep{zChaff} and MiniSat \citep{ES03}. We also use
BDD to perform simplification on the propositional formula and to
generate the CNF representation that the SAT solver takes as its
input.

Compared to the LWAASpin LTL model checker (\cite{hammer:truly}),
our tool performs quite poorly. As soon as a model is taken into
account, our procedure generate increasingly bigger propositional
formulae that the SAT solver struggle to solve. However for pure LTL
emptiness checking, our tool performs quite well.

It seems disappointing that our approach does not give good result
for model checking, however the reason seems to be that the
SAT-solvers we are using produce bad interpolants. In the future we
would like to interface our model checking tool with other SAT
solvers and interpolers.

There are also some optimizations that we have not finished to
implement. For instance, there are different ways of encoding the
bounded model checking problem into a propositional formula. We are
going to do some experiments do determine which encoding gives the
best result.

\chapter{Research plan}

I would like to extend the result I obtained for game semantics of
safe simply typed term to safe Idealized Algol terms (after having
given a proper definition of safe IA). Then I will investigate the
implication of such result in algorithmic game semantics.

In \cite{abramsky:mchecking_ia}, Abramsky has studied a language
called Serially Re-entrant Idealized Algol, or SRIA for short. This
language allows multiple occurrences or uses of arguments, as long
as they do not overlap in time. In the games of this language there
is at most one pending occurrence of a question at any time, so that
each move has a unique justifier and so justification pointers may
be ignored. I would like to find out whether there is a relationship
between SRIA and safe Idealized Algol.

I also want to investigate some application of game semantics to
program analysis and transformation by trying to extend the work of
Dimovski et al. (\cite{DBLP:conf/sas/DimovskiGL05}) on
data-absraction refinement based on game semantics.

Finally I will continue to work with Matthew Hagues and Luke Ong on
the LTL model checking problem.


\setcounter{chapter}{0}
\chapapp{Chapter}
\part{Summary of work so far}

The first chapter of this part is devoted to the presentation of the
basics and main results of game semantics. The categorical
interpretation of game semantics is presented as well as the full
abstraction result for \pcf. We also give a brief summary of the
main results in algorithmic game semantics. There is no personal
contribution in this chapter.

In the second chapter we present the \emph{Safe $\lambda$-Calculus}.
Originally, \emph{safety} has been introduced as a syntactical
restriction on higher-order grammars in order to show a decidability
result about MSO theory of infinite trees \citep{KNU02}. In
\cite{safety-mirlong2004}, Aehlig, de Miranda and Ong  proposed an
adaptation of the safety restriction to the $\lambda$-calculus. This
restriction gives rise to the Safe $\lambda$-Calculus. We first
present this calculus and then give a more general definition which
does not make any assumption on the types of the terms.

In the third chapter, following ideas described in
\cite{OngLics2006}, we introduce the notions of computation tree of
a simply-typed term and traversal over a computation tree. We prove
a theorem showing a correspondence between traversals of the
computation tree and the game semantics of a term. Based on that
correspondence, we give a characterisation of the game semantics of
safe terms by a property called ``incremental justification''. In
incrementally-justified strategies, pointers are superfluous (i.e.
they can be recovered uniquely from the underlying sequence of
moves). This simplification of the game semantics suggests some
potential applications in algorithmic game semantics. We finish the
chapter by extending the result to Safe \pcf\ and by giving the key
elements for an extension to full Safe Idealized Algol.


% first chapter
\chapter{Game semantics}

The aim of this chapter is to introduce game semantics. It starts
with a history of game semantics and a presentation of the full
abstraction problem for PCF which has been solved using game
semantics. It then goes on by introducing the basic notions of game
semantics and by giving a categorical interpretation of games.
Finally we show how games are used to define a syntax-independent
model of programming languages like PCF and Idealized Algol (IA).

This chapter is largely based on the tutorial by Samson Abramsky tutorial on Game Semantics \cite{AM98a}.
Most of the proof will be omitted and we refer the reader to
\cite{hylandong_pcf, abramsky94full} for a deeper description
of game semantics with complete proofs.

\section{History}

\subsection{Game semantics}

In the 1950s, Paul Lorenzen invented Game semantics as a tool to
study semantics of intuitionistic logic \citep{lor61}.

Four decade later, Abramsky proved the full completeness of
Multiplicative Linear Logic (MLL) using game semantics
\citep{abramsky92games}. Shortly after, game semantics has been used
as tool to study models of programming languages. In game semantics,
the meaning of a program is given by a strategy in a two-player
game. One player, the Opponent, represents the environment while the
other, the Proponent, represents the system.


\subsection{Model of programming languages}

Before the 1980s, there were many approaches to define models for
programming languages. Among the successful ones, there were the
axiomatic, operational and denotational semantics:
\begin{itemize}
\item Operational semantics gives a meaning to a program by describing the
behaviour of a machine executing the program. It is defined formally
by giving a state transition system.
\item Axiomatic semantics defined the behaviour of the program
with axioms and is used to prove program correctness by static
analysis of the code of the program.
\item The denotational semantics approach consists in mapping a program to a mathematical structure
having good properties such as compositionality. This mapping is
achieved by structural induction on the syntax of the program.
\end{itemize}

In the 1990s, three different independent research groups: Samson
Abramsky, Radhakrishnan Jagadeesan and Pasquale Malacaria
\citep{abramsky94full}, Martin Hyland and Luke Ong
\citep{hylandong_pcf} and Nickau \citep{Nickau:lfcs94} have
introduced game semantics, a new kind of semantics, in order to
solve a long standing problem in the semanticists community :
finding a fully abstract model for PCF.

\subsection{The problem of full abstraction for PCF}

PCF is a simple programming language introduced in a classical paper
by Plotkin ``LCF considered as a programming language''
(\cite{DBLP:journals/tcs/Plotkin77}). PCF is based on LCF, the Logic
of Computable Functions devised by Dana Scott in \cite{scott_lcf}.
It is a simply typed lambda calculus extended with arithmetic
operators, conditional and recursion.

The problem of the Full Abstraction for PCF goes back to the 1970s.
In \citep{scott93}, Scott gave a model for PCF based on domain
theory. This model gives a sound interpretation of observational
equivalence: if two terms have the same domain theoretic
interpretation then they are observationally equivalent. However the
converse is not true: there exist two PCF terms which are
observationally equivalent but have different domain theoretic
denotation. We say that the model is not fully abstract.

The key reason why the domain theoretic model of PCF is not fully
abstract is that the parallel-or operator defined by the following
truth table
\begin{center}
\begin{tabular}{l|lll}
p-or  & $\bot$ & tt & ff \\ \hline
$\bot$ & $\bot$ & tt & $\bot$\\
tt & tt & tt & tt\\
ff & $\bot$ & tt & ff\\
\end{tabular}
\end{center}
is not definable as a PCF term! It is possible to create two
different PCF terms that always behave the same except when they are
apply to a term computing p-or. Since p-or is not definable in PCF,
these two terms will have the same denotation. This implies that the
model is not fully abstract.

One can patch PCF by adding the operator $p-or$, the resulting
language ``PCF+p-or'' now becomes fully-abstracted by Scott domain
theoretic model \citep{DBLP:journals/tcs/Plotkin77}. However the
language we are now dealing with is strictly more powerful than PCF,
it allows parallel execution of commands whereas PCF only permits
sequential execution.

Another approach consists in getting rid of the undefinable elements
(like p-or) by strengthening the conditions on the function used in
the model (a condition stronger than strictness and continuity) but
unfortunately this approach did not succeed.

The only successful approaches to obtain a fully abstract model for
PCF were the ones taken by Ambramsky, Jagadeesan and Malacaria
\citep{abramsky94full}, Hyland and Ong \citep{hylandong_pcf} and
Nickau \citep{Nickau:lfcs94}, all based on game semantics.

This result has then been adapted to other varieties of programming
paradigm including languages with stores (Idealized Algol),
call-by-value \citep{honda99gametheoretic, abramsky98callbyvalue}
and call-by-name, general referencees
\citep{DBLP:conf/lics/AbramskyHM98}, polymorphism
\citep{DBLP:journals/apal/AbramskyJ05}, control features
(continuation and exception), non determinism, concurrency. In all
these cases, the game semantics model led to a syntax-independent
fully abstract model of the corresponding language.

\section{Games}
\label{sec:catgames}

We now introduce formally the notion of game that will be used in
the following section to give a model of the programming languages
PCF and Idealized Algol. The definitions are taken from
\cite{abramsky:game-semantics, hylandong_pcf, abramsky94full}.


\subsection{Arenas and Games}

The games we are interested in are two-players games. The players are named O for Opponent and P for Proponent.

The game played by O and P is constraint by something called
\emph{arena}. The arena defines the possible moves of the game. By
analogy with real board games, the arena represents the board
together with the rules that tell how players can make their moves
on the board. In fact the analogy with board game stops here. Our
games can be thought as dialog games: one person O interviews
another person P, P tries to answer the initial O-question by
possibly asking O some precisions about its initial question.
Moreover, the notion of winner and winning strategy will not be
relevant in our setting.


More formally, the arena can be seen as a forest of trees whose nodes are possible questions and leaves are possible answers.
The arena is partitioned into two kinds of moves: the moves that can be played by P and the ones that can be played by O.
A move is either a question to the other player or an answer to a question previously asked by the other player.

Each move of the game must be justified by another move that has already been played by the other player. This justification relation
is induced by the edges of the forest arena. Moreover, an answer must always be justified by the question that it answers and a question
is always justified by another question.

\begin{dfn}[Arena]
An arena is a structure $\langle M, \lambda, \vdash \rangle$ where:
\begin{itemize}
\item $M$ is the set of possible moves;
\item $(M,\vdash)$ is a forest of trees;

\item $\lambda : M \rightarrow \{ O, P\} \times \{Q, A\}$ is a labeling functions indicating whether a given move
    is a question or an answer and whether it can be played by O or by P.

    $\lambda = [\lambda^{OP},\lambda^{QA}]$ where $\lambda^{OP} : M \rightarrow  \{ O, P\}$
    and $\lambda^{QA} : M \rightarrow  \{ Q, A\}$.

    \begin{itemize}
    \item If $\lambda^{OP} (m) = O$, we call $m$ and O-move otherwise $m$ is a P-move.
    $\lambda^{QA} (m) = Q$ indicates that $m$ is a question otherwise $m$ is an answer.

    \item For any leaf $l$ of the tree $(M,\vdash)$, $\lambda^{QA} (l) = A$ and for any node
    $n \in (M,\vdash)$, $\lambda^{QA} (n) = Q$.
    \end{itemize}

\item The forest of tree $(M,\vdash)$ respect the following condition:
    \begin{itemize}
    \item[(e1)] The roots are O-moves: for any root $r$ of $(M,\vdash)$, $\lambda^{OP} (r) = O$.
    \item[(e2)] Answers are enabled by questions: $m \vdash n  \zand \lambda^{QA}(n) = A \imp \lambda^{QA}(m) = Q$.
    % Or more succinctly, if we write $\dashv$ the relation $\vdash^-1$: $\lambda^{QA} \left( \dashv( (\lambda^{QA})^{-1}(\{A\}) ) \right) = \{ O \}$
    \item[(e3)] A player move must be justified by a move played by the other player:
         $m\vdash n \imp \lambda^{OP}(m) \neq \lambda^{OP}(n)$.
    \end{itemize}
\end{itemize}
\end{dfn}

For commodity we write the set $\{O,P\} \times \{Q,A\}$ as $\{OQ,OA,PQ,PA\}$.
$\overline{\lambda}$ denotes the labeling function $\lambda$ with the question and answer swapped. For instance:
$$\overline{\lambda(m)} = OQ \iff \lambda(m) = PQ$$

The roots of the forest of tree $(M,\vdash)$ are the \emph{initial moves}.

For example, the simplest possible arena is written $\mathbf{1}$ and
denotes the arena which set of moves $M$ is empty.

\begin{exmp}[The flat arena]
\label{exmp:flatarena}

 Let $A$ be any countable set then the flat arena over $A$
is defined to be the arena $\langle M, \lambda, \vdash \rangle$ such
that $M$ has one move $q$ with $\lambda(q) = OQ$ and for each
element in $A$, there is a corresponding move $a_i$ in $M$ with
$\lambda(a_i) = PA$ for some $i \in \nat$. The enabling relation
$\vdash$ is defined to be $\{ q \vdash a_i \ | i \in \nat \}$.

This arena is represented by the following tree:
\begin{center}
  \pstree[levelsep=6ex]
    { \TR{$q$} }
    {    \TR{$a_1$} \TR{$a_2$} \TR{\ldots} }
\end{center}
The vertices represent the moves and the edges represent the
enabling relation.

The flat arena over $\nat$ and $\mathbb{B}$ is written
$\mathbf{int}$ and  $\mathbf{bool}$ respectively.

\end{exmp}

Once the arena has been defined, the bases of the game are set and the players have something to play with.
We now need to describe the state of the game, for that purpose
we introduced \emph{justified sequences of moves}. Sequence of moves are used to record the history of all the moves that have been
played.

\begin{dfn}[Justified sequence of moves]
A justified sequence is a sequence of moves $s$ together with an associated sequence of pointers. Any
move $m$ in the sequence that is not initial has as pointer that points to a previous move $n$ that justifies it (i.e. $n \vdash m$).
\end{dfn}

The pointers of a justified sequences are represented with arrows.
This is an example of justified sequence of moves:
$$\rnode{q4}{q}^4
\rnode{q3}{q}^3 \rnode{q2}{q}^2 \rnode{q3b}{q}^3 \rnode{q2b}{q}^2
\rnode{q1}{q}^1 \bkptrc{q3}{q4} \bkptrc{q2}{q3}
\bkptrc[ncurv=0.6]{q3b}{q4} \bkptrc{q2b}{q3b}$$

The first move of a justified sequence must be an O-move since
initial moves are all O-moves.

Notation: we write $s t$ or sometimes $s \cdot t$ do denote the
sequences obtain by concatenating $s$ and $t$. The empty sequence is
written $\epsilon$.

 A justified sequence has two particular subsequences which
will be of particular interest later on when we introduce
strategies. These subsequences are called the P-view and the O-view
of the sequence. The idea is that a view describes the local context
of the game. Here is the formal definition:

\begin{dfn}[View]
Given a justified sequence of moves $s$. We define the proponent view (P-view) noted $\pview{s}$ by induction:
\begin{align*}
\pview{\epsilon} &= \epsilon \\
\pview{s \cdot m} &= \pview{s} \cdot \ m && \mbox{ if $m$ is a P-move} \\
\pview{s \cdot m} &= m && \mbox{ if $m$ is initial (O-move) } \\
\pview{ s \cdot \rnode{m}{m} \cdot t \cdot \rnode{n}{n} \bkptra{50}{n}{m} } &=
 \pview{s} \cdot \rnode{mm}{m} \cdot \rnode{nn}{n} \bkptra{70}{nn}{mm} && \mbox{ if $n$ is a non initial O-move }
\end{align*}
The O-view $\oview{s}$ is defined similarly:
\begin{align*}
\oview{\epsilon} &= \epsilon \\
\oview{s \cdot m} &= \oview{s} \cdot \ m && \mbox{ if $m$ is a O-move} \\
\oview{ s \cdot \rnode{m}{m} \cdot t \cdot \rnode{n}{n} \bkptra{50}{n}{m} } &=
 \pview{s} \cdot \rnode{mm}{m} \cdot \rnode{nn}{n} \bkptra{70}{nn}{mm} && \mbox{ if $n$ is a P-move }
\end{align*}
\end{dfn}

In fact not all justified sequences will be of interest for the
games that we will use. We call \emph{legal position} any justified
sequence verifying two additional conditions: alternation and
visibility. Alternation says that players O and P plays
alternatively. Visibility expresses that each non-initial move is
justified by a move situated in the local context at that point.
Intuitively, the visibility condition gives some coherence to the
justification pointers of the sequence.

\begin{dfn}[Legal position]
A legal position is a justified sequence of move $s$ respecting the following constraint:
\begin{itemize}
\item Alternation: For any subsequence $m \cdot n$ of $s$, $\lambda^{OP}(m) \neq \lambda^{OP}(n)$.
\item Visibility: For any subsequence $t m$ of $s$ where $m$ is not initial, if $m$ is a P-move then $m$ points to a move in $\pview{s}$
and if $m$ is a O-move then $m$ points to a move in $\oview{s}$.
\end{itemize}

The set of legal position of an arena $A$ is noted $L_A$.
\end{dfn}

We say that a move $n$ is hereditarily justified by a move $m$ if there is a sequence of move
$m_1, \ldots, m_q$ such that:
$$ m \vdash m_1 \vdash m_2 \vdash \ldots m_q \vdash n$$
If a move has no justification pointer, we says that it is an
\emph{initial move} (in that case it must be a root of the forest
arena).

Suppose that $n$ is an occurrence of a move in the sequence $s$ then
$s \upharpoonright n$ denotes the subsequence of $s$ containing all the moves hereditarily justified by $n$.
Similarly, $s \upharpoonright I$ denotes the
subsequence of $s$ containing all the moves hereditarily justified by the moves in $I$.

\begin{dfn}[Game]
A game is a structure $\langle M, \lambda, \vdash, P \rangle$ such that
\begin{itemize}
\item $ \langle M, \lambda, \vdash \rangle$ is an arena.
\item $P$ is called the set of valid positions, it is:
    \begin{itemize}
    \item a non-empty prefix closed subset of the set of legal position
    \item closed by initial hereditary filtering: if $s$ is a valid position then for any set $I$ of occurrences of initial moves
    in $s$, $s\upharpoonright I$ is also a valid position.
    \end{itemize}
\end{itemize}
\end{dfn}

\begin{exmp}  Consider the flat arena  $\mathbf{int}$.
The set of valid position $P = \{ \epsilon, q \} \union \{ q \cdot
a_i \ | i \in \nat \}$ defines a game on the arena $\mathbf{int}$.
\end{exmp}

\subsection{Constructions on games}
\label{sec:gameconstruction}

We now define game constructors that will be useful later on.

Consider the two functions $f : A \rightarrow C$ and $g : B
\rightarrow C$, we write $[f,g]$ to denote the pairing of $f$ and
$g$ defined on the direct sum $A + B$. Given a game $A$ with a set
of moves $M_A$, we use the filtering operator $s \upharpoonright A$
do denote the subsequence of $s$ consisting of all moves in $M_A$.
Although this notation conflicts with the hereditarily filtering
operator, it should not cause any confusion.

\subsubsection{Tensor product}
Given two games $A$ and $B$ we define the tensor product constructor
$A \otimes B$ as follows:
\begin{eqnarray*}
  M_{A \otimes B} &=& M_A + M_B \\
  \lambda_{A\otimes B} &=& [\lambda_A,\lambda_B] \\
  \vdash_{A\otimes B} & = & \vdash_{A}\ \union\ \vdash_{B} \\
  P_{A\otimes B} & = & \{ s \in L_{A\otimes B} | s \upharpoonright A \in P_A \wedge s \ \upharpoonright B \in P_B  \}.
\end{eqnarray*}

In particular,  $n$ is initial in $A\otimes B$ if and only if $n$ is
initial in A or B. And $m \vdash_{A\otimes B} n$  holds if and only if $m
\vdash_{A} n$ or $m \vdash_{B} n$ holds.

\subsubsection{Function space}
The game $A \otimes B$ is defined as follows:
\begin{eqnarray*}
  M_{A \multimap B} &=& M_A + M_B \\
  \lambda_{A\multimap B} &=& [\overline{\lambda_A},\lambda_B] \\
  \vdash_{A\multimap B} & = & \vdash_{A}\ \union\ \vdash_{B}\ \union\  \{ (m,n) \ |\ m \mbox{ initial in } B \wedge n \mbox{ initial in } A \} \\
  P_{A\otimes B} & = & \{ s \in L_{A\otimes B} | s \upharpoonright A \in P_A \wedge s \ \upharpoonright B \in P_B  \}.
\end{eqnarray*}

\subsubsection{Cartesian product}
The game $A \& B$ is defined as follows:
\begin{eqnarray*}
  M_{A \& B} &=& M_A + M_B \\
  \lambda_{A\& B} &=& [\lambda_A,\lambda_B] \\
  \vdash_{A\& B} & = & \vdash_{A}\ \union\ \vdash_{B} \\
  P_{A\& B} & = & \{ s \in L_{A\otimes B} | s \upharpoonright A \in P_A \wedge s \ \upharpoonright B = \epsilon  \} \\
        &&   \union \{ s \in L_{A\otimes B} | s \upharpoonright A \in P_B \wedge s \ \upharpoonright A = \epsilon  \}.
\end{eqnarray*}

A play of the game $A \& B$ is either a play of $A$ or a play of $B$ whether a play
of the game $A \otimes B$ may be an interleaving of plays on $A$ and plays on $B$.

\subsection{Representation of plays}

Plays of the game are usually represented in a table diagram. The
columns of the table correspond to the different components of the
arena and each row corresponds to one move in the play. The first
row always represents an O-move, this is because O is the only
player who can open a game (since roots of the arena are O-moves).

As an example the play
$$\rnode{q1}{q}\
 \rnode{q2}{q}
 \ \rnode{a2}{8}
\  \rnode{a1}{12}
  \bkptrc{a1}{q1}
\bkptrc{a2}{q2} $$
on the
game $\textbf{int} \multimap \textbf{int} $ can be represented by
the following diagram:

\begin{center}
\begin{tabular}{cccc}
\textbf{int} & $\imp$ & \textbf{int} & \\
&& q & O\\
q  &&& P\\
8  &&& O\\
&& 12 & P
\end{tabular}
\end{center}

When it is necessary, the justification pointers of the play can also
be shown on the diagram.


\subsection{Strategy}

\subsubsection{Definition}

During a game, the player who has to play may have several choices
for his next move. The move that he makes is chosen according to a
given strategy.

A strategy is a rule telling the player which move to make when the
game is in a given position. More abstractly, a strategy is a
partial function mapping legal position where Proponent has to move
to P-moves.

\begin{dfn}[Strategy]
A strategy for player P on a given game $\langle M, \lambda, \vdash, P \rangle$ is a
non-empty set of even-length positions from $P$ such that:
\begin{enumerate}
\item (\emph{no unreachable position}) $sab \in \sigma \imp s \in \sigma$
\item (\emph{determinacy}) $sab, sac \in \sigma \quad \imp \quad  b = c$  and $b$ has the same justifier as
$c$.
\end{enumerate}
\end{dfn}

The idea is that the presence of the even-length sequence $s a b$ in
$\sigma$ tells the player P that whenever the game is in position
$s$ and player O plays the move $a$ then it must respond by playing
the move $b$.

The first condition ensures that the strategy $\sigma$ only
considers positions that the strategy itself could have led to in a
previous move. The second condition in the definition requires that
this choice of move is deterministic (i.e. there is a function $f$
from the set of odd length position to the set of moves $M$ such
that $f(s a) = b$).


For any game $A$, the smallest possible strategy is the strategy
that never respond given by $\{ \epsilon \}$. It is called the
\emph{empty strategy} and denoted $\bot$.

\subsubsection{Copy-cat strategy}

For any arena $A$ there is a strategy on the game $A \multimap A$
called the \emph{copy-cat strategy}. We write $A_1$ and $A_2$ to
denote the first and second copy of the arena $A$ in the game $A
\multimap A$. If $A$ is the arena $A_1$ then $A^\perp$ denotes the
arena $A_2$ and reciprocally.

Let $A$ be one of the arena $A_1$ or $A_2$. The copy-cat strategy
operates as follows: whenever P has to respond to an O-move played
in $A$, it replicates the move played by O in the arena $A^{\perp}$
after that $O$ has to respond in $A^{\perp}$ and $P$ replicates this
response in $(A^\perp)^\perp = A$ and so on and so forth.


More formally, the copy-cat strategy is defined by:
$$ \textsf{id}_A = \{ s \in P^{\textsf{even}}_{A \multimap A} \ | \ \forall t \sqsubseteq^{\textsf{even}} s\ .\ t \upharpoonright A_1 = t \upharpoonright A_2 \}$$
where $P^{\textsf{even}}_A$ denotes the valid position of even
length in the game $A$ and $t \sqsubseteq^{\textsf{even}} s$ denotes
that $t$ is an even length prefix of $s$.

The copy-cat strategy is also called \emph{identity strategy} since
it is the identity for strategy composition as we will see in the
next paragraph.

\begin{exmp} The copy-cat strategy on $\textbf{int}$ is:
$$\begin{array}{ccc}
\textbf{int} & \imp & \textbf{int} \\
&& q\\
q \\
n \\
&& n
\end{array}
$$
Note that we introduced this type of diagram to represent plays of
games but, as we can see here, the same diagrams can be used to
represent strategies when the play represented is general enough.

The copy-cat strategy on $\textbf{int} \typar \textbf{int}$ is given
by the following diagram:
$$\begin{array}{ccccccc}
(\textbf{int} & \imp & \textbf{int}) & \imp & (\textbf{int} & \imp & \textbf{int}) \\
&&&& && q\\
&& q\\
q \\
&&&& q \\
&&&& m \\
m\\
&& n \\
&&&& && n
\end{array}$$
\end{exmp}

\subsubsection{Composition}

It is well-known that any model of the simply typed lambda-calculus
is a cartesian closed category \citep{CroleRL:catt}. Games are used
to give a fully-abstract model of PCF, an extended simply typed
lambda calculus, therefore the game model should fit into a
cartesian closed category. This category will have games as objects
and strategies as morphisms. In a category, morphisms should be able
to compose together, therefore there should be an appropriate notion
of strategy composition.

Composition of strategies is an essential feature of game semantics.
As we will see in the following section, in the game model of PCF,
strategies represent programs. Therefore, strategy composition will
prove to be very useful : obtaining the model of a composed program
boils down to composing the strategies of the composing programs.

The way composition is defined for strategies is similar to
``parallel composition plus hiding'' in the trace semantics of CSP
\citep{hoare_csp}. Consider two strategies $\sigma : A \multimap B$
and $\tau : B \multimap C$ that we wish to compose.

For any sequence of moves $u$ on three arenas $A$, $B$, $C$, we call
projection of $s$ on the game $A \multimap B$ and we note $u
\upharpoonright A,B$ the subsequence of $s$ obtained by removing
from $u$ the moves in $C$ and pointers to moves in $C$. The
projection on $B \multimap C$ is defined similarly.

The definition of the projection on $A \multimap B$ differs
slightly: $u \upharpoonright A,C$ is the subsequence of $u$
consisting of the moves from $A$ and $C$ with some additional
pointers: we add a pointer from $a \in A$ to $c\in C$ whenever $a$
points to some move $b \in B$ itself pointing to $c$. All the
pointers to moves in $B$ are removed.


First we remark that for a given legal position $s$ in the game $A
\multimap C$, there is what is called an \emph{uncovering} of $s$.
The uncovering of $s$ is the maximal justified sequence of moves $u$
from the games $A$, $B$ and $C$ such that:
\begin{itemize}
\item The sequence $s$, considered as a pointer-less sequence, is a subsequence of
$u$;
\item the projection of $u$ on the game $A \multimap B$ lies in the
strategy $\sigma$;
\item the projection of $u$ on the game $B \multimap C$
lies in the strategy $\tau$;
\item and the projection of $u$ on the game $A \multimap C$ is a subsequence of $s$ (here the term ``subsequence'' refers to the sequence of nodes together with the auxiliary sequence of pointers).
\end{itemize}
This uncovering, noted $uncover(s, \sigma, \tau)$, is
defined uniquely for given strategies $\sigma$, $\tau$ and legal
position $s$ (this is proved in part II of \cite{hylandong_pcf}).

We define $\sigma \| \tau $ to be the set of uncovering of legal
positions in $A \multimap C$:
$$ \sigma \| \tau = \{ uncover(s, \sigma, \tau) \ | \ s \mbox{ is a legal position in } A \multimap C \}$$

The composition of $\sigma$, $\tau$ is defined to be the set of
projections of uncovering of legal positions in $A \multimap C$:

\begin{dfn}[Strategy composition]
Consider $\sigma : A \multimap B$ and  $\tau : B \multimap C$ two
strategies. We define $\sigma ; \tau$ to be:
$$ \sigma ; \tau = \{ u \upharpoonright A,C \ | \ u \in \sigma \|
\tau \}$$
\end{dfn}

It can be verified that composition is well-defined and associative
\citep{hylandong_pcf} and that the copy-cat strategy $\textsf{id}_A$ is the identity for composition.

\subsubsection{Constraint on strategies}

Different classes of strategies will be considered depending on the
features of the language that we want to model. Here is a list of
common restrictions that we will consider:
\begin{itemize}
\item \emph{Well-bracketing:} In a well-bracketed strategies the players always answer the last unanswered question (called the pending question) first.
If we represent Opponent's question as ``['', Proponent's answer as
``]'', Proponent's question as ``('' and Opponent's answers as ``)''
then requiring that the last pending question is answered first is
the same as requiring that the string representing the play is a
prefix of a well-bracketed sequence.

\item \emph{History-free strategies:} A strategy is history-free if the Proponent's move at any position of the game where he has to play
is determined by the last move of the Opponent. In other words, the
history prior to the last move is ignored by the Proponent when
deciding how to respond.

\item \emph{History-sensitive strategies:} The Proponent follows a history-sensitive strategy if he needs to have access to the full
history of the moves in order to decide which move to make.

\item \emph{Innocence:} a strategy is innocent if it determines Proponent's moves based on a restricted view of the history of the play, mainly the P-view
at that point. Such strategies can be specified by a partial
function mapping P-views to P-moves. However not every partial
function from P-views to P-moves gives rise to an innocent strategy
(a sufficient condition is given in \cite{hylandong_pcf}).
\end{itemize}

The formal definition of innocence follows:
\begin{dfn}[Innocence]
Given positions $sab, ta \in L_A$ where $sab$ has even length and
$\pview{sa} = \pview{ta}$, there is a unique extension of $ta$ by
the move $b$ together with a justification pointer such that
$\pview{sab} = \pview{sa}$. We write this extension
$\textsf{match}(sab,ta)$.

The strategy $\sigma:A$ is \emph{innocent} if and only if:
$$ \left(
     \begin{array}{c}
       \pview{sa} = \pview{ta} \\
       sab \in \sigma \\
       t\in \sigma \wedge ta \in P_A \\
     \end{array}
   \right)
\quad \imp\quad  \textsf{match}(sab,ta) \in \sigma$$

\end{dfn}


\subsection{Categorical interpretation of games}

In this section we recall some results about the categorical representation of Games.
These results with complete details and proofs can be found in \cite{McC96b,hylandong_pcf,abramsky94full}.
We refer the reader to \cite{CroleRL:catt} for more information about category theory.

We consider the category $\mathcal{G}$ whose objects are games and morphisms are
strategies. A morphism from $A$ to $B$ is a strategy on the game $A \multimap B$.

Three other sub-categories of $\mathcal{G}$ are considered: each of them correspond to some restriction on strategies:
$\mathcal{G}_i$ is the sub-category
of $\mathcal{G}$ whose morphisms are the innocent strategies,
$\mathcal{G}_b$ has only the well-bracketed strategies and $\mathcal{G}_{ib}$ has the innocent and well-bracketed strategies.

\begin{prop}
$\mathcal{G}$, $\mathcal{G}_i$, $\mathcal{G}_b$ and $\mathcal{G}_{ib}$ are categories.
\end{prop}

Proving this requires to prove that composition of strategies is well-defined, associative, has a unit (the copy-cat strategy), preserves innocence and
well-bracketedness. See \cite{hylandong_pcf,abramsky94full} for a proof.


\subsubsection{Monoidal structure}

We have already defined the tensor product on games in section \ref{sec:gameconstruction}.
We now define the corresponding transformation on morphisms:
given two strategies $\sigma : A \multimap B$ and $\tau : C \multimap D$ the strategy
$\sigma \otimes \tau : (A \otimes C) \multimap (B\otimes D)$ is defined by:
$$ \sigma \otimes \tau = \{ s \in L_{A \otimes C \multimap B\otimes D} \ s \upharpoonright A,B \in \sigma
\wedge s \upharpoonright C,D \in \tau \}$$

It can be shown that the tensor product is associative, commutative and has
$I = \langle \emptyset, \emptyset,\emptyset, \{ \epsilon \} \rangle $ as identity.
Hence the game categories $\mathcal{G}$ is a symmetric monoidal categories. Moreover
$\mathcal{G}_i$ and  $\mathcal{G}_b$ are sub-symmetric monoidal categories of $\mathcal{G}$,
and $\mathcal{G}_{ib}$ is a sub-symmetric monoidal category of $\mathcal{G}_i$, $\mathcal{G}_b$ and
$\mathcal{G}$.

\subsubsection{Closed structure}

For any game $A$, $B$ and $C$,
to any strategy $\sigma : A\otimes B \multimap C$, there is a corresponding strategy
$\tau : A\otimes B \multimap C$ obtained by relabeling the moves in $\sigma$. This transformation
is in fact an isomorphism: the hom-set $\mathcal{G}(A\otimes B, C)$ is isomorphic to the hom-set
$\mathcal{G}(A,B\multimap C)$. Hence $\mathcal{G}$ is an autonomous (i.e. symmetric monoidal closed) category.

$\mathcal{G}_i$ and  $\mathcal{G}_b$ are sub-autonomous categories of $\mathcal{G}$,
and $\mathcal{G}_{ib}$ is a sub-autonomous category of $\mathcal{G}_i$, $\mathcal{G}_b$ and
$\mathcal{G}$.

\subsubsection{Cartesian product}
The cartesian product defined in section \ref{sec:gameconstruction} is indeed a cartesian product in the category
$\mathcal{G}$, $\mathcal{G}_i$, $\mathcal{G}_b$ and $\mathcal{G}_{ib}$.

The projections $\pi_1:A \& B \rightarrow A$ and $\pi_1:A \& B \rightarrow B$ are given by the obvious copy-cat strategies.
Given two category morphisms $\sigma :C \rightarrow A$ and $\tau : C \rightarrow B$ the pairing function
$\langle \sigma, \tau \rangle : C \rightarrow A \& B$ is given by:
\begin{eqnarray*}
\langle \sigma, \tau \rangle &=& \{ s \in L_{C\multimap A\&B} \ | \ s \upharpoonright C,A \in \sigma \wedge s \upharpoonright B = \epsilon  \} \\
&\union& \{ s \in L_{C\multimap A\&B} \ | \ s \upharpoonright C,A \in \sigma \wedge s \upharpoonright B = \epsilon  \}
\end{eqnarray*}

\subsubsection{Cartesian closed structure}
Having defined the cartesian product is not enough to turn $\mathcal{G}$ into a cartesian closed category :
we also need to define a terminal object $I$ and the exponential construct $A \imp B$ for any two games $A$ and $B$.
In fact, this cannot be done in the current categories $\mathcal{G}$ and we have to move on to another category
of games noted $\mathcal{C}$ whose objects and morphisms are certain sub-classes of games and strategies.

Before introducing the category $\mathcal{C}$ we need some new definitions:


For any game $A$ we define the exponential game noted $!A$.
The game $!A$ corresponds to a repeated version of the game $A$. Plays of $!A$ are interleaving of plays of
$A$. It is defined as follows:
\begin{eqnarray*}
  M_{!A} &=& M_A \\
  \lambda_{!A} &=& \lambda_A \\
  \vdash_{!A} & = & \vdash_{A} \\
  P_{!A} & = & \{ s \in L_{!A} | \mbox{ for each initial move $m$, } s \upharpoonright m \in P_A \}
\end{eqnarray*}
The following equalities hold:
\begin{eqnarray*}
  !(A \& B) &=& !A \otimes !B\\
  I &=& !I
\end{eqnarray*}

\begin{dfn}[Well-opened games]
A game $A$ is well-opened if for any position $s \in P_A$ the only initial move is the first
one.
\end{dfn}

Well-opened games have single thread of dialog. Then can be turned into games with multiple-thread of dialog
using the promotion operator:

\begin{dfn}[Promotion]
Consider a well-opened game $B$.
Given a strategy on ${!A} \multimap B$, we define it promotion $\sigma^\dagger : {!A} \multimap {!B}$ to be the
strategy which plays several copies of $\sigma$. It is formally defined by:
$$ \sigma^\dagger = \{ s \in L_{{!A} \multimap !B} \ | \ \mbox{ for all initial $m$, } s \upharpoonright m \in \sigma  \}.$$
\end{dfn}

It can be shown that promotion is well-defined (it is indeed a strategy) and that it preserves innocence and
well-bracketedness.


We now introduce the category of well-opened games:
\begin{dfn}[Category of well-opened games]
The category $\mathcal{C}$ of well-opened games is defined as follow:
\begin{enumerate}
\item The objects are the well-opened games,
\item a morphism $\sigma : A \rightarrow B$ is a strategy for the game $!A \multimap B$,
\item the identity map for $A$ is the copy-cat strategy on $!A \multimap A$ (which is well-defined for well-opened games).
It is called dereliction, noted
$\textsf{der}_A$ and defined formally by:
$$ \textsf{der}_A = \{ s \in P^{\textsf{even}}_{{!A} \multimap A} \ | \ \forall t \sqsubseteq^{\textsf{even}} s \ . \ t \upharpoonright {!A} = t \upharpoonright A \},$$
\item composition of morphisms $\sigma : {!A} \multimap B$ and $\tau : {!B} \multimap C$ is defined to be
the strategy $\sigma^\dagger;\tau$ on the game ${!A} \multimap C$.
\end{enumerate}
\end{dfn}
$\mathcal{C}$ is a well-defined category and the three sub-categories
$\mathcal{C}_i$, $\mathcal{C}_b$, $\mathcal{C}_{ib}$ corresponding to sub-category
with innocent strategies, well-bracketed strategies and innocent and well-bracketed strategies respectively.


The category $\mathcal{C}$ has a terminal object $I$, for any two games $A$ and $B$ a product $A \& B$ and
an exponential $A \imp B$. Moreover the hom-sets $\mathcal{C}(A \& B,C)$ and
$\mathcal{C}(A,!B \multimap C)$ are isomorphic. Indeed:
\begin{eqnarray*}
\mathcal{C}(A\& B,C) &=& \mathcal{G}(!(A\& B),C) \\
&=& \mathcal{G}({!A}\otimes {!B}),C) \\
&\cong& \mathcal{G}({!A}, {!B} \multimap C) \qquad  \mbox{($\mathcal{G}$ is a closed monoidal category)}\\
&=& \mathcal{C}(A, {!B} \multimap C)
\end{eqnarray*}
Hence $\mathcal{C}$ is a cartesian closed category. Moreover $\mathcal{C}_i$ and $\mathcal{C}_b$
are sub-cartesian closed caterogies of $\mathcal{C}$ and $\mathcal{C}_{ib}$ is as sub-cartesian closed category
of each of $\mathcal{C}$, $\mathcal{C}_i$ and $\mathcal{C}_b$.



\subsubsection{Order enrichment}

Strategies can be ordered using the inclusion ordering.
The set of strategies on a given game $A$ is a pointed directed complete partial order under this ordering: the
least upper bounds is the union of two strategies and the least element is the empty strategy $\{ \epsilon \}$.

The category  $\mathcal{C}$ and  $\mathcal{G}$ are cpo-enriched.





directe It is possible to define an order on strategies


\subsection{Arena of order at most 2}
In this section, we consider a restricted class of arena and prove a
property on the games played on these arenas.

The height of the arena is the length of the longest sequence of moves
$m_1 \ldots m_h$ in $M$ such that $m_1 \vdash m_2 \vdash \ldots \vdash m_h$.

The order of an arena $\langle M, \lambda, \vdash \rangle$ is defined to be
$h-2$ where $h$ is the height of the forest of trees $(M, \vdash)$.


\begin{lem}[Pointers are superfluous up to order 2]
Let $A$ be the arena of order at most 2. Let $s$ be a justified sequence of moves in the arena $A$ satisfying
 alternation, visibility and well-bracketing then
the pointers of the sequence $s$ can be reconstructed uniquely.
\end{lem}



\begin{proof}
In the graphic representation of the arena, we display the sub-arena by decreasing order of sub-arena order.
It is safe to do so since in the definition of the forest of tree of an arena, the children nodes
are not ordered.

Let $A$ be an arena of order 2. We assume that $A$ has only one root. The arena $A$ has therefore the following shape:
\begin{center}
\
  \pstree[levelsep=6ex]
    { \TR{$q$} }
    {
\SubTree{$T_1$} \SubTree[linestyle=none]{$\ldots$} \SubTree{$T_n$}
    \TR{$a_1$} \TR{$a_2$} \TR{\ldots} }
\end{center}

where each triangle $T_i$ represents an arena of order 0 or 1.

We will see that the following proof can easily be adapted to take into account the general case of forest arenas (multiple roots).

We write $I_k$, for $k=0$ or $1$, the set of indices $i$ such that the arena $T_i$ has order $k$:
$$I_k = \{ i \in 1.. n\ |\ \order{T_i} = k \}$$

Here is a graphic representation of the arenas $T_i$ for $i \in I_0$ and $T_j$ for $j \in I_1$:
\begin{center}
\
  \pstree[levelsep=6ex]
    {\TR{$q^i$}}
    { \TR{$a_1^i$} \TR{$a_2^i$} \TR{\ldots} }
\hspace{2cm}
  \pstree[levelsep=6ex]
    { \TR{$p^j$} }
    {
      \pstree[levelsep=6ex]
        { \TR{$q^j$} }
        { \TR{$a_1^j$} \TR{$a_2^j$} \TR{\ldots} }
      \TR{$b_1^j$} \TR{$b_2^j$} \TR{\ldots}
    }
\end{center}



For any justified sequence of moves $u$, we write $?(u)$ for the
subsequence of $u$ consisting of the questions in the sequence $u$
that are still pending at the end of the sequence.

Let $L$ be the following language $L = \{\ p^i q^i\ | \ i \in I_1
\}$. We consider the following cases:

\begin{center}
\begin{tabular}{c|c|l|l}
Case & $\lambda_{OP}(m)$ & $?(u) \in$ & condition \\ \hline
0 & O & $\{ \epsilon \}$ \\
A & P & $q$ \\
B & O & $q \cdot L^* \cdot p^i$     & $i \in I_1$ \\
C & P & $q \cdot L^* \cdot p^i q^i$ & $i \in I_1$ \\
D & O & $q \cdot L^* \cdot q^i$      & $i \in I_0$ \\
\end{tabular}
\end{center}

We use the notation $\hat{s}$ to denote a legal and well-bracketed
\emph{justified} sequence of moves and $s$ to denote the same
sequence of moves with pointers removed.

Note that the well-bracketing condition already tells us how to
uniquely recover the pointers for P answer moves: a P-answers points
to the last pending question having the same tag. However for O
answers, we will see that the visibility condition already ensures
the unique recoverability of the pointer and that the
well-bracketing condition is not needed.


We prove by induction on the sequence of moves $u$ that $?(u)$
corresponds to either case 0, A, B, C or D and that the pointers in
$u$ can be recovered uniquely.

\textbf{Base cases:}

If $u$ is the empty sequence $\epsilon$ then there is no pointer to
recover and it corresponds to case 0.

If $u$ is a singleton then it must be the initial question $q$ and
there is not pointer to recover. This corresponds to case A.

\textbf{Step case:}

Consider a legal well-bracketed justified sequence $\hat{s}$ where
$s = u \cdot m$ and $m \in M_A$. The induction hypothesis tells us
that the pointers of $u$ can be recovered (and therefore the P-view
or O-view at that point can be computed) and that $u$ corresponds to
one of the cases 0,A,B,C or D.

We proceed by case analysis on $u$:

\begin{description}

\item[case 0] This case cannot happen because $?(u) = \epsilon$ ($u$ is a complete play) implies that there cannot be any further move $m$.

Indeed the visibility condition implies that $m$ must point to a
P-question in the O-view at that point. But since $u$ is a complete
play, the O-view is $\oview{\hat{u}} = q a$ which does not contain
any P-question. Hence the move $m$ cannot be justified and is not
valid.


\item[case A] $?(u) = q$ and the last move $m$ is played by P.
    There are several cases:
    \begin{itemize}
    \item $m$ is an answer $a_k$ (to the initial question
    $q$) for some $k$, then $m$ points to $q$:

    $\hat{s} = \justseq{ q & \ldots & m \pointto{ll}}$

    and $?(s) = \epsilon$ therefore $s$ correspond to the case 0 (complete play).

    \item $m = q^i$ where $q^i$ is an order 0 question ($i \in I_0$).
    Then $q^i$ points to the initial question $q$ and $s$ falls into category D.

    \item $m = p^i$, a first order question, then $p^i$ points to $q$,

    $?(s)= q p^i$ and it is O's turn after $s$ therefore $s$ falls into category B.

    \end{itemize}


\item[case B] $?(u) \in q \cdot L^* \cdot p^i$ where $i \in I_1$ and O plays the move $m$.

We now analyse the different possible O-moves:
\begin{itemize}
\item Suppose that O gives the (tagged) answer $b^j$ for some $j \in I_1$ then
the visibility condition constraints it to point to a question in
the O-view at that point.

We remark that the last move in $\hat{u}$ must be $p^i$. Indeed,
suppose that there is a move $x \in M_A$ such that $\hat{u} =
\justseq{q & \ldots & p^i\ x \pointto{ll}}$ then by visibility, the
O-move $x$ should points to a move in the O-view a that point. The
O-view is $q p^i$, therefore $x$ can only points to $p^i$. But then,
$p^i$ is not a pending question in $s$ which is a contradiction.


Therefore $\oview{\hat{u}} = \oview{ \justseq{ q & \ldots & p^i
\pointto{ll}} } = q p^i$.

Hence $b^j$ can only point to $p^i$ (and therefore $i=j$).

We then have $?(s) = ?(u \cdot b^i) \in  q \cdot L^*$ which is
covered by case A and C.

\item The only other possible O-move is $q^i$ which, again by the visibility condition, points necessarily
to the previous move $p^i$. We then have $?(s) = ?(u \cdot q^i) \in
q \cdot L^* \cdot p^i q^i$. This falls into category C.

\end{itemize}

\item[case C] $?(u) \in q \cdot L^* \cdot p^i q^i$ where $i \in I_1$ and the move $m$ is played by $P$.

Suppose $m$ is an answer, then the well-bracketing condition imposes
to answer to $q^i$ first. The move $m$ is therefore an integer $a^i$
pointing to $q^i$. We then have $?(s) = ?(u \cdot a^i) \in  q \cdot
L^* \cdot p^i$. This correspond to case B.


Suppose $m$ is a question then there are two cases:
\begin{itemize}
\item $m = q^j$ with $j \in I_0$, the pointer goes to the initial question $q$ and $s$ falls into category D.
\item $m = p^j$ with $j \in I_1$, the pointer goes to the initial question $q$ and $s$ falls into category B.
\end{itemize}

\item[case D] $?(u) \in q \cdot L^* \cdot q^i$ where $i \in I_0$ and the move $m$ is played by $O$.

    The same argument as in case B holds. However there is now another possible move:
    the answer $m = a^i_k$ for some $k$.  This moves can only points to
    $q^i$ (this is the only pending question tagged by $i \in I_0$).

    Then $?(\hat{s}) = ?(\hat{u}\cdot a^i_k) = ?(\justseq{ q & \ldots & q^i \pointto{ll} & \ldots & a^i_k \pointto{ll}}) \in q \cdot L^* $ therefore $s$ falls either into category A or C.

\end{description}

This completes the induction.

How to generalize the proof to arenas that have multiple roots
(forest arenas)? In fact there is no ambiguity since all the moves
are implicitly tagged according to the arena that they belong to.
Therefore in the induction, it suffices to ignore the moves that
belong to another tree (as if they were part of a different game
played in parallel).


\end{proof}


\subsection{Pointer-less strategies}
\label{subsec:ptrless_strat}

Up to order 2, the semantics of PCF terms is entirely defined by
pointer-less strategies. In other words, the pointers can be
uniquely reconstructed from any non justified sequence of moves
satisfying the visibility and well-bracketing condition.

At level 3 however, pointers cannot be omitted in general. Here is an example
taken from \cite{abramsky:game-semantics} illustrating this. Consider the
following two terms of type $((\nat \typar \nat) \typar \nat) \typar
\nat$:

$$M_1 = \lambda f . f (\lambda x . f (\lambda y .y ))$$
$$M_2 = \lambda f . f (\lambda x . f (\lambda y .x ))$$

We assign tags to the types in order to identify in which arena the
questions are asked: $((\nat^1 \typar \nat^2) \typar \nat^3) \typar
\nat^4$. Consider now the following pointer-less sequence of moves
$s = q^4 q^3 q^2 q^3 q^2 q^1$. It is possible to retrieve the
pointers of the first five moves but there is an ambiguity for the
last move: does it point to the first or second occurrence of $q^2$
in the sequence $s$?

Note that the visibility condition does not eliminate the ambiguity,
since the two occurrences of $q^2$ both appear in the P-view at that
point (after recovering the pointers of $s$ up to the second last
move we get:
$$s = \rnode{q4}{q}^4
\rnode{q3}{q}^3
\rnode{q2}{q}^2
\rnode{q3b}{q}^3
\rnode{q2b}{q}^2
\rnode{q1}{q}^1
\bkptrc{q3}{q4}
\bkptrc{q2}{q3}
\bkptrc[ncurv=0.6]{q3b}{q4}
\bkptrc{q2b}{q3b}$$

 therefore the P-view of $s$ is $s$ itself.)

In fact these two different possibilities correspond to two
different strategies. Suppose that the link goes to the first
occurrence of $q^2$ then it means that the proponent is requesting
the value of the variable $x$ bound in the subterm $\lambda x . f (
\lambda y. ... )$. If P needs to know the value of $x$, this is
because P is in fact following the strategy of the subterm $\lambda
y . x$. And the entire play is part of the strategy $\sem{M_2}$.

Similarly, if the link points to the second occurrence of $q^2$ then
the play belongs to the strategy $\sem{M_1}$.

\section{Game model for PCF}
\subsection{Syntax of the PCF language}
PCF is a simply-type $\lambda$-calculus with the following
additions: integer constants  (of ground type), first-order
arithmetic operators, if-then-else branching, and the recursion
combinator $Y_A : (A\rightarrow A)\rightarrow A$ for any type $A$.

The types of PCF are given by the following grammar:
$$ T ::= \texttt{exp}\ |\ T \rightarrow T$$

The following grammar gives the structure of terms:
\begin{eqnarray*}
 M ::= x\ |\ \lambda x :A . M \ |\ M M \ |\ \\
\ |\ n \ |\ \texttt{succ } M \ |\  \texttt{pred } M \\
\ |\ \texttt{cond } M M M \ |\ \texttt{Y}_A\ M
\end{eqnarray*}

where $x$ ranges over a set of countably many variables and $n$
ranges over the set of natural numbers.

Terms are generated according to the formation rules given in table
\ref{tab:pcf_formrules} where the judgement is of the form $ \Gamma  \vdash M : A$.

\begin{table}[htbp]
$$ (var) \rulef{}{x_1:A_1, x_2:A_2, \ldots x_n : A_n  \vdash x_i : A_i}\ i \in 1..n$$
$$ (app) \rulef{\Gamma \vdash M : A\rightarrow B \qquad \Gamma \vdash N:A}{\Gamma \vdash M\ N : B}
\qquad (abs) \rulef{\Gamma, x:A \vdash M : B}{\Gamma \vdash \lambda x :A . M : A\rightarrow B}$$

$$ (const) \rulef{}{\Gamma \vdash n :\texttt{exp}}
\qquad (succ) \rulef{\Gamma \vdash M:\texttt{exp} }{\Gamma \vdash \texttt{succ}\ M:\texttt{exp}}
\qquad (pred) \rulef{\Gamma \vdash M:\texttt{exp} }{\Gamma \vdash \texttt{pred}\ M:\texttt{exp}}$$

$$
(cond) \rulef{\Gamma \vdash M : exp \qquad \Gamma \vdash N_1 : exp \qquad \Gamma \vdash N_2 : exp }{\Gamma \vdash \texttt{cond}\ M\ N_1\ N_2}
\qquad  (rec) \rulef{\Gamma \vdash M : A\rightarrow A }{ \Gamma \vdash Y_A M : A}$$

\caption{Formation rules for PCF terms}
\label{tab:pcf_formrules}
\end{table}

\subsection{Operational semantics of PCF}

We give the big-step operational semantics of PCF. The notation $M \eval V$ means
that the closed term $M$ evaluates to the canonical form $V$. The canonical forms are given by the following
grammar:
$$V ::= n\ |\ \lambda x. M$$
In other word, a canonical form is either a number or a function.

The operational semantics is given for closed terms therefore the context $\Gamma$ is not present in
the evaluation rules.

The full operational semantics is given in table \ref{tab:bigstep_pcf}.

\begin{table}[htbp]
$$\rulef{}{V \eval V} \quad \mbox{ provided that $V$ is in canonical form.} $$

$$ \rulef{M \eval \lambda x. M' \quad M'\subst{x}{N}}{M N \eval V}$$

$$\rulef{M \eval n}{\texttt{succ}\ M \eval n+1}
\qquad \rulef{M \eval n+1}{\texttt{pred}\ M \eval n}
\qquad \rulef{M \eval 0}{\texttt{pred}\ M \eval 0}$$

$$\rulef{M \eval 0 \quad N_1 \eval V}{\texttt{cond}\ M N_1 N_2  \eval V}
\qquad
 \rulef{M \eval n+1 \quad N_2 \eval V}{\texttt{cond}\ M N_1 N_2  \eval V}$$

$$\rulef{M (\mathrm{Y} M) \eval V }{\texttt{Y} M \eval V}$$
\label{tab:bigstep_pcf}
\caption{Big-step operational semantics of PCF}
\end{table}



\section{Idealized Algol (IA)}
\label{sec:ia}

\subsection{The syntax of IA}
IA is an extension of PCF introduced by J.C. Reynold in
\cite{Reynolds81}. It adds imperative features such as local variables and sequential composition.

The description of the language that we give here follows the one of \cite{abramsky:game-semantics}.

On top of \texttt{exp}, PCF has the following two new types:
 \texttt{com} for commands and \texttt{var} for variables.

There is a constant \texttt{skip} of type \texttt{com} which corresponds to the command that do
nothing. Commands can be composed using the sequential composition operator \texttt{seq}.
Local variable are declared using the \texttt{new} operator, variable content is written
using \texttt{assign} and retrieved using \texttt{deref}.

The new formations rules are given in table \ref{tab:ia_formrules}.

\begin{table}[htbp]
$$ \rulef{\Gamma \vdash M : \texttt{com} \quad \Gamma \vdash N :A}
    {\Gamma \vdash \texttt{seq}_A \ M\ N\ : A} \quad A \in \{ \texttt{com}, \texttt{exp}\}$$

$$ \rulef{\Gamma \vdash M : \texttt{var} \quad \Gamma \vdash N : \texttt{exp}}
    {\Gamma \vdash \texttt{assign}\ M\ N\ : \texttt{com}}
\qquad
 \rulef{\Gamma \vdash M : \texttt{var}}
    {\Gamma \vdash \texttt{deref}\ M\ : \texttt{exp}}$$

$$ \rulef{\Gamma, x : \texttt{var} \vdash M : A}
    {\Gamma \vdash \texttt{new } x \texttt{ in } M} \quad A \in \{ \texttt{com}, \texttt{exp}\}$$

$$ \rulef{\Gamma \vdash M_1 : \texttt{exp} \rightarrow \texttt{com} \quad \Gamma \vdash M_2 : \texttt{exp}}
    {\Gamma \vdash \texttt{mkvar } M_1\ M_2\ : \texttt{var}}$$

\caption{Formation rules for IA terms}
\label{tab:ia_formrules}
\end{table}

If $\vdash M : A$ (i.e. $M$ can be formed with an empty context), we say that $M$ is a close term.

\subsection{Operational semantics}

In IA the semantics is given in a slightly different form from PCF.
In PCF, the evaluation rules were given for closed terms only. Suppose that we
proceed the same way for IA and consider the evaluation rule for the $\texttt{new}$ construct:
the conclusion is $\texttt{new } x:=0 \texttt{ in } M$ and the premise
is an evaluation for a certain term constructed from $M$, more precisely the term $M$
where \emph{some} occurrences of $x$ are replaced by the value $0$.
Because of the presence of the \texttt{assign} operator, we cannot simply replace all
the occurrences of $x$ in $M$ (the required substitution is  more complicated
than the substitution used for beta-reduction).


Therefore, instead of giving the semantics for closed term we consider terms
whose free variables are all of type \texttt{var}. These free variables are ``closed'' by mean of
stores. A store is a function mapping free variables of type \texttt{var} to natural numbers.
Suppose $\Gamma$ is a context containing only variable of type \texttt{var}, then we say that
$\Gamma$ is a \texttt{var}-context. A store whose domain $\Gamma$ is called a $\Gamma$-store.

The notation $s\ |\ x \mapsto n$ refers to the store that maps $x$ to $n$
and otherwise maps variables according to the store $s$.


The canonical forms for IA are given by the grammar:
$$ V ::= n\ |\ \lambda x. M\ |\ x\ |\  \texttt{mkvar} M N$$

where $n \in \nat$ and $x:var$.


A program is now defined by a term together with a $\Gamma$-store such that $\Gamma \vdash M : A$.
The evaluation semantics is expressed by the judgment form
$$s,M \eval s', V$$
where $s$ and $s'$ are $\Gamma$-stores,
$\Gamma \vdash M : A$ and $\Gamma \vdash V : A$ where $V$ is in canonical form.

The operational semantics for IA is given by the rule of PCF (table \ref{tab:bigstep_pcf})
together with the rules of table \ref{tab:bigstep_ia} where the following abbreviation is used:
$$ \rulef{M_1 \eval V_1 \quad M_2 \eval V_2}{M \eval V} \qquad \mbox{for} \qquad
  \rulef{s,M_1 \eval s',V_1 \quad s', M_2 \eval s'',V_2 }{s,M \eval s'',V}
$$


\begin{table}[htbp]
$$\mbox{\textbf{Sequencing }}
    \rulef{M \eval \iaskip \quad N \eval V}{\texttt{seq } M\ N \eval V}
$$

$$\mbox{\textbf{Variables }}
    \rulef{s,N \eval s',n \quad s',M \eval s'',x}{s, \assign\ M\ N \eval (s''\ |\ x \mapsto n),\iaskip}
\qquad
    \rulef{s,M \eval s',x }{s, \deref\ M \eval s',s'(x)}$$

$$\mbox{\texttt{\textbf{mkvar}}}
    \rulef{N \eval n \quad M \eval \texttt{mkvar } M_1\ M_2 \quad M_1\ n \eval \iaskip}
    {\assign\ M\ N \eval \iaskip}
\qquad
    \rulef{N \eval \texttt{mkvar } M_1\ M_2 \quad M_2\ \eval n}
    {\deref\ M \eval n}
$$

$$\mbox{\textbf{Block}}
    \rulef{(s\ |\ x \mapsto 0),M \eval (s'\ |\ x \mapsto n),V }
    {s, \texttt{new } x \texttt{ in } M \eval s',V}
$$

\label{tab:bigstep_ia}
\caption{Big-step operational semantics of IA}
\end{table}

\subsection{Game semantics}

As we have seen in section \ref{sec:catgames}, games and strategies
form a cartesian closed category, therefore games can model the simply-typed $\lambda$-calculus. Let us first
explain how this is achieved before extending the model to PCF and IA.

\subsubsection{Simply typed $\lambda$-calculus}

In the cartesian closed category $\mathcal{C}$, the objects are the arenas and the morphisms are the strategies.

In the games that we describe here, the Opponent represents the environment while
the Proponent plays according to a strategy imposed by the program itself.


Given a simple type $A$, we will model it as an arena $\sem{A}$.
A context $\Gamma = x_1 :A_1, \ldots x_n:A_n$ will be mapped to the arena
$\sem{\Gamma} = \sem{A_1} \times \ldots \times \sem{A_n}$ and a term $\Gamma \vdash M : A$
will be modeled by a strategy on the arena $\sem{\Gamma} \rightarrow \sem{A}$.
Since $\mathcal{C}$ is cartesian closed, there is is a terminal object $\textbf{1}$ (the empty arena) that
models the empty context ($\sem{\Gamma} = \textbf{1}$).


The base type \texttt{exp} is interpreted by the following flat arena of natural numbers noted $\nat$:
$$  \pstree[levelsep=6ex]
    {\TR[name=R]{q}}
    { \TR{1} \TR{2} \TR{\ldots}
    }
$$
In this arena, there is only one question: the initial O-question, P can then answer it by playing a natural number $i \in \nat$.
There are only two kinds strategy on this arena:
\begin{itemize}
\item the empty strategy where P never answer the initial question. This corresponds to a non terminating computation;
\item the strategies where P answers by playing a number $n$. This models the constants of the language.
\end{itemize}

Given the interpretation of base types, we define the interpretation of $A\rightarrow B$ by induction:
$$\sem{A \rightarrow B} = \sem{A} \Rightarrow \sem{B}$$

where the operator $\Rightarrow$ denotes the arena construction $!A
\multimap B$ which exists because $\mathcal{C}$ is cartesian closed.

Graphically if we represent the arena $A$ and $B$ by two triangles, the arena for $A \rightarrow B$ would be represented by:
\begin{center}
\psset{xunit=.5pt,yunit=.5pt,runit=.5pt}
\begin{pspicture}(150,80)
\rput[tr](150,80){ \pnode(27,40){a} \pstribox{A} }
\rput[bl](0,0){ \pnode(27,40){b} \pstribox{B} }
\ncline{->}{a}{b}
\end{pspicture}
\end{center}


Variables are interpreted by projection:
$$\sem{x_1 : A_1, \ldots, x_n:A_n \vdash x_i : A_i} = \pi_i : \sem{A_i} \times \ldots \times \sem{A_i} \times \ldots \times \sem{A_n} \rightarrow  \sem{A_i}$$

The abstraction $\Gamma \vdash \lambda x :A.M : A \rightarrow B$ is modeled by a strategy on the arena
$\sem{\Gamma} \rightarrow (\sem{A}\Rightarrow\sem{B})$. This strategy is obtain by using the currying operator of the
cartesian closed category:
$$\sem{\Gamma \vdash \lambda x :A.M : A \rightarrow B} = \Lambda( \sem{\Gamma, x :A \vdash M : B})$$

The application $\Gamma \vdash M N$ is modeled using the evaluation map $ev_{A,B} : (A\Rightarrow B)\times A \rightarrow B$:

$$\sem{\Gamma \vdash M N} = \langle \sem{\Gamma \vdash M, \Gamma \vdash N} \rangle; ev_{A,B}$$


\subsubsection{PCF}

We now show how to model the PCF constructs in the game semantics setting.
In the following, the sub-arena of a game are tagged in order to distinguish identical arenas that are present in different components of the game.
Moves are also tagged in the exponent in order to identify the sub-arena in which moves are played. We will omit the pointers in the play
when they are not essential for the understanding of the model (moreover we will see later on that under certain assumptions
up to order 2, pointers can be recovered uniquely).

The successor arithmetic operator is modeled by the following strategy on the arena $\nat^1 \Rightarrow \nat^0$:
$$\sem{\texttt{succ}} = \{q^0 \cdot q^1 \cdot n^1 \cdot (n+1)^0\ |\ n \in \nat \}$$

The predecessor arithmetic operator is denoted by the strategy
$$\sem{\texttt{pred}} = \{q^0 \cdot q^1 \cdot n^1 \cdot (n-1)^0\ |\ n >0 \} \union \{ q^0 \cdot q^1 \cdot 0^1 \cdot 0^0 \} $$

Then given a term $\Gamma \vdash \texttt{succ} M : \texttt{exp}$ we define:
$$\sem{\Gamma \vdash \texttt{succ } M : \texttt{exp}} = \sem{\Gamma \vdash M} ; \sem{\texttt{succ}} $$
$$\sem{\Gamma \vdash \texttt{pred } M : \texttt{exp}} = \sem{\Gamma \vdash M} ; \sem{\texttt{pred}} $$


The conditional operator is denoted by the following strategy on the arena $\nat^3 \times \nat^2 \times \nat ^1 \Rightarrow \nat^0$:
$$\sem{\texttt{cond}} =
    \{ q^0 \cdot q^3 \cdot 0 \cdot q^2 \cdot n^2 \cdot n^0 \ | \ n \in \nat \}
    \union
    \{ q^0 \cdot q^3 \cdot m \cdot q^2 \cdot n^2 \cdot n^0 \ | \ m >0, n \in \nat \}
    $$


Given a term $\Gamma \vdash \texttt{cond} M\ N_1\ N_2$ we define:
$$\sem{\Gamma \vdash \texttt{cond} M\ N_1\ N_2} =
\langle \sem{\Gamma \vdash M}, \sem{\Gamma \vdash N_1}, \sem{\Gamma \vdash N_2} \rangle ; \sem{\texttt{cond}}$$


The interpretation of the \texttt{Y} combinator is a bit more complicated.

Consider the term $\Gamma \vdash M : A \rightarrow A$, its semantics $f$ is a strategy on $\sem{\Gamma} \times \sem{A} \rightarrow \sem{A}$.
We define the chain $g_n$ of strategies on the arena $\sem{\Gamma} \rightarrow \sem{A}$ as follows:
\begin{eqnarray*}
g_0 &=& \perp \\
g_{n+1} &=&  F(g_n) = \langle id_{\sem{\Gamma}}, g_n\rangle ; f
\end{eqnarray*}

where $\perp$ denotes the empty strategy $\{ \epsilon \}$.

It is easy to see that indeed the $g_n$ forms a chain.
We define $\sem{\texttt{Y } M}$ to be the least upper bound of the chain $g_n$
(i.e. the  least fixed point of $F$). Its existence is guaranteed by the fact that
the category of games is cpo-enriched.

\subsubsection{IA}

It is easy to check that all the strategies given until now are well-bracketed and innocent.
From now on, we will only require well-bracketing and we will introduce strategies that are
not innocent. This is a necessity if we want to give a model of memory cells that correspond
to variables. The intuition behind this fact is that a cell needs to remember what was the last value written in it
in order to be able to return it when it is read, and this can only be done by looking at the whole history of moves,
not only those present in the P-view.





\subsection{Full-abstraction}
In this section we recall the standard full abstraction result proved in  \cite{abramsky94full}
and \cite{hylandong_pcf}.

A context noted $C[-]$ is a term containing a hole denoted by $-$. If $C[-]$ is a context then $C[A]$ denotes the term obtained
after replacing the hole by the term $A$.

\begin{dfn}[Observational preorder]
Let $\vdash M : A$ and $\vdash N : A$ be two closed terms. We define the relation $\sqsubseteq$ as follows:


$M \sqsubseteq N$ if and only if for all context $C[-]$ such that $C[M]$ and $C[M]$ are well-formed terms if
$C[M] \eval$ then $C[N] \eval$.
\end{dfn}


\begin{lem}[Soundness for PCF terms] Let $M$ be a PCF term.
If $M \eval V$ then $\sem{M} = \sem{V}$.
\end{lem}

\begin{lem}[Soundness for IA terms] Let $\Gamma \vdash M : A$ be an IA term and a $\Gamma$ store $s$.
If $s,M \eval s',V$ then the plays of $\sem{s,M} : I \multimap A \otimes !\Gamma$ which begin
with a move of $A$ are identical to those of $\sem{s',V}$.
\end{lem}


\begin{lem}[Computational adequacy for PCF terms]
All PCF terms are computable. (i.e. $\sem{M} \neq \perp$ implies $M \eval$)
\end{lem}

\begin{lem}[Computational adequacy for IA terms]
All IA terms are computable. (i.e. $\sem{M} \neq \perp$ implies $M \eval$)
\end{lem}


The following result follows from soundness and computational adequacy of the model.
\begin{prop}[Inequational soundness]
\label{prop:ineqsoundness}
Let $M$ and $N$ be two closed terms then
$$\sem{M} \subseteq \sem{N} \implies  M \sqsubseteq N $$
\end{prop}

\begin{prop}[Definability]
\label{prop:definability}
Let $\sigma$ be a compact well-bracketed on a game $A$ denoting a IA type. Then there is
an IA-term $M$ such that $\sem{M} = \sigma$.
\end{prop}

The final standard result of game semantics can then be proved using proposition \ref{prop:ineqsoundness} and \ref{prop:definability}:
\begin{thm}[Full abstraction]
Let $M$ and $N$ be two closed IA-terms.
$$\sem{M} \precsim_b \sem{N} \ \iff \ M \sqsubseteq N$$
\end{thm}

where $\precsim_b$ denotes the intrinsic preorder of the category $\mathcal{C}_b$.


\subsection{Call-by-Value first-order Idealized Algol}

Game semantics for call-by-value programming Language.


% second chapter
\chapter{Safe $\lambda$-Calculus}
In \cite{KNU02}, the authors introduced a restriction on
higher-order grammars called \emph{safety} in order to study the
infinite hierarchy of trees recognized by a higher-order pushdown
automaton. They proved that trees recognized by pushdown automata of
level $n$ coincide with trees generated by safe higher-order
grammars of level $n$. This characterisation permitted them to prove
the decidability of the monadic second-order theory of infinite
trees recognized by a higher-order pushdown automaton of any level.

Safety has also appeared in a different form in \cite{Dam82} under
the name \emph{restriction of derived types}. The forthcoming thesis
of Jolie de Miranda \citep{demirandathesis} contains a comparison of
safety and the restriction of derived types.

More recently, Ong proved in \cite{OngLics2006} that the safety
assumption of \cite{KNU02} is in fact not necessary. More precisely,
the paper shows that the MSO theory of trees generated by order-$n$
recursion schemes is $n$-EXPTIME complete.

For this particular problem, \emph{safety} happens to be an
artificial restriction. However when the \emph{safety} condition is
transposed to the simply-typed $\lambda$-calculus, it gives some
interesting properties. In particular, for safe terms, it becomes
unnecessary to rename variables when performing substitution.

This chapter starts with a presentation of the original version of
the safe $\lambda$-calculus where types are required to satisfy a
condition called homogeneity. We then give a more general definition
which does not require type homogeneity.

\section{Homogeneous safe $\lambda$-calculus}
\label{sec:safe_homog}

\subsection{Type homogeneity}
Let $Types$ be the set of simple types generated by the grammar $A
\, ::= \, o \; | \; A \funsp A$. Any type different from the base
type $o$ can be written $(A_1, \cdots, A_n, o)$ for some $n \geq 1$,
which is a shorthand for $A_1 \funsp \cdots \funsp A_n \funsp o$ (by
convention, $\rightarrow$ associates to the right). If $T=(A_1,
\cdots, A_n, o)$ then the arity of $T$, written $arity(T)$, is
defined to be $n$.

Suppose that a ranking function ${\sf rank} :
Types \funto (L, \leq)$ is given where $(L, \leq)$ is any linearly ordered
set. Possible candidates for the ranking function are:
\begin{itemize}
\item ${\sf ord} : Types \funto (\nat,\leq)$ with $\ord{o} = 0$
and $\ord{A \funsp B} = \max(\ord{A}+1, \ord{B})$;
\item ${\sf height} : Types \funto (\nat,\leq)$ with
$\slheight{A \funsp B} = 1 + \max(\slheight{A}, \slheight{B})$ and
$\slheight{o} = 0$ ;
\item ${\sf nparam} : Types \funto (\nat,\leq)$ with $\nparam{o} = 0$
and $\nparam{A_1, \cdots, A_n} = n$;
\item ${\sf ordernp} : Types \funto (\nat \times \nat,\leq)$ with $ {\sf ordernp} (t)  = \langle \order{t}, \nparam{t} \rangle$ for $t \in Types$.
\end{itemize}
Following \cite{KNU02}, we say that a type is {\sf rank}-homogeneous
if it is $o$ or if it is $(A_1, \cdots, A_n, o)$ with the condition
that $\rank{A_1} \geq \rank{A_2}\geq \cdots \geq \rank{A_n}$ and
each $A_1$, \ldots, $A_n$ is {\sf rank}-homogeneous.



Suppose that $\overline{A_1}$, $\overline{A_2}$, \ldots,
$\overline{A_n}$ are $n$ lists of types, where $A_{ij}$ denotes the
$j$th type of list $\overline{A_i}$ and $l_i$ the size of
$\overline{A_i}$, then the notation $A \; = \; (\overline{A_1} \, |
\, \cdots \, | \, \overline{A_r} \, | \, o)$ means that
\begin{itemize}
  \item $A$ is the type $(A_{11},A_{12},\cdots, A_{1l_1}, A_{21}, \cdots,A_{2l_2}, \cdots A_{n1},\cdots, A_{nl_n},o)$
  \item $\forall i: \forall u,v \in A_i : \rank u = \rank v $
  \item $\forall i,j . \forall u \in A_i . \forall v \in A_j . i<j \implies \rank u >
   \rank v $
\end{itemize}
and therefore $A$ is {\sf rank}-homogenous. This notation organises
the $A_{ij}$s into partitions according to their ranks. Suppose $B =
(\overline{B_1} \, | \, \cdots \, | \, \overline{B_m} \, | \, o)$,
we write $(\overline{A_1} \, | \, \cdots \, | \, \overline{A_n} \, |
\, {B})$ to mean
\[(\overline{A_1} \, | \, \cdots \, | \, \overline{A_n} \, | \,
\overline{B_1} \, | \, \cdots \, | \, \overline{B_m} \, | \, o).\]

From now on, we only consider the rank function {\sf ord}. We will
use the term ``homogeneous'' to refer to {\sf ord}-homogeneity.


\subsection{Safe Higher-order recursion scheme}
We now present the original notion of safety introduced in
\cite{KNU02} as a restriction on higher-order recursion schemes. We
introduce briefly the notion of higher-order recursion scheme. The
reader is refered to \cite{KNU02,demirandathesis,safety-mirlong2004}
for more details.

Suppose that $\Gamma$ is a set of typed symbol then the set of
\emph{applicative terms} written $\mathcal{T}(\Gamma)$ is the
closure of $\Gamma$ under the application rule i.e. if $s:
A\rightarrow B$ and $t:A$ are in $\mathcal{T}(\Gamma)$ then so is
$st :B$.

A higher-order recursion scheme is a deterministic grammar that can be used to define potentially infinite term trees:
\begin{dfn}[Higher-order recursion scheme]
A \emph{deterministic higher-order grammar} or \emph{higher-order recursion scheme} is a tuple $\mathcal{G} =
\langle \Sigma, \mathcal{N}, V, \mathcal{R}, S \rangle$, where
\begin{itemize}
\item $\Sigma$ is a ranked alphabet of terminals of order at most 1,
\item $V$ is a finite set of typed variables,
\item $\mathcal{N}$ is a finite set of homogeneously-typed non-terminals,
\item $S$ a distinguished symbol of $\mathcal{N}$ of ground type, called the start symbol,
\item $\mathcal{R}$ is a finite set of production rules, one for each $F : (A_1, \ldots, A_n, o) \in N$, of the form
    $$ F z_1 \ldots z_m \rightarrow e$$
where $z_i$ is a variable of type $A_i$ and $e$ is an applicative
term of type $o$ in $\mathcal{N}(\Sigma \union \mathcal{N} \union
\{z_1 \ldots z_m \} )$. The $z_i$s are called the \emph{parameters}
of the rule.
\end{itemize}
\end{dfn}
The order of a rewrite rule is the order of the non-terminal symbol
appearing on the left hand side of the rule. The order of a grammar
is the highest order of its non-terminals.

Safety is a syntactic restriction on higher-order grammars. It can be formulated as
follows:
\begin{dfn}[Safe Higher-order grammar]
  Let $G$ be a higher-order grammar $G$ of order $n$
    whose non-terminals are of homogeneous type.
    $G$ is \emph{unsafe} if and only if there is a rewrite rule $F z_1 \ldots z_m \rightarrow e$ where
   $e$ contains a subterm $t$ such that:
  \begin{enumerate}
    \item $t$ occurs in an operand position in $e$,
    \item $t$ is of order $k>0$,
    \item $t$ contains a parameter of order strictly less than $k$.
  \end{enumerate}
  $G$ is \emph{safe} if it is not unsafe.
\end{dfn}

Let us give an example taken from \cite{KNU02}:
\begin{exmp} Let $f:(o,o,o)$, $g,h:(o,o)$ and $a,b:o$ be $\Sigma$ constants.
 The grammar of level 3 with non-terminals $S:o$ and $F: ((o,o),o,o,o)$ and production rules:
\begin{eqnarray*}
    S &\rightarrow&  F g a b \\
    F \varphi x y &\rightarrow& f ( \mathcal{F} ( F \varphi x ) y (h y)) (f (\varphi x) y)
\end{eqnarray*}
is not safe because the term $\mathcal{F} \varphi x : (o,o)$ containing a variable of order $0$
occurs at an operand position in the right-hand side expression of the second rule.

On the other hand, the grammar with the following production rules is safe:
\begin{eqnarray*}
    S &\rightarrow&  F g a b \\
    F \varphi x y &\rightarrow& f ( \mathcal{F} ( F \varphi x ) y (h y)) (f (\varphi x) y)
\end{eqnarray*}
Moreover it can be shown that these two grammars are equivalent in the sense that they generate the same
infinite tree.
\end{exmp}


\subsection{Rules of the Safe $\lambda$-Calculus}

There is a correspondence between higher-order recursion schemes and
the simply-typed $\lambda$-calculus. The non-terminals of a
recursion scheme can be interpreted as $\lambda$-abstractions in the
simply-typed $\lambda$-calculus. The $\Sigma$-constants are
interpreted as ``constructors'' constants (in the sense of
constructor used in functional programming languages to represent
abstract data-types such as trees). The notions of variable and
application are directly transposed to the equivalent notions in the
simply-typed $\lambda$-calculus. Using this analogy it is possible
to derive a version of the safety restriction for the
$\lambda$-calculus.

The Safe $\lambda$-calculus has been first proposed in \cite{DBLP:conf/fossacs/AehligMO05},
a corrected definition appeared in \cite{Ong2005}. The definition that we give here is
slightly more general in the sense that we allow the use of
$\Sigma$-constants of any higher-order type whereas the original definition only allows first order constants.


The \textbf{Safe $\lambda$-Calculus} is a sub-system of the
simply-typed $\lambda$-calculus. Typing judgements (or
terms-in-context) are of the form:
\begin{equation}
\nonumber \seq{\overline{x_1}:\overline{A_1} \, | \, \cdots \, | \,
\overline{x_n} :  \overline{A_n}}{M : B}
\end{equation}
which is shorthand for $\seq{x_{11} : A_{11}, \cdots, x_{1r}:
A_{1r}, A_{21},\ldots }{M : B}$ such that the context variables are listed in decreasing type order and
 with the condition that $\ord{x_{ik}} < \ord{x_{jl}}$ for any $k, l$ and $i<j$.

\emph{Valid typing judgements} of the system are defined by
induction over the following rules, where $\Delta$ is a given
homogeneously-typed alphabet and $\Sigma$ is a set of
homogeneously-typed constants:

$$ \rulename{wk}
    {   \rulef{ \seq{\Gamma}{M:B} \qquad {\Gamma \subset \Delta} }
             { \seq{\Delta }{M : B}}
   }
\qquad
    \rulename{perm}
    {
      \rulef { \seq{\Gamma}{M:B} \qquad \sigma(\Gamma) \hbox{ homogeneous} }
            { \seq{\sigma(\Gamma)}{M : B} }
    }
$$

$$ \rulename{\Sigma\mbox{\textbf{-const}}}  \rulef{}{\seq{}{b : A}}\ b:A \in \Sigma
\qquad
 \rulename{var} \rulef{}{\seq{x_{ij} : A_{ij}\, }{x_{ij} : A_{ij}}}
$$

$$ \rulename{\lambda\mbox{\textbf{-abs}}}
\rulef{ {\seq{\overline{x_1} : \overline{A_1}\, | \, \cdots\, | \,
\overline{x_{n+1}} : \overline{A_{n+1}}}{M : B}} \qquad
\ord{\overline{A_{n+1}}} \geq \ord{B} -1}
    {\seq{\overline{x_1} :
\overline{A_1}\, | \, \cdots\, | \, \overline{x_{n}} :
\overline{A_{n}}}{\lterm{\overline{x_{n+1}} : \overline{A_{n+1}}}{M}
: (\overline{A_{n+1}} \, | \, B)}} $$

$$ \rulename{app} \rulef{{\seq{\Gamma}{M : (\overline{B_1} \, | \, \cdots \, | \, \overline{B_m} \, | \, o)} \qquad
\seq{\Gamma}{N_1 : B_{11}} \quad \cdots \quad \seq{\Gamma}{N_{l} :
B_{1l}} \qquad l = |\overline{B_1}| }}
    { \seq{\Gamma}{M N_1
\cdots N_{l} : (\overline{B_2} \, | \, \cdots \, | \,
\overline{B_m} \, | \, o)}} $$

$$ \rulename{app^+} \rulef
    {\seq{\Gamma}{M : (\overline{B_1} \, | \, \cdots \, | \, \overline{B_m} \, | \, o)} \qquad
    \seq{\Gamma}{N_1 : B_{11}} \quad \cdots \quad \seq{\Gamma}{N_{l} :
    B_{1l}} \qquad l < |\overline{B_1}| }
    { \seq{\Gamma}{M N_1
    \cdots N_{l} : (\overline{B} \, | \, \overline{B_2} \, | \ \cdots \, | \,
    \overline{B_m} \, | \, o)}} $$

where $\overline{B_1} = B_{11}, \ldots, B_{1l},\overline{B}$ with
the condition that every variable in $\Gamma$ has an order strictly greater
than $\ord{\overline{B_1}}$.






\begin{property}[Basic properties]
\label{proper:safe_basic_prop} Suppose $\Gamma \vdash M : B$ is a
valid judgment then
\begin{itemize}
\item[(i)] $B$ is homogeneous;
\item[(ii)] every free variable of $M$ has order at least $\ord{M}$;
\item[(iii)] $fv(M) \vdash M : B$,
\end{itemize}
where $fv(M) \subseteq \Gamma$ denotes the context constituted of the variables
in $\Gamma$ occurring free in $M$.
\end{property}
\begin{proof}
(i) and (ii) are proved by an easy structural induction. (iii) is
due to the fact that the weakening rule is the only rule which can
introduce a variable not occurring freely in $M$ in the context of a
typing judgement.
\end{proof}

We now define a special kind of substitution that performs
simultaneous substitution and permits variable capture i.e. that
does not rename variables when the substitution is performed on an
abstraction.

\begin{dfn}[Capture-permitting simultaneous substitution (for homogeneous safe terms)]
\label{dnf:safe_simsubst} We use the notation
$\subst{\overline{N}}{\overline{x}}$ for $\subst{N_1 \ldots N_n}{x_1
\ldots x_n}$ and $\overline{y}:\overline{A}$ for $y_1:A_1, \ldots,
y_p:A_p$. A safe term has necessarily one of the forms occurring on
the left-hand side of the following equations, where $M$, $N_1,
\ldots N_l$ are safe terms. The capture-permitting simultaneous
substitution is then defined by:
\begin{eqnarray*}
c \subst{\overline{N}}{\overline{x}} &=& c \quad \mbox{ where $c$ is a $\Sigma$-constant}\\
x_i \subst{\overline{N}}{\overline{x}} &=& N_i\\
 y \subst{\overline{N}}{\overline{x}} &=& y \quad \mbox{ if } y \not \neq x_i \mbox{ for all } i,\\
(M N_1 \ldots N_l) \subst{\overline{N}}{\overline{x}} &=& (M \subst{\overline{N}}{\overline{x}}) (N_1 \subst{\overline{N}}{\overline{x}}) \ldots  (N_l \subst{\overline{N}}{\overline{x}})\\
(\lambda \overline{y} : \overline{A}. M)
\subst{\overline{N}}{\overline{x}} &=& \lambda \overline{y} . M
\subst{\overline{N} \upharpoonright I}{\overline{x} \upharpoonright I} \\
&& \mbox{where } I  = \{ i \in 1..n \ | \ x_i \not \in \overline{y} \}
\end{eqnarray*}

where $ \upharpoonright$ is the index filtering operator: if $s$ is
a sequence and $I$ a set of indices then $s \upharpoonright I$ is
the subsequence of $s$ obtained by keeping only the element in $s$
at positions in $I$.
\end{dfn}

This substitution is well-defined for safe terms in the sense that safety is preserved by substitution:

\begin{lem}[Capture-permitting simultaneous substitution preserves safety]
\label{lem:subst_preserve_safety} Let $\Gamma \union \overline{x}
\vdash M$ be a safe term where $\overline{x}$ denotes a list of
variables (which do not necessarily belong to the same partition).

For any safe terms $\Gamma \vdash N_1, \cdots, \Gamma \vdash N_n$,
the capture-permitting simultaneous substitution $M[N_1 / x_1 ,
\cdots, N_n / x_n]$ is safe. In other words, the following judgment
is valid:
$$ \Gamma \vdash M[N_1 / x_1 , \cdots, N_n / x_n] $$
\end{lem}
\begin{proof}
An easy proof by an induction on the structure of the safe term.
\end{proof}



With the traditional substitution, it is necessary to rename
variables when performing substitution on an abstraction in order to
avoid possible variable capture. As a consequence, in order to
implement substitution one needs to have access to an unbound number
of variable names. An interesting property of the homogeneous safe
$\lambda$-calculus is that variable capture never occurs when
performing substitution. In other words, the traditional
substitution can be safely replaced by the capture-permitting
substitution:

\begin{lem}[No variable capture lemma]
\label{lem:homog_nocapture} In the safe $\lambda$-calculus, there is
no variable capture when performing the following capture-permitting
simultaneous substitution:
$$ M[N_1 / x_1 , \cdots, N_n / x_n] $$
provided that $\Gamma \union \overline{x} \vdash M$, $\Gamma \vdash  N_1, \cdots ,\Gamma \vdash  N_n$ are valid judgments.
\end{lem}

\begin{proof}
We prove the result by induction. The variable, constant and
application cases are trivial. For the abstraction case, suppose $M
= \lambda \overline{y} : \overline{A}. P$ where $\overline{y} = y_1
\ldots y_p$. The capture-permitting simultaneous substitution gives:
$$M \subst{\overline{N}}{\overline{x}} = \lambda \overline{y} . P
\subst{\overline{N} \upharpoonright I}{\overline{x} \upharpoonright
I} \mbox{ where } I  = \{ i \in 1..n \ | \ x_i \not \in \overline{y}
\}. $$


By the induction hypothesis there is no variable capture in $P
\subst{\overline{N} \upharpoonright I}{\overline{x} \upharpoonright
I}$. Hence variable capture can only happen when the variable $y_j$
occurs freely in $N_i$ and $x_i$ occurs freely in $P$ for some $i
\in I$ and $j \in 1..p$. In that case, property
\ref{proper:safe_basic_prop} (ii) gives:
$$ \ord{y_j} \geq \ord{N_i} = \ord{x_i}$$

Moreover $i\in I$ therefore $x_i \not \in \overline{y}$ and since $x_i$ occurs freely in $P$, $x_i$ must also occur freely in the safe term
$\lambda \overline{y}. P$. Thus, property \ref{proper:safe_basic_prop} (ii) gives:
$$ \ord{x_i} \geq \ord{\lambda y_1 \ldots y_p . T} \geq 1+ \ord{y_j} > \ord{y_j}$$

which, together with the previous equation, gives a contradiction.
\end{proof}




\subsection{Safe $\beta$-reduction}

We now introduce the notion of safe $\beta$-redex and show how to
reduce them using the capture-permitting simultaneous substitution.
We will then show that a safe $\beta$-reduction reduces to a safe
term.


In the simply-typed lambda calculus a redex is a term of the form $(\lambda x . M) N$.
We generalize this notion to the safe lambda calculus. We call multi-redex a term of the form
$(\lambda x_1 \ldots x_n . M) N_1 \ldots N_l$ (it is not required to have $n=l$).


We say that a multi-redex is safe if it respects the formation rules
of the safe $\lambda$-calculus: the multi-redex $(\lambda x_1 \ldots
x_n . M) N_1 \ldots N_l$ is a safe redex if the variable
$x_1,\ldots,x_n$ are abstracted altogether at once using the
abstraction rule and if the terms $N_1 \ldots N_l$ are applied to
the term $\lambda x_1 \ldots x_n . M$ at once using either the rule
$\rulename{app^+}$ or $\rulename{app}$. The formal definition
follows:

\begin{dfn}[Safe redex]
A safe redex is a term of the form:
$$(\lambda \overline{x} . M) N_1 \ldots N_l$$
such that
\begin{itemize}
  \item variables $\overline{x}=x_1\ldots x_n$ are abstracted
altogether by one occurrence of the rule $\rulename{abs}$ in the
proof tree (possibly followed by the weakening rule). This implies
that:
$$\ord{M} -1 \leq \ord{\overline{x}} = \ord{x_1} = \ldots = \ord{x_n};$$
\item the terms $(\lambda \overline{x} . M)$, $N_1$,
$N_l$ are applied together at once using either:
\begin{itemize}
    \item the rule $\rulename{app}$:
        $$   \rulef{
                    \Sigma \vdash \lambda \overline{x} . M : (\overline{B_1}|\ldots|\overline{B_m}|o)
                    \quad
                    \Sigma \vdash N_1         \quad \ldots \quad \Sigma \vdash N_l
                    \quad l = |\overline{B_1}|
            }
            {
            \Sigma \vdash (\lambda \overline{x} . M) N_1 \ldots N_l
            } (\mathbf{app}),
        $$
        in which case  $n\leq |\overline{B_1}| = l$;

\item or the rule $\rulename{app^+}$:
        $$   \rulef{
                    \Sigma \vdash \lambda \overline{x} . M : (\overline{B_1}|\ldots|\overline{B_m}|o)
                    \quad
                    \Sigma \vdash N_1         \quad \ldots \quad \Sigma \vdash N_l
                    \quad l < |\overline{B_1}|
            }
            {
            \Sigma \vdash (\lambda \overline{x} . L) N_1 \ldots N_l
            } (\mathbf{app^+}),
        $$
      in which case $n \leq |\overline{B_1}|$ and no relation holds between $n$ and $l$.
\end{itemize}
\end{itemize}
It is not required to have $n = |\overline{B_1}|$.
\end{dfn}

Note that there are safe terms of the form $(\lambda x_1 \ldots x_n
. M) N_1 \ldots N_l$ with $l>n$. For instance the term $ (\lambda f
. ((\lambda g h . h) a) ) a a$ of type $o \rightarrow o$ for some
constant $a:o \rightarrow o$ and variables $x : o$ and $f,g,h:o
\rightarrow o$, can be formed using the $\rulename{app}$ rule as
follows:
$$ \rulef{
    \emptyset \vdash (\lambda f . ((\lambda g h . h) a) ) : (o,o),(o,o),o,o
        \quad \emptyset \vdash a : o,o
        \quad \emptyset \vdash a : o,o
    }
    {
       \emptyset \vdash (\lambda f . ((\lambda g h . h) a) ) a a : o,o
    } \rulename{app}
$$


\begin{dfn}[Safe reduction $\beta_s$] \
\label{dfn:safereduction} For the sake of concision, the following
abbreviations are used $\overline{x} = x_1 \ldots x_n$,
$\overline{N} = N_1 \ldots N_l$, and when $n\geq l$, $\overline{x_L}
= x_1 \ldots x_l$, $\overline{x_R} = x_{l+1} \ldots x_n$.
\begin{itemize}
\item The relation $\beta_s$ is defined on the set of safe redex as follows:
\begin{eqnarray*}
\beta_s &=&
\{  \ (\lambda \overline{x} : \overline{A} . T) N_1 \ldots N_l \mapsto \lambda \overline{x_R}. T\subst{\overline{N}}{\overline{x_L}}  \\
&& \mbox{ where $(\lambda \overline{x} : \overline{A} . T) N_1 \ldots N_l$ is a safe redex such that $n> l$}
\} \\
&\union&
\{ \ (\lambda \overline{x} : \overline{A} . T) N_1 \ldots N_l \mapsto T\subst{\overline{N}}{\overline{x}} N_{n+1} \ldots N_l  \\
&& \mbox{ where $(\lambda \overline{x} : \overline{A} . T) N_1 \ldots N_l$ is a safe redex such that $n\leq l$}
\}
\end{eqnarray*}
where the notation $\subst{\overline{N}}{\overline{x}}$ denotes the capture-permitting simultaneous substitution.

\item
The safe $\beta$-reduction, written $\betasred$, is the closure of
the relation $\beta_s$ by compatibility with the formation rules of
the safe $\lambda$-calculus.
\end{itemize}
\end{dfn}



We observe that safe $\beta$-reduction is a certain kind of multi-steps $\beta$-reduction.
\begin{property}
$\betasred \subset \betaredtr$, i.e. the safe
$\beta$-reduction relation is included in the transitive closure of the $\beta$-reduction relation.
\end{property}
\begin{proof}
Suppose that $(M\mapsto N) \in \beta_s$. We show that $M \betared^* N$.
\begin{itemize}
\item Suppose that the safe-redex is
$M \equiv (\lambda \overline{x} : \overline{A} . T) N_1 \ldots N_l$ such that $n\leq l$ then:
\begin{eqnarray*}
 M &=_\alpha& (\lambda z_1 \ldots z_n .T [z_1,\ldots z_n /x_1,\ldots x_n] ) \ N_1  N_2 \ldots N_l
            \\
&& \mbox{where the $z_i$ are fresh variables}  \\
     &\betared& (\lambda z_2 \ldots z_n .T [z_1,\ldots z_n /x_1,\ldots x_n] \subst{N_1}{z_1} ) \ N_2 \ldots N_l \\
&& \mbox{ (because the $z_i$s do not occur freely in $N_1$) }\\
%%    &=_\alpha& (\lambda z_2 \ldots z_n .T [z_2,\ldots z_n /x_2,\ldots x_n] \subst{N_1}{x_1})\  N_2 \ldots N_l  \qquad \mbox{where the $z_i$ are fresh variables}  \\
    &\betared& \ldots \\
    &\betared& (T [z_1,\ldots z_n /x_1,\ldots x_n] \subst{N_1}{z_1}  \ldots \subst{N_n}{z_n})\  N_{n+1} \ldots N_l \\
    &\betared& (T [N_1\ldots N_l/x_1,\ldots x_l])\ N_{n+1} \ldots
    N_l,
\end{eqnarray*}
and since $T$ is safe, the substitution $T [N_1\ldots N_l/x_1,\ldots
x_l]$ in the last equation can be performed using the
capture-permitting substitution. Hence $M \betared^* N$.

\item
 Suppose that $M \equiv (\lambda \overline{x} : \overline{A} . T) N_1 \ldots N_l$ such that $n> l$, then necessarily
the redex must be formed using the $\rulename{app^+}$ rule. The
side-condition of this rules says that the free variables of the
terms $N_1, \ldots N_l$ have all order strictly greater than
$\ord{\overline{x}}$, hence the $x_i$s do not occur freely in $N_1,
\ldots N_l$. Therefore:
\begin{eqnarray*}
 M &=& (\lambda x_1 \ldots x_n .T) \ N_1  N_2 \ldots N_l  \\
     &\betared& (\lambda x_2 \ldots x_n .T \subst{N_1}{x_1} ) \ N_2 \ldots N_l \\
            && \mbox{(for $i \in 2..n$, $x_i$ does not occur freely in $N_1$)}\\
    &\betared& \ldots \\
    &\betared& \lambda x_{l+1} \ldots x_n . T \subst{N_1}{x_1}  \ldots \subst{N_l}{x_l} \\
        && \mbox{(for $i \in (l+1)..n$,  $x_i$ does not occur freely in $N_l$)}\\
    &\betared& \lambda x_{l+1} \ldots x_n . T [N_1\ldots, N_l /  \ x_1,\ldots, x_l] \\
        && \mbox{(the $x_i$ do not occur freely in $N_1, \ldots
        N_l$)},
\end{eqnarray*}
and since $T$ is safe, the substitution $T [N_1\ldots N_l/x_1,\ldots
x_l]$ in the last equation can be performed using the
capture-permitting substitution. Hence $M \betared^* N$.
\end{itemize}
\end{proof}

\begin{property} In the simply-typed $\lambda$-calculus:
\begin{enumerate}
\item $\betasred$ is strongly normalizing.
\item $\beta_s$ has the unique normal form property.
\item $\beta_s$ has the Church-Rosser property.
\end{enumerate}
\end{property}

\begin{proof}
1. This is because $\betasred \subset \betaredtr$ and, $\betared$ is
strongly normalizing in the simply-typed $\lambda$-calculus. 2. A
term has a safe redex iff it has a $\beta$-redex therefore the set
of $\beta_s$ normal forms is equal to the set of $\beta_s$ normal
forms. Hence, the unicity of $\beta$-normal form implies the unicity
of $\beta_s$-normal form. 3. is a consequence of 1 and 2.
\end{proof}


Capture-permitting simultaneous substitution preserves safety (lemma
\ref{lem:subst_preserve_safety}), consequently any safe redex
reduces to a safe term:

\begin{lem}[The safe reduction $\beta_s$ preserves safety]
\label{lem:homoh_safered_preserve_safety}
If $M$ is safe and $M \betasred N$ then $N$ is safe.
\end{lem}

\begin{proof}
It suffices to show that the relation $\beta_s$ preserves safety.
Consider the safe-redex $(s\mapsto t) \in \beta_s$ where $ s \equiv (\lambda x_1 \ldots x_n . M) N_1 \ldots N_l $ .
We proceed by case analysis on the last rule used to form the redex.
\begin{itemize}
\item Suppose the last rule used is $\rulename{app}$, then necessarily $n\leq l$ and the reduction is
$$(\lambda x_1 \ldots x_n . M) N_1 \ldots N_l \qquad \mapsto  \qquad t \equiv M[N_1 / x_1 , \cdots, N_n / x_n]\ N_{n+1} \ldots N_l.$$
The first premise of the rule $\rulename{app}$ tells us that $M$ is safe therefore using lemma \ref{lem:subst_preserve_safety} and
the application rule we obtain that $t$ is safe.

\item Suppose the last rule used is $\rulename{app^+}$ and $n> l$ then the reduction is
$$ (\lambda \overline{x_L} : \overline{A_L} \
\overline{x_R}: \overline{A_R} . T) \overline{N_L} \qquad \mapsto
\qquad t \equiv \lambda \overline{x_R}: \overline{A_R} .
T\subst{\overline{x_L}}{\overline{N_L}}.
$$
By lemma \ref{lem:subst_preserve_safety}, $T\subst{\overline{x_L}}{\overline{N_L}}$ is a safe term.
Using the rule $\rulename{abs}$ we derive that $t$ is safe.

\item Suppose the last rule used is $\rulename{app^+}$ and $n\leq l$ then the reduction is
$$(\lambda x_1 \ldots x_n . M) N_1 \ldots N_l \qquad \mapsto \qquad t \equiv M[N_1 / x_1 , \cdots, N_n / x_n]\ N_{n+1} \ldots N_l$$
We conclude that $t$ is safe similarly to case $\rulename{app}$.

\item Rule $\rulename{wk}$ $\rulename{seq}$: these cases reduce to one of the previous cases.
\end{itemize}
\end{proof}


\begin{rem}
\label{rem:betasred_notpreserv_unsafety} $\betasred$ \emph{does not}
preserves un-safety: given two terms $S$ safe and $U$ unsafe of the
same type, the term $(\lambda x y . y) U S$ is also unsafe but it
$\beta_s$-reduces to $S$ which is safe.
\end{rem}


\subsection{An alternative system of rules}


In this section, we will refine the formation rules
given in the previous section. We say that $\Gamma \vdash M : A$ verifies $P_i$ for $i \in \zset$ if the
variables in $\Gamma$ all have orders at least $\ord{A}+i$. We introduce the notation $\Gamma \vdash^{i} M : A$ for $i \in
\zset$ to mean that $\Gamma \vdash M : A$ is a valid judgment satisfying $P_i$.


We remark that if $\Gamma \vdash M : A$ then the variables in $\Gamma$ with order
strictly smaller than $M$ cannot occur freely in $M$ and therefore it is possible to restrict
the context to a smaller number of variables:

\begin{lem}[Context reduction]
\label{lem:restriction}

If $\Gamma \vdash^i M : A$ then $\Gamma' \vdash^{0} M : A$
where $$\Gamma' = \{ z \in \Gamma \ |
\ \ord{z} \geq \ord{M} \} = \Gamma \setminus \{ z \in \Gamma \ | \ \ord{M} + i \leq \ord{z} < \ord{M} \}$$
\end{lem}
\begin{proof}
If $i\geq 0$ then the result is trivial. Suppose $i<0$. We proceed
by structural induction and case analysis. We only give the details
for the application cases $\rulename{app}$ and $\rulename{app^+}$:
\begin{itemize}
\item Case of the rule $\rulename{app}$:

    \[ (\mathbf{app}) \
    \rulef
        {\seq{\Gamma}{M : (\overline{B_1} \, | \, \cdots \, | \, \overline{B_m} \, | \, o)} \qquad
            \seq{\Gamma}{N_1 : B_{11}} \quad \cdots \quad \seq{\Gamma}{N_{l} :
            B_{1l}} \qquad l = |\overline{B_1}| }
        { \seq{\Gamma}{M N_1
            \cdots N_{l} : (\overline{B_2} \, | \, \cdots \, | \,
            \overline{B_m} \, | \, o)}}
    \]

    If the conclusion verifies $P_i$ then, for all $z \in \Gamma$:
    \begin{eqnarray*}
    \ord{z} \geq 1 + \ord{\overline{B_2}} + i
    &=& 1 + \ord{\overline{B_1}} + \ord{\overline{B_2}} - \ord{\overline{B_1}} + i \\
    &=& \ord{M} + (\ord{\overline{B_2}} - \ord{\overline{B_1}} + i)
    \end{eqnarray*}
    Therefore the first premise satisfies $P_j$ where $j={\ord{\overline{B_2}} - \ord{\overline{B_1}} + i}$.
    Hence by the induction hypothesis,
    $$\Gamma' \vdash^{0} M : (\overline{B_1} \, | \, \cdots \, | \, \overline{B_m} \, | \, o)$$
    where $\Gamma' = \Gamma \setminus \{ z \in \Gamma \ | \ \ord{M} + j \leq \ord{z} < \ord{M} \}$.


    Similarly, for all $z \in \Sigma$:
    \begin{eqnarray*}
    \ord{z} \geq 1 + \ord{\overline{B_2}} + i
    &=& \ord{\overline{B_1}} + (1+\ord{\overline{B_2}} - \ord{\overline{B_1}} + i) \\
    &=& \ord{\overline{B_1}} + j+1
    \end{eqnarray*}
    Hence by the induction hypothesis:
    $$\Gamma'' \vdash^0 N_k : B_{1k} \mbox{ for } k \in 1..l$$
    where $\Gamma'' = \Gamma \setminus \{ z \in \Gamma \ | \ \ord{M} + j+1 \leq \ord{z} < \ord{M} \}$.

    Furthermore, $\Gamma'' = \Gamma' \union \{ z \in \Gamma \ | \ \ord{M} + j = \ord{z}\}$ therefore
    the weakening rule gives:
    $$\Gamma'' \vdash^{-1} M : (\overline{B_1} \, | \, \cdots \, | \, \overline{B_m} \, | \, o)$$

    Finally the $\rulename{app}$ rule gives:
    $$\rulef{\Gamma'' \vdash^{-1} M : (\overline{B_1} \, | \, \cdots \, | \, \overline{B_m} \, | \, o)
    \quad \Gamma'' \vdash^0 N_1 : B_{11} \quad \ldots \quad \Gamma'' \vdash^0 N_1 : B_{1l}
    }
        { \Gamma'' \vdash M N_1 \ldots N_l : (\overline{B_2} \, | \, \cdots \, | \,
            \overline{B_m} \, | \, o)}
    $$
    such that for all $z\in \Gamma''$:
    \begin{eqnarray*}
    \ord{z} \geq \ord{\overline{B_1}}
    &\geq& 1 + \ord{\overline{B_2}} = \ord{M N_1 \ldots N_l}
    \end{eqnarray*}

    Therefore:
    $$\Gamma'' \vdash^0 M N_1 \ldots N_l : (\overline{B_2} \, | \, \cdots \, | \,
            \overline{B_m} \, | \, o)$$

\item $\rulename{app^+}$  The side-condition of the rule $\rulename{app^+}$ ensures that the first premise
 verifies $P_0$. The conclusion of the rule has the same order as the first premise
 therefore the conclusion also verifies $P_0$.
\end{itemize}
\end{proof}


\begin{lem}
\label{lem:prooftree01only} If $\Gamma \vdash^{0} M : T$ or $\Gamma
\vdash^{-1} M : T$ then there is a valid proof tree showing $\Gamma
\vdash M : T$ such that all the judgments appearing in the proof
tree verify either $P_0$ or $P_{-1}$.
\end{lem}


\begin{proof}
Since $P_{-1}$ implies $P_0$, w.l.o.g. we can assume that the
judgment $\Gamma \vdash M : T$ satisfies $P_{-1}$. We show that
there is a proof tree for $\Gamma \vdash M : T$ where all the nodes
of the tree verify $P_0$ or $P_{-1}$. We proceed by structural
induction and case analysis on the last rule used to show $\Gamma
\vdash M : T$:
\begin{itemize}
\item Axiom $\rulename{\Sigma\mbox{\textbf{-const}}}$: the context is empty therefore the sequent verifies $P_{-1}$.

\item Axiom $\rulename{var}$: the context contains only the variable itself therefore the sequent verifies $P_0$.

\item Rule $\rulename{wk}$: The premise is $\Delta \vdash M : T$ with $\Delta \subset \Gamma$. Since
$\Gamma \vdash M : T$ verifies $P_{-1}$ and $\Delta \subset \Gamma$ the premise must also verify $P_{-1}$. We can conclude using the
induction hypothesis.

\item Rule $\rulename{perm}$: By the induction hypothesis.


\item Rule $\rulename{abs}$: the second premise of the rule guarantees that the first
premise verifies $P_{-1}$.

\item Rule $\rulename{app^+}$: The first premise has the same order as the
conclusion of the rule therefore the first premise verifies
$P_0$. The side-condition of the rule $\rulename{app^+}$ ensures that all the other premises verify $P_0$.

\item Rule $\rulename{app}$:

$$ \rulename{app} \
    \rulef{
        { \Gamma \vdash M : (\overline{A} \, | B)
        \qquad
        \Gamma \vdash N_1 : A_1 \quad \cdots \quad \Gamma \vdash N_{l} : A_l \qquad l = |\overline{A}|
        }
    }
    {
        \Gamma \vdash^0 M N_1 \cdots N_{l} : B
    }
$$

Applying lemma \ref{lem:restriction} to the first premise we obtain:
\begin{equation}
 \Sigma \vdash^0 M : (\overline{A} \, | B) \label{eq:seq1}
\end{equation}
where $\Sigma = \{ z \in \Gamma \ | \ \ord{z} \geq \ord{(\overline{A} \, | B)} \} = \{ z \in \Gamma \ | \ \ord{z} \geq 1 + \ord{\overline{A}} \}.$

Applying lemma \ref{lem:restriction} to each of the remaining
premises gives  :
$$ \Sigma' \vdash^0 N_i : A_i \quad \mbox{ for all } i \in 1..p$$
where $\Sigma' = \{ z \in \Gamma \ | \ \ord{z} \geq \ord{A_i} =
\ord{\overline{A}} \} \supseteq \Sigma.$

If the inclusion $\Sigma \subseteq \Sigma'$ is strict then we apply the weakening rule to sequent (\ref{eq:seq1}):
$$ \rulef{\Sigma \vdash^0 M : (\overline{A} \, | B)}{\Sigma' \vdash^{-1} M : (\overline{A} \, | B)} \rulename{wk} $$

Finally, we obtain the following proof tree:
$$  \rulef{
        \rulef{
            { \Sigma' \vdash^{-1} M : (\overline{A} \, | B)
            \qquad
            \Sigma' \vdash^0 N_1 : A_1 \quad \cdots \quad \Sigma' \vdash^0 N_{l} : A_l \qquad l = |\overline{A}|
            }
        }
        {
            \Sigma' \vdash^0 M N_1 \cdots N_{l} : B
        } \rulename{app}
    }
    {
         \Gamma \vdash^0 M N_1 \cdots N_{l} : B
    } \rulename{wk}
$$

where the last weakening rules is applied only if the inclusion $\Sigma' \subseteq \Gamma$ is strict.

We can now conclude by applying the induction hypothesis on the
sequents $\Sigma' \vdash^{-1} M$, $\Sigma' \vdash^0 N_1$, \ldots,
$\Sigma' \vdash^0 N_l$ .
\end{itemize}
\end{proof}

\subsubsection{An alternative definition of the homogeneous safe $\lambda$-calculus}

Using the observations that we have just made, we will now derive
new rules for the safe $\lambda$-calculus with homogeneous type. We
want a system of rules generating sequents that verify $P_0$. Also,
it must be able to generate intermediate sequents that do not
necessarily satisfy $P_0$ provided that they can be used to produce
\emph{in fine} terms satisfying $P_0$.

Because of the lemma \ref{lem:prooftree01only}, we know that the
only necessary intermediate sequents are those that either satisfy
$P_0$ or $P_{-1}$. Hence, we can assume by default that premises of
the rules all satisfy $P_{-1}$ at least.

First we define an additional rule expressing the fact that $P_0$
implies $P_{-1}$:
$$ \rulename{seq} \  \rulef{\Gamma \vdash^{0} M : A}{\Gamma \vdash^{-1} M : A} $$

The weakening rule can be rewritten as follows:
$$ \rulename{wk^{0}} \   \rulef{\Gamma \vdash^{0} M : A}{\Gamma , x : B \vdash^{0} M : A} \quad \ord{B} \geq \ord{A} $$
$$ \rulename{wk^{-1}} \   \rulef{\Gamma \vdash^{-1} M : A}{\Gamma , x : B \vdash^{-1} M : A} \quad \ord{B} \geq \ord{A} -1$$

Because of the context reduction lemma, any sequent verifying $P_{-1}$ can be obtained
by applying the weakening rule $\rulename{wk^{-1}}$ or the rule $\rulename{seq}$ to another sequent
verifying $P_0$. Therefore, with the exception of these two rules, we only need to use rules
whose conclusion sequents verify $P_0$:
\begin{itemize}
\item For the rules $\rulename{perm}$, $\rulename{const}$ and $\rulename{var}$, only the tagging of the sequents
changes:
$$ \rulename{var} \   \rulef{}{x : A\vdash^{0} x : A}
\qquad
\rulename{\Sigma\mbox{\textbf{-const}}}  \  \rulef{}{\vdash^0 b : A} \ b:A \in \Sigma
$$

$$
  \rulename{perm} \  \rulef{
      { \Gamma \vdash^0 M:B \qquad \sigma(\Gamma)  } \hbox{ homogeneous}
    }
      { \sigma(\Gamma) \vdash^0 M : B }
$$

\item $\rulename{abs}$ The abstraction rule has a side condition
expressing the fact that the premise verifies $P_0$ or $P_{-1}$. Since this is always true for sequents
generated by our new system of rules, we can drop the side condition:
$$ \rulename{abs} \   \rulef{\Gamma | \overline{x} : \overline{A} \vdash^{-1} M : B}
                                   {\Gamma  \vdash^{0} \lambda \overline{x} : \overline{A} . M : (\overline{A},B)}$$


\item $\rulename{app}$ The application rule has the following form:
$$ \rulename{app} \
    \rulef{
        { \Gamma \vdash^{-1} M : (\overline{A} \, | B)
        \qquad
        \Gamma \vdash^{-1} N_1 : A_1 \quad \cdots \quad \Gamma \vdash^{-1} N_{l} : A_l \qquad l = |\overline{A}|
        }
    }
    {
        \Gamma \vdash^0 M N_1 \cdots N_{l} : B
    }
$$

Since the first premise verifies $P_{-1}$, by property \ref{proper:safe_basic_prop}(ii) we have:
$$\forall z \in \Gamma : \ord{z} \geq 1 + \ord{\overline{A}} -1 = \ord{\overline{A}} = \ord{\overline{N}}$$
Hence, all the sequents of the premises but the first must verify $P_0$. The rule (app) is therefore given by:
$$ \rulename{app} \
    \rulef{
        { \Gamma \vdash^{-1} M : (\overline{A} \, | B)
        \qquad
        \Gamma \vdash^0 N_1 : A_1 \quad \cdots \quad \Gamma \vdash^0 N_{l} : A_l \qquad l = |\overline{A}|
        }
    }{
        \Gamma \vdash^0 M N_1 \cdots N_{l} : B
      }
$$

\item For the application rule $\rulename{app^+}$, the type of the sequent in the first premise has the same order
as the type of the conclusion premise, and since the conclusion
verifies $P_0$, the first premise must also verify $P_0$. The
side-condition implies that all the other sequents in the premise
verify $P_0$. Moreover since the first premise verifies $P_0$, the
side-condition must hold. Hence the rule becomes:
$$ \rulename{app^+} \
    \rulef{
        \Gamma \vdash^0 M : (\overline{B_1} \, | \, \cdots \, | \, \overline{B_m} \, | \, o) \qquad
        \Gamma \vdash^0 N_1 : B_{11} \quad \cdots \quad \Gamma \vdash^0 N_{l} : B_{1l} \qquad l < |\overline{B_1}|
    }
    {
        \Gamma \vdash^0 M N_1 \cdots N_{l} : (\overline{B} \, | \, \cdots \, | \, \overline{B_m} \, | \, o)
    }
$$
where $\overline{B_1} = B_{11}, \ldots, B_{1l},\overline{B}$.
Clearly, this rule can be equivalently stated as:
$$ \rulef{\Gamma \vdash^0 M : A\rightarrow B
                                        \qquad \Gamma \vdash^{0} N : A
                                   }
                                   {\Gamma  \vdash^{0} M N : B}$$
\end{itemize}

The full set of rules is given in table \ref{tab:homosafelmd_rules_refined}.

\begin{table}[htbp]
$$  \rulename{perm} \
    \rulef{
      { \Gamma \vdash^0 M:B \qquad \sigma(\Gamma)  } \hbox{ homogeneous}
    }
    { \sigma(\Gamma) \vdash^0 M : B
    }
\qquad
\rulename{seq} \  \rulef{\Gamma \vdash^{0} M : A}{\Gamma \vdash^{-1} M : A}
$$

$$
\rulename{\Sigma\mbox{\textbf{-const}}} \  \rulef{}{\vdash^0 b : A}\ b:A \in \Sigma
\qquad
 \rulename{var} \   \rulef{}{x : A\vdash^{0} x : A} $$

$$ \rulename{wk^{0}} \   \rulef{\Gamma \vdash^{0} M : A}{\Gamma , x : B \vdash^{0} M : A} \quad \ord{B} \geq \ord{A} $$

$$ \rulename{wk^{-1}} \   \rulef{\Gamma \vdash^{-1} M : A}{\Gamma , x : B \vdash^{-1} M : A} \quad \ord{B} \geq \ord{A} -1$$


$$ \rulename{app} \
    \rulef
        {   \Gamma \vdash^{-1} M : (\overline{A} \, | B)
            \qquad
            \Gamma \vdash^0 N_1 : A_1 \quad \cdots \quad \Gamma \vdash^0 N_{l} : A_l \qquad l = |\overline{A}|
        }
        {
            \Gamma \vdash^0 M N_1 \cdots N_{l} : B
        }
$$

$$ \rulename{app^+} \   \rulef{\Gamma \vdash^0 M : A\rightarrow B
                                        \qquad \Gamma \vdash^{0} N : A
                                   }
                                   {\Gamma  \vdash^{0} M N : B}$$

$$ \rulename{abs} \   \rulef{\Gamma| \overline{x} : \overline{A} \vdash^{-1} M : B}
                                   {\Gamma  \vdash^{0} \lambda \overline{x} : \overline{A} . M : (\overline{A}|B)}$$


where $\Gamma| \overline{x} : \overline{A}$ means that the lowest type-partition of the context is
$\overline{x} : \overline{A}$.
\caption{Alternative rules for the homogeneous safe lambda calculus}
\label{tab:homosafelmd_rules_refined}
\end{table}
%%%

\include{chap_safe_nonhomog}
% \include{safe_nonhomog_va}

% third chapter
\chapter{Computation trees, traversals and game semantics}

The aim of this chapter is to develop tools that will be used in the
next chapter to give a characterisation of the game semantics of the
Safe $\lambda$-Calculus. Establishing such a characterisation is
complicated by the fact that Safety is a syntactic restriction
whereas Game Semantics is by nature a syntax-independent semantics.
We therefore need to make an explicit correspondence between the
game denotation of a term and its syntax.

Our approach follows ideas recently introduced in
\cite{OngLics2006}, mainly the notion of computation tree of a
simply-typed $\lambda$-term and traversals over the computation
tree. A computation tree can be regarded as an abstract syntax tree
(AST) of the $\eta$-long normal form of a term. A traversal is a
justified sequence of nodes of the computation tree respecting some
formation rules. Traversals are used to describe computations. An
interesting property is that the \emph{P-view} of a traversal
(computed in the same way as P-view of plays in Game Semantics) is a
path in the computation tree.

The main result that we will prove in this chapter is called the
\emph{Correspondence Theorem} (theorem \ref{thm:correspondence}). It
states that traversals over the computation tree are just
representations of the uncovering of plays in the
strategy-denotation of the term. Hence there is an isomorphism
between the strategy denotation of a term and its revealed game
denotation (i.e. its strategy denotation where internal moves are
not hidden after composition). This theorem permits us to explore
the effect that a given syntactic restriction has on the strategy
denotating a term.

To really make use of the Correspondence Theorem, it will be
necessary to restate it in the standard game-semantic framework in
which internal moves are hidden. For that purpose, we will define a
\emph{reduction} operation on traversals responsible of eliminating
the ``internal nodes'' of the computation. This leads to a
correspondence between the standard game denotation of a term and
the set of reductions of traversals over its computation tree.
Fortunately, the reduction process preserves the good properties of
traversals. This is guaranteed by the facts that the P-view of the
reduction of a traversal is equal to the reduction of the P-view of
the traversal, and the O-view of a traversal is the same as the
O-view of its reduction (lemma \ref{lem:redtrav_trav}). \vspace{8pt}

\emph{Related works}: Traversals of a computation tree provide a way
to perform \emph{local computation} of $\beta$-reductions as opposed
to a global approach where the $\beta$-reduction is implemented by
performing substitutions. A notion of local computation of
$\beta$-reduction has been investigated in
\cite{DanosRegnier-Localandasynchronou} through the use of special
graphs called ``virtual nets'' that embed the lambda-calculus.

In \cite{DBLP:conf/lics/AspertiDLR94}, a notion of graph based on
Lamping's graphs \citep{lamping} is introduced to represent
$\lambda$-terms. The authors unify different notions of paths
(regular, legal, consistent and persistent paths) that have appeared
in the literature as ways to implement graph-based reduction of
lambda-expressions. We can regard a traversal as an alternative
notion of path adapted to the graph representation of
$\lambda$-expressions given by computation trees.



%Is there any unsafe term whose game semantics is a strategy where
%pointers can be recovered?
%
%The answer is yes: take the term $T_i = (\lambda x y . y) M_i S$
%where $i =1..2$ and $\Gamma \vdash_s S : A$. $T_1$ and $T_2$ both
%$\beta$-reduce to the safe term $S$, therefore
%$\sem{T_1}=\sem{T_2}=\sem{S}$. But $T_1$ is safe whereas $T_2$ is
%unsafe. Since it is possible to recover the pointer from the game
%semantics of $S$, it is as well possible to recover the pointer from
%the semantics of $T_2$ which is unsafe.

\section{Computation tree}
We work in the general setting of the simply-typed
$\lambda$-calculus extended with a fixed set $\Sigma$ of
higher-order constants.

\subsection{$\eta$-long normal form and computation tree}

The $\eta$-long normal form appeared in
\citep{DBLP:journals/tcs/JensenP76} and
\citep{DBLP:journals/tcs/Huet75} under the names \emph{long reduced
form} and \emph{$\eta$-normal form} respectively. It was then
investigated in \citep{huet76} under the name \emph{extensional
form}.

The $\eta$-expansion of $M: A\typear B$ is defined to be the term
$\lambda x . M x : A\typear B$ where $x:A$ is a fresh variable. A
term $M : (A_1,\ldots,A_n,o)$ can be expanded in several steps into
$\lambda \varphi_1 \ldots \varphi_l . M \varphi_1 \ldots \varphi_l$
where the $\varphi_i:A_i$ are fresh variables. The $\eta$-normal
form of a term is obtained by hereditarily $\eta$-expanding every
subterm occurring at an operand position.

\begin{dfn}[$\eta$-long normal form]
A simply-typed term is either an abstraction or it can be written uniquely as
$s_0 s_1 \ldots s_m$ where $m\geq0$ and $s_0$ is a variable, a $\Sigma$-constant or an abstraction.
The $\eta$-long normal form of a term $M$, written $\aux{M}$ or sometimes $\etanf{M}$,
is defined as follows:
\begin{align*}
\aux{\alpha s_1 \ldots s_m : (A_1,\ldots,A_n,o)} &= \lambda \overline{\varphi} . \alpha \aux{s_1}\ldots \aux{s_m} \aux{\varphi_1} \ldots \aux{\varphi_n}
& \mbox{with $m,n\geq0$}\\
%\aux{(\lambda x . s_0) s_1 \ldots s_m } &=& (\lambda x . \aux{s_0}) \aux{s_1} \aux{s_2} \ldots \aux{s_m}
\aux{\lambda x . s } &= \lambda x . \aux{s} \\
\aux{(\lambda x . s_0) s_1 \ldots s_m : (A_1,\ldots,A_n,o) } &= \lambda \overline{\varphi} . (\lambda x . \aux{s_0}) \aux{s_1} \ldots \aux{s_m} \aux{\varphi_1} \ldots \aux{\varphi_n}
& \mbox{with $m\geq 1,n\geq0$}
\end{align*}
where $x$ and each $\varphi_i : A_i$ are variables and $\alpha$ is
either a variable or a constant.
\end{dfn}

For $n=0$, the first clause in the definition becomes:
$$\aux{x s_1 \ldots s_m : o} = \lambda . x \aux{s_1} \aux{s_2} \ldots \aux{s_m},$$
and we deliberately keep the \textsl{dummy} lambda in the right-hand
side of the equation because it will play an important role in the
correspondence with game semantics.



Note that our version of the $\eta$-long normal form is defined not only for $\beta$-normal terms but also for any simply-typed term.
Moreover it is defined in such a way that $\beta$-normality is preserved:
\begin{lem}
The $\eta$-long normal form of a term in $\beta$-normal form is also in $\beta$-normal form.
\end{lem}
\begin{proof}
By induction on the structure of the term and the order of its type.
\emph{Base case}:
If $M=x:0$ then $\aux{x} = \lambda . x$ is also in $\beta$-nf.
\emph{Step case}:
The case $M = \aux{(\lambda x . s_0) s_1 \ldots s_m : (A_1,\ldots,A_n,o)}$ with $m>0$ is not possible since $M$ is in
$\beta$-normal form.
Suppose $M = \lambda x . s$ then $s$ is in $\beta$-nf. By the induction hypothesis $\aux{s}$ is also in $\beta$-nf and therefore
so is $\aux{M} = \lambda x . \aux{s}$.

Suppose $M= \alpha s_1 \ldots s_m : (A_1,\ldots,A_n,o)$. Let $i,j$
range over $1..n$ and $1..m$ respectively. The $s_j$ are in
$\beta$-nf and the $\varphi_i$ are variables of order smaller than
$M$, therefore by the induction hypothesis the $\aux{\varphi_i}$ and
the $\aux{s_j}$ are in $\beta$-nf. Hence $\aux{M}$ is also in
$\beta$-nf.
\end{proof}


The computation tree of term is a certain tree representation of its
$\eta$-long normal form. It is defined as follows:
\begin{dfn}[Computation tree]
For any term $M$ in $\eta$-normal form we define the tree $\tau(M)$ by induction
on the structure of the term.
Since $M$ is in $\eta$-normal form, there are only two cases:
$M$ is either an abstraction or it is of ground type and can be written uniquely as
$s_0 s_1 \ldots s_m : 0$ where $m\geq0$,  $s_0$ is a variable, a
constant or an abstraction and each of the $s_j$ for $j\in 1..m$ is in $\eta$-normal form:
\begin{itemize}
\item the tree for $\lambda x_1 \ldots x_n. s$ where $n\geq0$ and $s$ is not an abstraction is:
$$ \tau(\lambda x_1 \ldots x_n . s) =
      \pstree[levelsep=4ex]
        { \TR{\lambda x_1 \ldots x_n} }
        { \SubTree{\tau(s)^{-}} }
$$
where $\tau(s)^{-}$ denotes the tree obtained after deleting the
root of $\tau(s)$.


\item the tree for $\alpha s_1 \ldots s_m : o$ where $m\geq0$ and $\alpha$ is a variable or constant is:
$$ \tau( \alpha s_1 \ldots s_m) =
    \tree{\lambda}
    {
        \pstree[levelsep=4ex]
            { \TR{\alpha} }
            { \SubTree{\tau(s_1)} \SubTree[linestyle=none]{\ldots} \SubTree{\tau(s_m)}
            }
    }
$$


\item the tree for $(\lambda x.s_0) s_1 \ldots s_n : o$ where $n \geq 1$ is:
$$ \tau((\lambda x.s_0) s_1 \ldots s_n) =
    \tree{\lambda}
    {
        \pstree[levelsep=4ex]
            { \TR{@} }
            {
            \SubTree{\tau(\lambda x.s_0)}    \SubTree{\tau(s_1)} \SubTree[linestyle=none]{\ldots} \SubTree{\tau(s_n)}
            }
    }
$$
\end{itemize}

The \emph{computation tree} of a simply-typed term $M$ (whether or not in $\eta$-normal form) is written $\tau(M)$
and defined to be $\tau(M) = \tau(\etanf{M})$.
\end{dfn}

The nodes (and leaves) of the tree are of three kinds:
\begin{itemize}
\item $\lambda$-nodes labelled $\lambda \overline{x}$ (note that a $\lambda$-node represents several consecutive variable abstractions),
\item application nodes labelled @,
\item variable or constant nodes labelled $\alpha$ for some constant or variable $\alpha$.
\end{itemize}
We write $N$ for the set of nodes of $\tau(M)$, $N_\Sigma$ for the set of $\Sigma$-labelled nodes,
$N_@$ for the set of @-labelled nodes, $N_{var}$ for the set of variable nodes and
$N_{fv}$ for the subset of $N_{var}$ constituted of free-variable nodes.


Let $\mathcal{T}$ denote the set of $\lambda$-terms.
Each subtree of the computation tree $\tau(M)$ represents a subterm of $\aux{M}$.
We define the function $\kappa : N \rightarrow \mathcal{T}$ that maps a node $n \in N$ to the subterm of $\aux{M}$
represented by the subtree of $\tau(M)$ rooted at $n$.
In particular if $r$ is the root of $\tau(M)$ then $\kappa(r) = \aux{M}$.

\begin{dfn}[Node order]
\label{def:nodeorder}
The node-order function $\textsf{ord}$ is defined on nodes as follows:
\begin{eqnarray*}
\ord{r} &=& \ord{\lambda \overline{x} . M } \\
\ord{n} &=& \ord{\kappa(n)}, \hbox{ for $n \neq r$. }
\end{eqnarray*}
\end{dfn}

In particular, $\ord{@} = 0$, $\ord{\alpha:T} = \ord{T}$ if $\alpha$ is a variable or constant,
$\ord{\lambda \overline{\xi}} = \max_{z\in \overline{\xi}} \ord{z}$ is $\lambda \overline{\xi}$ is not the root,
and $\ord{\lambda \overline{\xi}} = 1 + \max_{z\in \overline{\xi}\union fv(M)} \ord{z}$ if it is the root with the convention that $\max \emptyset = -1$.

\noindent Some remarks:
\begin{itemize}
\item In a computation tree, nodes at even level are $\lambda$-nodes and nodes at odd level are either application nodes,
variable or constant nodes;

\item for any ground type variable or constant $\alpha$,
$\tau(\alpha) = \tau(\lambda . \alpha) =  \pstree[levelsep=3ex]
    { \TR{\lambda } }
    { \TR{\alpha}
    }$;

\item for any higher-order variable or constant $\alpha : (A_1,\ldots,A_p,o)$, the computation tree $\tau(\alpha)$ has the following form:
$ \pstree[levelsep=3ex]{\TR{\lambda}}
        {\pstree[levelsep=3ex]
                { \TR{\alpha} }
                { \tree{\lambda \overline{\xi_1}}{\TR{\ldots}} \TR{\ldots} \tree{\lambda \overline{\xi_p}}{\TR{\ldots}}
                }
        }
$;

\item for any tree of the form
        $ \pstree[levelsep=4ex]
            { \TR{\lambda \overline{\varphi}} }
            { \pstree[levelsep=3ex]
                {\TR{n}}
                {\TR{\lambda \overline{\xi_1}} \TR{\ldots} \TR{\lambda \overline{\xi_p}}}
            }
        $,
    we have $\ord{\kappa(n)}=0$.

\end{itemize}



\subsection{Pointers and justified sequence of nodes}

\begin{dfn}[Binder]
Let $n$ be a variable node of the computation tree labelled $x$. We
say that a node $n$ is bound by the node $m$, and $m$ is called the
binder of $n$, if $m$ is the closest node in the path from $n$ to
the root of the tree such that $m$ is labelled $\lambda
\overline{\xi}$ with $x\in \overline{\xi}$.
\end{dfn}

\begin{dfn}[Enabling]
The enabling relation $\vdash$ is defined on the set of nodes of the
computation tree. We write $m \vdash n$ and we say that $m$ enables
$n$ if and only if
\begin{itemize}
\item $n$ is a bound variable node and $m$ is the binder of $n$,
\item or $n$ is a free variable node and $m$ is the root of the computation tree,
\item or $n$ is a $\lambda$-node and $m$ is the parent node of $n$.
\end{itemize}
\end{dfn}

We call \emph{input-variable} a variable that is hereditarily justified by the root of the computation tree.
Free variables and variables bound by the root are example of input-variables.

\begin{dfn}[Justified sequence of nodes]
A \emph{justified sequence of nodes} is a sequence of
nodes of the computation tree $\tau(M)$ with pointers attached to the nodes. A node $n$ in the sequence
that is either a variable node or a lambda-node different from the root of the computation tree
has a pointer to a node $m$ occurring before $n$ in the sequence such that $m \vdash n$.
If $n$ points to $m$ then we say that $m$ \emph{justifies} $n$ and we represent the pointer in the sequence as follows:
$$\rnode{m}{m} \cdot \ldots \cdot \rnode{n}{n} \bkptr[nodesep=1pt]{40}{n}{m}$$
\end{dfn}
Note that justified sequences are also defined for open terms:
occurrences of nodes in $N_{fv}$ must point to an occurrence of the
root of the computation tree.


A pointer is sometime labeled with an index $i$: if $m$ is a
$\lambda$-node then it indicates that $n$ is labelled with the $i$th
variable abstracted in $m$; otherwise it indicates that $n$ is the
$i$th child of $m$. A pointer in a justified sequence of nodes has
therefore one of the following forms: \vspace{2pt}
$$
\rnode{m}{r} \cdot \ldots \cdot \rnode{n}{z} \bkptr[nodesep=1pt]{40}{n}{m}
\hspace{1.5cm}
\rnode{m}{\lambda \overline{\xi}} \cdot \ldots \cdot \rnode{n}{\xi_i} \bkptr[nodesep=1pt]{40}{n}{m} \bklabel{i}
\hspace{1.5cm}
\rnode{m}{@ } \cdot \ldots \cdot \rnode{n}{\lambda \overline{\eta}} \bkptr[nodesep=1pt]{40}{n}{m} \bklabel{j}
\hspace{1.5cm}
\rnode{m}{\alpha } \cdot \ldots \cdot \rnode{n}{\lambda \overline{\eta}} \bkptr[nodesep=1pt]{40}{n}{m} \bklabel{k}
$$
where $r$ denotes the root of $\tau(M)$, $z \in N_{fv}$, $\xi_1,
\ldots \xi_n$ are bound variables, $\alpha \in N_{\Sigma} \union
N_{var}$, $i \in 1..n$, $j$ ranges from $0$ to the number of
children nodes of @ minus 1 and $k \in 1 ..arity(\alpha)$.

The following numbering conventions are used:
\begin{itemize}
\item the first child of a @-node is numbered $0$,
\item the first child of a variable or constant node is numbered $1$,
\item variables in $\overline{\xi}$ are numbered from $1$ onward ($\overline{\xi} = \xi_1 \ldots \xi_n$).
\end{itemize}
We use the notation $n.i$ to denote the $i$th child of node $n$.


We write $s = t$ to denote that the justified sequences $t$ and $s$
have same nodes \emph{and} pointers. Justified sequence of nodes can
be ordered using the prefix ordering: $t \sqsubseteq t'$ if and only
if $t=t'$ or the sequence of nodes $t$ is a finite prefix of $t'$
(and the pointers of $t$ are the same as the pointers of the
corresponding prefix of $t'$). Note that with this definition,
infinite justified sequences can also be compared. This ordering
gives rise to a complete partial order.

We say that a node $n_0$ of a justified sequence is hereditarily justified by $n_p$ if there are nodes $n_1, n_2, \ldots n_{p-1}$ in
the sequence such that for all $i\in 0..p-1$, $n_i$ points to $n_{i+1}$.

If $N$ is a set of nodes and $s$ a justified sequence of nodes then
we write $s \upharpoonright N$ to denote the subsequence of $s$
obtained by keeping only the nodes that are hereditarily
justified by nodes in $N$. This subsequence is also a justified
sequence of nodes. If $n$ denotes a node of $\tau(M)$ we
abbreviate $s \upharpoonright \{ n \}$ into $ s\upharpoonright n$.

\begin{lem}
\label{lem:filtercontinous}
For any set of node $N$, the filtering function $\_ \upharpoonright N$ defined on the cpo of justified sequences ordered by the prefix ordering
is continuous.
\end{lem}
\begin{proof}
Clearly $\_ \upharpoonright N$ is monotonous.
Suppose that $(t_i)_{i\in\omega}$ is a chain of justified sequence of nodes. Let $u$ be a finite prefix of $(\bigvee t_i) \upharpoonright r$.
Then $u = s \upharpoonright r$ for some finite prefix $s$ of $\bigvee t_i$. Since $s$ is finite we must have $s \sqsubseteq t_j$ for some $j\in\omega$.
Therefore $u \sqsubseteq t_j \upharpoonright r \sqsubseteq \bigvee (t_j \upharpoonright r)$.
This is valid for any finite prefix $u$ therefore $(\bigvee t_i) \upharpoonright r \sqsubseteq \bigvee (t_j \upharpoonright r)$.
\end{proof}


\begin{dfn}[P-view of justified sequence of nodes]
The P-view of a justified sequence of nodes $t$ of $\tau(M)$, written $\pview{t}$, is defined as follows:
\begin{eqnarray*}
 \pview{\epsilon} &=&  \epsilon \\
 \pview{s \cdot n }  &=&  \pview{s} \cdot n \\
 \pview{s \cdot \rnode{m}{m} \cdot \ldots \cdot \rnode{lmd}{\lambda \overline{\xi}}} &=& \pview{s} \cdot \rnode{m2}{m} \cdot \rnode{lmd2}{\lambda \overline{\xi}}
   \bkptr[nodesep=1pt]{30}{lmd}{m}
   \bkptr[nodesep=1pt]{60}{lmd2}{m2} \\
 \pview{s \cdot r }  &=&  r
\end{eqnarray*}
where $r$ is the root of the tree $\tau(M)$ and $n$ ranges over
non-lambda nodes (i.e. $N_\Sigma \union N_@ \union N_{var}$).

In the second clause, the pointer associated to $n$ is preserved
from the left-hand side to the right-hand side i.e. if in the
left-hand side, $n$ points to some node in $s$ that is also present
in $\pview{s}$ then in the right-hand side, $n$ points to this
occurrence of the node in $\pview{s}$.

Similarly, in the third clause the pointer associated to $m$ is preserved.
\end{dfn}

We also define O-view, the dual notion of P-view:
\begin{dfn}[O-view of justified sequence of nodes]
The O-view of a justified sequence of nodes $t$ of $\tau(M)$, written $\oview{t}$, is defined as follows:
\begin{eqnarray*}
 \oview{\epsilon} &=&  \epsilon \\
 \oview{s \cdot \lambda \overline{\xi} }  &=&  \oview{s} \cdot \lambda \overline{\xi} \\
 \oview{s \cdot \rnode{m}{m} \cdot \ldots \cdot \rnode{x}{x}} &=& \oview{s} \cdot \rnode{m2}{m} \cdot \rnode{n2}{x} \\
   \bkptr[nodesep=1pt]{30}{x}{m}
   \bkptr[nodesep=1pt]{60}{n2}{m2}
 \oview{s \cdot n }  &=&  n
\end{eqnarray*}
where $x$ ranges over variable nodes and  $n$ ranges over non-lambda
nodes without pointer (i.e. $N_@ \union N_\Sigma$).

The pointer associated to $\lambda \overline{\xi}$ in the second
equality and the pointer associated to $m$ in the third equality are
preserved from the left-hand side to the right-hand side of the
equalities.
\end{dfn}

\begin{dfn}[Alternation and Visibility] \ \\
A justified sequence of nodes $s$ satisfies:
\begin{itemize}
\item \emph{Alternation} if for any two consecutive nodes in $s$, one is a $\lambda$-node
and the other is not;

\item \emph{P-visibility} if every variable node in $s$ points to a node occurring in the P-view a that point;

\item  \emph{O-visibility} if every lambda node in $s$ points to a node occurring in the O-view a that point.
\end{itemize}
\end{dfn}

\begin{property}
\label{proper:pview_visibility}
The P-view (resp. O-view) of a justified sequence verifying P-visibility (resp. O-visibility)
is a well-formed justified sequence verifying P-visibility (resp. P-visibility).
\end{property}
This is proved by an easy induction.

\subsection{Adding value-leaves to the computation tree}
\label{sec:adding_value_leaves}

We now add leaves to the computation tree that has been defined in the previous section.
These leaves, called \emph{value-leaves}, are attached to the nodes of the computation tree. Each
value-leaf corresponds to a possible value of the base type $o$.
We write $\mathcal{D}$ to denote the set of values of the base type
$o$. The values leaves are added as follows: every  %$\lambda$-node or variable
node $n \in \tau(M)$ has a child leaf denoted by $v_n$ for each possible value $v \in \mathcal{D}$.

%@-nodes and $\Sigma$-nodes do not have child leaves.

%If $n$ is a $\lambda$-node then its value-leaves are numbered from $1$ onwards.
%If $n$ is a variable or constant node then its children nodes are numbered from $1$ to $arity(n)$ and
%its value-leaves are numbered from $arity(n)+1$ onwards.
%If $n$ is an application node then its value-leaves are numbered from $1$ onwards.

Everything that we have defined for computation tree can be lifted
to this new version of computation tree. The node order of a
value-leaf is defined to be $0$. The enabling relation $\vdash$ is
extended so that every leaf is enabled by its parent node. The
definition of justified sequence does not change.
When representing a link in a justified sequence going from a value-leaf $v_n$ to a node $n$,
we label the link with $v$:
$$
\rnode{n}{n} \cdot \ldots \cdot \rnode{vn}{v_n} \bkptr[nodesep=1pt]{40}{vn}{n} \bklabel{v}
$$


For the definition
of P-view, O-view and visibility, value-leaves are treated as
$\lambda$-nodes if they are at odd level in the computation tree and
as variable nodes if there at a even level.

From now the term ``computation tree'' refers to this extended
definition.
\vspace{10pt}

Let $n$ be a node of a justified sequence of nodes.
% that is either a $\lambda$-node or a variable node.
If there is an occurrence of a value-leaf $v_n$ in the sequence that points to $n$ we say that
$n$ is \emph{matched} by $v_n$. If there is no value-leaf in the sequence that points to $n$ we
say that $n$ is an \emph{unmatched node}.
The last unmatched node is called the \emph{pending node}.
A justified sequence of nodes is \emph{well-bracketed} if
each value-leaf in the traversal points to the pending node at that point.

If $t$ is a traversal then we write $?(t)$ to denote the subsequence
of $t$ consisting only of unmatched nodes.

\subsection{Traversal of the computation tree}
\label{subsec:traversal} We first define traversals for computation
tree of simply-typed $\lambda$-terms with no interpreted constants.
We will then we show how to extend the definition to the general
setting of $\lambda$-calculus augmented with interpreted constants.

\subsubsection{Traversals for simply-typed $\lambda$-terms}
Intuitively, a \emph{traversal} is a justified sequence of nodes of the computation tree where each node
indicates a step that is taken during the evaluation of the term.

\begin{dfn}[Traversals for pure simply-typed $\lambda$-terms]
\label{def:traversal}
In the simply-typed $\lambda$-calculus with no constants,
a traversal over a computation tree $\tau(M)$
is a justified sequence of nodes defined by induction on the rules
given below. A \emph{maximal-traversal} is a traversal that cannot be
extended by any rule. If $T$ denotes a computation tree then we write $\travset(T)$
to denote the set of traversals of $T$. We also use the abbreviation $\travset(M)$ for $\travset(\tau(M))$.

\emph{Initialization rules}
\begin{itemize}
\item ($\epsilon$) The empty sequence of node $\epsilon$ is a traversal.
\item (Root) The length 1 sequence $r$, where $r$ is denotes the root of $\tau(M)$, is a traversal.
\end{itemize}

\emph{Structural rules}
\begin{itemize}
\item (Lam) Suppose that $t \cdot \lambda \overline{\xi}$ is a traversal and $n$ is the only child node of $\lambda \overline{\xi}$ in
the computation tree then
$$t \cdot \lambda \overline{\xi} \cdot n$$
is also a traversal
where $n$ points to the (only) occurrence of its enabler in $\pview{t \cdot \lambda \overline{\xi}}$.
In particular, if $n$ is a free variable node then $n$ points to the first node of $t$.

\item (App) If $t \cdot @$ is a traversal then so is
$$t \cdot \rnode{m}{@} \cdot
\rnode{n}{n} \bkptr[nodesep=1pt]{60}{n}{m} \bklabelc{0}
$$

i.e. the next visited node is the $0$th child node of @ : the
node corresponding to the operator of the application.
\end{itemize}

\emph{Input-variable rules}
\begin{itemize}
\item (InputVar$^0$) If $t = t_1 \cdot x \cdot t_2$ is a traversal where
$x$ is the pending node in $t$ (i.e. $?(t_2)=\epsilon$)
and $x$ is a ground-type input-variable then for any $v \in \mathcal{D}$ the following is a traversal
$$t_1 \cdot \rnode{x}{x} \cdot t_2 \cdot \rnode{xv}{v_x}
\bkptr[nodesep=1pt]{40}{xv}{x} \bklabelc{v}$$


\item (InputVar$^{\geq 1}$)
If $t = t_1 \cdot x \cdot t_2$ is a traversal where
$x$ is the pending node in $t$ (i.e. $?(t_2)=\epsilon$)
and $x$ is a higher-order input-variable then the following is a traversal:
$$t_1 \cdot \rnode{m}{x} \cdot t_2 \cdot
\rnode{n}{n} \bkptr[nodesep=1pt]{40}{n}{m} \bklabelc{i} \qquad
\mbox{ for } 1 \leq i \leq arity(x).$$
Moreover for any $v\in \mathcal{D}$ the sequence $t_1 \cdot \rnode{x}{x} \cdot t_2 \cdot
\rnode{xv}{v_x} \bkptr[nodesep=1pt]{40}{xv}{x} \bklabelc{v}$ is also a traversal.
\end{itemize}

\emph{Copy-cat rules}
\begin{itemize}
  \item (CCAnswer-@)
%  If $t \cdot \lambda \overline{\xi} \cdot \rnode{app}{@} \cdot \rnode{lz}{\lambda \overline{z}} \cdot \ldots \cdot  \rnode{lzv}{v_{\lambda \overline{z}}}
%              \bkptr[nodesep=1pt]{30}{lzv}{lz} \bklabelc{v}
%              \bkptr[nodesep=1pt]{40}{lz}{app} \bklabelc{0}$
%              is a traversal then so is:
%              $t \cdot \rnode{lmd}{\lambda \overline{\xi}} \cdot \rnode{app}{@} \cdot \rnode{lz}{\lambda \overline{z}} \cdot \ldots \cdot \rnode{lzv}{v_{\lambda \overline{z}}} \cdot
%              \rnode{lmdv}{v_{\lambda \overline{\xi}}}
%              \bkptr[nodesep=1pt]{30}{lzv}{lz} \bklabelc{v}
%              \bkptr[nodesep=1pt]{40}{lz}{app} \bklabelc{0}
%                \bkptr[nodesep=1pt]{30}{lmdv}{lmd} \bklabelc{v}$.
  If $t \cdot \rnode{app}{@} \cdot \rnode{lz}{\lambda \overline{z}} \cdot \ldots \cdot \rnode{lzv}{v_{\lambda \overline{z}}}
              \bkptr[nodesep=1pt]{30}{lzv}{lz} \bklabelc{v}
              \bkptr[nodesep=1pt]{40}{lz}{app} \bklabelc{0}$
              is a traversal then so is:
              $t \cdot \rnode{app}{@} \cdot \rnode{lz}{\lambda \overline{z}} \cdot \ldots \cdot \rnode{lzv}{v_{\lambda \overline{z}}} \cdot \rnode{appv}{v_@}
              \bkptr[nodesep=1pt]{30}{lzv}{lz} \bklabelc{v}
              \bkptr[nodesep=1pt]{40}{lz}{app} \bklabelc{0}
              \bkptr[nodesep=1pt]{30}{appv}{app} \bklabelc{v}$.


  \item (CCAnswer-$\lambda$) If $t \cdot \lambda \overline{\xi} \cdot \rnode{x}{x} \cdot \ldots \cdot  \rnode{xv}{v_x}
              \bkptr[nodesep=1pt]{30}{xv}{x} \bklabelc{v}$
              is a traversal then so is:
              $t \cdot \rnode{lmd}{\lambda \overline{\xi}} \cdot \rnode{x}{x} \cdot \ldots \cdot \rnode{xv}{v_x} \cdot
              \rnode{lmdv}{v_{\lambda \overline{\xi}}}
              \bkptr[nodesep=1pt]{20}{xv}{x} \bklabelc{v}
                \bkptr[nodesep=1pt]{20}{lmdv}{lmd} \bklabelc{v}$.

     \item (CCAnswer-var) If $t \cdot y \cdot \rnode{lmd}{\lambda \overline{\xi}}
                   \cdot \ldots
                   \cdot \rnode{lmdv}{v_{\lambda \overline{\xi}}} \bkptr[nodesep=1pt]{30}{lmdv}{lmd} \bklabelc{v}$ is a traversal,
                   where $y$ is a non input-variable node, then the following is also a traversal:
        $$t \cdot \rnode{y}{y}
            \cdot \rnode{lmd}{\lambda \overline{\xi}}
            \cdot \ldots
            \cdot \rnode{lmdv}{v_{\lambda \overline{\xi}}}
            \cdot \rnode{yv}{v_y}
                \bkptr[nodesep=3pt]{35}{yv}{y} \bklabelc{v}
                \bkptr[nodesep=1pt]{30}{lmdv}{lmd} \bklabelc{v}.$$


\item (Var)
If $t \cdot x_i$ is a traversal where $x_i$ is not an input-variable,
then the rule (Var) permits to visit the node corresponding to the subterm that would be substituted
for $x_i$ if all the $\beta$-redexes occurring in $M$ were reduced.

The binding node $\lambda \overline{x}$ necessarily occur previously
in the traversal. Since $x$ is not hereditarily justified by the
root, $\lambda \overline{x}$ is not the root of the tree and
therefore its justifier $p$ - which is also its parent node - occurs
immediately before itself it in the traversal. We do a case analysis
on $p$:

    \begin{itemize}
    \item Suppose $p$ is an @-node then $\lambda \overline{x}$ is necessarily the first child node of $p$
    and $p$ has exactly $|\overline{x}| + 1$ children nodes:
    $$\pstree[levelsep=7ex]{\TR{\stackrel{\vdots}{@^{[p]}}}}
    {   \pstree[linestyle=dotted,levelsep=4ex]{\TR{\lambda \overline{x}}\treelabel{0}}
            {\TR{x_i }}
        \tree{\lambda \overline{\eta_1}}{\vdots}\treelabel{1}
        \TR[edge=\dotedge]{}
        \tree{\lambda \overline{\eta_i}}{\vdots}\treelabel{i}
        \TR[edge=\dotedge]{}
        \tree{\lambda \overline{\eta_{|x|}}}{\vdots}\treelabel{|x|}
    }
    $$
    In that case, the next step of the traversal is a jump to $\lambda \overline{\eta_i}$ -- the $i$th child of
    @ -- which corresponds to the subterm that would be substituted for $x_i$ if the $\beta$-reduction was
    performed:
    \vspace{0.3cm}
    $$t' \cdot \rnode{n}{@^{[p]}} \cdot
    \rnode{lx}{\lambda \overline{x}} \cdot \ldots \cdot
    \rnode{x}{x_i} \cdot
    \rnode{mi}{\lambda \overline{\eta_i}} \cdot \ldots
    \bkptr[ncurv=0.45]{45}{mi}{n} \bklabel{i}
    \bkptr[ncurv=0.6]{50}{x}{lx} \bklabel{i} \in \travset(M)
    $$

    \item Suppose $p$ is variable node $y$, then
    necessarily the node $\lambda \overline{x}$ has been added to the traversal $t_{\leq y}$ using the (Var) rule
    (this is proved in proposition \ref{prop:pviewtrav_is_path}(i)).
    Therefore $y$ is substituted by the term $\kappa(\lambda \overline{x})$ during the evaluation of the term
    and we have $\ord{y} = \ord{\lambda \overline{x}}$.

    Consequently, during reduction, the variable $x_i$ is substituted by the subterm represented by
    $\lambda \overline{\eta_i}$ -- the $i$th child node of $y$.
    Hence the following justified sequence is also a traversal:
    \vspace{0.2cm}
    $$t' \cdot \rnode{n}{y^{[n]}} \cdot
    \rnode{lx}{\lambda \overline{x}} \cdot \ldots \cdot
    \rnode{x}{x_i} \cdot
    \rnode{mi}{\lambda \overline{\eta_i}} \cdot \ldots
    \bkptr[ncurv=0.6]{50}{x}{lx} \bklabel{i}
    \bkptr[ncurv=0.5]{50}{mi}{n} \bklabel{i}$$
    \end{itemize}
\end{itemize}
Note that a traversal always starts with the root of the tree.
\end{dfn}

\begin{rem}
Our notions of computation tree and traversal differ slightly from
\cite{OngLics2006}.

Firstly, our computation tree do not have nodes labelled with
(uninterpreted) first-order constants. On the other hand, there are
nodes which are labelled by free variables of any order. Since
uninterpreted constants can be regarded as free variables, we do not
lose any expressivity. The traversal rules (InputVar$^0$) and
(InputVar$^\geq 1$) provide a more general version of the (Sig) rule
of \cite{OngLics2006}.

Secondly we have introduced copy-cat rules that permit to visit the
value-leaves of the computation tree. The presence of value-leaves
is necessary to model free variables as well as the interpreted
constants present in extensions of the $\lambda$-calculus such as
\pcf\ or \ialgol.
\end{rem}

\begin{exmp}
Consider the following computation tree:
$$\tree{\lambda}
{
    \tree{@}
    {
        \pstree[levelsep=8ex,linestyle=dotted]{\TR{\lambda y}\treelabel{0} }
        {
            \pstree[levelsep=8ex]{\TR{y}}
            {
                \tree{\lambda \overline{\eta_1}}{\vdots} \treelabel{1}
                \TR[edge=\dotedge]{}
                \tree{\lambda \overline{\eta_i}}{\vdots}\treelabel{i}
                \TR[edge=\dotedge]{}
                \tree{\lambda \overline{\eta_n}}{\vdots}\treelabel{n}
            }
        }
        \pstree[levelsep=6ex,linestyle=dotted]{\TR{\lambda \overline{x}}\treelabel{1}}{ \tree{x_i}{\TR{} \TR{} } }
    }
}
$$
An example of traversal of this tree is:
\vspace{0.3cm}
$$ \lambda \cdot
\rnode{app}{@}  \cdot
\rnode{ly}{\lambda y} \cdot \ldots \cdot
\rnode{y}{y} \cdot
\rnode{lx}{\lambda \overline{x}} \cdot \ldots \cdot
\rnode{x}{x_i} \cdot
\rnode{leta}{\lambda \overline{\eta_i} } \cdot \ldots
\bkptr[ncurv=0.6,nodesep=0]{40}{x}{lx}  \bklabel{i}
\bkptr[ncurv=0.5]{50}{leta}{y}  \bklabel{i}
\bkptr[ncurv=0.6,nodesep=0]{40}{y}{ly}  \bklabel{1}
\bkptr[ncurv=0.5]{50}{lx}{app}  \bklabel{1}$$
\end{exmp}

\subsubsection{Traversals for interpreted constants}

\begin{dfn}[Well-behaved traversal rule]
\label{def:wellbehaved_traversal}
A traversal rule is \emph{well-behaved} if it can be stated under the following form:
$$\rulef{t = t_1\cdot n \cdot t_2 \in \travset \quad ?(t_2) = \epsilon \quad P(t)}
  { \stackrel{  \rule{0pt}{3pt} }{t' = t_1\cdot \rnode{n}{n} \cdot t_2 \cdot \rnode{m}{m} \in \travset} }
   \bkptr[nodesep=1pt]{35}{m}{n}
    \ m\in S(t)
   $$
such that:
\begin{enumerate}
  \item $n$ is a variable or a constant node;
  \item $P$ expresses some condition on $t$;
  \item $S(t)$ is some subset of $E(n)$, the set of children $\lambda$-nodes and value-leaves of $n$.
  If $S(t)$ has more than one element then the rule is non-deterministic.
\end{enumerate}
\end{dfn}
Note that if $t$ is well-bracketed then $t'$ is also well-bracketed
and if $?(t)$ satisfies alternation (resp. visibility) then so does $?(t')$.


The rules (InputVar$^0$) and (InputVar$^{\geq1}$) are two examples of
non-deterministic well-behaved traversal rules for which
$S(t)$ is exactly the set of all children-nodes and value-leaves of $n$:
$S(t) = \{ n.i \ |\ i \in 1..arity(n) \} \union  \{ v_n \ | \ v \in \mathcal{D} \} $.


In the presence of higher-order interpreted constants, additional rules must be specified to indicate how
the constant nodes should be traversed in the computation tree. These rules
are specific to the language that is being studied.
In the last section of this chapter we will define such traversals for the interpreted constants of
\pcf\ and \ialgol.

From now on, we consider a simply-typed $\lambda$-calculus language extended with
higher-order interpreted constants for which some constant traversal rules have been defined
and we take the following condition as a prerequisite:
\begin{center}
  \textbf{Condition (WB) :} the constant traversal rules are well-behaved.
\end{center}


\subsubsection{Some properties of traversals}

\begin{prop}
\label{prop:pviewtrav_is_path}
Let $t$ be a traversal. Then:
\begin{itemize}
\item[(i)] $t$ is a well-defined and well-bracketed justified sequence;
\item[(ii)] $?(t)$ is a well-defined justified sequence verifying alternation, P-visibility and O-visibility;
\item[(iii)] $\pview{?(t)}$ is a path in the computation tree going from the root to the last node in $?(t)$.
\end{itemize}
\end{prop}
This is the counterpart of proposition 6 from
\cite{OngHoMchecking2006} which is proved by induction on the
traversal rules. This proof can be easily adapted to take into
account the constant rules (using the assumption that constants
rules are well-behaved) and the presence of value-leaves in the
traversal.
\begin{proof}
We just give a partial proof of (i). We prove that in the second case of the (Var) rule, where $p$ is a variable node $y$,
the node $\lambda \overline{x}$ has necessarily been added to the traversal $t_{\leq y}$ using the (Var) rule.

Suppose that an (InputVar) rule is used to produce $t_{<y} \cdot y
\cdot \lambda \overline{x}$, then $y$ is an input-variable node and
also the parent node of $\lambda \overline{x}$. But $x$ is not an
input-variable therefore it cannot be the child node of an
input-variable. Hence (Var) is the only rule which can be used to
produce $t_{<y} \cdot y \cdot \lambda \overline{x}$.
\end{proof}

%In particular to prove that the copy-cat rules are well-defined, one needs to ensure that
%if the last two unmatched nodes are $y$ and $\lambda \overline{\xi}$ in that order, for some non input-variable node $y$ then necessary
%      $y$ and $\lambda \overline{\xi}$ are consecutive nodes in the traversal.
%    This is because in a traversal, a non input-variable $y$ is always followed by a lambda node and whenever this lambda node is answered
%    there is only one way to extend the traversal : by using the copy cat rule to answer the $y$ node.


\begin{dfn}[Traversal reduction]
Let $r$ be the root of the computation tree. We say that the
justified sequence of nodes \emph{$s$ is a reduction of the
traversal $t$} just when $s = t \upharpoonright r$.
\end{dfn}

Since @-nodes and $\Sigma$-constants do not have pointer, the
reduction of traversal contains only nodes in $N_\lambda \union
N_{var}$.


\begin{lem}
\label{lem:var_followedby_child} Let $M$ be a term in $\beta$-normal
form and $t$ be a traversal of $\tau(M)$. If $?(t) = u_1 \cdot
\rnode{m}{m} \cdot u_2 \cdot \rnode{n}{n}
\bkptr[nodesep=1pt]{20}{n}{m}$ where $m$ is a not a $\lambda$-node
then $u_2 = \epsilon$.
\end{lem}
\begin{proof}
By induction on the traversal rules. The only relevant rules are (Var), (CCAnswer-var), (InputVar$^0$), (InputVar$^{\geq 1}$)
and the constant rules.
Since the term is in $\beta$-normal form, there is no @-node in $\tau(M)$ and therefore (Var) cannot be used.
For the rules (CCAnswer-var), (InputVar$^0$) and (InputVar$^{\geq 1})$ we just use the well-bracketedness of traversals.
For the constant rules, the result is a consequence of condition (WB) stating that constant rules are the well-behaved.
\end{proof}

\begin{lem}[View of a traversal reduction]
\label{lem:redtrav_trav} Let $M$ be a term in $\beta$-normal form,
$r$ be the root of $\tau(M)$ and $t$ be a traversal of $\tau(M)$. We
have
\begin{itemize}
\item[(i)] $ \pview{?(t) \upharpoonright  r } = \pview{?(t)} \upharpoonright r$;
\item[(ii)] if the last node in $t$ is hereditarily justified by $r$ then $ \oview{?(t) \upharpoonright r } = \oview{?(t)}$.
\end{itemize}
\end{lem}

\begin{proof}
(i) By induction. Base case: it is trivially true for the empty
traversal $t = \epsilon$. Step case: consider a traversal $t$ and
suppose that the property (i) is verified for all traversal smaller
than $t$. There are three cases:
\begin{itemize}
\item Suppose $?(t) = t' \cdot r$ then:
    \begin{align*}
    \pview{?(t)} \upharpoonright  r
        &=  \pview{t' \cdot r } \upharpoonright  r       & (\mbox{definition of } ?(t))\\
        &=  r \upharpoonright  r                         & (\mbox{def. P-view})\\
        &=  r                                                & (\mbox{def. operator $\upharpoonright$})\\
        &=  \pview{(t' \upharpoonright  r ) \cdot r }    & (\mbox{def. P-view})\\
        &=  \pview{(t' \cdot r)  \upharpoonright  r }    & (\mbox{def. operator $\upharpoonright$})\\
        &= \pview{?(t) \upharpoonright  r }                & (\mbox{definition of } ?(t))
    \end{align*}

\item Suppose $?(t) = t' \cdot n$ where $n$ is a non-lambda
move. We have:
    \begin{equation}
    \pview{?(t)} = \pview{t' \cdot n} = \pview{t'} \cdot n  \label{eq_tprime}
    \end{equation}
    \begin{itemize}
    \item If $n$ is not hereditarily justified by $r$ then:
    \begin{align*}
    \pview{?(t)} \upharpoonright  r
        &= (\pview{t'} \cdot n) \upharpoonright  r  & (\mbox{equation \ref{eq_tprime}}) \\
        &= \pview{t'} \upharpoonright  r            & (n \mbox{ is not hereditarily justified by } r) \\
        &= \pview{t' \upharpoonright  r }           & (\mbox{induction hypothesis}) \\
        &= \pview{(t' \cdot n) \upharpoonright  r } & (n \mbox{ is not hereditarily justified by } r) \\
        &= \pview{?(t) \upharpoonright  r  }           & (\mbox{definition of } ?(t))
    \end{align*}

    \item If $n$ is hereditarily justified by $r$ then:
    \begin{align*}
    \pview{?(t)} \upharpoonright  r
    &= (\pview{t'} \cdot n) \upharpoonright  r      & (\mbox{equation \ref{eq_tprime}}) \\
    &= (\pview{t'} \upharpoonright  r  ) \cdot n    & (n \mbox{ is hereditarily justified by } r)\\
    &= \pview{t' \upharpoonright  r } \cdot n       & (\mbox{induction hypothesis}) \\
    &= \pview{(t' \upharpoonright  r ) \cdot n }    & (\mbox{P-view computation}) \\
    &= \pview{(t' \cdot n) \upharpoonright  r  }    & (n \mbox{ is hereditarily justified by } r) \\
    &= \pview{?(t) \upharpoonright  r  }               & (\mbox{definition of } ?(t))
    \end{align*}
    \end{itemize}


\item Suppose that $?(t) =  t' \cdot \rnode{m}{m} \cdot  u \cdot \rnode{lmd}{n}
    \bkptr[nodesep=1pt]{30}{lmd}{m}$ where $n$ is a $\lambda$-node then by lemma
    \ref{lem:var_followedby_child} we have $u = \epsilon$ and therefore:
        \begin{align*}
        \pview{?(t)} \upharpoonright  r
        &= \pview{t' \cdot \rnode{m}{m} \cdot \rnode{n}{n}} \upharpoonright  r
               \bkptr[nodesep=1pt]{60}{n}{m}                   & (u=\epsilon)\\
        &= (\pview{t'} \cdot \rnode{m}{m} \cdot \rnode{lmd}{n} ) \upharpoonright  r
               \bkptr[nodesep=1pt]{60}{lmd}{m}                 & (\mbox{P-view computation}) \\
        &= \pview{t'} \upharpoonright  r                & (m, n \mbox{ are not hereditarily justified by } r) \\
        &= \pview{t' \upharpoonright  r }               & \mbox{(induction hypothesis)} \\
        &= \pview{ (t' \cdot \rnode{m}{m} \cdot \rnode{lmd}{n}) \upharpoonright r }
                        \bkptr[nodesep=1pt,ncurv=0.7]{40}{lmd}{m}
                                                            & (\mbox{def. operator $\upharpoonright$ and } m, \lambda \mbox{ are not her. just. by } r) \\
        &= \pview{ ?(t) \upharpoonright r }                & \mbox{(def. of $?(t)$)}
        \end{align*}
\end{itemize}
(ii) By a straightforward induction similar to (i).
\end{proof}

\begin{lem}[Traversal of $\beta$-normal terms]
\label{lem:betaeta_trav}
Let $M$ be a $\beta$-normal term, $r$ be the root of the tree $\tau(M)$ and
$t$ be a traversal of $\tau(M)$.
For any node $n$ occurring in $t$:
\begin{eqnarray*}
r \mbox{ does not hereditarily justify } n  \  \iff \   n \mbox{ is
hereditarily justified by some node in } N_\Sigma.
\end{eqnarray*}
%\begin{itemize}
%\item[(i)]
%for any node $n$ occurring in $t$:
%\begin{eqnarray*}
%r \mbox{ does not hereditarily justify } n  \  \iff \   n \mbox{
%is hereditarily justified by some node in } N_\Sigma;
%\end{eqnarray*}
%\item[(ii)] For any $\lambda$-node $n$ occurring in $t$, $t_{\geq n} \in \travset(\kappa(n))$,
%
% where $t_{\geq n}$ denotes the justified sequence of nodes obtained by taking the suffix of $t$ starting at $n$ and
% such that any dangling link going from a variable node to a node preceding $n$ is ``fixed'' into a pointer going to $n$.
% \end{itemize}
\end{lem}
\begin{proof}
%(i)
 In a computation tree, the only nodes that do not have justification pointer are:
the root $r$, @-nodes and $\Sigma$-constant nodes. But since $M$ is
in $\beta$-normal form, there is no @-node in the computation tree.
Hence nodes are either hereditarily justified by $r$ or hereditarily
justified by a node in $N_\Sigma$. Moreover $r$ is not in $N_\Sigma$
therefore the ``or'' is exclusive : a node cannot be hereditarily
justified at the same time by $r$ and by some node in $N_\Sigma$.

%(ii) Since $M$ is in $\beta$-normal, the rules (App) and (Var) cannot be used. Therefore the traversals
%follow an inductive exploration of the computation tree without making any ``jump''.
%Consequently, by taking the prefix of $t$ starting at a $\lambda$-node, we obtain
%a traversal of a sub-computation tree of $\tau(M)$. However by taking the prefix we obtain some dangling pointers.
%The ``fix'' applied to the dangling pointers correspond to the
%The formal proof is by an easy induction on the traversal rules. For the constant rules, we appeal to well-behaviour of the rules.

\end{proof}


\section{Game semantics of simply-typed $\lambda$-calculus with $\Sigma$-constants}
\label{sec:assumptions}

We are working in the general setting of an applied simply-typed $\lambda$-calculus with a given set of higher-order constants $\Sigma$.
The operational semantics of these constants is given by certain reduction rules.
We assume that a fully abstract model of the calculus is provided by mean of a category of well-bracketed games.
For instance, if $\Sigma$ is the set of \pcf\ constants then we work in the category $\mathcal{C}_{b}$
of well-bracketed defined in section \ref{subsec:pcfgamemodel} of the first chapter.

We will use the alternative representation of strategy defined in remark \ref{rem:atlern_strategy}: a
strategy is given by a prefix-closed set instead of an ``even length
prefix''-closed set. In practice this means that we replace the set
of plays $\sigma$ by $\sigma \union \textsf{dom}(\sigma)$. This
permits to avoid considerations on the parity of the length of
traversals when we show the correspondence between traversals and
game semantics. We write $\sem{\Gamma \vdash M : A}$ for the strategy denoting the simply-typed term
$\Gamma \vdash M : A$ and $\prefset(S)$ to denote the
prefix-closure of the set $S$.


\subsection{Relationship between computation trees and arenas}

\subsubsection{Example}
Consider the following term $M \equiv \lambda f z . (\lambda g x . f (f x)) (\lambda y. y) z$ of type $(o \typear o) \typear o \typear o$.
Its $\eta$-long normal form is $\lambda f z . (\lambda g x . f (f x)) (\lambda y. y) (\lambda .z)$.
The computation tree is:

$$
\tree{\lambda f z}
{ \tree{@}
    {
        \tree{\lambda g x}
            { \tree{f}{   \tree{\lambda}{ \tree{f}{  \tree{\lambda}{\TR{x}}} }  }
            }
        \tree{\lambda y}{\TR{y}}
        \tree{\lambda}{\TR{z}}
    }
}
$$

The arena for the type $(o \typear o) \typear o \typear o$ is:
$$\tree{q^1}
{
    \tree{q^3}
        {  \tree{q^4}
                {\TR{a^4_1} \TR{\ldots}}
            \TR{a^3_1} \TR{\ldots} }
    \tree{q^2}
    { \TR{a^2_1} \TR{a^2_2}\TR{\ldots} }
    \TR{a_1} \TR{a_2}\TR{\ldots}
}
$$

\newlength{\yNull}
\def\bow{\quad\psarc{->}(0,\yNull){1.5ex}{90}{270}}

The figure below represents the computation tree (left) and the
arena (right). The dashed line defines a partial function $\varphi$
from the set of nodes in the computation tree to the set of moves.
For simplicity, we now omit answers moves when representing arenas.
$$
\tree{ \Rnode{root} {\lambda f z}^{[1]} }
     {  \tree{@^{[2]}}
        {   \tree{\lambda g x ^{[3]}}
                { \tree{\Rnode{f}{f^{[6]}}}{  \tree{\Rnode{lmd}\lambda^{[7]}}{ \tree{\Rnode{f2}{f^{[8]}}} {\tree{\Rnode{lmd2}\lambda^{[9]}}{\TR{x^{[10]}}}}}  }
                }
            \tree{\lambda y ^{[4]}}{\TR{y}}
            \tree{\lambda ^{[5]}}{\TR{\Rnode{z}z}}
        }
    }
\hspace{3cm}
  \tree[levelsep=12ex]{ \Rnode{q1}q^1 }
    {   \pstree[levelsep=4ex]{\TR{\Rnode{q3}q^3}}{\TR{\Rnode{q4}q^4}}
        \TR{\Rnode{q2}q^2}
        \TR{\Rnode{q5}q^5}
    }
\psset{nodesep=1pt,arrows=->,arcangle=-20,arrowsize=2pt 1,linestyle=dashed,linewidth=0.3pt}
\ncline{->}{root}{q1} \aput*{:U}{\varphi}
\ncarc{->}{z}{q2}
\ncline{->}{f}{q3}
\ncline{->}{lmd}{q4}
\ncline{->}{f2}{q3}
\ncline{->}{lmd2}{q4}
$$

Consider the justified sequence of moves $s \in \sem{M}$:
\vspace{0.2cm}
 $$s =
\rnode{q1}{q}^1\ \rnode{q3}{q}^3\ \rnode{q4}{q}^4\ \rnode{q3b}{q}^3\ \rnode{q4b}{q}^4\ \rnode{q2}{q}^2
\bkptr[offset=-3pt]{60}{q3}{q1}
\bkptr[offset=-3pt,ncurv=0.5]{60}{q3b}{q1}
\bkptr[offset=-3pt]{60}{q4}{q3}
\bkptr[offset=-3pt]{60}{q4b}{q3b}
\bkptr[offset=-3pt,ncurv=0.5]{60}{q2}{q1}
\in \sem{M}$$

There is a corresponding justified sequence of nodes in the computation tree:
\vspace{0.5cm}
$$r =
\rnode{q1}{\lambda f z} \cdot
\rnode{q3}{f}^{[6]} \cdot
\rnode{q4}{\lambda^{[7]}} \cdot
\rnode{q3b}{f}^{[8]} \cdot
\rnode{q4b}{\lambda^{[9]}} \cdot
\rnode{q2}{z}
\bkptr[ncurv=1]{60}{q3}{q1}
\bkptr[ncurv=1]{60}{q4}{q3}
\bkptr[ncurv=0.5]{75}{q3b}{q1}
\bkptr[ncurv=1]{50}{q4b}{q3b}
\bkptr[ncurv=0.4]{80}{q2}{q1}$$
such that $s_i = \varphi(r_i)$ for all $i < |s|$.

The sequence $r$ is in fact the reduction of the following
traversal: \vspace*{1cm}
$$t = \rnode{q1}{\lambda f
z} \cdot \rnode{n2}{@^{[2]}} \cdot \rnode{n3}{\lambda g x^{[3]}}
\cdot \rnode{q3}{f}^{[6]} \cdot \rnode{q4}{\lambda^{[7]}} \cdot
\rnode{q3b}{f}^{[8]} \cdot \rnode{q4b}{\lambda^{[9]}} \cdot
\rnode{n8}{x^{[10]}} \cdot \rnode{n9}{\lambda^{[5]}} \cdot
\rnode{q2}{z} \bkptr[ncurv=0.6]{60}{q3}{q1}
\bkptr[ncurv=1]{60}{q4}{q3} \bkptr[ncurv=0.4]{75}{q3b}{q1}
\bkptr[ncurv=0.8]{70}{q4b}{q3b} \bkptr[ncurv=0.4]{80}{q2}{q1}
\bkptr[ncurv=0.4]{60}{n3}{n2} \bkptr[ncurv=0.4]{60}{n8}{n3}
\bkptr[ncurv=0.4]{60}{n9}{n2}.
$$

By representing side-by-side the computation tree and the type arena of a term in $\eta$-normal form we have observed
that some nodes of the computation tree can be mapped to question moves of the arena.
In the next section, we show how to define this mapping in a systematic manner.

\subsubsection{Formal definition}

Let us establish precisely the relationship between arenas of the
game semantics and the computation trees. Let $\Gamma \vdash M : A$
be a term in $\eta$-long normal form. The computation tree $\tau(M)$
is represented by a pair $(V,E)$ where $V$ is the set of vertices of
the trees and $E$ is the edges relation. $V = N \union L$ where $N$
is the set of nodes and $L$ is the set of value-leaves.

The relation $E \subseteq V \times V$ gives the parent-child relation on the vertices of the tree.
We write $V_\$$ for $N_\$ \union (E(N_\$) \inter L)$ where $\$$ ranges over $\{@, var, \Sigma, fv \}$.


Let $\mathcal{D}$ be the set of values of the base type $o$. If $n$
is a node in $N$ then the value-leaves in
$E_l(n)$ attached to the node $n$ are written $v_n$ where $v$ ranges in $\mathcal{D}$.
Similarly, if $q$ is a question in $\sem{A}$ then the answer moves
enabled by $q$ are written $v_q$ where $v$ ranges in $\mathcal{D}$.

If $A$ is an arena and $q$ is a move in $A$ then we write $A_q$ to
denote the subarena of $A$ rooted at $q$.

\begin{dfn}[Relation between moves of the arena and nodes of the computation tree]
\label{def:phi_procedure}
We consider the computation tree of a simply-typed-term.
For any arena $A$, we define a function $f_A(n,q)$ taking two parameters:
a node $n$ of the computation tree and a question move $q$ of the arena $A$
such that $q$ and $n$ have the same type.
$f_A(n,q)$ returns a partial function from $V$ to $A$. It is defined as follows:
\noindent
\begin{itemize}
\item[case 1] If $n$ is labelled labelled $\lambda$ or is a ground type variable node then
        $$f_A(n,q) = \{ n \mapsto q \} \quad \union \quad  \{ v_n \mapsto v_q \ | \ v \in \mathcal{D} \}$$

\item[case 2]  If $n$ is a $\lambda$-node labelled $\lambda \overline{\xi} = \lambda \xi_1 \ldots \xi_p$ with $p\geq 1$ and with a child node labelled $\alpha$ and $\vdash( q ) = \{ q^1, \ldots, q^m \} \union \{  v_q : v \in \mathcal{D} \} $ then the computation tree and the arena $A_q$ have the following form
    (value-leaves and answer moves are not represented for simplicity):
    $$ \tree{ \Rnode{r}\lambda \overline{\xi}  ^{[n]}}
        {
            \tree[levelsep=6ex]{\alpha}
            {   \TR{\ldots} \TR{\ldots} \TR{\ldots}
            }
        }
    \hspace{3cm}
    \tree{ \Rnode{q0}q }
        {
            \tree[linestyle=dotted]{q^1}{\TR{} \TR{} }
            \tree[linestyle=dotted]{q^2}{\TR{} \TR{} }
            \TR{\ldots}
            \tree[linestyle=dotted]{q^p}{\TR{} \TR{} }
        }
    \psset{nodesep=1pt,arrows=->,arcangle=-20,arrowsize=2pt 1,linestyle=dashed,linewidth=0.3pt}
    \ncline{->}{r}{q0}
    \ncarc{->}{q2}{z}
    \ncline{->}{q3}{f}
    \ncline{->}{q4}{lmd}
    \ncline{->}{q3}{f2}
    \ncline{->}{q4}{lmd2}
    $$

    For each of the abstracted variable $\xi_i$ there exists a corresponding question move $q^i$ of the same order
    in the arena.  $f_A(n,q)$ maps each free occurrence of a variable $\xi_i$ to the corresponding move $q^i$:
    $$
    f_A(n,q) =  \{ n \mapsto q \} \quad  \union \quad  \{ v_n \mapsto v_q \ | \ v \in \mathcal{D} \}
                      \quad \union \quad  \Union_{\stackrel{\displaystyle m \in N | n \vdash m}{\displaystyle m \mbox{ labelled } \xi_i}} f_A( m, q^i)
    $$

\item[case 3] If $n$ is a variable node $x$ of higher-order type $(A_1,\ldots,A_m,o)$
with children nodes $\lambda \overline{\eta}_1$, \ldots, $\lambda \overline{\eta}_m$ and
$\vdash( q ) = \{ q^1, \ldots, q^m \} \union \{  v_q : v \in \mathcal{D} \} $ then the computation tree and the arena $A_q$ have the following form:
    $$\tree{\Rnode{r}{x^{[n]}}}
        {   \tree{\TR{\lambda \overline{\eta}_1}}{\vdots} \TR{\ldots}
        \tree{\TR{\lambda \overline{\eta}_m }}{\vdots}
        }
    \hspace{3cm}
    \tree{ \Rnode{q0}q }
        {
            \tree[linestyle=dotted]{\Rnode{q1}{q^1}}{\TR{} \TR{} }
            \tree[linestyle=dotted]{\Rnode{q2}{q^2}}{\TR{} \TR{} }
            \TR{\ldots}
            \tree[linestyle=dotted]{\Rnode{qm}{q^m}}{\TR{} \TR{} }
        }
    \psset{nodesep=1pt,arrows=->,arcangle=-20,arrowsize=2pt 1,linestyle=dashed,linewidth=0.3pt}
    \ncline{->}{r}{q0}
    \ncarc{->}{q2}{z}
    \ncline{->}{q3}{f}
    \ncline{->}{q4}{lmd}
    \ncline{->}{q3}{f2}
    \ncline{->}{q4}{lmd2}
    $$

    $f_A(n,q)$ maps each child node of $n$ to the corresponding question move $q^i$ of the same type
    in the arena $A_q$:
    $$f_A(n,q) =
         \{ n \mapsto q \} \quad \union\quad \{ v_n \mapsto v_q \ | \ v \in \mathcal{D}   \} \quad\union\quad     \Union_{i=1..m} f_A( \lambda \overline{\eta}_i, q^i)
    $$
\end{itemize}

Note that $f_A(n,q)$ is only a partial function from $V$ to $A$ since it is defined only
on nodes that are hereditarily justified by the root \emph{and} not hereditarily justified by a free variable node.
In other words, $f_A(n,q)$ is undefined on nodes that are hereditarily justified by $N_{fv} \union N_@ \union N_\Sigma$.
\end{dfn}

We write $\mathcal{M}_M$ to denote the following disjoint union of arenas:
$$\mathcal{M}_M = \sem{\Gamma \rightarrow T} \quad \uplus \quad  \biguplus_{n \in N \inter E\relimg{N_@ \union N_\Sigma} } \sem{type(\kappa(n))}.$$

Moves in $\mathcal{M}_M$ are implicitly tagged so it is possible to recover the arena in which they belong.


\begin{dfn}[Total mapping from nodes to moves]
Let $\Gamma \vdash M : T$ be a simply-typed term
with $\Gamma = x_1:X_1 \ldots x_p : X_p$.
We write $q_{\sem{\Gamma}}^1$, \ldots, $q_{\sem{\Gamma}}^p$ to denote the initial question moves of the
component $\Gamma$ of the arena $\sem{\Gamma \rightarrow T}$ and $q^0_A$ to denote the single initial question of any arena $A$
(arenas involved in the game semantics of pure simply-typed $\lambda$-calculus have only one root).
$r$ denotes the root of the computation tree.

We define the total function $\varphi_M : V_\lambda \union V_{var} \rightarrow \mathcal{M}_M$ as follows:
\begin{align*}
\varphi_M =
        f_{\sem{\Gamma \rightarrow T}}(r, q^0_{\sem{\Gamma \rightarrow T}}) \quad
    & \union \quad
    \Union_{n \in N_{fv} | n \mbox{ {\small labelled} } x_i }  f_{\sem{\Gamma \rightarrow T}}(n, q^i_{\sem{\Gamma}} ) \\
    & \union \quad
        \Union_{n \in N \inter E \relimg{N_@ \union N_\Sigma}}  f_{\sem{type(\kappa(n))}}(n, q^0_{\sem{type(\kappa(n))}} )
\end{align*}
When there is no ambiguity we just write $\varphi$ instead of $\varphi_M$.
\end{dfn}

Nodes of $\tau(M)$ are either hereditarily justified by the root, by
a @-node or by a $\Sigma$-node, therefore $\varphi_M$ is totally
defined on $V_\lambda \union V_{var} = V\setminus (V_@ \union
V_\Sigma)$.

\begin{exmp}
Consider the term $\lambda x . (\lambda g . g x) (\lambda y . y)$ with $x,y:o$ and $g:(o,o)$.
The diagram below represents the computation tree (middle), the arenas
$\sem{(o,o)\rightarrow o}$ (left), $\sem{o \rightarrow o}$ (right), $\sem{o\rightarrow o}$ (rightmost)
and the function $\varphi = f(\lambda x, q_{\lambda x}) \union f(\lambda g, q_{\lambda g}) \union f(\lambda y, q_{\lambda y})$
(dashed-lines).
$$
\psset{levelsep=4ex}
\pstree{\TR[name=root]{\lambda x}}
{
    \pstree{\TR[name=App]{@}}
    {
            \pstree{\TR[name=lg]{\lambda g}}
                { \pstree{\TR[name=lgg]{g}}{
                        \pstree{\TR[name=lgg1]{\lambda}}
                        { \TR[name=lgg1x]{x}  } } }
            \pstree{\TR[name=ly]{\lambda y}}
                    {\TR[name=lyy]{y}}
    }
}
\rput(5cm,-1cm){
  \pstree{\TR[name=A1lx]{q_{\lambda x}}}
        { \TR[name=A1x]{q_x} }
}
\rput(-6cm,-1.5cm){
    \pstree{\TR[name=A2lg]{q_{\lambda g}}}
    {
        \pstree{\TR[name=A2g]{q_g}}
        {  \TR[name=A2g1]{q_{g_1}}   }
    }}
\rput(2.5cm,-1.5cm){
    \pstree{\TR[name=A3ly]{q_{\lambda y}}}
        { \TR[name=A3y]{q_y}
        }
}
\psset{nodesep=1pt,arrows=->,arcangle=-20,arrowsize=2pt 1,linestyle=dashed,linewidth=0.3pt}
\ncline{->}{root}{A1lx} \mput*{f(\lambda x, q_{\lambda x})}
\ncarc{->}{lgg1x}{A1x}
\ncline{->}{lg}{A2lg} \mput*{f(\lambda g, q_{\lambda g})}
\ncline{->}{lgg}{A2g}
\ncline{->}{lgg1}{A2g1}
\ncline{->}{ly}{A3ly} \mput*{f(\lambda y, q_{\lambda y})}
\ncline{->}{lyy}{A3y}
$$
\end{exmp}

The following properties are immediate consequences of the definition of the procedure $f$:
\begin{property} \
\label{proper:phi_conserve_order}
\begin{itemize}
\item[(i)] $\varphi$ maps $\lambda$-nodes to O-questions, variable nodes to
P-questions, value-leaves of $\lambda$-nodes to P-answers and
value-leaves of variable nodes to O-answers;
\item[(ii)] $\varphi$ maps nodes of a given order to moves of the same order.
\end{itemize}
\end{property}
Remark: we recall that in definition \ref{def:nodeorder}, the
node-order is defined differently for the root $\lambda$-node and
other $\lambda$-nodes. This convention was chosen to guarantee that
property (ii) holds.

By extension, the function $\varphi$ is also defined on justified
sequences of nodes: if $t = t_0 t_1 \ldots$ is a justified sequence
of nodes in $V_\lambda \union V_{var}$ then $\varphi(t)$ is defined
to be the following sequence of moves:
$$\varphi(t) = \varphi(t_0)\ \varphi(t_1)\  \varphi(t_2) \ldots$$
where the pointers of $\varphi(t)$ are defined to be exactly those
of $t$. This definition implies that $\varphi : (V_\lambda \union
V_{var})^* \rightarrow \mathcal{M}^*$ regarded as a function from
pointer-less sequences of nodes to pointer-less sequences of moves
is a monoid homomorphism.

\begin{property}
\label{proper:phi_pview} Let $t$ be a justified sequence of nodes. The following properties hold:
\begin{itemize}
\item[(i)] $\varphi(t)$ and $t$ have the same pointers;
\item[(ii)] the P-view of $\varphi(t)$ and the P-view of $t$ are computed
identically: the set of indices of elements that must be removed
from both sequences in order to obtain their P-view is the same;
\item[(iii)] the O-view of $\varphi(t)$ and the O-view of $t$ are computed identically;
\item[(iv)] if $t$ is a justified sequence of nodes in $V_\lambda \union V_{var}$ then $?(\varphi(t)) =
\varphi(?(t))$,
\end{itemize}
where $?(\varphi(t))$ denotes the subsequence of $\varphi(t)$ consisting of the unanswered questions
and $?(t)$ denotes the subsequence of $t$ consisting of the unmatched nodes (see the
definition in section \ref{sec:adding_value_leaves}).
\end{property}
\begin{proof}
(i): By definition of $\varphi$, $t$ and $\varphi(t)$ have the same
pointers;

(ii) and (iii): $\varphi$ maps lambda nodes to O-question,
non-lambda nodes to P-question, value-leaves of lambda nodes to P-answers and
value-leaves of non-lambda to O-answers. Therefore since $t$ and $\varphi(t)$ have the
same pointers, the computations of the P-view (resp. O-view) of the
sequence of moves and the P-view (resp. O-view) of the sequence of
nodes follow the same steps;

(iv) is a consequence of (i).

\end{proof}


\subsection{Category of interaction games}
\label{sec:interaction_semantics}

In game semantics, strategy composition is achieved by performing a
CSP-like ``composition + hiding''. It is possible to define an
alternative semantics where the internal moves are not hidden when
performing composition. This semantics is named \emph{interaction}
semantics in \cite{DBLP:conf/sas/DimovskiGL05} and \emph{revealed
semantics} in \cite{willgreenlandthesis}.

In addition to the moves of the standard semantics, the interaction semantics contains certain
internal moves of the computation.
Consequently, the interaction semantics depends on the syntactical structure of the term and therefore cannot
lead to a full abstraction result. However this semantics will prove to be useful to identify
a correspondence between the game semantics
of a term and the traversals of its computation tree.

We will be interested in the interaction semantics computed from the
$\eta$-normal form of a term. However we do not want to keep all the internal moves. We will only keep the internal
moves that are produced when composing two subterms of the computation tree that are joined by an @-node.
This means that when computing the strategy of
$y N_1 \ldots N_p$ where $y$ is a variable, we keep the internal moves of $N_1$, \ldots, $N_p$, but
we omit the internal moves produced by the copy-cat projection strategy denoting $y$.

\begin{dfn}[Type-tree]
We call \emph{type decomposition tree} or \emph{type-tree}, a tree whose leaves are labelled with linear simple types
and nodes are labelled with symbol in $\{ ;, \times, \otimes, \dagger, \Lambda \}$.

Nodes labelled $;$, $\times$ or $\otimes$ are binary nodes and nodes labelled $\dagger$ or $\Lambda$ are unary nodes.

Every node or leaf of the tree has a linear type, this type is determined by the
structure of the tree as follows:
\begin{itemize}
\item a leaf has the type of its label;

\item a $\dagger$-node with the child node of type $!A \multimap B$ has type $!A \multimap !B$;

\item a $\Lambda$-node with the child node of type $A \otimes B \multimap C$ has type $A \multimap (B \multimap C)$;

\item a $\times$-node with two children nodes of type $A$
and $B$ has type $A \times B$;

\item a $\otimes$-node with two children nodes of type $A$
and $B$ has type $A\otimes B$;

\item a $;$-node with two children nodes of type $A\multimap B$
and $B \multimap C$ has type $A \multimap C$.
\end{itemize}

For a type-tree to be well-defined, the type of the children nodes
must be compatible with the meaning of the node, for instance the
two children nodes of a ;-node must be of type
$A\multimap B$ and $B\multimap C$.

We write $type(T)$ to denote the type represented by the root of the tree $T$. An we say that $T$ is a \emph{valid
tree decomposition} of $type(T)$.

If $T_1$ and $T_2$ are type-tree we write $T_1 \times T_2$ to denote the tree obtained by attaching $T_1$ and $T_2$ to a $\times$-node.
Similarly we use the notations $T_1 \otimes T_2$, $T_1 ; T_2$, $\Lambda(T_1)$ and  $T_1^\dagger$.
\end{dfn}


Let $T$ be a type-tree. Each leaf or node of type $A$ in $T$ can be mapped to the
(standard) arena $\sem{A}$. By taking the image of $T$ across this mapping we obtain a tree whose leaves and nodes are labelled by arenas.
This tree, written $\intersem{T}$, is called the \emph{interaction arena} of type $T$.
We write $root(\intersem{T})$ to denote the arena located at the root of the interaction arena $\intersem{T}$.

A \emph{revealed strategy} $\Sigma$ on the interaction arena
$\intersem{T}$ is a composition of several standard strategies where
certain internal moves are not hidden. Formally this can be defined
as follows:
\begin{dfn}[Revealed strategy]
A revealed strategy $\Sigma$ on a game $\intersem{T}$, written
$\Sigma: \intersem{T}$, is a tree type $T$ where
\begin{itemize}
\item each leaf $\sem{A}$ of
$\intersem{T}$ is annotated with a (standard) strategy $\sigma$ on the
game $\sem{A}$;
\item each $;$-node is annotated with a set of indices $U \subseteq \nat$.
\end{itemize}
\end{dfn}
A $;$-node with children of type $A\multimap B$ and $B\multimap C$ is annotated with a set of indices $U$ indicating
which components of $B$ should be uncovered when performing composition.
\begin{exmp}
The diagrams below represent a type-tree $T$ (left) the
corresponding interaction arena $\intersem{T}$ (middle) and an
revealed strategy $\Sigma$ (right):
$$
\pstree[levelsep=6ex]{\TR{;}}
        {
            \pstree[levelsep=6ex]{\TR{;}}
            { \TR{A\multimap B}
              \TR{B\multimap C}
            }
            \TR{C\multimap D}
        }
\hspace{1cm}
\pstree[levelsep=6ex]{\TR{;}}
        {
            \pstree[levelsep=6ex]{\TR{;}}
            { \TR{\sem{A\multimap B}}
              \TR{\sem{B\multimap C}}
            }
            \TR{\sem{C\multimap D}}
        }
\hspace{1cm}
\pstree[levelsep=6ex]{\TR{;^{\{0\}}}}
        {
            \pstree[levelsep=6ex]{\TR{;^{\{0\}}}}
            { \TR{A\multimap B^{\sigma_1}}
              \TR{B\multimap C^{\sigma_2}}
            }
            \TR{C\multimap D^{\sigma_3}}
        }
$$
\end{exmp}
A revealed strategy can also be written as an expression, for
instance the strategy represented above is given by the expression
$\Sigma = (\sigma_1 ;^{\{0\}} \sigma_2) ;^{\{0\}} \sigma_3$. We will
use the abbreviation $\Sigma_1 \fatsemi^U \Sigma_2$ for
$\Sigma_1^\dagger ; ^U \Sigma_2$.

\begin{dfn}[Composition of revealed strategies]
Suppose $\Sigma_1 : \intersem{T_1}$ and $\Sigma_2 : \intersem{T_2}$
are revealed strategies where $type(T_1) = A \multimap B$ and
$type(T_2) = B \multimap C$ then the \emph{interaction composition}
of $\Sigma_1$ and $\Sigma_2$ written $\Sigma_1 ; \Sigma_2$ is the
revealed strategy on $\intersem{T_1 ; T_2}$ obtained by copying the
annotation of the leaves and nodes from $\Sigma_1$ and $\Sigma_2$ to
the corresponding leaves and nodes of the type-tree $T_1 ; T_2$ and
by annotating the root node with $\emptyset$.
\end{dfn}

A play of the interaction semantics, called an \emph{uncovered
play}, is a play containing internal moves.
The moves are implicitly tagged so that it is possible to retrieve in which component
of which node or leaf-arenas the move belongs to. Note that a same move can belong to different node/leaf-arenas.
The internal moves of an interaction play on the game $\intersem{T}$ are those which do not
belong to the arena $root(\intersem{T})$.

For any uncovered play $s$ and any interaction arena $\intersem{T}$
we can define the filtering operator $s\upharpoonright \intersem{T}$ to be the
sequence of moves obtained from $s$ by keeping only the moves
belonging to a node or leaf-arena of $\intersem{T}$.


Revealed strategies can alternatively be represented by mean of sets
of uncovered plays instead of annotated type-trees. This set is
defined inductively on the structure of the annotated type-tree
$\Sigma$ as follows:
\begin{itemize}
\item for a leaf $\sem{A}$ of $\Sigma$ annotated by $\sigma :\sem{A}$, it is just the set of plays of the standard strategy $\sigma$;
\item for a $\otimes$-node with two children strategies $\Sigma_1$ and $\Sigma_2$, it is the tensor product written $\Sigma_1 \otimes \Sigma_2$;
\item for a $\times$-node, it is the pairing written $\langle \Sigma_1, \Sigma_2 \rangle$;
\item for a $\dagger$-node with a child strategy $\Sigma$, it is the promotion written $\Sigma^\dagger$;
\item for a $\Lambda$-node with a child strategy $\Sigma$, it is the same set of plays with the moves retagged appropriately;

\item for a $;^U$-node, it is the ``uncovered-composition'' of $\Sigma_1 : \intersem{T_1}$ and $\Sigma_2 :\intersem{T_2}$ which is written $\Sigma_1
;^U \Sigma_1$ and defined as follows: suppose that $type(T_1) = A
\multimap B_0 \times \ldots \times B_l$ and $type(T_2) = B_0 \times
\ldots \times B_l \multimap C$ then $\Sigma_1 ;^U \Sigma_1$ is the
set of uncovered plays obtained by performing the usual composition
while ignoring and copying the internal moves from arenas in
$\intersem{T_1}$ or $\intersem{T_2}$ and preserving any internal
move produced by the composition in some component $B_k$ for $k \in
U$. Formally:
$$ \Sigma_1 \| \Sigma_2 = \{ u \in int(\intersem{T}) \ | \ u \upharpoonright \intersem{T_1} \in \Sigma_1 \mbox{ and } u \upharpoonright \intersem{T_2} \in \Sigma_2 \}$$
$$ \Sigma_1 ;^{\{i_0, \ldots i_l\}} \Sigma_2 = \{ u \upharpoonright A, B_{i_0}, \ldots, B_{i_l}, C \ | \ u \in \Sigma_1 \| \Sigma_2 \}$$
where $int(\intersem{T})$ denotes the set of sequences of moves in (some arena of) $\intersem{T}$;
\end{itemize}
where the tensor product, pairing and promotion are defined similarly as in the standard game semantics.


We can now define the category $\mathcal{I}$ of interaction games:
\begin{dfn}[Category of interaction games]
The category of interaction games is denoted by $\mathcal{I}$. The
objects of $\mathcal{I}$ are those of $\mathcal{C}$ i.e. the arenas
$\sem{A}$ for some linear type $A$. The morphisms of the category
are the revealed strategies: a morphism from $A$ to $B$ is an
revealed strategy $\Sigma$ on some interaction arena $\intersem{T}$
such that $root(\intersem{T}) = \sem{!A\multimap B}$.

The composition of two morphisms $\Sigma_1$ and $\Sigma_2$ is given
by $\Sigma_1 \fatsemi \Sigma_2 = \Sigma_1^\dagger ; \Sigma_2$ where
$;$ denotes the revealed strategy composition. The identity on $A$
is the revealed strategy given by the single annotated leaf $\sem{!A
\multimap A}^{der_A}$.
\end{dfn}

It can be checked that this indeed defines a category. The constructions of the category $\mathcal{C}$ can be transposed to $\mathcal{I}$
making $\mathcal{I}$ a cartesian closed category.


\begin{dfn}[Valid strategy]
Consider a term $\Gamma \vdash M : A$ and an revealed strategy
$\Sigma : \intersem{T}$. We say that $\Sigma$ is a valid revealed
strategy for $M$ if $root(\intersem{T}) = \sem{\Gamma \rightarrow
A}$ or equivalently if $type(T) = \Gamma \rightarrow A$.
\end{dfn}


\subsubsection{Modeling the $\lambda$-calculus in $\mathcal{I}$}

We would like to use the category $\mathcal{I}$ to model terms of
the simply-typed lambda calculus.
Depending on the internal moves that we wish to hide, we obtain different possible interaction strategies for a given term.
We now fix a unique strategy denotation which is computed from the $\eta$-normal form of the term.

\begin{dfn}[Revealed denotation of a term]
\label{dfn:interactionstrategy_ofterms}
The \emph{revealed game denotation} or \emph{revealed
strategy} of $M$ written $\intersem{\Gamma \vdash M : A}$ is defined by structural induction on the $\eta$-long normal form of $M$ as follows:

Let $\overline{\xi} = \xi_1 : Y_1, \ldots \xi_n : Y_n$
and $z$ be a variable ranging in $\Gamma \union \overline{\xi}$. If $z\in \Gamma$ then $\pi_{z}$ denotes
the $i^{th}$ projection copycat strategy $\pi_i : \sem{\Gamma \union \overline{\xi}} \rightarrow \sem{X_i}$. If $z = \xi_j$ then
$\pi_{z}$ denotes the $(n+j)^{th}$ projection $\pi_{n+j} : \sem{\Gamma \union \overline{\xi}} \rightarrow \sem{Y_j}$.
\begin{eqnarray*}
\intersem{\Gamma \vdash \lambda \overline{\xi} . z } &=& \Lambda^n(\pi_{z})  \\
\intersem{\Gamma \vdash \lambda \overline{\xi} . z N_1 \ldots N_p} &=& \Lambda^n(\langle \pi_z, \intersem{\Gamma \vdash N_1 : A_1}, \ldots, \intersem{\Gamma \vdash N_p : A_p}  \rangle \fatsemi ^{\emptyset} ev^p) \\
\intersem{\Gamma \vdash \lambda \overline{\xi}. f N_1 \ldots N_p} &=& \langle \intersem{\Gamma \vdash N_1}, \ldots, \intersem{\Gamma \vdash N_p} \rangle \fatsemi^{0..p-1} \sem{f} \\
\intersem{\Gamma \vdash \lambda \overline{\xi} . N_0 \ldots N_p} &=& \Lambda^n(\langle \intersem{\Gamma \vdash N_0 : A_0}, \ldots, \intersem{\Gamma \vdash N_p : A_p}  \rangle \fatsemi^{0..p} ev^p)
\end{eqnarray*}
where $\Gamma \vdash N_0 : (A_1,\ldots,A_p,B)$, $\Gamma \vdash z : (A_1,\ldots,A_p,B)$, $\Gamma \vdash N_k : A_k$ for $k\in 1..p$,
$f : (A_1,\ldots,A_p,B) \in \Sigma$ and $ev^p$ denotes the evaluation strategy with $p$ parameters.

We write $\intersem{\Gamma \rightarrow A}_M$ to denote the
interaction arena of the revealed strategy $\intersem{\Gamma \vdash
M : A}$.
\end{dfn}
Note that when computing $\intersem{z N_1 \ldots N_p}$, for some variable $z$, the internal moves of $N_1$, \ldots, $N_p$ are preserved but
we omit the internal moves produced by the copy-cat projection strategy denoting $z$.



%\begin{dfn}[Revealed denotation of a term]
%\label{dfn:interactionstrategy_ofterms} Let $\Gamma \vdash M : A$ be
%a term with $\Gamma = x_1:X_1, \ldots, x_k:X_k$. Let $\pi_i :
%\sem{\Gamma \rightarrow X_i}$ denote the $i$th projection copycat
%strategy and $ev^p$ denote the evaluation strategy with $p$
%parameters.
%
%The \emph{revealed game denotation of $M$} or \emph{revealed
%strategy of $M$} written $\intersem{\Gamma \vdash M : A}$ is the
%revealed strategy defined by structural induction on the computation
%tree $\tau(M)$ as follows:
%
%\begin{tabularx}{14cm}{cX}
%$\tree[levelsep=6ex]{\lambda \xi_1\ldots \xi_n}{\TR{x_i}}$ &
%       $\intersem{M} = \Lambda^n(\pi_i)$ \\ \hline
%$ \tree[levelsep=6ex]{\lambda \xi_1\ldots \xi_n}
%        { \tree[levelsep=6ex]{x_i}
%            {   \TR{\tau(N_1)} \TR{\ldots} \TR{\tau(N_p)}}}
%    $
%&    where $\Gamma \vdash x_i : (A_1,\ldots,A_p,B)$ and $\Gamma \vdash N_j : A_j$ for $j\in 1..p$
%    $$\intersem{M} = \Lambda^n(\langle \pi_i, \intersem{\Gamma \vdash N_1 : A_1}, \ldots, \intersem{\Gamma \vdash N_p : A_p}  \rangle
%    \fatsemi ^{1..p} ev^p)$$
%\\ \hline
%$ \tree[levelsep=6ex]{\lambda \xi_1\ldots \xi_n}
%        { \tree[levelsep=6ex]{@}
%            {   \TR{\tau(N_0)} \TR{\ldots} \TR{\tau(N_p)}}}
%    $ &
%    where $\Gamma \vdash N_0 : (A_1,\ldots,A_p,B)$ and $\Gamma \vdash N_j : A_j$ for $j\in 1..p$
%    $$\intersem{M} = \Lambda^n(\langle \intersem{\Gamma \vdash N_0 : A_0}, \ldots, \intersem{\Gamma \vdash N_p : A_p}  \rangle
%    \fatsemi^{0..p} ev^p)$$
%\end{tabularx}
%\vspace{10pt}
%
%We write $\intersem{\Gamma \rightarrow A}_M$ to denote the
%interaction arena of the revealed strategy $\intersem{\Gamma \vdash
%M : A}$.
%\end{dfn}
%


\begin{exmp}
Consider the term $\lambda x . (\lambda f . f x) (\lambda y . y)$.
Its computation tree is:
$$
\tree{\lambda x} {
    \pstree[levelsep=4ex]{\TR{@}}
    {       \pstree[levelsep=4ex]{\TR{\lambda f}}
                { \tree{f}{  \tree{\lambda}{ \TR{x}  } } }
            \pstree[levelsep=4ex]{\TR{\lambda y}}
                    {\TR{y}}
    } }
$$
and its revealed strategy is $\langle \sem{ x:X \vdash \lambda f . f
x} , \sem{ x:X \vdash \lambda y . y} \rangle \fatsemi^{\{0,1\}}
ev_2$.
\end{exmp}


\subsubsection{From interaction semantics to standard semantics and vice-versa}

In the standard semantics, given two strategies $\sigma : A \rightarrow B$, $\tau : B \rightarrow C$ and
a sequence $s \in \sigma \fatsemi \tau$, it is possible to (uniquely) recover the internal moves. The uncovered sequence is written
${\bf u}(s, \sigma, \tau)$. The algorithm to obtain this unique uncovering is given in part II of \cite{hylandong_pcf}.

Given a term $M$, we can completely uncover the internal moves of a
sequence $s\in\sem{M}$ by performing the uncovering recursively at
every @-node of the computation tree. This operation is called
\emph{full-uncovering with respect to $M$}.

Conversely, the standard semantics can be recovered from the
interaction semantics by filtering the moves, keeping only those
played in the root arena:
\begin{eqnarray}
 \sem{\Gamma \vdash M : A} = \intersem{\Gamma \vdash M : A} \upharpoonright \sem{\Gamma \rightarrow T} \label{eqn:int_std_gamsem}
\end{eqnarray}


\subsubsection{Full abstraction}

Let $\mathcal{I'}$ denote lluf sub-category of $\mathcal{I}$
consisting only of strategies $\Sigma$ with a single annotated leaf
and no nodes. We have the following lemma:
\begin{lem}[$\mathcal{I'}$ is isomorphic to $\mathcal{C}$]
$\mathcal{I'} \cong \mathcal{C}$
\end{lem}
\begin{proof}
We define the functor $F:\mathcal{I'} \rightarrow \mathcal{C}$
by $F(A) = A$ for any object $A\in \mathcal{I'}$ and for $\Sigma \in \mathcal{I'}(A,B)$,
$F(\Sigma)$ is defined to be the annotation $\sigma$ of the only leaf in $\Sigma$.
The functor $G:\mathcal{C} \rightarrow \mathcal{I'}$ is defined by
$G(A) = A$ for any object $A\in \mathcal{C}$ and for $\sigma \in \mathcal{C}(A,B)$,
$G(\sigma)$ is the tree formed with the single annotated leaf $\sem{A}^\sigma$.
Then $F;G =id_{\mathcal{I'}}$ and $G;F =id_{\mathcal{C}}$.
\end{proof}

Consequently the lluf sub-category $\mathcal{I'}$ is fully abstract for the simply-typed lambda calculus.
Note that this is a major difference with $\mathcal{I}$ which is not fully-abstract since there may be several maps denoting a given
term.





\subsection{The correspondence theorem for the pure simply-typed $\lambda$-calculus}
In this section, we establish a
connection between the interaction semantics of a simply-typed term without constants ($\Sigma = \emptyset$)
and the traversals of its computation tree.

\subsubsection{Removing @-nodes from traversals}

When defining computation trees, it was necessary to introduce
application nodes (labelled @) in order to connect the operator and
the operand of an application. The presence of @-nodes has also
another advantage: it ensures that the lambda-nodes are all at even
level in the computation tree. Consequently a traversal respects
Alternation.

Application nodes are however redundant in the sense that they do
not play any role in the computation of the term. In other words,
the @-nodes occurring in traversals are superfluous. In fact it is
necessary to filter them out if we want to establish the
correspondence with the interaction game semantics.

\begin{dfn}[Filtering @-nodes in traversals]
\label{dfn:appnode_filter}
Let $t$ be a traversal of $\tau(M)$.
We write $t-@$ for the sequence of nodes with pointers obtained by
\begin{itemize}
\item removing from $t$ all @-nodes and value-leaves of a @-node;
\item replacing any link pointing to an @-node by a link pointing to the immediate predecessor of @ in $t$.
\end{itemize}

Suppose $u = t-@$ is a sequence of nodes obtained by applying the
previously defined transformation on the traversal $t$, then $t$ can
be partially recovered from $u$ by reinserting the @-nodes as
follows. For each @-node @ in the computation tree with parent node
denoted by $p$, we perform the following operations:
\begin{enumerate}
\item replace every occurrence of the pattern $p \cdot n$, where $n$ is a $\lambda$-nodes,
by $p \cdot @ \cdot n$;
\item replace any link in $u$ starting from a $\lambda$-node and pointing to $p$ by a link pointing to the inserted @-node;
\item if there is an occurrence in $u$ of a value-leaf $v_p$ pointing to $p$ then insert a value-leaf $v_@$
immediately before $v_p$ and make it points to the node immediately
following $p$ (which is also the $@$-node that we inserted in 1).
\end{enumerate}
We write $u+@$ for this second transformation.
\end{dfn}
These transformations are well-defined because in a traversal, an @-node
always occurs in-between two nodes $n_1$ and $n_2$ such that  $n_1$ is the parent node of @
and $n_2$ is the first child node of @ in the computation tree:
$$      \pstree[levelsep=4ex]{\TR{n_1}\treelabel{0} }
        {
            \pstree[levelsep=3ex]{\TR{@}}
            {
                \tree{n_2}{\vdots}
                \TR[edge=\dedge]{}
                \TR[edge=\dedge]{}
            }
        }
$$
Remark: $t-@$ is not a proper justified sequence
since after removing a @-node, any $\lambda$-node justified by @ will become
justified by the parent of @ which is also a $\lambda$-node.

The following lemma follows directly from the definition:
\begin{lem}
\label{lem:minus_at_plus_at}
For any traversal $t$ we have $(t-@)+@ \sqsubseteq t$ and if $t$ does not end with an @-node then
$(t-@)+@ = t$.
\end{lem}

Let $M$ be a term and $r$ be the root of $\tau(M)$. We introduce the following notations:
\begin{eqnarray*}
\travset(M)^{-@} &=& \{ t - @ \ | \  t \in \travset(M) \} \\
\travset(M)^{\upharpoonright r} &=& \{ t  \upharpoonright r \ | \  t  \in \travset(M) \} .
\end{eqnarray*}

\begin{lem}
Let $M$ be a pure simply-typed term and $r$ be the root of $\tau(M)$.
If $M$ is in $\beta$-normal form then $t = t \upharpoonright r = t - @$ for any $t \in \travset(M)$.
Consequently $\travset(M)^{-@} = \travset(M) =  \travset(M)^{\upharpoonright r }$.
\end{lem}
\begin{proof}
This is because the computation tree of a term in $\beta$-normal
does not contain any @-node and therefore all the nodes are
hereditarily justified by the root.
\end{proof}



\begin{lem}[Filtering lemma] Let $\Gamma \vdash M :T$ be a term and $r$ be the root of $\tau(M)$.
\label{lem:varphi_filter}
For any traversal $t$ of the computation tree we have
$\varphi(t-@) \upharpoonright \sem{\Gamma \rightarrow T} = \varphi(t\upharpoonright r)$.
Consequently:
$$ \varphi(\travset^{-@}(M)) \upharpoonright \sem{\Gamma \rightarrow T} = \varphi(\travset^{\upharpoonright r}(M)).$$
\end{lem}
\begin{proof}
    From the definition of $\varphi$, the nodes of the computation tree that are mapped by $\varphi$
    to moves of the arena $\sem{\Gamma \rightarrow T}$ are exactly the nodes that are hereditarily justified by $r$.
    The result follows from the fact that @-nodes are not hereditarily justified by the root.
\end{proof}

The function $\varphi$ regarded as a function from the set of vertices $V_\lambda \union V_{var}$ of the computation tree to moves in arenas is not injective.
For instance the two occurrences of $x$ in the computation tree of the term $\lambda f x. f x x$ are mapped to the same question. However
the function $\varphi$ regarded as a function from sequences of nodes to sequences of moves is injective:
\begin{lem}[$\varphi$ is injective]
\label{lem:varphiinjective}
$\varphi$ regarded as a function defined on the set of
sequences of nodes is injective in the sense that for any two traversals $t_1$ and $t_2$:
\begin{itemize}
\item[(i)] if $\varphi (t_1 - @ ) = \varphi (t_2 - @ )$ then $t_1-@ =t_2 -@$;
\item[(ii)] if $\varphi (t_1 \upharpoonright r ) = \varphi (t_2 \upharpoonright r )$ then $t_1\upharpoonright r = t_2\upharpoonright r$.
\end{itemize}
\end{lem}
\begin{proof}
(i) The set of traversals of a computation tree verifies the following property:
\begin{equation}
t \cdot n_1, t \cdot n_2 \in \travset \mbox{ where } n_1 \neq n_2 \mbox{ and $n_1, n_2$ are not @-node implies } \varphi(n_1) \neq \varphi(n_2). \label{lem:varphiinjective:eq1}
\end{equation}
Indeed, the only possible case where $\varphi$ maps two different
nodes to the same move is when $n_1$ and $n_2$ are two nodes
labelled with the same variable $x$. Hence the two traversals $t
\cdot n_1$ and $t \cdot n_2$ must have been formed using either rule
(Lam) or (App). But these two rules are deterministic and their
domain of definition is disjoint. This contradict the fact that $n_1
\neq n_2$.

Now suppose that $t_1-@\neq t_2-@$ then necessarily $t_1 \neq t_2$. Therefore
 $t_1 = t' \cdot n_1 \cdot u_1$ and $t_2 = t' \cdot n_2 \cdot u_2$ for some sequences $t'$, $u_1$, $u_2$
and some nodes $n_1\neq n_2$. By property \ref{lem:varphiinjective:eq1} we have $\varphi(n_1) \neq \varphi(n_2)$.
If we regard sequences of nodes and moves as \emph{pointer-less} sequences then we are allowed to write the following:
$$ (t' \cdot n_1 \cdot u_1) - @ = (t' - @) \cdot n_1 \cdot (u_1 -@),$$
and since $\varphi_M$ is a monoid homomorphism (provided that we ignore the justification pointers) we have:
$$ \varphi(t_1-@) = \varphi(t'-@) \cdot \varphi(n_1) \cdot \varphi(u_1) \neq \varphi(t'-@) \cdot \varphi(n_2) \cdot \varphi(u_2) = \varphi(t_2-@).$$

(ii) Again, suppose that $t \upharpoonright r \neq t' \upharpoonright r$ then
 $t_1 = t'_1 \cdot n_1 \cdot u_1$ and $t_2 = t_2' \cdot n_2 \cdot u_2$ for some sequences $t_1'$, $t_2'$, $u_1$, $u_2$
 such that $t'_1 \upharpoonright r = t'_2 \upharpoonright r $
and some nodes $n_1 \neq n_2$ both hereditarily justified by the root.
For the same reason as in (i), we must have $\varphi(n_1) \neq \varphi(n_2)$. Hence:
$$ \varphi(t_1\upharpoonright r) =
        \varphi(t'_1\upharpoonright r) \cdot \varphi(n_1) \cdot \varphi(u_1 \upharpoonright r)
    \neq \varphi(t'_1\upharpoonright r) \cdot \varphi(n_2) \cdot \varphi(u_2 \upharpoonright r)
         = \varphi(t_2\upharpoonright r).$$
\end{proof}

\begin{cor} \
\label{cor:varphi_bij}
\begin{itemize}
\item[(i)] $\varphi$ defines a bijection from $\travset(M)^{-@}$
to $\varphi(\travset(M)^{-@})$;
\item[(ii)] $\varphi$ defines a bijection from $\travset(M)^{\upharpoonright r}$ to
$\varphi(\travset(M)^{\upharpoonright r})$.
\end{itemize}
\end{cor}

\subsubsection{The correspondence theorem}
We are now going to state and prove the correspondence theorem
for the pure simply-typed $\lambda$-calculus without constants ($\Sigma = \emptyset$).
The result extends immediately to the simply-typed $\lambda$-calculus with \emph{uninterpreted} constants by
considering constants as being free variables.
We use the cartesian closed category of games $\mathcal{C}$ (defined in section \ref{subsec:pcfgamemodel} of the first chapter) as
a model of the simply-typed $\lambda$-calculus. We write $\sem{\Gamma \vdash M : A}$ for the strategy denoting the simply-typed term
$\Gamma \vdash M : A$.

\begin{prop}
\label{prop:rel_gamesem_trav} Let $\Gamma \vdash M : T$ be a term of
the pure simply-typed $\lambda$-calculus and $r$ be the root of
$\tau(M)$. We have:
\begin{itemize}
\item[(i)]  $\varphi_M(\travset(M)^{-@}) = \intersem{M}$
\item[(ii)] $\varphi_M(\travset(M)^{\upharpoonright r}) = \sem{M}$.
\end{itemize}
\end{prop}


\begin{rem} The proof that follows is quite tedious but the idea is simple. Let us give the intuition.
    We start by reducing the problem to the case of closed terms only. Then the proof proceeds by induction on the structure of the computation tree.
    It is straightforward to prove the result for term that are abstraction of a single variable.
    Now consider an application $M$ with the following computation tree $\tau(M)$:
    $$ \tree[levelsep=4ex]{\lambda \overline{\xi}}
        { \tree[levelsep=4ex]{@}
            {   \TR{\tau(N_0)} \TR{\ldots} \TR{\tau(N_p)}}}
    $$

    A traversal of $\tau(M)$ proceeds as follows: it starts at the root $\lambda \overline{\xi}$ of the tree $\tau(M)$ (rule
    (Root)), it then passes the node @ (rule (Lam)).
    After this initialization part, it proceeds by traversing the term $N_0$ (rule (App)).
    At some point, while traversing $N_0$, some variable $y_i$ bound by the root of $N_0$ is visited. The traversal
    of $N_0$ is interrupted and there is a jump (rule (Var)) to the root of $\tau(N_i)$. The process goes on by traversing $\tau(N_i)$.
    When traversing $N_i$, if the traversal encounters a variable bound by the root of $\tau(N_i)$ then the traversal of $N_i$ is interrupted and
    the traversal of $N_0$ resumes.  This schema is repeated until the traversal of $\tau(N_0)$ is completed\footnote{Since we are considering
    simply-typed terms, the traversal does indeed terminate. However this will not be true anymore in the \pcf\ case.}.

    The traversal of $M$ is therefore made of an initialization part followed by an interleaving of a traversal of $N_0$ and
    several traversals of $N_i$ for $i=1..p$. This schema is reminiscent of the way the evaluation copycat map $ev$ works in game semantics.

    The key idea is that every time the traversal pauses the traversal of a subterm and switches to another one,
    the jump is permitted by one of the four copycat rules (Var), (CCAnswer-@), (CCAnswer-$\lambda$) or (CCAnswer-var).
    We show by (a second) induction that these copycat rules defines exactly what the copycat strategy $ev$ performs on sets of moves.

%    In the game semantics, the evaluation map (a copy-cat strategy) copies this opening move to an initial move $m_0$ in the game
%    $B_0$ and the game continues in $B_0$. We reflect this in the traversal : we make $t$ follow
%    the ``script'' given by the traversal $t^0_{m_0}$.
%    The rule (App) allow us to initiate this simulation  by visiting the  first move in $t^0_{m_0}$: the root of $\tau(N_0)$.
%
%    This simulation continues until it reaches a node $\alpha_0$ which is hereditarily justified by the root
%    $\tau(N_0)$: $\alpha_0$ is present in the reduction of traversal of $t^0_{m_0}$ therefore $\varphi_{N_0}(\alpha_0)$ is an un-hidden move played in $A_0$.
%
%    In the game semantics this corresponds to a move played in a component $A_k$ for some $k\in 1..p$ of
%    of the game $B_0$ in which case the evaluation map copies the move to an initial move $m_1$ in the corresponding component $B_k$.
%
%    To reflect this the traversal now opens up a new thread and simulates the traversal $t^k_{m_1}$.  Again, this simulation stops when we reach a node
%    $\alpha_1$ in $t^k_{m_1}$ which is hereditarily justified by the root of $\tau(N_k)$: $\alpha_1$ must be present in the reduction of traversal
%    of $t^k_{m_1}$ therefore $\varphi_{N_k}(\alpha_1)$ is an un-hidden move played in $A_k$.
%    In the game semantics, this move $\alpha$ is copied back to the component $B_k$ of the game $B_0$.
%
%    The traversal now resumes the simulation of $t^0_{m_0}$. And the process goes continuously.
\end{rem}

Let us fix some notation: we write $s\upharpoonright A,B$ for the
sequence obtained from $s$ by keeping only the moves that are in $A$ or $B$ and by removing any link pointing to a move that
has been removed.
If $m$ is an initial move, we write $s \upharpoonright m$ to
denote the thread of $s$ initiated by $m$, i.e. the sequence obtained from $s$ by keeping all the moves
hereditarily justified by $m$.
We also write $s \upharpoonright A,B,m$ where $m$ is an initial move
for the sequence obtained from $s \upharpoonright A,B$ by keeping
all moves hereditarily justified by $m$.



\begin{proof}
(i) Suppose $\Gamma = \xi_1:X_1,\ldots \xi_n:X_n$. Then we have:
\begin{eqnarray*}
\intersem{\Gamma \vdash M:T} &=& \Lambda^n( \intersem{\emptyset \vdash \lambda \xi_1\ldots \xi_n . M: (X_1,\ldots,X_n,T) } ) \\
        &\simeq& \intersem{\emptyset \vdash \lambda \xi_1\ldots \xi_n . M: (X_1,\ldots,X_n,T) }.
\end{eqnarray*}
Similarly the computation tree $\tau(M)$ is isomorphic to
$\tau(\lambda \xi_1\ldots \xi_n . M)$ (up to a renaming of the root
of the computation tree) therefore $\travset(M)$ is also isomorphic
to $\travset(\lambda \xi_1\ldots \xi_n . M)$. Hence we can make the
assumption that $M$ is a closed term. If we prove that the property
is true for all closed terms of a given height then it will be
automatically true for any open term of the same height.


Let us assume that $M$ is already in $\eta$-long normal form. We
proceed by induction on the height of the tree $\tau(M)$ and by
case analysis on the structure of the computation tree:
\begin{itemize}
  \item (abstraction of a variable): $M \equiv \lambda \overline{\xi} .
  x$.  Since $M$ is in $\eta$-long normal form, $x$ must be of ground type and since $M$ is
      closed we have $x = \xi_i \in \overline{\xi}$ for some $i$.
      Hence $\tau(M)$ has the following shape:
        $$ \tree[levelsep=6ex]{ \lambda \overline{\xi}^{[0]} }{\TR{\xi_i^{[1]}}}$$
        The arena is of the following form (only question moves are represented):
        $$ \tree{ q_0 }
        {   \tree[linestyle=dotted]{q^1}{\TR{} \TR{} }
            \tree[linestyle=dotted]{q^2}{\TR{} \TR{} }
            \TR{\ldots}
            \tree[linestyle=dotted]{q^n}{\TR{} \TR{} }
        }$$

        Let $\pi_i$ denote the $i$th projection of the interaction game
        semantics. We have:
        \begin{align*}
        \intersem{M} &= \intersem{\emptyset \vdash \lambda \overline{\xi} . \xi_i} \\
                     &= \Lambda^n(\intersem{\overline{\xi} \vdash  \xi_i}) \\
                     &= \Lambda^n(\pi_i) \\
                     &\cong \pi_i \\
                     &= \prefset(\{ q_0 \cdot q^i \cdot v_{q^i} \cdot v_{q_0} \ | \ v\in \mathcal{D}
                     \}).
        \end{align*}

        Since $M$ is in $\beta$-normal we have $\travset(M)^{-@} = \travset(M)$.
        It is easy to see that the set of traversals of $M$ is the set of prefix of
        the traversal $\lambda \overline{\xi} \cdot \xi_i \cdot v_{\xi_i} \cdot v_{\lambda \overline{\xi}}$:
        $$ \travset^{-@}(M) = \travset(M) = \prefset( \lambda \overline{\xi} \cdot \xi_i \cdot v_{\xi_i} \cdot v_{\lambda \overline{\xi}})
        $$

        The pointers of the traversal $\lambda \overline{\xi} \cdot \xi_i \cdot v_{\xi_i} \cdot
        v_{\lambda \overline{\xi}}$ are the same as the play $q_0 \cdot q^i \cdot v_{q^i} \cdot
        v_{q_0}$, therefore since $\varphi_M(\lambda \overline{\xi}) = q_0$ and $\varphi_M(\xi_i) =
        q^i$ we have:
        $$ \varphi_M(\travset^{-@}(M)) = \intersem{M}.$$


    \item (abstraction of an application): we have $M = \lambda \overline{\xi} . N_0 N_1 \ldots N_p$. Let $\Gamma$ be the context
    $\Gamma = \overline{\xi} : \overline{X}$. Then we have the following sequents:
    $\emptyset \vdash M : (X_1,\ldots,X_n,o)$,
    $\Gamma \vdash N_0 N_1 \ldots N_p : o$,
    $\Gamma \vdash N_i : B_i$ for $i\in 0..p$ with $B_0 = (B_1,\ldots,B_p,o)$ and $p\geq 1$.

    There are two subcases, either $N_0 \equiv \xi_i$ where $\alpha$ is a variable in $\overline{\xi}$ and the tree has the following form:
    $$ \tree[levelsep=6ex]{\lambda \overline{\xi}^{[0]}}
        { \tree[levelsep=6ex]{\xi_i^{[1]}}
            {   \TR{\tau(N_1)} \TR{\ldots} \TR{\tau(N_p)}}}
    $$
    or $N_0$ is not a variable and the tree $\tau(M)$ has the following form:
    $$ \tree[levelsep=6ex]{\lambda \overline{\xi}^{[0]}}
        { \tree[levelsep=6ex]{@^{[1]}}
            {
            \tree[levelsep=6ex]{\lambda y_1 \ldots y_p}{\ldots}
            \TR{\tau(N_1)} \TR{\ldots} \TR{\tau(N_p)}}}
    $$

    We only consider the second case since the first one can be treated
    similarly. Moreover we make the assumption that $p=1$. It is
    straightforward to generalize to any $p\geq1$.
    We write $\lambda \overline{z}$ to denote the root of the tree $\tau(N_1)$.


    We have:
    \begin{align*}
    \intersem{M}
        &=  \Lambda^n( \intersem{\Gamma \vdash N_0 N_1 : o} )
            & \mbox{(game semantics for abstraction)}\\
        &\cong  \intersem{\Gamma \vdash N_0 N_1 : o}
            & \mbox{(up to moves retagging)}\\
        &=  \langle \intersem{\Gamma \vdash N_0}, \intersem{\Gamma \vdash N_1} \rangle \fatsemi^{0..1} ev
            & \mbox{(game semantics for application)}\\
        &=  \langle \varphi_{N_0} (\travset^{-@}(N_0)), \varphi_{N_1}(\travset^{-@}(N_1) \rangle \fatsemi^{0..1} ev
            & \mbox{(induction hypothesis)}\\
        &=  \langle \varphi_{M} (\travset^{-@}(N_0)), \varphi_{M}(\travset^{-@}(N_1)) \rangle \fatsemi^{0..1} ev
            & \mbox{($\varphi_M = f(0,q_0) \union \varphi_{N_0} \union \varphi_{N_1}$)} \\
        &=  \underbrace{\langle \varphi_{M} (\travset^{-@}(N_0)), \varphi_{M}(\travset^{-@}(N_1)) \rangle}_{\sigma} \parallel ev
            & \mbox{($\fatsemi^{0..1}$ and $\parallel$ are the same operator)}
    \end{align*}


    The strategies $\sigma$ and $ev$ are defined on the arena $!A \multimap B$ and $!B \multimap C$ respectively where:
    \begin{eqnarray*}
        A &=& \intersem{\Gamma} = \intersem{X_1} \times \ldots \times \intersem{X_n}\\
        B &=& \intersem{B_0} \times \intersem{B_1} = \intersem{B_1' \rightarrow o'} \times \intersem{B_1} \\
        C &=& \intersem{o}
    \end{eqnarray*}

    We have $u \in \intersem{M} \cong \sigma^{\dag} \parallel ev$ if and only if
    \begin{eqnarray*}
      &&      \left\{
            \begin{array}{ll}
                u \in int(!A,!B,C)\\
                u \upharpoonright !A,!B  \in \sigma^\dagger \\
                u \upharpoonright !B,C  \in  ev
            \end{array}
            \right. \\
    & \mbox{or equivalently} & \left\{
    \begin{array}{ll}
        u \in int(!A,!B,C) \\
        \hbox{for any initial $m$ in $u \upharpoonright !A,!B$ there is $j \in 0..p$ such that } \\
        \left\{\begin{array}{ll}
            u \upharpoonright !A,B_j, m \in \varphi_{M} (\travset^{-@}(N_j)) \label{eq:def_z} \\
            u \upharpoonright !A, B_k,m = \epsilon \quad \mbox{ for every } k\neq j \label{eq:b}
        \end{array}
        \right.
    \end{array}
    \right.
    \end{eqnarray*}


    We first prove that $\intersem{M} \subseteq \varphi_{M}( \travset^{-@}(M)
    )$.


    Suppose $u \in \intersem{M}$. We give a constructive proof that
    there exists a sequence of nodes $t$ in $N$ such that $\varphi_M(t-@) = u$ by induction on the length of $u$.
    Let $q_o$ be the initial question of the arena $\sem{M}$ and $q_1$ the initial question of $\sem{N_0}$.

    Base cases:
    \begin{itemize}
    \item $u=\epsilon$ then $\varphi(\epsilon) = u$ where the traversal $\epsilon$ is formed with the rule ($\epsilon$).
    \item If $|u|=1$ then $u=q_0$ is the initial move in $C$ and $\varphi(\lambda \overline{\xi}) = u$. The traversal
    $\lambda \overline{\xi}$ is formed with the rule (Root).
    \end{itemize}

    Step cases: Suppose that $u' = \varphi_M(t'-@)$ and $u = u' \cdot m \in \intersem{M}$ with $|u|>1$ for some traversal $t'$ of $\tau(M)$.
    Let us write $m^1$ for the last move in $u'$.

    \begin{enumerate}
    \item Suppose $m \in C$. In $C$ there are no internal moves, the only moves of $C$ are therefore $q_0$ and
    $v_{q_0}$ for some $v\in\mathcal{D}$. But $q_0$ can occur only once in $u$, therefore since $|u|>1$ we must have $m = v_{q_0}$
    for some $v\in \mathcal{D}$.  Since $m$ is an answer move to the initial question, it must be
    the duplication  (performed by the copy-cat evaluation strategy) of the move $m^1$ played in $o'$.
    Hence $m^1=v_{q_1}$. By the induction hypothesis, $n'$ -- the last move in $t'$ -- is equal to
    $\varphi(m^1) = v_{\lambda y_1}$.

    By property \ref{proper:phi_pview}(iv), $?(u') = \varphi(?(t'-@))$ and
    since $q_0$ is the pending question in $u'$, the first node of $t'$ is also the pending node in $t'$.
    This permits us to use the rule (CCAnswer-$\lambda$) to produce the traversal $t = t' \cdot v_{\lambda \overline{\xi}}$
    where $v_{\lambda \overline{\xi}}$ points to the first node in $t'$. Clearly, $\varphi(t-@) = u$.



    \item Suppose that $m,m^1 \in A \union B_0$.
    The strategy $ev$ is responsible for switching thread in $B_0$ therefore, in the interaction semantics,
    there must be a copycat move in-between two moves belonging to two different threads.
    Since $m$ and $m^1$ are consecutive moves in the sequence $u$, they must belong to the same thread i.e. there are
    hereditarily justified  by the same initial $m_0$ in $B_0$.


    We then have $(u \upharpoonright !A, !B)\upharpoonright m_0 = \varphi_{N_0}(t_0-@)$ for some traversal $t_0$ of $N_0$.
    Consequently  $\varphi_{N_0}(n^1) = m^1$ and $\varphi_{N_0}(n) = m$
    where $n^1 \cdot n$ are the last two moves in $t_0-@$.

    $n$ points to some node in $t_0$ that also occurs in $t'$. Let us call $n^2$ this node.
    Since $(u \upharpoonright !A, !B)\upharpoonright m_0 = \varphi_{N_0}(t_0-@)$,
    $n_2$ must have the same position in $t'$ as the node justifying $m$ in $u'$.
    Hence we just need to take $t = t' \cdot n$ where $n$ points to $n^2$ in $t'$.

    The sequence $t$ is indeed a valid traversal of $\tau(M)$
    because the rule used by the traversal $t_0$
    of $\tau(N_0)$ to visit the node $n$ after $n^1$ can also be used by the traversal $t'$ of $\tau(M)$
    to visit $n$ after $n^1$.
    This can be checked formally by inspecting all the traversal rules. The key reason is that
    all the nodes in $t_0-@$ are present in $t'$ with the same pointers but with some nodes interleaved in between.
    However these interleaved nodes are inserted in a way that still permits to use the traversal rule.

    \item Suppose that $m,m^1 \in A \union B_1$.
    The proof is similar to the previous case.

    \item Suppose that $m \in A \union B_0$ and $m^1 \in A \union B_1$.

    $t$ is obtained from $t-@$ using the transformation $+@$. We apply the same transformation to $u$ in order
    to make $O$-questions and $P$-questions in $u$ match with $\lambda$-nodes and variable nodes in $t'$ respectively.
    We write this sequence $u+@$.
    The $+@$ operation inserts nodes in the sequence but not at the end,
    therefore $m^1$, the last move in $u'$, is also the last move in $u'+@$.
    Let us note $n^1$ for the last move in $t'$.

        \begin{enumerate}
        \item If $n^1$ is the application node @ then it must be the parent of the node $\lambda y_1$ since it
        is the only non-internal @-node present in $t'$.
        Therefore $t'=\lambda \overline{\xi} \cdot @$ and $u= q_0 \cdot m$.
        But $m$ is the copy of $q_0$ replicated by $ev$ in $o'$ therefore $m=q_1$.
        Applying the (App) rule on $t'$ produces the traversal $\lambda \overline{\xi} \cdot @ \cdot \lambda y_1$
        with $\varphi((\lambda \overline{\xi} \cdot @ \cdot \lambda y_1)-@ ) = q_0 \cdot q_1 = u$.

        \item If $n^1$ is a variable node then $m^1$ is a P-move and $m$ is an O-move
            and therefore $m$ is the copy of $m^1$ duplicated in $B_1$ by the evaluation strategy.
            Consequently, $m^1$ points to some $m^2$ and $m$ points to the node preceding $m^2$ denoted by $m^3$.
            The diagram below shows an example of such sequence:
                $$
                \begin{array}{cccccccc}
                & (B_1' &\rightarrow & o') & \times & B_1 & \rightarrow & o' \\
                O & &&&&&& \rnode{q0}{q_0 (\lambda \overline{\xi})} \\
                P & &&&&& \\
                O & && \rnode{q1}{q_1 (\lambda \overline{y})} \\
                P & \rnode{m3}{m^3 (y_1)} \\
                O & &&&& \rnode{m2}{m^2 (\lambda \overline{z})} \\
                P & &&&& \rnode{m1}{m^1 (z_i)} \\
                O & \rnode{m}{m} \\
                \end{array}
                \ncline[nodesep=3pt]{->}{q1}{q0} \mput*{@}
                \nccurve[nodesep=3pt,ncurv=2,angleA=180,angleB=180]{->}{m1}{m2}
                \ncarc[nodesep=3pt,ncurv=1,angleA=90,angleB=180]{->}{m3}{q1}
                \ncarc[nodesep=3pt,ncurv=1,angleA=90,angleB=180]{->}{m}{m3}
                \ncline[nodesep=3pt]{->}{m2}{q0}
                $$

        $t'$  and $u+@$ have the following forms:
        \begin{eqnarray*}
                t'&=& \ldots \cdot n^3 \cdot \rnode{n2}{n^2} \cdot \ldots \cdot \rnode{n1}{n^1} \\ \\
                u+@ &=& \ldots \cdot \rnode{m3}{m^3} \cdot \rnode{m2}{m^2} \cdot \ldots \cdot \rnode{m1}{m^1} \cdot \rnode{m}{m}
            \bkptr{30}{m1}{m2} \bkptr{30}{m}{m3}
            \bkptr{30}{n1}{n2}
        \end{eqnarray*}

        Since $n^1$ is a variable node, $n^2$ must be a $\lambda$-node.
        $n^3$ could be either a variable node or an @-node. In fact $n^3$ is necessarily a variable node. Indeed,
        $n^3$ is mapped to $m^3$ by $\varphi_{N_0}$ and $m^3$ belongs to $\sem{B_i'}$ (i.e. it is not
        an internal move of $\intersem{B_i'}$). The function $\varphi_{N_0}$ is defined in such a way that
        only nodes which are hereditarily justified by the root of $\tau(N_0)$ are mapped to nodes in $\sem{B_1'}$.
        Hence $n^3$ is hereditarily justified by the root and consequently it cannot be an @-node.

        Hence $n^1$ is a variable node, $n^2$ is a $\lambda$-node and $n^3$ is a variable node. We
        can therefore apply the (Var) rule to $t'$ and we obtain a traversal of the following form:

        \begin{eqnarray*}
            t&=& \ldots \cdot \rnode{n3}{n^3} \cdot \rnode{n2}{n^2} \cdot \ldots \cdot \rnode{n1}{n^1} \cdot \rnode{n}{n}
            \bkptr{30}{n1}{n2} \bkptr{30}{n}{n3}
        \end{eqnarray*}

        We have $\varphi(t'-@) = u'$ by the induction hypothesis and $\varphi(n) = m$ by definition of $\varphi$.
        Therefore since $m$ and $n$ point to the same position we have $\varphi(t-@) = u$.

        \item If $n^1$ is the value-leaf of a variable node then we proceed the same way as in the previous case:
        $n^1$ is a value-leaf of the variable node $n^2$ and we can use the
        (CCAnswer-$\lambda$) rule to extend the traversal $t'$.

        \item Suppose that $n^1$ is a lambda node, in which case $m^1$ is an O-move, then
        necessarily, $m^1$ is a move copied by the evaluation strategy
         from $B_1'$ to $B_1$. The move following $m^1$ should also be played in $B_1$ before being copied
         back to $B_1'$ by the evaluation strategy. But since $m \in B_0$, this case does not happen.


        \item If $n^1$ is a value-leaf of a lambda node then $n^2$ is a lambda node and $n^3$ is a variable node.
        We can therefore use the rule (CCAnswer-var) or (CCAnswer-@) to extend the traversal $t'$.
        \end{enumerate}

    \item Suppose $m \in A \union B_1$ and $m^1 \in A \union B_0$ then
    the proof is similar to the previous case.
    \end{enumerate}


  For the converse, $\varphi_{M}( \travset^{-@}(M) ) \subseteq \intersem{M}$, it is an easy induction
  on the traversal rules. We omit the details here.
\end{itemize}

(ii) is an immediate consequence of (i):
\begin{align*}
\sem{M} &= \intersem{M} \upharpoonright \sem{\Gamma \rightarrow T} & \mbox{(eq. \ref{eqn:int_std_gamsem})} \\
        &= \varphi_M(\travset^{-@}(M)) \upharpoonright \sem{\Gamma \rightarrow T} & \mbox{(by (i))}\\
        &= \varphi_M(\travset^{\upharpoonright r}(M)) & \mbox{(lemma \ref{lem:varphi_filter})}
\end{align*}
\end{proof}


Putting corollary \ref{cor:varphi_bij} and proposition
\ref{prop:rel_gamesem_trav} together we obtain the following theorem
which establish a correspondence between the game-denotation of a
term and the set of traversals of its computation tree:

\begin{thm}[The Correspondence Theorem]
\label{thm:correspondence}
 For any pure simply-typed term $\Gamma \vdash M$,
$\varphi_M$ defines a bijection from $\travset(M)^{\upharpoonright
r}$ to $\sem{M}$ and a bijection from $\travset(M)^{-@}$ to
$\intersem{M}$:
\begin{eqnarray*}
 \varphi_M  &:& \travset(\Gamma \vdash M)^{\upharpoonright r} \stackrel{\cong}{\longrightarrow} \sem{\Gamma \vdash M} \\
 \varphi_M  &:& \travset(\Gamma \vdash M)^{-@} \stackrel{\cong}{\longrightarrow} \intersem{\Gamma \vdash M}
\end{eqnarray*}

Moreover if $M$ is in $\beta$-normal form and $s$ is a
\emph{maximal} play then  $t$ is a \emph{maximal} traversal.
\end{thm}

\begin{proof}
The first part is an immediate consequence of corollary
\ref{cor:varphi_bij} and proposition
\ref{prop:rel_gamesem_trav}.

Finally, if $M$ is in $\beta$-normal form then
$\travset(M)^{\upharpoonright r} = \travset(M)$
therefore $\varphi$ is a bijection from $\travset(M)$ to
$\sem{M}$. Suppose $s$ is a maximal play and suppose $t' \sqsubseteq
t$ then since $\varphi$ is monotonous we have $s = \varphi(t) \sqsubseteq
\varphi(t')$. But $s$ is maximal therefore $s = \varphi(t') =
\varphi(t)$ and because $\varphi$ is injective we have $t'=t$.
\end{proof}

The following diagram recapitulates the main results of this section:
$$
\xymatrix @C=6pc{
                                           & \travset(M)^{-@} \ar@/_/[dl]_{+@}  \ar[r]^{\varphi_M}_\cong & \intersem{M} \ar@/_/[dd]_{\_ \upharpoonright \sem{\Gamma\rightarrow T}} \\
\travset(M) \ar@/_/[ur]_{-@}^{} \ar[dr]^{\_ \upharpoonright r}  \\
                                           & \travset(M)^{\upharpoonright r} \ar[r]^{\varphi_M}_\cong & \sem{M} \ar@/_/[uu]^{\cong}_{\mbox{full uncovering}}
}
$$


% fourth chapter
\include{chap_gsemsafety}

% fifth chapter
\chapter{Further possible developments}

In the previous chapter, we have given an account of the game
semantics of Safe $\lambda$-Calculus. However the nature of this
calculus is still not well known. We propose the following possible
roadmap for further research:
\begin{enumerate}
\item prove or disprove that observational equivalence is decidable for Safe \ialgol;
\item find a categorical interpretation of the Safe $\lambda$-Calculus;
\item study the proof theory obtained by the Curry-Howard isomorphism and determine whether it has nice properties that can be helpful in theorem proving;
\item In \cite{DBLP:conf/tlca/LeivantM93}, the $\lambda$-calculus is used to
give several characterisations of the complexity class P. We would
like to investigate whether, by following similar techniques, we can
obtain a characterisation of a different complexity class using the
Safe $\lambda$-Calculus.
\end{enumerate}


In a more general direction of research, we would like to study the
class of languages for which pointers are uniquely recoverable. We
name this class PUR for ``Pointer Uniquely Recoverable''.

We proved that Safe $\lambda$-Calculus is a PUR-language. Another
example is the Serially Re-entrant Idealized Algol (SRIA) proposed
by Abramsky  in \cite{abramsky:mchecking_ia}. This language allows
multiple occurrences or uses of arguments, as long as they do not
overlap in time. In the game semantics denotation of a SRIA term
there is at most one pending occurrence of a question at any time.
Each move has therefore a unique justifier and consequently
justification pointers may be ignored. Safe \ialgol\ is not a
sublanguage of SRIA. One reason for this is that none of the two
Kierstead terms $\lambda f . f (\lambda x . f (\lambda y .y ))$ and
$\lambda f . f (\lambda x . f (\lambda y .x ))$ are Serially
Re-entrant whereas the first one is safe. Conversely, SRIA is not a
sublanguage of Safe \ialgol\ since the term $\lambda f g. f (\lambda
x . g (\lambda y .x ))$ where $f,g:((o,o),o)$ belongs to SRIA but
not to Safe \ialgol. SRIA and Safe \ialgol\ are therefore two
different examples of languages with pointer-less game semantics.

Finitary $\ialgol_2$ is also an example of PUR-language for which
observational equivalence is decidable. As we indicated in the first
chapter, decidability of observational equivalence is a very
appealing property which has immediate applications in the domain of
program verification. Intuitively, PUR-languages seem to be good
candidates of languages for which observational equivalence is
decidable. It would be interesting to discover classes of PUR
languages having this appealing property.

Another possible way to generate PUR-languages might be to constrain
the types of an existing language. In \cite{DBLP:conf/tlca/Joly01},
a notion of ``complexity'' is defined for $\lambda$-terms. It is
proved that a type $T$ can be generated from a finite set of
combinators if and only if there is a constant bounding the
complexity of every closed normal $\lambda$-term of type $T$;
consequently, the only inhabited finitely generated types are the
type of rank $\leq 2$ and the types $(A_1, A_2, \ldots, A_n, o)$
such that for all $i = 1..n$: $A_i = o$ , $A_i = o \rightarrow o$ or
$A_i = o^k \rightarrow o \rightarrow o$.

We know that imposing the first of these two type restrictions to
Finitary \ialgol\ leads to a PUR language. Is is also the case when
imposing the second type restriction?


\bibliographystyle{plain}
\bibliography{../../bib/gamesem,../../bib/modelchecking,../../bib/proganalys,../../bib/higherorder}

%adds the bibliography to the table of contents
\addcontentsline{toc}{chapter}
         {\protect\numberline{Bibliography\hspace{-96pt}}}
\end{document}
