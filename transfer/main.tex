\documentclass[nocenter,sfbold]{thesis}

\chapapp{Chapter}

\input preambule.tex

%\includeonly{chap_previouswork,chap_gamesem,safe_background,safe_homog,safe_nonhomog,safe_gamesem2,chap_further}
%\includeonly{safe_gamesem2}


\author{William Blum}
\title{Transfer thesis}
\institution{Oxford University Computing Laboratory}

\begin{document}
\maketitle \tableofcontents

\part{Previous work and plan of proposed work}

%\chapapp{Chapter}
\chapapp{}

\chapter{First-Year work}
%\addcontentsline{toc}{chapter}{First-Year work}

\section{Coursework}
I have attended the following courses: \emph{Automata Logic and
Games} in Hilary term 2005, \emph{Domain theory} in Michaelmas term
2005 and \emph{Categories Proofs and Programs} in Hilary term 2006.

\section{Teaching}

I was the demonstrator for \emph{Network and Operating Systems}
practicals in Hilary term 2005, I tutored two groups of students for
the \emph{Introduction to Specification} classes (Hilary 2006) and I
was the marker for one group.

\section{Meetings and conferences}
\begin{itemize}
\item I attended Bonn spring school on GAMES in March 2005;

\item  I attended BCTCS (British Colloquium in
Theoretical Computer Science) in Nottingham in March 2005 where I
gave a presentation based on my MSc dissertation ``Termination
analysis of a subset of CoreML'';

\item I attended PAT \emph{Program transformation and Analysis} in Copenhagen, July 2005;

\item Marktoberdorf Summer School;
\item CSL (Computer Science Logic) August 2005:
I helped to organise the conference;
\item I visited the Isaac Newton Institute in Cambridge in February
2006.
\end{itemize}
I have also done a presentation during the Computer Laboratory open
days.


\section{Research}

\subsection{Game semantics}

During the past months, I have studied a restriction of
lambda-calculus called ``safe lambda-calculus''. \emph{Safety} is a
syntactic property originally defined in \cite{KNU02} for
higher-order recursion schemes (grammars). In their paper they
proved that the MSO theory of the term tree generated by a safe
recursion scheme of level $n$ is decidable. More recently, Ong
proved in \cite{OngLics2006} that the safety assumption is in fact
not necessary for the decidability of MSO theories.

I am interested in the transposition of the safety property from
grammars to lambda terms. A definition of the safe
$\lambda$-calculus was first given in a technical report by Aehlig,
de Miranda and Ong in \cite{safety-mirlong2004}. One interesting
property is that performing substitution on safe terms does not
require a renaming of the variable.

I have investigated different possible definitions of a safe lambda
calculus and have proposed a more general notion of safety that does
not assume homogeneity of types while still preserving  the ``no
variable renaming'' property.

I also tried to relate the safety restriction and the
\emph{size-change termination} property defined in
\cite{jones01,jones04}. Jones conjectured that any simply-typed term
is size-change terminating, however Damien Sereni disproved this
conjecture by exhibiting a class of counter-examples
(\cite{serenistypesct05}). It turns out that the simply-typed terms
of this class are all safe (but not necessary of homogeneous type)
and not size-change terminating. This suggests that there is no real
interesting relation between safety and size-change termination.


Recently, inspired by my reading on game semantics
\citep{abramsky:game-semantics-tutorial} and by the techniques
developed by Luke Ong in \citep{OngLics2006}, I have proved a result
on the game semantics of safe terms: the pointers in the game
semantics of safe simply-typed terms can be recovered uniquely from
the sequence of moves. This result is similar to the standard result
in game semantics which says that pointers of strategies can be
recovered uniquely for arena of order 2 at most.


\subsection{Verification}

In parallel, I worked on a separate project with Matthew Hagues and
Luke Ong. We developed a SAT-based  model checker for verifying
Linear Temporal Logic (LTL) formulae on programs expressed as finite
state machines. Our approach combines techniques presented in two
papers: \cite{hammer:truly, DBLP:conf/cav/McMillan03}.

In \cite{DBLP:conf/cav/McMillan03}, McMillan describes an
acceleration technique for the SAT-based Bounded Model Checking
problem based on Craig interpolants. His algorithm significantly
improves the performance of the standard SAT-based model checking
method in the case of positive instances.

In \citep{hammer:truly}, Hammer \emph{et al.} introduced a new kind
of automata called \emph{Linearly Weak Alternating Automata},
abbreviated LWAA. The set of languages recognized by these automata
are exactly the set of languages definable in LTL. There is a
straightforward translation from LTL formulae to LWAAs. The size of
the resulting automaton is linear in the size of the LTL formula.
Checking emptiness of LWAA then amounts to searching the
configuration graph for a lasso verifying certain conditions.

Our approach can be summarized as follows: we translate the model
checking problem into an emptiness checking of LWAA. The automata is
empty if and only if the formula is true. The emptiness of the
automaton is expressed in term of a reachability problem. As in the
traditional SAT-based bounded-model checking approach
(\cite{biere99symbolic}), we construct a boolean formula which is
satisfiable if and only if the desired configuration is reachable in
at most $k$ steps (i.e. there is a counter-example of length $k$ at
most).

Furthermore, instead of using the traditional SAT-solver technique,
which iterates $k$ until the completeness threshold is reached, we
use the acceleration method described in
\cite{DBLP:conf/cav/McMillan03}. The principle is the following: for
every iteration of $k$, if the formula is not satisfiable then we
perform some over-approximation of the set of initial configuration.

Suppose that the final configuration becomes reachable in $k$ steps
from the over-approximated initial configuration then we are still
uncertain whether the formula has a valid counter-example because
the counter-example obtained may be spuriously created by the
over-approximation. We therefore increase $k$ and move on to the
next iteration. However, if after performing several
over-approximations we reach a fixed point and the formula is still
not satisfiable (not counter-example of length $k$ at most) then we
know that there cannot be any counter-example of any length. We have
therefore reached the completeness threshold and we know that the
formula is true.


There are two reasons why we think that our approach may lead to a
gain of performance. Firstly, although determining emptiness of a
LWAA is more costly than determining emptiness of a B\"uchi
automaton, we save time during the construction of the automaton
because the size of a LWAA is linear in the length of the formula as
opposed to the standard translation which produces a B\"uchi
automaton of size exponential in the length of the formula.
Secondly, in the case where there is no counter-example, McMillan's
acceleration method based on over-approximation permits quick
detection of attaintment of the completeness threshold.


%\cite{ckos2005}
We have produced an experimental implementation in OCaml and C. The
program parses a file in the NuSMV format (\cite{CAV02:nusmv})
containing the kripke structure of the model and the set of LTL
properties to verify. Our tools can be interfaced with two SAT
solvers: ZChaff \citep{zChaff} and MiniSat \citep{ES03}. We also use
BDD to perform simplification on the propositional formula and to
generate the CNF representation that the SAT solver takes as input.

Compared to the LWAASpin LTL model checker (\cite{hammer:truly}),
our tool performs quite poorly. As soon as a model is taken into
account, our procedure generates increasingly bigger propositional
formulae that the SAT solver struggles to solve. However, for pure
LTL emptiness checking, our tool performs quite well.

It seems disappointing that our approach does not give good results
for model checking, however the reason seems to be that the
SAT-solvers we are using produce bad interpolants. In the future, we
would like to interface our model checking tool with other SAT
solvers and interpolers.

Furthermore, there are optimizations that we have not finished to
implement. These include the optimization of the encoding of the
bounded model checking problem into a propositional formula. We
propose to do some experimental tests do discover the encoding
giving the best performance.

\chapter{Research plan}
%\addcontentsline{toc}{chapter}{Research plan}

My research plan for the coming year is as follows: first I will
continue to work on the Safe $\lambda$-calculus. My immediate goal
is to extend the result I obtained about the unique recoverability
of pointers in the game semantics of Safe simply-typed
$\lambda$-calculus to the case of other languages like Safe
Idealized Algol. I also wish to investigate applications in
algorithmic game semantics. There are also further questions about
Safe $\lambda$-calculus that have to be addressed: what is the
categorical interpretation of Safe $\lambda$-calculus? What kind of
proof theory do we obtain by the Curry-Howard isomorphism? Which
complexity class is characterized by the Safe-$\lambda$ calculus?

In parallel to that line of research, I will continue to work with
Matthew Hagues and Luke Ong on the LTL model checking problem.


%I also want to investigate some application of game semantics to
%program analysis and transformation by trying to extend the work of
%Dimovski \emph{et al.} (\cite{DBLP:conf/sas/DimovskiGL05}) on
%data-absraction refinement based on game semantics.



\part{Research proposal}

The first chapter of this part is devoted to the presentation of the
basics and main results of game semantics. The categorical
interpretation of game semantics is presented as well as the full
abstraction result for \pcf. We also give a brief summary of the
main results in algorithmic game semantics. There is no personal
contribution in this chapter.

In the second chapter we present the \emph{Safe $\lambda$-calculus}.
Originally, \emph{safety} has been introduced as a syntactical
restriction on higher-order grammars in order to show a decidability
result about MSO theory of infinite trees \citep{KNU02}. In
\cite{safety-mirlong2004}, Aehlig, de Miranda and Ong  proposed an
adaptation of the safety restriction to the $\lambda$-calculus. This
restriction gives rise to the Safe $\lambda$-calculus. We first
present this calculus and then give a more general definition which
does not make any assumption on the types of the terms.

In the third chapter, following ideas described in
\cite{OngLics2006}, we introduce the notions of computation tree of
a simply-typed term and traversal over a computation tree. We prove
a theorem showing a correspondence between traversals of the
computation tree and the game semantics of a term. Based on that
correspondence, we give a characterisation of the game semantics of
safe terms by a property called ``incremental justification''. In
incrementally-justified strategies, pointers are superfluous (i.e.
they can be recovered uniquely from the sequence of moves). This
simplification of the game semantics suggests some potential
applications in algorithmic game semantics. We finish the chapter by
extending the result to Safe \pcf\ and by giving the key elements
for an extension to full Safe Idealized Algol.


% first chapter
\chapter{Game semantics}

The aim of this chapter is to introduce game semantics. It starts
with a history of game semantics and a presentation of the full
abstraction problem for PCF which has been solved using game
semantics. It then goes on by introducing the basic notions of game
semantics and by giving a categorical interpretation of games.
Finally we show how games are used to define a syntax-independent
model of programming languages like PCF and Idealized Algol (IA).

This chapter is largely based on the tutorial by Samson Abramsky
tutorial on Game Semantics \cite{abramsky:game-semantics-tutorial}.
Many details and proofs will be omitted and we refer the reader to
\cite{hylandong_pcf, abramsky94full} for a complete description of
game semantics.

\section{History}

\subsection{Game semantics}

In the 1950s, Paul Lorenzen invented Game semantics as a new
approach to study semantics of intuitionistic logic \citep{lor61}.
In this setting, the notion of logical truth is modeled using game
theoretic concepts such as the existence of winning strategy.

Four decade later, game semantics is used to prove the full
completeness of Multiplicative Linear Logic (MLL)
\citep{abramsky92games,HO93a}. Shortly after, a connection between
games and linear logic has been established. Game semantics has then
been used as a new paradigm to study formal models of programming
languages. The idea is to model the execution of a program as a game
played by two protagonists: the Opponent representing the
environment and the Proponent representing the system. The meaning
of the program is then modeled by a strategy for the Proponent.


Subsequently, these game-based model have been used to give a
solution to the long-standing problem of ``Full abstraction of PCF''
\citep{abramsky94full, hylandong_pcf,Nickau:lfcs94}.

Based on that major result, and in a more applied direction, games
have been used as a new tool for software verification
\cite{ghicamccusker00}. This open-up a new field called Algorithmic
Game Semantics \citep{Abr02}.




\subsection{Model of programming languages}

Before the 1980s, there were many approaches to define models for
programming languages. Among the successful ones, there were the
axiomatic, operational and denotational semantics:
\begin{itemize}
\item Operational semantics gives a meaning to a program by describing the
behaviour of a machine executing the program. It is defined formally
by giving a state transition system.
\item Axiomatic semantics defined the behaviour of the program
with axioms and is used to prove program correctness by static
analysis of the code of the program.
\item The denotational semantics approach consists in mapping a program to a mathematical structure
having good properties such as compositionality. This mapping is
achieved by structural induction on the syntax of the program.
\end{itemize}

In the 1990s, three different independent research groups: Samson
Abramsky, Radhakrishnan Jagadeesan and Pasquale Malacaria
\citep{abramsky94full}, Martin Hyland and Luke Ong
\citep{hylandong_pcf} and Nickau \citep{Nickau:lfcs94} have
introduced game semantics, a new kind of semantics, in order to
solve a long standing problem in the semanticists community :
finding a fully abstract model for PCF.

\subsection{The problem of full abstraction for PCF}

PCF is a simple programming language introduced in a classical paper
by Plotkin ``LCF considered as a programming language''
(\cite{DBLP:journals/tcs/Plotkin77}). PCF is based on LCF, the Logic
of Computable Functions devised by Dana Scott in \cite{scott_lcf}.
It is a simply typed lambda calculus extended with arithmetic
operators, conditional and recursion.

The problem of the Full Abstraction for PCF goes back to the 1970s.
In \citep{scott93}, Scott gave a model for PCF based on domain
theory. This model gives a sound interpretation of observational
equivalence: if two terms have the same domain theoretic
interpretation then they are observationally equivalent. However the
converse is not true: there exist two PCF terms which are
observationally equivalent but have different domain theoretic
denotation. We say that the model is not fully abstract.

The key reason why the domain theoretic model of PCF is not fully
abstract is that the parallel-or operator defined by the following
truth table
\begin{center}
\begin{tabular}{l|lll}
p-or  & $\bot$ & tt & ff \\ \hline
$\bot$ & $\bot$ & tt & $\bot$\\
tt & tt & tt & tt\\
ff & $\bot$ & tt & ff\\
\end{tabular}
\end{center}
is not definable as a PCF term! It is possible to create two
different PCF terms that always behave the same except when they are
apply to a term computing p-or. Since p-or is not definable in PCF,
these two terms will have the same denotation. This implies that the
model is not fully abstract.


It is possible to patch PCF by adding the operator $p-or$, the
resulting language ``PCF+p-or'' becomes fully-abstracted by Scott
domain theoretic model \citep{DBLP:journals/tcs/Plotkin77}. However
the language we are now dealing with is strictly more powerful than
PCF, it allows parallel execution of commands whereas PCF only
permits sequential execution.

Another approach consists in eliminating  the undefinable elements
(like p-or) by strengthening the conditions on the function used in
the model. This approach has been followed by Berry in
\cite{berry-stable,gberry-thesis} where he gives a model based on
stable functions, a class of function smaller than the class of
strict and continuous function. Unfortunately this approach did not
succeed.

The only successful approaches to obtain a fully abstract model for
PCF were the ones taken by Ambramsky, Jagadeesan and Malacaria
\citep{abramsky94full}, Hyland and Ong \citep{hylandong_pcf} and
Nickau \citep{Nickau:lfcs94}, all based on game semantics.

This result has then been adapted to other varieties of programming
paradigm including languages with stores (Idealized Algol),
call-by-value \citep{honda99gametheoretic, abramsky98callbyvalue}
and call-by-name, general referencees
\citep{DBLP:conf/lics/AbramskyHM98}, polymorphism
\citep{DBLP:journals/apal/AbramskyJ05}, control features
(continuation and exception), non determinism, concurrency. In all
these cases, the game semantics model led to a syntax-independent
fully abstract model of the corresponding language.

\section{Games}
\label{sec:catgames}

We now introduce formally the notion of game that will be used in
the following section to give a model of the programming languages
PCF and Idealized Algol. The definitions are taken from
\cite{abramsky:game-semantics-tutorial, hylandong_pcf,
abramsky94full}.


\subsection{Arenas and Games}

The games we are interested in are two-players games. The players are named O for Opponent and P for Proponent.

The game played by O and P is constraint by something called
\emph{arena}. The arena defines the possible moves of the game. By
analogy with real board games, the arena represents the board
together with the rules that tell how players can make their moves
on the board. In fact the analogy with board game stops here. Our
games can be thought as dialog games: one person O interviews
another person P, P tries to answer the initial O-question by
possibly asking O some precisions about its initial question.
Moreover, the notion of winner and winning strategy will not be
relevant in our setting.


More formally, the arena can be seen as a forest of trees whose nodes are possible questions and leaves are possible answers.
The arena is partitioned into two kinds of moves: the moves that can be played by P and the ones that can be played by O.
A move is either a question to the other player or an answer to a question previously asked by the other player.

Each move of the game must be justified by another move that has already been played by the other player. This justification relation
is induced by the edges of the forest arena. Moreover, an answer must always be justified by the question that it answers and a question
is always justified by another question.

\begin{dfn}[Arena]
An arena is a structure $\langle M, \lambda, \vdash \rangle$ where:
\begin{itemize}
\item $M$ is the set of possible moves;
\item $(M,\vdash)$ is a forest of trees;

\item $\lambda : M \rightarrow \{ O, P\} \times \{Q, A\}$ is a labeling functions indicating whether a given move
    is a question or an answer and whether it can be played by O or by P.

    $\lambda = [\lambda^{OP},\lambda^{QA}]$ where $\lambda^{OP} : M \rightarrow  \{ O, P\}$
    and $\lambda^{QA} : M \rightarrow  \{ Q, A\}$.

    \begin{itemize}
    \item If $\lambda^{OP} (m) = O$, we call $m$ and O-move otherwise $m$ is a P-move.
    $\lambda^{QA} (m) = Q$ indicates that $m$ is a question otherwise $m$ is an answer.

    \item For any leaf $l$ of the tree $(M,\vdash)$, $\lambda^{QA} (l) = A$ and for any node
    $n \in (M,\vdash)$, $\lambda^{QA} (n) = Q$.
    \end{itemize}

\item The forest of tree $(M,\vdash)$ respect the following condition:
    \begin{itemize}
    \item[(e1)] The roots are O-moves: for any root $r$ of $(M,\vdash)$, $\lambda^{OP} (r) = O$.
    \item[(e2)] Answers are enabled by questions: $m \vdash n  \zand \lambda^{QA}(n) = A \imp \lambda^{QA}(m) = Q$.
    % Or more succinctly, if we write $\dashv$ the relation $\vdash^-1$: $\lambda^{QA} \left( \dashv( (\lambda^{QA})^{-1}(\{A\}) ) \right) = \{ O \}$
    \item[(e3)] A player move must be justified by a move played by the other player:
         $m\vdash n \imp \lambda^{OP}(m) \neq \lambda^{OP}(n)$.
    \end{itemize}
\end{itemize}
\end{dfn}

For commodity we write the set $\{O,P\} \times \{Q,A\}$ as $\{OQ,OA,PQ,PA\}$.
$\overline{\lambda}$ denotes the labeling function $\lambda$ with the question and answer swapped. For instance:
$$\overline{\lambda(m)} = OQ \iff \lambda(m) = PQ$$

The roots of the forest of tree $(M,\vdash)$ are the \emph{initial moves}.

For example, the simplest possible arena is written $\mathbf{1}$ and
denotes the arena which set of moves $M$ is empty.

\begin{exmp}[The flat arena]
\label{exmp:flatarena}

 Let $A$ be any countable set then the flat arena over $A$
is defined to be the arena $\langle M, \lambda, \vdash \rangle$ such
that $M$ has one move $q$ with $\lambda(q) = OQ$ and for each
element in $A$, there is a corresponding move $a_i$ in $M$ with
$\lambda(a_i) = PA$ for some $i \in \nat$. The enabling relation
$\vdash$ is defined to be $\{ q \vdash a_i \ | i \in \nat \}$.

This arena is represented by the following tree:
\begin{center}
  \pstree[levelsep=6ex]
    { \TR{$q$} }
    {    \TR{$a_1$} \TR{$a_2$} \TR{\ldots} }
\end{center}
The vertices represent the moves and the edges represent the
enabling relation.

The flat arena over $\nat$ and $\mathbb{B}$ is written
$\mathbf{int}$ and  $\mathbf{bool}$ respectively.

\end{exmp}

Once the arena has been defined, the bases of the game are set and the players have something to play with.
We now need to describe the state of the game, for that purpose
we introduced \emph{justified sequences of moves}. Sequence of moves are used to record the history of all the moves that have been
played.

\begin{dfn}[Justified sequence of moves]
A justified sequence is a sequence of moves $s$ together with an associated sequence of pointers. Any
move $m$ in the sequence that is not initial has as pointer that points to a previous move $n$ that justifies it (i.e. $n \vdash m$).
\end{dfn}

The pointers of a justified sequences are represented with arrows.
This is an example of justified sequence of moves:
$$\rnode{q4}{q}^4
\rnode{q3}{q}^3 \rnode{q2}{q}^2 \rnode{q3b}{q}^3 \rnode{q2b}{q}^2
\rnode{q1}{q}^1 \bkptrc{q3}{q4} \bkptrc{q2}{q3}
\bkptrc[ncurv=0.6]{q3b}{q4} \bkptrc{q2b}{q3b}$$

The first move of a justified sequence must be an O-move since
initial moves are all O-moves.

Notation: we write $s t$ or sometimes $s \cdot t$ do denote the
sequences obtain by concatenating $s$ and $t$. The empty sequence is
written $\epsilon$.

 A justified sequence has two particular subsequences which
will be of particular interest later on when we introduce
strategies. These subsequences are called the P-view and the O-view
of the sequence. The idea is that a view describes the local context
of the game. Here is the formal definition:

\begin{dfn}[View]
Given a justified sequence of moves $s$. We define the proponent view (P-view) noted $\pview{s}$ by induction:
\begin{align*}
\pview{\epsilon} &= \epsilon \\
\pview{s \cdot m} &= \pview{s} \cdot \ m && \mbox{ if $m$ is a P-move} \\
\pview{s \cdot m} &= m && \mbox{ if $m$ is initial (O-move) } \\
\pview{ s \cdot \rnode{m}{m} \cdot t \cdot \rnode{n}{n} \bkptra{50}{n}{m} } &=
 \pview{s} \cdot \rnode{mm}{m} \cdot \rnode{nn}{n} \bkptra{70}{nn}{mm} && \mbox{ if $n$ is a non initial O-move }
\end{align*}
The O-view $\oview{s}$ is defined similarly:
\begin{align*}
\oview{\epsilon} &= \epsilon \\
\oview{s \cdot m} &= \oview{s} \cdot \ m && \mbox{ if $m$ is a O-move} \\
\oview{ s \cdot \rnode{m}{m} \cdot t \cdot \rnode{n}{n} \bkptra{50}{n}{m} } &=
 \pview{s} \cdot \rnode{mm}{m} \cdot \rnode{nn}{n} \bkptra{70}{nn}{mm} && \mbox{ if $n$ is a P-move }
\end{align*}
\end{dfn}

In fact not all justified sequences will be of interest for the
games that we will use. We call \emph{legal position} any justified
sequence verifying two additional conditions: alternation and
visibility. Alternation says that players O and P plays
alternatively. Visibility expresses that each non-initial move is
justified by a move situated in the local context at that point.
Intuitively, the visibility condition gives some coherence to the
justification pointers of the sequence.

\begin{dfn}[Legal position]
A legal position is a justified sequence of move $s$ respecting the following constraint:
\begin{itemize}
\item Alternation: For any subsequence $m \cdot n$ of $s$, $\lambda^{OP}(m) \neq \lambda^{OP}(n)$.
\item Visibility: For any subsequence $t m$ of $s$ where $m$ is not initial, if $m$ is a P-move then $m$ points to a move in $\pview{s}$
and if $m$ is a O-move then $m$ points to a move in $\oview{s}$.
\end{itemize}

The set of legal position of an arena $A$ is noted $L_A$.
\end{dfn}

We say that a move $n$ is hereditarily justified by a move $m$ if there is a sequence of move
$m_1, \ldots, m_q$ such that:
$$ m \vdash m_1 \vdash m_2 \vdash \ldots m_q \vdash n$$
If a move has no justification pointer, we says that it is an
\emph{initial move} (in that case it must be a root of the forest
arena).

Suppose that $n$ is an occurrence of a move in the sequence $s$ then
$s \upharpoonright n$ denotes the subsequence of $s$ containing all the moves hereditarily justified by $n$.
Similarly, $s \upharpoonright I$ denotes the
subsequence of $s$ containing all the moves hereditarily justified by the moves in $I$.

\begin{dfn}[Game]
A game is a structure $\langle M, \lambda, \vdash, P \rangle$ such that
\begin{itemize}
\item $ \langle M, \lambda, \vdash \rangle$ is an arena.
\item $P$ is called the set of valid positions, it is:
    \begin{itemize}
    \item a non-empty prefix closed subset of the set of legal position
    \item closed by initial hereditary filtering: if $s$ is a valid position then for any set $I$ of occurrences of initial moves
    in $s$, $s\upharpoonright I$ is also a valid position.
    \end{itemize}
\end{itemize}
\end{dfn}

\begin{exmp}  Consider the flat arena  $\mathbf{int}$.
The set of valid position $P = \{ \epsilon, q \} \union \{ q \cdot
a_i \ | i \in \nat \}$ defines a game on the arena $\mathbf{int}$.
\end{exmp}

\subsection{Constructions on games}
\label{sec:gameconstruction}

We now define game constructors that will be useful later on.

Consider the two functions $f : A \rightarrow C$ and $g : B
\rightarrow C$, we write $[f,g]$ to denote the pairing of $f$ and
$g$ defined on the direct sum $A + B$. Given a game $A$ with a set
of moves $M_A$, we use the filtering operator $s \upharpoonright A$
do denote the subsequence of $s$ consisting of all moves in $M_A$.
Although this notation conflicts with the hereditarily filtering
operator, it should not cause any confusion.

\subsubsection{Tensor product}
Given two games $A$ and $B$ we define the tensor product constructor
$A \otimes B$ as follows:
\begin{eqnarray*}
  M_{A \otimes B} &=& M_A + M_B \\
  \lambda_{A\otimes B} &=& [\lambda_A,\lambda_B] \\
  \vdash_{A\otimes B} & = & \vdash_{A}\ \union\ \vdash_{B} \\
  P_{A\otimes B} & = & \{ s \in L_{A\otimes B} | s \upharpoonright A \in P_A \wedge s \ \upharpoonright B \in P_B  \}.
\end{eqnarray*}

In particular,  $n$ is initial in $A\otimes B$ if and only if $n$ is
initial in A or B. And $m \vdash_{A\otimes B} n$  holds if and only if $m
\vdash_{A} n$ or $m \vdash_{B} n$ holds.

\subsubsection{Function space}
The game $A \otimes B$ is defined as follows:
\begin{eqnarray*}
  M_{A \multimap B} &=& M_A + M_B \\
  \lambda_{A\multimap B} &=& [\overline{\lambda_A},\lambda_B] \\
  \vdash_{A\multimap B} & = & \vdash_{A}\ \union\ \vdash_{B}\ \union\  \{ (m,n) \ |\ m \mbox{ initial in } B \wedge n \mbox{ initial in } A \} \\
  P_{A\otimes B} & = & \{ s \in L_{A\otimes B} | s \upharpoonright A \in P_A \wedge s \ \upharpoonright B \in P_B  \}.
\end{eqnarray*}

Graphically if we draw a triangle to represent an arena $A$ then the
arena for $A \multimap B$ is represented as follows:
\begin{center}
\psset{xunit=.5pt,yunit=.5pt,runit=.5pt}
\begin{pspicture}(150,80)
\rput[tr](150,80){ \pnode(27,40){b} \pstribox{B} } \rput[bl](0,0){
\pnode(27,40){a} \pstribox{A} } \ncline{->}{a}{b}
\end{pspicture}
\end{center}

\subsubsection{Cartesian product}
The game $A \& B$ is defined as follows:
\begin{eqnarray*}
  M_{A \& B} &=& M_A + M_B \\
  \lambda_{A\& B} &=& [\lambda_A,\lambda_B] \\
  \vdash_{A\& B} & = & \vdash_{A}\ \union\ \vdash_{B} \\
  P_{A\& B} & = & \{ s \in L_{A\otimes B} | s \upharpoonright A \in P_A \wedge s \ \upharpoonright B = \epsilon  \} \\
        &&   \union \{ s \in L_{A\otimes B} | s \upharpoonright A \in P_B \wedge s \ \upharpoonright A = \epsilon  \}.
\end{eqnarray*}

A play of the game $A \& B$ is either a play of $A$ or a play of $B$ whether a play
of the game $A \otimes B$ may be an interleaving of plays on $A$ and plays on $B$.

\subsection{Representation of plays}

Plays of the game are usually represented in a table diagram. The
columns of the table correspond to the different components of the
arena and each row corresponds to one move in the play. The first
row always represents an O-move, this is because O is the only
player who can open a game (since roots of the arena are O-moves).

As an example the play
$$\rnode{q1}{q}\
 \rnode{q2}{q}
 \ \rnode{a2}{8}
\  \rnode{a1}{12}
  \bkptrc{a1}{q1}
\bkptrc{a2}{q2} $$
on the
game $\textbf{int} \multimap \textbf{int} $ can be represented by
the following diagram:

\begin{center}
\begin{tabular}{cccc}
\textbf{int} & $\imp$ & \textbf{int} & \\
&& q & O\\
q  &&& P\\
8  &&& O\\
&& 12 & P
\end{tabular}
\end{center}

When it is necessary, the justification pointers of the play can also
be shown on the diagram.


\subsection{Strategy}

\subsubsection{Definition}

During a game, the player who has to play may have several choices
for his next move. The move that he makes is chosen according to a
given strategy.

A strategy is a rule telling the player which move to make when the
game is in a given position. More abstractly, a strategy is a
partial function mapping legal position where Proponent has to move
to P-moves.

\begin{dfn}[Strategy]
A strategy for player P on a given game $\langle M, \lambda, \vdash, P \rangle$ is a
non-empty set of even-length positions from $P$ such that:
\begin{enumerate}
\item (\emph{no unreachable position}) $sab \in \sigma \imp s \in \sigma$
\item (\emph{determinacy}) $sab, sac \in \sigma \quad \imp \quad  b = c$  and $b$ has the same justifier as
$c$.
\end{enumerate}
\end{dfn}

The idea is that the presence of the even-length sequence $s a b$ in
$\sigma$ tells the player P that whenever the game is in position
$s$ and player O plays the move $a$ then it must respond by playing
the move $b$.

The first condition ensures that the strategy $\sigma$ only
considers positions that the strategy itself could have led to in a
previous move. The second condition in the definition requires that
this choice of move is deterministic (i.e. there is a function $f$
from the set of odd length position to the set of moves $M$ such
that $f(s a) = b$).


For any game $A$, the smallest possible strategy is the strategy
that never respond given by $\{ \epsilon \}$. It is called the
\emph{empty strategy} and denoted $\bot$.

\subsubsection{Copy-cat strategy}

For any arena $A$ there is a strategy on the game $A \multimap A$
called the \emph{copy-cat strategy}. We write $A_1$ and $A_2$ to
denote the first and second copy of the arena $A$ in the game $A
\multimap A$. If $A$ is the arena $A_1$ then $A^\perp$ denotes the
arena $A_2$ and reciprocally.

Let $A$ be one of the arena $A_1$ or $A_2$. The copy-cat strategy
operates as follows: whenever P has to respond to an O-move played
in $A$, it replicates the move played by O in the arena $A^{\perp}$
after that $O$ has to respond in $A^{\perp}$ and $P$ replicates this
response in $(A^\perp)^\perp = A$ and so on and so forth.


More formally, the copy-cat strategy is defined by:
$$ \textsf{id}_A = \{ s \in P^{\textsf{even}}_{A \multimap A} \ | \ \forall t \sqsubseteq^{\textsf{even}} s\ .\ t \upharpoonright A_1 = t \upharpoonright A_2 \}$$
where $P^{\textsf{even}}_A$ denotes the valid position of even
length in the game $A$ and $t \sqsubseteq^{\textsf{even}} s$ denotes
that $t$ is an even length prefix of $s$.

The copy-cat strategy is also called \emph{identity strategy} since
it is the identity for strategy composition as we will see in the
next paragraph.

\begin{exmp} The copy-cat strategy on $\textbf{int}$ is:
$$\begin{array}{ccc}
\textbf{int} & \imp & \textbf{int} \\
&& q\\
q \\
n \\
&& n
\end{array}
$$
Note that we introduced this type of diagram to represent plays of
games but, as we can see here, the same diagrams can be used to
represent strategies when the play represented is general enough.

The copy-cat strategy on $\textbf{int} \typar \textbf{int}$ is given
by the following diagram:
$$\begin{array}{ccccccc}
(\textbf{int} & \imp & \textbf{int}) & \imp & (\textbf{int} & \imp & \textbf{int}) \\
&&&& && q\\
&& q\\
q \\
&&&& q \\
&&&& m \\
m\\
&& n \\
&&&& && n
\end{array}$$
\end{exmp}

\subsubsection{Composition}

It is well-known that any model of the simply typed lambda-calculus
is a cartesian closed category \citep{CroleRL:catt}. Games are used
to give a fully-abstract model of PCF, an extended simply typed
lambda calculus, therefore the game model should fit into a
cartesian closed category. This category will have games as objects
and strategies as morphisms. In a category, morphisms should be able
to compose together, therefore there should be an appropriate notion
of strategy composition.

Composition of strategies is an essential feature of game semantics.
As we will see in the following section, in the game model of PCF,
strategies represent programs. Therefore, strategy composition will
prove to be very useful : obtaining the model of a composed program
boils down to composing the strategies of the composing programs.

The way composition is defined for strategies is similar to
``parallel composition plus hiding'' in the trace semantics of CSP
\citep{hoare_csp}. Consider two strategies $\sigma : A \multimap B$
and $\tau : B \multimap C$ that we wish to compose.

For any sequence of moves $u$ on three arenas $A$, $B$, $C$, we call
projection of $s$ on the game $A \multimap B$ and we note $u
\upharpoonright A,B$ the subsequence of $s$ obtained by removing
from $u$ the moves in $C$ and pointers to moves in $C$. The
projection on $B \multimap C$ is defined similarly.

The definition of the projection on $A \multimap B$ differs
slightly: $u \upharpoonright A,C$ is the subsequence of $u$
consisting of the moves from $A$ and $C$ with some additional
pointers: we add a pointer from $a \in A$ to $c\in C$ whenever $a$
points to some move $b \in B$ itself pointing to $c$. All the
pointers to moves in $B$ are removed.


First we remark that for a given legal position $s$ in the game $A
\multimap C$, there is what is called an \emph{uncovering} of $s$.
The uncovering of $s$ is the maximal justified sequence of moves $u$
from the games $A$, $B$ and $C$ such that:
\begin{itemize}
\item The sequence $s$, considered as a pointer-less sequence, is a subsequence of
$u$;
\item the projection of $u$ on the game $A \multimap B$ lies in the
strategy $\sigma$;
\item the projection of $u$ on the game $B \multimap C$
lies in the strategy $\tau$;
\item and the projection of $u$ on the game $A \multimap C$ is a subsequence of $s$ (here the term ``subsequence'' refers to the sequence of nodes together with the auxiliary sequence of pointers).
\end{itemize}
This uncovering, noted $uncover(s, \sigma, \tau)$, is
defined uniquely for given strategies $\sigma$, $\tau$ and legal
position $s$ (this is proved in part II of \cite{hylandong_pcf}).

We define $\sigma \| \tau $ to be the set of uncovering of legal
positions in $A \multimap C$:
$$ \sigma \| \tau = \{ uncover(s, \sigma, \tau) \ | \ s \mbox{ is a legal position in } A \multimap C \}$$

The composition of $\sigma$, $\tau$ is defined to be the set of
projections of uncovering of legal positions in $A \multimap C$:

\begin{dfn}[Strategy composition]
Consider $\sigma : A \multimap B$ and  $\tau : B \multimap C$ two
strategies. We define $\sigma ; \tau$ to be:
$$ \sigma ; \tau = \{ u \upharpoonright A,C \ | \ u \in \sigma \|
\tau \}$$
\end{dfn}

It can be verified that composition is well-defined and associative
\citep{hylandong_pcf} and that the copy-cat strategy $\textsf{id}_A$ is the identity for composition.

\subsubsection{Constraint on strategies}

Different classes of strategies will be considered depending on the
features of the language that we want to model. Here is a list of
common restrictions that we will consider:
\begin{itemize}
\item \emph{Well-bracketing:} In a well-bracketed strategies the players always answer the last unanswered question (called the pending question) first.
If we represent Opponent's question as ``['', Proponent's answer as
``]'', Proponent's question as ``('' and Opponent's answers as ``)''
then requiring that the last pending question is answered first is
the same as requiring that the string representing the play is a
prefix of a well-bracketed sequence.

\item \emph{History-free strategies:} A strategy is history-free if the Proponent's move at any position of the game where he has to play
is determined by the last move of the Opponent. In other words, the
history prior to the last move is ignored by the Proponent when
deciding how to respond.

\item \emph{History-sensitive strategies:} The Proponent follows a history-sensitive strategy if he needs to have access to the full
history of the moves in order to decide which move to make.

\item \emph{Innocence:} a strategy is innocent if it determines Proponent's moves based on a restricted view of the history of the play, mainly the P-view
at that point. Such strategies can be specified by a partial
function mapping P-views to P-moves called the \emph{view function}. However not every partial
function from P-views to P-moves gives rise to an innocent strategy
(a sufficient condition is given in \cite{hylandong_pcf}).
\end{itemize}

The formal definition of innocence follows:
\begin{dfn}[Innocence]
Given positions $sab, ta \in L_A$ where $sab$ has even length and
$\pview{sa} = \pview{ta}$, there is a unique extension of $ta$ by
the move $b$ together with a justification pointer such that
$\pview{sab} = \pview{sa}$. We write this extension
$\textsf{match}(sab,ta)$.

The strategy $\sigma:A$ is \emph{innocent} if and only if:
$$ \left(
     \begin{array}{c}
       \pview{sa} = \pview{ta} \\
       sab \in \sigma \\
       t\in \sigma \wedge ta \in P_A \\
     \end{array}
   \right)
\quad \imp\quad  \textsf{match}(sab,ta) \in \sigma$$

\end{dfn}


\subsection{Categorical interpretation}

In this section we recall some results about the categorical representation of Games.
These results with complete details and proofs can be found in \cite{McC96b,hylandong_pcf,abramsky94full}.
We refer the reader to \cite{CroleRL:catt} for more information about category theory.

We consider the category $\mathcal{G}$ whose objects are games and morphisms are
strategies. A morphism from $A$ to $B$ is a strategy on the game $A \multimap B$.

Three other sub-categories of $\mathcal{G}$ are considered: each of them correspond to some restriction on strategies:
$\mathcal{G}_i$ is the sub-category
of $\mathcal{G}$ whose morphisms are the innocent strategies,
$\mathcal{G}_b$ has only the well-bracketed strategies and $\mathcal{G}_{ib}$ has the innocent and well-bracketed strategies.

\begin{prop}
$\mathcal{G}$, $\mathcal{G}_i$, $\mathcal{G}_b$ and $\mathcal{G}_{ib}$ are categories.
\end{prop}

Proving this requires to prove that composition of strategies is well-defined, associative, has a unit (the copy-cat strategy), preserves innocence and
well-bracketedness. See \cite{hylandong_pcf,abramsky94full} for a proof.


\subsubsection{Monoidal structure}

We have already defined the tensor product on games in section \ref{sec:gameconstruction}.
We now define the corresponding transformation on morphisms:
given two strategies $\sigma : A \multimap B$ and $\tau : C \multimap D$ the strategy
$\sigma \otimes \tau : (A \otimes C) \multimap (B\otimes D)$ is defined by:
$$ \sigma \otimes \tau = \{ s \in L_{A \otimes C \multimap B\otimes D} \ s \upharpoonright A,B \in \sigma
\wedge s \upharpoonright C,D \in \tau \}$$

It can be shown that the tensor product is associative, commutative and has
$I = \langle \emptyset, \emptyset,\emptyset, \{ \epsilon \} \rangle $ as identity.
Hence the game categories $\mathcal{G}$ is a symmetric monoidal categories. Moreover
$\mathcal{G}_i$ and  $\mathcal{G}_b$ are sub-symmetric monoidal categories of $\mathcal{G}$,
and $\mathcal{G}_{ib}$ is a sub-symmetric monoidal category of $\mathcal{G}_i$, $\mathcal{G}_b$ and
$\mathcal{G}$.

\subsubsection{Closed structure}

Given the games $A$, $B$ and $C$, we can transform strategies on $A\otimes B \multimap C$ to
strategies on $A \multimap (B \multimap C)$ by retagging the moves to the appropriate arenas. This transformation
defines an isomorphism noted $\Lambda_B$ and called currying. Therefore the hom-set $\mathcal{G}(A\otimes B, C)$ is isomorphic to the hom-set
$\mathcal{G}(A,B\multimap C)$ which makes $\mathcal{G}$ an autonomous (i.e. symmetric monoidal closed) category.

We write $ev_{A,B} : (A \multimap B) \otimes A \rightarrow B$ to denote the \emph{evaluation strategy} obtained by uncurrying the
identity map on $A \rightarrow B$. $ev_{A,B}$ is in fact the copycat strategy for the game
$(A \multimap B) \otimes A \rightarrow B$.

$\mathcal{G}_i$ and  $\mathcal{G}_b$ are sub-autonomous categories of $\mathcal{G}$,
and $\mathcal{G}_{ib}$ is a sub-autonomous category of $\mathcal{G}_i$, $\mathcal{G}_b$ and
$\mathcal{G}$.

\subsubsection{Cartesian product}
The cartesian product defined in section \ref{sec:gameconstruction} is indeed a cartesian product in the category
$\mathcal{G}$, $\mathcal{G}_i$, $\mathcal{G}_b$ and $\mathcal{G}_{ib}$.

The projections $\pi_1:A \& B \rightarrow A$ and $\pi_1:A \& B \rightarrow B$ are given by the obvious copy-cat strategies.
Given two category morphisms $\sigma :C \rightarrow A$ and $\tau : C \rightarrow B$ the pairing function
$\langle \sigma, \tau \rangle : C \rightarrow A \& B$ is given by:
\begin{eqnarray*}
\langle \sigma, \tau \rangle &=& \{ s \in L_{C\multimap A\&B} \ | \ s \upharpoonright C,A \in \sigma \wedge s \upharpoonright B = \epsilon  \} \\
&\union& \{ s \in L_{C\multimap A\&B} \ | \ s \upharpoonright C,A \in \sigma \wedge s \upharpoonright B = \epsilon  \}
\end{eqnarray*}

\subsubsection{Cartesian closed structure}
Having defined the cartesian product is not enough to turn $\mathcal{G}$ into a cartesian closed category :
we also need to define a terminal object $I$ and the exponential construct $A \imp B$ for any two games $A$ and $B$.
In fact, this cannot be done in the current categories $\mathcal{G}$ and we have to move on to another category
of games noted $\mathcal{C}$ whose objects and morphisms are certain sub-classes of games and strategies.

Before introducing the category $\mathcal{C}$ we need some new definitions:


For any game $A$ we define the exponential game noted $!A$.
The game $!A$ corresponds to a repeated version of the game $A$. Plays of $!A$ are interleaving of plays of
$A$. It is defined as follows:
\begin{eqnarray*}
  M_{!A} &=& M_A \\
  \lambda_{!A} &=& \lambda_A \\
  \vdash_{!A} & = & \vdash_{A} \\
  P_{!A} & = & \{ s \in L_{!A} | \mbox{ for each initial move $m$, } s \upharpoonright m \in P_A \}
\end{eqnarray*}
The following equalities hold:
\begin{eqnarray*}
  !(A \& B) &=& !A \otimes !B\\
  I &=& !I
\end{eqnarray*}

\begin{dfn}[Well-opened games]
A game $A$ is well-opened if for any position $s \in P_A$ the only initial move is the first
one.
\end{dfn}

Well-opened games have single thread of dialog. Then can be turned into games with multiple-thread of dialog
using the promotion operator:

\begin{dfn}[Promotion]
Consider a well-opened game $B$.
Given a strategy on ${!A} \multimap B$, we define it promotion $\sigma^\dagger : {!A} \multimap {!B}$ to be the
strategy which plays several copies of $\sigma$. It is formally defined by:
$$ \sigma^\dagger = \{ s \in L_{{!A} \multimap !B} \ | \ \mbox{ for all initial $m$, } s \upharpoonright m \in \sigma  \}.$$
\end{dfn}

It can be shown that promotion is well-defined (it is indeed a strategy) and that it preserves innocence and
well-bracketedness.


We now introduce the category of well-opened games:
\begin{dfn}[Category of well-opened games]
The category $\mathcal{C}$ of well-opened games is defined as follow:
\begin{enumerate}
\item The objects are the well-opened games,
\item a morphism $\sigma : A \rightarrow B$ is a strategy for the game $!A \multimap B$,
\item the identity map for $A$ is the copy-cat strategy on $!A \multimap A$ (which is well-defined for well-opened games).
It is called dereliction, noted
$\textsf{der}_A$ and defined formally by:
$$ \textsf{der}_A = \{ s \in P^{\textsf{even}}_{{!A} \multimap A} \ | \ \forall t \sqsubseteq^{\textsf{even}} s \ . \ t \upharpoonright {!A} = t \upharpoonright A \},$$
\item composition of morphisms $\sigma : {!A} \multimap B$ and $\tau : {!B} \multimap C$
noted $\sigma \fatsemi \tau : {!A} \multimap C$ is defined as $\sigma^\dagger;\tau$.
\end{enumerate}
\end{dfn}
$\mathcal{C}$ is a well-defined category and the three sub-categories
$\mathcal{C}_i$, $\mathcal{C}_b$, $\mathcal{C}_{ib}$ corresponding to sub-category
with innocent strategies, well-bracketed strategies and innocent and well-bracketed strategies respectively.


The category $\mathcal{C}$ has a terminal object $I$, for any two games $A$ and $B$ a product $A \& B$ and
an exponential $A \imp B$ defined to be $!A \multimap B$. The hom-sets $\mathcal{C}(A \& B,C)$ and
$\mathcal{C}(A,!B \multimap C)$ are isomorphic. Indeed:
\begin{eqnarray*}
\mathcal{C}(A\& B,C) &=& \mathcal{G}(!(A\& B),C) \\
&=& \mathcal{G}({!A}\otimes {!B},C) \\
&\cong& \mathcal{G}({!A}, {!B} \multimap C) \qquad  \mbox{($\mathcal{G}$ is a closed monoidal category)}\\
&=& \mathcal{C}(A, {!B} \multimap C)
\end{eqnarray*}
Hence $\mathcal{C}$ is a cartesian closed category. Moreover $\mathcal{C}_i$ and $\mathcal{C}_b$
are sub-cartesian closed caterogies of $\mathcal{C}$ and $\mathcal{C}_{ib}$ is as sub-cartesian closed category
of each of $\mathcal{C}$, $\mathcal{C}_i$ and $\mathcal{C}_b$.





\subsubsection{Order enrichment}

Strategies can be ordered using the inclusion ordering. Under this
ordering, the set of strategies on a given game $A$ is a pointed
directed complete partial order : the least upper bounds is the
union of two strategies and the least element is the empty strategy
$\{ \epsilon \}$.

Moreover all the operators on strategies that we have defined so far
(composition, tensor product, ...) are continuous. Hence the
category $\mathcal{C}$ and $\mathcal{G}$ are cpo-enriched.

This significant characteristic will prove to be extremely useful
when it comes to model programming languages with recursion such as
PCF.


\subsubsection{Intrinsic preoder}

We now define a pre-ordering on strategies. We assume that we are working in one of the categories
$\mathcal{C}$, $\mathcal{C}_i$, $\mathcal{C}_b$, $\mathcal{C}_{ib}$.

Let $\Sigma$ be the game with a single question $q$ and single answer $a$. There are only two strategies on $\Sigma$:
$\bot = \{ \epsilon \}$ and $\top = \{ \epsilon, q a \}$ which are both innocent and well-bracketed. These strategies are used
to test strategies: for any strategy $\sigma : {\bf 1} \rightarrow A$ and for any test strategy $\alpha : A \rightarrow \Sigma$ we say that $\sigma$
passes the test $\alpha$ if $\sigma \fatsemi \alpha = \top$.

The intrinsic preorder noted $\lesssim$ is then defined as follows:
for any strategy $\sigma,\tau$ on the game $A$, $\sigma \lesssim \tau$ if $\tau$ passes all the test passed by $\sigma$. Formally:
$$ \sigma \lesssim \tau \quad \iff \quad \forall \alpha : A \rightarrow \Sigma. \sigma \fatsemi \tau = \top \imp \tau \fatsemi \alpha = \top$$

One can check that the relation $\lesssim$ is indeed a preorder on the set of strategies of the considered category.
This preorder defines classes of equivalence: two strategies are in the same equivalence class if no test can distinguish them.
The quotiented category is written $\bf C/\lesssim$ where $\bf C$ ranges over $\{ \mathcal{C}_i, \mathcal{C}_i, \mathcal{C}_b, \mathcal{C}_{ib} \}$.

Later on we will state the full abstraction of the game semantics model of PCF. This result will
be proved in the quotiented category.

\subsection{Special case of arenas of order 2 at most}
In this section, we consider a restricted class of arena and prove a
property on the games played on these arenas.

The height of the arena is the length of the longest sequence of moves
$m_1 \ldots m_h$ in $M$ such that $m_1 \vdash m_2 \vdash \ldots \vdash m_h$.

The order of an arena $\langle M, \lambda, \vdash \rangle$ is defined to be
$h-2$ where $h$ is the height of the forest of trees $(M, \vdash)$.


\begin{lem}[Pointers are superfluous up to order 2]
Let $A$ be the arena of order at most 2. Let $s$ be a justified sequence of moves in the arena $A$ satisfying
 alternation, visibility, well-openedness and well-bracketing then
the pointers of the sequence $s$ can be reconstructed uniquely.
\end{lem}



\begin{proof}
We represent an arena graphically as a forest of trees. We choose to display the sub-trees of a given node
from left to right by decreasing order of the sub-arena order. This reordering is harmless since reordering children nodes
produces isomorphic arenas.

Let $A$ be an arena of order 2.
The justified sequence that we consider are well-opened therefore there is only one initial move in the sequence (the first move). Consequently
if $A$ is a forest arena (i.e. with multiple roots), the problem can be reduced to the case of single root arena just by replacing
the arena $A$ by the tree of the forest $A$ whose root is the first move of the justified sequence.
Therefore we assume that $A$ has a single root. $A$ has the following shape:
\begin{center}
\
  \pstree[levelsep=6ex]
    { \TR{$q$} }
    {
\SubTree{$T_1$} \SubTree[linestyle=none]{$\ldots$} \SubTree{$T_n$}
    \TR{$a_1$} \TR{$a_2$} \TR{\ldots} }
\end{center}

where each triangle $T_i$ represents an arena of order 0 or 1.


We write $I_k$, for $k=0$ or $1$, the set of indices $i$ such that the arena $T_i$ has order $k$:
$$I_k = \{ i \in 1.. n\ |\ \order{T_i} = k \}$$

Here is a graphic representation of the arenas $T_i$ for $i \in I_0$ and $T_j$ for $j \in I_1$:
\begin{center}
\
  \pstree[levelsep=6ex]
    {\TR{$q^i$}}
    { \TR{$a_1^i$} \TR{$a_2^i$} \TR{\ldots} }
\hspace{2cm}
  \pstree[levelsep=6ex]
    { \TR{$p^j$} }
    {
      \pstree[levelsep=6ex]
        { \TR{$q^j$} }
        { \TR{$a_1^j$} \TR{$a_2^j$} \TR{\ldots} }
      \TR{$b_1^j$} \TR{$b_2^j$} \TR{\ldots}
    }
\end{center}



For any justified sequence of moves $u$, we write $?(u)$ for the
subsequence of $u$ consisting of the questions in the sequence $u$
that are still pending at the end of the sequence.

Let $L$ be the following language $L = \{\ p^i q^i\ | \ i \in I_1
\}$. We consider the following cases:

\begin{center}
\begin{tabular}{c|c|l|l}
Case & $\lambda_{OP}(m)$ & $?(u) \in$ & condition \\ \hline
0 & O & $\{ \epsilon \}$ \\
A & P & $q$ \\
B & O & $q \cdot L^* \cdot p^i$     & $i \in I_1$ \\
C & P & $q \cdot L^* \cdot p^i q^i$ & $i \in I_1$ \\
D & O & $q \cdot L^* \cdot q^i$      & $i \in I_0$ \\
\end{tabular}
\end{center}

We use the notation $\hat{s}$ to denote a legal and well-bracketed
\emph{justified} sequence of moves and $s$ to denote the same
sequence of moves with pointers removed.

Note that the well-bracketing condition already tells us how to
uniquely recover the pointers for P answer moves: a P-answers points
to the last pending question having the same tag. However for O
answers, we will see that the visibility condition already ensures
the unique recoverability of the pointer and that the
well-bracketing condition is not needed.


We prove by induction on the sequence of moves $u$ that $?(u)$
corresponds to either case 0, A, B, C or D and that the pointers in
$u$ can be recovered uniquely.

\textbf{Base cases:}

If $u$ is the empty sequence $\epsilon$ then there is no pointer to
recover and it corresponds to case 0.

If $u$ is a singleton then it must be the initial question $q$ and
there is not pointer to recover. This corresponds to case A.

\textbf{Step case:}

Consider a legal well-bracketed justified sequence $\hat{s}$ where
$s = u \cdot m$ and $m \in M_A$. The induction hypothesis tells us
that the pointers of $u$ can be recovered (and therefore the P-view
or O-view at that point can be computed) and that $u$ corresponds to
one of the cases 0,A,B,C or D.

We proceed by case analysis on $u$:

\begin{description}

\item[case 0] This case cannot happen because $?(u) = \epsilon$ ($u$ is a complete play) implies that there cannot be any further move $m$.

Indeed the visibility condition implies that $m$ must point to a
P-question in the O-view at that point. But since $u$ is a complete
play, the O-view is $\oview{\hat{u}} = q a$ which does not contain
any P-question. Hence the move $m$ cannot be justified and is not
valid.


\item[case A] $?(u) = q$ and the last move $m$ is played by P.
    There are several cases:
    \begin{itemize}
    \item $m$ is an answer $a_k$ (to the initial question
    $q$) for some $k$, then $m$ points to $q$:

    $\hat{s} = \justseq{ q & \ldots & m \pointto{ll}}$

    and $?(s) = \epsilon$ therefore $s$ correspond to the case 0 (complete play).

    \item $m = q^i$ where $q^i$ is an order 0 question ($i \in I_0$).
    Then $q^i$ points to the initial question $q$ and $s$ falls into category D.

    \item $m = p^i$, a first order question, then $p^i$ points to $q$,

    $?(s)= q p^i$ and it is O's turn after $s$ therefore $s$ falls into category B.

    \end{itemize}


\item[case B] $?(u) \in q \cdot L^* \cdot p^i$ where $i \in I_1$ and O plays the move $m$.

We now analyse the different possible O-moves:
\begin{itemize}
\item Suppose that O gives the (tagged) answer $b^j$ for some $j \in I_1$ then
the visibility condition constraints it to point to a question in
the O-view at that point.

We remark that the last move in $\hat{u}$ must be $p^i$. Indeed,
suppose that there is a move $x \in M_A$ such that $\hat{u} =
\justseq{q & \ldots & p^i\ x \pointto{ll}}$ then by visibility, the
O-move $x$ should points to a move in the O-view a that point. The
O-view is $q p^i$, therefore $x$ can only points to $p^i$. But then,
$p^i$ is not a pending question in $s$ which is a contradiction.


Therefore $\oview{\hat{u}} = \oview{ \justseq{ q & \ldots & p^i
\pointto{ll}} } = q p^i$.

Hence $b^j$ can only point to $p^i$ (and therefore $i=j$).

We then have $?(s) = ?(u \cdot b^i) \in  q \cdot L^*$ which is
covered by case A and C.

\item The only other possible O-move is $q^i$ which, again by the visibility condition, points necessarily
to the previous move $p^i$. We then have $?(s) = ?(u \cdot q^i) \in
q \cdot L^* \cdot p^i q^i$. This falls into category C.

\end{itemize}

\item[case C] $?(u) \in q \cdot L^* \cdot p^i q^i$ where $i \in I_1$ and the move $m$ is played by $P$.

Suppose $m$ is an answer, then the well-bracketing condition imposes
to answer to $q^i$ first. The move $m$ is therefore an integer $a^i$
pointing to $q^i$. We then have $?(s) = ?(u \cdot a^i) \in  q \cdot
L^* \cdot p^i$. This correspond to case B.


Suppose $m$ is a question then there are two cases:
\begin{itemize}
\item $m = q^j$ with $j \in I_0$, the pointer goes to the initial question $q$ and $s$ falls into category D.
\item $m = p^j$ with $j \in I_1$, the pointer goes to the initial question $q$ and $s$ falls into category B.
\end{itemize}

\item[case D] $?(u) \in q \cdot L^* \cdot q^i$ where $i \in I_0$ and the move $m$ is played by $O$.

    The same argument as in case B holds. However there is now another possible move:
    the answer $m = a^i_k$ for some $k$.  This moves can only points to
    $q^i$ (this is the only pending question tagged by $i \in I_0$).

    Then $?(\hat{s}) = ?(\hat{u}\cdot a^i_k) = ?(\justseq{ q & \ldots & q^i \pointto{ll} & \ldots & a^i_k \pointto{ll}}) \in q \cdot L^* $ therefore $s$ falls either into category A or C.

\end{description}

This completes the induction.
\end{proof}


\subsection{Pointers are necessary}
\label{subsec:pointer_necessary}

Up to order 2, the semantics of PCF terms is entirely defined by
pointer-less strategies. In other words, the pointers can be
uniquely reconstructed from any non justified sequence of moves
satisfying the visibility and well-bracketing condition.

At level 3 however, pointers cannot be omitted in general. Here is
an example taken from \cite{abramsky:game-semantics-tutorial}
illustrating this. Consider the following two terms, called the
Kierstead terms, of type $((\nat \typar \nat) \typar \nat) \typar
\nat$:

$$M_1 = \lambda f . f (\lambda x . f (\lambda y .y ))$$
$$M_2 = \lambda f . f (\lambda x . f (\lambda y .x ))$$

We assign tags to the types in order to identify in which arena the
questions are asked: $((\nat^1 \typar \nat^2) \typar \nat^3) \typar
\nat^4$. Consider now the following pointer-less sequence of moves
$s = q^4 q^3 q^2 q^3 q^2 q^1$. It is possible to retrieve the
pointers of the first five moves but there is an ambiguity for the
last move: does it point to the first or second occurrence of $q^2$
in the sequence $s$?

Note that the visibility condition does not eliminate the ambiguity,
since the two occurrences of $q^2$ both appear in the P-view at that
point (after recovering the pointers of $s$ up to the second last
move we get:
$$s = \rnode{q4}{q}^4
\rnode{q3}{q}^3
\rnode{q2}{q}^2
\rnode{q3b}{q}^3
\rnode{q2b}{q}^2
\rnode{q1}{q}^1
\bkptrc{q3}{q4}
\bkptrc{q2}{q3}
\bkptrc[ncurv=0.6]{q3b}{q4}
\bkptrc{q2b}{q3b}$$

 therefore the P-view of $s$ is $s$ itself.)

In fact these two different possibilities correspond to two
different strategies. Suppose that the link goes to the first
occurrence of $q^2$ then it means that the proponent is requesting
the value of the variable $x$ bound in the subterm $\lambda x . f (
\lambda y. ... )$. If P needs to know the value of $x$, this is
because P is in fact following the strategy of the subterm $\lambda
y . x$. And the entire play is part of the strategy $\sem{M_2}$.

Similarly, if the link points to the second occurrence of $q^2$ then
the play belongs to the strategy $\sem{M_1}$.

\section{The fully abstract game model for PCF}

In this section we introduce the functional languages PCF. We then
describe the game model introduced in \cite{abramsky94full} and
finally we will state the full abstraction result.

\subsection{The syntax of PCF}
PCF is a simply-type $\lambda$-calculus with the following
additions: integer constants  (of ground type), first-order
arithmetic operators, if-then-else branching, and the recursion
combinator $Y_A : (A\rightarrow A)\rightarrow A$ for any type $A$.

The types of PCF are given by the following grammar:
$$ T ::= \texttt{exp}\ |\ T \rightarrow T$$

and the structure of terms is given by:
\begin{eqnarray*}
 M ::= x\ |\ \lambda x :A . M \ |\ M M \ |\ \\
\ |\ n \ |\ \texttt{succ } M \ |\  \texttt{pred } M \\
\ |\ \texttt{cond } M M M \ |\ \texttt{Y}_A\ M
\end{eqnarray*}

where $x$ ranges over a set of countably many variables and $n$
ranges over the set of natural numbers.

Terms are generated according to the formation rules given in table
\ref{tab:pcf_formrules} where the judgement is of the form $ \Gamma  \vdash M : A$.

\begin{table}[htbp]
$$ (var) \rulef{}{x_1:A_1, x_2:A_2, \ldots x_n : A_n  \vdash x_i : A_i}\ i \in 1..n$$
$$ (app) \rulef{\Gamma \vdash M : A\rightarrow B \qquad \Gamma \vdash N:A}{\Gamma \vdash M\ N : B}
\qquad (abs) \rulef{\Gamma, x:A \vdash M : B}{\Gamma \vdash \lambda x :A . M : A\rightarrow B}$$

$$ (const) \rulef{}{\Gamma \vdash n :\texttt{exp}}
\qquad (succ) \rulef{\Gamma \vdash M:\texttt{exp} }{\Gamma \vdash \texttt{succ}\ M:\texttt{exp}}
\qquad (pred) \rulef{\Gamma \vdash M:\texttt{exp} }{\Gamma \vdash \texttt{pred}\ M:\texttt{exp}}$$

$$
(cond) \rulef{\Gamma \vdash M : exp \qquad \Gamma \vdash N_1 : exp \qquad \Gamma \vdash N_2 : exp }{\Gamma \vdash \texttt{cond}\ M\ N_1\ N_2}
\qquad  (rec) \rulef{\Gamma \vdash M : A\rightarrow A }{ \Gamma \vdash Y_A M : A}$$

\caption{Formation rules for PCF terms}
\label{tab:pcf_formrules}
\end{table}

\subsection{Operational semantics of PCF}

We give the big-step operational semantics of PCF. The notation $M \eval V$ means
that the closed term $M$ evaluates to the canonical form $V$. The canonical forms are given by the following
grammar:
$$V ::= n\ |\ \lambda x. M$$
In other word, a canonical form is either a number or a function.

The full operational semantics is given in table
\ref{tab:bigstep_pcf}. The evaluation rules are defined for closed
terms only therefore the context $\Gamma$ is not present in the
rules. We write $M \eval$ if $M \eval V$ for some value $V$.

\begin{table}[htbp]
$$\rulef{}{V \eval V} \quad \mbox{ provided that $V$ is in canonical form.} $$

$$ \rulef{M \eval \lambda x. M' \quad M'\subst{x}{N} \eval V}{M N \eval V}$$

$$\rulef{M \eval n}{\texttt{succ}\ M \eval n+1}
\qquad \rulef{M \eval n+1}{\texttt{pred}\ M \eval n}
\qquad \rulef{M \eval 0}{\texttt{pred}\ M \eval 0}$$

$$\rulef{M \eval 0 \quad N_1 \eval V}{\texttt{cond}\ M N_1 N_2  \eval V}
\qquad
 \rulef{M \eval n+1 \quad N_2 \eval V}{\texttt{cond}\ M N_1 N_2  \eval V}$$

$$\rulef{M (\mathrm{Y} M) \eval V }{\texttt{Y} M \eval V}$$
\label{tab:bigstep_pcf}
\caption{Big-step operational semantics of PCF}
\end{table}



\subsection{Game model of PCF}
\label{subsec:pcfgamemodel}

As we have seen in section \ref{sec:catgames}, games and strategies
form a cartesian closed category, therefore games can model the
simply-typed $\lambda$-calculus. We are now about to make this
connection concrete by explicitly giving the strategy corresponding
to a given $\lambda$-term. We will then extend the game model to PCF
and IA.

\subsubsection{Simply-typed $\lambda$-calculus fragment}

In the games that we are considering, the Opponent represents the
environment and the Proponent represents the lambda term. Opponent
opens the game by asking a question such as ``What is the output of
the function?'', the proponent then may then ask further information
such that ``What is the input of the function?'' O can then provide
$P$ with an answer (the value of the input) or can pursue with
another question. The dialog goes on until O gets the answer to his
initial question.

O represents the environment, he is responsible for proving input
values while P plays from the term's point of view: he is
responsible for performing the computation and returning the output
to O. P plays according to the strategy that is associated to the
$\lambda$-term being modeled.

We recall that in the cartesian closed category $\mathcal{C}$, the
objects are the arenas and the morphisms are the strategies. Given a
simple type $A$, we will model it as an arena $\sem{A}$. A context
$\Gamma = x_1 :A_1, \ldots x_n:A_n$ will be mapped to the arena
$\sem{\Gamma} = \sem{A_1} \times \ldots \times \sem{A_n}$ and a term
$\Gamma \vdash M : A$ will be modeled by a strategy on the arena
$\sem{\Gamma} \rightarrow \sem{A}$. Since $\mathcal{C}$ is cartesian
closed, there is is a terminal object $\textbf{1}$ (the empty arena)
that models the empty context ($\sem{\Gamma} = \textbf{1}$).


Let $\omega$ denotes the set of natural numbers. Consider the
following flat arena over $\omega$:
$$  \pstree[levelsep=6ex]
    {\TR[name=R]{q}}
    { \TR{1} \TR{2} \TR{\ldots}
    }
$$
Then the base type \texttt{exp} is interpreted by the flat game
$\nat$ over the previous arena where the set of valid position is:
$$P_N = \{ \epsilon, q \} \union \{ qn \ | \ n \in \omega \}$$


In this arena, there is only one question: the initial O-question, P
can then answer by playing a natural number $i \in \omega$. There
are only two kinds strategy on this arena:
\begin{itemize}
\item the empty strategy where P never answer the initial question. This corresponds to a non terminating computation;
\item the strategies where P answers by playing a number $n$. This models the numerical constants of the language.
\end{itemize}

Given the interpretation of base types, we define the interpretation
of $A\rightarrow B$ by induction:
$$\sem{A \rightarrow B} = \sem{A} \Rightarrow \sem{B}$$

where the operator $\Rightarrow$ denotes the arena construction $!A
\multimap B$, the exponential object of the cartesian closed
category $\mathcal{C}$.



Variables are interpreted by projection:
$$\sem{x_1 : A_1, \ldots, x_n:A_n \vdash x_i : A_i} = \pi_i : \sem{A_i} \times \ldots \times \sem{A_i} \times \ldots \times \sem{A_n} \rightarrow  \sem{A_i}$$

The abstraction $\Gamma \vdash \lambda x :A.M : A \rightarrow B$ is
modeled by a strategy on the arena $\sem{\Gamma} \rightarrow
(\sem{A}\Rightarrow\sem{B})$. This strategy is obtain by using the
currying operator of the cartesian closed category:
$$\sem{\Gamma \vdash \lambda x :A.M : A \rightarrow B} = \Lambda( \sem{\Gamma, x :A \vdash M : B})$$

The application $\Gamma \vdash M N$ is modeled using the evaluation
map $ev_{A,B} : (A\Rightarrow B)\times A \rightarrow B$:

$$\sem{\Gamma \vdash M N} = \langle \sem{\Gamma \vdash M, \Gamma \vdash N} \rangle \fatsemi ev_{A,B}$$


\subsubsection{PCF fragment}

We now show how to model PCF constructs in the game semantics
setting. In the following, the sub-arena of a game are tagged in
order to distinguish identical arenas present in different
components of the game. Moves are also tagged in the exponent in
order to identify the sub-arena in which moves are played. We will
omit the pointers in the play when there is no ambiguity.

The successor arithmetic operator is modeled by the following
strategy on the arena $\nat^1 \Rightarrow \nat^0$:
$$\sem{\texttt{succ}} = \{q^0 \cdot q^1 \cdot n^1 \cdot (n+1)^0\ |\ n \in \nat \}$$

The predecessor arithmetic operator is denoted by the strategy
$$\sem{\texttt{pred}} = \{q^0 \cdot q^1 \cdot n^1 \cdot (n-1)^0\ |\ n >0 \} \union \{ q^0 \cdot q^1 \cdot 0^1 \cdot 0^0 \} $$

Then given a term $\Gamma \vdash \texttt{succ }M : \texttt{exp}$ we
define:
$$\sem{\Gamma \vdash \texttt{succ } M : \texttt{exp}} = \sem{\Gamma \vdash M} \fatsemi \sem{\texttt{succ}} $$
$$\sem{\Gamma \vdash \texttt{pred } M : \texttt{exp}} = \sem{\Gamma \vdash M} \fatsemi \sem{\texttt{pred}} $$


The conditional operator is denoted by the following strategy on the
arena $\nat^3 \times \nat^2 \times \nat ^1 \Rightarrow \nat^0$:
$$\sem{\texttt{cond}} =
    \{ q^0 \cdot q^3 \cdot 0 \cdot q^2 \cdot n^2 \cdot n^0 \ | \ n \in \nat \}
    \union
    \{ q^0 \cdot q^3 \cdot m \cdot q^2 \cdot n^2 \cdot n^0 \ | \ m >0, n \in \nat \}
    $$


Given a term $\Gamma \vdash \texttt{cond}\ M\ N_1\ N_2$ we define:
$$\sem{\Gamma \vdash \texttt{cond}\ M\ N_1\ N_2} =
\langle \sem{\Gamma \vdash M}, \sem{\Gamma \vdash N_1}, \sem{\Gamma
\vdash N_2} \rangle \fatsemi \sem{\texttt{cond}}$$


The interpretation of the \texttt{Y} combinator is a bit more
complicated.

Consider the term $\Gamma \vdash M : A \rightarrow A$, its semantics
$f$ is a strategy on $\sem{\Gamma} \times \sem{A} \rightarrow
\sem{A}$. We define the chain $g_n$ of strategies on the arena
$\sem{\Gamma} \rightarrow \sem{A}$ as follows:
\begin{eqnarray*}
g_0 &=& \perp \\
g_{n+1} &=&  F(g_n) = \langle id_{\sem{\Gamma}}, g_n\rangle \fatsemi f
\end{eqnarray*}

where $\perp$ denotes the empty strategy $\{ \epsilon \}$.

It is easy to see that the $g_n$ forms a chain. We define
$\sem{\texttt{Y } M}$ to be the least upper bound of the chain $g_n$
(i.e. the  least fixed point of $F$). Its existence is guaranteed by
the fact that the category of games is cpo-enriched.

Since all the strategies that we have given are innocent and
well-bracketed, the game model of PCF can be interpreted in any of
the four categories $\mathcal{C}$, $\mathcal{C}_i$, $\mathcal{C}_b$,
$\mathcal{C}_{ib}$.



\subsection{Full-abstraction of PCF}
In this section we state the full abstraction result proved in
\cite{abramsky94full} and \cite{hylandong_pcf}.


\subsubsection{Observational preorder}

A context noted $C[-]$ is a term containing a hole denoted by $-$.
If $C[-]$ is a context then $C[A]$ denotes the term obtained after
replacing the hole by the term $A$.

If $M$ is a PCF term then we write $C[M]$ to denote the term
obtained after replacing the hole by the term $M$. $C[M]$ is
well-formed provided that $M$ has the appropriate type. Remark: this
capture permitting substitution must be distinguished from the
capture-free substitution which is noted $M[N/x]$ for any two terms
$M$ and $N$.


\begin{dfn}[Observational preorder]
We define the relation on terms $\obspre$ as follows: suppose $M$
and $N$ are two closed terms of the same type then:
\begin{eqnarray*}
M \obspre N &\iff& \parbox{10cm}{for all context $C[-]$ such that
                $C[M]$ and $C[N]$ are well-formed closed term of type \texttt{exp},
                    $C[M] \eval$ implies $C[N] \eval$}
\end{eqnarray*}
Observational equivalence is defined as the reflexive closure of
$\obspre$ noted $\obseq$.
\end{dfn}

Said informally, two programs are observationally equivalent if then
can be safely interchanged in any program context.

\subsubsection{Soundness and adequacy}
A model of a programming language is said to be \emph{sound} or
\emph{inequationally sound} if whenever the denotation of two
programs are equal then the two programs are observationally
equivalent, or more formally if for any closed terms $M$ and $N$ of
the same type:
$$ \sem{M} \subseteq \sem{N} \imp M \obspre N.$$

In a way, soundness is the minimum one can require for a model of
programming language: it guarantees that we can reason about the
program by manipulating the object of the denotational model.

It can be shown that the game model of PCF is sound for evaluation
and computationally adequate. These two properties imply the
soundness of the game model:

We said that the evaluation relation $\eval$ is sound if the
denotation is preserved by evaluation:
\begin{lem}[Soundness of evaluation]
\label{lem:evalsoundness}
 Let $M$ be a PCF term then
$$M \eval V \quad \imp \quad \sem{M} = \sem{V}.$$
\end{lem}

\begin{dfn}[Computable terms] \
\begin{itemize}
\item A closed term $\vdash M$ of base type is computable if $\sem{M} \neq \bot$
implies $M \eval$.
\item A higher-order closed term $\vdash M : A\rightarrow B$ is computable if $M N$ is computable for any computable closed term $\vdash  N:A$.
\item An open term $x_1 : A_1, \ldots, x_n : A_n \vdash M : A\rightarrow B$ is computable if $\vdash M [N_1/x_1, \ldots N_n/x_n]$ is computable
for all computable closed terms $N_1:A_1, \ldots, N_n:A_n$.
\end{itemize}
\end{dfn}

A model is \emph{computationally adequate} if all
terms are computable.
\begin{lem}[Computational adequacy]
\label{lem:computadequacy}
The game model of PCF is
computationally adequate.
\end{lem}
We refer the reader to \cite{abramsky:game-semantics-tutorial} for
the proofs.

Inequational soundness follows from the last two lemmas:
\begin{prop}[Inequational soundness]
\label{prop:ineqsoundness} Let $M$ and $N$ be two closed terms then
$$\sem{M} \subseteq \sem{N} \implies  M \obspre N $$
\end{prop}
\begin{proof}
  Suppose that $\sem{M} \subseteq \sem{N}$ and $C[M] \eval$ for some context $C[-]$. Then by compositionality of game semantics we also have
  $C[\sem{M}] \subseteq C[\sem{N}]$.
  Lemma \ref{lem:evalsoundness} gives $\sem{C[M]} \neq \bot$, therefore $\sem{C[N]} \neq \bot$.
  Lemma \ref{lem:computadequacy} then implies that $C[N] \eval$.
  Hence $M \obspre N$.
\end{proof}

\subsubsection{Definability}

We will now consider only strategies that are innocent and
well-bracketed (i.e. we work in the category $\mathcal{C}_{ib}$).

The definability result says that every compact element of the model
is the denotation of some term.
The compact morphisms of the category $\mathcal{C}_{ib}$ are those
with finite view-function.

The economical syntax of PCF prevents us from stating this
result directly: we need to consider an extension of PCF with some additional
constants. Indeed, there are strategies that are not the denotation of any term
in PCF, for instance the ternary conditional strategy : this
strategy denotes the computation that tests the value of its first
parameter, if its equal to zero or one then it returns the value of
the second or third parameter respectively, otherwise it returns the
value of the fourth parameter. This strategy is illustrated by the
following diagram:
$$
\begin{array}{ccccccccc}
!\bf N & \otimes & !\bf N & \otimes & !\bf N & \otimes & !\bf N & \multimap & !\bf N \\
&&&&&&&&q \\
q \\
0 \\
&& q \\
&& n \\
&&&&&&&&n \\
\hline
&&&&&&&&q \\
q \\
1 \\
&&&& q \\
&&&& n \\
&&&&&&&&n \\
\hline
&&&&&&&&q \\
q \\
m>1 \\
&&&&&& q \\
&&&&&& n \\
&&&&&&&&n \\
\end{array}
$$

It is possible to simulate this computation in PCF using the conditional operator, for instance the following term is a potential candidate:
$$ T = \texttt{cond}\ M\  N_1 (\texttt{cond}\  (\texttt{pred } M)\  N_2\  N_3)$$

Unfortunately the game semantics of $T$ is not given by the strategy that we have just defined, it is instead the following one:
$$
\begin{array}{ccccccccc}
!\bf N & \otimes & !\bf N & \otimes & !\bf N & \otimes & !\bf N & \multimap & !\bf N \\
&&&&&&&&q \\
q \\
0 \\
&& q \\
&& n \\
&&&&&&&&n \\
\hline
&&&&&&&&q \\
q \\
1 \\
q \\
0 \\
&&&& q \\
&&&& n \\
&&&&&&&&n \\
\hline
&&&&&&&&q \\
q \\
m>1 \\
q \\
m-1>0 \\
&&&&&& q \\
&&&&&& n \\
&&&&&&&&n \\
\end{array}
$$

To make up for this deficiency we add a family of terms to PCF: the $k$-ary conditionals:
$$ \texttt{case}_k\ N\ N_1\ N_2\ \ldots\ N_k$$
with the desired operational semantics:
$$ \rulef{M \eval i \quad N_{i+1} \eval V}{\texttt{case}_k\ N\ N_1\ N_2\ \ldots\ N_k\ \eval V}\ i \in \{0, \ldots,k-1\}.$$
The denotation of this term is given by the first strategy illustrated above.
The extended language is called PCF'.

We can now prove the definability result:
\begin{prop}[Definability]
\label{prop:definability} Let $A$ be a PCF type and $\sigma$ be a compact innocent and well-bracketed
strategy on $A$. There exists a PCF' term $M$ such that $\sem{M} = \sigma$.
\end{prop}

Note that definability is proved for PCF' and not for PCF.
Nevertheless, PCF' is a conservative extension of PCF: if $M$ and $N$ are terms such that for any PCF-context $C[-]$,
$C[M] \eval \imp C[N] \eval$ then the same is true for any PCF'-context. This is because $\texttt{case}_k$ constructs can be ``simulated''
in PCF, for instance $\texttt{case}_3$ can be replaced by the PCF term $T$ which shares the same operational semantics.

This observation will allow us to use definability in PCF' to
prove the full-abstraction of PCF.


\subsubsection{Full abstraction}

Full abstraction of PCF cannot be stated directly in the category $\mathcal{C}_{ib}$. Instead we need to consider the quotiented category
$\mathcal{C}_{ib}/\lesssim_{ib}$.

First we need to show that $\mathcal{C}_{ib}/\lesssim_{ib}$ is a model of PCF.
$\mathcal{C}_{ib}/\lesssim_{ib}$ is a posset-enriched cartesian closed category. The game semantics of the basic types and constants of PCF
can be transposed from $\mathcal{C}_{ib}$ to $\mathcal{C}_{ib}/\lesssim_{ib}$. Unfortunately it is not know whether $\mathcal{C}_{ib}/\lesssim_{ib}$
is enriched over the category of CPOs. However it can be proved that it is a rational category \citep{abramsky94full}
and this suffices to ensure that $\mathcal{C}_{ib}/\lesssim_{ib}$ is indeed a model of PCF.

The full abstraction of the game model then follows from
proposition \ref{prop:ineqsoundness} and \ref{prop:definability}:
\begin{thm}[Full abstraction]
Let $M$ and $N$ be two closed IA-terms.
$$\sem{M} \precsim_{ib} \sem{N} \ \iff \ M \obspre N$$
where $\precsim_{ib}$ denotes the intrinsic preorder of the category $\mathcal{C}_{ib}$.
\end{thm}

\section{The fully abstract game model for Idealized Algol (\ialgol)}

We now extend the work of the previous section to the language
\ialgol, an imperative extension of PCF. We start by giving the
syntax and operational semantics of the language, we then describe
the game model which was introduced in \cite{abramsky99full}.
Finally we will state the full abstraction result for the game
model.

\subsection{The syntax of \ialgol}
IA is an extension of PCF introduced by J.C. Reynold in
\cite{Reynolds81}. It adds imperative features such as local
variables and sequential composition.

On top of \texttt{exp}, PCF has the following two new types:
\texttt{com} for commands and \texttt{var} for variables. There is a
constant \texttt{skip} of type \texttt{com} which corresponds to the
command that do nothing.

Commands can be composed using the
sequential composition operator $\texttt{seq}_A$: suppose that $M$ and $N$ are of type
\texttt{com} and $A$ respectively then they can be composed to form the term
$S = \texttt{seq}_A M N : \texttt{com}$. $S$ denotes program that executes $M$ until it terminates and then
behave like $N:A$. If $A = \texttt{exp}$ then the expression is allowed to have a side-effect and $S$ returns the expression computed by $N$, if
$A = \texttt{com}$ then the command $N$ is executed after $M$.
We say that the language has \emph{active expressions} to indicate the presence of the
sequential operator $\texttt{seq}_{exp}$ in the language.


Local variable are
declared using the \texttt{new} operator, variable content is altered
using \texttt{assign} and retrieved using \texttt{deref}.

In addition IA has the constant \texttt{mkvar} that can be used to
create a particular kind of variables. The \texttt{mkvar} operator
works in an object oriented fashion: it takes two arguments, the
first one is a function (called the acceptor) that affects a value
to the variable and the second argument is an expression that
returns a value from the variable. This mechanism is very much like
the ``set/get'' object programming paradigm used by C++ programmers.

Variables created with \texttt{mkvar} are less constraint than the
variables created with \texttt{new}. Indeed, variables created with
\texttt{new} act like memory cells, they obey the following rule: the value read
from the variable is always the last value that has been assigned to
it. This rule does not apply to variables created with
\texttt{mkvar}. For instance the variable:
$$\texttt{mkvar}\ (\lambda v.\texttt{skip})\ 0$$
will always return $0$ even if another number has been assigned it.


One may think that this addition to the language is artificial,
however the full abstraction result of the game model of IA relies
upon this addition. At present, it is still an open problem to find
a fully abstract model of IA deprived of \texttt{mkvar}.

Judgement are of the form $\Gamma \vdash M : A$.
If the judgement $\Gamma = \emptyset$ we say that $M$ is a closed term.
The set of additional formations rules completing those of PCF are
given in table \ref{tab:ia_formrules}.

\begin{table}[htbp]
$$ \rulef{\Gamma \vdash M : \texttt{com} \quad \Gamma \vdash N :A}
    {\Gamma \vdash \texttt{seq}_A \ M\ N\ : A} \quad A \in \{ \texttt{com}, \texttt{exp}\}$$

$$ \rulef{\Gamma \vdash M : \texttt{var} \quad \Gamma \vdash N : \texttt{exp}}
    {\Gamma \vdash \texttt{assign}\ M\ N\ : \texttt{com}}
\qquad
 \rulef{\Gamma \vdash M : \texttt{var}}
    {\Gamma \vdash \texttt{deref}\ M\ : \texttt{exp}}$$

$$ \rulef{\Gamma, x : \texttt{var} \vdash M : A}
    {\Gamma \vdash \texttt{new } x \texttt{ in } M} \quad A \in \{ \texttt{com}, \texttt{exp}\}$$

$$ \rulef{\Gamma \vdash M_1 : \texttt{exp} \rightarrow \texttt{com} \quad \Gamma \vdash M_2 : \texttt{exp}}
    {\Gamma \vdash \texttt{mkvar } M_1\ M_2\ : \texttt{var}}$$

\caption{Formation rules for IA terms}
\label{tab:ia_formrules}
\end{table}


\subsection{Operational semantics of \ialgol}

The operational semantics of IA is given in a slightly different form compared to PCF.
Instead of giving the semantics for closed terms we consider terms
whose free variables are all of type \texttt{var}. Terms are
``closed'' by mean of stores. A store is a function mapping free
variables of type \texttt{var} to natural numbers. Suppose $\Gamma$
is a context containing only variables of type \texttt{var}, then we
say that $\Gamma$ is a \texttt{var}-context. A store with domain
$\Gamma$ is called a $\Gamma$-store. The notation $s\ |\ x \mapsto
n$ refers to the store that maps $x$ to $n$ and otherwise maps
variables according to the store $s$.

%%%% The following is poorly written:
%
%In PCF, the evaluation rules were given for closed terms only.
%Suppose that we proceed the same way for IA and consider the
%evaluation rule for the $\texttt{new}$ construct: the conclusion is
%$\texttt{new } x:=0 \texttt{ in } M$ and the premise is an
%evaluation for a certain term constructed from $M$, more precisely
%the term $M$ where \emph{some} occurrences of $x$ are replaced by
%the value $0$. Because of the presence of the \texttt{assign}
%operator, we cannot simply replace all the occurrences of $x$ in $M$
%(the required substitution is  more complicated than the
%substitution used for beta-reduction).


The canonical forms for IA are given by the grammar:
$$ V ::= \texttt{skip}\ |\ n\ |\ \lambda x. M\ |\ x\ |\  \texttt{mkvar}\ M\ N$$

where $n \in \nat$ and $x: \texttt{var}$.


In \ialgol, a program is a term together with a $\Gamma$-store such
that $\Gamma \vdash M : A$. The evaluation semantics is expressed by
the judgment form:
$$s,M \eval s', V$$
where $s$ and $s'$ are $\Gamma$-stores, $V$ is a canonical form and $\Gamma \vdash V : A$.

The operational semantics for IA is given by the rule of PCF (table \ref{tab:bigstep_pcf})
together with the rules of table \ref{tab:bigstep_ia} where the following abbreviation is used:
$$ \rulef{M_1 \eval V_1 \quad M_2 \eval V_2}{M \eval V} \qquad \mbox{for} \qquad
  \rulef{s,M_1 \eval s',V_1 \quad s', M_2 \eval s'',V_2 }{s,M \eval s'',V}
$$


\begin{table}[htbp]
$$\mbox{\textbf{Sequencing }}
    \rulef{M \eval \iaskip \quad N \eval V}{\texttt{seq } M\ N \eval V}
$$

$$\mbox{\textbf{Variables }}
    \rulef{s,N \eval s',n \quad s',M \eval s'',x}{s, \iaassign\ M\ N \eval (s''\ |\ x \mapsto n),\iaskip}
\qquad
    \rulef{s,M \eval s',x }{s, \iaderef\ M \eval s',s'(x)}$$

$$\mbox{\texttt{\textbf{mkvar}}}
    \rulef{N \eval n \quad M \eval \texttt{mkvar}\ M_1\ M_2 \quad M_1\ n \eval \iaskip}
    {\iaassign\ M\ N \eval \iaskip}
\qquad
    \rulef{N \eval \texttt{mkvar } M_1\ M_2 \quad M_2\ \eval n}
    {\iaderef\ M \eval n}
$$

$$\mbox{\textbf{Block}}
    \rulef{(s\ |\ x \mapsto 0),M \eval (s'\ |\ x \mapsto n),V }
    {s, \texttt{new } x \texttt{ in } M \eval s',V}
$$

\label{tab:bigstep_ia}
\caption{Big-step operational semantics of IA}
\end{table}


\subsection{Game model of \ialgol}

All the strategies used to model PCF are well-bracketed and
innocent. On the other hand, to obtain a model of IA we need to
introduce strategies that are not innocent.
This is necessary to model memory cell variable created with the \texttt{new} operator.
The intuition is that a cell needs to
remember what was the last value written in it in order to be able
to return it when it is read, and this can only be done by looking
at the whole history of moves, not only those present in the P-view.

Hence we now consider the categories $\mathcal{C}$ and $\mathcal{C}_b$.

\subsubsection{Base types}

The type \texttt{com} is modeled by the flat game with a single initial question \texttt{run} and a single answer
\texttt{done}. The idea is that O can request the execution of a command by playing \texttt{run}, P then execute the command
and if it terminates acknowledge it by playing \texttt{done}.

The variable type \texttt{var} is modeled by the game $\mathtt{com^{\bf N} \times exp}$ illustrated below:
\begin{center}
\begin{pspicture}(10cm,1.7cm)
$\rput[b]{0}(3cm,0){
\pstree[treemode=U,levelsep=8ex,nodesep=2pt]
    {\TR[name=R]{\mathtt{ok}}}
    { \TR{\mathtt{write}_0} \TR{\mathtt{write}_1} \TR{\mathtt{write}_2} \TR{\ldots}
    }
}
\rput[b]{0}(8cm,0){
\pstree[levelsep=8ex,nodesep=2pt]
    { \TR[name=R]{\mathtt{read}} }
    { \TR{0} \TR{1} \TR{2} \TR{\ldots} }
    }$
\end{pspicture}
\end{center}

\subsubsection{Constants}

\texttt{skip} is interpreted by the strategy $\{ \epsilon, \iarun \cdot \iadone \}$.
The sequential composition $\mathtt{seq_{exp}}$ is interpreted by the following strategy:
$$
\begin{array}{ccccc}
!\mathtt{com} & \otimes & ! \mathtt{exp} & \multimap & \mathtt{exp}\\
&&&&q\\
\iarun\\
\iadone\\
&&q\\
&&n\\
&&&&n
\end{array}
$$

Assignment \iaassign and dereferencing \iaderef are denoted  by the
following strategies (left and right respectively):
$$
\begin{array}{ccccc}
!\mathtt{var} & \otimes & ! \mathtt{exp} & \multimap & \mathtt{com}\\
&&&&q\\
&&q\\
&&n\\
\iawrite_n\\
\iaok\\
&&&&\iadone
\end{array}
\hspace{3cm}
\begin{array}{ccccc}
!\mathtt{var} & \multimap & \mathtt{exp}\\
&&q\\
\iaread\\
n\\
&&n
\end{array}
$$

\iamkvar is modeled by the paired strategy $\langle \iamkvar_{acc} , \iamkvar_{exp}
\rangle$ where $\iamkvar_{acc}$ and $\iamkvar_{exp}$ are the following strategies:
$$
\begin{array}{ccccccc}
(\mathtt{!exp} & \multimap & \mathtt{com}) & \otimes & !\mathtt{exp} & \multimap & \mathtt{com}^\omega\\
&&&&&&\iawrite_n\\
&&\iarun\\
q\\
n\\
&&\iadone \\
&&&&&&\iaok
\end{array}
\hspace{2cm}
\begin{array}{ccccccc}
(\mathtt{!exp} & \multimap & \mathtt{com}) & \otimes & !\mathtt{exp} & \multimap & \mathtt{exp}\\
&&&&&&\iaread\\
&&&&\iaread\\
&&&&n\\
&&&&&&n
\end{array}
$$


The strategies used until now are all innocent. In order to model the \ianew operator, we need to introduce non-innocent strategies, sometimes called
\emph{knowing strategies}. We define the knowing well-bracketed strategy $cell : I \multimap !\mathtt{var}$ that models a storage cell: it responds to \iawrite\
with \iaok\ and responds
to \iaread\ with the last value written or $0$ if no value has yet been written.

Consider the term $\Gamma,x:\mathtt{var} \vdash M : A$ modeled by $\sem{M}$ then the term
 $\Gamma \vdash \ianew\ x \texttt{ in } M : A$  will be modeled by the strategy $(id_{\sem{\Gamma}} \otimes cell) \fatsemi\fatsemi \sem{M}$ on the game
 $!\Gamma \multimap \iacom$.

\subsection{Full abstraction of \ialgol}

We now state the full abstraction result. All the details are omitted, the reader is refered
to \cite{abramsky:game-semantics-tutorial,AM97a} for the proofs.

\subsubsection{Inequational soundness}

The inequational soundness result can be also proved for \ialgol.
Proving soundness of the evaluation requires a bit more work than in the PCF case because
the store needs to be made explicit. Also, an appropriate notion of \emph{computable term} must be defined
that takes into account the presence of stores in the evaluation semantics.
Again it is possible to prove that the model is computational adequate.
The inequational soundness then follows from evaluation soundness and computational adequacy:

%\begin{lem}[Soundness for IA terms] Let $\Gamma \vdash M : A$ be an IA term and a $\Gamma$ store $s$.
%If $s,M \eval s',V$ then the plays of $\sem{s,M} : I \multimap A
%\otimes !\Gamma$ which begin with a move of $A$ are identical to
%those of $\sem{s',V}$.
%\end{lem}

\begin{prop}[Inequational soundness]
\label{prop:ia_ineqsoundness} Let $M$ and $N$ be two \ialgol\ closed terms then
$$\sem{M} \subseteq \sem{N} \implies  M \obspre N $$
\end{prop}

\subsubsection{Definability}

The proof of definability is based on a factoring argument: strategies in
$\mathcal{G}_b$ can all be obtained by composing the non-innocent strategy $cell$ with an innocent strategy.
The strategy $cell$ can therefore be viewed as a generic non-innocent strategy. Using this factorization argument,
it is possible to prove the definability result:
\begin{prop}[Definability]
\label{prop:ia_definability} Let $\sigma$ be a compact well-bracketed
strategy on a game $A$ denoting a IA type. There is an IA-term $M$ such
that $\sem{M} = \sigma$.
\end{prop}

\subsubsection{Full abstraction}

Full abstraction for the model $\mathcal{C}_b$ is a consequence of proposition
\ref{prop:ia_ineqsoundness} and \ref{prop:ia_definability}:
\begin{thm}[Full abstraction]
Let $M$ and $N$ be two closed \ialgol-terms.
$$\sem{M} \precsim_b \sem{N} \ \iff \ M \obspre N$$
where $\precsim_b$ denotes the intrinsic preorder of the category
$\mathcal{C}_b$.
\end{thm}


\section{Algorithmic game semantics}

After the resolution of the ``Full Abstraction of PCF'' problem,
game semantics has become a very successful paradigm in fundamental
computer science. It has permitted to give full abstract semantics
for a variety of programming languages. More recently, game
semantics has emerged as a new approach to program verification and
program analysis. In particular in the paper \cite{ghicamccusker00},
the authors considered a fragment of Idealized Algol for which the
game semantics of programs can be expressed simply using regular
expressions. In this setting, observational equivalence of programs
becomes decidable. Consequently, numbers of interesting verification
problem become solvable. This development opened up a new direction
of research called \emph{Algorithmic game semantics}.

\subsection{Characterization of observational equivalence}

In \citep{AM97a} it is shown that observational equivalence of IA is
characterized by complete plays.

A play of a game is \emph{complete} if it is maximal and all
question have been answered. A game is \emph{simple} if the complete
plays are exactly those in which the initial question has been
answered. It can be shown that for any IA type $T$, $\sem{T}$ is a
simple game. The following characterization theorem holds for simple
games:
\begin{thm}[Characterization Theorem for Simple Game (Abramsky, McCusker 1997)]
Let $\sigma$ and $\tau$ be strategies on a simple game $A$ then:
$$\sigma \leq \tau \iff \textsf{comp}(\sigma) = \textsf{comp}(\tau)$$
\end{thm}
Therefore terms in IA are fully described by the complete plays of
the corresponding strategies.

\subsection{Finitary fragments of Idealized algol}
We introduce
some fragments of the language \ialgol. Firstly, \emph{Finitary
Idealized Algol} denotes the recursion-free sub-fragment of \ialgol\
over finite ground types. A term $\Gamma \vdash M:T$ of finitary
Idealized algol is an $i^{th}$-order term if $T$ is of order $i$ at
most and the variables in $\Gamma$ are of order strictly less than
$i$. $\ialgol_i$ denotes the fragment of finitary Idealized Algol
consisting of terms of $i^{th}$-order terms. $\ialgol_i +
\textsf{while}$ denotes the fragment $\ialgol_i$ augmented with
primitive recursion. Finally $\ialgol_i + \textsf{Y}_j$ where $j
\leq i$ denotes the fragment $\ialgol_i$ augmented with a set of
fixed-point iterators $\textsf{Y}_A : (A\rightarrow A ) \rightarrow
A$ for any type $A$ of order $j$ at most.

We recall the observational equivalence decision problem: given two
$\beta$-normal forms $M$ and $N$ in a given fragment of \ialgol,
does $M \approx N$ hold?

This problem has been investigated and decidability results have
been obtained for a complete class of fragments of Idealized Algol.
These results help us to understand the limits of Algorithmic Game
Semantics. We now present briefly those results.

\subsubsection{$\ialgol_2$ fragment}
In \cite{ghicamccusker00}, Dan R. Ghica and Guy McCusker considered the $\ialgol_2$ fragment.
They show that in $\ialgol_2$ the set of complete plays are
representable by extended regular languages.

\begin{lem}[Ghica and McCusker 2000]
For any $\ialgol_2$-term $\Gamma \vdash M : T$, the set of complete
plays of $\sem{\Gamma \vdash M : T}$ is regular.
\end{lem}
Since equivalence of regular expression is decidable, this shows
decidability of observational equivalence of $\ialgol_2$-terms. In
the same paper they show that the same result holds for the
$\ialgol_2 +\textsf{while}$ fragment.

In \cite{Ong02}, it is shown that observational equivalence is
undecidable for $\ialgol_2 + \textsf{Y}_1$.


\subsubsection{Other fragments of IA}

Observational equivalence is decidable for $\ialgol_3$. This is
proved in \cite{Ong02} by reduction to the \emph{Deterministic
Push-down Automata Equivalence} problem. Unfortunately, this result
does not extend beyond order $3$: Murawski showed in
\cite{murawski03program} that the problem is undecidable for
$\ialgol_i$ with $i\geq4$.

However in $\ialgol_3 + \textsf{while}$ the problem becomes
decidable: it is shown in \cite{C:MW05} that the problem is EXPTIME
in $\ialgol_2 + \textsf{while}$ and $\ialgol_3 + \textsf{while}$.

Moreover in \cite{C:MOW05} it is shown that $\ialgol_i +Y_0$, for $i
= 1, 2, 3$ is as difficult as the DPDA equivalence problem. This
problem is decidable \citep{DBLP:journals/tcs/Senizergues01} but no
complexity result is known about it. We only know that it is
primitive recursive \citep{stirling02}.

\subsubsection{The complete classification}
\begin{center}
\begin{tabular}{rcccc}
Fragment  & pure & +while & +Y0 & +Y1 \\ \hline \hline
$\ialgol_0$ & PTIME & ??? & ??? & $\times$  \\
$\ialgol_1$ & coNP & PSPACE & DPDA EQUIV & ??? \\
$\ialgol_2$ & PSPACE & PSPACE & DPDA EQUIV & undecidable \\
$\ialgol_3$ &EXPTIME & EXPTIME & DPDA EQUIV & undecidable \\
$\ialgol_i, i \geq 4$  & undecidable & undecidable & undecidable
& undecidable
\end{tabular}
\end{center}

The $\times$ symbol denotes undefined \ialgol\ fragments.

The coNP and PSPACE results are due to Murawski \citep{Mur04b}.

%\input dataref


% second chapter
\chapter{Safe $\lambda$-calculus}
    \section{Background}

\todobox{background on safety}

\subsection{Homogeneous type}
\label{sec:homotypes}

Let $Types$ be the set of simple types generated by the grammar $A
\, ::= \, o \; | \; A \funsp A$. Any type different from the base
type $o$ can be written $(A_1, \cdots, A_n, o)$ for some $n \geq 1$,
which is a shorthand for $A_1 \funsp \cdots \funsp A_n \funsp o$ (by
convention, $\rightarrow$ associates to the right).

We suppose that a ranking function has been defined: ${\sf rank} :
Types \funto (L, \leq)$ where $(L, \leq)$ is any linearly ordered
set. Possible candidates for the ranking function are:
\begin{itemize}
\item ${\sf ord} : Types \funto (\nat,\leq)$ with $\ord{o} = 0$
and $\ord{A \funsp B} = \max(\ord{A}+1, \ord{B})$.
\item ${\sf height} : Types \funto (\nat,\leq)$ with
\begin{eqnarray*}
     \slheight{o} &=& 0 \\
     \slheight{A \funsp B} &=& 1 + \max(\slheight{A}, \slheight{B})
\end{eqnarray*}
\item ${\sf nparam} : Types \funto (\nat,\leq)$ with $\nparam{o} = 0$
and $\nparam{A_1, \cdots, A_n} = n$.
\item ${\sf ordernp} : Types \funto (\nat \times \nat,\leq)$ with $ {\sf ordernp} (t)  = (\order{t}, \nparam{t})$ for $t \in Types$.
\end{itemize}


Following \cite{KNU02}, a type is rank-homogeneous if it is $o$ or
if it is $(A_1, \cdots, A_n, o)$ with the condition that $\rank{A_1}
\geq \rank{A_2}\geq \cdots \geq \rank{A_n}$ and each $A_1$, \ldots,
$A_n$ is rank-homogeneous.



Suppose that $\overline{A_1}$, $\overline{A_2}$, \ldots,
$\overline{A_n}$ are $n$ lists of types, where $A_{ij}$ denotes the
$j^{th}$ type of list $\overline{A_i}$ and $l_i$ the size of
$\overline{A_i}$. Then the notation $A \; = \; (\overline{A_1} \, |
\, \cdots \, | \, \overline{A_r} \, | \, o)$ means that
\begin{itemize}
  \item $A$ is the type $(A_{11},A_{12},\cdots, A_{1l_1}, A_{21}, \cdots,A_{2l_2}, \cdots A_{n1},\cdots, A_{nl_n},o)$
  \item $\forall i: \forall u,v \in A_i : \rank u = \rank v $
  \item $\forall i,j . \forall u \in A_i . \forall v \in A_j . i<j \implies \rank u >
   \rank v $
\end{itemize}
Consequently, $A$ is rank-homogenous. This notation organises the
$A_{ij}$s into partitions according to their ranks. Suppose $B =
(\overline{B_1} \, | \, \cdots \, | \, \overline{B_m} \, | \, o)$.
We write $(\overline{A_1} \, | \, \cdots \, | \, \overline{A_n} \, |
\, {B})$ to mean
\[(\overline{A_1} \, | \, \cdots \, | \, \overline{A_n} \, | \,
\overline{B_1} \, | \, \cdots \, | \, \overline{B_m} \, | \, o).\]

From now on, we only consider the rank function {\sf ord}. The term
``homogeneous'' will be used to refer to {\sf ord}-homogeneity.

    \section{Homogeneous safe $\lambda$-calculus}
\label{sec:safe_homog}

\subsection{Type homogeneity}
Let $Types$ be the set of simple types generated by the grammar $A
\, ::= \, o \; | \; A \funsp A$. Any type different from the base
type $o$ can be written $(A_1, \cdots, A_n, o)$ for some $n \geq 1$,
which is a shorthand for $A_1 \funsp \cdots \funsp A_n \funsp o$ (by
convention, $\rightarrow$ associates to the right).

We suppose that a ranking function has been defined: ${\sf rank} :
Types \funto (L, \leq)$ where $(L, \leq)$ is any linearly ordered
set. Possible candidates for the ranking function are:
\begin{itemize}
\item ${\sf ord} : Types \funto (\nat,\leq)$ with $\ord{o} = 0$
and $\ord{A \funsp B} = \max(\ord{A}+1, \ord{B})$.
\item ${\sf height} : Types \funto (\nat,\leq)$ with
\begin{eqnarray*}
     \slheight{o} &=& 0 \\
     \slheight{A \funsp B} &=& 1 + \max(\slheight{A}, \slheight{B})
\end{eqnarray*}
\item ${\sf nparam} : Types \funto (\nat,\leq)$ with $\nparam{o} = 0$
and $\nparam{A_1, \cdots, A_n} = n$.
\item ${\sf ordernp} : Types \funto (\nat \times \nat,\leq)$ with $ {\sf ordernp} (t)  = (\order{t}, \nparam{t})$ for $t \in Types$.
\end{itemize}


Following \cite{KNU02}, a type is rank-homogeneous if it is $o$ or
if it is $(A_1, \cdots, A_n, o)$ with the condition that $\rank{A_1}
\geq \rank{A_2}\geq \cdots \geq \rank{A_n}$ and each $A_1$, \ldots,
$A_n$ is rank-homogeneous.



Suppose that $\overline{A_1}$, $\overline{A_2}$, \ldots,
$\overline{A_n}$ are $n$ lists of types, where $A_{ij}$ denotes the
$j$th type of list $\overline{A_i}$ and $l_i$ the size of
$\overline{A_i}$. Then the notation $A \; = \; (\overline{A_1} \, |
\, \cdots \, | \, \overline{A_r} \, | \, o)$ means that
\begin{itemize}
  \item $A$ is the type $(A_{11},A_{12},\cdots, A_{1l_1}, A_{21}, \cdots,A_{2l_2}, \cdots A_{n1},\cdots, A_{nl_n},o)$
  \item $\forall i: \forall u,v \in A_i : \rank u = \rank v $
  \item $\forall i,j . \forall u \in A_i . \forall v \in A_j . i<j \implies \rank u >
   \rank v $
\end{itemize}
Consequently, $A$ is rank-homogenous. This notation organises the
$A_{ij}$s into partitions according to their ranks. Suppose $B =
(\overline{B_1} \, | \, \cdots \, | \, \overline{B_m} \, | \, o)$.
We write $(\overline{A_1} \, | \, \cdots \, | \, \overline{A_n} \, |
\, {B})$ to mean
\[(\overline{A_1} \, | \, \cdots \, | \, \overline{A_n} \, | \,
\overline{B_1} \, | \, \cdots \, | \, \overline{B_m} \, | \, o).\]

From now on, we only consider the rank function {\sf ord}. The term
``homogeneous'' will be used to refer to {\sf ord}-homogeneity.


\subsection{Rules}

We present here the definition of the safe $\lambda$-calculus given
in \cite{Ong2005} (a corrected version of
\cite{DBLP:conf/fossacs/AehligMO05}). In the following we shall
consider terms-in-context $\seq{\Gamma}{M : A}$ of the simply-typed
$\lambda$-calculus. Let $\Delta$ be a simply-typed alphabet i.e.,
each symbol in $\Delta$ has a simple type. We write
$\terms{A}{\Delta}$ for the set of terms of type $A$ built up from
the set $\Delta$ understood as constant symbols, \emph{without}
using $\lambda$-abstraction.


The \textbf{Safe $\lambda$-Calculus} is a sub-system of the
simply-typed $\lambda$-calculus. Typing judgements (or
terms-in-context) are of the form
\begin{equation}
\nonumber \seq{\overline{x_1}:\overline{A_1} \, | \, \cdots \, | \,
\overline{x_n} :  \overline{A_n}}{M : B}
\end{equation}
which is shorthand for $\seq{x_{11} : A_{11}, \cdots, x_{1r}:
A_{1r}, A_{21},\ldots }{M : B}$ such that the context variables are listed in decreasing type order and
 with the condition that $\ord{x_{ik}} < \ord{x_{jl}}$ for any $k, l$ and $i<j$.

\emph{Valid typing judgements} of the
system are defined by induction over the following rules, where
$\Delta$ is a given homogeneously-typed alphabet:

$$ \rulename{wk}
    {   \rulef{ \seq{\Sigma}{M:B} \qquad {\Sigma \subset \Delta} }
             { \seq{\Delta }{M : B}}
   }
\qquad
    \rulename{perm}
    {
      \rulef { \seq{\Gamma}{M:B} \qquad \sigma(\Gamma) \hbox{ homogeneous} }
            { \seq{\sigma(\Gamma)}{M : B} }
    }
$$


$$ \rulename{\Sigma\mbox{\textbf{-const}}}  \rulef{b : o^r \rightarrow o \in \Sigma} {\seq{}{b : o^r \rightarrow o}}
\qquad
 \rulename{var} \rulef{}{\seq{x_{ij} : A_{ij}\, }{x_{ij} : A_{ij}}}
$$

$$ \rulename{\lambda\mbox{\textbf{-abs}}}
\rulef{ {\seq{\overline{x_1} : \overline{A_1}\, | \, \cdots\, | \,
\overline{x_{n+1}} : \overline{A_{n+1}}}{M : B}} \qquad
\ord{\overline{A_{n+1}}} \geq \ord{B} -1}
    {\seq{\overline{x_1} :
\overline{A_1}\, | \, \cdots\, | \, \overline{x_{n}} :
\overline{A_{n}}}{\lterm{\overline{x_{n+1}} : \overline{A_{n+1}}}{M}
: (\overline{A_{n+1}} \, | \, B)}} $$

$$ \rulename{app} \rulef{{\seq{\Gamma}{M : (\overline{B_1} \, | \, \cdots \, | \, \overline{B_m} \, | \, o)} \qquad
\seq{\Gamma}{N_1 : B_{11}} \quad \cdots \quad \seq{\Gamma}{N_{l} :
B_{1l}} \qquad l = |\overline{B_1}| }}
    { \seq{\Gamma}{M N_1
\cdots N_{l} : (\overline{B_2} \, | \, \cdots \, | \,
\overline{B_m} \, | \, o)}} $$


$$ \rulename{app^+} \rulef
    {\seq{\Gamma}{M : (\overline{B_1} \, | \, \cdots \, | \, \overline{B_m} \, | \, o)} \qquad
    \seq{\Gamma}{N_1 : B_{11}} \quad \cdots \quad \seq{\Gamma}{N_{l} :
    B_{1l}} \qquad l < |\overline{B_1}| }
    { \seq{\Gamma}{M N_1
    \cdots N_{l} : (\overline{B} \, | \, \cdots \, | \,
    \overline{B_m} \, | \, o)}} $$

where $\overline{B_1} = B_{11}, \ldots, B_{1l},\overline{B}$ with
the condition that every variable in $\Gamma$ has an order strictly greater
than $\ord{\overline{B_1}}$.


\begin{property}[Basic properties]
\label{proper:safe_basic_prop} Suppose $\Gamma \vdash M : B$ is a
valid judgment then
\begin{itemize}
\item[(i)] $B$ is homogeneous
\item[(ii)] Every free variables of $M$ has order at least $\ord{M}$
\item[(iii)] $fv(M) \vdash M : B$
\end{itemize}
where $fv(M) \subseteq \Gamma$ denotes the context constituted of the variables
in $\Gamma$ occurring free in $M$.
\end{property}
\begin{proof}
(i) and (ii) An easy proof by structural induction.
(iii) Because the weakening rule is the only rule which can introduce in the context some variable not occurring freely in $M$.
\end{proof}

We now define a special kind of substitution that performs simultaneous substitution and
that permits variable capture (i.e. that does not rename variables when the substitution is performed on an abstraction).

\begin{dfn}[Capture permitting simultaneous substitution (for homogeneous safe terms)]
\label{dnf:safe_simsubst}
 We use the notation $\subst{\overline{N}}{\overline{x}}$ for $\subst{N_1 \ldots N_n}{x_1 \ldots x_n}$ and
$\overline{y}:\overline{A}$ for $y_1:A_1, \ldots, y_p:A_p$.
A safe term must have one of the forms occurring on the left-hand side of the following equations, where
the terms $M$, $N_1, \ldots N_l$ are safe terms:
\begin{eqnarray*}
c \subst{\overline{N}}{\overline{x}} &=& c \quad \mbox{ where $c$ is a $\Sigma$-constant}\\
x_i \subst{\overline{N}}{\overline{x}} &=& N_i\\
 y \subst{\overline{N}}{\overline{x}} &=& y \quad \mbox{ if } y \not \neq x_i \mbox{ for all } i,\\
(M N_1 \ldots N_l) \subst{\overline{N}}{\overline{x}} &=& (M \subst{\overline{N}}{\overline{x}}) (N_1 \subst{\overline{N}}{\overline{x}}) \ldots  (N_l \subst{\overline{N}}{\overline{x}})\\
(\lambda \overline{y} : \overline{A}. M)
\subst{\overline{N}}{\overline{x}} &=& \lambda \overline{y} . M
\subst{\overline{N} \upharpoonright I}{\overline{x} \upharpoonright I} \\
&& \mbox{where } I  = \{ i \in 1..n \ | \ x_i \not \in \overline{y} \}
\end{eqnarray*}

where $ \upharpoonright$ is the index filtering operator: if $s$ is a sequence and $I$ a set of indices then
$s \upharpoonright I$ is the subsequence of $s$ obtained by removing from $s$ all the elements
at a position that is not in $I$.
\end{dfn}

This substitution is well-defined for safe terms in the sense that safety is preserved by substitution:

\begin{lem}[Capture-permitting simultaneous substitution preserves safety]
\label{lem:subst_preserve_safety}
Let $\Gamma \union \overline{x} \vdash M$ be a safe term where $\overline{x}$ denotes a list of variables
(not necessarily belonging to the same partition).

Then for any safe terms safe terms $\Gamma \vdash N_1, \cdots, \Gamma \vdash N_n$,
the capture permitting simultaneous substitution $M[N_1 / x_1 , \cdots, N_n / x_n]$ is safe. i.e. the following judgment is valid :
$$ \Gamma \vdash M[N_1 / x_1 , \cdots, N_n / x_n] $$
\end{lem}
\begin{proof}
An easy proof by an induction on the structure of the safe term.
\end{proof}



With the traditional substitution, it is necessary to rename variables when performing substitution on an abstraction
in order to avoid possible captures of variables. Hence in general implementing substitution requires to have
access to an unbound number
of variables names.
An interesting property of the homogeneous safe $\lambda$-calculus is that
variable capture never occurs when performing substitution.
Consequently, the capture permitting substitution will produce the same terms as the traditional substitution:

\begin{lem}[No variable capture lemma]
\label{lem:homog_nocapture}
In the safe $\lambda$-calculus, there is no capture of variable
when performing the following capture permitting simultaneous substitution:
$$ M[N_1 / x_1 , \cdots, N_n / x_n] $$
provided that $\Gamma \union \overline{x} \vdash M$, $\Gamma \vdash  N_1, \cdots ,\Gamma \vdash  N_n$ are valid judgments.
\end{lem}

\begin{proof}
We prove the result by induction. The variable, constant and application cases are trivial.
For the abstraction case, suppose $M = \lambda \overline{y} : \overline{A}. P$ where $\overline{y} = y_1 \ldots y_p$. The capture permitting
simultaneous substitution gives:
$$M \subst{\overline{N}}{\overline{x}} = \lambda \overline{y} . P
\subst{\overline{N} \upharpoonright I}{\overline{x} \upharpoonright I}$$

where $I  = \{ i \in 1..n \ | \ x_i \not \in \overline{y} \}$.


By the induction hypothesis there is no variable capture in $P
\subst{\overline{N} \upharpoonright I}{\overline{x} \upharpoonright I}$.

Hence the only possible case of variable capture happens when for some $i \in I$ and $j \in 1..p$ the variable $y_j$ occurs freely
in $N_i$ and $x_i$ occurs freely in $P$. In that case, property \ref{proper:safe_basic_prop} (ii) gives:
$$ \ord{y_j} \geq \ord{N_i} = \ord{x_i}$$

Moreover $i\in I$ therefore $x_i \not \in \overline{y}$ and since $x_i$ occurs freely in $P$, $x_i$ must also occur freely in the safe term
$\lambda \overline{y}. P$. Thus, property \ref{proper:safe_basic_prop} (ii) gives:
$$ \ord{x_i} \geq \ord{\lambda y_1 \ldots y_p . T} \geq 1+ \ord{y_j} > \ord{y_j}$$

Hence we reach a contradiction.
\end{proof}




\subsection{Safe $\beta$-reduction}

We now introduce the notion of safe $\beta$-redex and show how such redex can be reduced using the
capture-permitting simultaneous substitution. We will then show that
a safe $\beta$-reduction reduces to a safe term.


In the simply-typed lambda calculus a redex is a term of the form $(\lambda x . M) N$.
We generalize this notion to the safe lambda calculus. We call multi-redex a term of the form
$(\lambda x_1 \ldots x_n . M) N_1 \ldots N_l$ (it is not required to have $n=l$).


We say that a multi-redex is safe if it respects the formation rules of the safe $\lambda$-calculus. More precisely,
the multi-redex $(\lambda x_1 \ldots x_n . M) N_1 \ldots N_l$ is a safe redex if the variable $x_1,\ldots,x_n$
are abstracted altogether at once using the abstraction rule and if the terms $N_1 \ldots N_l$ are applied to the
term $\lambda x_1 \ldots x_n . M$ at once using either the application rule $\rulename{app^+}$ or $\rulename{app}$.

Note that there exist safe terms of the form $(\lambda x_1 \ldots x_n . M) N_1 \ldots N_l$
such that $l>n$. For instance the following term:
$$ (\lambda f g . ((\lambda h i . i) a) ) (\lambda x.x) (\lambda x.x) (\lambda x.x)$$
where is $a$ constant, $x : o$ and $f,g,h,i,a:o \rightarrow o$ can be formed using the $\rulename{app}$ rule as follows:
$$ \rulef{
    \emptyset \vdash (\lambda f g . ((\lambda h i . i) a) )
        \quad \emptyset \vdash (\lambda x.x)
        \quad \emptyset \vdash (\lambda x.x)
        \quad \emptyset \vdash (\lambda x.x)
    }
    {
       \emptyset \vdash (\lambda f g . ((\lambda h i . i) a) ) (\lambda x.x) (\lambda x.x) (\lambda x.x)
    } \rulename{app}
$$


The formal definition follows:

\begin{dfn}[Safe redex]
A safe redex is a term of the form:
$$(\lambda \overline{x} . M) N_1 \ldots N_l$$
such that the variables $\overline{x}=x_1\ldots x_n$ are abstracted altogether by one occurrence
of the rule $\rulename{abs}$ in the proof tree (possibly followed by the weakening rule)
which implies that:
$$\ord{M} -1 \leq \ord{\overline{x}} = \ord{x_1} = \ldots = \ord{x_n}$$
and such that the terms $(\lambda \overline{x} . M)$, $N_1$, $N_l$ are applied together at
once using either:
\begin{itemize}
    \item the rule $\rulename{app}$ with
        $$   \rulef{
                    \Sigma \vdash \lambda \overline{x} . M : (\overline{B_1}|\ldots|\overline{B_m}|o)
                    \quad
                    \Sigma \vdash N_1         \quad \ldots \quad \Sigma \vdash N_l
                    \quad l = |\overline{B_1}|
            }
            {
            \Sigma \vdash (\lambda \overline{x} . M) N_1 \ldots N_l
            } (\mathbf{app})
        $$

        In which case  $n\leq |\overline{B_1}| = l$.

\item or the rule $\rulename{app^+}$ with:
        $$   \rulef{
                    \Sigma \vdash \lambda \overline{x} . M : (\overline{B_1}|\ldots|\overline{B_m}|o)
                    \quad
                    \Sigma \vdash N_1         \quad \ldots \quad \Sigma \vdash N_l
                    \quad l < |\overline{B_1}|
            }
            {
            \Sigma \vdash (\lambda \overline{x} . L) N_1 \ldots N_l
            } (\mathbf{app^+})
        $$

      In which case $n \leq |\overline{B_1}|$ and no relation holds between $n$ and $l$.
\end{itemize}
Note that we do not necessarily have $n = |\overline{B_1}|$.
\end{dfn}


\begin{dfn}[Safe reduction $\beta_s$] \
For concision the following abbreviations are used $\overline{x} = x_1 \ldots
x_n$, $\overline{N} = N_1 \ldots N_l$, and when $n\geq l$, $\overline{x_L} = x_1 \ldots
x_l$, $\overline{x_R} = x_{l+1} \ldots x_n$.
\begin{itemize}
\item The relation $\beta_s$ is defined on the set of safe redex as follows:
\begin{eqnarray*}
\beta_s &=&
\{  \ (\lambda \overline{x} : \overline{A} . T) N_1 \ldots N_l \mapsto \lambda \overline{x_R}. T\subst{\overline{N}}{\overline{x_L}}  \\
&& \mbox{ where $(\lambda \overline{x} : \overline{A} . T) N_1 \ldots N_l$ is a safe redex such that $n> l$}
\} \\
&\union&
\{ \ (\lambda \overline{x} : \overline{A} . T) N_1 \ldots N_l \mapsto T\subst{\overline{N}}{\overline{x}} N_{n+1} \ldots N_l  \\
&& \mbox{ where $(\lambda \overline{x} : \overline{A} . T) N_1 \ldots N_l$ is a safe redex such that $n\leq l$}
\}
\end{eqnarray*}
where the notation $\subst{\overline{N}}{\overline{x}}$ denotes the capture-permitting simultaneous substitution.

\item
The safe $\beta$-reduction noted $\betasred$ is the closure
of the relation $\beta_s$ by compatibility with the formation rules
of the safe $\lambda$-calculus.
\end{itemize}
\end{dfn}



We observe that safe $\beta$-reduction is a certain kind of multi-steps $\beta$-reduction.
\begin{property}
$\betasred \subset \betaredtr$, i.e. the safe
$\beta$-reduction relation is included in the transitive closure of the $\beta$-reduction relation.
\end{property}
\begin{proof}
Suppose that $(M\mapsto N) \in \beta_s$. We show that $M \betared^* N$.
\begin{itemize}
\item Suppose that the safe-redex is
$M \equiv (\lambda \overline{x} : \overline{A} . T) N_1 \ldots N_l$ such that $n\leq l$ then:
\begin{eqnarray*}
 M &=_\alpha& (\lambda z_1 \ldots z_n .T [z_1,\ldots z_n /x_1,\ldots x_n] ) \ N_1  N_2 \ldots N_l
            \\
&& \mbox{where the $z_i$ are fresh variables}  \\
     &\betared& (\lambda z_2 \ldots z_n .T [z_1,\ldots z_n /x_1,\ldots x_n] \subst{N_1}{z_1} ) \ N_2 \ldots N_l \\
&& \mbox{ (because the $z_i$ do not occur freely in $N_1$) }\\
%%    &=_\alpha& (\lambda z_2 \ldots z_n .T [z_2,\ldots z_n /x_2,\ldots x_n] \subst{N_1}{x_1})\  N_2 \ldots N_l  \qquad \mbox{where the $z_i$ are fresh variables}  \\
    &\betared& \ldots \\
    &\betared& (T [z_1,\ldots z_n /x_1,\ldots x_n] \subst{N_1}{z_1}  \ldots \subst{N_n}{z_n})\  N_{n+1} \ldots N_l \\
    &\betared& (T [N_1\ldots N_l/x_1,\ldots x_l])\ N_{n+1} \ldots N_l
\end{eqnarray*}
And since $T$ is safe, the substitution $T [N_1\ldots N_l/x_1,\ldots x_l]$ in the last equation
can be performed using the capture-permitting substitution. Hence $M \betared^* N$.

\item
 Suppose that $M \equiv (\lambda \overline{x} : \overline{A} . T) N_1 \ldots N_l$ such that $n> l$, then necessarily
the redex must be formed using the $\rulename{app^+}$ rule. The side-condition of this rules
says that the free variables of the terms $N_1, \ldots N_l$ have all order strictly greater than $\ord{\overline{x}}$,
hence the $x_i$ do not occur freely in $N_1, \ldots N_l$. Therefore:

\begin{eqnarray*}
 M &=& (\lambda x_1 \ldots x_n .T) \ N_1  N_2 \ldots N_l  \\
     &\betared& (\lambda x_2 \ldots x_n .T \subst{N_1}{x_1} ) \ N_2 \ldots N_l \\
            && \mbox{(for $i \in 2..n$, $x_i$ does not occur freely in $N_1$)}\\
    &\betared& \ldots \\
    &\betared& \lambda x_{l+1} \ldots x_n . T \subst{N_1}{x_1}  \ldots \subst{N_l}{x_l} \\
        && \mbox{(for $i \in (l+1)..n$,  $x_i$ does not occur freely in $N_l$)}\\
    &\betared& \lambda x_{l+1} \ldots x_n . T [N_1\ldots, N_l /  \ x_1,\ldots, x_l] \\
        && \mbox{(the $x_i$ do not occur freely in $N_1, \ldots N_l$)}
\end{eqnarray*}
And since $T$ is safe, the substitution $T [N_1\ldots N_l/x_1,\ldots x_l]$ in the last equation
can be performed using the capture-permitting substitution. Hence $M \betared^* N$.
\end{itemize}
\end{proof}

\begin{property} In the simply typed $\lambda$-calculus setting:
\begin{enumerate}
\item $\betasred$ is strongly normalizing.
\item $\beta_s$ has the unique normal form property.
\item $\beta_s$ has the Church-Rosser property.
\end{enumerate}
\end{property}

\begin{proof}
1. This is because $\betasred \subset \betaredtr$ and $\betared$ is strongly normalizing (in the simply typed lambda calculus).
2. A term has a safe redex iff it has a $\beta$-redex therefore
the set of $\beta_s$ normal form is equal to the set of $\beta_s$
normal form. Hence, the unicity of $\beta$ normal form implies the
unicity of $\beta_s$ normal form.
3. is a consequence of 1 and 2.
\end{proof}




Since capture-permitting simultaneous substitution preserves safety (lemma \ref{lem:subst_preserve_safety}),
consequently any safe redex reduces to a safe term:

\begin{lem}[The safe reduction $\beta_s$ preserves safety]
\label{lem:homoh_safered_preserve_safety}
If $M$ is safe and $M \betasred N$ then $N$ is safe.
\end{lem}

\begin{proof}
It suffices to show that the relation $\beta_s$ preserves safety.
Consider the safe-redex $(s\mapsto t) \in \beta_s$ where $ s \equiv (\lambda x_1 \ldots x_n . M) N_1 \ldots N_l $ .
We proceed by case analysis on the the last rule used to form the redex.
\begin{itemize}
\item Suppose the last rules used is $\rulename{app}$, then necessarily $n\leq l$ and the reduction is:
$$(\lambda x_1 \ldots x_n . M) N_1 \ldots N_l \qquad \mapsto  \qquad t \equiv M[N_1 / x_1 , \cdots, N_n / x_n]\ N_{n+1} \ldots N_l$$
The first premise of the rule $\rulename{app}$ tells us that $M$ is safe therefore using lemma \ref{lem:subst_preserve_safety} and
the application rule we obtain that $t$ is safe.

\item Suppose the last rules used is $\rulename{app^+}$ and $n> l$ then the reduction is
$$ (\lambda \overline{x_L} : \overline{A_L} \
\overline{x_R}: \overline{A_R} . T) \overline{N_L} \qquad \mapsto
\qquad t \equiv \lambda \overline{x_R}: \overline{A_R} .
T\subst{\overline{x_L}}{\overline{N_L}}
$$
By lemma \ref{lem:subst_preserve_safety}, $T\subst{\overline{x_L}}{\overline{N_L}}$ is a safe term.
Using the rule $\rulename{abs}$ we derive that $t$ is safe.

\item Suppose the last rules used is $\rulename{app^+}$ and $n\leq l$ then the reduction is
$$(\lambda x_1 \ldots x_n . M) N_1 \ldots N_l \qquad \mapsto \qquad t \equiv M[N_1 / x_1 , \cdots, N_n / x_n]\ N_{n+1} \ldots N_l$$
Similarly to case $\rulename{app}$, we conclude that $t$ is safe.

\item Rule $\rulename{wk}$ $\rulename{seq}$: these cases reduce to one of the previous cases.
\end{itemize}
\end{proof}


\begin{rem}
\label{rem:betasred_notpreserv_unsafety} $\betasred$ \emph{does not} preserves un-safety: given two terms of
the same type $S$ safe and $U$ unsafe, the term $(\lambda x y . y) U S$ is also unsafe
but it $\beta_s$-reduces to $S$ which is safe.
\end{rem}


\subsection{An alternative system of rules}


In this section, we will refine the formation rules
given in the previous section. We say that $\Gamma \vdash M : A$ verifies $P_i$ for $i \in \zset$ if the
variables in $\Gamma$ all have orders at least $\ord{A}+i$. We introduce the notation $\Gamma \vdash^{i} M : A$ for $i \in
\zset$ to mean that $\Gamma \vdash M : A$ is a valid judgment satisfying $P_i$.


We remark that if $\Gamma \vdash M : A$ then the variables in $\Gamma$ with order
strictly smaller than $M$ cannot occur freely in $M$ and therefore it is possible to restrict
the context to a smaller number of variables:

\begin{lem}[Context reduction]
\label{lem:restriction}

If $\Gamma \vdash^i M : A$ then $\Gamma' \vdash^{0} M : A$
where $$\Gamma' = \{ z \in \Gamma \ |
\ \ord{z} \geq \ord{M} \} = \Gamma \setminus \{ z \in \Gamma \ | \ \ord{M} + i \leq \ord{z} < \ord{M} \}$$
\end{lem}
\begin{proof}
If $i\geq 0$ then the result is trivial.
Suppose $i<0$. We proceed by structural induction and by case analysis.
We only give the details for the application cases $\rulename{app}$ and $\rulename{app^+}$:
\begin{itemize}
\item Case of the rule $\rulename{app}$:

    \[ (\mathbf{app})
    \rulef
        {\seq{\Gamma}{M : (\overline{B_1} \, | \, \cdots \, | \, \overline{B_m} \, | \, o)} \qquad
            \seq{\Gamma}{N_1 : B_{11}} \quad \cdots \quad \seq{\Gamma}{N_{l} :
            B_{1l}} \qquad l = |\overline{B_1}| }
        { \seq{\Gamma}{M N_1
            \cdots N_{l} : (\overline{B_2} \, | \, \cdots \, | \,
            \overline{B_m} \, | \, o)}}
    \]

    If the conclusion verifies $P_i$ then, for all $z \in \Gamma$:
    \begin{eqnarray*}
    \ord{z} \geq 1 + \ord{\overline{B_2}} + i
    &=& 1 + \ord{\overline{B_1}} + \ord{\overline{B_2}} - \ord{\overline{B_1}} + i \\
    &=& \ord{M} + (\ord{\overline{B_2}} - \ord{\overline{B_1}} + i)
    \end{eqnarray*}
    Therefore the first premise satisfies $P_j$ where $j={\ord{\overline{B_2}} - \ord{\overline{B_1}} + i}$.
    Hence by the induction hypothesis,
    $$\Gamma' \vdash^{0} M : (\overline{B_1} \, | \, \cdots \, | \, \overline{B_m} \, | \, o)$$
    where $\Gamma' = \Gamma \setminus \{ z \in \Gamma \ | \ \ord{M} + j \leq \ord{z} < \ord{M} \}$.


    Similarly, for all $z \in \Sigma$:
    \begin{eqnarray*}
    \ord{z} \geq 1 + \ord{\overline{B_2}} + i
    &=& \ord{\overline{B_1}} + (1+\ord{\overline{B_2}} - \ord{\overline{B_1}} + i) \\
    &=& \ord{\overline{B_1}} + j+1
    \end{eqnarray*}
    Hence by the induction hypothesis:
    $$\Gamma'' \vdash^0 N_k : B_{1k} \mbox{ for } k \in 1..l$$
    where $\Gamma'' = \Gamma \setminus \{ z \in \Gamma \ | \ \ord{M} + j+1 \leq \ord{z} < \ord{M} \}$.

    Furthermore, $\Gamma'' = \Gamma' \union \{ z \in \Gamma \ | \ \ord{M} + j = \ord{z}\}$ therefore
    the weakening rule gives:
    $$\Gamma'' \vdash^{-1} M : (\overline{B_1} \, | \, \cdots \, | \, \overline{B_m} \, | \, o)$$

    Finally the $\rulename{app}$ rule gives:
    $$\rulef{\Gamma'' \vdash^{-1} M : (\overline{B_1} \, | \, \cdots \, | \, \overline{B_m} \, | \, o)
    \quad \Gamma'' \vdash^0 N_1 : B_{11} \quad \ldots \quad \Gamma'' \vdash^0 N_1 : B_{1l}
    }
        { \Gamma'' \vdash M N_1 \ldots N_l : (\overline{B_2} \, | \, \cdots \, | \,
            \overline{B_m} \, | \, o)}
    $$
    such that for all $z\in \Gamma''$:
    \begin{eqnarray*}
    \ord{z} \geq \ord{\overline{B_1}}
    &\geq& 1 + \ord{\overline{B_2}} = \ord{M N_1 \ldots N_l}
    \end{eqnarray*}

    Therefore:
    $$\Gamma'' \vdash^0 M N_1 \ldots N_l : (\overline{B_2} \, | \, \cdots \, | \,
            \overline{B_m} \, | \, o)$$

\item $\rulename{app^+}$  The side-condition of the rule $\rulename{app^+}$ ensures that the first premise
 verify $P_0$. The conclusion of the rule has the same order as the first premise
 therefore the conclusion also verifies $P_0$.
\end{itemize}
\end{proof}


\begin{lem}
\label{lem:prooftree01only}
If $\Gamma \vdash^{0} M : T$ or $\Gamma \vdash^{-1} M : T$ then there is valid proof tree
showing $\Gamma \vdash M : T$ such that all the judgments
appearing in the proof tree verify either $P_0$ or $P_{-1}$.
\end{lem}


\begin{proof}
Since $P_{-1}$ implies $P_0$, we assume that the judgment $\Gamma \vdash M : T$ satisfies $P_{-1}$.

We show that there is a proof tree for
$\Gamma \vdash M : T$ where all the nodes of the tree verify $P_0$
or $P_{-1}$. We proceed by structural induction and case analysis on the last rule used
to show $\Gamma \vdash M : T$.
\begin{itemize}
\item Axiom $\rulename{\Sigma\mbox{\textbf{-const}}}$: the context is empty therefore the sequent verifies $P_{-1}$.

\item Axiom $\rulename{var}$: the context contains only the variable itself therefore the sequent verifies $P_0$.

\item Rule $\rulename{wk}$: If $\Delta \vdash M : T$ verifies $P_{-1}$ then in particular $\Gamma
\vdash M : T$ verifies $P_{-1}$ for any $\Gamma \subseteq \Delta$.

\item Rule $\rulename{perm}$: By the induction hypothesis.


\item Rule $\rulename{abs}$: the second premise of the rule guarantees that the first
premise verifies $P_{-1}$.

\item Rule $\rulename{app^+}$: The first premise has the same order has the
conclusion of the rule therefore the first premise verifies
$P_0$. The side-condition of the rule $\rulename{app^+}$ ensures that all the other premises verify $P_0$.

\item Rule $\rulename{app}$:

$$ \rulename{app}
    \rulef{
        { \Gamma \vdash M : (\overline{A} \, | B)
        \qquad
        \Gamma \vdash N_1 : A_1 \quad \cdots \quad \Gamma \vdash N_{l} : A_l \qquad l = |\overline{A}|
        }
    }
    {
        \Gamma \vdash^0 M N_1 \cdots N_{l} : B
    }
$$

Applying lemma \ref{lem:restriction} to the first premise we obtain:
\begin{equation}
 \Sigma \vdash^0 M : (\overline{A} \, | B) \label{eq:seq1}
\end{equation}
where $\Sigma = \{ z \in \Gamma \ | \ \ord{z} \geq \ord{(\overline{A} \, | B)} \} = \{ z \in \Gamma \ | \ \ord{z} \geq 1 + \ord{\overline{A}} \} $

Since $\ord{\overline{A}} = \ord{A_1} = \ldots = \ord{A_l}$ by applying lemma \ref{lem:restriction} to all the premises but the first we have
for all $i \in 1..p$ :
$$ \Sigma' \vdash^0 N_i : A_i $$
where
$\Sigma' = \{ z \in \Gamma \ | \ \ord{z} \geq \ord{\overline{A}} \} \supseteq \Sigma$

If the inclusion $\Sigma \subseteq \Sigma'$ is strict then we apply the weakening rule to sequent (\ref{eq:seq1}):
$$ \rulef{\Sigma \vdash^0 M : (\overline{A} \, | B)}{\Sigma' \vdash^{-1} M : (\overline{A} \, | B)} \rulename{wk} $$

We obtain the following proof tree:
$$  \rulef{
        \rulef{
            { \Sigma' \vdash^{-1} M : (\overline{A} \, | B)
            \qquad
            \Sigma' \vdash^0 N_1 : A_1 \quad \cdots \quad \Sigma' \vdash^0 N_{l} : A_l \qquad l = |\overline{A}|
            }
        }
        {
            \Sigma' \vdash^0 M N_1 \cdots N_{l} : B
        } \rulename{app}
    }
    {
         \Gamma \vdash^0 M N_1 \cdots N_{l} : B
    } \rulename{wk}
$$

where the last weakening rules is applied only if the inclusion $\Sigma' \subseteq \Gamma$ is strict.

We can now conclude using the induction hypothesis on the sequents $\Sigma' \vdash^{-1} M$,
$\Sigma' \vdash^0 N_1$, \ldots, $\Sigma' \vdash^0 N_l$ .
\end{itemize}
\end{proof}

\subsubsection{Refining the rules of the homogeneous safe $\lambda$-calculus}

Using the observations that we have just made, we will now derive the rules
of the safe $\lambda$-calculus with homogeneous type. We want a system
of rules generating sequents that verify $P_0$. Also, it must be able to generate intermediate
sequents that do not necessarily satisfy $P_0$ provided that they can be used
to produce \emph{in fine} terms satisfying $P_0$.

Because of the lemma \ref{lem:prooftree01only}, we know that the only necessary intermediate sequents
are those that either satisfy $P_0$ or $P_{-1}$.
Hence, we will assume by default that premises of the rules all satisfy $P_{-1}$. We will see that in
some cases, some premise will actually satisfy $P_0$.

First we define an additional rules expressing the fact that $P_0$ implies $P_{-1}$:
$$ \rulename{seq} \quad \rulef{\Gamma \vdash^{0} M : A}{\Gamma \vdash^{-1} M : A} $$

The weakening rule specializes into two rules:
$$ \rulename{wk^{0}} \quad  \rulef{\Gamma \vdash^{0} M : A}{\Gamma , x : B \vdash^{0} M : A} \quad \ord{B} \geq \ord{A} $$
$$ \rulename{wk^{-1}} \quad  \rulef{\Gamma \vdash^{-1} M : A}{\Gamma , x : B \vdash^{-1} M : A} \quad \ord{B} \geq \ord{A} -1$$

Because of the context reduction lemma, any sequent verifying $P_{-1}$ can be obtained
by applying the weakening rule $\rulename{wk^{-1}}$ or the rule $\rulename{seq}$ to another sequent
verifying $P_0$. Therefore, with the exception of these two rules, we only need to use rules
whose conclusion sequents verify $P_0$:
\begin{itemize}
\item For the rules $\rulename{perm}$, $\rulename{const}$ and $\rulename{var}$, only the tagging of the sequents
changes:
$$ \rulename{var} \quad  \rulef{}{x : A\vdash^{0} x : A}
\qquad
  \rulename{perm} \rulef{
      { \Gamma \vdash^0 M:B \qquad \sigma(\Gamma)  } \hbox{ homogeneous}
    }
      { \sigma(\Gamma) \vdash^0 M : B }
$$

$$ \rulename{const}
    \rulef{}{ \vdash^0 b : o^r \rightarrow o} \quad b : o^r \rightarrow o \in \Sigma
$$

\item $\rulename{abs}$ The definition of the abstraction rule has a side condition
expressing the fact that the premise verifies $P_0$ or $P_{-1}$. Since this is always true for sequents
generated by our new system of rules, we can drop the side condition:
$$ \rulename{abs} \quad  \rulef{\Gamma | \overline{x} : \overline{A} \vdash^{-1} M : B}
                                   {\Gamma  \vdash^{0} \lambda \overline{x} : \overline{A} . M : (\overline{A},B)}$$


\item $\rulename{app}$ The application rule has the following form:
$$ \rulename{app}
    \rulef{
        { \Gamma \vdash^{-1} M : (\overline{A} \, | B)
        \qquad
        \Gamma \vdash^{-1} N_1 : A_1 \quad \cdots \quad \Gamma \vdash^{-1} N_{l} : A_l \qquad l = |\overline{A}|
        }
    }
    {
        \Gamma \vdash^0 M N_1 \cdots N_{l} : B
    }
$$

Since the first premise verifies $P_{-1}$, by property \ref{proper:safe_basic_prop}(ii) we have:
$$\forall z \in \Gamma : \ord{z} \geq 1 + \ord{\overline{A}} -1 = \ord{\overline{A}} = \ord{\overline{N}}$$
Hence, all the sequents of the premises but the first must verify $P_0$. The rule (app) is therefore given by:
$$ \rulename{app}
    \rulef{
        { \Gamma \vdash^{-1} M : (\overline{A} \, | B)
        \qquad
        \Gamma \vdash^0 N_1 : A_1 \quad \cdots \quad \Gamma \vdash^0 N_{l} : A_l \qquad l = |\overline{A}|
        }
    }{
        \Gamma \vdash^0 M N_1 \cdots N_{l} : B
      }
$$

\item For the application rule $\rulename{app^+}$, the type of the sequent in the first premise has the same order
as the type of the conclusion premises, therefore since the conclusion verifies $P_0$, the first premise also verifies $P_0$.
The side-condition implies that that all the other sequents in the premise verify $P_0$. Moreover the fact
that the first premise verifies $P_0$ ensure that the side-condition always holds. Hence the rule becomes:
$$ \rulename{app^+}
    \rulef{
        \Gamma \vdash^0 M : (\overline{B_1} \, | \, \cdots \, | \, \overline{B_m} \, | \, o) \qquad
        \Gamma \vdash^0 N_1 : B_{11} \quad \cdots \quad \Gamma \vdash^0 N_{l} : B_{1l} \qquad l < |\overline{B_1}|
    }
    {
        \Gamma \vdash^0 M N_1 \cdots N_{l} : (\overline{B} \, | \, \cdots \, | \, \overline{B_m} \, | \, o)
    }
$$
where $\overline{B_1} = B_{11}, \ldots, B_{1l},\overline{B}$.
Clearly, this rule can be equivalently stated as:
$$ \rulef{\Gamma \vdash^0 M : A\rightarrow B
                                        \qquad \Gamma \vdash^{0} N : A
                                   }
                                   {\Gamma  \vdash^{0} M N : B}$$
\end{itemize}

The full set of rules is given in table \ref{tab:homosafelmd_rules_refined}.

\begin{table}[htbp]
$$  \rulename{perm}
    \rulef{
      { \Gamma \vdash^0 M:B \qquad \sigma(\Gamma)  } \hbox{ homogeneous}
    }
    { \sigma(\Gamma) \vdash^0 M : B
    }
\qquad
\rulename{seq} \quad \rulef{\Gamma \vdash^{0} M : A}{\Gamma \vdash^{-1} M : A}
$$

$$
 \rulename{const}
    \rulef{}
        { \vdash^0 b : o^r \rightarrow o} \quad b : o^r \rightarrow o \in \Sigma
\qquad
 \rulename{var} \quad  \rulef{}{x : A\vdash^{0} x : A} $$

$$ \rulename{wk^{0}} \quad  \rulef{\Gamma \vdash^{0} M : A}{\Gamma , x : B \vdash^{0} M : A} \quad \ord{B} \geq \ord{A} $$

$$ \rulename{wk^{-1}} \quad  \rulef{\Gamma \vdash^{-1} M : A}{\Gamma , x : B \vdash^{-1} M : A} \quad \ord{B} \geq \ord{A} -1$$


$$ \rulename{app}
    \rulef
        {   \Gamma \vdash^{-1} M : (\overline{A} \, | B)
            \qquad
            \Gamma \vdash^0 N_1 : A_1 \quad \cdots \quad \Gamma \vdash^0 N_{l} : A_l \qquad l = |\overline{A}|
        }
        {
            \Gamma \vdash^0 M N_1 \cdots N_{l} : B
        }
$$

$$ \rulename{app^+} \quad  \rulef{\Gamma \vdash^0 M : A\rightarrow B
                                        \qquad \Gamma \vdash^{0} N : A
                                   }
                                   {\Gamma  \vdash^{0} M N : B}$$

$$ \rulename{abs} \quad  \rulef{\Gamma| \overline{x} : \overline{A} \vdash^{-1} M : B}
                                   {\Gamma  \vdash^{0} \lambda \overline{x} : \overline{A} . M : (\overline{A}|B)}$$


where $\Gamma| \overline{x} : \overline{A}$ means that the lowest type-partition of the context is
$\overline{x} : \overline{A}$.
\caption{Alternative rules for the homogeneous safe lambda calculus}
\label{tab:homosafelmd_rules_refined}
\end{table}
%%%

    \clearpage

\section{Non homogeneous safe $\lambda$-calculus - VERSION B}

In section \ref{sec:safe_alt}, we have presented a safe lambda
calculus in the setting of homogeneous types. In this section, we
give a general notion of safety for the simply typed
$\lambda$-calculus. The rules we give here do not assume homogeneity
of the types.

We will call safe terms the simply typed lambda terms that are
typable within the following system of formation rules:

\subsection{Rules}

 We use a set of sequents of the form $\Gamma \vdash^{i} M :
A$ where the meaning is ``variables in $\Gamma$ have orders at least
$\ord{A}+i$'' where $i \leq 0$. The following set of rules are
defined for $i \in \nat$:

$$ \rulename{var} \quad  \rulef{}{x : A\vdash^{0} x : A} $$

$$ \rulename{seq} \quad \rulef{\Gamma \vdash^0 M : A}{\Gamma \vdash^{-1} M : A}
\qquad  \rulename{wk^i} \quad  \rulef{\Gamma \vdash^i M : A}{\Gamma , x : B \vdash^i M : A} \quad \ord{B} \geq \ord{A} + i $$

$$ \rulename{app} \quad  \rulef{\Gamma \vdash^{-1} M : (A,\ldots,A_l,B)
                                        \qquad \Gamma \vdash^{0} N_1 : A_1
                                        \quad \ldots \quad \Gamma \vdash^{0} N_l : A_l  }
                                   {\Gamma  \vdash^0 M N_1 \ldots N_l : B}
                                    \qquad
                                   \forall y \in \Gamma : \ord{y} \geq \ord{B}$$

$$ \rulename{abs^i} \quad  \rulef{\Gamma, \overline{x} : \overline{A} \vdash^{i} M : B}
                                   {\Gamma  \vdash^{0} \lambda \overline{x} : \overline{A} . M : (\overline{A},B)} \qquad
                                   \forall y \in \Gamma : \ord{y} \geq \ord{\overline{A},B}$$


Note that:
\begin{itemize}
\item $(\overline{A},B)$ denotes the type $(A_1,A_2, \ldots, A_n, B)$;
\item all the types appearing in the rule are not required to be homogeneous. For instance
for the type $(A,\ldots,A_l,B)$ in the rule $\rulename{app}$ it is not necessary that $\ord{A_l} \geq \ord{B}$;
\item the environment $\Gamma, \overline{x}:\overline{A}$ is not stratified. In particular, variables in $\overline{x}$ do not necessarily have the same order;
\item In the abstraction rule, the side-condition imposes that at least all the variable of the lowest order
in the context are abstracted. However other variables can also be
abstracted together with the lowest order variables. Moreover there
is not constraint on the order on which the variables are abstracted
(contrary to what happens in the homogeneous case);
\item The sequents $\Gamma \vdash^0 M$ are the \emph{safe terms} that we want to generate.
Other terms are only used as intermediate sequents in a proof tree.
\end{itemize}

\todobox{Problem: with this definition, safety is not preserved by $\eta$-expansion. For instance
if $M:(A_1,\ldots,A_l,o)$ where $(A_1,\ldots,A_l,o)$ is not homogeneous. The term
$M x_1 \ldots x_l$ where $x_i :A_i$ will not be a valid term for this system of rules. Therefore the
$\eta$-expansion will not be a valid term neither.}

\begin{exmp}
Suppose $x:o$, $f:o\rightarrow o$ and $\varphi:(o\rightarrow
o)\rightarrow o$ then the term $$\vdash^0 \lambda x f \varphi .
\varphi : o \rightarrow (o\rightarrow o) \rightarrow ((o\rightarrow
o)\rightarrow o) \rightarrow (o\rightarrow o)\rightarrow o$$ is
valid although its type is not homogeneous
\end{exmp}


\begin{lem}[Basic properties]
\label{lem:nonhomosafe_basic_prop} Suppose $\Gamma \vdash^i M : B$
is a valid judgment then every variable in $\Gamma$ has order at
least $ord(M)+i$.
\end{lem}
\begin{proof}
An easy induction. The step case for the application is: suppose
$\Gamma \vdash^{i+\delta} M N : B$ where $\Gamma \vdash^i M :
A\rightarrow B$. Then by induction we have $\forall y \in \Gamma :
\ord{y} \geq \ord{A\rightarrow B} + i = \max(1+\ord{A}, \ord{B})+i =
\delta + \ord{B} + i \geq \min(i+\delta,0) + \ord{B}$.
\end{proof}

\subsection{Substitution in the safe lambda calculus}

The traditional notion of substitution, on which the
$\lambda$-calculus is based on, is the following one:
\begin{dfn}[Substitution]
\label{dfn:subst}
\begin{eqnarray*}
c \subst{t}{x} &=& c \quad \mbox{where $c$ is a $\Sigma$-constant}\\
x \subst{t}{x} &=& t\\
 y\subst{t}{x} &=& y \quad \mbox{for } x \not \neq y,\\
(M_1 M_2) \subst{t}{x} &=& (M_1 \subst{t}{x}) (M_2 \subst{t}{x})\\
(\lambda x . M) \subst{t}{x} &=& \lambda x . M\\
(\lambda y . M) \subst{t}{x} &=& \lambda z . M \subst{z}{y}
\subst{t}{x} \mbox{where $z$ is a fresh variable and $x\not = y$}
\end{eqnarray*}
\end{dfn}

In the setting of the safe lambda calculus, the notion of
substitution can be simplified. Indeed, we remark that for safe
$\lambda$-terms there is no need to rename variables when performing
substitution:

\begin{lem}[No variable capture lemma]
\label{lem:noclash} There is no variable capture when performing
substitution on a safe term.
\end{lem}
\begin{proof}
Suppose that a capture occurs during the substitution $M[N/\varphi]$
where $M$ and $N$ are safe. Then the following conditions must hold:
\begin{enumerate}
\item $\varphi:A, \Gamma \vdash^0 M$,
\item $\Gamma' \vdash^0 N$,
\item there is a subterm $\lambda \overline{x} . L$ in $M$ (where the abstraction is taken as wide as possible) such that:
\item $\varphi \in fv(\lambda \overline{x} . L)$ (and therefore $\varphi \in fv(L)$),
\item $x \in fv(N)$ for some $x \in \overline{x}$.
\end{enumerate}

By lemma \ref{lem:nonhomosafe_basic_prop} and (v) we have:

\begin{equation}
\ord{x} \geq \ord{N} = \ord{\varphi} \label{eq:xigeqphi}
\end{equation}

The abstraction $\lambda \overline{x} . L$ (taken as large as
possible) is a subterm of $M$, therefore there is a node $\Sigma
\vdash^u \lambda \overline{x} . L$  for some $u$ in the proof tree
of $\varphi:A, \Gamma \vdash^0 M$.

There are only three kinds of rules that can produce an abstraction:
$\rulename{abs^i}$, $\rulename{seq}$ and $\rulename{wk^i}$. The only
one that can introduce the abstraction is $\rulename{abs^i}$.
Therefore the proof tree has the following form:

$$ \rulef{
    \rulef{
        \rulef{
            \rulef  {\ldots}
                   {\Sigma' \vdash^0 \lambda \overline{x} . L} \rulename{abs^i}
        }
        {\ldots} r_1
    }
    {\vdots} r_2
    }
    { \Sigma \vdash^u \lambda \overline{x} . L } r_l
    \qquad \mbox{where } r_j \in \{ \rulename{seq},\ \rulename{wk^i}\ |\ i \in \nat \},
            \quad j\in 1..l.
$$


Since $\varphi \in fv (L)$ we must have $\varphi \in \Sigma'$ and
since $\Sigma' \vdash^0 \lambda \overline{x} . L$, by lemma
\ref{lem:nonhomosafe_basic_prop} we have:

$$\ord{\varphi} \geq \ord{\lambda \overline{x} . L} \geq \max(1+ \ord{x}, \ord{L}) > \ord{x}$$

which contradicts equation (\ref{eq:xigeqphi}).
\end{proof}

Hence, in the safe lambda calculus setting, we can omit to to rename
variable when performing substitution. The equation
$$(\lambda x . M) \subst{t}{y} = \lambda z . M \subst{z}{x}
\subst{t}{y} \mbox{where $z$ is a fresh variable}$$ becomes
$$(\lambda x . M) \subst{t}{y} = \lambda x . M \subst{t}{y}$$



Unfortunately, this notion of substitution is still not adequate for
the purpose of the safe simply-typed lambda calculus. The problem is
that performing a single $\beta$-reduction on a safe term will not
necessarily produce another safe term.

To fix this problem, we need to be able to reduce several
consecutive $\beta$-redex at the same time until we obtain a safe
term. Consequently, we need a mean of performing several
substitutions at the same time. To achieve this, we introduce the
\emph{simultaneous substitution},
 a generalization of the standard substitution given in definition \ref{dfn:subst}.

\begin{dfn}[Simultaneous substitution]
\label{dnf:simsubst}
 We use the notation
$\subst{\overline{N}}{\overline{x}}$ for $\subst{N_1 \ldots N_n}{x_1
\ldots x_n}$:
\begin{eqnarray*}
c \subst{\overline{N}}{\overline{x}} &=& c \quad \mbox{where $c$ is a $\Sigma$-constant}\\
x_i \subst{\overline{N}}{\overline{x}} &=& N_i\\
 y \subst{\overline{N}}{\overline{x}} &=& y \quad \mbox{ if } y \not \neq x_i \mbox{ for all } i,\\
(M N) \subst{\overline{N}}{\overline{x}} &=& (M \subst{\overline{N}}{\overline{x}}) (N \subst{\overline{N}}{\overline{x}}) \\
(\lambda x_i . M) \subst{\overline{N}}{\overline{x}} &=& \lambda x_i
. M
\subst{N_1 \ldots N_{i-1} N_{i+1}\ldots N_n}{x_1 \ldots x_{i-1} x_{i+1}\ldots x_n} \\
(\lambda y . M)
\subst{\overline{N}}{\overline{x}} &=& \lambda z . M \subst{z}{y} \subst{\overline{N}}{\overline{x}} \\
&& \mbox{where $z$ is a fresh variables and } y \neq x_i \mbox{ for
all } i
\end{eqnarray*}
\end{dfn}

In general, variable captures should be avoided, this explains why
the definition of simultaneous substitution uses auxiliary fresh
variables. However in the current setting, lemma \ref{lem:noclash}
can clearly be transposed to the simultaneous substitution therefore
there is no need to rename variable.

The notion of substitution that we need is therefore the
\emph{capture permitting simultaneous substitution} defined as
follow:

\begin{dfn}[Capture permitting simultaneous substitution]
 We use the notation
$\subst{\overline{N}}{\overline{x}}$ for $\subst{N_1 \ldots N_n}{x_1
\ldots x_n}$:
\begin{eqnarray*}
c \subst{\overline{N}}{\overline{x}} &=& c \quad \mbox{where $c$ is a $\Sigma$-constant}\\
 x_i \subst{\overline{N}}{\overline{x}} &=& N_i\\
 y \subst{\overline{N}}{\overline{x}} &=& y \quad \mbox{where } x \not \neq y_i \mbox{ for all } i,\\
(M_1 M_2) \subst{\overline{N}}{\overline{x}} &=& (M_1 \subst{\overline{N}}{\overline{x}}) (M_2 \subst{\overline{N}}{\overline{x}})\\
(\lambda x_i . M) \subst{\overline{N}}{\overline{x}} &=& \lambda x_i
. M
\subst{N_1 \ldots N_{i-1} N_{i+1}\ldots N_n}{x_1 \ldots x_{i-1} x_{i+1}\ldots x_n} \\
(\lambda y . M) \subst{\overline{N}}{\overline{x}} &=& \lambda y . M
\subst{\overline{N}}{\overline{x}} \mbox{where $y \not = x_i$ for
all $i$} \qquad \mathbf{(\star)}
\end{eqnarray*}
The symbol $\mathbf{(\star)}$ identifies the equation that changed
compared to the previous definition.
\end{dfn}

\begin{lem}
\label{lem:subst_preserve_i}
$$ \Gamma,\overline{x} : \overline{A}\vdash^i M : T
\quad \mbox{and} \quad \Gamma \vdash^0 N_k : B_k \mbox{, } k \in
1..n \qquad \mbox{ implies } \qquad \Gamma \vdash^i
M[\overline{N}/\overline{x}] : T$$
\end{lem}

\begin{proof}
Suppose that $\Gamma,\overline{x}: \overline{A} \vdash^i M :T$ and
$\Gamma \vdash^0 N_k : B_k$ for $k \in 1..n$.

We prove $\Gamma \vdash^i M[\overline{N}/\overline{x}]$ by induction
on the size of the proof tree of $\Gamma,\overline{x}:\overline{A}
\vdash^i M : T$ and by case analysis on the last rule used. We just
give the detail for the abstraction case. Suppose that the property
is verified for terms whose proof tree is smaller than $M$. Suppose
$\Gamma,\overline{x}:\overline{A} \vdash^0 \lambda \overline{y} :
\overline{C}. P : (\overline{C}|D)$ where $\Gamma,
\overline{x}:\overline{A}, \overline{y}:\overline{C} \vdash^i P :
D$, then by the induction hypothesis $\Gamma,
\overline{y}:\overline{C} \vdash^i
P\subst{\overline{N}}{\overline{x}} : D$. Applying the rule
$\rulename{abs^i}$ gives $\Gamma \vdash^0 \lambda
\overline{y}:\overline{C} . P \subst{\overline{N}}{\overline{x}}$.
\end{proof}

\subsection{Safe-redex}
In the simply-typed lambda calculus a redex is a term of the form
$(\lambda x . M) N$. We generalize this definition to the safe
lambda calculus:
\begin{dfn}[Safe redex]
We call safe redex a term of the form $(\lambda \overline{x} . M)
N_1 \ldots N_l$ such that:
\begin{itemize}
\item $ \Gamma \vdash^0 (\lambda \overline{x} . M) N_1 \ldots N_l $
\item the variable $\overline{x}=x_1\ldots x_n$ are abstracted altogether by one occurrence of the rule $\rulename{abs}$ in the proof tree.
\item The terms $(\lambda \overline{x} . M)$, $N_1$, $N_l$ are applied together at once using the $\rulename{app}$ rule :
$$   \rulef{
            \Sigma \vdash^{-1} \lambda \overline{x} . M
            \quad
            \Sigma \vdash^0 N_1         \quad \ldots \quad \Sigma \vdash^0 N_l
    }
    {
       \Sigma \vdash^0 (\lambda \overline{x} . L) N_1 \ldots N_l
    } (\mathbf{app})
$$
Consequently each $N_i$ is safe.

\item $l\leq n$
\end{itemize}
\end{dfn}

Note that the condition $l\leq n$ in the definition is not too
restrictive because if $l>n$ then the application rule is too wide
and can in fact be replaced by an application of exactly $n$ terms
followed by another application for the remaining terms $N_{n+1},
\ldots, N_l$.


\todobox{Define the safe reduction:
Consider the safe-redex $(\lambda \overline{x} . M) N_1 \ldots N_l$, it reduces
to $\lambda x_l \ldots x_n . M \subst{N_1 \ldots N_l}{x_1 \ldots x_l}$. The relation $\beta_s$ is defined
on safe-redex: $(s\mapsto t) \in \beta_s$ iff $s \equiv (\lambda \overline{x} . M) N_1 \ldots N_l$ is a safe redex and
$t \equiv \lambda x_l \ldots x_n . M \subst{N_1 \ldots N_l}{x_1 \ldots x_l}$
}

\todobox{Show that $\betasred \subseteq \betared^*$.}

Using the previous lemma, we will now prove that safe-reduction
produces safe terms.

\begin{lem}
\label{lem:safereduction} A safe redex reduces to a safe term.
\end{lem}

\begin{proof}
We note $\overline{A}$ for $A_1, \ldots , A_n$, $\overline{x}'$ for
$x_1 \ldots x_l$ and $\overline{x}''$ for $x_{l+1} \ldots x_n$.

A safe-redex has a proof tree of the following form:
$$
   \rulef{
        \rulef{
            \rulef{
                \rulef{
                    \rulef
                        { \rulef
                            {\vdots}
                            {\Sigma',\overline{x}:\overline{A}\vdash^i L:C  }
                        }
                        {\Sigma' \vdash^0 \lambda \overline{x} . L : \overline{A}|C} \rulename{abs^i}
                }
                {\vdots} r_1
            }
            {\vdots} r_2
            }
            { \Sigma \vdash^{-1} \lambda \overline{x} . L : A_1, \ldots , A_l|B} r_q
            \quad
            \Sigma \vdash^0 N_1 : A_1
            \quad \ldots \quad \Sigma \vdash^0 N_l : A_l
    }
    {
       \Sigma \vdash^0 (\lambda \overline{x} . L) N_1 \ldots N_l : B
    } (\mathbf{app})
$$
with the following conditions:
\begin{enumerate}
\item for $j\in 1..q$, $r_j \in \{ \rulename{seq}, \rulename{wk^0}, \rulename{wk^{-1}} \}$ therefore
$\Sigma = \Sigma' \union \Delta$ where $\Delta$ contains the
variables introduced by the rules $r_1 \ldots r_q$.

\item $A_1, \ldots , A_l|B = A_1, \ldots , A_n|C$ and $l\leq n$. Therefore
$\ord{B} \geq \ord{C}$.
\item The side condition of the rule $\rulename{abs}$ gives: $\forall z \in \Sigma : \ord{z} \geq \ord{B}$
\end{enumerate}


The conditions 2 and 3 ensure that $\forall z \in \Delta : \ord{z}
\geq \ord{C}$ therefore we can use the weakening rule to introduce
all the variable of $\Delta$ in the context of the sequent
$\Sigma',\overline{x}:\overline{A}\vdash^i L:C$:

$$\rulef{\rulef{ \Sigma',\overline{x}:\overline{A}\vdash^i L:C  }
        {\vdots} (wk^i_0)}
        {\Sigma,\overline{x}:\overline{A}\vdash^i L:C} (wk^i_0)
$$

By lemma \ref{lem:subst_preserve_i} we obtain:
$$ \Sigma, \overline{x}'':\overline{A}'' \vdash^i L\subst{N_1 \ldots N_l}{\overline{x}'}$$
Finally using the abstraction rule:
$$ \Sigma \vdash^0 \lambda \overline{x}'':\overline{A}'' . L\subst{N_1 \ldots N_l}{\overline{x}'}$$
\end{proof}



\subsection{Examples}
\subsubsection{Example 1}
Let $f,g:o\rightarrow o$, $x,y:o\rightarrow
o$, $\Gamma = g:o\rightarrow o$ and $\Gamma' = g:o\rightarrow o,
y:o$. The term $(\lambda f x . x) g y $ is safe:


$$ \rulef{
        \rulef{
            \rulef{
                \rulef{\vdots}{\Gamma \vdash^{-1} \lambda f x. x}      \qquad \axiomf{\Gamma \vdash^0 g} }
            {\Gamma \vdash^0 (\lambda f x. x) g} \rulename{app}
        }
        { \Gamma' \vdash^{-1} (\lambda f x. x) g } \rulename{wk^{-1}}
        \qquad \axiomf{\Gamma' \vdash^0 y}
    }
    { \Gamma' \vdash^0 (\lambda f x. x) g y } \rulename{app}
$$


And the two occurrences of the application rule cannot be merged as
follow:
$$ \rulef{{\Gamma' \not\vdash^{-1} \lambda f x. x} \qquad \Gamma' \not\vdash^0 g \qquad \Gamma' \vdash^0 y}
    {\Gamma' \vdash^0 (\lambda f x. x) g y } \rulename{app}$$


\subsection{Particular case of homogeneously-safe lambda terms}

We look at a particular sub-class of lambda terms. The types of
these terms respect a property call homogeneity as defined in
section \ref{sec:homotypes}. A type $(A_1, A_2, \ldots A_n, o)$ is
said to be homogeneous whenever $\order{A_1} \geq \order{A_2} \geq
\ldots \geq  \order{A_n}$ and each of the $A_i$ are homogeneous. A
term is homogeneous if its type is homogeneous.


In their definition of safety (\cite{KNU02}), Knapik et al. require
that all the recursion equations of a safe recursion scheme have a
homogeneous type.

In the rules defining safety for the non-homogeneous case, the only
rule that can potentially introduce a non-homogeneous term from a
homogeneous one is the abstraction rule. But abstractions correspond
exactly to recursion equations in the
recursion scheme setting of Knapik et al. Therefore requiring that
recursions equation have homogeneous type is the same as requiring
that all sequents appearing in the proof tree of a safe term are of
homogeneous type.

We say that a term is homogeneously-safe if its type is homogeneous
and there is a proof tree showing its safety where all the sequents
of the proof tree are of homogenous type.

\begin{lem}[Context reduction]
\label{lem:context_reduction} If $\Gamma \vdash^i M : A$ then there
is a context $\Gamma' \subseteq \Gamma$ such that $\Gamma' \vdash^0
M : A$.
\end{lem}
\begin{proof}
An easy induction.
\end{proof}


\begin{lem}
\label{lem:homog_judg_zero_minusone} If a term is homogeneously-safe
then there is valid proof tree showing that it is safe containing
only judgments of the form $\Gamma \vdash^{k} M : T$ with $k\in
\{-1,0\}$.
\end{lem}

\begin{proof}
Assume that $\Gamma \vdash^{0} S : T_S$ with $T_S$ homogeneous. We
prove the result by induction on the size of the proof tree and by
case analysis on the last rule used to obtain $\Gamma \vdash^{0} S :
T_S$.

We give the details of the proof for the application and abstraction
case:
\begin{itemize}
\item Rule $\rulename{abs^i}$ for some $i$:
$$ \rulename{abs^i} \quad  \rulef{\Gamma, \overline{x} : \overline{A} \vdash^i M : B}
                                   {\Gamma  \vdash^{0} \lambda \overline{x} : \overline{A} . M : (\overline{A},B)} \qquad
                                   \forall y \in \Gamma : \ord{y} \geq \ord{\overline{A},B}$$

Type homogeneity requires that $\ord{\overline{x}} = \ord{A} \geq
\ord{B}$. Therefore the premise of the rule can be replaced by
$\Gamma, \overline{x} : \overline{A} \vdash^0 M : B$.

The induction hypothesis permits to conclude.

\item Rule $\rulename{app}$:
$$ \rulename{app} \quad  \rulef{\Gamma \vdash^{-1} M : (A,\ldots,A_l,B)
                                        \qquad \Gamma \vdash^{0} N_1 : A_1
                                        \quad \ldots \quad \Gamma \vdash^{0} N_l : A_l  }
                                   {\Gamma  \vdash^0 M N_1 \ldots N_l : B}
                                    \qquad
                                   \forall y \in \Gamma : \ord{y} \geq \ord{B}$$

For the first premise of the rule we apply lemma
\ref{lem:context_reduction}: there is a context $\Gamma'$ such that
$\Gamma' \subseteq \Gamma$ and:
$$ \Gamma' \vdash^{0} M : (A,\ldots,A_l,B) $$
Now by the induction hypothesis we know that there is a proof tree
showing $\Gamma' \vdash^{0} M : (A,\ldots,A_l,B)$ with judgement of
the form $\Sigma \vdash^{k} P : T$ with $k\in \{-1,0\}$.

By applying the weakening rule, we lift this result to $\Gamma
\vdash^{0} M : (A,\ldots,A_l,B)$.

For all the other premises we can directly apply the induction
hypothesis.

We conclude using the rule $\rulename{app}$.
\end{itemize}
\end{proof}

This lemma permits us to derive rules specialized for the
homogeneously-safe lambda calculus.

\subsubsection{The application rule} Let us derive the
application rules specialized for the case of homogeneous types. We
recall the rule $\rulename{app}$:
$$ \rulename{app} \  \rulef{\Gamma \vdash^{-1} M : (A,\ldots,A_l,B)
                                        \qquad \Gamma \vdash^{0} N_1 : A_1
                                        \quad \ldots \quad \Gamma \vdash^{0} N_l : A_l  }
                                   {\Gamma  \vdash^0 M N_1 \ldots N_l : B}
                                    \quad
                                   \forall y \in \Gamma : \ord{y} \geq \ord{B}$$

Type homogeneity implies that $\ord{A_1} \geq \ldots \geq \ord{A_l}
\geq \ord{B} - 1$.

We can make the assumption that $\ord{A_1} = \ldots = \ord{A_l}$ (if it is not the case,
we can replace the application rule by several consecutive application rules
respecting this condition).


\begin{itemize}
\item Suppose that $A_1, \ldots A_l$ forms a type partition, then we
have $\ord{A_l} \geq \ord{B}$, the side-condition disappears and
the rule becomes:
$$ \rulename{app_1} \quad  \rulef{\Gamma \vdash^{-1} M : \overline{A} | B
                                        \qquad \Gamma \vdash^{0} N_1 :
                                        A_1
                                        \quad \ldots \quad \Gamma \vdash^{0} N_l :
                                        A_l
                                        }
                                   {\Gamma  \vdash^{0} M N_1 \ldots N_l : B}
$$

where $\overline{A} = A_1, \ldots A_l$

\item  Suppose that $A_1, \ldots A_l$ do not form a type partition, then we
have $\ord{A_l} = \ord{B} - 1$. The side-condition becomes
$\forall y \in \Gamma : \ord{y} \geq 1 +\ord{A_l} = \ord{\overline{A}|B}$. Therefore the side-condition
can be omitted provided that we replace the $-1$ exponent in the first premise by a $0$:
$$ \rulename{app_2} \quad  \rulef{\Gamma \vdash^0 M : (A,\ldots,A_l,B)
                                        \qquad \Gamma \vdash^{0} N_1 : A_1
                                        \quad \ldots
                                        \quad \Gamma \vdash^{0} N_l : A_l
                                   }
                                   {\Gamma  \vdash^0 M N_1 \ldots N_l : B}
$$

Equivalently we can replace this rule by the following one:
$$ \rulename{app_2'} \quad  \rulef{\Gamma \vdash^0 M : A\rightarrow B
                                        \qquad \Gamma \vdash^{0} N : A }
                                {\Gamma  \vdash^{0} M N : B}$$
\end{itemize}

\subsubsection{The abstraction rule}

Let us derive the abstraction rule specialized for the case of
homogeneous types. We recall the rule $\rulename{abs}$:
$$ \rulename{abs^i} \quad  \rulef{\Gamma, \overline{x} : \overline{A} \vdash^{i} M : B}
                                   {\Gamma  \vdash^{0} \lambda \overline{x} : \overline{A} . M : (\overline{A},B)} \qquad
                                   \forall y \in \Gamma : \ord{y} \geq \ord{\overline{A},B}$$

The context $\Gamma$ is now partitionned according to the order of
the variables. The partitions are written in decreasing order of
type order. The notation $\Gamma | \overline{x}:\overline{A}$ means
that $\overline{x}:\overline{A}$ is the lowest partition of the
context.

We also use the notation $(\overline{A}|B)$ to denote the
homogeneous type $(A_1, A_2, \ldots A_n, B)$ where $\ord{A_1} =
\ord{A_2} =  \ldots \ord{A_n} \geq \ord{B} -1$.


Suppose that we abstract the single variable $\overline{x} = x$,
then in order to respect the side condition, we need to abstract all
variables of order lower or equal to $\ord{x}$. In particular we
need to abstract the partition of the order of $x$.

Moreover to respect type homogeneity, we need to abstract variables
of the lowest order first.

Hence we can change the abstraction rule so that it only allows
abstraction of the lowest variable partition. The rule can then be
used repeatedely if further partitions need to be abstracted. We
obtained the following rule where the side-condition has
disappeared:

$$ \rulename{abs^i} \quad  \rulef{\Gamma| \overline{x} : \overline{A} \vdash^{-1} M : B}
                                   {\Gamma  \vdash^{0} \lambda \overline{x} : \overline{A} . M : (\overline{A}|B)}$$


\subsubsection{The rules of the homogeneous safe $\lambda$-calculus}

Table \ref{tab:homosafelmd_rules} recapitulates the entire set of rules:

\begin{table}[htbp]
$$  \rulename{perm} {
      { \Gamma \vdash^0 M:B \qquad \sigma(\Gamma)  } \hbox{ homogeneous}
    \over
      { \sigma(\Gamma) \vdash^0 M : B }
    }
\qquad \rulename{seq} \quad \rulef{\Gamma \vdash^{0} M : A}{\Gamma
\vdash^{-1} M : A}
$$

$$
 \rulename{const}
    { \over { \vdash^0 b : o^r \rightarrow o}} \quad b : o^r \rightarrow o \in \Sigma
\qquad
 \rulename{var} \quad  \rulef{}{x : A\vdash^{0} x : A} $$

$$ \rulename{wk^{0}} \quad  \rulef{\Gamma \vdash^{0} M : A}{\Gamma , x : B \vdash^{0} M : A} \quad \ord{B} \geq \ord{A} $$

$$ \rulename{wk^{-1}} \quad  \rulef{\Gamma \vdash^{-1} M : A}{\Gamma , x : B \vdash^{-1} M : A} \quad \ord{B} \geq \ord{A} -1$$

$$ \rulename{app} \quad  \rulef{\Gamma \vdash^{-1} M : \overline{A} | B
                                        \qquad \Gamma \vdash^{0} N_1 : A_1
                                        \quad \ldots \quad \Gamma \vdash^{0} N_l : A_l
                                        \qquad l = |\overline{A}|
                                        }
                                   {\Gamma  \vdash^{0} M N_1 \ldots N_l : B}
$$


$$ \rulename{app^0} \quad  \rulef{\Gamma \vdash^0 M : A\rightarrow B
                                        \qquad \Gamma \vdash^{0} N : A
                                   }
                                   {\Gamma  \vdash^{0} M N : B}$$

$$ \rulename{abs} \quad  \rulef{\Gamma| \overline{x} : \overline{A} \vdash^{-1} M : B}
                                   {\Gamma  \vdash^{0} \lambda \overline{x} : \overline{A} . M : (\overline{A}|B)}$$
\caption{Rules of the homogeneous safe lambda calculus}
\label{tab:homosafelmd_rules}
\end{table}


We observe that these rules correspond exactly to the rules given in the previous section
in table \ref{tab:homosafelmd_rules_refined}.

    % \clearpage
\section{Non-homogeneous safe $\lambda$-calculus - VERSION A}

In section \ref{sec:safe_alt}, we have presented a safe lambda calculus in the setting of homogeneous types.
In this section, we try to give a general notion of safety for the simply typed $\lambda$-calculus.
The rules we give here do not assume homogeneity of the types.

We will call safe terms the simply typed lambda terms that are typable within the following system of formation rules:

\subsection{Rules}

 We use a set of sequents of the form $\Gamma \vdash^{i} M :
A$ where the meaning is ``variables in $\Gamma$ have orders at least
$\ord{A}+i$'' where $i \leq 0$. The following set of rules are
defined for $i \leq 0$:

$$ \mathbf{(seq^i_\delta)} \quad \rulef{\Gamma \vdash^{i} M : A}{\Gamma \vdash^{i-\delta} M : A} \quad i \leq 0, \delta > 0  $$

$$ \mathbf{(var)} \quad  \rulef{}{x : A\vdash^{0} x : A} $$

$$ \mathbf{(wk^i)} \quad  \rulef{\Gamma \vdash^i M : A}{\Gamma , x : B \vdash^i M : A} \quad \ord{B} \geq \ord{A} + i $$

$$ \mathbf{(app^i)} \quad  \rulef{\Gamma \vdash^i M : A\rightarrow B
                                        \qquad \Gamma \vdash^{0} N : A}
                                   {\Gamma  \vdash^{\min(i+\delta,0)} M N : B}
                                    \qquad
                                   \delta = \max\left(0, 1 + \ord{A} - \ord{B}\right)$$

$$ \mathbf{(abs^i)} \quad  \rulef{\Gamma, \overline{x} : \overline{A} \vdash^{i} M : B}
                                   {\Gamma  \vdash^{0} \lambda \overline{x} : \overline{A} . M : (\overline{A},B)} \qquad
%                                   \left\{
%                                     \begin{array}{ll}
                                       \forall y \in \Gamma : \ord{y} \geq \ord{\overline{A},B}\\
%                                       \forall y \in fv(M) : \ord{y} \geq \ord{B}
%                                     \end{array}
%                                   \right.
                                   $$


Note that:
\begin{itemize}
\item $(\overline{A},B)$ denotes the type $(A_1,A_2, \ldots, A_n, B)$;
\item all the types appearing in the rule are not required to be homogeneous. For instance in the rule $\mathbf{(app^i)}$ for the type $A \rightarrow B$
it is not necessary that $\ord{A} \geq \ord{B}$;
\item the environment $\Gamma, \overline{x}$ is not stratified. In particular, variables in $\overline{x}$ do not necessarily have the same
order;
\item In the abstraction rule, the side-condition imposes that at least all the variable of the lowest order
in the context are abstracted. However other variables can also be
abstracted together with the lowest order variables. Moreover there
is not constraint on the order on which the variables are abstracted
(contrary to what happens in the homogeneous case);
\item The sequents $\Gamma \vdash^0 M$ are the \emph{safe terms} that we want to generate.
Other terms are only used as intermediate sequents in a proof tree.
\end{itemize}

\begin{exmp}
Suppose $x:o$, $f:o\rightarrow o$ and $\varphi:(o\rightarrow
o)\rightarrow o$ then the term $$\vdash^0 \lambda x f \varphi .
\varphi : o \rightarrow (o\rightarrow o) \rightarrow ((o\rightarrow
o)\rightarrow o) \rightarrow (o\rightarrow o)\rightarrow o$$ is
valid although its type is not homogeneous
\end{exmp}


\begin{lem}[Basic properties]
\label{va_lem:nonhomosafe_basic_prop} Suppose $\Gamma \vdash^i M : B$
is a valid judgment then every variable in $\Gamma$ has order at
least $ord(M)+i$.
\end{lem}
\begin{proof}
An easy induction. The step case for the application is: suppose
$\Gamma \vdash^{i+\delta} M N : B$ where $\Gamma \vdash^i M :
A\rightarrow B$. Then by induction we have $\forall y \in \Gamma :
\ord{y} \geq \ord{A\rightarrow B} + i = \max(1+\ord{A}, \ord{B})+i =
\delta + \ord{B} + i \geq \min(i+\delta,0) + \ord{B}$.
\end{proof}

\subsection{Substitution in the safe lambda calculus}

The traditional notion of substitution, on which the $\lambda$-calculus is based on, is the following one:
\begin{dfn}[Substitution]
\label{va_dfn:subst}
\begin{eqnarray*}
c \subst{t}{x} &=& c \quad \mbox{where $c$ is a $\Sigma$-constant}\\
x \subst{t}{x} &=& t\\
 y\subst{t}{x} &=& y \quad \mbox{for } x \not \neq y,\\
(M_1 M_2) \subst{t}{x} &=& (M_1 \subst{t}{x}) (M_2 \subst{t}{x})\\
(\lambda x . M) \subst{t}{x} &=& \lambda x . M\\
(\lambda y . M) \subst{t}{x} &=& \lambda z . M \subst{z}{y}
\subst{t}{x} \mbox{where $z$ is a fresh variable and $x\not = y$}
\end{eqnarray*}
\end{dfn}

In the setting of the safe lambda calculus, the notion of substitution can be simplified.
Indeed, we remark that for safe $\lambda$-terms there is no need to rename variables
when performing substitution:

\begin{lem}[No variable capture lemma]
\label{va_lem:noclash}
There is no variable capture when performing substitution on a safe term.
\end{lem}
\begin{proof}
Suppose that a capture occurs during the substitution $M[N/\varphi]$
where $M$ and $N$ are safe. Then the following conditions must hold:
\begin{enumerate}
\item $\varphi:A, \Gamma \vdash^0 M$,
\item $\Gamma' \vdash^0 N$,
\item there is a subterm $\lambda \overline{x} . L$ in $M$ (where the abstraction is taken as wide as possible) such that:
\item $\varphi \in fv(\lambda \overline{x} . L)$ (and therefore $\varphi \in fv(L)$),
\item $x \in fv(N)$ for some $x \in \overline{x}$.
\end{enumerate}

By lemma \ref{va_lem:nonhomosafe_basic_prop} and (v) we have:

\begin{equation}
\ord{x} \geq \ord{N} = \ord{\varphi} \label{va_eq:xigeqphi}
\end{equation}

The abstraction $\lambda \overline{x} . L$ (taken as large as possible)
is a subterm of $M$, therefore there is a node $\Sigma \vdash^u \lambda \overline{x} . L$  for some $u$ in the
proof tree of $\varphi:A, \Gamma \vdash^0 M$.

There are only three kinds of rules that can produce an abstraction:
$\mathbf{(abs^i)}$, $\mathbf{(seq^i_\delta)}$ and $\mathbf{(wk^i)}$.
The only one that can introduce the abstraction is
$\mathbf{(abs^i)}$. Therefore the proof tree has the following form:
$$ \rulef{
    \rulef{
        \rulef{
            \rulef  {\ldots}
                   {\Sigma' \vdash^0 \lambda \overline{x} . L} \mathbf{(abs^i)}
        }
        {\ldots} r_1
    }
    {\vdots} r_2
    }
    { \Sigma \vdash^u \lambda \overline{x} . L } r_l
    \qquad \mbox{where for } j\in 1..l: r_j \in \{ \mathbf{(seq^i_\delta)},\ \mathbf{(wk^i)}\ |\ i \in \zset, \delta > 0 \}.
$$


Since $\varphi \in fv (L)$ we must have $\varphi \in \Sigma'$ and
since $\Sigma' \vdash^0 \lambda \overline{x} . L$, by lemma
\ref{va_lem:nonhomosafe_basic_prop} we have:
$$\ord{\varphi} \geq \ord{\lambda \overline{x} . L} \geq \max(1+ \ord{x}, \ord{L}) > \ord{x}$$

which contradicts equation (\ref{va_eq:xigeqphi}).
\end{proof}

Hence, in the safe lambda calculus setting, we can omit to
to rename variable when performing substitution. The equation
$$(\lambda x . M) \subst{t}{y} = \lambda z . M \subst{z}{x}
\subst{t}{y} \mbox{where $z$ is a fresh variable}$$
becomes
$$(\lambda x . M) \subst{t}{y} = \lambda x . M \subst{t}{y}$$



Unfortunately, this notion of substitution is still not adequate for
the purpose of the safe simply-typed lambda calculus. The problem is
that performing a single $\beta$-reduction on a safe term will not
necessarily produce another safe term.

To fix this problem, we need to be able to reduce several
consecutive $\beta$-redex at the same time until we obtain a safe
term. Consequently, we need a mean of performing several
substitutions at the same time. To achieve this, we introduce the
\emph{simultaneous substitution},
 a generalization of the standard substitution given in definition \ref{va_dfn:subst}.

\begin{dfn}[Simultaneous substitution]
 We use the notation
$\subst{\overline{N}}{\overline{x}}$ for $\subst{N_1 \ldots N_n}{x_1
\ldots x_n}$:
\begin{eqnarray*}
c \subst{\overline{N}}{\overline{x}} &=& c \quad \mbox{where $c$ is a $\Sigma$-constant}\\
x_i \subst{\overline{N}}{\overline{x}} &=& N_i\\
 y \subst{\overline{N}}{\overline{x}} &=& y \quad \mbox{ if } y \not \neq x_i \mbox{ for all } i,\\
(M N) \subst{\overline{N}}{\overline{x}} &=& (M \subst{\overline{N}}{\overline{x}}) (N \subst{\overline{N}}{\overline{x}}) \\
(\lambda x_i . M) \subst{\overline{N}}{\overline{x}} &=& \lambda x_i . M
\subst{N_1 \ldots N_{i-1} N_{i+1}\ldots N_n}{x_1 \ldots x_{i-1} x_{i+1}\ldots x_n} \\
(\lambda y . M)
\subst{\overline{N}}{\overline{x}} &=& \lambda z . M \subst{z}{y} \subst{\overline{N}}{\overline{x}} \\
&& \mbox{where $z$ is a fresh variables and } y \neq x_i \mbox{ for all } i
\end{eqnarray*}
\end{dfn}

In general, variable captures should be avoided, this explains why the definition
of simultaneous substitution uses auxiliary fresh variables.
However in the current setting, lemma \ref{va_lem:noclash} can clearly be transposed to
the simultaneous substitution therefore there is no need to rename variable.

The notion of substitution that we need is therefore
the \emph{capture permitting simultaneous substitution} defined as follows:

\begin{dfn}[Capture permitting simultaneous substitution]
 We use the notation
$\subst{\overline{N}}{\overline{x}}$ for $\subst{N_1 \ldots N_n}{x_1
\ldots x_n}$:
\begin{eqnarray*}
c \subst{\overline{N}}{\overline{x}} &=& c \quad \mbox{where $c$ is a $\Sigma$-constant}\\
 x_i \subst{\overline{N}}{\overline{x}} &=& N_i\\
 y \subst{\overline{N}}{\overline{x}} &=& y \quad \mbox{where } x \not \neq y_i \mbox{ for all } i,\\
(M_1 M_2) \subst{\overline{N}}{\overline{x}} &=& (M_1 \subst{\overline{N}}{\overline{x}}) (M_2 \subst{\overline{N}}{\overline{x}})\\
(\lambda x_i . M) \subst{\overline{N}}{\overline{x}} &=& \lambda x_i . M
\subst{N_1 \ldots N_{i-1} N_{i+1}\ldots N_n}{x_1 \ldots x_{i-1} x_{i+1}\ldots x_n} \\
(\lambda y . M) \subst{\overline{N}}{\overline{x}} &=& \lambda y . M \subst{\overline{N}}{\overline{x}} \mbox{where $y \not = x_i$ for all $i$}
\qquad \mathbf{(\star)}
\end{eqnarray*}
The symbol $\mathbf{(\star)}$ identifies the equation that changed compared to the previous definition.
\end{dfn}

\begin{lem}
\label{va_lem:subst_preserve_i}
$$ \Gamma,\overline{x} : \overline{A}\vdash^i M : T
\quad \mbox{and} \quad \Gamma \vdash^0 N_k : B_k \mbox{, } k \in
1..n \qquad \mbox{ implies } \qquad \Gamma \vdash^i
M[\overline{N}/\overline{x}] : T$$
\end{lem}

\begin{proof}
Suppose that $\Gamma,\overline{x}: \overline{A} \vdash^i M :T$ and
$\Gamma \vdash^0 N_k : B_k$ for $k \in 1..n$.

We prove $\Gamma \vdash^i M[\overline{N}/\overline{x}]$ by induction
on the size of the proof tree of $\Gamma,\overline{x}:\overline{A}
\vdash^i M : T$ and by case analysis on the last rule used. We just
give the detail for the abstraction case. Suppose that the property
is verified for terms whose proof tree is smaller than $M$. Suppose
$\Gamma,\overline{x}:\overline{A} \vdash^0 \lambda \overline{y} :
\overline{C}. P : (\overline{C}|D)$ where $\Gamma,
\overline{x}:\overline{A}, \overline{y}:\overline{C} \vdash^i P :
D$, then by the induction hypothesis $\Gamma,
\overline{y}:\overline{C} \vdash^i
P\subst{\overline{N}}{\overline{x}} : D$. Applying the rule
$\rulename{abs^i}$ gives $\Gamma \vdash^0 \lambda
\overline{y}:\overline{C} . P \subst{\overline{N}}{\overline{x}}$.
\end{proof}


\begin{cor}[Simultaneous substitution preserves safety]
If $M$ is safe and $N_k$ is safe for $k \in 1..n$ then  $M[\overline{N}/\overline{x}]$ is safe
\end{cor}

\subsection{Safe-redex}

In the simply-typed lambda calculus a redex is a term of the form
$(\lambda x . M) N$. We generalize this definition to the safe
lambda calculus:

\begin{dfn}[Safe redex]
We call safe redex a term of the form $(\lambda \overline{x} . M)
N_1 \ldots N_l$ such that:
\begin{itemize}
\item $ \Gamma \vdash^0 (\lambda \overline{x} . M) N_1 \ldots N_l $
\item the variable $\overline{x}=x_1\ldots x_n$ are abstracted altogether by one occurrence of the rule $\rulename{abs}$ in the proof tree.
%\item The terms $(\lambda \overline{x} . M)$, $N_1$, $N_l$ are applied together at once using the $\rulename{app}$ rule :
%$$   \rulef{
%            \Sigma \vdash^{-1} \lambda \overline{x} . M
%            \quad
%            \Sigma \vdash^0 N_1         \quad \ldots \quad \Sigma \vdash^0 N_l
%    }
%    {
%       \Sigma \vdash^0 (\lambda \overline{x} . L) N_1 \ldots N_l
%    } (\mathbf{app})
%$$
%Consequently each $N_i$ is safe.

\item $l\leq n$
\end{itemize}

\end{dfn}

Consequently, not all multi-$\beta$-redex of the form $(\lambda
\overline{x} . M) N_1 \ldots N_l$ is a safe-redex. It is important
to require that the abstraction $\lambda \overline{x}$ can be done
at once, otherwise we would not be able to prove that reducing a
safe-redex produces a safe term.


\todobox{Define the safe reduction: Consider the safe-redex
$(\lambda \overline{x} . M) N_1 \ldots N_l$, it reduces to $\lambda
x_l \ldots x_n . M \subst{N_1 \ldots N_l}{x_1 \ldots x_l}$. The
relation $\beta_s$ is defined on safe-redex: $(s\mapsto t) \in
\beta_s$ iff $s \equiv (\lambda \overline{x} . M) N_1 \ldots N_l$ is
a safe redex and $t \equiv \lambda x_l \ldots x_n . M \subst{N_1
\ldots N_l}{x_1 \ldots x_l}$ }

\todobox{Show that $\betasred \subseteq \betared^*$.}


Using the previous lemma, we will now prove that reducing a
safe-redex produces a safe term:

\begin{lem}
\label{va_lem:safereduction} A safe redex $(\lambda \overline{x} . M )
\overline{N}$ where the $N_i$ are safe reduces to the safe term
$M\subst{\overline{N}}{\overline{x}}$.
\end{lem}

\begin{proof}
%% NEED TO ADAPT THE FOLLOWING TO VERSION A OF THE RULES:
%%We note $\overline{A}$ for $A_1, \ldots , A_n$, $\overline{x}'$ for
%%$x_1 \ldots x_l$ and $\overline{x}''$ for $x_{l+1} \ldots x_n$.
%%
%%A safe-redex has a proof tree of the following form:
%%$$
%%   \rulef{
%%        \rulef{
%%            \rulef{
%%                \rulef{
%%                    \rulef
%%                        { \rulef
%%                            {\vdots}
%%                            {\Sigma',\overline{x}:\overline{A}\vdash^i L:C  }
%%                        }
%%                        {\Sigma' \vdash^0 \lambda \overline{x} . L : \overline{A}|C} \rulename{abs^i}
%%                }
%%                {\vdots} r_1
%%            }
%%            {\vdots} r_2
%%            }
%%            { \Sigma \vdash^{-1} \lambda \overline{x} . L : A_1, \ldots , A_l|B} r_q
%%            \quad
%%            \Sigma \vdash^0 N_1 : A_1
%%            \quad \ldots \quad \Sigma \vdash^0 N_l : A_l
%%    }
%%    {
%%       \Sigma \vdash^0 (\lambda \overline{x} . L) N_1 \ldots N_l : B
%%    } (\mathbf{app})
%%$$
%%with the following conditions:
%%\begin{enumerate}
%%\item for $j\in 1..q$, $r_j \in \{ \rulename{seq}, \rulename{wk^0}, \rulename{wk^{-1}} \}$ therefore
%%$\Sigma = \Sigma' \union \Delta$ where $\Delta$ contains the
%%variables introduced by the rules $r_1 \ldots r_q$.
%%
%%\item $A_1, \ldots , A_l|B = A_1, \ldots , A_n|C$ and $l\leq n$. Therefore
%%$\ord{B} \geq \ord{C}$.
%%\item The side condition of the rule $\rulename{abs}$ gives: $\forall z \in \Sigma : \ord{z} \geq \ord{B}$
%%\end{enumerate}
%%
%%
%%The conditions 2 and 3 ensure that $\forall z \in \Delta : \ord{z}
%%\geq \ord{C}$ therefore we can use the weakening rule to introduce
%%all the variable of $\Delta$ in the context of the sequent
%%$\Sigma',\overline{x}:\overline{A}\vdash^i L:C$:
%%
%%$$\rulef{\rulef{ \Sigma',\overline{x}:\overline{A}\vdash^i L:C  }
%%        {\vdots} (wk^i_0)}
%%        {\Sigma,\overline{x}:\overline{A}\vdash^i L:C} (wk^i_0)
%%$$
%%
%%By lemma \ref{va_lem:subst_preserve_i} we obtain:
%%$$ \Sigma, \overline{x}'':\overline{A}'' \vdash^i L\subst{N_1 \ldots N_l}{\overline{x}'}$$
%%Finally using the abstraction rule:
%%$$ \Sigma \vdash^0 \lambda \overline{x}'':\overline{A}'' . L\subst{N_1 \ldots N_l}{\overline{x}'}$$
\end{proof}




\subsection{Examples}

\subsubsection{Example 1 - Damien Sereni SCT counter-example}
In \cite{serenistypesct05}, the following counter-example is given
to show that not all simply-typed terms are size-change terminating
(see \cite{jones01} for a definition of size-change termination):

$$ E =  (\lambda a . a (\lambda b . a (\lambda c d .d))) (\lambda e . e (\lambda f .f))$$
where:
\begin{eqnarray*}
a &:& ((\tau \typear \tau) \typear \mu \typear \mu) \typear \mu \typear \mu \\
b &:& \tau \typear \tau \\
c &:& \tau \typear \tau \\
d &:& \mu \\
e &:& (\tau \typear \tau) \typear \mu \typear \mu \\
f &:& \tau
\end{eqnarray*}




\subsection{Particular case of homogeneously-safe lambda terms}

We look at a particular class of lambda terms: those having a
homogeneous type (as defined in section \ref{sec:homotypes}). We
recall that a type $(A_1, A_2, \ldots A_n, o)$ is said to be
homogeneous whenever $\order{A_1} \geq \order{A_2} \geq \ldots \geq
\order{A_n}$ and each of the $A_i$ are homogeneous. A term is
homogeneous if its type is homogeneous.


In their definition of safety (\cite{KNU02}), Knapik et al. require
that all the recursion equations of a safe recursion scheme have a
homogeneous type.

In the rules defining safety for the non-homogeneous case, the only
rule that can potentially introduce a non-homogeneous term from a
homogeneous one is the abstraction rule. But such a term (lambda
abstraction) corresponds exactly to a recursion equation in the
recursion scheme setting of Knapik et al. Therefore requiring that
recursions equation have homogeneous type is the same as requiring
that all sequents appearing in the proof tree of a safe term are of
homogeneous type.

We say that a term is homogeneously-safe if its type is homogeneous
and there is a proof tree showing its safety where all the sequents
of the proof tree are of homogenous type.

We are now going to specialize the rules of the safe
$\lambda$-calculus to obtain a system of rules for
homogeneously-safe terms.

\subsubsection{The application rule}
We recall the rule $\mathbf{(app^i)}$:
$$
 \mathbf{(app^i)} \quad  \rulef{\Gamma \vdash^i M : A\rightarrow B
                                        \qquad \Gamma \vdash^{0} N : A}
                                   {\Gamma  \vdash^{u} M N : B}
\quad \mbox{where } u = \min(i+\max\left(0, 1 + \ord{A} -
\ord{B}\right),0)
$$

Because of type homogeneity we have $\ord{A\typear B } = 1 +
\ord{A}$. The second premise gives $\forall z \in \Gamma : \ord{z}
\geq \ord{A} = 1 + \ord{A} - 1$. Hence the exponent $i$ in the first
premise can be replaced by $-1$.


Moreover type homogeneity implies $\ord{A} \geq \ord{B}-1$ therefore
$1 + \ord{A} - \ord{B} \geq 0$ and
$$ u = \min(i+1 + \ord{A} - \ord{B},0) = \min(\ord{A} - \ord{B},0)$$

\begin{itemize}
\item Suppose that $\ord{A} \geq \ord{B}$ then $u=0$ and we obtain the following rule:
$$ \mathbf{(app_1)} \quad  \rulef{\Gamma \vdash^{-1} M : A\rightarrow B
                                        \qquad \Gamma \vdash^{0} N : A }
                                   {\Gamma  \vdash^{0} M N : B}
                                    \qquad \ord{A} \geq \ord{B}$$

\item Suppose that $\ord{A} = \ord{B} - 1$ then
$ u = -1$ and we obtain the following rule:
$$ \mathbf{(app_2)} \quad  \rulef{\Gamma \vdash^{-1} M : A\rightarrow B
                                        \qquad \Gamma \vdash^{0} N : A
                                   }
                                   {\Gamma  \vdash^{-1} M N : B}
                                    \qquad \ord{A} = \ord{B} - 1$$
\end{itemize}



\subsubsection{The abstraction rule}

Let us derive the abstraction rule specialized for the case of
homogeneous types. We recall the rule $\rulename{abs}$:
$$ \rulename{abs^i} \quad  \rulef{\Gamma, \overline{x} : \overline{A} \vdash^{i} M : B}
                                   {\Gamma  \vdash^{0} \lambda \overline{x} : \overline{A} . M : (\overline{A},B)} \qquad
                                   \forall y \in \Gamma : \ord{y} \geq \ord{\overline{A},B}$$

We also use the notation $(\overline{A}|B)$ to denote the
homogeneous type $(A_1, A_2, \ldots A_n, B)$ where $\ord{A_1} =
\ord{A_2} =  \ldots \ord{A_n} \geq \ord{B} -1$.


Suppose that we abstract the single variable $\overline{x} = x$,
then in order to respect the side condition, we need to abstract all
variables of order lower or equal to $\ord{x}$. In particular we
need to abstract the partition of the order of $x$. Moreover to
respect type homogeneity, we need to abstract variables of the
lowest order first.

Hence we can change the abstraction rule so that it only allows
abstraction of the lowest variable partition. The rule can then be
used repeatedly if further partitions need to be abstracted.

The context $\Gamma$ is partitioned according to the order of the
variables. The partitions are written in decreasing order of type
order. The notation $\Gamma | \overline{x}:\overline{A}$ means that
$\overline{x}:\overline{A}$ is the lowest partition of the context.
We obtained the following rule:
$$ \rulename{abs^i} \quad  \rulef{\Gamma| \overline{x} : \overline{A} \vdash^{-1} M : B}
                                   {\Gamma  \vdash^{0} \lambda \overline{x} : \overline{A} . M : (\overline{A}|B)}$$

Note that the side-condition has disappeared.

\subsubsection{The other rules}

\begin{lem}
If a term is homogeneously-safe then there is valid proof tree
showing that it is safe containing only judgments of the form
$\Gamma \vdash^{k} M : T$ with $k\in \{-1,0\}$.
\end{lem}

\begin{proof}
This is a direct consequence from the fact that the sequents
appearing in the rules $\rulename{app}$ and $\rulename{abs}$ are all
of the form $\Gamma \vdash^{k} M : T$ with $k\in \{-1,0\}$. The
result can be proved formally with an easy structural induction.
\end{proof}

This lemma permits us to simplify the rules $\rulename{wk^i}$,
$\rulename{var}$, $\rulename{const}$, $\rulename{seq}$ and
$\rulename{perm}$. Table \ref{va_tab:homosafelmd_rules} recapitulates
the entire set of rules.

\begin{table}[htbp]
$$  \rulename{perm} {
      { \Gamma \vdash^0 M:B \qquad \sigma(\Gamma)  } \hbox{ homogeneous}
    \over
      { \sigma(\Gamma) \vdash^0 M : B }
    }
\qquad \rulename{seq} \quad \rulef{\Gamma \vdash^{0} M : A}{\Gamma
\vdash^{-1} M : A}
$$

$$
 \rulename{const}
    { \over { \vdash^0 b : o^r \rightarrow o}} \quad b : o^r \rightarrow o \in \Sigma
\qquad
 \rulename{var} \quad  \rulef{}{x : A\vdash^{0} x : A} $$

$$ \rulename{wk^{0}} \quad  \rulef{\Gamma \vdash^{0} M : A}{\Gamma , x : B \vdash^{0} M : A} \quad \ord{B} \geq \ord{A} $$

$$ \rulename{wk^{-1}} \quad  \rulef{\Gamma \vdash^{-1} M : A}{\Gamma , x : B \vdash^{-1} M : A} \quad \ord{B} \geq \ord{A} -1$$


$$ \mathbf{(app_1)} \quad  \rulef{\Gamma \vdash^{-1} M : A\rightarrow B
                                        \qquad \Gamma \vdash^{0} N : A }
                                   {\Gamma  \vdash^{0} M N : B}
                                    \qquad \ord{A} \geq \ord{B}$$

$$ \mathbf{(app_2)} \quad  \rulef{\Gamma \vdash^{-1} M : A\rightarrow B
                                        \qquad \Gamma \vdash^{0} N : A
                                   }
                                   {\Gamma  \vdash^{-1} M N : B}
                                    \qquad \ord{A} = \ord{B} - 1$$

$$ \rulename{abs} \quad  \rulef{\Gamma| \overline{x} : \overline{A} \vdash^{-1} M : B}
                                   {\Gamma  \vdash^{0} \lambda \overline{x} : \overline{A} . M : (\overline{A}|B)}$$
\caption{Rules of the homogeneously-safe lambda calculus}
\label{va_tab:homosafelmd_rules}
\end{table}

\subsubsection{Comparison with the rules of table \ref{tab:homosafelmd_rules_refined}}

The application rules are the only rules that do not match the
definition of table \ref{tab:homosafelmd_rules_refined}.

Here is a counter-example. Suppose:
\begin{eqnarray*}
 x&:&o\\
 \varphi, \theta &:& (o \typear o) \typear o \\
 f &:& \tau = (o \typear o) \typear (o \typear o) \typear o \\
\emptyset &\vdash^0& M \equiv (\lambda \varphi \theta . \varphi (\lambda x . x)) \\
f:\tau &\vdash^0& N \equiv f (\lambda x . x)
\end{eqnarray*}
Then we have $ f:\tau \vdash^{-1} M $ and using the
$\rulename{app_2}$ we get:
$$ f : \tau \vdash^{-1} M N$$

This term is not valid for the system of rules given in table \ref{tab:homosafelmd_rules_refined} simply because
for this system of rules if $\Gamma \vdash^i M$ then for all variable $x$ \emph{occurring free} in $M$, $\ord{x}\geq\ord{M}$. However
here $f$ occurs freely in $M N$ and $\ord{f} = 2 < 3 = \ord{M N}$.

%We observe that these rules correspond exactly to the rules given in
%the previous section in table \ref{tab:homosafelmd_rules_refined}.


% third chapter
\def\cmptre#1{\tau(#1)}
\def\aux#1{\lceil #1\rceil}
\def\nf#1{\eta_{\sf nf}(#1)}

\section{Game semantics of safe $\lambda$-terms}

In this section we will prove that the safety condition
of section \ref{sec:safe_alt} leads to a pointer economy in the game
semantics: for safe $\lambda$-terms the pointers from the game semantics can be reconstructed uniquely from the moves of
the play.

The example of section \ref{subsec:ptrless_strat} gives the intuition.
Remember that in order to distinguish the terms $M_1$ and $M_2$,
we introduced pointers in strategies. In the safe $\lambda$-calculus
this ambiguity disappears because $M_2$ is not a safe term. Indeed, in the
sub-term $f (\lambda y . x)$, the free variable $x$
has the same order as $y$ but $x$ is not abstracted together
with $y$.


%\begin{enumerate}
%\item
%Is there any unsafe term whose game semantics is a strategy where
%pointers can be recovered?
%
%The answer is yes: take the term $T_i = (\lambda x y . y) M_i S$
%where $i =1..2$ and $\Gamma \vdash_s S : A$. $T_1$ and $T_2$ both
%$\beta$-reduce to the safe term $S$, therefore
%$\sem{T_1}=\sem{T_2}=\sem{S}$. But $T_1$ is safe whereas $T_2$ is
%unsafe. Since it is possible to recover the pointer from the game
%semantics of $S$, it is as well possible to recover the pointer from
%the semantics of $T_2$ which is unsafe.
%
%\item
%Is there any unsafe $\beta$-normal form whose game semantics is a
%strategy where pointers can be recovered?
%\end{enumerate}






\subsection{$\eta$-long normal form}

The $\eta$-expansion of $M: A\typear B$ is defined to be the term $\lambda x . M x : A\typear B$ where $x:A$ is a fresh variable.
It is easy to check that if $M$ is safe then $\lambda x . M x$ is also safe.

Consider the term $M : (A_1,\ldots,A_n,o)$, it can be expanded in several steps into
$\lambda \varphi_1 \ldots \varphi_l . M \varphi_1 \ldots \varphi_l$
where the $\varphi_i:A_i$ are fresh variables.

The $\eta$-normal form of a term is obtained by hereditarily $\eta$-expanding every sub-term occurring
at an operand position:

\begin{dfn}[$\eta$-long normal form]
A term is either an abstraction or it can be written uniquely as
$s_0 s_1 \ldots s_m$ where $m\geq0$ and $s_0$ is a variable, a
constant or an abstraction.

The $\eta$-long normal form of a term $M$ is denoted $\aux{M}$ and
is defined as follow:
\begin{eqnarray*}
\aux{x s_1 \ldots s_m : (A_1,\ldots,A_n,o)} &=& \lambda \overline{\varphi} . x \aux{s_1} \aux{s_2} \ldots \aux{s_m} \aux{\varphi_1} \ldots \aux{\varphi_n} \\
\aux{x s_1 \ldots s_m : o} &=& \lambda . x \aux{s_1} \aux{s_2} \ldots \aux{s_m} \\
\aux{(\lambda x . s_0) s_1 \ldots s_m } &=& (\lambda x . \aux{s_0}) \aux{s_1} \aux{s_2} \ldots \aux{s_m}
\end{eqnarray*}
where $m \geq 0$ and $x$ is either a variable or a constant.
\end{dfn}

The $\eta$-long normal form appeared in \citep{DBLP:journals/tcs/JensenP76}
under the name \emph{long reduced form}
and in \citep{DBLP:journals/tcs/Huet75}
under the name \emph{$\eta$-normal form}. It was then investigated in \citep{huet76}
under the name \emph{extensional form}.


A term can be represented by a tree defined formally by induction on the structure
of its $\eta$-long normal form as follow:

\begin{dfn}[Computation tree]
The computation tree associated to the term $s$ is noted
$\cmptre{s}$. It is obtained by applying the following rules
inductively \emph{on the $\eta$-long normal form} of $s$. In the
following $x$ is either a variable or a constant.
\begin{itemize}
\item the tree for $\lambda x_1 \ldots x_n. M$ where $M$ is not an abstraction is:
$$ \cmptre{\lambda x_1 \ldots x_n . M} =
  \pstree[levelsep=4ex]
    { \TR{\lambda x_1 \ldots x_n} }
    { \SubTree{\cmptre{M}}
    }
$$


\item the tree for $x s_1 \ldots s_n$ is:
$$ \cmptre{ x s_1 \ldots s_n} =
  \pstree[levelsep=4ex]
    { \TR{x @} }
    { \SubTree{\cmptre{s_1}} \SubTree[linestyle=none]{\ldots} \SubTree{\cmptre{s_n}}
    }
$$

\item the tree for $x$ is the single leaf $x$.

\item the tree for $(\lambda x.s_0) s_1 \ldots s_n$ is:
$$ \cmptre{(\lambda x.s_0) s_1 \ldots s_n} =
  \pstree[levelsep=4ex]
    { \TR{@} }
    {
    \SubTree{\cmptre{\lambda x.s_0}}    \SubTree{\cmptre{s_1}} \SubTree[linestyle=none]{\ldots} \SubTree{\cmptre{s_n}}
    }
$$
\end{itemize}
\end{dfn}

Example: if $x$ is a variable or a constant then
$ \cmptre{\lambda . x} =
  \pstree[levelsep=3ex]
    { \TR{\lambda } }
    { \TR{x}
    }$

The nodes (and leaves) of the tree are of three kinds:
\begin{itemize}
\item $\lambda$-node labeled $\lambda \overline{x}$. A $\lambda$-node represents several consecutive abstractions of variables.
\item application node labeled $@$
\item operator-application nodes labeled $x @$ where the operator $x$ is
either a variable or a constant.
\end{itemize}

A sub-tree of the computation tree represents a $\lambda$-term. We
define the map $\kappa : N \rightarrow \mathcal{T}$ where $N$
denotes the set of nodes and leaves of the computation tree
$\tau(s)$ and $\mathcal{T}$ denotes the set of $\lambda$-terms.
$\kappa$ associates to any node $n$ of the tree the $\lambda$-term
$\kappa(n)$ that is represented by the sub-tree of $\tau(s)$ rooted
at $n$. In particular if $r$ is the root of the tree $\tau(s)$ then
$\kappa(s) = \aux{s}$.



Consider the computation tree $\tau(s)$ of a term $s$ in $\eta$-long normal form. Then:
\begin{itemize}
\item One can check that nodes at even level are abstraction
node and nodes at odd level are either application nodes,
operator-application nodes, variable or constant nodes (the root level being numbered $0$).

\item Suppose that a variable $x$ occurs in $s$. The corresponding node in the tree has of one of the two following forms:
    \begin{itemize}
    \item $ \pstree[levelsep=3ex]
        { \TR{\lambda } }
        { \TR{x}
        }$ where $\ord{x} = 0$

    \item $ \pstree[levelsep=3ex]
                { \TR{x @} }
                { \TR{\lambda \overline{\xi_1}} \TR{\ldots} \TR{\lambda \overline{\xi_p}}}
        $ where $\ord{x} > 0$ and $x:(A_1,\ldots,A_p,o)$
    \end{itemize}

\item    Moreover for any abstraction node
        $ \pstree[levelsep=4ex]
            { \TR{\lambda \overline{\varphi}} }
            { \pstree[levelsep=3ex]
                {\TR{@^{[n]}}}
                {\TR{\lambda \overline{\xi_1}} \TR{\ldots} \TR{\lambda \overline{\xi_p}}}
            }
        $
    we have $\ord{\kappa(@^{[n]})}=0$

\end{itemize}

\subsubsection{Pointers and justified sequence of nodes}

We introduce pointer in the computation tree: a node $n$ labeled
with the variable $x$ points to a lambda node $m$ labeled $\lambda
\overline{\varphi}$ if and only if the variable $x$ is bound in
$\kappa(m)$ by the abstraction $\lambda \overline{\varphi}$. In that
case we say that the node $n$ is bound by $m$. Additionally a lambda node points to its parent node.

To sum up a justification pointer goes upward from a variable occurrence to its bindings or from an abstraction node to its parent node.


We call \emph{traversal} or \emph{justified sequence of nodes} any sequence of nodes of the computation tree together
with links as defined above.

\subsection{Correspondence with game semantics}

By representing side-by-side the computation tree and the type arena of a term in $\eta$-normal form we observe
that for each question move of the arena there are corresponding nodes in the computation tree.

\begin{exmp}
Consider the following term $M \equiv \lambda f z . (\lambda g x . f (f x)) (\lambda y. y) z$ of type $(o \typear o) \typear o \typear o$.
Its $\eta$-long normal form is $\lambda f z . (\lambda g x . f (f x)) (\lambda y. y) (\lambda .z)$.
The computation tree is:

$$
\tree{\lambda f z}
{ \tree{@}
    {
        \tree{\lambda g x}
            { \tree{f@}{   \tree{\lambda}{ \tree{f@}{  \tree{\lambda}{\TR{x}}} }  }
            }
        \tree{\lambda y}{\TR{y}}
        \tree{\lambda}{\TR{z}}
    }
}
$$

The arena for the type $(o \typear o) \typear o \typear o$ is:
$$\tree{q^1}
{
    \tree{q^3}
        {  \tree{q^4}
                {\TR{a^4_1} \TR{\ldots}}
            \TR{a^3_1} \TR{\ldots} }
    \tree{q^2}
    { \TR{a^2_1} \TR{a^2_2}\TR{\ldots} }
    \TR{a_1} \TR{a_2}\TR{\ldots}
}
$$

\newlength{\yNull}
\def\bow{\quad\psarc{->}(0,\yNull){1.5ex}{90}{270}}

We now omit the answers moves when we represent the arena.
The arena is represented on the right and the computation tree on the left.

The dashed line defines a relation $\varphi$ from the set of question moves to nodes in the computation tree.
$\varphi$ maps each question to one or more nodes in the computation tree:
$$
\tree{ \Rnode{root} {\lambda f z}^{[1]} }
     {  \tree{@^{[2]}}
        {   \tree{\lambda g x ^{[3]}}
                { \tree{\Rnode{f}{f@^{[6]}}}{  \tree{\Rnode{lmd}\lambda^{[7]}}{ \tree{\Rnode{f2}{f@^{[8]}}} {\tree{\Rnode{lmd2}\lambda^{[9]}}{\TR{x^{[10]}}}}}  }
                }
            \tree{\lambda y ^{[4]}}{\TR{y}}
            \tree{\lambda ^{[5]}}{\TR{\Rnode{z}z}}
        }
    }
\hspace{3cm}
  \tree[levelsep=12ex]{ \Rnode{q1}q^1 }
    {   \pstree[levelsep=4ex]{\TR{\Rnode{q3}q^3}}{\TR{\Rnode{q4}q^4}}
        \TR{\Rnode{q2}q^2}
    }
\psset{nodesep=1pt,arrows=->,arcangle=-20,arrowsize=2pt 1,linestyle=dashed,linewidth=0.3pt}
\ncline{<-}{root}{q1} \aput*{:U}{\varphi}
\ncarc{->}{q2}{z}
\ncline{->}{q3}{f}
\ncline{->}{q4}{lmd}
\ncline{->}{q3}{f2}
\ncline{->}{q4}{lmd2}
$$

Consider the justified sequence of moves $s \in \sem{M}$:
\vspace{0.5cm}
 $$s =
\rnode{q1}{q}^1\
\rnode{q3}{q}^3\
\rnode{q4}{q}^4\
\rnode{q3b}{q}^3\
\rnode{q4b}{q}^4\
\rnode{q2}{q}^2
\bkptrc{q3}{q1}
\bkptrc[ncurv=0.5]{q3b}{q1}
\bkptrc{q4}{q3}
\bkptrc{q4b}{q3b}
\bkptrc[ncurv=0.5]{q2}{q1}
\in \sem{M}$$

There is a corresponding justified sequence of nodes in the computation tree:
\vspace{0.5cm}
$$t =
\rnode{q1}{\lambda f z} \cdot
\rnode{q3}{f@}^{[6]} \cdot
\rnode{q4}{\lambda^{[7]}} \cdot
\rnode{q3b}{f@}^{[8]} \cdot
\rnode{q4b}{\lambda^{[9]}} \cdot
\rnode{q2}{z}
\bkptra[ncurv=1]{60}{q3}{q1}
\bkptra[ncurv=1]{60}{q4}{q3}
\bkptra[ncurv=0.4]{75}{q3b}{q1}
\bkptra[ncurv=0.8]{70}{q4b}{q3b}
\bkptra[ncurv=0.4]{80}{q2}{q1}$$
such that $t_i = \varphi(s_i)$ for all $i < |s|$.

We see on this example that the game semantics and the computation tree are somehow related to each other.
\end{exmp}

Let us now define precisely the relationship between game semantics and computation tree.

First we define the relation $\varphi$ that maps question-moves to
set of nodes in the computation tree.

\begin{dfn}[Relation between question-moves and nodes]
\label{def:phi_procedure}
Let $\Gamma \vdash M : A$ be a term in $\eta$-long normal form.
Suppose $(M,\vdash)$ is the arena $\sem{A}$ where $M$ is the set of moves and $\vdash$ is the enabling relation.
$M^Q \subseteq M$ denotes the set of question-moves.
We note $(N,E)$ the computation tree $\tau(M)$ where $N$ is the set of nodes and leaves of the tree and $E$ is the parent-child relation.


We give an algorithm to compute the partial function $\varphi^{-1} : N \rightarrow M^Q$.
The total function $\varphi : M^Q \rightarrow \mathcal{P}(N)$ can then be obtained from
$\varphi^{-1}$.

We define $\varphi^{-1} = f(0,q^0)$ where the index $0$ denotes
the root of the computation tree, $q^0$ is the root of the arena
(in the game semantics of simply-typed lambda calculus the arenas have
a single root) and $f$ is defined below.

The procedure $f$ takes two parameters: $n$ is the index number of a
$\lambda$-node in the computation tree and $q$ is a question move of the arena
of $\Gamma \vdash M : A$ such that $q$ and $\kappa(n)$ have the same type.

Similarly we define the procedure $g$ that
takes two parameters: $n$ is the index number of a
$x@$-labelled node or a $x$-labelled node in the computation tree and $q$ is a question move
of the arena of $\Gamma \vdash M : A$ such that $q$ and $x$ have the same type.
\\

\noindent
\begin{description}
\item[\textbf{Procedure} $f(nd,q)$]
    where $nd$ is a $\lambda$-node.

    \begin{itemize}
    \item If $\ord{\kappa(nd)} = 0$ then the term is of ground type therefore
    the game for $M$ is played on the flat arena
    with only one question $q$. Moreover the node $nd$ of the computation tree is labelled with $\lambda$.\\
    \textbf{return} $\{ nd \mapsto q \}$.

    \item $\ord{\kappa(nd)} > 0$. The computation tree and the arena
    have the following form:
    $$ \tree[levelsep=6ex]{ \Rnode{r}\lambda \overline{\xi}_n  ^{[nd]}}
        {
            \tree[levelsep=6ex]{x@^{[nd+1]}}
            {   \TR{\lambda^{[\ldots]}} \TR{\ldots} \TR{\lambda^{[\ldots]}}
            }
        }
    \hspace{3cm}
    \tree{ \Rnode{q0}q }
        {
            \tree[linestyle=dotted]{q^1}{\TR{} \TR{} }
            \tree[linestyle=dotted]{q^2}{\TR{} \TR{} }
            \TR{\ldots}
            \tree[linestyle=dotted]{q^n}{\TR{} \TR{} }
        }
    \psset{nodesep=1pt,arrows=->,arcangle=-20,arrowsize=2pt 1,linestyle=dashed,linewidth=0.3pt}
    \ncline{->}{r}{q0}
    \ncarc{->}{q2}{z}
    \ncline{->}{q3}{f}
    \ncline{->}{q4}{lmd}
    \ncline{->}{q3}{f2}
    \ncline{->}{q4}{lmd2}
    $$

    such that $\Gamma, \overline{\xi}_n \vdash \kappa(x@^{[nd+1]}) : o$.

    For each of the abstracted variable $\xi_i$ there is a corresponding question move $q_i$ of the same order
    in the arena.  Each free occurrence of the variable $\xi_i$ is mapped to the move $q_i$
    by the procedure $g$.

    $$\mathbf{return} \ \{ nd \mapsto q \}
    \union
    \Union_{\stackrel{i=1..n}{\xi_i^{[k]} \in Desc(nd)}} g ( k, q_i)
    \quad \union \quad
    \Union_{\stackrel{i=1..n}{@\xi_i^{[k]} \in Desc(nd)}} g ( k, q_i)
    $$
    where $Desc(nd)$ is the set of descendants of  node $nd$
    (nodes $m$ such that there is a path from node $nd$ to node $m$ in the computation
    tree).

    \end{itemize}

\item[\textbf{Procedure} $g(nd,q)$]\  \\
The procedure is not defined on $\lambda$-nodes or $@$ nodes.
This is ok since all the calls to $g$ in $f$ are of the type $g(nd,q)$ where $nd$ denotes a $@x$-node or a $x$-node.

\begin{itemize}
\item[case 1] Suppose that $nd$ is labelled with $x$ then we must have $x:0$. \textbf{return}
$\{ nd \mapsto q \}$.

\item[case 2] If $nd$ is labelled with $x@$ then $x:(A_1|\ldots|A_m|o)$.
The computation tree and the arena  have the following form:

    $$\tree[levelsep=6ex]{\Rnode{r}{x@^{[nd]}}}
        {   \tree{\TR{\lambda^{[k_1]}}}{\vdots} \TR{\ldots}
        \tree{\TR{\lambda^{[k_m]}}}{\vdots}
        }
    \hspace{3cm}
    \tree{ \Rnode{q0}q }
        {
            \tree[linestyle=dotted]{\Rnode{q1}{q^1}}{\TR{} \TR{} }
            \tree[linestyle=dotted]{\Rnode{q2}{q^2}}{\TR{} \TR{} }
            \TR{\ldots}
            \tree[linestyle=dotted]{\Rnode{qm}{q^m}}{\TR{} \TR{} }
        }
    \psset{nodesep=1pt,arrows=->,arcangle=-20,arrowsize=2pt 1,linestyle=dashed,linewidth=0.3pt}
    \ncline{->}{r}{q0}
    \ncarc{->}{q2}{z}
    \ncline{->}{q3}{f}
    \ncline{->}{q4}{lmd}
    \ncline{->}{q3}{f2}
    \ncline{->}{q4}{lmd2}
    $$

    such that $\Gamma, \overline{\xi}_n \vdash \kappa(x@^{[nd]}) : o$.

    For each of the children node of $nd$
    there is a corresponding question move $q_i$ of the same type
    in the arena.
    $$\mathbf{return} \ \{ nd \mapsto q \} \union
    \Union_{i=1..m} f ( k_i, q_i)
    $$
\end{itemize}
\end{description}
\end{dfn}

One can check that the inverse of $\varphi^{-1}$ is indeed a total function : $\varphi : M^Q \rightarrow N$.
Moreover $\varphi$ is such that player O questions are associated to $\lambda$-nodes
and player P questions are associated to $x @$-nodes or $x$-nodes.


Notation: Suppose $s$ is a justified sequence of moves then we note $\tilde{s}$
the subsequence of $s$ consisting of question-moves only:
\begin{eqnarray*}
\tilde{} &: L_A &\longrightarrow (M^Q)^* \\
& s &\longmapsto \tilde{s} = s \upharpoonright M^Q
\end{eqnarray*}
where $M^Q$ denotes the set of question-moves.
$\tilde{s}$ is also a justified sequence of move (there is no dangling pointer since questions-moves points to other question-moves).
If $s = u\upharpoonright A,B$ then clearly $\tilde{s} = \tilde{u} \upharpoonright A,B$.

\begin{prop}[Justified sequence of moves -- traversal in the computation tree]
\label{prop:cor_trav_seq}
Let $s$ be a justified sequence of moves such that $s \in \sem{\Gamma \vdash M:A}$ then
there is a traversal $t = n_0 n_1 \ldots$ of $\tau(M)$ such that $\tilde{s} = \varphi^{-1} (t)$ (i.e. $\tilde{s} = \varphi^{-1}(n_0)\ \varphi^{-1}(n_1) \ldots$)
\end{prop}
\begin{proof}
By induction on the structure of $M$ in $\eta$-long normal form.
Let $\Gamma = \{y_1, : B_1, \ldots , y_r : B_r \}$ and $s$ be as justified sequence of moves in $\sem{\Gamma \vdash M}$.

\begin{itemize}
  \item Suppose that $\ord{M} = 0$ then the arena is of the following form (only question moves are represented):
    $$ \tree{ q }
        {   \tree[linestyle=dotted]{q^1}{\TR{} \TR{} }
            \tree[linestyle=dotted]{q^2}{\TR{} \TR{} }
            \TR{\ldots}
            \tree[linestyle=dotted]{q^r}{\TR{} \TR{} }
        }$$
        \begin{itemize}
        \item Suppose that $M = \lambda. y_i : o$. Then $\sem{M} = \pi_i$ and the only question move that can occur
                in $s \in \pi_i$ is $q$. Take $t = \tilde{s}[\lambda/q]$ where $\lambda$ denotes the root of the tree $\tau(M)$.
                Then it is clear that $\varphi^{-1}(t) = \tilde{s}$.

        \item $\Gamma \vdash h_0 h_1 \ldots h_p : o$ where $\Gamma \vdash h_0 : (A_1,\ldots,A_p,o)$ and $\Gamma \vdash h_1 : A_1, \ldots, \Gamma \vdash h_p:A_p$.
        Then $$
            \sem{\Gamma \vdash h_0 h_1 \ldots h_p : o} = \overbrace{\langle \sem{\Gamma \vdash h_0}, \ldots, \sem{\Gamma \vdash h_p} \rangle}^\sigma ; ev_{A_1,\ldots,A_p,o}
            $$
        with $\sigma : A \longrightarrow B$ , $ev_{A_1,\ldots,A_p,o} : B \longrightarrow C$ where
        \begin{eqnarray*}
         A &=& \sem{\Gamma} \\
         B &=& \sem{(A_1,\ldots,A_p,o)} \times H_1 \times \ldots \times H_p \\
         C &=& \sem{o} \\
         H_j &=& \sem{A_j} \mbox{ for }j \in 1..p
        \end{eqnarray*}

        Consequently:
        \begin{equation}
            s = u \upharpoonright A, C \label{eq:def_s}
        \end{equation}
        for some $u \in \sigma \parallel ev_{A_1,\ldots,A_p,o}$, more precisely for some $u$ such that:
        \begin{eqnarray*}
          u &\in& int(A,B,C) \\
          w = u \upharpoonright A,B & \in& \sigma \\
          u \upharpoonright B,C & \in & ev_{A_1,\ldots,A_p,o}
        \end{eqnarray*}

        $w \in \sigma$ implies that for some $j \in 1..p$:
        \begin{eqnarray}
          z = w \upharpoonright A, H_j &\in& \sem{h_j}  \label{eq:def_z} \\
          \mbox{ and for every } k\neq j &:& w \upharpoonright H_k = \epsilon \label{eq:b}
        \end{eqnarray}

        Equation \ref{eq:a} allows us to use the induction hypothesis on the term $h_j$:
        there is a traversal $t$ of the tree $\tau(h_j)$ such that $\varphi^{-1}_{h_j}(t) = \tilde{z}$
        where $\varphi^{-1}_{h_j}$ denotes the function obtained from the procedure
        given in definition \ref{def:phi_procedure} applied on the term $h_j$.

        Let $q_C$ denotes the only question of the arena $C = \sem{o}$.

        We introduce a new operator $\upharpoonright^C$. This operator works similarly
        to the filtering operator $\upharpoonright$ except that it does not filter
        elements that belong to $C$.
        Hence for any justified sequence of moves $s \in int(A,B,C)$
        we have $s  = s \upharpoonright^C A,B$.
        and for arena $A$, $H_j$ we have $s\upharpoonright A, H_j =
        (s\upharpoonright^C A, H_j) \upharpoonright A, H_j $.

        We define the following justified sequence of moves:
        \begin{eqnarray}
            w^\ast &=& \tilde{u} \upharpoonright^C A,B \ =\  \tilde{u} \label{eq:def_wstar} \\
            z^\ast &=& \tilde{w} \upharpoonright^C A, H_j \label{eq:def_zstar}
        \end{eqnarray}

        Since $z = w \upharpoonright A, H_j$ we have
        $\tilde{z} = \tilde{w} \upharpoonright A, H_j = (\tilde{w} \upharpoonright^C A,H_j) \upharpoonright A, H_j
        = z^\ast \upharpoonright A, H_j$. Similarly $w^\ast \upharpoonright A,B =\tilde{w}$.


        Moreover for $k\neq h : w \upharpoonright H_k = \epsilon$, said differently
        moves in $w$ hereditarily justified by moves in $B$ are in fact all
        hereditarily justified by moves in $H_j$.  Hence:
        \begin{equation}
         z^\ast \quad=\quad \tilde{w} \upharpoonright^C A, H_j \quad = \quad \tilde{w} \upharpoonright^C A, B  \quad = \quad  \tilde{w}
        \end{equation}


        We know that there is a traversal $t$ of $\tau(h_j)$ such that
        $\varphi^{-1}_{h_j}(t) = \tilde{z}$.

        The domain of $\varphi^{-1}_{h_j}$ is a subset of $N_{h_j}$, the set of nodes of $\tau(h_j)$, therefore it does not contain
        the root of $\tau(M)$ noted $\lambda$. The domain of the function $\varphi^{-1}_{M}$ is a subset of $N$,
        the set of nodes  of $\tau(M)$.

        The function $\varphi^{-1}$ has been defined by an inductive procedure
        guaranteeing that $\varphi^{-1}_{h_j}$ and $\varphi^{-1}_{M}$
        agree on the intersection of their respective domain, hence we have
        \begin{equation}
        \varphi^{-1}_{M}(t) = \tilde{z}. \label{eq:def_t}
        \end{equation}
        with $\varphi^{-1}_{M}(\lambda) = q_C$.

        From $t$ we can construct the justified sequence $t^\ast$ of as follow:
        we insert the node $\lambda$ at the same positions as where the move $q_C$ appears in $z^\ast$.

        We then clearly have:
        \begin{equation}
        \varphi^{-1}_{M}(t^\ast) = z^\ast. \label{eq:def_tstar}
        \end{equation}


        The filtering operator $\upharpoonright$ is defined on node traversals as follow:
        $t\upharpoonright A,C$ is the subsequence of the traversal $t$ constituted of the nodes $n$
        verifying $\varphi^{-1}_M(n) \in A,C$.


        We remove from $t^\ast$ all the elements at positions $i$ such that $\tilde{u}_i \in B$.
        We obtain a justified sequence of nodes $t' = t^\ast \upharpoonright A,C$
        verifying
            $$\varphi^{-1}_{M}(t^\ast) = \tilde{s}$$


        \textbf{Example}: Let us illustrate the steps of the demonstration on a short example.
         For clarity the pointers are not specified in the following justified sequences.
        $$
        \xymatrix @=3pt{
          & \tilde{u} &=& m_0 & m_1 & m_2 & m_3 & m_4 & m_5 & m_6 & m_7 & m_8 & \ldots \\
          & &\in& A & B & C & A & A & C & B & B & C & \ldots \\
          \mbox{ (equation \ref{eq:def_s})} & \tilde{s} &=& m_0 & & m_2 & m_3 & m_4 & m_5 &  & & m_8 & \ldots \\
          \mbox{ (equation \ref{eq:def_wstar})} & w^\ast &=& m_0 & m_1 & q_C & m_3 & m_4 & q_C & m_6 & m_7 & q_C & \ldots \\
          \mbox{ (equation \ref{eq:def_z})} & \tilde{z}  &=& m_0 & m_1 &  & m_3 & m_4 &  & m_6 & m_7 &  & \ldots  \\ \\ \\
          \mbox{ (equation \ref{eq:def_t})} & t &=& N_0 \ar[uuu]^{\varphi^{-1}_{h_j}} & N_1\ar[uuu] &  & N_3 \ar[uuu] & N_4\ar[uuu] & & N_6\ar[uuu] & N_7\ar[uuu] &  & \ldots \\ \\
          \mbox{ (equation \ref{eq:def_zstar})} & z^\ast  &=& m_0 & m_1 & q_C & m_3 & m_4 & q_C & m_6 & m_7 & q_C & \ldots  \\ \\ \\
          \mbox{ (equation \ref{eq:def_tstar})} & t^\ast &=& N_0 \ar[uuu]^{\varphi^{-1}_{h_j}} & N_1\ar[uuu] & \lambda\ar[uuu]^{\varphi^{-1}_{M}} & N_3 \ar[uuu] & N_4\ar[uuu] & \lambda\ar[uuu] & N_6\ar[uuu] & N_7\ar[uuu] & \lambda\ar[uuu] & \ldots \\
          t^\ast\upharpoonright A,C =& t' &=& N_0 &  & \lambda & N_3 &  & \lambda & N_6 &  & \lambda & \ldots \\
          }
        $$
        Note that there is only one question $q_C$ in the arena $C$ therefore $m_2 = m_5 = m_8 = q_C$.
        It is easy to check that $\varphi^{-1}(t') = \tilde{s}$


        \end{itemize}

    \item Suppose that $\ord{M} > 0$ then $M = \lambda \overline{\xi} . N : \overline{A}|B$ and :
    $$ \sem{\Gamma : M} = \Lambda( \sem{\Gamma,\overline{\xi} \vdash N: B} )$$
\end{itemize}

\end{proof}

\todomargin{to finish}

\subsection{Pointers in the game semantics of safe terms are recoverable}

\begin{prop}
The pointers in the game semantics of safe terms are
uniquely recoverable.
\end{prop}

\begin{proof}
Let $M$ be a safe term, we consider its $\eta$-long normal form $\aux{M}$.
$\aux{M}$ is also safe because safety is preserved by $\eta$-expansion.

We consider a justified sequence of move $s \in \sem{M}$.
First, the pointers for O and P answers can all be recovered by using the well-bracketing condition.

For O-question, the justification pointer always points to its
parent node in the computation tree.

For P-question, suppose P ask for the value of variable $x$. Then
there may be several choices for the destination of the pointer but
we claim that in the case of safe terms, it should point to the
closest node in the path from the root to P-question whose
order is greater than the order of $x$. The order of a lambda-node
$\lambda \overline{\xi}$ being equal to $\max_{i=1..n} \ord{\xi_i}$ the


\todomargin{to finish}
\end{proof}


% fourth chapter
\chapter{Further possible developments}

In the previous chapter, we have given an account of the game
semantics of Safe $\lambda$-Calculus. However the nature of this
calculus is still not well known. We propose the following possible
roadmap for further research:
\begin{enumerate}
\item prove or disprove that observational equivalence is decidable for Safe \ialgol;
\item find a categorical interpretation of the Safe $\lambda$-Calculus;
\item study the proof theory obtained by the Curry-Howard isomorphism and determine whether it has nice properties that can be helpful in theorem proving;
\item determine which complexity class is characterized by the Safe-$\lambda$ calculus.
\end{enumerate}


In a more general direction of research, we would like to study the
class of languages for which pointers are uniquely recoverable. We
name this class PUR for ``Pointer Uniquely Recoverable''.

We proved that Safe $\lambda$-Calculus is a PUR-language. Another
example is the Serially Re-entrant Idealized Algol (SRIA) proposed
by Abramsky  in \cite{abramsky:mchecking_ia}. This language allows
multiple occurrences or uses of arguments, as long as they do not
overlap in time. In the game semantics denotation of a SRIA term
there is at most one pending occurrence of a question at any time.
Each move has therefore a unique justifier and consequently
justification pointers may be ignored. Safe \ialgol\ is not a
sublanguage of SRIA. One reason for this is that none of the two
Kierstead terms $\lambda f . f (\lambda x . f (\lambda y .y ))$ and
$\lambda f . f (\lambda x . f (\lambda y .x ))$ are Serially
Re-entrant whereas the first one is safe. Conversely, SRIA is not a
sublanguage of Safe \ialgol\ since the term $\lambda f g. f (\lambda
x . g (\lambda y .x ))$ where $f,g:((o,o),o)$ belongs to SRIA but
not to Safe \ialgol. SRIA and Safe \ialgol\ are therefore two
different examples of languages with pointer-less game semantics.

Finitary $\ialgol_2$ is also an example of PUR-language for which
observational equivalence is decidable. As we indicated in the first
chapter, decidability of observational equivalence is a very
appealing property which has immediate applications in the domain of
program verification. Intuitively, PUR-languages seem to be good
candidates of languages for which observational equivalence is
decidable. It would be interesting to discover classes of PUR
languages having this appealing property.

Another possible way to generate PUR-languages might be to constrain
the types of an existing language. In \cite{DBLP:conf/tlca/Joly01},
a notion of ``complexity'' is defined for $\lambda$-terms. It is
proved that a type $T$ can be generated from a finite set of
combinators if and only if there is a constant bounding the
complexity of every closed normal $\lambda$-term of type $T$;
consequently, the only inhabited finitely generated types are the
type of rank $\leq 2$ and the types $(A_1, A_2, \ldots, A_n, o)$
such that for all $i = 1..n$: $A_i = o$ , $A_i = o \rightarrow o$ or
$A_i = o^k \rightarrow o \rightarrow o$.

We know that imposing the first of these two type restrictions to
Finitary \ialgol\ leads to a PUR language. Is is also the case when
imposing the second type restriction?


\bibliographystyle{plainnat}
\bibliography{gamesem,modelchecking,proganalys,higherorder}

%adds the bibliography to the table of contents
\addcontentsline{toc}{chapter}
         {\protect\numberline{Bibliography\hspace{-96pt}}}
\end{document}
