%%% preambule

\usepackage[draft]{graphicx}
\usepackage{a4wide}
\usepackage{pst-tree}
\usepackage{stmaryrd}
\usepackage[all]{xypic}
\usepackage{amsmath, amsthm, amssymb}
\usepackage{bbm,latexsym}
\usepackage{manfnt}
\usepackage{natbib}


\newtheorem{thm}{Theorem}[section]
\newtheorem{cor}[thm]{Corollary}
\newtheorem{lem}[thm]{Lemma}
\newtheorem{prop}[thm]{Proposition}
\theoremstyle{remark}
\newtheorem{rem}[thm]{Remark}
\newtheorem{exmp}[thm]{Example}
\newtheorem{property}[thm]{Property}


\theoremstyle{definition}
\newtheorem{dfn}[thm]{Definition}
\newtheorem{algo}[thm]{Algorithm}


\newcommand\betared{\rightarrow_\beta}
\newcommand\betaredtr{\twoheadrightarrow_\beta} % transitive closure of the beta reduction
\newcommand\betasred{\rightarrow_{\beta_s}}
\newcommand\betasredO{\rightarrow_{\beta_s^1}}
\newcommand\betasredT{\rightarrow_{\beta_s^2}}
\newcommand\subst[2]{\left[ #1/#2 \right]}

%\newcommand\bot{\perp}
\newcommand\typar{\Rightarrow}
\newcommand\dps{\displaystyle}
\newcommand\rulef[2]{\frac{\dps #1}{#2}}
\newcommand\axiomf[1]{\frac{\dps}{#1}}
\newcommand\union{\cup}
\newcommand\Union{\bigcup}

\newcommand\ord[1]{{\sf ord}(#1)}

\newcommand{\lsem}{[\![} % \llbracket
\newcommand{\rsem}{]\!]} % \rrbracket
\newcommand{\sem}[1]{{\lsem #1 \rsem}}
\newcommand{\makeset}[1]{\{\,{#1}\,\}}


%%% game semantics
\newcommand\typexp{\texttt{exp}}
\newcommand\natbf{\mathbf{N}}
\newcommand\zset{\mathbb{Z}}
\newcommand\eval{\Downarrow}

\newcommand{\SubTree}[2][]{\Tr[ref=t]{\pstribox[#1]{#2}}}
\newcommand{\SubTreeE}[2][]{\Tr[ref=t]{\pstribox[#1]{#2}}}

\newcommand\dedge{\ncline[linestyle=dotted]}
\newcommand{\tree}[2][levelsep=4ex]{\pstree[levelsep=4ex,#1]{\TR{#2}}}


\newcommand\imp{\Rightarrow}
\newcommand\zand{\wedge}


% ia
\newcommand\iaskip{\texttt{skip}}
\newcommand\deref{\texttt{deref}}
\newcommand\assign{\texttt{assign}}


% justified sequence of move
\newcommand\jseq{\stackrel{\curvearrowleft}{=}} %equality of justseq

% back pointer using psttricks
\newcommand{\bkptrc}[1][nodesep=3pt]{\nccurve[nodesep=3pt,ncurv=1,angleA=90,angleB=90,#1]{->}}
\newcommand{\bkptra}[2][nodesep=3pt]{\ncarc[nodesep=3pt,ncurv=1,arcangleA=-#2, arcangleB=-#2,#1]{->}}


% backpointer using xypic
\newcommand{\justseq}[2][12pt]{\xymatrix @=#1@M=0pt{ #2 }}
\newcommand{\pointto}[1]{\ar@/_/[#1]}
\newcommand{\apointto}[1]{\ar@/_1pc/[#1]}
\newcommand{\oview}[1]{\llcorner #1 \lrcorner}
\newcommand{\pview}[1]{\ulcorner #1 \urcorner}


%%% Safe lambda calculus
\newcommand\funto{\longrightarrow}
\newcommand\rank[1]{{\sf rank}(#1)}
\newcommand\order[1]{{\sf order}(#1)}
\newcommand\slheight[1]{{\sf height}(#1)}
\newcommand\nparam[1]{{\sf nparam}(#1)}
\newcommand\nat{\mathbb{N}}

\newcommand{\typear}{\rightarrow}
\newcommand{\rulename}[1]{\mathbf{(#1)}}


\newcommand\textbfit[1]{{\bf\em #1}\index{#1}}
\newcommand\blambda{\hbox{\boldmath $\lambda$}}
\newcommand\lterm[2]{{\blambda{#1}.{#2}}}
\newcommand\terms[2]{{\cal T}^{#1}(#2)}
\newcommand{\funsp}{\rightarrow}
\newcommand\level[1]{{\sf level}(#1)}
\newcommand\seq[2]{{{#1} \vdash {#2}}}


% model checking
\newcommand\entail{\vdash}

% todo symbol
\newcommand\todo{\textdbend}
\newcommand\todomargin[1]{\marginpar{\textdbend #1}}
\newcommand\todobox[1]{\colorbox{lightgray}{\parbox[h]{0.9\textwidth}{#1}}\marginpar[\textdbend]{\textdbend}}
