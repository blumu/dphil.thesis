% justified sequence of move
\def\justseq#1{\xymatrix @=12pt@M=0pt{ #1 }}
\def\pointto#1{\ar@/_/[#1]}

\def\oview#1{\llcorner #1 \lrcorner}
\def\seq#1{\langle #1 \rangle}

First note that the well-bracketing condition allows us to recover
the pointers for all the P answer moves (simply make the answers point
to the last pending question with the same tag).

For any justified sequence of moves $u$, we write $?(u)$ for the subsequence of $u$ consisting of the questions in the sequence $u$ that are still pending at the end of the sequence.

Let $L$ be the following language $L = \{ q'_i q_i\ | \ i \in I_1 \}$.
We consider the following cases:

\begin{center}
\begin{tabular}{c|c|l|l}
Case & $\lambda_{OP}(m)$ & $?(u) \in$ & condition \\ \hline
0 & O & $\{ \epsilon \}$ \\
A & P & $q$ \\
B & O & $q \cdot L^* \cdot q'_i$     & $i \in I_1$ \\
C & P & $q \cdot L^* \cdot q'_i q_i$ & $i \in I_1$ \\
D & O & $q \cdot L^* \cdot q_i$      & $i \in I_0$ \\
\end{tabular}
\end{center}

We use the notation $\hat{s}$ to denote a legal and well-bracketed \emph{justified} sequence of moves and
$s$ to denote the same sequence of moves with pointers removed.

We prove by induction on the sequence of moves $u$ that $u$ corresponds to either
case 0, A, B, C or D and that the pointers in $u$ can be recovered uniquely.

\textbf{Base cases:}

If $u$ is the empty sequence $\epsilon$ then there is no pointer to recover and it corresponds to case 0.

If $u$ is a singleton then it must be the initial question $q$ and there is not pointer to recover. This corresponds
to case A.

\textbf{Step case:}

Consider a legal well-bracketed justified sequence $\hat{s}$ where $s = u \cdot m$ and $m \in M_A$.
The induction hypothesis tells us
that the pointers of $u$ can be recovered (and therefore the P-view or O-view at that point can be
computed) and that $u$ corresponds to one of the cases 0,A,B,C or D.

We proceed by case analysis depending on the case $u$ corresponds to:

\begin{description}

\item[case 0] This case cannot happen because $?(u) = \epsilon$ ($u$ is a complete play) implies that there cannot be any further move $m$.

Indeed the visibility condition implies that $m$ must point to
a P-question in the O-view at that point. But since $u$ is a complete play, the O-view is $\oview{\hat{u}} = q a$ which
does not contain any P-question. Hence the move $m$ cannot be justified and is not valid.


\item[case A] $?(u) = q$ and the last move $m$ is played by P.
    There are several cases:
    \begin{itemize}
    \item $m = a \in \natbf$ is an answer to the initial question
    $q$, then $m$ points to $q$:

    $\hat{s} = \xymatrix @=12pt@M=0pt{ q & \ldots & m \ar@/_/[ll]}$

    and $?(s) = \epsilon$ therefore $s$ correspond to the case 0 (complete play).

    \item $m = q_i$ where $q_i$ is an order 0 question ($i \in I_0$).
    Then $q_i$ points to the initial question $q$ and $s$ falls into category D.

    \item $m = q'_i$, a first order question, then $q'_i$ points to $q$,

    $?(s)= q q'_i$ and it is O's turn after $s$ therefore $s$ falls into category B.

    \end{itemize}


\item[case B] $?(u) \in q \cdot L^* \cdot q'_i$ where $i \in I_1$ and O plays the move $m$.

We now analyse the different possible O-moves:
\begin{itemize}
\item Suppose that O gives the (tagged) answer $a'_j$ for some $j \in I_1$ then
the visibility condition constraints it to point to a question in the O-view at that point.

We remark that the last move in $\hat{u}$ must be $q'_i$.
Indeed, suppose that there is a move $x \in M_A$ such that $\hat{u} = \justseq{q & \ldots & q'_i\ x \pointto{ll}}$
then by visibility, the O-move $x$ should points to a move in the O-view a that point. The O-view is $q q'_i$, therefore $x$ can only points to
$q'_i$. But then, $q'_i$ is not a pending question in $s$ which is a contradiction.


Therefore $\oview{\hat{u}} = \oview{ \justseq{ q & \ldots & q'_i \pointto{ll}} } = q q'_i$.

Hence $a'_j$ can only point to $q'_i$ (and therefore $i=j$).

We then have $?(s) = ?(u \cdot a'_i) \in  q \cdot L^*$ which is covered by case A and C.

\item The only other possible O-move is $q_i$ which, again by the visibility condition, points necessarily
to the previous move $q'_i$.
We then have $?(s) = ?(u \cdot q_i) \in   q \cdot L^* \cdot q'_i q_i$. This falls into category C.

\end{itemize}

\item[case C] $?(u) \in q \cdot L^* \cdot q'_i q_i$ where $i \in I_1$ and the move $m$ is played by $P$.

Suppose $m$ is an answer, then the well-bracketing condition imposes to answer to $q_i$ first.
The move $m$ is therefore an integer $a_i$ pointing to $q_i$.
We then have $?(s) = ?(u \cdot a_i) \in  q \cdot L^* \cdot q'_i$. This correspond to case B.


Suppose $m$ is a question then there are two cases:
\begin{itemize}
\item $m = q_j$ with $j \in I_0$, the pointer goes to the initial question $q$ and $s$ falls into category D.
\item $m = q'_j$ with $j \in I_1$, the pointer goes to the initial question $q$ and $s$ falls into category B.
\end{itemize}

\item[case D] $?(u) \in q \cdot L^* \cdot q_i$ where $i \in I_0$ and the move $m$ is played by $O$.

    The same argument as in case B holds. However there is now another possible move:
    the answer $m = a_i \in \natbf$.  This moves can only points to
    $q_i$ (this is the only pending question tagged by $i \in I_0$).

    Then $?(\hat{s}) = ?(\hat{u}\cdot a_i) = ?(\xymatrix @=12pt@M=0pt{ q & \ldots & q_i \ar@/_/[ll] & \ldots & a_i \ar@/_/[ll]}) \in q \cdot L^* $ therefore $s$ falls either into category A or C.

\end{description}
