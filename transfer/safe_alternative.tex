\section{Homogeneous safe $\lambda$-calculus}
\label{sec:safe_alt}

We recall the definition of the safe $\lambda$-calculus given in
\cite{Ong2005}.

\subsection{Rules}

These rules are a corrected version of
\cite{DBLP:conf/fossacs/AehligMO05}

In the following we shall consider terms-in-context $\seq{\Gamma}{M
: A}$ of the simply-typed $\lambda$-calculus. Let $\Delta$ be a
simply-typed alphabet i.e., each symbol in $\Delta$ has a simple
type. We write $\terms{A}{\Delta}$ for the set of terms of type $A$
built up from the set $\Delta$ understood as constant symbols,
\emph{without} using $\lambda$-abstraction.


The \textbfit{Safe $\lambda$-Calculus} is a sub-system of the
simply-typed $\lambda$-calculus. Typing judgements (or
terms-in-context) are of the form
\begin{equation}
\nonumber \seq{\overline{x_1}:\overline{A_1} \, | \, \cdots \, | \,
\overline{x_n} :  \overline{A_n}}{M : B}
\end{equation}
which is shorthand for $\seq{x_{11} : A_{11}, \cdots, x_{1r}:
A_{1r}, \cdots}{M : B}$. \emph{Valid typing judgements} of the
system are defined by induction over the following rules, where
$\Delta$ is a given homogeneously-typed alphabet:

$$ \rulename{wk}
    {   \rulef{ \seq{\Sigma}{M:B} \qquad {\Sigma \subset \Delta} }
             { \seq{\Delta }{M : B}}
   }
\qquad
    \rulename{perm}
    {
      \rulef { \seq{\Gamma}{M:B} \qquad \sigma(\Gamma) \hbox{ homogeneous} }
            { \seq{\sigma(\Gamma)}{M : B} }
    }
$$


$$ \rulename{\Sigma\mbox{\textbf{-const}}}  \rulef{b : o^r \rightarrow o \in \Sigma} {\seq{}{b : o^r \rightarrow o}}
\qquad
 \rulename{var} \rulef{}{\seq{x_{ij} : A_{ij}\, }{x_{ij} : A_{ij}}}
$$

$$ \rulename{\lambda\mbox{\textbf{-abs}}}
\rulef{ {\seq{\overline{x_1} : \overline{A_1}\, | \, \cdots\, | \,
\overline{x_{n+1}} : \overline{A_{n+1}}}{M : B}} \qquad
\ord{\overline{A_{n+1}}} \geq \ord{B} -1}
    {\seq{\overline{x_1} :
\overline{A_1}\, | \, \cdots\, | \, \overline{x_{n}} :
\overline{A_{n}}}{\lterm{\overline{x_{n+1}} : \overline{A_{n+1}}}{M}
: (\overline{A_{n+1}} \, | \, B)}} $$

$$ \rulename{app} \rulef{{\seq{\Gamma}{M : (\overline{B_1} \, | \, \cdots \, | \, \overline{B_m} \, | \, o)} \qquad
\seq{\Gamma}{N_1 : B_{11}} \quad \cdots \quad \seq{\Gamma}{N_{l} :
B_{1l}} \qquad l = |\overline{B_1}| }}
    { \seq{\Gamma}{M N_1
\cdots N_{l} : (\overline{B_2} \, | \, \cdots \, | \,
\overline{B_m} \, | \, o)}} $$


$$ \rulename{app^+} \rulef
    {\seq{\Gamma}{M : (\overline{B_1} \, | \, \cdots \, | \, \overline{B_m} \, | \, o)} \qquad
    \seq{\Gamma}{N_1 : B_{11}} \quad \cdots \quad \seq{\Gamma}{N_{l} :
    B_{1l}} \qquad l < |\overline{B_1}| }
    { \seq{\Gamma}{M N_1
    \cdots N_{l} : (\overline{B} \, | \, \cdots \, | \,
    \overline{B_m} \, | \, o)}} $$

where $\overline{B_1} = B_{11}, \ldots, B_{1l},\overline{B}$ with
the condition that every variable in $\Gamma$ has an order strictly greater
than $\ord{\overline{B_1}}$.


\begin{lem}[Basic properties]
\label{lem:safe_basic_prop} Suppose $\Gamma \vdash M : B$ is a
valid judgment then
\begin{itemize}
\item[(i)] $B$ is homogeneous
\item[(ii)] Every free variables of $M$ has order at least $\ord{M}$
\end{itemize}
\end{lem}

The following corollary is a consequence of the second point in the previous lemma:
\begin{property}
If $\Gamma \vdash M : B$ then $fv(M) \vdash M : B$ where $fv(M) \subseteq \Gamma$ denotes the context which variables are
are exactly the free variables of $M$.
\end{property}

We now define a special kind of substitution that performs simultaneous substitution and
that permits variable capture (i.e. does not rename variables when the substitution is performed on an abstraction).

\begin{dfn}[Capture permitting simultaneous substitution (for homogeneous safe terms)]
\label{dnf:safe_simsubst}
 We use the notation $\subst{\overline{N}}{\overline{x}}$ for $\subst{N_1 \ldots N_n}{x_1 \ldots x_n}$ and
$\overline{y}:\overline{A}$ for $y_1:A_1, \ldots, y_p:A_p$.
A safe term must have one of the forms occurring on the left-hand side of the following equations, where
the terms $M$, $N_1, \ldots N_l$ are safe terms:
\begin{eqnarray*}
c \subst{\overline{N}}{\overline{x}} &=& c \quad \mbox{ where $c$ is a $\Sigma$-constant}\\
x_i \subst{\overline{N}}{\overline{x}} &=& N_i\\
 y \subst{\overline{N}}{\overline{x}} &=& y \quad \mbox{ if } y \not \neq x_i \mbox{ for all } i,\\
(M N_1 \ldots N_l) \subst{\overline{N}}{\overline{x}} &=& (M \subst{\overline{N}}{\overline{x}}) (N_1 \subst{\overline{N}}{\overline{x}}) \ldots  (N_l \subst{\overline{N}}{\overline{x}})\\
(\lambda \overline{y} : \overline{A}. M)
\subst{\overline{N}}{\overline{x}} &=& \lambda \overline{y} . M
\subst{\overline{N} \upharpoonright I}{\overline{x} \upharpoonright I} \\
&& \mbox{where } I  = \{ i \in 1..n \ | \ x_i \not \in \overline{y} \}
\end{eqnarray*}

where $ \upharpoonright$ is the index filtering operator: if $s$ is a sequence and $I$ a set of indices then
$s \upharpoonright I$ is the subsequence of $s$ obtained by removing from $s$ all the elements
at a position that is not in $I$.
\end{dfn}

We now prove that in the homogeneous safe $\lambda$-calculus, variable capture never occurs. Therefore
the capture permitting substitution that we use is equivalent to the traditional substitution (where an unbound number of
variable names are required to avoid capture).

\begin{lem}[No variable capture lemma]
In the safe $\lambda$-calculus, there is no capture of variable
when performing the following capture permitting simultaneous substitution:
$$ M[N_1 / x_1 , \cdots, N_n / x_n] $$
where $\Gamma,\overline{x} \vdash M$, $\Gamma \vdash  N_1, \ldots \Gamma \vdash  N_n$.

% provided the substitution is performed simultaneously on
%\emph{all} free variables of the same order in $M$
%i.e.~$\makeset{x_1, \cdots, x_n}$ is the set variables of the same
%order as $x_1$ occurring freely in $M$.
\end{lem}

\begin{proof}
We prove the result by induction. The variable, constant and application cases are trivial.
For the abstraction case, suppose $M = \lambda \overline{y} : \overline{A}. P$ where $\overline{y} = y_1 \ldots y_p$. The capture permitting
simultaneous substitution gives:
$$M \subst{\overline{N}}{\overline{x}} = \lambda \overline{y} . P
\subst{\overline{N} \upharpoonright I}{\overline{x} \upharpoonright I}$$

%The variables $\overline{x}$ all occurs freely in $M$ therefore
%the variable in $\overline{x}$ cannot be abstracted under the $\lambda \overline{y}$. Hence $I = \emptyset$
%and:
%$$M \subst{\overline{N}}{\overline{x}} = \lambda \overline{y} . P
%\subst{\overline{N}}{\overline{x}}$$
%By the induction hypothesis there is no variable capture in $P\subst{\overline{N}}{\overline{x}}$.

By the induction hypothesis there is no variable capture in $P
\subst{\overline{N} \upharpoonright I}{\overline{x} \upharpoonright I}$.

Hence the only possible case of variable capture is when for some $i \in I$ and $j \in 1..p$ the variable $y_j$ occurs freely
in $N_i$ and $x_i$ occurs freely in $P$. In that case, lemma \ref{lem:safe_basic_prop} (ii) gives:
$$ \ord{y_j} \geq \ord{N_i} = \ord{x_i}$$

Moreover since $x_i$ occurs freely in $P$ and $i\in I$, $x_i$ must occur freely also in the safe term $\lambda \overline{y}. P$
and lemma \ref{lem:safe_basic_prop} (ii) gives:
$$ \ord{x_i} \geq \ord{\lambda y_1 \ldots y_p . T} \geq 1+ \ord{y_j} > \ord{y_j}$$

Hence we reach a contradiction.
\end{proof}


\subsection{Safe $\beta$-reduction}

We now introduce the notion of safe $\beta$-redex and show how such redex can be reduced using the
capture-permitting simultaneous substitution. We will then show that
a safe $\beta$-reduction reduces to a safe term.


In the simply-typed lambda calculus a redex is a term of the form $(\lambda x . M) N$.
We generalize this notion to the safe lambda calculus. We call multi-redex a term of the form
$(\lambda x_1 \ldots x_n . M) N_1 \ldots N_l$ (it is not required to have $n=l$).


We say that a multi-redex is safe if it respects the formation rules of the safe $\lambda$-calculus. More precisely,
the multi-redex $(\lambda x_1 \ldots x_n . M) N_1 \ldots N_l$ is a safe redex if the variable $x_1,\ldots,x_n$
are abstracted altogether at once using the abstraction rule and if the terms $N_1 \ldots N_l$ are applied to the
term $\lambda x_1 \ldots x_n . M$ at once using either the application rule $\rulename{app^+}$ or $\rulename{app}$.

Note that there exist safe terms of the form $(\lambda x_1 \ldots x_n . M) N_1 \ldots N_l$
such that $l>n$. For instance the following term:
$$ (\lambda f g . ((\lambda h i . i) a) ) (\lambda x.x) (\lambda x.x) (\lambda x.x)$$
where is $a$ constant, $x : o$ and $f,g,h,i,a:o \rightarrow o$ can be formed using the $\rulename{app}$ rule as follows:
$$ \rulef{
    \emptyset \vdash (\lambda f g . ((\lambda h i . i) a) )
        \quad \emptyset \vdash (\lambda x.x)
        \quad \emptyset \vdash (\lambda x.x)
        \quad \emptyset \vdash (\lambda x.x)
    }
    {
       \emptyset \vdash (\lambda f g . ((\lambda h i . i) a) ) (\lambda x.x) (\lambda x.x) (\lambda x.x)
    } \rulename{app}
$$


The formal definition follows:

\begin{dfn}[Safe redex]
A safe redex is a term having one of the following two forms:

\begin{itemize}
    \item $(\lambda \overline{x} . M) N_1 \ldots N_l$ formed with the $\rulename{app}$ rule such that
        \begin{itemize}
        \item the variable $\overline{x}=x_1\ldots x_n$ are abstracted altogether by one occurrence of the rule $\rulename{abs}$ in the proof tree.
                Consequently:
                $$\ord{M} \leq \ord{\overline{x}} = \ord{x_1} = \ldots = \ord{x_n}$$

        \item The terms $(\lambda \overline{x} . M)$, $N_1$, $N_l$ are applied together at once using the rule $\rulename{app}$:
        $$   \rulef{
                    \Sigma \vdash \lambda \overline{x} . M : (\overline{B_1}|\ldots|\overline{B_m}|o)
                    \quad
                    \Sigma \vdash N_1         \quad \ldots \quad \Sigma \vdash N_l
                    \quad l = |\overline{B_1}|
            }
            {
            \Sigma \vdash (\lambda \overline{x} . L) N_1 \ldots N_l
            } (\mathbf{app})
        $$

        \item $n \leq l$. This is because the variables $x_1, \ldots, x_n$ all belong to the lowest partition $\overline{B_1}$ hence $n\leq |\overline{B_1}| = l$.

        \end{itemize}

\item $(\lambda \overline{x} . M) N_1 \ldots N_l$ formed with the $\rulename{app^+}$ rule such that:

        \begin{itemize}
        \item the variable $\overline{x}=x_1\ldots x_n$ are abstracted altogether by one occurrence of the rule $\rulename{abs}$ in the proof tree.
                Consequently  $\ord{M} \leq \ord{\overline{x}} = \ord{x_1} = \ldots = \ord{x_n}$

        \item The terms $(\lambda \overline{x} . M)$, $N_1$, $N_l$ are applied together at once using the rule $\rulename{app^+}$:
        $$   \rulef{
                    \Sigma \vdash \lambda \overline{x} . M : (\overline{B_1}|\ldots|\overline{B_m}|o)
                    \quad
                    \Sigma \vdash N_1         \quad \ldots \quad \Sigma \vdash N_l
                    \quad l < |\overline{B_1}|
            }
            {
            \Sigma \vdash (\lambda \overline{x} . L) N_1 \ldots N_l
            } (\mathbf{app^+})
        $$

      \item $n \leq |\overline{B_1}|$ (we do not necessarily have $n = |\overline{B_1}|$).

        \end{itemize}

\end{itemize}
\end{dfn}


\begin{dfn}[Safe reduction $\beta_s$] \
For concision the following abbreviations are used $\overline{x} = x_1 \ldots
x_n$, $\overline{N} = N_1 \ldots N_l$, and when $n\geq l$, $\overline{x_l} = x_1 \ldots
x_l$, $\overline{x_r} = x_{l+1} \ldots x_n$.
\begin{itemize}
\item The relation $\beta_s$ is defined on the set of safe redex as follows:
\begin{eqnarray*}
\beta_s &=&
\{  \ (\lambda \overline{x} : \overline{A} . T) N_1 \ldots N_l \mapsto \lambda \overline{x_r}. T\subst{\overline{N}}{\overline{x_l}}  \\
&& \mbox{ where $(\lambda \overline{x} : \overline{A} . T) N_1 \ldots N_l$ is a safe redex such that $n> l$}
\} \\
&\union&
\{ \ (\lambda \overline{x} : \overline{A} . T) N_1 \ldots N_l \mapsto T\subst{\overline{N}}{\overline{x}} N_{n+1} \ldots N_l  \\
&& \mbox{ where $(\lambda \overline{x} : \overline{A} . T) N_1 \ldots N_l$ is a safe redex such that $n\leq l$}
\}
\end{eqnarray*}
where the notation $\subst{\overline{N}}{\overline{x}}$ denotes the capture-permitting simultaneous substitution.

\item
The safe $\beta$-reduction noted $\betasred$ is the closure
of the relation $\beta_s$ by compatibility with the formation rules
of the safe $\lambda$-calculus.
\end{itemize}
\end{dfn}



We observe that safe $\beta$-reduction is a certain kind of multi-steps $\beta$-reduction.
\begin{property}
$\betasred \subset \betaredtr$, i.e. the safe
$\beta$-reduction relation is included in the transitive closure of the $\beta$-reduction relation.
\end{property}
\begin{proof}
Suppose that $(M\mapsto N) \in \beta_s$. We show that $M \betared^* N$.
\begin{itemize}
\item Suppose that the safe-redex is
$M \equiv (\lambda \overline{x} : \overline{A} . T) N_1 \ldots N_l$ such that $n\leq l$ then:
\begin{eqnarray*}
 M &=_\alpha& (\lambda z_1 \ldots z_n .T [z_1,\ldots z_n /x_1,\ldots x_n] ) \ N_1  N_2 \ldots N_l
            \\
&& \mbox{where the $z_i$ are fresh variables}  \\
     &\betared& (\lambda z_2 \ldots z_n .T [z_1,\ldots z_n /x_1,\ldots x_n] \subst{N_1}{z_1} ) \ N_2 \ldots N_l \\
&& \mbox{ (because the $z_i$ do not occur freely in $N_1$) }\\
%%    &=_\alpha& (\lambda z_2 \ldots z_n .T [z_2,\ldots z_n /x_2,\ldots x_n] \subst{N_1}{x_1})\  N_2 \ldots N_l  \qquad \mbox{where the $z_i$ are fresh variables}  \\
    &\betared& \ldots \\
    &\betared& (T [z_1,\ldots z_n /x_1,\ldots x_n] \subst{N_1}{z_1}  \ldots \subst{N_n}{z_n})\  N_{n+1} \ldots N_l \\
    &\betared& (T [N_1\ldots N_l/x_1,\ldots x_l])\ N_{n+1} \ldots N_l
\end{eqnarray*}
And since $T$ is safe, the substitution $T [N_1\ldots N_l/x_1,\ldots x_l]$ in the last equation
can be performed using the capture-permitting substitution. Hence $M \betared^* N$.

\item
 Suppose that $M \equiv (\lambda \overline{x} : \overline{A} . T) N_1 \ldots N_l$ such that $n> l$, then necessarily
the redex must be formed using the $\rulename{app^+}$ rule. The side-condition of this rules
says that the free variables of the terms $N_1, \ldots N_l$ have all order strictly greater than $\ord{\overline{x}}$,
hence the $x_i$ do not occur freely in $N_1, \ldots N_l$. Therefore:

\begin{eqnarray*}
 M &=& (\lambda x_1 \ldots x_n .T) \ N_1  N_2 \ldots N_l  \\
     &\betared& (\lambda x_2 \ldots x_n .T \subst{N_1}{x_1} ) \ N_2 \ldots N_l \\
            && \mbox{(for $i \in 2..n$, $x_i$ does not occur freely in $N_1$)}\\
    &\betared& \ldots \\
    &\betared& \lambda x_{l+1} \ldots x_n . T \subst{N_1}{x_1}  \ldots \subst{N_l}{x_l} \\
        && \mbox{(for $i \in (l+1)..n$,  $x_i$ does not occur freely in $N_l$)}\\
    &\betared& \lambda x_{l+1} \ldots x_n . T [N_1\ldots, N_l /  \ x_1,\ldots, x_l] \\
        && \mbox{(the $x_i$ do not occur freely in $N_1, \ldots N_l$)}
\end{eqnarray*}
And since $T$ is safe, the substitution $T [N_1\ldots N_l/x_1,\ldots x_l]$ in the last equation
can be performed using the capture-permitting substitution. Hence $M \betared^* N$.
\end{itemize}
\end{proof}

\begin{property} In the simply typed $\lambda$-calculus setting:
\begin{enumerate}
\item $\betasred$ is strongly normalizing.
\item $\beta_s$ has the unique normal form property.
\item $\beta_s$ has the Church-Rosser property.
\end{enumerate}
\end{property}

\begin{proof}
1. This is because $\betasred \subset \betaredtr$ and $\betared$ is strongly normalizing (in the simply typed lambda calculus).
2. A term has a safe redex iff it has a $\beta$-redex therefore
the set of $\beta_s$ normal form is equal to the set of $\beta_s$
normal form. Hence, the unicity of $\beta$ normal form implies the
unicity of $\beta_s$ normal form.
3. is a consequence of 1 and 2.
\end{proof}



\begin{lem}[Capture-permitting simultaneous substitution preserves safety]
\label{lem:subst_preserve_safety}
Let $\Gamma, \overline{x} \vdash M$ be a safe term such that $\overline{x}$ is the lowest partition of the environment.

Then for any safe terms safe terms $\Gamma \vdash N_1$, \ldots, $\Gamma \vdash N_n$,
the capture permitting simultaneous substitution
$$ \Gamma \vdash M[N_1 / x_1 , \cdots, N_n / x_n] $$
is safe.
\end{lem}
\begin{proof}
An easy proof by an induction similar to the proof of the previous lemma.
\end{proof}

since capture-permitting simultaneous substitution preserves safety, a direct consequence is that
a safe redex $(\lambda \overline{x} . M) N_1 \ldots N_l$ reduces to a safe term:

\begin{lem}[The safe reduction $\beta_s$ preserves safety] If $M$ is safe and $M \betasred N$ then $N$ is safe.
\end{lem}

\begin{proof}
It suffices to show that the relation $\beta_s$ preserves safety.
Consider the safe-redex $(s\mapsto t) \in \beta_s$ where $ s \equiv (\lambda x_1 \ldots x_n . M) N_1 \ldots N_l $ .
We proceed by case analysis on the the last rule used to form the redex.
\begin{itemize}
\item Suppose the last rules used is $\rulename{app}$, then necessarily $n\leq l$ and the reduction is:
$$(\lambda x_1 \ldots x_n . M) N_1 \ldots N_l \qquad \mapsto  \qquad t \equiv M[N_1 / x_1 , \cdots, N_n / x_n]\ N_{n+1} \ldots N_l$$
where $\ord{M} \leq \ord{x_1} = \ldots = \ord{x_n}$.

The first premise of the rule $\rulename{app}$ tells us that $M$ is safe and that $\overline{x}$ is the lowest partition of the context.
Therefore by lemma \ref{lem:subst_preserve_safety} and using the application rule we obtain that $t$ is safe.

\item Suppose the last rules used is $\rulename{app^+}$ and $n> l$ then the reduction is
$$ (\lambda \overline{x_l} : \overline{A_l} \
\overline{x_r}: \overline{A_r} . T) \overline{N_l} \qquad \mapsto
\qquad t \equiv \lambda \overline{x_r}: \overline{A_r} .
T\subst{\overline{x_l}}{\overline{N_l}}
$$
where $\ord{T} \leq \ord{x_1} = \ldots = \ord{x_n}$

By lemma \ref{lem:subst_preserve_safety}, $T\subst{\overline{x_l}}{\overline{N_l}}$ is a safe term.
Using the rule $\rulename{abs}$ we derive that $t$ is safe.

\item Suppose the last rules used is $\rulename{app^+}$ and $n\leq l$ then the reduction is
$$(\lambda x_1 \ldots x_n . M) N_1 \ldots N_l \qquad \mapsto \qquad t \equiv M[N_1 / x_1 , \cdots, N_n / x_n]\ N_{n+1} \ldots N_l$$
where $\ord{M} \leq \ord{x_1} = \ldots = \ord{x_n}$. The treatment of this case is identical to case $\rulename{app}$.

\item Rule $\rulename{wk}$ $\rulename{seq}$: these cases lead back to the previous cases.
\end{itemize}
\end{proof}


\begin{rem}
\label{rem:betasred_notpreserv_unsafety} $\betasred$ preserves
safety but \emph{does not} preserves un-safety: given two terms of
the same type $S$ safe and $U$ unsafe, the term $(\lambda x y . y) U S$ is also unsafe
but it $\beta_s$-reduces to $S$ which is safe.
\end{rem}


\subsection{An alternative system of rules}


In this section, we will refine the formation rules
given in the previous section. We say that $\Gamma \vdash M : A$ verifies $P_i$ if all the
variables in $\Gamma$ have orders at least $\ord{A}+i$ and we introduce the notation $\Gamma \vdash^{i} M : A$ for $i \in
\zset$ to mean that $\Gamma \vdash M : A$ and $\Gamma \vdash M : A$
satisfies $P_i$.


The following lemma says that if $\Gamma \vdash M : A$ then variable in the context $\Gamma$ with order
strictly smaller than $M$ do not occur freely in $M$ and therefore the context can be restricted to a smaller number of variables.

\begin{lem}[Context reduction]
\label{lem:restriction}

Suppose that $\Gamma \vdash M : A$ satisfies $P_i$ with $i\leq0$, then we have
$\Gamma' \vdash^{0} M : A$ where $$\Gamma' = \{ z \in \Gamma \ |
\ \ord{z} \geq \ord{M} \} = \Gamma \setminus \{ z \in \Gamma \ | \ \ord{M} + i \leq \ord{z} < \ord{M} \}$$
\end{lem}
\begin{proof}
By induction, the only non trivial cases are $\rulename{app}$ and $\rulename{app^+}$:
\begin{itemize}
\item Case of the rule $\rulename{app}$:

    \[ (\mathbf{app})
    \rulef
        {\seq{\Gamma}{M : (\overline{B_1} \, | \, \cdots \, | \, \overline{B_m} \, | \, o)} \qquad
            \seq{\Gamma}{N_1 : B_{11}} \quad \cdots \quad \seq{\Gamma}{N_{l} :
            B_{1l}} \qquad l = |\overline{B_1}| }
        { \seq{\Gamma}{M N_1
            \cdots N_{l} : (\overline{B_2} \, | \, \cdots \, | \,
            \overline{B_m} \, | \, o)}}
    \]

    If the conclusion verifies $P_i$ then, for all $z \in \Gamma$:
    \begin{eqnarray*}
    \ord{z} \geq 1 + \ord{\overline{B_2}} + i
    &=& 1 + \ord{\overline{B_1}} + \ord{\overline{B_2}} - \ord{\overline{B_1}} + i \\
    &=& \ord{M} + (\ord{\overline{B_2}} - \ord{\overline{B_1}} + i)
    \end{eqnarray*}
    Therefore the first premise satisfies $P_j$ where $j={\ord{\overline{B_2}} - \ord{\overline{B_1}} + i}$

    and $j< 0$ because of the homogeneity of the type. Hence by the induction hypothesis,
    there is a context $$\Gamma' = \{ z \in \Gamma \ |
    \ \ord{z} \geq \ord{M} \} = \Gamma \setminus \{ z \in \Gamma \ | \ \ord{M} + j \leq \ord{z} < \ord{M} \}$$
    such that $\Gamma' \vdash^{0} M : (\overline{B_1} \, | \, \cdots \, | \, \overline{B_m} \, | \, o)$.


    Similarly:
    \begin{eqnarray*}
    \ord{z} \geq 1 + \ord{\overline{B_2}} + i
    &=& \ord{\overline{B_1}} + (1+\ord{\overline{B_2}} - \ord{\overline{B_1}} + i) \\
    &=& \ord{\overline{B_1}} + j+1
    \end{eqnarray*}

    where $j+1\leq 0$, hence by the induction hypothesis for $k : 1..l$ there is a
    context $$\Gamma'' = \{ z \in \Gamma \ |
    \ \ord{z} \geq \ord{N_k} \} = \Gamma \setminus \{ z \in \Gamma \ | \ \ord{M} + j+1 \leq \ord{z} < \ord{M} \}$$
    such that $\Gamma'' \vdash N_k : B_{1k}$ verifies $P_0$.

    But since $\Gamma' = \Gamma'' \union \{ z \in \Gamma \ | \ \ord{M} + j = \ord{z}\}$, we have for $k : 1..l$:

    $\Gamma' \vdash N_k : B_{1k}$ satisfies $P_{-1}$.


    By applying  the $\rulename{app}$ rule we obtain:
    $$\Gamma' \vdash M N_1 \ldots N_l : (\overline{B_2} \, | \, \cdots \, | \,
    \overline{B_m} \, | \, o)$$
    where for all $z\in \Gamma'$:
    \begin{eqnarray*}
    \ord{z} \geq 1 + \ord{\overline{B_1}}
    &>& 1 + \ord{\overline{B_2}} = \ord{M N_1 \ldots N_l}
    \end{eqnarray*}

\item $\rulename{app^+}$  The side-condition of the rule $\rulename{app^+}$ ensures that all the premises
 verify $P_0$. The conclusion of the rule has the same order as the first premise
 therefore the conclusion verifies $P_0$.
\end{itemize}
\end{proof}


The following lemma is a direct consequence of the context restriction lemma:
\begin{lem}
\label{lem:prooftree01only}
If $\Gamma \vdash^{0} M : T$ then there is valid proof tree
showing that $\Gamma \vdash M : T$ such that all the judgments
of the proof tree verify $P_0$ or $P_{-1}$.
\end{lem}


\begin{proof}
Suppose that $\Gamma \vdash M : T$ such that it satisfies $P_{-1}$.

We show that there is a proof tree for
$\Gamma \vdash M : T$ where all the nodes of the tree verify $P_0$
or $P_{-1}$ by case analysis on the last rule used to prove $\Gamma \vdash M : T$ .

\begin{itemize}
\item $\rulename{wk}$ If $\Delta \vdash M : T$ verifies $P_{-1}$ then in particular $\Gamma
\vdash M : T$ verifies $P_{-1}$ for any $\Gamma \subset \Delta$.

\item $\rulename{perm}$ By the induction hypothesis.

\item $\rulename{\Sigma\mbox{\textbf{-const}}}$ the context is empty therefore the sequent verifies $P_{-1}$.

\item $\rulename{var}$ the context contains only the variable itself : verifies $P_0$.

\item $\rulename{abs}$ the second premise of the rule guarantees that the first
premise verifies $P_{-1}$.

\item $\rulename{app^+}$ The first premise has the same order has the
conclusion of the rule therefore the first premise verifies
$P_0$. The side-condition of the rule $\rulename{app^+}$ ensures that the other premises verify $P_0$.

\item $\rulename{app}$ Using lemma \ref{lem:restriction}, the induction hypothesis and the rule $\rulename{app}$.

\end{itemize}
\end{proof}

\subsubsection{Refining the rules of the homogeneous safe $\lambda$-calculus}

Using the observations that we have just made, we will now refine the rules
of the safe $\lambda$-calculus with homogeneous type. We would like to obtain a system
of rules generating sequents that verify $P_0$. Those sequents correspond to the ``safe'' terms.

The system of rules must be able to generate intermediate sequents that are then used to produce term
satisfying $P_0$. Because of the lemma \ref{lem:prooftree01only}, we know that the only necessary intermediate sequents
are those that either satisfy $P_0$ or $P_{-1}$. In other word, it is useless to produce sequents that do not satisfy
$P_0$ or $P_{-1}$. We will therefore use the notation $\vdash^0$ and $\vdash^{-1}$ instead of $\vdash$ to precise
whether a given sequent satisfies $P_0$ or $P_{-1}$.

The first refinement consist in constraining the weakening rule so that
it only permits the addition of variable having an order big enough:
$$ \rulename{wk^{0}} \quad  \rulef{\Gamma \vdash^{0} M : A}{\Gamma , x : B \vdash^{0} M : A} \quad \ord{B} \geq \ord{A} $$
$$ \rulename{wk^{-1}} \quad  \rulef{\Gamma \vdash^{-1} M : A}{\Gamma , x : B \vdash^{-1} M : A} \quad \ord{B} \geq \ord{A} -1$$

There is also an additional rules expressing the fact that $P_0$ implies $P_{-1}$:

$$ \rulename{seq} \quad \rulef{\Gamma \vdash^{0} M : A}{\Gamma \vdash^{-1} M : A} $$


Because of the context reduction lemma, any sequent verifying $P_1$ can be obtained
by applying the weakening rule $\rulename{wk^{-1}}$ or the rule $\rulename{seq}$ on a sequent
verifying $P_0$. Therefore, with the exception these two rules, we can safely force all the rules to
have a conclusion sequent verifying $P_0$:
\begin{itemize}
\item  For the rules $\rulename{perm}$, $\rulename{const}$ and $\rulename{var}$, only the tagging of the sequent changes:

$$ \rulename{var} \quad  \rulef{}{x : A\vdash^{0} x : A}
\qquad
  \rulename{perm} \rulef{
      { \Gamma \vdash^0 M:B \qquad \sigma(\Gamma)  } \hbox{ homogeneous}
    }
      { \sigma(\Gamma) \vdash^0 M : B }
$$

$$ \rulename{const}
    \rulef{}{ \vdash^0 b : o^r \rightarrow o} \quad b : o^r \rightarrow o \in \Sigma
$$

\item  The previous definition of the abstraction rule has a side condition
expressing the fact that the premise verifies $P_0$ or $P_{-1}$. Since this is always true for sequents
generated by our new system of rules, we can drop the side condition:
$$ \rulename{abs} \quad  \rulef{\Gamma | \overline{x} : \overline{A} \vdash^{-1} M : B}
                                   {\Gamma  \vdash^{0} \lambda \overline{x} : \overline{A} . M : (\overline{A},B)}$$


\item The application rule $\rulename{app}$ has the following form:
$$ \rulename{app}
    \rulef{
        { \Gamma \vdash^{-1} M : (\overline{A} \, | B)
        \qquad
        \Gamma \vdash^? N_1 : A_1 \quad \cdots \quad \Gamma \vdash^? N_{l} : A_l \qquad l = |\overline{A}|
        }
    }
    {
        \Gamma \vdash^0 M N_1 \cdots N_{l} : B
    }
$$

Suppose that the sequent in the first premise verifies $P_{-1}$, then by Lemma \ref{lem:safe_basic_prop}(ii)
we have:
$$\forall z \in \Gamma : \ord{z} \geq 1 + \ord{\overline{A}} -1 = \ord{\overline{A}} = \ord{\overline{N}}$$
Hence, all the sequents of the premises but the first one verify $P_0$. The rule (app) is therefore given by:
$$ \rulename{app}
    \rulef{
        { \Gamma \vdash^{-1} M : (\overline{A} \, | B)
        \qquad
        \Gamma \vdash^0 N_1 : A_1 \quad \cdots \quad \Gamma \vdash^0 N_{l} : A_l \qquad l = |\overline{A}|
        }
    }{
        \Gamma \vdash^0 M N_1 \cdots N_{l} : B
      }
$$

\item For the application rule $\rulename{app^+}$, the type of the sequent in the first premise has the same order
as the type of the conclusion premises, therefore since the conclusion verifies $P_0$, the first premise also verifies $P_0$.
The side-condition implies that that all the other sequents in the premise verify $P_0$. Moreover the fact
that the first premise verifies $P_0$ ensure that the side-condition holds. Hence the rule becomes:
$$ \rulename{app^+}
    \rulef{
        \Gamma \vdash^0 M : (\overline{B_1} \, | \, \cdots \, | \, \overline{B_m} \, | \, o) \qquad
        \Gamma \vdash^0 N_1 : B_{11} \quad \cdots \quad \Gamma \vdash^0 N_{l} : B_{1l} \qquad l < |\overline{B_1}|
    }
    {
        \Gamma \vdash^0 M N_1 \cdots N_{l} : (\overline{B} \, | \, \cdots \, | \, \overline{B_m} \, | \, o)
    }
$$
where $\overline{B_1} = B_{11}, \ldots, B_{1l},\overline{B}$.
This rule can be equivalently stated as:
$$ \rulef{\Gamma \vdash^0 M : A\rightarrow B
                                        \qquad \Gamma \vdash^{0} N : A
                                   }
                                   {\Gamma  \vdash^{0} M N : B}$$
\end{itemize}

The full set of rules is given in table \ref{tab:homosafelmd_rules_refined}

\begin{table}[htbp]
$$  \rulename{perm}
    \rulef{
      { \Gamma \vdash^0 M:B \qquad \sigma(\Gamma)  } \hbox{ homogeneous}
    }
    { \sigma(\Gamma) \vdash^0 M : B
    }
\qquad
\rulename{seq} \quad \rulef{\Gamma \vdash^{0} M : A}{\Gamma \vdash^{-1} M : A}
$$

$$
 \rulename{const}
    \rulef{}
        { \vdash^0 b : o^r \rightarrow o} \quad b : o^r \rightarrow o \in \Sigma
\qquad
 \rulename{var} \quad  \rulef{}{x : A\vdash^{0} x : A} $$

$$ \rulename{wk^{0}} \quad  \rulef{\Gamma \vdash^{0} M : A}{\Gamma , x : B \vdash^{0} M : A} \quad \ord{B} \geq \ord{A} $$

$$ \rulename{wk^{-1}} \quad  \rulef{\Gamma \vdash^{-1} M : A}{\Gamma , x : B \vdash^{-1} M : A} \quad \ord{B} \geq \ord{A} -1$$


$$ \rulename{app}
    \rulef
        {   \Gamma \vdash^{-1} M : (\overline{A} \, | B)
            \qquad
            \Gamma \vdash^0 N_1 : A_1 \quad \cdots \quad \Gamma \vdash^0 N_{l} : A_l \qquad l = |\overline{A}|
        }
        {
            \Gamma \vdash^0 M N_1 \cdots N_{l} : B
        }
$$

$$ \rulename{app^+} \quad  \rulef{\Gamma \vdash^0 M : A\rightarrow B
                                        \qquad \Gamma \vdash^{0} N : A
                                   }
                                   {\Gamma  \vdash^{0} M N : B}$$

$$ \rulename{abs} \quad  \rulef{\Gamma| \overline{x} : \overline{A} \vdash^{-1} M : B}
                                   {\Gamma  \vdash^{0} \lambda \overline{x} : \overline{A} . M : (\overline{A}|B)}$$


where $\Gamma| \overline{x} : \overline{A}$ means that the lowest type-partition of the context is
$\overline{x} : \overline{A}$.
\caption{Alternative rules for the homogeneous safe lambda calculus}
\label{tab:homosafelmd_rules_refined}
\end{table}
%%%
