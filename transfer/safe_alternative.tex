\section{Safe $\lambda$-calculus - Another presentation}
\label{sec:safe_alt}

We recall the definition of the safe $\lambda$-calculus given in
\cite{Ong2005}.

\subsection{ Rules}

These rules are a corrected version of
\cite{DBLP:conf/fossacs/AehligMO05}

In the following we shall consider terms-in-context $\seq{\Gamma}{M
: A}$ of the simply-typed $\lambda$-calculus. Let $\Delta$ be a
simply-typed alphabet i.e., each symbol in $\Delta$ has a simple
type. We write $\terms{A}{\Delta}$ for the set of terms of type $A$
built up from the set $\Delta$ understood as constant symbols,
\emph{without} using $\lambda$-abstraction.


The \textbfit{Safe $\lambda$-Calculus} is a sub-system of the
simply-typed $\lambda$-calculus. Typing judgements (or
terms-in-context) are of the form
\begin{equation}
\nonumber \seq{\overline{x_1}:\overline{A_1} \, | \, \cdots \, | \,
\overline{x_n} :  \overline{A_n}}{M : B}
\end{equation}
which is shorthand for $\seq{x_{11} : A_{11}, \cdots, x_{1r}:
A_{1r}, \cdots}{M : B}$. \emph{Valid typing judgements} of the
system are defined by induction over the following rules, where
$\Delta$ is a given homogeneously-typed alphabet:

\[ {    { \seq{\Sigma}{M:B} \qquad {\Sigma \subset \Delta} }
    \over
        { \seq{\Delta }{M : B}}
   }
   (\mbox{wk})
\]

\[  {
      { \seq{\Gamma}{M:B} \qquad \sigma(\Gamma) \hbox{ homogeneous} }
    \over
      { \seq{\sigma(\Gamma)}{M : B} }
    }
    (\mbox{perm})
\]


\[{ {b : o^r \rightarrow o \in \Sigma } \over {\seq{}{b : o^r \rightarrow o}}}  (\Sigma\mbox{-const})  \]

\[{ \over
{\seq{\overline{x_{ij}} : \overline{A_{ij}}\, }{x_{ij} :
A_{ij}}}}(\mbox{var})\]

\[
{ {\seq{\overline{x_1} : \overline{A_1}\, | \, \cdots\, | \,
\overline{x_{n+1}} : \overline{A_{n+1}}}{M : B}} \qquad
\ord{\overline{A_{n+1}}} \geq \ord{B} \over {\seq{\overline{x_1} :
\overline{A_1}\, | \, \cdots\, | \, \overline{x_{n}} :
\overline{A_{n}}}{\lterm{\overline{x_{n+1}} : \overline{A_{n+1}}}{M}
: (\overline{A_{n+1}} \, | \, B)}} } (\mbox{$\lambda$-abs})\]

\[ {{\seq{\Gamma}{M : (\overline{B_1} \, | \, \cdots \, | \, \overline{B_m} \, | \, o)} \qquad
\seq{\Gamma}{N_1 : B_{11}} \quad \cdots \quad \seq{\Gamma}{N_{l} :
B_{1l}} \qquad l = |\overline{B_1}| } \over{ \seq{\Gamma}{M N_1
\cdots N_{l_1} : (\overline{B_2} \, | \, \cdots \, | \,
\overline{B_m} \, | \, o)}}} (\mbox{app})\]


\[ {{\seq{\Gamma}{M : (\overline{B_1} \, | \, \cdots \, | \, \overline{B_m} \, | \, o)} \qquad
\seq{\Gamma}{N_1 : B_{11}} \quad \cdots \quad \seq{\Gamma}{N_{l} :
B_{1l}} \qquad l < |\overline{B_1}| } \over{ \seq{\Gamma}{M N_1
\cdots N_{l_1} : (\overline{B_2} \, | \, \cdots \, | \,
\overline{B_m} \, | \, o)}}} (\mbox{app+})\]

where $\overline{B_1} = B_{11}, \ldots, B_{1l},\overline{B}$ with
the condition that every variable in $\Sigma$ has an order greater
than $\ord{\overline{B_1}}$.


\begin{lem}[Basic properties]
\label{lem:safe_basic_prop} Suppose $\Gamma \vdash_s M : B$ is a
valid judgment then

\begin{itemize}
\item[(i)] $B$ is homogeneous
\item[(ii)] Every free variables of $M$ has order at least $ord(M)$
\end{itemize}
\end{lem}


\begin{dfn}[Simultaneous substitution for safe terms]
\label{dnf:safe_simsubst}
 We use the notation
$\subst{\overline{N}}{\overline{x}}$ for $\subst{N_1 \ldots N_n}{x_1
\ldots x_n}$:
\begin{eqnarray*}
x_i \subst{\overline{N}}{\overline{x}} &=& N_i\\
 y \subst{\overline{N}}{\overline{x}} &=& y \quad \mbox{ if } y \not \neq x_i \mbox{ for all } i,\\
(M N_1 \ldots N_l) \subst{\overline{N}}{\overline{x}} &=& (M \subst{\overline{N}}{\overline{x}}) (N_1 \subst{\overline{N}}{\overline{x}}) \ldots  (N_l \subst{\overline{N}}{\overline{x}})\\
(\lambda x_i . M) \subst{\overline{N}}{\overline{x}} &=& \lambda x_i
. M
\subst{N_1 \ldots N_{i-1} N_{i+1}\ldots N_n}{x_1 \ldots x_{i-1} x_{i+1}\ldots x_n} \\
(\lambda \overline{y} : \overline{A}. T)
\subst{\overline{N}}{\overline{x}} &=& \lambda \overline{z} . T
\subst{\overline{z}}{\overline{y}}
\subst{\overline{N}}{\overline{x}} \\
&& \mbox{where $T$ is a safe
term and $\overline{z} = z_1, \ldots z_p$ are all fresh variables}\\
\end{eqnarray*}
\end{dfn}

 Remark: On safe terms, simultaneous substitution can be achieved inductively by only performing
 simultaneous substitution on smaller sub-terms that are safe.

We now prove that the ``no variable clash lemma'' also hold with
this new definition of the homogeneous safe $\lambda$-calculus.

\begin{lem}[No variable clash lemma]
In the safe $\lambda$-calculus, there is no clash of variable name
when performing substitution:
\[ \qquad M[N_1 / x_1 , \cdots, N_n / x_n] \]
 provided the substitution is performed simultaneously on
\emph{all} free variables of the same order in $M$
i.e.~$\makeset{x_1, \cdots, x_n}$ is the set variables of the same
order as $x_1$ that occur free in $M$.
\end{lem}

\begin{proof}
First we note that if the substitutions were consecutive ($M
\subst{N_1}{x_1}\ldots \subst{N_n}{x_n}$) instead of being
simultaneous then a variable capture would arise if some $N_i$ has a
free occurrence of a variable $x_j$ with $j>i$. However this capture
does not happen when performing the substitutions simultaneously as
follow: $M \subst{N_1, \ldots N_n}{x_1, \ldots x_n}$.

Suppose that a variable capture occurs in the term $M$: $M$ has a
subterm $\lambda y_1 \ldots y_p. T$ such that some $x_i$ appears
freely in $T$ and some $y_k$ appears freely in $N_i$. Because of the
previous remark,
 we can assume that the subterm $\lambda y_1 \ldots y_p . T$ is safe.

Since $x_i$ appears freely in the safe term $\lambda y_1 \ldots y_p
. T$, by Lemma \ref{lem:safe_basic_prop} (ii) we get:
$$ ord(x_i) \geq ord(\lambda y_1 \ldots y_p . T) \geq 1+ ord(y_k) > ord(y_k)$$

Since $y_k$ appears freely in the safe term $N_i$, Lemma
\ref{lem:safe_basic_prop} (ii) gives:

$$ ord(y_k) \geq ord(N_i) = ord(x_i)$$

Hence we reach a contradiction.
\end{proof}
