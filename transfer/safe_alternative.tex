\section{Homogeneous safe $\lambda$-calculus}
\label{sec:safe_alt}

We recall the definition of the safe $\lambda$-calculus given in
\cite{Ong2005}.

\subsection{Rules}

These rules are a corrected version of
\cite{DBLP:conf/fossacs/AehligMO05}

In the following we shall consider terms-in-context $\seq{\Gamma}{M
: A}$ of the simply-typed $\lambda$-calculus. Let $\Delta$ be a
simply-typed alphabet i.e., each symbol in $\Delta$ has a simple
type. We write $\terms{A}{\Delta}$ for the set of terms of type $A$
built up from the set $\Delta$ understood as constant symbols,
\emph{without} using $\lambda$-abstraction.


The \textbfit{Safe $\lambda$-Calculus} is a sub-system of the
simply-typed $\lambda$-calculus. Typing judgements (or
terms-in-context) are of the form
\begin{equation}
\nonumber \seq{\overline{x_1}:\overline{A_1} \, | \, \cdots \, | \,
\overline{x_n} :  \overline{A_n}}{M : B}
\end{equation}
which is shorthand for $\seq{x_{11} : A_{11}, \cdots, x_{1r}:
A_{1r}, \cdots}{M : B}$. \emph{Valid typing judgements} of the
system are defined by induction over the following rules, where
$\Delta$ is a given homogeneously-typed alphabet:

\[ (\mbox{\textbf{wk}})
{    { \seq{\Sigma}{M:B} \qquad {\Sigma \subset \Delta} }
    \over
        { \seq{\Delta }{M : B}}
   }
\qquad
    (\mbox{\textbf{perm}})
  {
      { \seq{\Gamma}{M:B} \qquad \sigma(\Gamma) \hbox{ homogeneous} }
    \over
      { \seq{\sigma(\Gamma)}{M : B} }
    }
\]


\[ (\mathbf{\Sigma}\mbox{\textbf{-const}})    {   {b : o^r \rightarrow o \in \Sigma } \over {\seq{}{b : o^r \rightarrow o}}}
\qquad
 (\mbox{\textbf{var}}) { \over
{\seq{\overline{x_{ij}} : \overline{A_{ij}}\, }{x_{ij} :
A_{ij}}}}\]

\[ (\mbox{$\mathbf{\lambda}$\textbf{-abs}})
{ {\seq{\overline{x_1} : \overline{A_1}\, | \, \cdots\, | \,
\overline{x_{n+1}} : \overline{A_{n+1}}}{M : B}} \qquad
\ord{\overline{A_{n+1}}} \geq \ord{B} -1 \over {\seq{\overline{x_1} :
\overline{A_1}\, | \, \cdots\, | \, \overline{x_{n}} :
\overline{A_{n}}}{\lterm{\overline{x_{n+1}} : \overline{A_{n+1}}}{M}
: (\overline{A_{n+1}} \, | \, B)}} } \]

\[ (\mbox{\textbf{app}}) {{\seq{\Gamma}{M : (\overline{B_1} \, | \, \cdots \, | \, \overline{B_m} \, | \, o)} \qquad
\seq{\Gamma}{N_1 : B_{11}} \quad \cdots \quad \seq{\Gamma}{N_{l} :
B_{1l}} \qquad l = |\overline{B_1}| } \over{ \seq{\Gamma}{M N_1
\cdots N_{l_1} : (\overline{B_2} \, | \, \cdots \, | \,
\overline{B_m} \, | \, o)}}} \]


\[ (\mbox{\textbf{app+}}) {{\seq{\Gamma}{M : (\overline{B_1} \, | \, \cdots \, | \, \overline{B_m} \, | \, o)} \qquad
\seq{\Gamma}{N_1 : B_{11}} \quad \cdots \quad \seq{\Gamma}{N_{l} :
B_{1l}} \qquad l < |\overline{B_1}| } \over{ \seq{\Gamma}{M N_1
\cdots N_{l_1} : (\overline{B} \, | \, \cdots \, | \,
\overline{B_m} \, | \, o)}}} \]

where $\overline{B_1} = B_{11}, \ldots, B_{1l},\overline{B}$ with
the condition that every variable in $\Gamma$ has an order greater
than $\ord{\overline{B_1}}$.


\begin{lem}[Basic properties]
\label{lem:safe_basic_prop} Suppose $\Gamma \vdash M : B$ is a
valid judgment then

\begin{itemize}
\item[(i)] $B$ is homogeneous
\item[(ii)] Every free variables of $M$ has order at least $ord(M)$
\end{itemize}
\end{lem}

We now define a special kind of substitution that performs simultaneous substitution and
that permits variable capture (i.e. no renaming occurs in the abstraction case).

\begin{dfn}[Capture permitting simultaneous substitution (for homogeneous safe terms)]
\label{dnf:safe_simsubst}
 We use the notation $\subst{\overline{N}}{\overline{x}}$ for $\subst{N_1 \ldots N_n}{x_1 \ldots x_n}$ and
$\overline{y}:\overline{A}$ for $y_1:A_1, \ldots, y_p:A_p$.
A safe term must have one of the forms occurring on the left-hand side of the following equations, where
the terms $M$, $N_1, \ldots N_l$ are safe terms:
\begin{eqnarray*}
c \subst{\overline{N}}{\overline{x}} &=& c \quad \mbox{ where $c$ is a $\Sigma$-constant}\\
x_i \subst{\overline{N}}{\overline{x}} &=& N_i\\
 y \subst{\overline{N}}{\overline{x}} &=& y \quad \mbox{ if } y \not \neq x_i \mbox{ for all } i,\\
(M N_1 \ldots N_l) \subst{\overline{N}}{\overline{x}} &=& (M \subst{\overline{N}}{\overline{x}}) (N_1 \subst{\overline{N}}{\overline{x}}) \ldots  (N_l \subst{\overline{N}}{\overline{x}})\\
(\lambda \overline{y} : \overline{A}. M)
\subst{\overline{N}}{\overline{x}} &=& \lambda \overline{y} . M
\subst{\overline{N} \upharpoonright I}{\overline{x} \upharpoonright I} \\
&& \mbox{where } I  = \{ i \in 1..n \ | \ x_i \not \in \overline{y} \}
\end{eqnarray*}

where $ \upharpoonright$ is the index filtering operator: if $s$ is a sequence and $I$ a set of indices then
$s \upharpoonright I$ is the subsequence of $s$ obtained by removing from $s$ all the elements
at a position that is not in $I$.
\end{dfn}

We now prove that in the homogeneous safe $\lambda$-calculus, variable capture never occurs. Therefore
the capture permitting substitution that we use is equivalent to the traditional substitution (where an unbound number of
variable names are required to avoid capture).

\begin{lem}[No variable capture lemma]
In the safe $\lambda$-calculus, there is no capture of variable
when performing the following capture permitting simultaneous substitution on safe terms $M$, $N_1, \ldots N_n$:
\[ \qquad M[N_1 / x_1 , \cdots, N_n / x_n] \]
 provided the substitution is performed simultaneously on
\emph{all} free variables of the same order in $M$
i.e.~$\makeset{x_1, \cdots, x_n}$ is the set variables of the same
order as $x_1$ occurring freely in $M$.
\end{lem}

\begin{proof}
We prove the result by induction. The variable, constant and application cases are trivial.
For the abstraction case, suppose $M = \lambda \overline{y} : \overline{A}. P$. The capture permitting
simultaneous substitution gives:
$$M \subst{\overline{N}}{\overline{x}} = \lambda \overline{y} . P
\subst{\overline{N} \upharpoonright I}{\overline{x} \upharpoonright I}$$

The variables $\overline{x}$ all occurs freely in $M$ therefore
the variable in $\overline{x}$ cannot be abstracted under the $\lambda \overline{y}$. Hence $I = \emptyset$
and:
$$M \subst{\overline{N}}{\overline{x}} = \lambda \overline{y} . P
\subst{\overline{N}}{\overline{x}}$$

The free variable of $P$ that have the same order as $x_1$ are exactly the variables in $\overline{x}$ therefore by applying the
induction hypothesis we know that there is no variable capture in $P
\subst{\overline{N}}{\overline{x}}$.

Hence the only possible case of variable capture that can occur is when a variable $y_j$ for some $j \in 1..p$ occurs freely
in $N_i$. By lemma \ref{lem:safe_basic_prop} (ii) we get:
$$ ord(y_j) \geq ord(N_i) = ord(x_i)$$

Moreover, since $x_i$ occurs freely in the safe term $\lambda \overline{y}. P$, again by lemma \ref{lem:safe_basic_prop} (ii)
we get:
$$ ord(x_i) \geq ord(\lambda y_1 \ldots y_p . T) \geq 1+ ord(y_j) > ord(y_j)$$

Hence we reach a contradiction.

%First we note that if we perform the substitutions consecutively ($M
%\subst{N_1}{x_1}\ldots \subst{N_n}{x_n}$) instead of
%simultaneously, then a variable capture may arise if for some $i$ and $j>i$, $x_j$ occurs freely
%in $N_i$. This problem does not arise when performing the substitutions simultaneously as
%follow: $M \subst{N_1, \ldots N_n}{x_1, \ldots x_n}$.

%Remark: We observe from this definition, that simultaneous substitution on safe terms is achieved
%inductively by performing simultaneous substitution only on smaller sub-terms that are safe.

%Suppose that a variable capture occurs in the term $M$: $M$ has a
%subterm $\lambda y_1 \ldots y_p. T$ such that some $x_i$ appears
%freely in $T$ and some $y_k$ appears freely in $N_i$. Because of the
%previous remark,
% we can assume that the subterm $\lambda y_1 \ldots y_p . T$ is safe.

\end{proof}


\todobox{Prove that safety is preserved by
capture-permitting simultaneous substitution provided the substitution is performed simultaneously on
\emph{all} free variables of the same order in $M$.
}

\todobox{Prove that any redex occurring in a safe term can be reduced using the capture-permitting simultaneous substitution
(in particular for redex formed with the (app+) rule)}


\subsection{Rules - second version}


In this section, we will refine the formation rules
given in the previous section. We say that $\Gamma \vdash M : A$ verifies $P_i$ if all the
variables in $\Gamma$ have orders at least $\ord{A}+i$ and we introduce the notation $\Gamma \vdash^{i} M : A$ for $i \in
\zset$ to mean that $\Gamma \vdash M : A$ and $\Gamma \vdash M : A$
satisfies $P_i$.


The following lemma says that if $\Gamma \vdash M : A$ then variable in the context $\Gamma$ with order
strictly smaller than $M$ do not occur freely in $M$ and therefore the context can be restricted to a smaller number of variables.

\begin{lem}[Context reduction]
\label{lem:restriction}

Suppose that $\Gamma \vdash M : A$ satisfies $P_i$ with $i\leq0$, then we have
$\Gamma' \vdash^{0} M : A$ where $$\Gamma' = \{ z \in \Gamma \ |
\ \ord{z} \geq \ord{M} \} = \Gamma \setminus \{ z \in \Gamma \ | \ \ord{M} + i \leq \ord{z} < \ord{M} \}$$
\end{lem}
\begin{proof}
By induction, the only non trivial cases are (app) and (app+):
\begin{itemize}
\item (app)

    \[ {{\seq{\Gamma}{M : (\overline{B_1} \, | \, \cdots \, | \, \overline{B_m} \, | \, o)} \qquad
    \seq{\Gamma}{N_1 : B_{11}} \quad \cdots \quad \seq{\Gamma}{N_{l} :
    B_{1l}} \qquad l = |\overline{B_1}| } \over{ \seq{\Gamma}{M N_1
    \cdots N_{l_1} : (\overline{B_2} \, | \, \cdots \, | \,
    \overline{B_m} \, | \, o)}}} (\mbox{app})\]

    If the conclusion verifies $P_i$ then, for all $z \in \Gamma$:
    \begin{eqnarray*}
    \ord{z} \geq 1 + \ord{\overline{B_2}} + i
    &=& 1 + \ord{\overline{B_1}} + \ord{\overline{B_2}} - \ord{\overline{B_1}} + i \\
    &=& \ord{M} + (\ord{\overline{B_2}} - \ord{\overline{B_1}} + i)
    \end{eqnarray*}
    Therefore the first premise satisfies $P_j$ where $j={\ord{\overline{B_2}} - \ord{\overline{B_1}} + i}$

    and $j< 0$ because of the homogeneity of the type. Hence by the induction hypothesis,
    there is a context $$\Gamma' = \{ z \in \Gamma \ |
    \ \ord{z} \geq \ord{M} \} = \Gamma \setminus \{ z \in \Gamma \ | \ \ord{M} + j \leq \ord{z} < \ord{M} \}$$
    such that $\Gamma' \vdash^{0} M : (\overline{B_1} \, | \, \cdots \, | \, \overline{B_m} \, | \, o)$.


    Similarly:
    \begin{eqnarray*}
    \ord{z} \geq 1 + \ord{\overline{B_2}} + i
    &=& \ord{\overline{B_1}} + (1+\ord{\overline{B_2}} - \ord{\overline{B_1}} + i) \\
    &=& \ord{\overline{B_1}} + j+1
    \end{eqnarray*}

    where $j+1\leq 0$, hence by the induction hypothesis for $k : 1..l$ there is a
    context $$\Gamma'' = \{ z \in \Gamma \ |
    \ \ord{z} \geq \ord{N_k} \} = \Gamma \setminus \{ z \in \Gamma \ | \ \ord{M} + j+1 \leq \ord{z} < \ord{M} \}$$
    such that $\Gamma'' \vdash N_k : B_{1k}$ verifies $P_0$.

    But since $\Gamma' = \Gamma'' \union \{ z \in \Gamma \ | \ \ord{M} + j = \ord{z}\}$, we have for $k : 1..l$:

    $\Gamma' \vdash N_k : B_{1k}$ satisfies $P_{-1}$.


    We can now apply the (App) rule and get
    $$\Gamma' \vdash M N_1 \ldots N_l : (\overline{B_2} \, | \, \cdots \, | \,
    \overline{B_m} \, | \, o)$$
    where for all $z\in \Gamma'$:
    \begin{eqnarray*}
    \ord{z} \geq 1 + \ord{\overline{B_1}}
    &>& 1 + \ord{\overline{B_2}} = \ord{M N_1 \ldots N_l}
    \end{eqnarray*}

\item (app+) The side-condition of the rule (app+) ensures that all but the first premise verify $P_0$.
The first premise has the same order has the
conclusion of the rule therefore the first premise verifies
$P_{i}$. We can therefore conclude by using the induction hypothesis and applying the rule (app+).
\end{itemize}
\end{proof}


The following lemma is a direct consequence of the context restriction lemma:
\begin{lem}
\label{lem:prooftree01only}
If $\Gamma \vdash^{0} M : T$ then there is valid proof tree
showing that $\Gamma \vdash M : T$ such that all the judgments
of the proof tree verify $P_0$ or $P_{-1}$.
\end{lem}


\begin{proof}
Suppose that $\Gamma \vdash M : T$ such that it satisfies $P_{-1}$.

We show that there is a proof tree for
$\Gamma \vdash M : T$ where all the nodes of the tree verify $P_0$
or $P_{-1}$ by case analysis on the last rule used to prove $\Gamma \vdash M : T$ .

\begin{itemize}
\item (wk) If $\Delta \vdash M : T$ verifies $P_{-1}$ then in particular $\Gamma
\vdash M : T$ verifies $P_{-1}$ for any $\Gamma \subset \Delta$.

\item (perm) trivial

\item ($\Sigma$) the context is empty therefore the sequent verifies $P_{-1}$.

\item (var) the context contains only the variable itself : verifies $P_0$.

\item (abs) the second premise of the rule guarantees that the first
premise verifies $P_{-1}$.

\item (app+) The first premise has the same order has the
conclusion of the rule therefore the first premise verifies
$P_{-1}$. The side-condition of the rule (app+) ensures that the other premises verify $P_0$.

\item (app) Using lemma \ref{lem:restriction} and then applying the rule (app).

\end{itemize}
\end{proof}

\subsubsection{Refining the rules of the homogeneous safe $\lambda$-calculus}

Using the observations that we have just made, we will now refine the rules
of the safe $\lambda$-calculus with homogeneous type. We would like to obtain a system
of rules generating sequents that verify $P_0$. Those sequents correspond to the ``safe'' terms.

The system of rules must be able to generate intermediate sequents that are then used to produce term
satisfying $P_0$. Because of the lemma \ref{lem:prooftree01only}, we know that the only necessary intermediate sequents
are those that either satisfy $P_0$ or $P_{-1}$. In other word, it is useless to produce sequents that do not satisfy
$P_0$ or $P_{-1}$. We will therefore use the notation $\vdash^0$ and $\vdash^{-1}$ instead of $\vdash$ to precise
whether a given sequent satisfies $P_0$ or $P_{-1}$.

The first refinement consist in constraining the weakening rule so that
it only permits the addition of variable with an order big enough:
$$ \mathbf{(wk^{0})} \quad  \rulef{\Gamma \vdash^{0} M : A}{\Gamma , x : B \vdash^{0} M : A} \quad \ord{B} \geq \ord{A} $$
$$ \mathbf{(wk^{-1})} \quad  \rulef{\Gamma \vdash^{-1} M : A}{\Gamma , x : B \vdash^{-1} M : A} \quad \ord{B} \geq \ord{A} -1$$

There is also an additional rules expressing the fact that $P_0$ implies $P_{-1}$:

$$ \mathbf{(seq)} \quad \rulef{\Gamma \vdash^{0} M : A}{\Gamma \vdash^{-1} M : A} $$


Because of the context reduction lemma, any sequent verifying $P_1$ can be obtained
by applying the weakening rule ($\mathbf{wk^{-1}}$) or the rule (\textbf{seq}) on a sequent
verifying $P_0$. Therefore, with the exception these two rules, we can safely force all the rules to
have a conclusion sequent verifying $P_0$:
\begin{itemize}
\item  For the rules \textbf{(perm)}, \textbf{(const)} and \textbf{(var)}, only the tagging of the sequent changes:

$$ \mathbf{(var)} \quad  \rulef{}{x : A\vdash^{0} x : A}
\qquad
  (\mbox{\textbf{perm}}) {
      { \Gamma \vdash^0 M:B \qquad \sigma(\Gamma)  } \hbox{ homogeneous}
    \over
      { \sigma(\Gamma) \vdash^0 M : B }
    }
$$

\[ \textbf{(const)}
    { \over { \vdash^0 b : o^r \rightarrow o}} \quad b : o^r \rightarrow o \in \Sigma
    \]

\item  The previous definition of the abstraction rule has a side condition
expressing the fact that the premise verifies $P_0$ or $P_{-1}$. Since this is always true for sequents
generated by our new system of rules, we can drop the side condition:
$$ \mathbf{(abs)} \quad  \rulef{\Gamma, \overline{x} : \overline{A} \vdash^{-1} M : B}
                                   {\Gamma  \vdash^{0} \lambda \overline{x} : \overline{A} . M : (\overline{A},B)}$$


\item The application rule \textbf{(app)} has the following form:
\[ (\mbox{\textbf{app}})
    {
        { \Gamma \vdash^{-1} M : (\overline{A} \, | B)
        \qquad
        \Gamma \vdash^? N_1 : A_1 \quad \cdots \quad \Gamma \vdash^? N_{l} : A_l \qquad l = |\overline{A}|
        }
    \over{
        \Gamma \vdash^0 M N_1 \cdots N_{l_1} : B
       }
    }
\]

Suppose that the sequent in the first premise verifies $P_{-1}$, then by Lemma \label{lem:safe_basic_prop}(ii)
we have:
$$\forall z \in \Gamma : \ord{z} \geq 1 + \ord{\overline{A}} -1 = \ord{\overline{A}} = \ord{\overline{N}}$$
Hence, all the sequents of the premises but the first one verify $P_0$. The rule (app) is therefore given by:
\[ (\mbox{\textbf{app}})
    {
        { \Gamma \vdash^{-1} M : (\overline{A} \, | B)
        \qquad
        \Gamma \vdash^0 N_1 : A_1 \quad \cdots \quad \Gamma \vdash^0 N_{l} : A_l \qquad l = |\overline{A}|
        }
    \over{
        \Gamma \vdash^0 M N_1 \cdots N_{l_1} : B
       }
    }
\]

\item For the application rule \textbf{(app+)}, the type of the sequent in the first premise has the same order
as the type of the conclusion premises, therefore since the conclusion verifies $P_0$, the first premise also verifies $P_0$.
The side condition exactly translates the fact that that all the other sequents in the premise verify $P_0$. Hence the rule becomes:
\[ (\mbox{\textbf{app+}}) {
    {
        \Gamma \vdash M : (\overline{B_1} \, | \, \cdots \, | \, \overline{B_m} \, | \, o) \qquad
        \Gamma \vdash^0 N_1 : B_{11} \quad \cdots \quad \Gamma \vdash^0 N_{l} : B_{1l} \qquad l < |\overline{B_1}|
    }
    \over{
        \Gamma \vdash^0 M N_1 \cdots N_{l_1} : (\overline{B} \, | \, \cdots \, | \, \overline{B_m} \, | \, o)
    }
} \]
where $\overline{B_1} = B_{11}, \ldots, B_{1l},\overline{B}$.
This rule can be equivalently stated as:
$$ \rulef{\Gamma \vdash^0 M : A\rightarrow B
                                        \qquad \Gamma \vdash^{0} N : A
                                   }
                                   {\Gamma  \vdash^{0} M N : B}$$
\end{itemize}

The full set of rules is given in table \ref{tab:homosafelmd_rules_refined}

\begin{table}[htbp]
$$  (\mbox{\textbf{perm}}) {
      { \Gamma \vdash^0 M:B \qquad \sigma(\Gamma)  } \hbox{ homogeneous}
    \over
      { \sigma(\Gamma) \vdash^0 M : B }
    }
\qquad
\mathbf{(seq)} \quad \rulef{\Gamma \vdash^{0} M : A}{\Gamma \vdash^{-1} M : A}
$$

$$
 \textbf{(const)}
    { \over { \vdash^0 b : o^r \rightarrow o}} \quad b : o^r \rightarrow o \in \Sigma
\qquad
 \mathbf{(var)} \quad  \rulef{}{x : A\vdash^{0} x : A} $$

$$ \mathbf{(wk^{0})} \quad  \rulef{\Gamma \vdash^{0} M : A}{\Gamma , x : B \vdash^{0} M : A} \quad \ord{B} \geq \ord{A} $$

$$ \mathbf{(wk^{-1})} \quad  \rulef{\Gamma \vdash^{-1} M : A}{\Gamma , x : B \vdash^{-1} M : A} \quad \ord{B} \geq \ord{A} -1$$


$$ (\mbox{\textbf{app}})
    \rulef
        {   \Gamma \vdash^{-1} M : (\overline{A} \, | B)
            \qquad
            \Gamma \vdash^0 N_1 : A_1 \quad \cdots \quad \Gamma \vdash^0 N_{l} : A_l \qquad l = |\overline{A}|
        }
        {
            \Gamma \vdash^0 M N_1 \cdots N_{l_1} : B
        }
$$

$$ \mathbf{(app^+)} \quad  \rulef{\Gamma \vdash^0 M : A\rightarrow B
                                        \qquad \Gamma \vdash^{0} N : A,
                                   }
                                   {\Gamma  \vdash^{0} M N : B}$$

$$ \mathbf{(abs)} \quad  \rulef{\Gamma| \overline{x} : \overline{A} \vdash^{-1} M : B}
                                   {\Gamma  \vdash^{0} \lambda \overline{x} : \overline{A} . M : (\overline{A}|B)}$$


where $\Gamma| \overline{x} : \overline{A}$ means that the lowest type-partition of the context is
$\overline{x} : \overline{A}$.
\caption{Rules of the homogeneous safe lambda calculus}
\label{tab:homosafelmd_rules_refined}
\end{table}
%%%
