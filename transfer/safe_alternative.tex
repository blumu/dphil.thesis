\section{Homogeneous safe $\lambda$-calculus}
\label{sec:safe_alt}

We recall the definition of the safe $\lambda$-calculus given in
\cite{Ong2005}.

\subsection{ Rules}

These rules are a corrected version of
\cite{DBLP:conf/fossacs/AehligMO05}

In the following we shall consider terms-in-context $\seq{\Gamma}{M
: A}$ of the simply-typed $\lambda$-calculus. Let $\Delta$ be a
simply-typed alphabet i.e., each symbol in $\Delta$ has a simple
type. We write $\terms{A}{\Delta}$ for the set of terms of type $A$
built up from the set $\Delta$ understood as constant symbols,
\emph{without} using $\lambda$-abstraction.


The \textbfit{Safe $\lambda$-Calculus} is a sub-system of the
simply-typed $\lambda$-calculus. Typing judgements (or
terms-in-context) are of the form
\begin{equation}
\nonumber \seq{\overline{x_1}:\overline{A_1} \, | \, \cdots \, | \,
\overline{x_n} :  \overline{A_n}}{M : B}
\end{equation}
which is shorthand for $\seq{x_{11} : A_{11}, \cdots, x_{1r}:
A_{1r}, \cdots}{M : B}$. \emph{Valid typing judgements} of the
system are defined by induction over the following rules, where
$\Delta$ is a given homogeneously-typed alphabet:

\[ {    { \seq{\Sigma}{M:B} \qquad {\Sigma \subset \Delta} }
    \over
        { \seq{\Delta }{M : B}}
   }
   (\mbox{wk})
\]

\[  {
      { \seq{\Gamma}{M:B} \qquad \sigma(\Gamma) \hbox{ homogeneous} }
    \over
      { \seq{\sigma(\Gamma)}{M : B} }
    }
    (\mbox{perm})
\]


\[{ {b : o^r \rightarrow o \in \Sigma } \over {\seq{}{b : o^r \rightarrow o}}}  (\Sigma\mbox{-const})  \]

\[{ \over
{\seq{\overline{x_{ij}} : \overline{A_{ij}}\, }{x_{ij} :
A_{ij}}}}(\mbox{var})\]

\[
{ {\seq{\overline{x_1} : \overline{A_1}\, | \, \cdots\, | \,
\overline{x_{n+1}} : \overline{A_{n+1}}}{M : B}} \qquad
\ord{\overline{A_{n+1}}} \geq \ord{B} \over {\seq{\overline{x_1} :
\overline{A_1}\, | \, \cdots\, | \, \overline{x_{n}} :
\overline{A_{n}}}{\lterm{\overline{x_{n+1}} : \overline{A_{n+1}}}{M}
: (\overline{A_{n+1}} \, | \, B)}} } (\mbox{$\lambda$-abs})\]

\[ {{\seq{\Gamma}{M : (\overline{B_1} \, | \, \cdots \, | \, \overline{B_m} \, | \, o)} \qquad
\seq{\Gamma}{N_1 : B_{11}} \quad \cdots \quad \seq{\Gamma}{N_{l} :
B_{1l}} \qquad l = |\overline{B_1}| } \over{ \seq{\Gamma}{M N_1
\cdots N_{l_1} : (\overline{B_2} \, | \, \cdots \, | \,
\overline{B_m} \, | \, o)}}} (\mbox{app})\]


\[ {{\seq{\Gamma}{M : (\overline{B_1} \, | \, \cdots \, | \, \overline{B_m} \, | \, o)} \qquad
\seq{\Gamma}{N_1 : B_{11}} \quad \cdots \quad \seq{\Gamma}{N_{l} :
B_{1l}} \qquad l < |\overline{B_1}| } \over{ \seq{\Gamma}{M N_1
\cdots N_{l_1} : (\overline{B_2} \, | \, \cdots \, | \,
\overline{B_m} \, | \, o)}}} (\mbox{app+})\]

where $\overline{B_1} = B_{11}, \ldots, B_{1l},\overline{B}$ with
the condition that every variable in $\Sigma$ has an order greater
than $\ord{\overline{B_1}}$.


\begin{lem}[Basic properties]
\label{lem:safe_basic_prop} Suppose $\Gamma \vdash_s M : B$ is a
valid judgment then

\begin{itemize}
\item[(i)] $B$ is homogeneous
\item[(ii)] Every free variables of $M$ has order at least $ord(M)$
\end{itemize}
\end{lem}




\begin{dfn}[Simultaneous substitution for safe terms]
\label{dnf:safe_simsubst}
 We use the notation
$\subst{\overline{N}}{\overline{x}}$ for $\subst{N_1 \ldots N_n}{x_1
\ldots x_n}$:
\begin{eqnarray*}
x_i \subst{\overline{N}}{\overline{x}} &=& N_i\\
 y \subst{\overline{N}}{\overline{x}} &=& y \quad \mbox{ if } y \not \neq x_i \mbox{ for all } i,\\
(M N_1 \ldots N_l) \subst{\overline{N}}{\overline{x}} &=& (M \subst{\overline{N}}{\overline{x}}) (N_1 \subst{\overline{N}}{\overline{x}}) \ldots  (N_l \subst{\overline{N}}{\overline{x}})\\
(\lambda x_i . M) \subst{\overline{N}}{\overline{x}} &=& \lambda x_i
. M
\subst{N_1 \ldots N_{i-1} N_{i+1}\ldots N_n}{x_1 \ldots x_{i-1} x_{i+1}\ldots x_n} \\
(\lambda \overline{y} : \overline{A}. T)
\subst{\overline{N}}{\overline{x}} &=& \lambda \overline{z} . T
\subst{\overline{z}}{\overline{y}}
\subst{\overline{N}}{\overline{x}} \\
&& \mbox{where $T$ is a safe
term and $\overline{z} = z_1, \ldots z_p$ are all fresh variables}\\
\end{eqnarray*}
\end{dfn}

 Remark: On safe terms, simultaneous substitution can be achieved inductively by only performing
 simultaneous substitution on smaller sub-terms that are safe.

We now prove that the ``no variable clash lemma'' also hold with
this new definition of the homogeneous safe $\lambda$-calculus.

\begin{lem}[No variable clash lemma]
In the safe $\lambda$-calculus, there is no clash of variable name
when performing substitution:
\[ \qquad M[N_1 / x_1 , \cdots, N_n / x_n] \]
 provided the substitution is performed simultaneously on
\emph{all} free variables of the same order in $M$
i.e.~$\makeset{x_1, \cdots, x_n}$ is the set variables of the same
order as $x_1$ that occur free in $M$.
\end{lem}

\begin{proof}
First we note that if the substitutions were consecutive ($M
\subst{N_1}{x_1}\ldots \subst{N_n}{x_n}$) instead of being
simultaneous then a variable capture would arise if some $N_i$ has a
free occurrence of a variable $x_j$ with $j>i$. However this capture
does not happen when performing the substitutions simultaneously as
follow: $M \subst{N_1, \ldots N_n}{x_1, \ldots x_n}$.

Suppose that a variable capture occurs in the term $M$: $M$ has a
subterm $\lambda y_1 \ldots y_p. T$ such that some $x_i$ appears
freely in $T$ and some $y_k$ appears freely in $N_i$. Because of the
previous remark,
 we can assume that the subterm $\lambda y_1 \ldots y_p . T$ is safe.

Since $x_i$ appears freely in the safe term $\lambda y_1 \ldots y_p
. T$, by Lemma \ref{lem:safe_basic_prop} (ii) we get:
$$ ord(x_i) \geq ord(\lambda y_1 \ldots y_p . T) \geq 1+ ord(y_k) > ord(y_k)$$

Since $y_k$ appears freely in the safe term $N_i$, Lemma
\ref{lem:safe_basic_prop} (ii) gives:

$$ ord(y_k) \geq ord(N_i) = ord(x_i)$$

Hence we reach a contradiction.
\end{proof}


\subsection{Rules - second version}


In this section, we will refine the formation rules
given in the previous section.

We say that $\Gamma \vdash M : A$ verifies $P_i$ if all the
variables in $\Gamma$ have orders at least $\ord{A}+i$.

We introduce the notation $\Gamma \vdash^{i} M : A$ for $i \in
\zset$ to mean that $\Gamma \vdash M : A$ and $\Gamma \vdash M : A$
satisfies $P_i$.


The following lemma says that if $\Gamma \vdash M : A$ then variable in the context $\Gamma$ with order
strictly smaller than $M$ do not occur freely in $M$ and therefore the context can be restricted.

\begin{lem}[Context restriction]
\label{lem:restriction}

Suppose that $\Gamma \vdash M : A$ satisfies $P_i$ with $i\leq0$, then we have
$\Gamma' \vdash^{0} M : A$ where $$\Gamma' = \{ z \in \Gamma \ |
\ \ord{z} \geq \ord{M} \} = \Gamma \setminus \{ z \in \Gamma \ | \ \ord{M} + i \leq \ord{z} < \ord{M} \}$$
\end{lem}
\begin{proof}
By induction, the only non trivial cases are (app) and (app+):
\begin{itemize}
\item (app)

    \[ {{\seq{\Gamma}{M : (\overline{B_1} \, | \, \cdots \, | \, \overline{B_m} \, | \, o)} \qquad
    \seq{\Gamma}{N_1 : B_{11}} \quad \cdots \quad \seq{\Gamma}{N_{l} :
    B_{1l}} \qquad l = |\overline{B_1}| } \over{ \seq{\Gamma}{M N_1
    \cdots N_{l_1} : (\overline{B_2} \, | \, \cdots \, | \,
    \overline{B_m} \, | \, o)}}} (\mbox{app})\]

    If the conclusion verifies $P_i$ then, for all $z \in \Gamma$:
    \begin{eqnarray*}
    \ord{z} \geq 1 + \ord{\overline{B_2}} + i
    &=& 1 + \ord{\overline{B_1}} + \ord{\overline{B_2}} - \ord{\overline{B_1}} + i \\
    &=& \ord{M} + (\ord{\overline{B_2}} - \ord{\overline{B_1}} + i)
    \end{eqnarray*}
    Therefore the first premise satisfies $P_j$ where $j={\ord{\overline{B_2}} - \ord{\overline{B_1}} + i}$

    and $j< 0$ because of the homogeneity of the type. Hence by the induction hypothesis,
    there is a context $$\Gamma' = \{ z \in \Gamma \ |
    \ \ord{z} \geq \ord{M} \} = \Gamma \setminus \{ z \in \Gamma \ | \ \ord{M} + j \leq \ord{z} < \ord{M} \}$$
    such that $\Gamma' \vdash^{0} M : (\overline{B_1} \, | \, \cdots \, | \, \overline{B_m} \, | \, o)$.


    Similarly:
    \begin{eqnarray*}
    \ord{z} \geq 1 + \ord{\overline{B_2}} + i
    &=& \ord{\overline{B_1}} + (1+\ord{\overline{B_2}} - \ord{\overline{B_1}} + i) \\
    &=& \ord{\overline{B_1}} + j+1
    \end{eqnarray*}

    where $j+1\leq 0$, hence by the induction hypothesis for $k : 1..l$ there is a
    context $$\Gamma'' = \{ z \in \Gamma \ |
    \ \ord{z} \geq \ord{N_k} \} = \Gamma \setminus \{ z \in \Gamma \ | \ \ord{M} + j+1 \leq \ord{z} < \ord{M} \}$$
    such that $\Gamma'' \vdash N_k : B_{1k}$ verifies $P_0$.

    But since $\Gamma' = \Gamma'' \union \{ z \in \Gamma \ | \ \ord{M} + j = \ord{z}\}$, we have for $k : 1..l$:

    $\Gamma' \vdash N_k : B_{1k}$ satisfies $P_{-1}$.


    We can now apply the (App) rule and get
    $$\Gamma' \vdash M N_1 \ldots N_l : (\overline{B_2} \, | \, \cdots \, | \,
    \overline{B_m} \, | \, o)$$
    where for all $z\in \Gamma'$:
    \begin{eqnarray*}
    \ord{z} \geq 1 + \ord{\overline{B_1}}
    &>& 1 + \ord{\overline{B_2}} = \ord{M N_1 \ldots N_l}
    \end{eqnarray*}

\item (app+) The first premise has the same order has the
conclusion of the rule therefore the first premise verifies
$P_{i}$. The other premises verify $P_{i+1}$. Again using the induction hypothesis
and by applying the rule (App+) we get the result wanted.
\end{itemize}

\end{proof}


The following lemma is a direct consequence of the context restriction lemma:
\begin{lem}
%\label{}
If $\Gamma \vdash^{0} M : T$ then there is valid proof tree
showing that $\Gamma \vdash M : T$ such that all the judgments
of the proof tree verify $P_0$ or $P_{-1}$.
\end{lem}


\begin{proof}
Suppose that $\Gamma \vdash M : T$ such that it satisfies $P_{-1}$.

We show that there is a proof tree for
$\Gamma \vdash M : T$ where all the nodes of the tree verify $P_0$
or $P_{-1}$ by case analysis on the last rule used to prove $\Gamma \vdash M : T$ .

\begin{itemize}
\item (wk) If $\Delta \vdash M : T$ verifies $P_{-1}$ then in particular $\Gamma
\vdash M : T$ verifies $P_{-1}$ for any $\Gamma \subset \Delta$.

\item (perm) trivial

\item ($\Sigma$) the context is empty therefore the sequent verifies $P_{-1}$.

\item (var) the context contains only the variable itself : verifies $P_0$.

\item (abs) the second premise of the rule guarantees that the first
premise verifies $P_{-1}$.

\item (app+) The first premise has the same order has the
conclusion of the rule therefore the first premise verifies
$P_{-1}$. The other premises have an order strictly smaller that the
order of the conclusion of the rule therefore they verify $P_0$

\item (app) Using lemma \ref{lem:restriction} and then applying the rule (app).

\end{itemize}
\end{proof}

We are interested by the sequents that verify $P_0$. For that purpose, we now refine the previous system of rules
of the safe lambda calculus such that it only generates sequents that are either satisfying $P_0$ or
that can be used to produce term that satisfy $P_0$.

Thanks to the previous lemma, we know that it is useless to produce sequents that do not satisfy
$P_0$ or $P_{-1}$. We will therefore use the notation $\vdash^0$ and $\vdash^{-1}$ instead of $\vdash$ to precise
whether a given sequent satisfies $P_0$ or $P_{-1}$.

The first refinement consist in constraining the weakening rule so that
it only permits the addition of variable with an order big enough:
$$ \mathbf{(wk^{0})} \quad  \rulef{\Gamma \vdash^{0} M : A}{\Gamma , x : B \vdash^{0} M : A} \quad \ord{B} \geq \ord{A} $$
$$ \mathbf{(wk^{-1})} \quad  \rulef{\Gamma \vdash^{-1} M : A}{\Gamma , x : B \vdash^{-1} M : A} \quad \ord{B} \geq \ord{A} -1$$

There is also an additional rules expressing the fact that $P_0$ implies $P_{-1}$:

$$ \mathbf{(seq)} \quad \rulef{\Gamma \vdash^{0} M : A}{\Gamma \vdash^{-1} M : A} $$

In the rules \textbf{(perm)}, \textbf{(const)} and \textbf{(var)}, only the tagging of the sequent changes:

$$ \mathbf{(var)} \quad  \rulef{}{x : A\vdash^{0} x : A} $$

\[  (\mbox{\textbf{perm}}) {
      { \Gamma \vdash^s M:B \qquad \sigma(\Gamma)  } \hbox{ homogeneous}
    \over
      { \sigma(\Gamma) \vdash^s M : B }
    }
\]


\[ \textbf{(const)}
    { \over { \vdash^0 b : o^r \rightarrow o}} \quad b : o^r \rightarrow o \in \Sigma
    \]





%%%
