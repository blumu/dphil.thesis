
In \cite{KNU02}, Knapik et al. studied the infinite hierarchy of
trees recognized by a higher-order pushdown automaton. They
introduce a restriction on higher-order grammars called
\emph{safety} and they show that trees recognized by pushdown
automata of level $n$ coincides with trees generated by safe
-higher-order grammars of level $n$. This characterization permits
them to show that the monadic second-order theory of an infinite
tree recognized by a higher-order pushdown automaton of any level is
decidable.

Safety has also appeared in a different form in \cite{Dam82} under
the name \emph{restriction of derived types}. The forthcoming thesis
of Jolie de Miranda \citep{demirandathesis} contains a comparison of
safety and the restriction of derived types.

More recently, Ong proved in \cite{OngLics2006} that the safety
assumption of \cite{KNU02} is in fact not necessary. More precisely,
the paper shows that MSO theory of trees generated by order-$n$
recursion schemes is $n$-EXPTIME complete.

For this particular problem, \emph{safety} happens to be an
artificial restriction. However when the \emph{safety} condition is
transposed to the simply-type $\lambda$-calculus, it gives some
interesting properties. In particular, alpha-conversion of terms
becomes unnecessary when performing substitution on safe terms. We
therefore think that \emph{safety} is a property that is worth
studying.

We first give a definition of the Safe $\lambda$-calculus and then
