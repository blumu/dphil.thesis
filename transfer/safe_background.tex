\section{Background}

\todobox{background on safety}

\subsection{Homogeneous type}
\label{sec:homotypes}

Let $Types$ be the set of simple types generated by the grammar $A
\, ::= \, o \; | \; A \funsp A$. Any type different from the base
type $o$ can be written $(A_1, \cdots, A_n, o)$ for some $n \geq 1$,
which is a shorthand for $A_1 \funsp \cdots \funsp A_n \funsp o$ (by
convention, $\rightarrow$ associates to the right).

We suppose that a ranking function has been defined: ${\sf rank} :
Types \funto (L, \leq)$ where $(L, \leq)$ is any linearly ordered
set. Possible candidates for the ranking function are:
\begin{itemize}
\item ${\sf ord} : Types \funto (\nat,\leq)$ with $\ord{o} = 0$
and $\ord{A \funsp B} = \max(\ord{A}+1, \ord{B})$.
\item ${\sf height} : Types \funto (\nat,\leq)$ with
\begin{eqnarray*}
     \slheight{o} &=& 0 \\
     \slheight{A \funsp B} &=& 1 + \max(\slheight{A}, \slheight{B})
\end{eqnarray*}
\item ${\sf nparam} : Types \funto (\nat,\leq)$ with $\nparam{o} = 0$
and $\nparam{A_1, \cdots, A_n} = n$.
\item ${\sf ordernp} : Types \funto (\nat \times \nat,\leq)$ with $ {\sf ordernp} (t)  = (\order{t}, \nparam{t})$ for $t \in Types$.
\end{itemize}


Following \cite{KNU02}, a type is rank-homogeneous if it is $o$ or
if it is $(A_1, \cdots, A_n, o)$ with the condition that $\rank{A_1}
\geq \rank{A_2}\geq \cdots \geq \rank{A_n}$ and each $A_1$, \ldots,
$A_n$ is rank-homogeneous.



Suppose that $\overline{A_1}$, $\overline{A_2}$, \ldots,
$\overline{A_n}$ are $n$ lists of types, where $A_{ij}$ denotes the
$j$th type of list $\overline{A_i}$ and $l_i$ the size of
$\overline{A_i}$. Then the notation $A \; = \; (\overline{A_1} \, |
\, \cdots \, | \, \overline{A_r} \, | \, o)$ means that
\begin{itemize}
  \item $A$ is the type $(A_{11},A_{12},\cdots, A_{1l_1}, A_{21}, \cdots,A_{2l_2}, \cdots A_{n1},\cdots, A_{nl_n},o)$
  \item $\forall i: \forall u,v \in A_i : \rank u = \rank v $
  \item $\forall i,j . \forall u \in A_i . \forall v \in A_j . i<j \implies \rank u >
   \rank v $
\end{itemize}
Consequently, $A$ is rank-homogenous. This notation organises the
$A_{ij}$s into partitions according to their ranks. Suppose $B =
(\overline{B_1} \, | \, \cdots \, | \, \overline{B_m} \, | \, o)$.
We write $(\overline{A_1} \, | \, \cdots \, | \, \overline{A_n} \, |
\, {B})$ to mean
\[(\overline{A_1} \, | \, \cdots \, | \, \overline{A_n} \, | \,
\overline{B_1} \, | \, \cdots \, | \, \overline{B_m} \, | \, o).\]

From now on, we only consider the rank function {\sf ord}. The term
``homogeneous'' will be used to refer to {\sf ord}-homogeneity.
