\def\aux#1{\lceil #1\rceil}
\def\nf#1{\eta_{\sf nf}(#1)}

\section{Game semantics of safe $\lambda$-terms}

We would like to find out whether the safety condition defined in
\cite{Ong2005} leads to a pointer economy in the corresponding game
semantics.

The example of section \ref{subsec:ptrless_strat} is a good example
to start with. We observe that for this particular example and in
the safe $\lambda$-calculus setting, the ambiguity that led us to
the addition of pointers to strategies disappear. More precisely,
$M_1$ is a safe term whereas $M_2$ is not. Indeed, there is a free
occurrence of the variable $x$ of type $o$ in the subterm $f
(\lambda y . x)$ which is not abstracted together with $y$ of type
$o$.


\begin{enumerate}
\item
Is it the case that in general, the pointers from the semantics of
safe $\lambda$-terms can be reconstructed uniquely from the moves of
the play?


\item
Is there any unsafe term whose game semantics is a strategy where
pointers can be recovered?

The answer is yes: take the term $T_i = (\lambda x y . y) M_i S$
where $i =1..2$ and $\Gamma \vdash_s S : A$. $T_1$ and $T_2$ both
$\beta$-reduce to the safe term $S$, therefore
$\sem{T_1}=\sem{T_2}=\sem{S}$. But $T_1$ is safe whereas $T_2$ is
unsafe. Since it is possible to recover the pointer from the game
semantics of $S$, it is as well possible to recover the pointer from
the semantics of $T_2$ which is unsafe.

\item
Is there any unsafe $\beta$-normal form whose game semantics is a
strategy where pointers can be recovered?


\end{enumerate}






\subsection{$\eta$-long normal form}

The $\eta$-expansion of $M: A\typear B$ is defined to be the term $\lambda x . M x : A\typear B$ where $x:A$ is a fresh variable.
It is easy to check that if $M$ is safe then $\lambda x . M x$ is also safe.

Consider the term $M : (A_1,\ldots,A_n,o)$, it can be expanded in several steps into
$\lambda \varphi_1 \ldots \varphi_l . M \varphi_1 \ldots \varphi_l$
where the $\varphi_i:A_i$ are fresh variables.

The $\eta$-normal form of a term is obtained by hereditarily $\eta$-expanding every sub-term occurring
at an operand position:

\begin{dfn}[$\eta$-normal form]
A term is either an abstraction or it can be written uniquely as $s_0 s_1 \ldots s_m$ where $s_0$ is a variable, a constant
or an abstraction and $m\geq0$.

The $\eta$-normal form of a term $M$ is denoted $\aux{M}$ and is defined as follow:
\begin{eqnarray*}
\aux{x s_1 \ldots s_m : (A_1,\ldots,A_n,o)} &=& \lambda \overline{\varphi} . x \aux{s_1} \aux{s_2} \ldots \aux{s_m} \aux{\varphi_1} \ldots \aux{\varphi_n} \\
&& \mbox{where } m \geq 0 \\
\aux{x s_1 \ldots s_m : o} &=& \lambda . x \aux{s_1} \aux{s_2} \ldots \aux{s_m} \\
&& \mbox{where } m \geq 0 \\
\aux{(\lambda x . s_0) s_1 \ldots s_m } &=& (\lambda x . \aux{s_0}) \aux{s_1} \aux{s_2} \ldots \aux{s_m}
\end{eqnarray*}
where $x$ is either a variable or a constant.
\end{dfn}

The $\eta$-long normal form appeared in \citep{DBLP:journals/tcs/JensenP76}
under the name \emph{long reduced form}
and in \citep{DBLP:journals/tcs/Huet75}
under the name \emph{$\eta$-normal form}. It was then investigated in \citep{huet76}
under the name \emph{extensional form}.



Terms in $\eta$-long normal form can be represented by a tree. The nodes
of the tree are of three kinds: abstraction node noted $\lambda \overline{x}$,
application node noted $@$ or operator-application nodes noted $x @$ where the operator $x$
 is either a variable or a constant.

The tree is defined formally by induction on the structure of the term in $\eta$-long normal form.
A term in $\eta$-long normal form is necessarily an abstraction.
\begin{itemize}
\item the tree for $\lambda . x$ where $x$ is a variable or a constant is:
$$ \sem{\lambda . x} =
  \pstree[levelsep=6ex]
    { \TR{\lambda } }
    { \TR{x}
    }$$
\item the tree for $\lambda x_1 \ldots x_n. M$ where $M$ is not an abstraction is:
$$ \sem{\lambda x_1 \ldots x_n . M} =
  \pstree[levelsep=6ex]
    { \TR{\lambda x_1 \ldots x_n} }
    { \SubTree{\sem{M}}
    }
$$
\item the tree for $x s_1 \ldots s_n$ where $x$ is a variable or a constant is:
$$ \sem{x s_1 \ldots s_n} =
  \pstree[levelsep=6ex]
    { \TR{x @} }
    { \SubTree{\sem{s_1}} \SubTree[linestyle=none]{\ldots} \SubTree{\sem{s_n}}
    }
$$
\item the tree for $(\lambda x.s_0) s_1 \ldots s_n$ is:
$$ \sem{(\lambda x.s_0) s_1 \ldots s_n} =
  \pstree[levelsep=6ex]
    { \TR{@} }
    {
    \SubTree{\sem{\lambda x.s_0}}    \SubTree{\sem{s_1}} \SubTree[linestyle=none]{\ldots} \SubTree{\sem{s_n}}
    }
$$
\end{itemize}






\subsection{Pointers in the game semantics of safe terms are recoverable}

\todomargin{rework this paragraph!}

We claim that the pointers in the game semantics of a safe term are
uniquely recoverable.

\todobox{Need to prove that safety is preserved by $\eta$-expansion.}

Consider a term $M$ safe, we can assume that $M$ is in $\eta$ normal
form.

The term can be represented by a computation tree: nodes at even
depth (starting at level 0) correspond to $\lambda$ and nodes at odd
length corresponds to either application $@$, variable $x$ or
variable followed by an application $f@$. A $\lambda$ node
represented consecutive abstraction of variables.

There justification pointers going upward from variable occurrences
to their bindings.

In the game semantics of the term $M$, the pointers for O and P
answers can be recovered by using the well-bracketing condition.

For O-question, the justification pointer always points to its
parent node in the computation tree.

For P-question, suppose P ask for the value of variable $x$. Then
there may be several choices for the destination of the pointer but
we claim that in the case of safe terms, it should point to the
closest parent node (in the path from the root to P-question) whose
order is greater than the order of $x$.
