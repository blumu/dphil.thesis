\def\cmptre#1{\tau(#1)}
\def\aux#1{\lceil #1\rceil}
\def\nf#1{\eta_{\sf nf}(#1)}

\section{Game semantics of safe $\lambda$-terms}

In this section we will prove that the safety condition
of section \ref{sec:safe_alt} leads to a pointer economy in the game
semantics: for safe $\lambda$-terms the pointers from the game semantics can be reconstructed uniquely from the moves of
the play.

The example of section \ref{subsec:ptrless_strat} gives the intuition.
Remember that in order to distinguish the terms $M_1$ and $M_2$,
we introduced pointers in strategies. In the safe $\lambda$-calculus
this ambiguity disappears because $M_2$ is not a safe term. Indeed, in the
sub-term $f (\lambda y . x)$, the free variable $x$
has the same order as $y$ but $x$ is not abstracted together
with $y$.


%\begin{enumerate}
%\item
%Is there any unsafe term whose game semantics is a strategy where
%pointers can be recovered?
%
%The answer is yes: take the term $T_i = (\lambda x y . y) M_i S$
%where $i =1..2$ and $\Gamma \vdash_s S : A$. $T_1$ and $T_2$ both
%$\beta$-reduce to the safe term $S$, therefore
%$\sem{T_1}=\sem{T_2}=\sem{S}$. But $T_1$ is safe whereas $T_2$ is
%unsafe. Since it is possible to recover the pointer from the game
%semantics of $S$, it is as well possible to recover the pointer from
%the semantics of $T_2$ which is unsafe.
%
%\item
%Is there any unsafe $\beta$-normal form whose game semantics is a
%strategy where pointers can be recovered?
%\end{enumerate}






\subsection{$\eta$-long normal form}

The $\eta$-expansion of $M: A\typear B$ is defined to be the term $\lambda x . M x : A\typear B$ where $x:A$ is a fresh variable.
It is easy to check that if $M$ is safe then $\lambda x . M x$ is also safe.

Consider the term $M : (A_1,\ldots,A_n,o)$, it can be expanded in several steps into
$\lambda \varphi_1 \ldots \varphi_l . M \varphi_1 \ldots \varphi_l$
where the $\varphi_i:A_i$ are fresh variables.

The $\eta$-normal form of a term is obtained by hereditarily $\eta$-expanding every sub-term occurring
at an operand position:

\begin{dfn}[$\eta$-long normal form]
A term is either an abstraction or it can be written uniquely as
$s_0 s_1 \ldots s_m$ where $m\geq0$ and $s_0$ is a variable, a
constant or an abstraction.

The $\eta$-long normal form of a term $M$ is denoted $\aux{M}$ and
is defined as follows:
\begin{eqnarray*}
\aux{x s_1 \ldots s_m : (A_1,\ldots,A_n,o)} &=& \lambda \overline{\varphi} . x \aux{s_1} \aux{s_2} \ldots \aux{s_m} \aux{\varphi_1} \ldots \aux{\varphi_n} \\
\aux{x s_1 \ldots s_m : o} &=& \lambda . x \aux{s_1} \aux{s_2} \ldots \aux{s_m} \\
\aux{(\lambda x . s_0) s_1 \ldots s_m } &=& (\lambda x . \aux{s_0}) \aux{s_1} \aux{s_2} \ldots \aux{s_m}
\end{eqnarray*}
where $m \geq 0$ and $x$ is either a variable or a constant.
\end{dfn}

The $\eta$-long normal form appeared in \citep{DBLP:journals/tcs/JensenP76}
under the name \emph{long reduced form}
and in \citep{DBLP:journals/tcs/Huet75}
under the name \emph{$\eta$-normal form}. It was then investigated in \citep{huet76}
under the name \emph{extensional form}.


A term can be represented by a tree defined formally by induction on the structure
of its $\eta$-long normal form as follows:

\begin{dfn}[Computation tree]
The computation tree associated to the term $s$ is noted
$\cmptre{s}$. It is obtained by applying the following rules
inductively \emph{on the $\eta$-long normal form} of $s$. In the
following $x$ is either a variable or a constant.
\begin{itemize}
\item the tree for $\lambda x_1 \ldots x_n. M$ where $M$ is not an abstraction is:
$$ \cmptre{\lambda x_1 \ldots x_n . M} =
  \pstree[levelsep=4ex]
    { \TR{\lambda x_1 \ldots x_n} }
    { \SubTree{\cmptre{M}}
    }
$$


\item the tree for $x s_1 \ldots s_n$ is:
$$ \cmptre{ x s_1 \ldots s_n} =
  \pstree[levelsep=4ex]
    { \TR{x @} }
    { \SubTree{\cmptre{s_1}} \SubTree[linestyle=none]{\ldots} \SubTree{\cmptre{s_n}}
    }
$$

\item the tree for $x$ is the single leaf $x$.

\item the tree for $(\lambda x.s_0) s_1 \ldots s_n$ is:
$$ \cmptre{(\lambda x.s_0) s_1 \ldots s_n} =
  \pstree[levelsep=4ex]
    { \TR{@} }
    {
    \SubTree{\cmptre{\lambda x.s_0}}    \SubTree{\cmptre{s_1}} \SubTree[linestyle=none]{\ldots} \SubTree{\cmptre{s_n}}
    }
$$
\end{itemize}
\end{dfn}

Example: if $x$ is a variable or a constant then
$ \cmptre{\lambda . x} =
  \pstree[levelsep=3ex]
    { \TR{\lambda } }
    { \TR{x}
    }$

The nodes (and leaves) of the tree are of three kinds:
\begin{itemize}
\item $\lambda$-node labeled $\lambda \overline{x}$. A $\lambda$-node represents several consecutive abstractions of variables.
\item application node labeled $@$
\item operator-application nodes labeled $x @$ where the operator $x$ is
either a variable or a constant.
\end{itemize}

A sub-tree of the computation tree represents a $\lambda$-term. We
define the map $\kappa : N \rightarrow \mathcal{T}$ where $N$
denotes the set of nodes and leaves of the computation tree
$\tau(s)$ and $\mathcal{T}$ denotes the set of $\lambda$-terms.
$\kappa$ associates to any node $n$ of the tree the $\lambda$-term
$\kappa(n)$ that is represented by the sub-tree of $\tau(s)$ rooted
at $n$. In particular if $r$ is the root of the tree $\tau(s)$ then
$\kappa(s) = \aux{s}$.



Consider the computation tree $\tau(s)$ of a term $s$ in $\eta$-long normal form. Then:
\begin{itemize}
\item One can check that nodes at even level are abstraction
node and nodes at odd level are either application nodes,
operator-application nodes, variable or constant nodes (the root level being numbered $0$).

\item Suppose that a variable $x$ occurs in $s$. The corresponding node in the tree has of one of the two following forms:
    \begin{itemize}
    \item $ \pstree[levelsep=3ex]
        { \TR{\lambda } }
        { \TR{x}
        }$ where $\ord{x} = 0$

    \item $ \pstree[levelsep=3ex]
                { \TR{x @} }
                { \TR{\lambda \overline{\xi_1}} \TR{\ldots} \TR{\lambda \overline{\xi_p}}}
        $ where $\ord{x} > 0$ and $x:(A_1,\ldots,A_p,o)$
    \end{itemize}

\item    Moreover for any abstraction node
        $ \pstree[levelsep=4ex]
            { \TR{\lambda \overline{\varphi}} }
            { \pstree[levelsep=3ex]
                {\TR{@^{[n]}}}
                {\TR{\lambda \overline{\xi_1}} \TR{\ldots} \TR{\lambda \overline{\xi_p}}}
            }
        $
    we have $\ord{\kappa(@^{[n]})}=0$

\end{itemize}


\subsubsection{Pointers and justified sequence of nodes}

\begin{dfn}[Binder]
Let $n$ be a node of the computation tree labelled $x$ or $x@$.
We say a node $n$ is bound by the node $m$ if $m$ is
the closest node in the path from $n$ to the root of the tree such that
$m$ is labelled $\lambda \overline{\xi}$ with $x\in \xi$.
$m$ is called the binder of $n$
\end{dfn}

\begin{dfn}[Enabling]
The enabling relation noted $\vdash$ over the set of nodes of the computation tree is defined as follows:
\begin{itemize}
\item a node $n$ labeled $x$ or $x@$ is enabled by its binder node:
\item a lambda node labelled $\lambda \overline{\xi}$ is enabled by its parent node labelled $@$ or $x@$ for some $x$:
\end{itemize}
If $m \vdash n$ we say that the node $n$ is enabled by $m$ or that $m$ enables $n$.
\end{dfn}



\begin{dfn}[Justified sequence of nodes]
A \emph{justified sequence of nodes} is an alternating sequence of lambda and non lambda nodes
from the computation tree $\tau(M)$ together with pointers.
Each node $n$ of the sequence that is either
a lambda-node, a $x$-node or a $x@$-node where $x$ is a variable not occurring free in $M$,
and that is not the root of the computation tree has a pointer to a previous node $m$ in the sequence
such that $m \vdash n$.

If $n$ points to $m$ we say that $m$ justifies $n$ and we note it:
$$\justseq{m & \ldots & n \apointto{ll}}$$

Hence the pointers in a justified sequence of nodes must be of one of the following forms:
$$\xymatrix @=12pt@M=0pt{
\lambda \overline{\xi} & \ldots & \xi_i @ \apointto{ll} \\
\lambda \overline{\xi}  & \ldots & \xi_i \apointto{ll} \\
@  & \ldots & \lambda \overline{\xi} \apointto{ll} \\
x@ & \ldots & \lambda \overline{\xi} \apointto{ll}}
$$
\end{dfn}

Note that the previous definition has been adapted to the case of non closed terms $M$. Suppose that $M$ is opened then
there may be free variables occurring as nodes in the computation tree. Such node occurring in a justified sequence has
no pointer.

We say that a node $n_0$ of a justified sequence is hereditarily justified by $n_p$ if there are nodes $n_1, n_2, \ldots n_{p-1}$ in
the sequence such that for all $i\in 0..p-1$, $n_i$ points to $n_{i+1}$.

If $N$ is a set of nodes and $s$ a justified sequence of nodes then we note $s \upharpoonright N$ the
subsequence of $s$ obtained after removing the nodes that are not hereditarily justified by nodes in $N$.
This subsequence is also a justified sequence of nodes.

Let $r$ be the root of the tree $\tau(M)$ then the justified sequence
$s \upharpoonright \{ r \}$ is such that no node is labelled by a free variable in $M$.


\begin{dfn}[Equality of justified sequence]
We note $s \jseq t$ to denotes that the two justified sequences $t$ and $s$ are the equal (same nodes and same pointers).
\end{dfn}

\todomargin{Definition traversal}
\begin{dfn}[Traversal]
\end{dfn}

\begin{dfn}[Reduced-traversal]
A \emph{reduced-traversal} is a justified sequence of nodes $r$ such that for some traversal $t$:
$$ r = t \upharpoonright \{ r \}$$
where $r$ denotes the root of the computation tree.

We say that $r$ is the reduction of the traversal $t$.
\end{dfn}

The nodes of a reduced-traversal are either lambda nodes or nodes labelled $x@$ or $x$ (no nodes of type $@$ or $f \in \Sigma$).
This is because nodes of type $@$ or $f \in \Sigma$ in a traversal have no pointer.

\subsection{Correspondence with game semantics}

By representing side-by-side the computation tree and the type arena of a term in $\eta$-normal form we observe
that for each question move of the arena there are corresponding nodes in the computation tree.

\begin{exmp}
Consider the following term $M \equiv \lambda f z . (\lambda g x . f (f x)) (\lambda y. y) z$ of type $(o \typear o) \typear o \typear o$.
Its $\eta$-long normal form is $\lambda f z . (\lambda g x . f (f x)) (\lambda y. y) (\lambda .z)$.
The computation tree is:

$$
\tree{\lambda f z}
{ \tree{@}
    {
        \tree{\lambda g x}
            { \tree{f@}{   \tree{\lambda}{ \tree{f@}{  \tree{\lambda}{\TR{x}}} }  }
            }
        \tree{\lambda y}{\TR{y}}
        \tree{\lambda}{\TR{z}}
    }
}
$$

The arena for the type $(o \typear o) \typear o \typear o$ is:
$$\tree{q^1}
{
    \tree{q^3}
        {  \tree{q^4}
                {\TR{a^4_1} \TR{\ldots}}
            \TR{a^3_1} \TR{\ldots} }
    \tree{q^2}
    { \TR{a^2_1} \TR{a^2_2}\TR{\ldots} }
    \TR{a_1} \TR{a_2}\TR{\ldots}
}
$$

\newlength{\yNull}
\def\bow{\quad\psarc{->}(0,\yNull){1.5ex}{90}{270}}

We now omit the answers moves when we represent the arena.
The arena is represented on the right and the computation tree on the left.

The dashed line defines a relation $\varphi$ from the set of question moves to nodes in the computation tree.
$\varphi$ maps each question to one or more nodes in the computation tree:
$$
\tree{ \Rnode{root} {\lambda f z}^{[1]} }
     {  \tree{@^{[2]}}
        {   \tree{\lambda g x ^{[3]}}
                { \tree{\Rnode{f}{f@^{[6]}}}{  \tree{\Rnode{lmd}\lambda^{[7]}}{ \tree{\Rnode{f2}{f@^{[8]}}} {\tree{\Rnode{lmd2}\lambda^{[9]}}{\TR{x^{[10]}}}}}  }
                }
            \tree{\lambda y ^{[4]}}{\TR{y}}
            \tree{\lambda ^{[5]}}{\TR{\Rnode{z}z}}
        }
    }
\hspace{3cm}
  \tree[levelsep=12ex]{ \Rnode{q1}q^1 }
    {   \pstree[levelsep=4ex]{\TR{\Rnode{q3}q^3}}{\TR{\Rnode{q4}q^4}}
        \TR{\Rnode{q2}q^2}
    }
\psset{nodesep=1pt,arrows=->,arcangle=-20,arrowsize=2pt 1,linestyle=dashed,linewidth=0.3pt}
\ncline{<-}{root}{q1} \aput*{:U}{\varphi}
\ncarc{->}{q2}{z}
\ncline{->}{q3}{f}
\ncline{->}{q4}{lmd}
\ncline{->}{q3}{f2}
\ncline{->}{q4}{lmd2}
$$

Consider the justified sequence of moves $s \in \sem{M}$:
\vspace{0.5cm}
 $$s =
\rnode{q1}{q}^1\
\rnode{q3}{q}^3\
\rnode{q4}{q}^4\
\rnode{q3b}{q}^3\
\rnode{q4b}{q}^4\
\rnode{q2}{q}^2
\bkptrc{q3}{q1}
\bkptrc[ncurv=0.5]{q3b}{q1}
\bkptrc{q4}{q3}
\bkptrc{q4b}{q3b}
\bkptrc[ncurv=0.5]{q2}{q1}
\in \sem{M}$$

There is a corresponding justified sequence of nodes in the computation tree:
\vspace{0.5cm}
$$r =
\rnode{q1}{\lambda f z} \cdot
\rnode{q3}{f@}^{[6]} \cdot
\rnode{q4}{\lambda^{[7]}} \cdot
\rnode{q3b}{f@}^{[8]} \cdot
\rnode{q4b}{\lambda^{[9]}} \cdot
\rnode{q2}{z}
\bkptra[ncurv=1]{60}{q3}{q1}
\bkptra[ncurv=1]{60}{q4}{q3}
\bkptra[ncurv=0.4]{75}{q3b}{q1}
\bkptra[ncurv=0.8]{70}{q4b}{q3b}
\bkptra[ncurv=0.4]{80}{q2}{q1}$$
such that $r_i = \varphi(s_i)$ for all $i < |s|$.

The sequence $r$ is in fact a reduced-traversal, it is the reduction of the following traversal of the tree:
\vspace{1cm}
$$t =
\rnode{q1}{\lambda f z} \cdot
\rnode{n2}{@^{[2]}} \cdot
\rnode{n3}{\lambda g x^{[3]}} \cdot
\rnode{q3}{f@}^{[6]} \cdot
\rnode{q4}{\lambda^{[7]}} \cdot
\rnode{q3b}{f@}^{[8]} \cdot
\rnode{q4b}{\lambda^{[9]}} \cdot
\rnode{n8}{x^{[10]}} \cdot
\rnode{n9}{\lambda^{[5]}} \cdot
\rnode{q2}{z}
\bkptra[ncurv=0.6]{60}{q3}{q1}
\bkptra[ncurv=1]{60}{q4}{q3}
\bkptra[ncurv=0.4]{75}{q3b}{q1}
\bkptra[ncurv=0.8]{70}{q4b}{q3b}
\bkptra[ncurv=0.4]{80}{q2}{q1}
\bkptra[ncurv=0.4]{60}{n3}{n2}
\bkptra[ncurv=0.4]{60}{n8}{n3}
\bkptra[ncurv=0.4]{60}{n9}{n2}
$$

We see on this example that the game semantics and the computation tree are somehow related to each other.
\end{exmp}


Let us now analyze precisely the relationship between the game semantics and the computation tree.

Let $\Gamma \vdash M : A$ be a term in $\eta$-long normal form.
Suppose $(M^{QA},\vdash)$ is the arena $\sem{A}$ where $M^{QA}$ is the set of moves and $\vdash$ is the enabling relation.
$M^Q \subseteq M^{QA}$ denotes the set of question-moves.
We note $(N,E)$ the computation tree $\tau(M)$ where $N$ is the set of nodes and leaves of the tree and $E$ is the parent-child relation.

We give an algorithm computing the partial function $\varphi^{-1} : N \rightarrow M^Q$ which maps
some question-moves to set of nodes in the computation tree.

\begin{dfn}[Relation between question-moves and nodes of the computation tree]
\label{def:phi_procedure}
We start by defining two preliminary procedures.
The procedure $f$ takes two parameters: $n$ is the index number of a
$\lambda$-node in the computation tree and $q$ is a question move of the arena
of $\Gamma \vdash M : A$ such that $q$ and $\kappa(n)$ have the same type.

Similarly we define the procedure $g$ that
takes two parameters: $n$ is the index number of a
$x@$-labelled node or a $x$-labelled node in the computation tree and $q$ is a question move
of the arena of $\Gamma \vdash M : A$ such that $q$ and $x$ have the same type.
\\

\noindent
\begin{description}
\item[\textbf{Procedure} $f(nd,q)$]
    where $nd$ is a $\lambda$-node.

    \begin{itemize}
    \item If $\ord{\kappa(nd)} = 0$ then the term is of ground type therefore
    the game for $M$ is played on the flat arena
    with only one question $q$. Moreover the node $nd$ of the computation tree is labelled with $\lambda$.\\
    \textbf{return} $\{ nd \mapsto q \}$.

    \item $\ord{\kappa(nd)} > 0$. The computation tree and the arena
    have the following form:
    $$ \tree[levelsep=6ex]{ \Rnode{r}\lambda \overline{\xi}  ^{[nd]}}
        {
            \tree[levelsep=6ex]{x@^{[nd+1]}}
            {   \TR{\lambda^{[\ldots]}} \TR{\ldots} \TR{\lambda^{[\ldots]}}
            }
        }
    \hspace{3cm}
    \tree{ \Rnode{q0}q }
        {
            \tree[linestyle=dotted]{q^1}{\TR{} \TR{} }
            \tree[linestyle=dotted]{q^2}{\TR{} \TR{} }
            \TR{\ldots}
            \tree[linestyle=dotted]{q^n}{\TR{} \TR{} }
        }
    \psset{nodesep=1pt,arrows=->,arcangle=-20,arrowsize=2pt 1,linestyle=dashed,linewidth=0.3pt}
    \ncline{->}{r}{q0}
    \ncarc{->}{q2}{z}
    \ncline{->}{q3}{f}
    \ncline{->}{q4}{lmd}
    \ncline{->}{q3}{f2}
    \ncline{->}{q4}{lmd2}
    $$

    such that $\Gamma, \overline{\xi} \vdash \kappa(x@^{[nd+1]}) : o$.

    For each of the abstracted variable $\xi_i$ there is a corresponding question move $q_i$ of the same order
    in the arena.  Each free occurrence of the variable $\xi_i$ is mapped to the move $q_i$
    by the procedure $g$.

    $$\mathbf{return} \ \{ nd \mapsto q \}
    \union
    \Union_{\stackrel{i=1..n}{\xi_i^{[k]} \in Desc(nd)}} g ( k, q_i)
    \quad \union \quad
    \Union_{\stackrel{i=1..n}{@\xi_i^{[k]} \in Desc(nd)}} g ( k, q_i)
    $$
    where $Desc(nd)$ is the set of descendants of  node $nd$
    (nodes $m$ such that there is a path from node $nd$ to node $m$ in the computation
    tree).

    \end{itemize}

\item[\textbf{Procedure} $g(nd,q)$]\  \\
The procedure is not defined on $\lambda$-nodes or $@$ nodes.
This is ok since all the calls to $g$ in $f$ are of the type $g(nd,q)$ where $nd$ denotes a $@x$-node or a $x$-node.

\begin{itemize}
\item[case 1] Suppose that $nd$ is labelled with $x$ then we must have $x:0$. \textbf{return}
$\{ nd \mapsto q \}$.

\item[case 2] If $nd$ is labelled with $x@$ then $x:(A_1|\ldots|A_m|o)$.
The computation tree and the arena  have the following form:

    $$\tree[levelsep=6ex]{\Rnode{r}{x@^{[nd]}}}
        {   \tree{\TR{\lambda^{[k_1]}}}{\vdots} \TR{\ldots}
        \tree{\TR{\lambda^{[k_m]}}}{\vdots}
        }
    \hspace{3cm}
    \tree{ \Rnode{q0}q }
        {
            \tree[linestyle=dotted]{\Rnode{q1}{q^1}}{\TR{} \TR{} }
            \tree[linestyle=dotted]{\Rnode{q2}{q^2}}{\TR{} \TR{} }
            \TR{\ldots}
            \tree[linestyle=dotted]{\Rnode{qm}{q^m}}{\TR{} \TR{} }
        }
    \psset{nodesep=1pt,arrows=->,arcangle=-20,arrowsize=2pt 1,linestyle=dashed,linewidth=0.3pt}
    \ncline{->}{r}{q0}
    \ncarc{->}{q2}{z}
    \ncline{->}{q3}{f}
    \ncline{->}{q4}{lmd}
    \ncline{->}{q3}{f2}
    \ncline{->}{q4}{lmd2}
    $$

    such that $\Gamma, \overline{\xi} \vdash \kappa(x@^{[nd]}) : o$.

    For each of the children node of $nd$
    there is a corresponding question move $q_i$ of the same type
    in the arena.
    $$\mathbf{return} \ \{ nd \mapsto q \} \union
    \Union_{i=1..m} f ( k_i, q_i)
    $$
\end{itemize}
\end{description}

We define $\varphi^{-1}$ as follows:
$$\varphi^{-1} = f(0,q^0)$$
where the index $0$ denotes the root of the computation tree and $q^0$ is the root of the arena
(in the game semantics of simply-typed lambda calculus the arenas have
a single root).
\end{dfn}

The function $\varphi^{-1} : N \rightarrow M^Q$ is a partial function. One can check that its inverse
$\varphi : M^Q \rightarrow \mathcal{P}(N)$ is a total function.
Moreover $\varphi$ is such that player O questions are associated to $\lambda$-nodes
and player P questions are associated to $x @$-nodes or $x$-nodes.


We extend the function $\varphi^{-1}$ from nodes to justified sequences: suppose $t$
is a justified sequences of nodes $t = t_0 t_1 \ldots$. Then $\varphi^{-1}(t)$ is defined to
be the following justified sequence of question-moves of the arena $\sem{A}$:
$$\varphi^{-1}(t) = \varphi^{-1}(t_0)\ \varphi^{-1}(t_1)\  \varphi^{-1}(t_2) \ldots$$
where the pointers of the justified sequence of move $\varphi^{-1}(t)$ are defined to be exactly those
of the justified sequences of nodes $t$.

\begin{dfn}[Question-move filtering]
Suppose $s$ is a justified sequence of moves then we note $\tilde{s}$
the subsequence of $s$ consisting of question-moves only:
\begin{eqnarray*}
\tilde{} &: L_A &\longrightarrow (M^Q)^* \\
& s &\longmapsto \tilde{s} = s \upharpoonright M^Q
\end{eqnarray*}
where $M^Q$ denotes the set of question-moves.
$\tilde{s}$ is also a justified sequence of move (there is no dangling pointer since questions-moves points to other question-moves).
If $s = u\upharpoonright A,B$ then clearly $\tilde{s} = \tilde{u} \upharpoonright A,B$.
\end{dfn}


\begin{prop}[Relation between game semantics and reduced-traversals]
\label{prop:rel_gamesem_redtrav_closed}
Let $\emptyset \vdash M : A$ be a closed term.
Let $r$ denotes the root of $\tau(M)$. If $s$ is a justified sequence of moves such
that $s \in \sem{\Gamma \vdash M:A}$ then there is a reduced-traversal of nodes of $\tau(M)$
$t = n_0 n_1 \ldots$ such that:
 $$\tilde{s} \jseq  \varphi^{-1} (t)$$
\end{prop}
\begin{proof}
%Let $\Gamma = \{y_1, : B_1, \ldots , y_r : B_r \}$ and

Let us assume that $M$ is already in $\eta$-long normal form.
Let $s$ be as justified sequence of moves in $\sem{M}$.

We proceed by induction on the height of of
the tree $\tau(M)$ and by case analysis on the structure of the term.

\begin{itemize}
  \item (variable) $M = x$. This case does not happen since $M$ is a closed term.

    \item (application) The term $M$ is of the form $\emptyset \vdash N_0 N_1 \ldots N_p : o$ where $N_0$ is not
    a variable (since $M$ is closed) and:
    \begin{eqnarray*}
    \emptyset &\vdash& N_0 : (A_1,\ldots,A_p,o)\\
    \emptyset &\vdash& N_i : A_i \mbox{ for } i \in 1..p
    \end{eqnarray*}

    The tree $\tau(M)$ has the following form:
    $$ \tree[levelsep=6ex]{\lambda}
        { \tree[levelsep=6ex]{@^{[1]}}
            {   \TR{\tau(N_0)} \TR{\ldots} \TR{\tau(N_p)}}}
    $$

    We have:
    $$\sem{M} = \sem{\emptyset \vdash N_0 N_1 \ldots N_p : o} = \overbrace{\langle \sem{\emptyset \vdash N_0}, \ldots, \sem{\emptyset \vdash N_p} \rangle}^\sigma ; ev_{A_1,\ldots,A_p,o}$$
    with $\sigma : \textbf{1} \longrightarrow B$ and $ev_{A_1,\ldots,A_p,o} : B \longrightarrow \sem{o}$ where
    $$ B = \sem{(A_1,\ldots,A_p,o)} \times \sem{A_1} \times \ldots \times \sem{A_p} $$

    Since $s \in \sem{M} = \sem{\emptyset \vdash N_0 N_1 \ldots N_p : o}$ we have:
    \begin{equation*}
        s = u \upharpoonright \textbf{1},\sem{o} = u \upharpoonright \sem{o}
    \end{equation*}
    for some $u \in \sigma \parallel ev_{A_1,\ldots,A_p,o}$.

    Let $q_C$ denotes the only question of the arena $\sem{o}$ then $s \in \{ q_C \}^*$.
    We construct the justified sequence $t$ by replacing the moves $q_C$ in $\tilde{s}$ by the root node
    $\lambda \overline{\xi}$:
    $$ t \jseq  \tilde{s} [\lambda \overline{\xi} / q_C].$$

    Since $\tilde{s} \in \{ q_C \}^*$ and $\varphi^{-1}_{M}(\lambda \overline{\xi}) = q_C$, we have:
    $$\varphi^{-1}_{M} (t) = \varphi^{-1}_{M}( \tilde{s} [\lambda \overline{\xi} / q_C] )
        = \tilde{s} [\lambda \overline{\xi} / q_C] [q_C / \lambda \overline{\xi}]
        = \tilde{s}$$
    Since all the moves in $\tilde{s}$ are initial, they do not have pointers. Hence
    $$ \varphi^{-1}_{M} (t) \jseq \tilde{s} $$


    \item (variable-abstraction)
        $M = \lambda \overline{\xi} . x$.  Then $x$ must be of ground type $o$ and since $M$ is closed
        $x = \xi_i \in \overline{\xi}$ .
        Then $\tau(M)$ has the following shape:
        $$ \tree[levelsep=6ex]{ \lambda \overline{\xi}^{[0]} }{\TR{x^{[1]}}}$$
        The arena is of the following form (only question moves are represented):
        $$ \tree{ q }
        {   \tree[linestyle=dotted]{q^1}{\TR{} \TR{} }
            \tree[linestyle=dotted]{q^2}{\TR{} \TR{} }
            \TR{\ldots}
            \tree[linestyle=dotted]{q^n}{\TR{} \TR{} }
            \TR{q^x}
        }$$
        where $q^x$ denotes the root of the flat arena $\sem{o}$.
        We have:
        $$ \sem{M} = \sem{\emptyset \vdash \lambda \overline{\xi} . \xi_i} = \Lambda^n(\sem{\overline{\xi} \vdash  \xi_i}) = \Lambda^n(\pi_i)$$
        where $\pi_i$ denotes the $i$-th projection

        Therefore the only question-moves that can occur in $s \in \sem{M}$ are the initial question $q$ and the question $q_x$.
        We take $t \jseq \tilde{s} [0/q, 1/q^x]$ which is a valid justified sequence of nodes since
        node $1$ ($x$) is bound by node $0$ ($\lambda \overline{\xi}$). Clearly we have $\varphi^{-1}(t) \jseq \tilde{s}$.

    \item (application-abstraction) The term $M$ is of the form $M = \lambda \overline{\xi} . x N_1 \ldots N_p$ where
    \begin{eqnarray*}
    \emptyset &\vdash& M : (X_1,\ldots,X_n,o) \\
    \Gamma &\vdash& x N_1 \ldots N_p : o \\
    \Gamma &\vdash& x : (A_1,\ldots,A_p,o) \\
    \Gamma &\vdash& N_i : A_i \mbox{ for } i \in 1..p \\
    \Gamma &=& \overline{\xi} : \overline{X}
    \end{eqnarray*}

    The tree $\tau(M)$ has the following form:
    $$ \tree[levelsep=6ex]{\lambda \overline{\xi}^{[0]}}
        { \tree[levelsep=6ex]{x @^{[1]}}
            {   \TR{\tau(N_0)} \TR{\ldots} \TR{\tau(N_p)}}}
    $$
    where $x \in \overline{\xi}$ (since $M$ is a closed term).

    We have:
    \begin{eqnarray*}
    \sem{ \emptyset \vdash M} &=& \Lambda^n( \sem{\Gamma \vdash x N_1 \ldots N_p : o} ) \\
    \sem{\Gamma \vdash x N_1 \ldots N_p : o} &=& \overbrace{\langle \sem{\Gamma \vdash N_0}, \ldots, \sem{\Gamma \vdash N_p} \rangle}^\sigma ; ev_{A_1,\ldots,A_p,o}
    \end{eqnarray*}

    with $\sigma : A \longrightarrow B$ and $ev_{A_1,\ldots,A_p,o} : B \longrightarrow C$, the arena being defined as follows:
    \begin{eqnarray*}
        A &=& \sem{\Gamma} = \sem{X_1} \times \ldots \times \sem{X_n}\\
        B &=& H_0 \times H_1 \times \ldots \times H_p \\
        C &=& \sem{o}\\
        H_0 &=& \sem{(A_1,\ldots,A_p,o)} \\
        H_j &=& \sem{A_j} \mbox{ for }j \in 1..p
    \end{eqnarray*}

    Since $s \in \sem{M}$ we must have $s\in \sem{\Gamma \vdash x N_1 \ldots N_p : o}$ therefore:
    \begin{equation}
        s = u \upharpoonright A, C \label{eq:def_s}
    \end{equation}
    for some $u \in \sigma \parallel ev_{A_1,\ldots,A_p,o}$, more precisely for some $u$ such that:
    \begin{eqnarray}
        u &\in& int(A,B,C) \nonumber \\
        w = u \upharpoonright A,B & \in& \sigma       \label{eq:def_w}\\
        u \upharpoonright B,C & \in & ev_{A_1,\ldots,A_p,o} \nonumber
    \end{eqnarray}

    $w \in \sigma$ implies that for some $j \in 1..p$:
    \begin{eqnarray}
        z = w \upharpoonright A, H_j &\in& \sem{\Gamma \vdash N_j}  \label{eq:def_z} \\
        \mbox{ and for every } k\neq j &:& w \upharpoonright H_k = \epsilon \label{eq:b}
    \end{eqnarray}

    We cannot use the induction hypothesis on $\Gamma \vdash N_j : A_j$ because it is not a closed terms!
    We therefore consider the closed term $N'_j = \lambda \overline{\xi} . N_j$:
        $$\emptyset \vdash N'_j : (X_1, \ldots, X_n,A_j)$$

    We have $z \in \sem{N_j} = \Lambda^n(\sem{\Gamma \vdash N_j})$. But $\Lambda^n(\sem{\Gamma \vdash N_j})$
    and $\sem{\Gamma \vdash N_j}$ are the same strategies up to an isomorphism that
    retaggs the moves. Hence the equation \ref{eq:def_z} gives $z \in \sem{N'_j}$ where $N'_j$ is a closed term. Since
    the term $N'_j$ has the same height as the term $N_j$ which is strictly smaller than the height
    of the term $M$, we can use the induction hypothesis on $N'_j$:
    there is a justified sequence $t'$ of the tree $\tau(N'_j)$
    such that $\varphi^{-1}_{N'_j}(t') \jseq \tilde{z}$.




    From equation \ref{eq:def_w} we have:
    \begin{equation}
        \tilde{w} = \tilde{u} \upharpoonright A,B
        \label{eq:def_wtilde}
    \end{equation}

    From equation \ref{eq:b}, for $k\neq j : w \upharpoonright H_k = \epsilon$. Since $B =
    H_0 \times H_1 \times \ldots \times H_p$ this implies that moves in $w$ hereditarily justified by moves in $B$ are in fact all
    hereditarily justified by moves in $H_j$.  Hence $z = w \upharpoonright A, H_j = w \upharpoonright A, B$ and:
    \begin{eqnarray*}
            \tilde{z} =& \tilde{w} \upharpoonright A,B \\
            =& (\tilde{u} \upharpoonright A,B) \upharpoonright A,B &\mbox{(by eq \ref{eq:def_wtilde})} \\
            =& \tilde{u} \upharpoonright A,B \\
            =& \tilde{w}  & \mbox{(by eq \ref{eq:def_wtilde})}
    \end{eqnarray*}


    Hence $\varphi^{-1}_{N'_j}(t') \jseq \tilde{w}$.


    For $i \in 1..p$, the tree $\tau(N_i)$ (left) and $\tau(N'_i)$ (right) have the following form:
    $$ \tree[levelsep=6ex]{ \lambda \overline{y_i}^{[n_i]} }
    { \tree[levelsep=6ex,linestyle=dotted]{}
        {   \TR{} \TR{} \TR{} }}
    \hspace{2cm}
    \tree[levelsep=6ex]{ \lambda \overline{\xi}\  \overline{y_i}^{[n_i]} }
    { \tree[levelsep=6ex,linestyle=dotted]{}
        {   \TR{} \TR{} \TR{} }}
    $$
    where only the label of the root differs between the two trees.

    From the justified sequence $t'$ of $\tau(N'_j)$ we construct the justified sequence $t$ of
    $\tau(M)$ by mapping the nodes of $\tau(N'_j)$ to the corresponding node in the subtree $\tau(N_j)$
    of $\tau(M)$. Moreover we change all the pointers going from a node $\xi_i$ or $\xi_i @$ to the root node of $\tau(N'_j)$
    into a pointer starting from the corresponding node $\xi_i$ or $\xi_i @$ to the root of the tree $\tau(M)$.
    $t$ is a valid justified sequence of nodes of $\tau(M)$ since the nodes labelled $\xi_i$ and $\xi_i @$
    are bound by the root $\lambda \overline{\xi}$.

    The function $\varphi^{-1} : Nodes \rightarrow M^Q$ has been defined by an inductive procedure
    guaranteeing that $\varphi^{-1}_{N_j}$ is equal to the function $\varphi^{-1}_{M}$ restricted to
    the set of nodes of $\tau(N_j)$. Hence we have:
    \begin{equation}
    \varphi^{-1}_{M}(t) \jseq \tilde{w}. \label{eq:def_t}
    \end{equation}

    Let $q_C$ denotes the only question of the arena $C = \sem{o}$.
    We can construct from $t$ the justified sequence $t^\ast$ as follows:
    we insert the root node $\lambda \overline{\xi}$ at the positions where the move $q_C$ occurs in $\tilde{u}$.
    Since $\varphi^{-1}_{M}(\lambda \overline{\xi}) = q_C$, we have:
    $$\varphi^{-1}_{M}(t^\ast) = \tilde{u}$$

    Now we transfer links from $\tilde{u}$ to $t^\ast$: for each
    move $\tilde{u}_k$ pointing to $\tilde{u}_i = q_C$ in $\tilde{u}$ for some $0 \leq i \leq k$,
    we add a pointer in $t^\ast$ going from $t^\ast_k$ to $t^\ast_i = \lambda \overline{\xi}$.

    Then the pointers of the traversal $t^\ast$ and the sequence of move $\tilde{u}$ are the same and we obtain:
    \begin{equation}
    \varphi^{-1}_{M}(t^\ast) \jseq \tilde{u}. \label{eq:def_tstar}
    \end{equation}

    We need to ensure that the sequence $t^\ast$ is a valid justified sequence of $\tau(M)$.
    This requires us to check the validity of the the pointers introduced in $t^\ast$.
    Suppose $u_k$ points to $q_C$ in $\tilde{u}$ then $u_k$ belongs to the arena $A$. More precisely it is one
    of the root $q^1 \ldots q^n$ of the arena $A$. Therefore $u_k$ must belong to the sequence $\tilde{w}$ and
    for some $r \geq 0$, $u_k = \phi^{-1}_{N'_j}(t'_r)$ and $t^\ast_k = t'_r$.

    The arena of the term $\emptyset \vdash N'_j = \lambda \overline{\xi} . N_j$ is of the following form (only question moves are represented):
        $$ \tree{ q }
        {   \tree[linestyle=dotted]{q^1}{\TR{} \TR{} }
            \tree[linestyle=dotted]{q^2}{\TR{} \TR{} }
            \TR{\ldots}
            \tree[linestyle=dotted]{q^n}{\TR{} \TR{} }
        }$$
    We observe from the definition of $\phi^{-1}_{N'_j}$ that the
    only node that are mapped to the question  $q^1 \ldots q^n$ are labelled $\xi_i$ or $\xi_i @$ for some $i\in 1..n$.
    Consequently $t^\ast_k = \xi_i$ and the node $t^\ast_k$ is bound by the node $\lambda \overline{\xi}$.
    Hence the added pointer is indeed valid.\\


    For any justified sequence of nodes $t$ we define the justified sequence of nodes $t \upharpoonright A,C$
    to be the subsequence of $t$ constituted of nodes $n$
    verifying $\varphi^{-1}_M(n) \in A \union C$ together with the same
    pointers as the justified sequence of move $\varphi^{-1}_M(t) \upharpoonright A,C$ (here the
    operator $\upharpoonright$ denotes the filtering operator
    defined on justified sequence of moves).

    The effect of this transformation is to remove from $t^\ast$ all the elements at positions $i$ such that $\tilde{u}_i \in B$.
    We obtain a justified sequence of nodes $t^\dagger = t^\ast \upharpoonright A,C$
    verifying
        $$\varphi^{-1}_{M}(t^\dagger) \jseq \tilde{s}$$


    \textbf{Example}: Let us illustrate each step of the demonstration on a short example.
        For clarity only some relevant pointers are specified in the following justified sequences.
    $$
    \xymatrix @=3pt{
        & \tilde{u} &=& q_C & m_1 & q_C & m_3 & m_4 & q_C & m_6 & m_7 & q_C & \ldots \\
        & &\in& C & B & C & A & A & C & B & B & C & \ldots \\
        \mbox{ (equation \ref{eq:def_s})} & \tilde{s} &=& q_C & & q_C & m_3  \ar@/_1pc/[l] & m_4 & q_C &  & & q_C & \ldots \\
        \mbox{ (equation \ref{eq:def_w})} & \tilde{w}  &=&  & m_1 &  & m_3 & m_4 &  & m_6 & m_7 &  & \ldots  \\ \\ \\
        \mbox{ (equation \ref{eq:def_t})} & t &=&  & n_1\ar[uuu]^{\varphi^{-1}_{N'_j}} &  & n_3 \ar[uuu] & n_4\ar[uuu] & & n_6\ar[uuu] & n_7\ar[uuu] &  & \ldots \\ \\
        & \tilde{u}  &=& q_C & m_1 & q_C & m_3 & m_4 & q_C & m_6 & m_7 & q_C & \ldots  \\ \\ \\
        \mbox{ (equation \ref{eq:def_tstar})} & t^\ast &=& \lambda \ar@{=>}[uuu]^{\varphi^{-1}_{M}} & n_1\ar[uuu]^{\varphi^{-1}_{N'_j}} & \lambda\ar@{=>}[uuu] & n_3 \ar[uuu] \ar@/_1pc/[l] & n_4\ar[uuu] & \lambda\ar@{=>}[uuu] & n_6\ar[uuu] & n_7\ar[uuu] & \lambda\ar@{=>}[uuu] & \ldots \\
        t^\ast\upharpoonright A,C =& t^\dagger &=& \lambda &  & \lambda & n_3\ar@/_1pc/[l] & n_4 & \lambda & &  & \lambda & \ldots \\
        }
    $$
    Note that there is only one question $q_C$ in the arena $C$ therefore $m_2 = m_5 = m_8 = q_C$.
    It is easy to check that $\varphi^{-1}(t') \jseq \tilde{s}$
\end{itemize}
\end{proof}

The previous proposition has an equivalent for open terms. Its proof is omitted here. It
follows the same step as the previous proof.
\begin{prop}[Relation between game semantics and reduced-traversals for open terms]
\label{prop:rel_gamesem_redtrav_opened}
Suppose $\Gamma \vdash M : A$. Let $r$ denotes the root of $\tau(M)$. If $s$ is a justified sequence of moves such
that $s \in \sem{\Gamma \vdash M:A}$ then there is a reduced-traversal of nodes of $\tau(M)$
$t = n_0 n_1 \ldots$ such that:
 $$\tilde{s} \upharpoonright \sem{A} \jseq  \varphi^{-1} (t \upharpoonright \{r \})$$
and
$$\forall i \geq 0 . \tilde{s}_i = \varphi^{-1}(t_i) \not \in \sem{A}
\implies \left\{
        \begin{array}{ll}
            t_i = x \mbox{ or } t_i = x@, \hbox{where } x \in \Gamma \\
            \varphi^{-1}(t \upharpoonright \{t_i\}) \jseq \tilde{s} \upharpoonright \{ \varphi^{-1}(t_i) \}
        \end{array}
        \right.
$$
\end{prop}

\subsection{Pointers in the game semantics of safe terms are recoverable}

\begin{dfn}[P-view of justified sequence of nodes]
The P-view of a justified sequence of nodes $t$ of $\tau{M}$ noted $\pview{t}$ is defined as follows:
\begin{eqnarray*}
 \pview{t \lambda \overline{\xi} }  &=&  \pview{t}\  \lambda \overline{\xi} \qquad \mbox{where $\lambda \overline{\xi}$ is the root of } \tau{M}\\
 \pview{\justseq{t & n & \ldots & \lambda \overline{\xi} }}  &=& \pview{t} n \lambda \overline{\xi}\\
\def\justseq#1{\xymatrix @=12pt@M=0pt{ #1 }}
\def\pointto#1{\ar@/_/[#1]}
\def\apointto#1{\ar@/_1pc/[#1]}
 \pview{\lambda \overline{\xi}} &=&  \lambda \overline{\xi} \qquad \mbox{elsewhere}\\
\end{eqnarray*}
\end{dfn}

\begin{property}
The P-view of a justified sequence of nodes that does not contain any node labelled $@$ is
subsequence of a path from the root to a node in the computation tree.
\end{property}



The order of a lambda-node $\lambda \overline{\xi}$ in the computation tree is defined to be $\max_{i=1..n}
\ord{\xi_i}$.

The computation tree of safe terms verifies a property called regularity:
\begin{property}[Computation tree of safe terms are regular]
\label{proper:regularity}
If $M$ is a safe term then any node of the computation tree $\tau(M)$ labelled $x$ or $x@$ where
$x$ is a variable bound in $M$ is bound by the first $\lambda$-node in the path to the root that has
order greater or equal to $\ord{x}$.
\end{property}
\begin{proof}
For a safe term the first $\lambda$-node $\lambda \overline{\xi}$ in the path to the root such that
$x \in \overline{\xi}$ is also the first $\lambda$-node whose order is greater or equal to $\ord{x}$.
\end{proof}


\begin{prop}
The pointers in the game semantics of safe terms are
uniquely recoverable.
\end{prop}

\begin{proof}
Let $\Gamma \vdash M : A$ be an opened safe term where $\Gamma = y_1:Y_1, \ldots y_n:Y_n$.
We assume that $M$ is in $\eta$-long normal form. It is a safe assumption
because safety is preserved by $\eta$-expansion.

Consider a justified sequence of move $s \in \sem{\Gamma \vdash M}$. Firstly,
we remark that the pointers for O and P answer moves in $s$ can all be
recovered thanks to the well-bracketing condition.

Consider the closed term $M'  = \lambda \overline{y} . M$.
Up to a retagging of the moves, the justified sequence of moves $s$ belongs to the strategy
$\sem{\vdash \lambda \overline{y} . M} = \Lambda^n(\sem{\Gamma \vdash M})$.
Proposition \ref{prop:rel_gamesem_redtrav_closed}tells us that
there is a justified sequence of nodes $t$ of $\tau(M')$ such that:
$$\phi^{-1}_{M'}(t) \jseq \tilde{s}.$$



For O-question, the justification pointer always points to its
parent node in the computation tree.

Suppose $q \in s$ is a P-question then because of property \ref{proper:regularity} it should
point to the closest node in the path from the root to P-question
whose order is greater than the order of $x$.

\todomargin{to finish}
\end{proof}
