\clearpage

\section{Non homogeneous safe $\lambda$-calculus - VERSION B}

In section \ref{sec:safe_alt}, we have presented a safe lambda
calculus in the setting of homogeneous types. In this section, we
give a general notion of safety for the simply typed
$\lambda$-calculus. The rules we give here do not assume homogeneity
of the types.

We will call safe terms the simply typed lambda terms that are
typable within the following system of formation rules:

\subsection{Rules}

 We use a set of sequents of the form $\Gamma \vdash^{i} M :
A$ where the meaning is ``variables in $\Gamma$ have orders at least
$\ord{A}+i$'' where $i \leq 0$. The following set of rules are
defined for $i \in \nat$:

$$ \rulename{var} \quad  \rulef{}{x : A\vdash^{0} x : A} $$

$$ \rulename{seq} \quad \rulef{\Gamma \vdash^0 M : A}{\Gamma \vdash^{-1} M : A}
\qquad  \rulename{wk^i} \quad  \rulef{\Gamma \vdash^i M : A}{\Gamma , x : B \vdash^i M : A} \quad \ord{B} \geq \ord{A} + i $$

$$ \rulename{app} \quad  \rulef{\Gamma \vdash^{-1} M : (A,\ldots,A_l,B)
                                        \qquad \Gamma \vdash^{0} N_1 : A_1
                                        \quad \ldots \quad \Gamma \vdash^{0} N_l : A_l  }
                                   {\Gamma  \vdash^0 M N_1 \ldots N_l : B}
                                    \qquad
                                   \forall y \in \Gamma : \ord{y} \geq \ord{B}$$

$$ \rulename{abs^i} \quad  \rulef{\Gamma, \overline{x} : \overline{A} \vdash^{i} M : B}
                                   {\Gamma  \vdash^{0} \lambda \overline{x} : \overline{A} . M : (\overline{A},B)} \qquad
                                   \forall y \in \Gamma : \ord{y} \geq \ord{\overline{A},B}$$


Note that:
\begin{itemize}
\item $(\overline{A},B)$ denotes the type $(A_1,A_2, \ldots, A_n, B)$;
\item all the types appearing in the rule are not required to be homogeneous. For instance
for the type $(A,\ldots,A_l,B)$ in the rule $\rulename{app}$ it is not necessary that $\ord{A_l} \geq \ord{B}$;
\item the environment $\Gamma, \overline{x}:\overline{A}$ is not stratified. In particular, variables in $\overline{x}$ do not necessarily have the same order;
\item In the abstraction rule, the side-condition imposes that at least all the variable of the lowest order
in the context are abstracted. However other variables can also be
abstracted together with the lowest order variables. Moreover there
is not constraint on the order on which the variables are abstracted
(contrary to what happens in the homogeneous case);
\item The sequents $\Gamma \vdash^0 M$ are the \emph{safe terms} that we want to generate.
Other terms are only used as intermediate sequents in a proof tree.
\end{itemize}

\todobox{Problem: with this definition, safety is not preserved by $\eta$-expansion. For instance
if $M:(A_1,\ldots,A_l,o)$ where $(A_1,\ldots,A_l,o)$ is not homogeneous. The term
$M x_1 \ldots x_l$ where $x_i :A_i$ will not be a valid term for this system of rules. Therefore the
$\eta$-expansion will not be a valid term neither.}

\begin{exmp}
Suppose $x:o$, $f:o\rightarrow o$ and $\varphi:(o\rightarrow
o)\rightarrow o$ then the term $$\vdash^0 \lambda x f \varphi .
\varphi : o \rightarrow (o\rightarrow o) \rightarrow ((o\rightarrow
o)\rightarrow o) \rightarrow (o\rightarrow o)\rightarrow o$$ is
valid although its type is not homogeneous
\end{exmp}


\begin{lem}[Basic properties]
\label{lem:nonhomosafe_basic_prop} Suppose $\Gamma \vdash^i M : B$
is a valid judgment then every variable in $\Gamma$ has order at
least $ord(M)+i$.
\end{lem}
\begin{proof}
An easy induction. The step case for the application is: suppose
$\Gamma \vdash^{i+\delta} M N : B$ where $\Gamma \vdash^i M :
A\rightarrow B$. Then by induction we have $\forall y \in \Gamma :
\ord{y} \geq \ord{A\rightarrow B} + i = \max(1+\ord{A}, \ord{B})+i =
\delta + \ord{B} + i \geq \min(i+\delta,0) + \ord{B}$.
\end{proof}

\subsection{Substitution in the safe lambda calculus}

The traditional notion of substitution, on which the
$\lambda$-calculus is based on, is the following one:
\begin{dfn}[Substitution]
\label{dfn:subst}
\begin{eqnarray*}
c \subst{t}{x} &=& c \quad \mbox{where $c$ is a $\Sigma$-constant}\\
x \subst{t}{x} &=& t\\
 y\subst{t}{x} &=& y \quad \mbox{for } x \not \neq y,\\
(M_1 M_2) \subst{t}{x} &=& (M_1 \subst{t}{x}) (M_2 \subst{t}{x})\\
(\lambda x . M) \subst{t}{x} &=& \lambda x . M\\
(\lambda y . M) \subst{t}{x} &=& \lambda z . M \subst{z}{y}
\subst{t}{x} \mbox{where $z$ is a fresh variable and $x\not = y$}
\end{eqnarray*}
\end{dfn}

In the setting of the safe lambda calculus, the notion of
substitution can be simplified. Indeed, we remark that for safe
$\lambda$-terms there is no need to rename variables when performing
substitution:

\begin{lem}[No variable capture lemma]
\label{lem:noclash} There is no variable capture when performing
substitution on a safe term.
\end{lem}
\begin{proof}
Suppose that a capture occurs during the substitution $M[N/\varphi]$
where $M$ and $N$ are safe. Then the following conditions must hold:
\begin{enumerate}
\item $\varphi:A, \Gamma \vdash^0 M$,
\item $\Gamma' \vdash^0 N$,
\item there is a subterm $\lambda \overline{x} . L$ in $M$ (where the abstraction is taken as wide as possible) such that:
\item $\varphi \in fv(\lambda \overline{x} . L)$ (and therefore $\varphi \in fv(L)$),
\item $x \in fv(N)$ for some $x \in \overline{x}$.
\end{enumerate}

By lemma \ref{lem:nonhomosafe_basic_prop} and (v) we have:

\begin{equation}
\ord{x} \geq \ord{N} = \ord{\varphi} \label{eq:xigeqphi}
\end{equation}

The abstraction $\lambda \overline{x} . L$ (taken as large as
possible) is a subterm of $M$, therefore there is a node $\Sigma
\vdash^u \lambda \overline{x} . L$  for some $u$ in the proof tree
of $\varphi:A, \Gamma \vdash^0 M$.

There are only three kinds of rules that can produce an abstraction:
$\rulename{abs^i}$, $\rulename{seq}$ and $\rulename{wk^i}$. The only
one that can introduce the abstraction is $\rulename{abs^i}$.
Therefore the proof tree has the following form:

$$ \rulef{
    \rulef{
        \rulef{
            \rulef  {\ldots}
                   {\Sigma' \vdash^0 \lambda \overline{x} . L} \rulename{abs^i}
        }
        {\ldots} r_1
    }
    {\vdots} r_2
    }
    { \Sigma \vdash^u \lambda \overline{x} . L } r_l
    \qquad \mbox{where } r_j \in \{ \rulename{seq},\ \rulename{wk^i}\ |\ i \in \nat \},
            \quad j\in 1..l.
$$


Since $\varphi \in fv (L)$ we must have $\varphi \in \Sigma'$ and
since $\Sigma' \vdash^0 \lambda \overline{x} . L$, by lemma
\ref{lem:nonhomosafe_basic_prop} we have:

$$\ord{\varphi} \geq \ord{\lambda \overline{x} . L} \geq \max(1+ \ord{x}, \ord{L}) > \ord{x}$$

which contradicts equation (\ref{eq:xigeqphi}).
\end{proof}

Hence, in the safe lambda calculus setting, we can omit to to rename
variable when performing substitution. The equation
$$(\lambda x . M) \subst{t}{y} = \lambda z . M \subst{z}{x}
\subst{t}{y} \mbox{where $z$ is a fresh variable}$$ becomes
$$(\lambda x . M) \subst{t}{y} = \lambda x . M \subst{t}{y}$$



Unfortunately, this notion of substitution is still not adequate for
the purpose of the safe simply-typed lambda calculus. The problem is
that performing a single $\beta$-reduction on a safe term will not
necessarily produce another safe term.

To fix this problem, we need to be able to reduce several
consecutive $\beta$-redex at the same time until we obtain a safe
term. Consequently, we need a mean of performing several
substitutions at the same time. To achieve this, we introduce the
\emph{simultaneous substitution},
 a generalization of the standard substitution given in definition \ref{dfn:subst}.

\begin{dfn}[Simultaneous substitution]
\label{dnf:simsubst}
 We use the notation
$\subst{\overline{N}}{\overline{x}}$ for $\subst{N_1 \ldots N_n}{x_1
\ldots x_n}$:
\begin{eqnarray*}
c \subst{\overline{N}}{\overline{x}} &=& c \quad \mbox{where $c$ is a $\Sigma$-constant}\\
x_i \subst{\overline{N}}{\overline{x}} &=& N_i\\
 y \subst{\overline{N}}{\overline{x}} &=& y \quad \mbox{ if } y \not \neq x_i \mbox{ for all } i,\\
(M N) \subst{\overline{N}}{\overline{x}} &=& (M \subst{\overline{N}}{\overline{x}}) (N \subst{\overline{N}}{\overline{x}}) \\
(\lambda x_i . M) \subst{\overline{N}}{\overline{x}} &=& \lambda x_i
. M
\subst{N_1 \ldots N_{i-1} N_{i+1}\ldots N_n}{x_1 \ldots x_{i-1} x_{i+1}\ldots x_n} \\
(\lambda y . M)
\subst{\overline{N}}{\overline{x}} &=& \lambda z . M \subst{z}{y} \subst{\overline{N}}{\overline{x}} \\
&& \mbox{where $z$ is a fresh variables and } y \neq x_i \mbox{ for
all } i
\end{eqnarray*}
\end{dfn}

In general, variable captures should be avoided, this explains why
the definition of simultaneous substitution uses auxiliary fresh
variables. However in the current setting, lemma \ref{lem:noclash}
can clearly be transposed to the simultaneous substitution therefore
there is no need to rename variable.

The notion of substitution that we need is therefore the
\emph{capture permitting simultaneous substitution} defined as
follow:

\begin{dfn}[Capture permitting simultaneous substitution]
 We use the notation
$\subst{\overline{N}}{\overline{x}}$ for $\subst{N_1 \ldots N_n}{x_1
\ldots x_n}$:
\begin{eqnarray*}
c \subst{\overline{N}}{\overline{x}} &=& c \quad \mbox{where $c$ is a $\Sigma$-constant}\\
 x_i \subst{\overline{N}}{\overline{x}} &=& N_i\\
 y \subst{\overline{N}}{\overline{x}} &=& y \quad \mbox{where } x \not \neq y_i \mbox{ for all } i,\\
(M_1 M_2) \subst{\overline{N}}{\overline{x}} &=& (M_1 \subst{\overline{N}}{\overline{x}}) (M_2 \subst{\overline{N}}{\overline{x}})\\
(\lambda x_i . M) \subst{\overline{N}}{\overline{x}} &=& \lambda x_i
. M
\subst{N_1 \ldots N_{i-1} N_{i+1}\ldots N_n}{x_1 \ldots x_{i-1} x_{i+1}\ldots x_n} \\
(\lambda y . M) \subst{\overline{N}}{\overline{x}} &=& \lambda y . M
\subst{\overline{N}}{\overline{x}} \mbox{where $y \not = x_i$ for
all $i$} \qquad \mathbf{(\star)}
\end{eqnarray*}
The symbol $\mathbf{(\star)}$ identifies the equation that changed
compared to the previous definition.
\end{dfn}

\begin{lem}
$$ \Gamma,\overline{x} : \overline{A}\vdash^i M : T
\quad \mbox{and} \quad \Gamma \vdash^0 N_k : B_k \mbox{, } k \in
1..n \qquad \mbox{ implies } \qquad \Gamma \vdash^i
M[\overline{N}/\overline{x}] : T$$
\end{lem}

\begin{proof}
Suppose that $\Gamma,\overline{x}: \overline{A} \vdash^i M :T$ and
$\Gamma \vdash^0 N_k : B_k$ for $k \in 1..n$.

We prove $\Gamma \vdash^i M[\overline{N}/\overline{x}]$ by induction
on the size of the proof tree of $\Gamma,\overline{x}:\overline{A}
\vdash^i M : T$ and by case analysis on the last rule used. We just
give the detail for the abstraction case. Suppose that the property
is verified for terms whose proof tree is smaller than $M$. Suppose
$\Gamma,\overline{x}:\overline{A} \vdash^0 \lambda \overline{y} :
\overline{C}. P : (\overline{C}|D)$ where $\Gamma,
\overline{x}:\overline{A}, \overline{y}:\overline{C} \vdash^i P :
D$, then by the induction hypothesis $\Gamma,
\overline{y}:\overline{C} \vdash^i
P\subst{\overline{N}}{\overline{x}} : D$. Applying the rule
$\rulename{abs^i}$ gives $\Gamma \vdash^0 \lambda
\overline{y}:\overline{C} . P \subst{\overline{N}}{\overline{x}}$.
\end{proof}

\subsection{Safe-redex}
In the simply-typed lambda calculus a redex is a term of the form
$(\lambda x . M) N$. We generalize this definition to the safe
lambda calculus:
\begin{dfn}[Safe redex]
We call safe redex a term of the form $(\lambda \overline{x} . M)
N_1 \ldots N_l$ such that:
\begin{itemize}
\item $ \Gamma \vdash^0 (\lambda \overline{x} . M) N_1 \ldots N_l $
\item the variable $\overline{x}=x_1\ldots x_n$ are abstracted altogether by one occurrence of the rule $\rulename{abs}$ in the proof tree.
\item The terms $(\lambda \overline{x} . M)$, $N_1$, $N_l$ are applied together at once using the $\rulename{app}$ rule :
$$   \rulef{
            \Sigma \vdash^{-1} \lambda \overline{x} . M
            \quad
            \Sigma \vdash^0 N_1         \quad \ldots \quad \Sigma \vdash^0 N_l
    }
    {
       \Sigma \vdash^0 (\lambda \overline{x} . L) N_1 \ldots N_l
    } (\mathbf{app})
$$
Consequently each $N_i$ is safe.

\item $l\leq n$
\end{itemize}
\end{dfn}

Note that the condition $l\leq n$ in the definition is not too
restrictive because if $l>n$ then the application rule is too wide
and can in fact be replaced by an application of exactly $n$ terms
followed by another application for the remaining terms $N_{n+1},
\ldots, N_l$.


\todobox{Define the safe reduction:
Consider the safe-redex $(\lambda \overline{x} . M) N_1 \ldots N_l$, it reduces
to $\lambda x_l \ldots x_n . M \subst{N_1 \ldots N_l}{x_1 \ldots x_l}$. The relation $\beta_s$ is defined
on safe-redex: $(s\mapsto t) \in \beta_s$ iff $s \equiv (\lambda \overline{x} . M) N_1 \ldots N_l$ is a safe redex and
$t \equiv \lambda x_l \ldots x_n . M \subst{N_1 \ldots N_l}{x_1 \ldots x_l}$
}

\todobox{Show that $\betasred \subseteq \betared^*$.}

Using the previous lemma, we will now prove that safe-reduction
produces safe terms.

\begin{lem}
\label{lem:safereduction} A safe redex reduces to a safe term.
\end{lem}

\begin{proof}
We note $\overline{A}$ for $A_1, \ldots , A_n$, $\overline{x}'$ for
$x_1 \ldots x_l$ and $\overline{x}''$ for $x_{l+1} \ldots x_n$.

A safe-redex has a proof tree of the following form:
$$
   \rulef{
        \rulef{
            \rulef{
                \rulef{
                    \rulef
                        { \rulef
                            {\vdots}
                            {\Sigma',\overline{x}:\overline{A}\vdash^i L:C  }
                        }
                        {\Sigma' \vdash^0 \lambda \overline{x} . L : \overline{A}|C} \rulename{abs^i}
                }
                {\vdots} r_1
            }
            {\vdots} r_2
            }
            { \Sigma \vdash^{-1} \lambda \overline{x} . L : A_1, \ldots , A_l|B} r_q
            \quad
            \Sigma \vdash^0 N_1 : A_1
            \quad \ldots \quad \Sigma \vdash^0 N_l : A_l
    }
    {
       \Sigma \vdash^0 (\lambda \overline{x} . L) N_1 \ldots N_l : B
    } (\mathbf{app})
$$
with the following conditions:
\begin{enumerate}
\item for $j\in 1..q$, $r_j \in \{ \rulename{seq}, \rulename{wk^0}, \rulename{wk^{-1}} \}$ therefore
$\Sigma = \Sigma' \union \Delta$ where $\Delta$ contains the
variables introduced by the rules $r_1 \ldots r_q$.

\item $A_1, \ldots , A_l|B = A_1, \ldots , A_n|C$ and $l\leq n$. Therefore
$\ord{B} \geq \ord{C}$.
\item The side condition of the rule $\rulename{abs}$ gives: $\forall z \in \Sigma : \ord{z} \geq \ord{B}$
\end{enumerate}


The conditions 2 and 3 ensure that $\forall z \in \Delta : \ord{z}
\geq \ord{C}$ therefore we can use the weakening rule to introduce
all the variable of $\Delta$ in the context of the sequent
$\Sigma',\overline{x}:\overline{A}\vdash^i L:C$:

$$\rulef{\rulef{ \Sigma',\overline{x}:\overline{A}\vdash^i L:C  }
        {\vdots} (wk^i_0)}
        {\Sigma,\overline{x}:\overline{A}\vdash^i L:C} (wk^i_0)
$$

By lemma \ref{lem:safereduction} we obtain:
$$ \Sigma, \overline{x}'':\overline{A}'' \vdash^i L\subst{N_1 \ldots N_l}{\overline{x}'}$$
Finally using the abstraction rule:
$$ \Sigma \vdash^0 \lambda \overline{x}'':\overline{A}'' . L\subst{N_1 \ldots N_l}{\overline{x}'}$$
\end{proof}



\subsection{Examples}
\subsubsection{Example 1}
Let $f,g:o\rightarrow o$, $x,y:o\rightarrow
o$, $\Gamma = g:o\rightarrow o$ and $\Gamma' = g:o\rightarrow o,
y:o$. The term $(\lambda f x . x) g y $ is safe:


$$ \rulef{
        \rulef{
            \rulef{
                \rulef{\vdots}{\Gamma \vdash^{-1} \lambda f x. x}      \qquad \axiomf{\Gamma \vdash^0 g} }
            {\Gamma \vdash^0 (\lambda f x. x) g} \rulename{app}
        }
        { \Gamma' \vdash^{-1} (\lambda f x. x) g } \rulename{wk^{-1}}
        \qquad \axiomf{\Gamma' \vdash^0 y}
    }
    { \Gamma' \vdash^0 (\lambda f x. x) g y } \rulename{app}
$$


And the two occurrences of the application rule cannot be merged as
follow:
$$ \rulef{{\Gamma' \not\vdash^{-1} \lambda f x. x} \qquad \Gamma' \not\vdash^0 g \qquad \Gamma' \vdash^0 y}
    {\Gamma' \vdash^0 (\lambda f x. x) g y } \rulename{app}$$


\subsection{Particular case of homogeneously-safe lambda terms}

We look at a particular sub-class of lambda terms. The types of
these terms respect a property call homogeneity as defined in
section \ref{sec:homotypes}. A type $(A_1, A_2, \ldots A_n, o)$ is
said to be homogeneous whenever $\order{A_1} \geq \order{A_2} \geq
\ldots \geq  \order{A_n}$ and each of the $A_i$ are homogeneous. A
term is homogeneous if its type is homogeneous.


In their definition of safety (\cite{KNU02}), Knapik et al. require
that all the recursion equations of a safe recursion scheme have a
homogeneous type.

In the rules defining safety for the non-homogeneous case, the only
rule that can potentially introduce a non-homogeneous term from a
homogeneous one is the abstraction rule. But such a term (lambda
abstraction) corresponds exactly to a recursion equation in the
recursion scheme setting of Knapik et al. Therefore requiring that
recursions equation have homogeneous type is the same as requiring
that all sequents appearing in the proof tree of a safe term are of
homogeneous type.

We say that a term is homogeneously-safe if its type is homogeneous
and there is a proof tree showing its safety where all the sequents
of the proof tree are of homogenous type.

\begin{lem}[Context reduction]
\label{lem:context_reduction} If $\Gamma \vdash^i M : A$ then there
is a context $\Gamma' \subseteq \Gamma$ such that $\Gamma' \vdash^0
M : A$.
\end{lem}
\begin{proof}
An easy induction.
\end{proof}


\begin{lem}
\label{lem:homog_judg_zero_minusone} If a term is homogeneously-safe
then there is valid proof tree showing that it is safe containing
only judgments of the form $\Gamma \vdash^{k} M : T$ with $k\in
\{-1,0\}$.
\end{lem}

\begin{proof}
Assume that $\Gamma \vdash^{0} S : T_S$ with $T_S$ homogeneous. We
prove the result by induction on the size of the proof tree and by
case analysis on the last rule used to obtain $\Gamma \vdash^{0} S :
T_S$.

We give the details of the proof for the application and abstraction
case:
\begin{itemize}
\item Rule $\rulename{abs^i}$ for some $i$:
$$ \rulename{abs^i} \quad  \rulef{\Gamma, \overline{x} : \overline{A} \vdash^i M : B}
                                   {\Gamma  \vdash^{0} \lambda \overline{x} : \overline{A} . M : (\overline{A},B)} \qquad
                                   \forall y \in \Gamma : \ord{y} \geq \ord{\overline{A},B}$$

Type homogeneity requires that $\ord{\overline{x}} = \ord{A} \geq
\ord{B}$. Therefore the premise of the rule can be replaced by
$\Gamma, \overline{x} : \overline{A} \vdash^0 M : B$.

The induction hypothesis permits to conclude.

\item Rule $\rulename{app}$:
$$ \rulename{app} \quad  \rulef{\Gamma \vdash^{-1} M : (A,\ldots,A_l,B)
                                        \qquad \Gamma \vdash^{0} N_1 : A_1
                                        \quad \ldots \quad \Gamma \vdash^{0} N_l : A_l  }
                                   {\Gamma  \vdash^0 M N_1 \ldots N_l : B}
                                    \qquad
                                   \forall y \in \Gamma : \ord{y} \geq \ord{B}$$

For the first premise of the rule we apply lemma
\ref{lem:context_reduction}: there is a context $\Gamma'$ such that
$\Gamma' \subseteq \Gamma$ and:
$$ \Gamma' \vdash^{0} M : (A,\ldots,A_l,B) $$
Now by the induction hypothesis we know that there is a proof tree
showing $\Gamma' \vdash^{0} M : (A,\ldots,A_l,B)$ with judgement of
the form $\Sigma \vdash^{k} P : T$ with $k\in \{-1,0\}$.

By applying the weakening rule, we lift this result to $\Gamma
\vdash^{0} M : (A,\ldots,A_l,B)$.

For all the other premises we can directly apply the induction
hypothesis.

We conclude using the rule $\rulename{app}$.
\end{itemize}
\end{proof}

This lemma permits us to derive rules specialized for the
homogeneously-safe lambda calculus.

\subsubsection{The application rule} Let us derive the
application rules specialized for the case of homogeneous types. We
recall the rule $\rulename{app}$:
$$ \rulename{app} \  \rulef{\Gamma \vdash^{-1} M : (A,\ldots,A_l,B)
                                        \qquad \Gamma \vdash^{0} N_1 : A_1
                                        \quad \ldots \quad \Gamma \vdash^{0} N_l : A_l  }
                                   {\Gamma  \vdash^0 M N_1 \ldots N_l : B}
                                    \quad
                                   \forall y \in \Gamma : \ord{y} \geq \ord{B}$$

Type homogeneity implies that $\ord{A_1} \geq \ldots \geq \ord{A_l}
\geq \ord{B} - 1$.

We can make the assumption that $\ord{A_1} = \ldots = \ord{A_l}$ (if it is not the case,
we can replace the application rule by several consecutive application rules
respecting this condition).


\begin{itemize}
\item Suppose that $A_1, \ldots A_l$ forms a type partition, then we
have $\ord{A_l} \geq \ord{B}$, the side-condition disappears and
the rule becomes:
$$ \rulename{app_1} \quad  \rulef{\Gamma \vdash^{-1} M : \overline{A} | B
                                        \qquad \Gamma \vdash^{0} N_1 :
                                        A_1
                                        \quad \ldots \quad \Gamma \vdash^{0} N_l :
                                        A_l
                                        }
                                   {\Gamma  \vdash^{0} M N_1 \ldots N_l : B}
$$

where $\overline{A} = A_1, \ldots A_l$

\item  Suppose that $A_1, \ldots A_l$ do not form a type partition, then we
have $\ord{A_l} = \ord{B} - 1$. The side-condition becomes
$\forall y \in \Gamma : \ord{y} \geq 1 +\ord{A_l} = \ord{\overline{A}|B}$. Therefore the side-condition
can be omitted provided that we replace the $-1$ exponent in the first premise by a $0$:
$$ \rulename{app_2} \quad  \rulef{\Gamma \vdash^0 M : (A,\ldots,A_l,B)
                                        \qquad \Gamma \vdash^{0} N_1 : A_1
                                        \quad \ldots
                                        \quad \Gamma \vdash^{0} N_l : A_l
                                   }
                                   {\Gamma  \vdash^0 M N_1 \ldots N_l : B}
$$

Equivalently we can replace this rule by the following one:
$$ \rulename{app_2'} \quad  \rulef{\Gamma \vdash^0 M : A\rightarrow B
                                        \qquad \Gamma \vdash^{0} N : A }
                                {\Gamma  \vdash^{0} M N : B}$$
\end{itemize}

\subsubsection{The abstraction rule}

Let us derive the abstraction rule specialized for the case of
homogeneous types. We recall the rule $\rulename{abs}$:
$$ \rulename{abs^i} \quad  \rulef{\Gamma, \overline{x} : \overline{A} \vdash^{i} M : B}
                                   {\Gamma  \vdash^{0} \lambda \overline{x} : \overline{A} . M : (\overline{A},B)} \qquad
                                   \forall y \in \Gamma : \ord{y} \geq \ord{\overline{A},B}$$

The context $\Gamma$ is now partitionned according to the order of
the variables. The partitions are written in decreasing order of
type order. The notation $\Gamma | \overline{x}:\overline{A}$ means
that $\overline{x}:\overline{A}$ is the lowest partition of the
context.

We also use the notation $(\overline{A}|B)$ to denote the
homogeneous type $(A_1, A_2, \ldots A_n, B)$ where $\ord{A_1} =
\ord{A_2} =  \ldots \ord{A_n} \geq \ord{B} -1$.


Suppose that we abstract the single variable $\overline{x} = x$,
then in order to respect the side condition, we need to abstract all
variables of order lower or equal to $\ord{x}$. In particular we
need to abstract the partition of the order of $x$.

Moreover to respect type homogeneity, we need to abstract variables
of the lowest order first.

Hence we can change the abstraction rule so that it only allows
abstraction of the lowest variable partition. The rule can then be
used repeatedely if further partitions need to be abstracted. We
obtained the following rule where the side-condition has
disappeared:

$$ \rulename{abs^i} \quad  \rulef{\Gamma| \overline{x} : \overline{A} \vdash^{-1} M : B}
                                   {\Gamma  \vdash^{0} \lambda \overline{x} : \overline{A} . M : (\overline{A}|B)}$$


\subsubsection{The rules of the homogeneous safe $\lambda$-calculus}

Table \ref{tab:homosafelmd_rules} recapitulates the entire set of rules:

\begin{table}[htbp]
$$  \rulename{perm} {
      { \Gamma \vdash^0 M:B \qquad \sigma(\Gamma)  } \hbox{ homogeneous}
    \over
      { \sigma(\Gamma) \vdash^0 M : B }
    }
\qquad \rulename{seq} \quad \rulef{\Gamma \vdash^{0} M : A}{\Gamma
\vdash^{-1} M : A}
$$

$$
 \rulename{const}
    { \over { \vdash^0 b : o^r \rightarrow o}} \quad b : o^r \rightarrow o \in \Sigma
\qquad
 \rulename{var} \quad  \rulef{}{x : A\vdash^{0} x : A} $$

$$ \rulename{wk^{0}} \quad  \rulef{\Gamma \vdash^{0} M : A}{\Gamma , x : B \vdash^{0} M : A} \quad \ord{B} \geq \ord{A} $$

$$ \rulename{wk^{-1}} \quad  \rulef{\Gamma \vdash^{-1} M : A}{\Gamma , x : B \vdash^{-1} M : A} \quad \ord{B} \geq \ord{A} -1$$

$$ \rulename{app} \quad  \rulef{\Gamma \vdash^{-1} M : \overline{A} | B
                                        \qquad \Gamma \vdash^{0} N_1 : A_1
                                        \quad \ldots \quad \Gamma \vdash^{0} N_l : A_l
                                        \qquad l = |\overline{A}|
                                        }
                                   {\Gamma  \vdash^{0} M N_1 \ldots N_l : B}
$$


$$ \rulename{app^0} \quad  \rulef{\Gamma \vdash^0 M : A\rightarrow B
                                        \qquad \Gamma \vdash^{0} N : A
                                   }
                                   {\Gamma  \vdash^{0} M N : B}$$

$$ \rulename{abs} \quad  \rulef{\Gamma| \overline{x} : \overline{A} \vdash^{-1} M : B}
                                   {\Gamma  \vdash^{0} \lambda \overline{x} : \overline{A} . M : (\overline{A}|B)}$$
\caption{Rules of the homogeneous safe lambda calculus}
\label{tab:homosafelmd_rules}
\end{table}


We observe that these rules correspond exactly to the rules given in the previous section
in table \ref{tab:homosafelmd_rules_refined}.
