\section{Non homogeneous safe $\lambda$-calculus}

In section \ref{sec:safe_alt}, we have presented a safe lambda calculus in the setting of homogeneous types.
In this section, we try to give a general notion of safety for the simply typed $\lambda$-calculus.
The rules we give here do not assume homogeneity of the types.

We will call safe terms the simply typed lambda terms that are typable within the following system of formation rules:

\subsection{Rules}

 We use a set of sequents of the form $\Gamma \vdash^{i} M :
A$ where the meaning is ``variables in $\Gamma$ have orders at least
$\ord{A}+i$'' where $i \in \zset$. The following set of rules are
defined for $i \in \zset$:

$$ \mathbf{(seq^i_\delta)} \quad \rulef{\Gamma \vdash^{i} M : A}{\Gamma \vdash^{i-\delta} M : A} \quad i \in \zset, \delta > 0  $$

$$ \mathbf{(var)} \quad  \rulef{}{x : A\vdash^{0} x : A} $$

$$ \mathbf{(wk^i)} \quad  \rulef{\Gamma \vdash^i M : A}{\Gamma , x : B \vdash^i M : A} \quad \ord{B} \geq \ord{A} + i $$

$$ \mathbf{(app^i)} \quad  \rulef{\Gamma \vdash^i M : A\rightarrow B
                                        \qquad \Gamma \vdash^{0} N : A}
                                   {\Gamma  \vdash^{i+\delta} M N : B}
                                    \qquad
                                   \delta = \max\left(0, 1 + \ord{A} - \ord{B}\right)$$

$$ \mathbf{(abs^i)} \quad  \rulef{\Gamma, \overline{x} : \overline{A} \vdash^{i} M : B}
                                   {\Gamma  \vdash^{0} \lambda \overline{x} : \overline{A} . M : (\overline{A},B)} \qquad
                                   \forall y \in \Gamma : \ord{y} \geq \ord{\overline{A},B}$$


Note that:
\begin{itemize}
\item $(\overline{A},B)$ denotes the type $(A_1,A_2, \ldots, A_n, B)$;
\item all the types appearing in the rule are not required to be homogeneous. For instance in the rule $\mathbf{(app^i)}$, the type $A \rightarrow B$ is not necessarily
homogeneous;
\item the environment $\Gamma, \overline{x}$ is not stratified. In particular, variables in $\overline{x}$ do not necessarily have the same order. Also
there may be variable in $\Gamma$ of order smaller than $\ord{x}$
for some variable $x$ in $\overline{x}$.
\item The sequents $\Gamma \vdash^0 M$ are the \emph{safe terms} that we want to generate.
Other terms are only used as intermediate sequents in a proof tree.
\end{itemize}


\begin{rem}
\label{rem:rulesineg}

This set of rules is equivalent (in term of safe terms that can be
generated) to the same set of rules where $i$ is restricted to be a
negative integer and where the rule $\mathbf{(app^i)}$ becomes:
$$ \mathbf{(app^i)} \quad  \rulef{\Gamma \vdash^i M : A\rightarrow B
                                        \qquad \Gamma \vdash^{0} N : A}
                                   {\Gamma  \vdash^{\min(i+\delta,0)} M N : B}
                                    \qquad
                                   \delta = \max\left(0, 1 + \ord{A} - \ord{B}\right) \quad i \leq 0 $$

With this new set of rules, the sequents of the form $\Gamma
\vdash^{k} M$ with $k>0$ cannot be derived anymore, however, the set
of safe terms that can built remain the same. Indeed, suppose that
we derive $\Gamma \vdash^0 M$ using the sequent $\Gamma \vdash^k N$
with $k>0$ somewhere in the proof. Then an easy induction shows that
the sequent $\Gamma \vdash^0 N$ can as well be derived (using
the rule $(seq^i_\delta)$ for $i\leq 0$).
\end{rem}

\begin{lem}[Basic properties]
\label{lem:nonhomosafe_basic_prop} Suppose $\Gamma \vdash^0 M : B$
is a valid judgment then every variable in $\Gamma$ has order at
least $ord(M)$.
\end{lem}
\begin{proof}
An easy induction on the proof tree shows that if $\Gamma \vdash^{i}
M : A$ then the variables in $\Gamma$ have orders at least
$\ord{A}+i$. The induction step for the application is: suppose
$\Gamma  \vdash^{i+\delta} M N : B$ where $\Gamma \vdash^i M :
A\rightarrow B$. Then by induction we have $\forall y \in \Gamma :
\ord{y} \geq \ord{A\rightarrow B} + i = \max(1+\ord{A}, \ord{B})+i =
\delta + \ord{B} + i$.
\end{proof}

\subsection{Substitution in the safe lambda calculus}

The traditional notion of substitution, on which the $\lambda$-calculus is based on, is the following one:
\begin{dfn}[Substitution]
\label{dfn:subst}
\begin{eqnarray*}
c \subst{t}{x} &=& c \quad \mbox{where $c$ is a $\Sigma$-constant}\\
x \subst{t}{x} &=& t\\
 y\subst{t}{x} &=& y \quad \mbox{for } x \not \neq y,\\
(M_1 M_2) \subst{t}{x} &=& (M_1 \subst{t}{x}) (M_2 \subst{t}{x})\\
(\lambda x . M) \subst{t}{x} &=& \lambda x . M\\
(\lambda y . M) \subst{t}{x} &=& \lambda z . M \subst{z}{y}
\subst{t}{x} \mbox{where $z$ is a fresh variable and $x\not = y$}
\end{eqnarray*}
\end{dfn}

In the setting of the safe lambda calculus, the notion of substitution can be simplified.
Indeed, we remark that for safe $\lambda$-terms there is no need to rename variables
when performing substitution:

\begin{lem}[No variable capture lemma]
\label{lem:noclash}
There is no variable capture when performing substitution on a safe term.
\end{lem}
\begin{proof}
Suppose that a capture occurs during the substitution $M[N/\varphi]$
where $M$ and $N$ are safe. Then the following conditions must hold:
\begin{enumerate}
\item $\varphi:A, \Gamma \vdash^0 M$,
\item $\Gamma' \vdash^0 N$,
\item there is a subterm $\lambda \overline{x} . L$ in $M$ (where the abstraction is taken as wide as possible) such that:
\item $\varphi \in fv(\lambda \overline{x} . L)$ (and therefore $\varphi \in fv(L)$),
\item $x \in fv(N)$ for some $x \in \overline{x}$.
\end{enumerate}

By lemma \ref{lem:nonhomosafe_basic_prop} and (v) we have:

\begin{equation}
\ord{x} \geq \ord{N} = \ord{\varphi} \label{eq:xigeqphi}
\end{equation}

The abstraction $\lambda \overline{x} . L$ (taken as large as possible)
is a subterm of $M$, therefore there is a node $\Sigma \vdash^u \lambda \overline{x} . L$  for some $u$ in the
proof tree of $\varphi:A, \Gamma \vdash^0 M$.

There are only three kinds of rules that can produce an abstraction:
$\mathbf{(abs^i)}$, $\mathbf{(seq^i_\delta)}$ and $\mathbf{(wk^i)}$.
The only one that can introduce the abstraction is
$\mathbf{(abs^i)}$. Therefore the proof tree has the following form:

$$ \rulef{
    \rulef{
        \rulef{
            \rulef  {\ldots}
                   {\Sigma' \vdash^0 \lambda \overline{x} . L} \mathbf{(abs^i)}
        }
        {\ldots} r_1
    }
    {\vdots} r_2
    }
    { \Sigma \vdash^u \lambda \overline{x} . L } r_l
    \qquad \mbox{where } r_j \in \{ \mathbf{(seq^i_\delta)},\ \mathbf{(wk^i)}\ |\ i \in \zset, \delta > 0 \},
            \quad j\in 1..l.
$$


Since $\varphi \in fv (L)$ we must have $\varphi \in \Sigma'$ and
since $\Sigma' \vdash^0 \lambda \overline{x} . L$, by lemma
\ref{lem:nonhomosafe_basic_prop} we have:

$$\ord{\varphi} \geq \ord{\lambda \overline{x} . L} \geq \max(1+ \ord{x}, \ord{L}) > \ord{x}$$

which contradicts equation (\ref{eq:xigeqphi}).
\end{proof}

Hence, in the safe lambda calculus setting, we can omit to
to rename variable when performing substitution. The equation
$$(\lambda x . M) \subst{t}{y} = \lambda z . M \subst{z}{x}
\subst{t}{y} \mbox{where $z$ is a fresh variable}$$
becomes
$$(\lambda x . M) \subst{t}{y} = \lambda x . M \subst{t}{y}$$



Unfortunately, this notion of substitution is still not adequate for the purpose of the
safe simply type lambda calculus. The problem is that performing a single
$\beta$-reduction on a safe term will not necessarily produce another safe term.

To fix this problem, we need to be able to reduce several consecutive
$\beta$-redex at the same time until we obtain a safe term. In other words, we need a mean of performing
several substitutions at the same time.
To achieve this, we introduce the \emph{simultaneous substitution}. It is
 a generalization of the standard substitution given in definition \ref{dfn:subst}.

\begin{dfn}[Simultaneous substitution]
\label{dnf:simsubst}
 We use the notation
$\subst{\overline{N}}{\overline{x}}$ for $\subst{N_1 \ldots N_n}{x_1
\ldots x_n}$:
\begin{eqnarray*}
c \subst{\overline{N}}{\overline{x}} &=& c \quad \mbox{where $c$ is a $\Sigma$-constant}\\
x_i \subst{\overline{N}}{\overline{x}} &=& N_i\\
 y \subst{\overline{N}}{\overline{x}} &=& y \quad \mbox{ if } y \not \neq x_i \mbox{ for all } i,\\
(M N) \subst{\overline{N}}{\overline{x}} &=& (M \subst{\overline{N}}{\overline{x}}) (N \subst{\overline{N}}{\overline{x}}) \\
(\lambda x_i . M) \subst{\overline{N}}{\overline{x}} &=& \lambda x_i . M
\subst{N_1 \ldots N_{i-1} N_{i+1}\ldots N_n}{x_1 \ldots x_{i-1} x_{i+1}\ldots x_n} \\
(\lambda y . M)
\subst{\overline{N}}{\overline{x}} &=& \lambda z . M \subst{z}{y} \subst{\overline{N}}{\overline{x}} \\
&& \mbox{where $z$ is a fresh variables and } y \neq x_i \mbox{ for all } i
\end{eqnarray*}
\end{dfn}

In general, variable captures should be avoided, this explains why the definition
of simultaneous substitution uses auxiliary fresh variables.
However in the current setting, lemma \ref{lem:noclash} can clearly be transposed to
the simultaneous substitution therefore there is no need to rename variable.

The notion of substitution that we need is therefore
the \emph{capture permitting simultaneous substitution} defined as follow:

\begin{dfn}[Capture permitting simultaneous substitution]
 We use the notation
$\subst{\overline{N}}{\overline{x}}$ for $\subst{N_1 \ldots N_n}{x_1
\ldots x_n}$:
\begin{eqnarray*}
c \subst{\overline{N}}{\overline{x}} &=& c \quad \mbox{where $c$ is a $\Sigma$-constant}\\
 x_i \subst{\overline{N}}{\overline{x}} &=& N_i\\
 y \subst{\overline{N}}{\overline{x}} &=& y \quad \mbox{where } x \not \neq y_i \mbox{ for all } i,\\
(M_1 M_2) \subst{\overline{N}}{\overline{x}} &=& (M_1 \subst{\overline{N}}{\overline{x}}) (M_2 \subst{\overline{N}}{\overline{x}})\\
(\lambda x_i . M) \subst{\overline{N}}{\overline{x}} &=& \lambda x_i . M
\subst{N_1 \ldots N_{i-1} N_{i+1}\ldots N_n}{x_1 \ldots x_{i-1} x_{i+1}\ldots x_n} \\
(\lambda y . M) \subst{\overline{N}}{\overline{x}} &=& \lambda y . M \subst{\overline{N}}{\overline{x}} \mbox{where $y \not = x_i$ for all $i$}
\qquad \mathbf{(\star)}
\end{eqnarray*}
The symbol $\mathbf{(\star)}$ identifies the equation that changed compared to the previous definition.
\end{dfn}

\begin{prop}
$$ \Gamma,\overline{x} \vdash^i M : A
\quad \mbox{and} \quad \Gamma \vdash^0 N_k : B_k \mbox{, } k \in 1..n
\qquad \mbox{ implies } \qquad
\Gamma \vdash^i M[\overline{N}/\overline{x}]$$
\end{prop}

\begin{proof}
Suppose that $\Gamma,\overline{x} \vdash^i M : A$ and  $\Gamma \vdash^0 N_k : B_k$ for $k \in 1..n$.

We prove $\Gamma \vdash^i M[\overline{N}/\overline{x}]$ by induction on the size of the proof
tree of $\Gamma,\overline{x} \vdash^i M : A$ and by case analysis on the last rule used.
We just give the detail for the abstraction case. Suppose that the property is verified for terms whose proof tree is
smaller than $M$. Suppose $\Gamma,\overline{x} \vdash^0 \lambda \overline{y} : \overline{C}. T : (\overline{C}|D)$ where
$\Gamma, \overline{y}:\overline{C},\overline{x} \vdash^i T : D$, then
by the induction hypothesis $\Gamma, \overline{y}:\overline{C} \vdash^i T\subst{\overline{N}}{\overline{x}} : D$.
We can then apply the rule $\mathbf{(abs^i)}$ and we get
$\Gamma \vdash^0 \lambda \overline{y}:\overline{C} . T \subst{\overline{N}}{\overline{x}}$.
\end{proof}


\begin{cor}[Simultaneous substitution preserves safety]
\label{cor:subst_preserve_safety}
If $M$ is safe and $N_k$ is safe for $k \in 1..n$ then  $M[\overline{N}/\overline{x}]$ is safe
\end{cor}

\begin{cor}
Consider a safe multi-redex $(\lambda \overline{x} . M ) \overline{N}$ where the $N_i$ are safe, then
we can reduced it to $M\subst{\overline{N}}{\overline{x}}$. This reduction produces
a safe term \textbf{provided that $M$ is also a safe term}.
\end{cor}


It is important to note that not all multi-$\beta$-reduction conserve safety. Only the reductions that
peal off enough $\lambda$-abstraction will reach a safe term and therefore preserve safety.\\


\subsection{Simultaneous $\beta$ reduction}

We now define a notion of beta reduction that realizes simultaneous
substitution. Consider a simply-typed term $P$. A simultaneous
$\beta$-redex is a $P$ sub-term of the kind

$$R_1 \equiv (\lambda x_1 x_2 \ldots x_n . M) N_1 N_2 \ldots N_n$$

Reduction is only performed if the simultaneous $\beta$-redex
encompasses as many lambda abstraction of the same order as
possible. Such a redex (which cannot be extended to take into
account one more lambda abstraction of the same order) is called a
$\beta_s$-redex.


Example: consider a term $P$ with a subterm $((\lambda x_1 x_2
\ldots x_n . M) N_1 N_2 \ldots N_n) N_{n+1}$. Suppose that $M$ is
the abstraction $M \equiv \lambda x_{n+1} . U$ where $ord(x_{n+1}) =
x_1$. Then the redex $R_1$ will not be considered since it can be
enlarged as the redex $(\lambda x_1 x_2 \ldots x_n x_{n+1}. M) N_1
N_2 \ldots N_n N_{n+1}$. Now suppose instead that the term is formed
in such a way that there is no $N_{n+1}$ applied on the right of
$R_1$ then the redex $R_1$ will be considered (whether or not $M$ is
an abstraction).

We now give the formal definitions:

The following abbreviations are used $\overline{x} = x_1 \ldots
x_n$, $\overline{N} = N_1 \ldots N_n$, $\overline{x_l} = x_1 \ldots
x_l$, $\overline{x_r} = x_{l+1} \ldots x_n$, $\overline{N_l} = N_1
\ldots N_l$ and $\lambda \overline{x} : \overline{A} . T = \lambda
x_1^{A_1} \ldots x_1^{A_n} . T$.

\begin{dfn}[$\beta_s$-redex]
A safe simply typed lambda term is a redex if it has one of the
following forms:
\begin{itemize}
\item $(\lambda \overline{x} : \overline{A} . T) \overline{N}$
\quad with $|\overline{x}| = |\overline{N}| = n$, $ord(T) \leq
ord(\overline{x}) = ord(x_1) = \ldots = ord(x_n)$.

\item $(\lambda \overline{x_l} : \overline{A_l} \ \overline{x_r}: \overline{A_r} . T) \overline{N_l}$
\quad with $|\overline{x_l}| = |\overline{N_l}| = l$, $ord(T) \leq
ord(\overline{x}) = ord(x_1) = \ldots = ord(x_n)$.
\end{itemize}

These two cases correspond respectively to the formation rules (App)
and (App+) of the safe lambda calculus.

\end{dfn}

\begin{dfn}[Simultaneous $\beta$-reduction] \
\begin{itemize}
\item The relation $\beta_s$ is defined on the set of $\beta_s$-redex.
\begin{eqnarray*}
\beta_s &=&
 \{  \left( (\lambda \overline{x} : \overline{A} . T) \overline{N}, T\subst{\overline{x}}{\overline{N}} \right) \\
&& \mbox{ where }
     |\overline{x}| = |\overline{N}| = n \mbox{ and } ord(T) \leq ord(\overline{x}) = ord(x_1) = \ldots = ord(x_n)
\} \\
 && \union \quad \\
&& \{
    \left( (\lambda \overline{x_l} : \overline{A_l} \  \overline{x_r}: \overline{A_r} . T) \overline{N_l}, \lambda \overline{x_r}: \overline{A_r} . T\subst{\overline{x_l}}{\overline{N_l}} \right) \\
&& \mbox{ where }
 |\overline{x}| = |\overline{N}| = n \mbox{ and } ord(T) \leq ord(\overline{x}) = ord(x_1) = \ldots = ord(x_n)
\}
\end{eqnarray*}

Note that in the second case, the substitution is done under the
$\lambda \overline{x_r}$. The side condition of the formation rule
(App+) guarantees that there will not be any variable capture.

\item
The simultaneous $\beta$-reduction noted $\betasred$ is the closure
of the relation $\beta_s$ by compatibility with the formation rules
of the safe $\lambda$-calculus.
\end{itemize}
\end{dfn}

Note that $\beta_s$-redex are the only redex that can be reduced by
$\betasred$.




\subsection{Some properties of $\beta_s$ reduction}

We remark that $\betasred \subset \betaredtr$ (i.e. the simultaneous
$\beta$-reduction relation) is included in the transitive closure of
the $\beta$-reduction relation. More precisely, if $M \betasred N$
then $M \betaredtr N$. Simultaneous $\beta$-reduction is a certain
kind of multi-steps $\beta$-reduction.

\begin{lem} In the simply typed $\lambda$-calculus setting:
\begin{enumerate}
\item $\betasred$ is strongly normalizing.
\item $\beta_s$ has the unique normal form property.
\item $\beta_s$ has the Church-Rosser property.
\end{enumerate}
\end{lem}

\begin{proof}
\begin{enumerate}

%Proof of weak normalization:
%We know that $\betared$ is strongly normalizing: any reduction
%strategy leads to a normal form. In particular any strategy that
%performs consecutive $\beta$-redex of the same order consecutively
%will lead to a normal form. Since any such $\beta$ reduction
%strategy is also a $\beta_s$ reduction strategy, we can conclude
%that $\betasred$ is weakly normalizing.

\item This is because $\betasred \subset \betaredtr$ and $\betared$ is strongly normalizing (in the simply typed lambda calculus).

\item
A term has a $\beta_s$-redex iff it has a $\beta$-redex therefore
the set of $\beta_s$ normal form is equal to the set of $\beta_s$
normal form. Hence, the unicity of $\beta$ normal form implies the
unicity of $\beta_s$ normal form.

\item is a consequence of (i) and (ii).
\end{enumerate}
\end{proof}


%%
%%\begin{dfn}[Safe sub-terms]
%%Given a safe term $\Gamma \vdash_s M : A$, we define the set
%%$Safesub(M)$ of safe sub-terms of a $M$:
%%
%%\def\longtype#1#2{(\overline{{#1}_1} \, | \, \cdots \, | \, \overline{{#1}_{#2}} \, | \, o)}
%%
%%\begin{eqnarray*}
%%Safesub(b) &=& b  \qquad \mbox{ ($b \in \Sigma$ a first order constant)} \\
%%Safesub(x) &=& x \qquad \mbox{ ($x$ a variable) } \\
%%Safesub(M N_1 \ldots N_l) &=& \{M N_1 \ldots N_l \} \union Safesub(M) \union \Union_{i=1..l} Safesub(N_i) \\
%%&&\mbox{ where $M : \longtype{B}{n}$ and $ord(N_1) = \ldots =
%%ord(N_l) = ord(\overline{B})$}.\\
%%Safesub(\lambda x_1 \ldots x_n . N) &=& \{ \lambda x_1 \ldots x_n . N \} \union Safesub(R) \\
%%&&\mbox{ where $ord(x_1) = \ldots = ord(x_n)$ and $ord(x_1) \geq ord(N)$}\\
%%&&\mbox{ and $R = N$ if $x_1 \in fv(M)$ and $R = \lambda x_2 \ldots x_n . N$ otherwise.}\\
%%\end{eqnarray*}
%%
%%\end{dfn}
%%
%%It is easy to check that $Safesub(M)$ is indeed the set of all safe
%%sub-terms of $M$.



\begin{lem}
$\beta_s$-reduction preserves safety. (i.e. $M$ safe term and $M
\beta_s N$ implies $N$ safe)
\end{lem}


\begin{proof}
Simultaneous substitution preserves safety (property
\ref{cor:subst_preserve_safety}), therefore we just need to prove
that the relation $\beta_s$ preserves safety and the result will
follow:

 Suppose $s\ \beta_s\ t$ then $s$ is a $\beta_s$-redex. There are two kinds of them
 depending on which rule has been used last to form the redex.

\begin{itemize}
\item Suppose the last rules used is (App), then the redex is
$$s \equiv (\lambda x_1 \ldots x_n . M) N_1 \ldots N_n \qquad \betasred \qquad M[N_1 / x_1 , \cdots, N_n / x_n] \equiv t$$
where $ord(M) \leq ord(x_1) = \ldots = ord(x_n)$

The first premise of the rule (App) tells us that $M$ is safe,
therefore since substitution preserves safety, (property
\ref{prop:subst_preserve_safety}), $t$ is safe.

\item Suppose the last rules used is (App+), then the redex is

 $$
s \equiv  (\lambda \overline{x_l} : \overline{A_l} \
\overline{x_r}: \overline{A_r} . T) \overline{N_l} \qquad \betasred
\qquad \lambda \overline{x_r}: \overline{A_r} .
T\subst{\overline{x_l}}{\overline{N_l}} \equiv  t
$$
where $ord(T) \leq ord(x_1) = \ldots = ord(x_n)$

$T\subst{\overline{x_l}}{\overline{N_l}}$ is safe for the same
reason as in the first case. We can then apply the rule (Abs) and
that prove the safety of $t$.
\end{itemize}
\end{proof}



\begin{rem}
\label{rem:betasred_notpreserv_unsafety} While $\betasred$ preserves
safety it does not however preserves un-safety: given two terms of
the same type, one safe $\Gamma \vdash_s S : A$ and the other unsafe
$\Gamma \vdash U : A$, the term $(\lambda x y . y) U S$ is unsafe
but it $\beta_s$-reduces to $S$ which is safe.
\end{rem}

\subsection{Pointer-less strategies}
\label{subsec:ptrless_strat}

Up to order 2, the semantics of PCF terms is entirely defined by
pointer-less strategies. In other words, the pointers can be
uniquely reconstructed from any non justified sequence of moves
satisfying the visibility and well-bracketing condition.

At level 3 however, pointers cannot be omitted. There is an example
in \cite{abramsky:game-semantics} to illustrate this. Consider the
following two terms of type $((\nat \typar \nat) \typar \nat) \typar
\nat$:

$$M_1 = \lambda f . f (\lambda x . f (\lambda y .y ))$$
$$M_2 = \lambda f . f (\lambda x . f (\lambda y .x ))$$

We assign tags to the types in order to identify in which arena the
questions are asked: $((\nat^1 \typar \nat^2) \typar \nat^3) \typar
\nat^4$. Consider now the following pointer-less sequence of moves
$s = q^4 q^3 q^2 q^3 q^2 q^1$. It is possible to retrieve the
pointers of the first five moves but there is an ambiguity for the
last move: does it point to the first or second occurrence of $q^3$
in the sequence $s$?

Note that the visibility condition does not eliminate the ambiguity,
since the two occurrences of $q^3$ both appear in the P-view at that
point (after recovering the pointers of $s$ up to the second last
move we get $s = \justseq{ q^4 & q^3 \pointto{l} & q^2 \pointto{l} &
q^3 \pointto{ll} & q^2 \pointto{l} & q^1 }$ , therefore the P-view
of $s$ is $s$ itself.)

In fact these two different possibilities correspond to two
different strategies. Suppose that the link goes to the first
occurrence of $q^3$ then it means that the proponent is requesting
the value of the variable $x$ bound in the subterm $\lambda x . f (
\lambda y. ... )$. If P needs to know the value of $x$, this is
because P is in fact following the strategy of the subterm $\lambda
y . x$. And the entire play is part of the strategy $\sem{M_2}$.

Similarly, if the link points to the second occurrence of $q^3$ then
the play belongs to the strategy $\sem{M_1}$.

\subsection{Game semantics of safe $\lambda$ terms}

We would like to find out whether the safety condition defined in
\cite{Ong2005} leads to a pointer economy in the corresponding game
semantics.

The example of section \ref{subsec:ptrless_strat} is a good example
to start with. We observe that for this particular example and in
the safe $\lambda$-calculus setting, the ambiguity that led us to
the addition of pointers to strategies disappear. More precisely,
$M_1$ is a safe term whereas $M_2$ is not. Indeed, there is a free
occurrence of the variable $x$ of type $o$ in the subterm $f
(\lambda y . x)$ which is not abstracted together with $y$ of type
$o$.


\begin{enumerate}
\item
Is it the case that in general, the pointers from the semantics of
safe $\lambda$-terms can be reconstructed uniquely from the moves of
the play?


\item
Is there any unsafe term whose game semantics is a strategy where
pointers can be recovered?

The answer is yes: take the term $T_i = (\lambda x y . y) M_i S$
where $i =1..2$ and $\Gamma \vdash_s S : A$. $T_1$ and $T_2$ both
$\beta$-reduce to the safe term $S$, therefore
$\sem{T_1}=\sem{T_2}=\sem{S}$. But $T_1$ is safe whereas $T_2$ is
unsafe. Since it is possible to recover the pointer from the game
semantics of $S$, it is as well possible to recover the pointer from
the semantics of $T_2$ which is unsafe.

\item
Is there any unsafe $\beta$-normal form whose game semantics is a
strategy where pointers can be recovered?


\end{enumerate}


\subsection{$\eta$-extension}
\todobox{definition of $\eta$-extension, $\eta$-normal form}

Let $\eta$-normal form of a term is the term obtained after
hereditarily $\eta$-expanding every subterm.

\def\etanfaux#1{\lceil#1 \rceil}
\def\etanf#1{\eta-nf\left( #1 \right)}

\begin{eqnarray*}
%\etanf{\lambda x . M} = \lambda x . \etanf{M}\\
\end{eqnarray*}

\subsection{Pointers in the game semantics of safe terms are recoverable}

\todomargin{rework this paragraph!}

We claim that the pointers in the game semantics of a safe term are
uniquely recoverable.

\todobox{Need to prove that safety is preserved by $\eta$-expansion.}

Consider a term $M$ safe, we can assume that $M$ is in $\eta$ normal
form.

The term can be represented by a computation tree: nodes at even
depth (starting at level 0) correspond to $\lambda$ and nodes at odd
length corresponds to either application $@$, variable $x$ or
variable followed by an application $f@$. A $\lambda$ node
represented consecutive abstraction of variables.

There justification pointers going upward from variable occurrences
to their bindings.

In the game semantics of the term $M$, the pointers for O and P
answers can be recovered by using the well-bracketing condition.

For O-question, the justification pointer always points to its
parent node in the computation tree.

For P-question, suppose P ask for the value of variable $x$. Then
there may be several choices for the destination of the pointer but
we claim that in the case of safe terms, it should point to the
closest parent node (in the path from the root to P-question) whose
order is greater than the order of $x$.
