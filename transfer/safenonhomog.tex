
\subsection{Particular case of homogeneously-safe lambda terms}

We look at a particular sub-class of lambda terms. The types of
these terms respect a property call homogeneity as defined in
section \ref{sec:homotypes}. A type $(A_1, A_2, \ldots A_n, o)$ is
said to be homogeneous whenever $\order{A_1} \geq \order{A_2} \geq
\ldots \geq  \order{A_n}$ and each of the $A_i$ are homogeneous. A
term is homogeneous if its type is homogeneous.


In their definition of safety (\cite{KNU02}), Knapik et al. require
that all the recursion equations of a safe recursion scheme have a
homogeneous type.

In the rules defining safety for the non-homogeneous case, the only
rule that can potentially introduce a non-homogeneous term from a
homogeneous one is the abstraction rule. But such a term (lambda
abstraction) corresponds exactly to a recursion equation in the
recursion scheme setting of Knapik et al. Therefore requiring that
recursions equation have homogeneous type is the same as requiring
that all sequents appearing in the proof tree of a safe term are of
homogeneous type.

We say that a term is homogeneously-safe if its type is homogeneous
and there is a proof tree showing its safety where all the sequents
of the proof tree are of homogenous type.

\begin{lem}[Context reduction]
\label{lem:context_reduction} If $\Gamma \vdash^i M : A$ then there
is a context $\Gamma' \subseteq \Gamma$ such that $\Gamma' \vdash^0
M : A$.
\end{lem}
\begin{proof}
An easy induction.
\end{proof}


\begin{lem}
\label{lem:homog_judg_zero_minusone} If a term is homogeneously-safe
then there is valid proof tree showing that it is safe containing
only judgments of the form $\Gamma \vdash^{k} M : T$ with $k\in
\{-1,0\}$.
\end{lem}

\begin{proof}
Assume that $\Gamma \vdash^{0} S : T_S$ with $T_S$ homogeneous.

We prove the result by induction on the size of the proof tree and
by case analysis on the last rule used to obtain $\Gamma \vdash^{0}
S : T_S$.

We give the details for the application and abstraction rule:
\begin{itemize}
\item Rule $\mathbf{(abs^i)}$ for some $i$:
$$ \mathbf{(abs^i)} \quad  \rulef{\Gamma, \overline{x} : \overline{A} \vdash^i M : B}
                                   {\Gamma  \vdash^{0} \lambda \overline{x} : \overline{A} . M : (\overline{A},B)} \qquad
                                   \forall y \in \Gamma : \ord{y} \geq \ord{\overline{A},B}$$

Type homogeneity requires that $\ord{\overline{x}} = \ord{A} \geq
\ord{B}$. Therefore the premise of the rule can be replaced by
$\Gamma, \overline{x} : \overline{A} \vdash^0 M : B$.

The induction hypothesis permits to conclude.

\item Rule $\mathbf{(app)}$:
$$ \mathbf{(app)} \quad  \rulef{\Gamma \vdash^{-1} M : (A,\ldots,A_l,B)
                                        \qquad \Gamma \vdash^{0} N_1 : A_1
                                        \quad \ldots \quad \Gamma \vdash^{0} N_l : A_l  }
                                   {\Gamma  \vdash^0 M N_1 \ldots N_l : B}
                                    \qquad
                                   \forall y \in \Gamma : \ord{y} \geq \ord{B}$$

For the first premise of the rule, we apply lemma
\ref{lem:context_reduction}: there is a context $\Gamma'$ such that
$\Gamma' \subseteq \Gamma$ and:
$$ \Gamma' \vdash^{0} M : (A,\ldots,A_l,B) $$
Now by the induction hypothesis we know that there is a proof tree
showing $\Gamma' \vdash^{0} M : (A,\ldots,A_l,B)$ with judgement of
the form $\Sigma \vdash^{k} P : T$ with $k\in \{-1,0\}$.

By applying the weakening rule, we lift this result to $\Gamma
\vdash^{0} M : (A,\ldots,A_l,B)$.

For all the other premises we can directly apply the induction
hypothesis.

We conclude using the rule $\textbf{(app)}$.
\end{itemize}
\end{proof}

This lemma permits us to derive rules specialized for the
homogeneously-safe lambda calculus.

\subsubsection{The application rule} We are now about to derive the
application rules specialized for the case of homogeneous types. We
recall the rule $\mathbf{(app)}$:
$$ \mathbf{(app)} \quad  \rulef{\Gamma \vdash^{-1} M : (A,\ldots,A_l,B)
                                        \qquad \Gamma \vdash^{0} N_1 : A_1
                                        \quad \ldots \quad \Gamma \vdash^{0} N_l : A_l  }
                                   {\Gamma  \vdash^0 M N_1 \ldots N_l : B}
                                    \qquad
                                   \forall y \in \Gamma : \ord{y} \geq \ord{B}$$

Type homogeneity implies that $\ord{A_1} \geq \ldots \geq \ord{A_l}
\geq \ord{B} - 1$.

We can make the assumption that $\ord{A_1} = \ldots = \ord{A_l}$ (if it is not the case,
we can replace the application rule by several consecutive application rules
respecting this condition).


\begin{itemize}
\item Suppose that $A_1, \ldots A_l$ forms a type partition, then we
have $\ord{A_l} \geq \ord{B}$, the side-condition disappears and
the rule becomes:
$$ \mathbf{(app_1)} \quad  \rulef{\Gamma \vdash^{-1} M : \overline{A} | B
                                        \qquad \Gamma \vdash^{0} N_1 :
                                        A_1
                                        \quad \ldots \quad \Gamma \vdash^{0} N_l :
                                        A_l
                                        }
                                   {\Gamma  \vdash^{0} M N_1 \ldots N_l : B}
$$

where $\overline{A} = A_1, \ldots A_l$

\item  Suppose that $A_1, \ldots A_l$ do not form a type partition, then we
have $\ord{A_l} = \ord{B} - 1$. The side-condition becomes
$\forall y \in \Gamma : \ord{y} \geq 1 +\ord{A_l} = \ord{\overline{A}|B}$. Therefore the side-condition
can be omitted provided that we replace the $-1$ exponent in the first premise by a $0$:
$$ \mathbf{(app_2)} \quad  \rulef{\Gamma \vdash^0 M : (A,\ldots,A_l,B)
                                        \qquad \Gamma \vdash^{0} N_1 : A_1
                                        \quad \ldots
                                        \quad \Gamma \vdash^{0} N_l : A_l
                                   }
                                   {\Gamma  \vdash^0 M N_1 \ldots N_l : B}
$$

Equivalently we could replace this rule by the following one:
$$ \mathbf{(app_2')} \quad  \rulef{\Gamma \vdash^0 M : A\rightarrow B
                                        \qquad \Gamma \vdash^{0} N : A }
                                {\Gamma  \vdash^{0} M N : B}$$
\end{itemize}

\subsubsection{The abstraction rule}

Let us derive the abstraction rule specialized for the case of
homogeneous types. We recall the rule $\mathbf{(abs)}$:
$$ \mathbf{(abs^i)} \quad  \rulef{\Gamma, \overline{x} : \overline{A} \vdash^{i} M : B}
                                   {\Gamma  \vdash^{0} \lambda \overline{x} : \overline{A} . M : (\overline{A},B)} \qquad
                                   \forall y \in \Gamma : \ord{y} \geq \ord{\overline{A},B}$$

We now partitionned the context $\Gamma$ according to the order of
the variables. The partition are written in decreasing order of type
order. The notation $\Gamma | \overline{x}:\overline{A}$ means that
$\overline{x}:\overline{A}$ is the lowest partition of the context.

We also use the notation $(\overline{A}|B)$ to denote the
homogeneous type $(A_1, A_2, \ldots A_n, B)$ where $\ord{A_1} =
\ord{A_2} =  \ldots \ord{A_n} \geq \ord{B} -1$.


Suppose that we abstract the single variable $\overline{x} = x$,
then in order to respect the side condition, we need to abstract all
variables of order lower or equal to $\ord{x}$. In particular we
need to abstract the partition of the order of $x$.

Moreover to respect type homogeneity, we need to abstract variables
of the lowest order first.

Hence we can change the abstraction rule so that it only allows
abstraction of the lowest variable partition. The rule can then be
used repeatedely if further partitions need to be abstracted. We
obtained the following rule where the side-condition has
disappeared:

$$ \mathbf{(abs^i)} \quad  \rulef{\Gamma| \overline{x} : \overline{A} \vdash^{-1} M : B}
                                   {\Gamma  \vdash^{0} \lambda \overline{x} : \overline{A} . M : (\overline{A}|B)}$$


\subsubsection{The rules of the homogeneous safe $\lambda$-calculus}

Table \ref{tab:homosafelmd_rules} recapitulates the entire set of rules:

\begin{table}[htbp]
$$  (\mbox{\textbf{perm}}) {
      { \Gamma \vdash^0 M:B \qquad \sigma(\Gamma)  } \hbox{ homogeneous}
    \over
      { \sigma(\Gamma) \vdash^0 M : B }
    }
\qquad
\mathbf{(seq)} \quad \rulef{\Gamma \vdash^{0} M : A}{\Gamma \vdash^{-1} M : A}
$$

$$
 \textbf{(const)}
    { \over { \vdash^0 b : o^r \rightarrow o}} \quad b : o^r \rightarrow o \in \Sigma
\qquad
 \mathbf{(var)} \quad  \rulef{}{x : A\vdash^{0} x : A} $$

$$ \mathbf{(wk^{0})} \quad  \rulef{\Gamma \vdash^{0} M : A}{\Gamma , x : B \vdash^{0} M : A} \quad \ord{B} \geq \ord{A} $$

$$ \mathbf{(wk^{-1})} \quad  \rulef{\Gamma \vdash^{-1} M : A}{\Gamma , x : B \vdash^{-1} M : A} \quad \ord{B} \geq \ord{A} -1$$

$$ \mathbf{(app)} \quad  \rulef{\Gamma \vdash^{-1} M : \overline{A} | B
                                        \qquad \Gamma \vdash^{0} N_1 : A_1
                                        \quad \ldots \quad \Gamma \vdash^{0} N_l : A_l
                                        \qquad l = |\overline{A}|
                                        }
                                   {\Gamma  \vdash^{0} M N_1 \ldots N_l : B}
$$


$$ \mathbf{(app^0)} \quad  \rulef{\Gamma \vdash^0 M : A\rightarrow B
                                        \qquad \Gamma \vdash^{0} N : A
                                   }
                                   {\Gamma  \vdash^{0} M N : B}$$

$$ \mathbf{(abs)} \quad  \rulef{\Gamma| \overline{x} : \overline{A} \vdash^{-1} M : B}
                                   {\Gamma  \vdash^{0} \lambda \overline{x} : \overline{A} . M : (\overline{A}|B)}$$
\caption{Rules of the homogeneous safe lambda calculus}
\label{tab:homosafelmd_rules}
\end{table}


We observe that these rules correspond exactly to the rules given in the previous section
in table \ref{tab:homosafelmd_rules_refined}.
